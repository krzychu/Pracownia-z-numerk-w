\documentclass[12pt,a4paper,wide,authortitle]{mwart}

\usepackage[polish]{babel}%\usepackage{polski}
\usepackage[utf8]{inputenc}
\usepackage[OT4]{fontenc}
\usepackage{amsmath} %amssymb
\usepackage{amsthm}
\numberwithin{figure}{section}  
\numberwithin{table}{section}  
\linespread{1.3}
\usepackage{multirow}
\usepackage[small, bf]{caption}
\newtheorem{lm}{Lemat}
\usepackage{graphicx}
\usepackage{wrapfig}
\usepackage{subfigure}
\newcounter{myenum}
\newenvironment{myenumerate}{\setcounter{myenum}{1}\def\item{\par{\bf
      \arabic{myenum} - \ }\stepcounter{myenum}}}{\newline}

\begin{document}
\title{\LARGE \textbf{Pracownia z Analizy Numerycznej}\\Rozwiązanie zadania P3.14}
\author{Krzysztof Chrobak\thanks{\textit{E-mail}:
\texttt{krzychuch@gmail.com}} \and Aleksandra Spyra\thanks{\textit{E-mail}:
\texttt{aleksandra.spyra180@gmail.com}}}
\date {Wrocław, \today}
\maketitle
\tableofcontents
\newpage
\section{Wstęp}
Macierze stosuje się do reprezentacji danych i obliczeń w wielu dziedzinach naukowych i technicznych m. in. w matematyce, fizyce czy grafice komputerowej. Często istotne jest również przprowadzanie działań na macierzach, takich jak mnożenie, dodatkowo w jak najkrótszym czasie. Nie dziwi nas więc fakt, że powstało wiele algorytmów obliczjących iloczyn macierzy. W naszej pracy przedstawimy i zbadamy doświadczalnie dwa spośród nich.

Głównym celem naszego zadania jest porównanie pod względem dokładności i szybkości dwuch algorytmów mnożenia macierzy rozmiaru $n \in [4, 500]$: algorytmu naturalnego i algorytmu Strassena. Stawiamy przed sobą dodatkowe zadanie jakim jest rozszerzenie zakresu wielkości danych oraz znalezienie przedziałów wielkości macierzy dla których określony algorytm daje dokładniejsze wyniki czy też jest wydajniejszy czasowo.

W rozdziale \S2 znajduje się krótki opis algorytmów, następny rozdział przedstawia realizację zadania. Tam znajduje się raport z przeprowadzonych doświadczeń, jak również przegląd otrzymanych wyników w postaci tabel i wykresów. Rozdział \S 5 zawiera nie tylko wnioski oraz obserwacje zebrane przez nas podczas rozwiązywania problemu, ale także został wzbogacony o rozważania teoretyczne, które pozwoliły nam uzyskać szersze spojrzenie na obydwa algorytmy.

Do obliczeń wykorzystaliśmy napisany przez nas program w języku C++, implementujący dwie wspomniane powyżej metody liczenia iloczynu macierzy. Pliki z wszstkimi testami oraz wynikami zostały umieszczone w katalogu \textit{testy}.   

\section{Opis metod}
\subsection{Rozkład QR}
\subsubsection{Metoda Grama-Schmidta}
\subsubsection{Metoda Householdera}
\subsection{Rozkład LU}

\section{Znajdowanie macierzy odwrotnej}
\subsection{Rozkład QR}
Załóżmy, że mamy już obliczony rozkład macierzy $A=QR$. Wobec tego
odwrotność danej macierzy, możemy obliczyć wykorzystując tożsamość:
$$A^{-1}=(QR)^{-1}= R^{-1}Q^{-1}.$$
Wystarczy więc, że obliczymy osobno odwrotności macierzy ortogonalnej $Q$ i
górnotrójkątnej $R$. Znalezienie odwrotności macierzy $Q$ jest zadaniem
trywialnym. Wynika to z tego, że jest to macierz ortogonalna i
wobec czego jej odwrotność to transpozycja
$Q^{-1}=Q^{T}$. Zadanie to sprowadza się więc do zamiany wierszy z
kolumnami.

Więcej, ale wciąż niewiele, trudności przysparza nam znalezienie odwrotności
macierzy górnotrójkątej $R$. Oznaczmy przez $r_{ij}$ elementy macierzy $R$,
zaś $x_ij$ elementy jej odwrotności, $1 \leq i,j \leq n$. Wówczas elementy
macierzy $R^{-1}$ możemy obliczyć w następujący sposób:
$$ x_{ij}=
\begin{cases}
0, & \text{jeśli } j<i \\[8pt]
\displaystyle \frac{1}{r_{ij}}, & \text{jeśli } i=j \\[8pt]
\displaystyle -\frac{\sum_{k=0}^{j-i}a_{i,i+k}x_{i+k,j}}{a_{ii}}, & \text{jeśli } j>i, 
\end{cases}
$$
co wynika z rozwiązania układu równań $RR^{-1}=I$.
\subsection{Rozkład LU}
Jeżeli mamy rozkład $LU$ macierzy to, podobnie jak dla dekompozycji $QR$,
wykorzystujemy wzór na odwrotność iloczynu macierzy. Stąd $A^{-1}=U^{-1}L^{-1}$,
wystarczy więc umieć odwracać macierze trójkątne. To już jednak umiemy, odwróciliśmy
bowiem macierz $R$ z poprzedniego podrozdziału. Odwrotność macierzy dolnotrójkątnej
uzyskuje się analogicznie do górnotrójkątnej.


\section{Porównanie szybkości działania metod}
\section{Porównanie dokładności metod}
Jeśli chodzi o badanie dokładności uzyskanych wyników sprawa nie jest już tak
prosta. W zależności od tego, jaką struktrę ma macierz, analizowane przez nas
algorytmy mogą zachowawać się różnie. Postanowiliśmy wiec przeprowadzić
eksperymenty dla różnych rodzajów macierzy z zakresu wielkośći $[10, 200]$. Przypożądkujemy numer indentyfikujący
dany typ macierzy, aby uprościć dalsze rozważania. Mianowicie przez kolejne cyfry
oznaczamy:
\begin{myenumerate}
\item macierz wygenerowaną losowo,
\item macierz trójkątną,
\item macierz Hilberta (jak wiemy jest to macierz źle uwarunkowana),
\item macierz z dominującą przekątną,
\item macierz, której wyznaczik jest bliski zeru,
\item macierz, dla której jeden z minorów jest zerowy,
\item macierz trójprzekątniową.
\end{myenumerate}

Pierwsze przeprowadzone przez nas badanie polegało na wykonaniu rozkładu $QR$
obydwiema metodami i obliczeniu dla uzyskanych wyników wartośći wyrażenia:
$$\mathrm{F}(QQ^T - I),$$
gdzie $\mathrm{F}$ oznacza \textit{normę Frobeniusa} macierzy, którą definujemy
dla danej macierzy $A=(a_{ij})$ jako:
$$F(A)=(\sum_{i=1}^{n} \sum_{j=1}^{n} a_{ij}^2)^\frac{1}{2}.$$
Innymi słowy sprawdzamy, w jakim stopniu wyznaczona macierz $Q$ jest ortogonalna. Im większy
współczynnik, tym obliczona macierz jest bardziej oddalona od porządanej. Częściowe wyniki zostały przedstwione
w tabeli na rysunku \textbf{5.1}. Pełne dane można znaleźć w katalogu \textit{prog/testy}. Pierwszy wiersz tabeli 
określa użytą metodę, drugi zaś wielkość macierzy. W pierwszej kolumnie znajdują się identyfikatory poszczególnych typów
macierzy.
\begin{figure}[h!tb]
\begin{center}
\begin{tabular}{|l||c|c|c|c||c|c|c|c|}
\hline
\multirow{2}{*}{} & \multicolumn{4}{|c||}{\textbf{\textit{GS}}} & \multicolumn{4}{|c|}{\textbf{\textit{HH}}}\\
\cline{2-9}
&\textbf{10} & \textbf{40} & \textbf{100} & \textbf{190} & \textbf{10} & \textbf{40} & \textbf{100} & \textbf{190} \\
\hline
\hline
\textbf{1} & $0.5824$ & $80.9396$ & $87.5263$ & $97.9383$ & $0.2926$ & $122.4385$ & $1.1238$ & $2.0456$ \\
\hline
\textbf{2} & $1.0041$ & $5.9157$ & $16.0520$ & $157.3206$ & $0.1491$ & $0.4351$ & $1.1557$ & $2.1509$ \\
\hline
\textbf{3} & $ 0.3179$ & $4.8791$ & $20.1218$ & $106.4283$ & $0.1105$ & $0.4917$ & $1.1883$ & $2.1245$ \\
\hline
\textbf{4} & $ 0.2962$ & $5.5866$ & $38.1047$ & $408.4966$ & $0.2899$ & $0.5077$ & $1.1076$ & $2.3334$ \\
\hline
\textbf{5} & $ 0.3169$ & $17.6216$ & $72.5043$ & $58.2336$ & $0.1160$ & $0.5325$ & $1.2233$ & $2.1119$ \\
\hline
\textbf{6} & $ 39.8250$ & $35.6775$ & $49.8930$ & $836.6792$ & $0.1480$ & $0.4992$ & $1.1016$ & $2.1021$ \\
\hline
\textbf{7} & $ 0.3563$ & $660.1143$ & $621.1397$ & $46.6918$ & $0.1622$ & $0.4681$ & $1.1303$ & $2.1675$ \\
\hline
\end{tabular}
\caption{Tabela dla wyrażenia $\mathrm{F}(QQ^T - I)$, jednostka: $10^{14}$.} 
\end{center}
\end{figure}

Analizując tę tabelę, można zaobserwować, że wartość badanego wspołczynnika nie zależy od typu macierzy, zależy
zaś od jej wielkości. Dotyczy to zarówno algorytmu $HH$ jak i $GS$. Jednocześnie widzimy, że algorytm Householdera
w tym przypadku sprawuje się lepiej, niekiedy nawet wartość współczynnika jest dla niego dwa rzędy niższa. 

Kolejny eksperymet również dotyczy wyznaczania rozkładu $QR$. Tym razem przyjrzymy się dla obydwu metod
i danej macierzy $A$ wartości wyrażenia:
$$\mathrm{F}(A-QT).$$
Tym samym chcemy się przekonać, jak bardzo rozkład $QR$ deformuje macierz wejściową $A$ w przypadku obu alorytmów
i która z metod jest mniej dokładna. Częściowe wyniki uzyskane przy wykonywaniu tego zadania znajdują się w tabeli
\textbf{5.2}.
\begin{figure}[h!tb]
\begin{center}
\begin{tabular}{|l||c|c|c|c||c|c|c|c|}
\hline
\multirow{2}{*}{} & \multicolumn{4}{|c||}{\textbf{\textit{GS}}} & \multicolumn{4}{|c|}{\textbf{\textit{HH}}}\\
\cline{2-9}
&\textbf{10} & \textbf{40} & \textbf{100} & \textbf{190} & \textbf{10} & \textbf{40} & \textbf{100} & \textbf{190} \\
\hline
\hline
\textbf{1} & $0.1013$ & $17.1612$ & $4.6160$ & $16.2619$ & $0.1616$ & $5.6800$ & $2.7392$ & $6.7934$ \\
\hline
\textbf{2} & $0.0689$ & $0.6461$ & $3.6861$ & $45.8629$ & $0.1253$ & $0.7209$ & $ 2.8090$ & $7.0702$ \\
\hline
\textbf{3} & $0.0739$ & $0.5330$ & $3.1111$ & $23.2040$ & $0.1268$ & $0.7758$ & $2.7903$ & $7.1237$ \\
\hline
\textbf{4} & $0.1038$ & $0.8578$ & $6.5664$ & $49.8840$ & $0.1986$ & $0.7444$ & $2.7274$ & $7.3635$ \\
\hline
\textbf{5} & $0.0764$ & $6.8850$ & $23.1308$ & $13.7058$ & $0.0979$ & $0.7653$ & $2.7367$ & $6.9651$ \\
\hline
\textbf{6} & $9.9341$ & $5.9521$ & $9.1611$ & $262.4094$ & $0.0850$ & $0.7019$ & $2.7706$ & $6.9461$ \\
\hline
\textbf{7} & $0.0492$ & $260.6983$ & $31.3226$ & $21.3792$ & $0.1278$ & $0.7038$ & $2.7631$ & $7.1066$ \\
\hline
\end{tabular}
\caption{Tabela dla wyrażenia $\mathrm{F}(A-QT)$, jednostka: $10^{-12}$.} 
\end{center}
\end{figure}
Tak jak w poprzednim przypadku, tak i tutaj wartość wyznacznika dla obydwu metod w niewielkim stopniu
zależy od typu badanej macierzy, zależy zaś w podobny sposób od wielkośći macierzy. Na tym nie kończą
się zaobserwowane podobieństwa. Ponownie bardziej efektowny okazuje się być algorytm Householdera, przy
czym odpowiednie wielkości normy różnią się o jeden bądź dwa rzęd wielkości dla większych macierzy. Skupimy
się więc w dalszych rozważaniach na metodzie \textit{HH}.
 



\section{Podsumowanie}
Wykonane przez nas zadanie pozwoliło nam zapoznać się z metodami wykonywania rozkładu
$QR$, który wykorzystaliśmy do odwracania macierzy kwadratowych. Ponadto zaobserwowaliśmy,
że nie nie zawsze otrzymany wynik przy pomocy badanych metod jest bliski wynikowi oczekiwanemu.

Zbierzmy teraz zebrane przez nas obserwacje i wnioski. Przede wszystkim przypomnijmy, że wszystkie
badane przez nas metody miały podobny czas działania, który nie zależy od struktury macierzy.
Wynika to z prostego faktu, zarówno metody \textit{HH}, \textit{GS}, jak i \textit{LU} mają taką
samą złożoność obliczeniową równą $O(n^3)$. W rozdziale \S5 opisaliśmy przeprowadzone przez nas
badania nad dokładnością metod, która okazuje się zależeć nie tylko od wielkości macierze, ale
także od jej struktury. Jeśli chodzi o rozkład $QR$, metoda \textit{HH} okazała się lepsza od
metody \textit{GS}. Wobec tego to właśnie tę metodę zestawiliśmy z algorytmami wykorzystującym
rozkład $LU$ i biblioteką \verb|numpy|. Dokładności uzyskane za pomocą metod \textit{HH} i
\textit{LU} okazały się zbliżone dla wszystkich badanych rodzajów macierzy, oprócz tych z zerującym
się minorem głównym. W tym przypadku ewidentnie dokładniejsza okazała się być metoda Householdera.
Zauważyliśmy ponadto, że obydwa algorytmy nie radzą sobie z odwracaniem macierzy Hilberta, błędy
uzyskane były katastrofalnie duże. Podobne obserwacje uzyskaliśmy przy porównywaniu naszych metod z
biblioteką \textit{numpy}. Satysfakcjonujące wyniki otrzymaliśmy dla macierzy losowych, z dominującą
przekątną, o wyznaczniku bliskim zeru i trójdiagonalnych. Ponownie metoda \textit{HH} dała lepsze wyniki dla
macierzy typu \textbf{6}. Liczony wskaźnik był dla nich około $10^{12}$ razy mniejszy.

\begin{thebibliography}{99}
\bibitem{Notatki}   S. Lewanowicz, {\it Notatki do wykładu z analizy numerycznej}, Wrocław, 2010.
\bibitem{Kincaid} Kincaid D., Cheney W., {\it Numerical analysis}, California, 1991.
\bibitem{Janki} Dryja M., J. i M. Jankowscy {\it Przegląd metod i algorytmów numerycznych}, Warszawa, 1982.
\bibitem{Dahlquist} Dahlquist G., Bjorck A. {\it Metody numeryczne}, Warszawa, 1983.
\bibitem{NAL} Kiełbasiński A., Schwetlick H. {\it Numeryczna algebra liniowa}, Warszawa, 1992. 
\bibitem{Dow} Math4All, \verb|http://www.math4all.in/public_html/linear%20algebra/proof8.5.11.html|.
\end{thebibliography}


\end{document}

