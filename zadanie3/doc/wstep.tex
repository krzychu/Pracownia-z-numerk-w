\section{Wstęp}
Macierze to sposób reprezentacji danych i przekształceń, z którego w dzisiejszych
czasach nie korzystają wyłącznie matematycy. Szeroko stosowane są one
w informatyce, przykładowo w grafice komputerowej. Ważne jest aby wszystkie
obliczenia macierzowe realizować w sposób możliwie najszybszy i najdokładniejszy.
Wiele operacji znacznie upraszcza się, jeśli macierz przedstawimy jako
iloczyn macierzy prostszej postaci, dla których znamy algorytmy pozwalające
nam w efektywny sposób dokonywać obliczenia. Taki rozkład macierzy stanowi
dekompozycja $QR$, która przedstawia macierz jako iloczyn macierzy ortogonalnej
i macierzy trójkątnej górnej. Jak się okazuje, istnieją różne algorytmy,
wyznaczające taki rozkład, który jest cenny, ze względu na szerokie zastosowanie.

Głównym celem naszego zadania jest opracowanie metod służących do
znajdowania dekompozycji $QR$. Skupiamy naszą uwagę na dwóch algorytmach: \textit{metodzie Grama-Schmidta},
oznaczanej jako \textit{GS}
i \textit{metodzie Householdera},
którą będziemy skrótowo określać jako \textit{HH}. Stawiamy przed sobą problem, jak wyznaczyć
macierz odwrotną mając dany rozkład $QR$ macierzy. Dodatkowo postaramy się
porównać pod względem dokładności i szybkości oba algorytmy,
zestawiając je także z algorytmem wyznaczania odwrotności przy użyciu
dekompozycji LU, którą poznaliśmy podczas tegorocznych zajęć.

Rozdział \S2 zawiera definicje poszczególnych dekompozycji. Również w jego
obrębie znajduje się opis wykorzystanych przez nas metod Grama-Schmidta i Householdera.
Kolejna część zawiera opis sposobu, w jaki mając daną dekompozycję $QR$
uzyskać odwrotność wejściowej macierzy. Jak się okaże, jest to bardzo proste
w realizacji. Rozdział \S4 stanowi porównanie szybkości znajdowania
odwrotności za pomocą rozkładu $LU$ oraz $QR$ z wykorzystaniem obydwu
wymienionych powyżej metod. W następnym rozdziale zaś, zamieściliśmy
relację z doświadczeń, które pozwoliły nam zbadać dokładność wszystkich
trzech wspomnianych sposobów odwracania macierzy. Aby zrealizować
tę część zadania, wykorzystaliśmy nie tylko macierze generowane losowo, ale
także zrobiliśmy próby dla macierzy o szczególnych właściwościach,
przykładowo dla macierzy źle uwarunkowanych, czy macierzy trójdiagonalnych.
Porównaliśmy nasze algorytmy także z wynikami jakie generuje biblioteka \textit{numpy}.
Dane, które zostały zgromadzone podczas badań, postaraliśmy się
umiejętnie przedstawić przy pomocy wykresów i tabel.
Ostatnia część sprawozdania to swoiste zebranie wniosków i obserwacji
nagromadzonych podczas realizacji naszego zadania.

Implementację przedstawionych metod i algorytmów można odnaleźć w katalogu \textit{prog}.




