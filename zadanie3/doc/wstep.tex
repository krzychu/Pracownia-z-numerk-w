\section{Wstęp}
Macierze to sposób reprezentacji danych i przekształceń, z którego w dzisiajeszych
czasach nie korzystają wyłącznie matematycy. Szeroko stosowane są one
w informatyce, przykładwo w gafice komputerowej. Ważne jest aby wszystkie
obliczenia macierzowe realizować w sposób możliwie najszybszy i najdokładniejszy.
Wiele opercji znacznie upraszcza się, jeśli macierz przedstawimy jako
iloczyn macierzy prostszej postaci, dla których znamy algorytmy pozwalające
nam w efekttywny sposób dokonywać obliczeń. Taki rozkład macierzy stanowi
dekompozycja QR, która przedstawia macierz jako iloczyn macierzy oragonalnej
i macierzy trójkątnej górnej. Jak się okazuje, istnieją różne algortmy,
wyznaczające taki rozkład, który jest cenny, ze względu na szerokie zastosowanie.

Głównym celem naszego zadania jest opracowanie metod służących do
znajdowania dekompozycji QR. Skupiamy naszą uwagę na dwóch algorytmach: \textit{metodzie Grama-Schmidta}
i \textit{metodzie Householdera}. Stawiamy przed sobą problem, jak wyznaczyć
macierz odwrotną mając dany rozkład QR macierzy. Dodatkowo postaramy się
porównać pod wględem dokładności i szybkości oba algorytmy,
zestawiając je także z algorytmem wyznaczania odwrotnośći przy użyciu
dekompozycji LU, którą poznaliśmy podczas tegorocznych zajęć.
Rozdział \S2 zawiera definicje poszczególnych dekompozycji. Również w jego
obrębie znajduje się opis wykorzystanych przez nas metod Grama-Schmidta i Householdera.

Kolejna część zawiera opis sposobu, w jaki mając daną dekompozycję QR
uzyskać odwrotność wejściowej macierzy. Jak się okaże, jest to bardzo proste
w realizacji. Rozdział \S4 stanowi porównanie szbkośći znajdowania
odwrotności za pomocą rozkładu LU oraz QR z wykorzystaniem metod
Grama-Schmidta i Householdera. W następnym rozdziale zaś, zamieściliśmy
relację z doświadczeń, które pozwoliły nam zbadać dokładność wszystkich
trzech wspomnianych wyżej sposobów odwracania macierzy. Aby zralizować
tę część zadania, wykorzystaliśmy nie tylko macierze generowane losowo, ale
także zrobiliśmy próby dla macierzy o szczególnych właściwośćiach,
przykładowo dla macierzy źle uwarunkowanych, czy macierzy trójdiagonalnych.
Dane, które zostały zgromadzone podczas badań, postaraliśmy się
w dwóch ostatnio wsponianych rozdziałach zwizualizować przy pomocy wykresów.
Osatnia część sprawozdania to swoiste zebranie wniosków i obserwacji
nagromadzonych podczas realizacji naszego zadania.
Implementację przedstawionych metod i algorytmów można odnaleźć w katalogu \textit{prog}.



