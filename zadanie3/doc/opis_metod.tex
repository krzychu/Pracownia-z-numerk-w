\section{Opis metod}
W rozdziale tym wytłumaczymy ideę dekompozycji $QR$ oraz $LU$ macierzy oraz
podane zostaną metody do wyznaczaczania poszczególnych rozkładów, przy czym dla
dekompozycji $QR$ są to metody zwane \textit{metodą Grama-Schmidta} i
\textit{metodą Householdera} (w celu zgłębienia informacji na temat tych metod można zajrzeć do \cite{NAL}).
\subsection{Rozkład QR}
Dekompozycja $QR$ macierzy kwadratowej $A \in M_{n \times n}$ jest to rozkład
macierzy $A$ na dwie macierze kwadratowe $Q \in M_{n \times n}$ i $R \in M_{n
\times n}$ takie, że:
$$A=QR,$$
przy czym $Q$ jest macierzą ortogonalną ($QQ^T=I$), zaś R jest macierzą trójkątną
górną.

Ciekawą własnością tej dekompozycji jest to, że isnieje ona dla dowolnej
macierzy kwadratowej. Jeżeli przyjmiemy dodatkowo, że elementy diagonali
macierzy $R$ są wszystkie nieujemne, rozkład taki jest wyznaczony jednoznacznie.
Dowód powyższej własności jest dowodem konstrukcjyjnym
i opiera sie o metodę orotgonalizacji GS, której opis znajduje się w
dalszej części sprawozdania (wspomniany dowód można znaleźć na stronie \cite{Dow}).

Istnieje również dekompozycja $QR$ dla macierzy prostokątnych. Niech macierz
$A \in M_{n \times n}$, $m \geq n$. Wówczas rozkładamy macierz $A$ na macierz
ortogonalną $Q \in M_{m \in m}$ i macierz trójkątną górną $R \in M_{m \in n}$.
Jak można zauważyć, dolne $m-n$ wierszy macierzy $R$ są zerowe, stosuje się więc
podział macierzy $Q$ i $R$ na $R_1 \in M_{n \times n}$ i dwie macierze posiadające
ortogonalne kolumny, mianowicie $Q_1 \in M_{m \times n}$ i $Q_2 \in M_{m, m-n}$.
Zachodzi równość:
$$A=QR=(Q_1, Q_2)=\begin{pmatrix} R_1 \\ 0 \end{pmatrix}=Q_1R_1.$$
W dalszych rozważąniac ograniczymy się jednak do macierzy kwadratowych.
\subsubsection{Metoda Grama-Schmidta}
Jest to pierwsza rozważana przez nas metoda dekompozycji $QR$. Definiujemy
operator projekcji jako:
$$\mathrm{proj_u}(v)=\frac{<v,u>}{<u,u>}u,$$
gdzie $<x,y>$ oznacza iloczyn skalarny wektorów $x$ i $y$. Innymi słowy, projekcja
to rzut wektora $v$ na prostą rozpinaną przez wektor $u$. Danymi wejściowymi
metody Grama-Schmidta są kolejne kolumny macierzy $A=[a_1,...,a_n]$.
Oznaczamy przez ${u_k}$ i ${e_k}, 1 \leq k \leq n,$ następujące wektory:
$$ u_k = a_k-\sum_{j=1}^{k-1} \mathrm{proj_{e_j}a_k}, \hspace{2cm}
e_k = \frac{u_k}{\|u_k\|}.$$
Przekształcająć odpowiednio podane równości i kożystając z faktu, że
$<e_k, a_k> = \|u_k\|$ i tego, żę wektory $e_k$ są wektorami jednostkowymi dla
$1 \leq k \leq n$ otrzymujemy, że:
$$ a_k = \sum_{i=1}^{k} <e_i,a_k>e_i.$$
Powyższą równość możemy zapisać w postaci macierzowej $A=QR$, które są
postaci takiej, jakiej oczekiwaliśmy. Macierz $Q$ bowiem jest macierzą ortogonalną,
zaś $R$ górnotrójkątną i możemy je zapisać w postaci:
$$ Q = [e_1,...,e_n], \hspace{1cm} R = \begin{pmatrix} \langle\ e_1,a_1
\rangle & \langle e_1,a_2\rangle & \langle e_1, a_3\rangle & \ldots \\ 0
& \langle e_2, a_2\rangle & \langle e_2, a_3\rangle & \ldots \\ 0 & 0 &
\langle e_3, a_3\rangle & \ldots \\ \vdots & \vdots & \vdots & \ddots
\end{pmatrix}.$$
\subsubsection{Metoda Householdera}
Druga metoda rozkładu $QR$, którą pozanaliśmy, korzysta z odbić Housholdera.
Dla danego wektora $w$ \textbf{odbicie} (macierz) \textbf{Householdera}
definiujemy jako:
$$H=I-2ww^T.$$
Można dostrzec, że wartość $Hx$ jest odbiciem lustrznym wektora $x$ względem
hiperpłaszczyzny prostopadłej do wektora $w$. Ponadto macierz $H$ jest równa
macierzy $H^{-1}$ i $H^T$, w szczególności jest to więc przekształcenie
ortogonalne. W badanej przez nas metodzie Hausholdera odbić tych używa się
do przekształacania wektorów w taki sposób, aby wszystkie współrzędne,
poza jedną, uległy wyzerowaniu. Niech więc $x$ będzie wektorem kolumnowym,
a $e_1$ wektorem postaci $(1,0,...,0)^T$. Przez $u, v, Q$ oznaczmy odpowiednio $u=x+\|x\|e_1$,
$v=\frac{u}{\|u\|}$ oraz $Q=I-2vv^T$. Uzyskujemy wówczas macierz $Q$
będącą odbiciem Householdera i  widzimy, że wartość $Qx$ jest postaci $(\|x\|,0,...,0)^T$.
To właśnie ten proces jest wykorzystyany do transformacji wejśćiowej
macierzy A do macierzy górnotrójkątnej. W pierwszej kolejności mnożymy
macierz A przez macierz Householdera $Q_1$ tak, aby wszystkie elementy
prócz pierwszego elementy kolumny $a_1$ się wyzerowały, czyli tak, aby
zachodziła równość:
$$ Q_1A = \begin{pmatrix} \|a_1\|&\star&\dots&\star\\ 0 & & & \\ \vdots & &
A' & \\ 0 & & & \end{pmatrix}.$$
Powtarzamy to działanie dla macierzy $A'$, uzyskując macierz $Q_2'$.
Aby operować na miecirzach tego samego rozmiaru przyjmujemy:
$$Q_k=\begin{pmatrix} I_{k-1} & 0 \\ 0 & Q_k' \end{pmatrix}.$$
Po powtórzeniu tego procesu n-krotnie, uzyskujemy:
$$R=Q_{n-1}...Q_1A=QA.$$
$Q$ jako iloczyn macierzy ortagonalnych jest macierzą ortogonalną, R jest
macierzą górnotrójkątną, a wieć uzyskaliśmy to, czego oczekiwaliśmy, czyli
rozkład $QR$ macierzy $A$.
\subsection{Rozkład LU}
Rozkład $LU$ macierzy polega na przedstawieniu $A$ jako iloczynu macierzy
dolnotrójkątnej $L$ i górnotrójkątnej $U$. Dokładny opis postaci i metody
uzysiwania tej dekompozycji został podany na wykładzie i można go znaleźć w
odpowiadających mu notatkach. Warto jedynie podkreślić, że nie dla każdej
macierzy taki rozkład istnieje.
