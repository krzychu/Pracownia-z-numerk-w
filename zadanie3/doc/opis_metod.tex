\section{Opis metod}
W rozdziale tym wytłumaczona została idea dekompozycji QR oraz LU macierzy oraz
podane zostały metody do wyznaczaczania poszczególnych rozkładów, przy czym dla
dekompozycji QR są to metody zwane \textit{metodą Grama-Schmidta} i
\textit{metodą Householdera}.
\subsection{Rozkład QR}
Dekompozycja QR macierzy kwadratowej $A \in M_{n \times n}$ jest to rozkład
macierzy A na dwie macierze kwadratowe $Q \in M_{n \times n}$ i $R \in M_{n
\times n}$ takie, że:
$$A=QR,$$
przy czym Q jest macierzą ortogonalną ($QQ^T=I$), zaś R jest macierzą trójkątną
górną.

Ciekawą własnością tej dekompozycji jest to, że isnieje ona dla dowolnej
macierzy kwadratowej. Jeżeli przyjmiemy dodatkowo, że elementy diagonali
macierzy R są wszystkie nieujemne, rozkład taki jest wyznaczony jednoznacznie.
Dowód powyższej własności jest dowodem konstrukcjyjnym\footnote{Można go znależć
na stronie http://www.math4all.in/public\_html/linear\%20algebra/proof8.5.11.html}
i opiera sie o metodęorotgonalizacji Grama-Schmitda, której opis znajduje się w
dalszej częśći sprawozdania.

Istnieje również dekompozycja QR dla macierzy prostokątnych. Niech macierz
$A \in M_{n \times n}$, $m \geq n$. Wówczas rozkładamy macierz A na macierz
ortogonalną $Q \in M_{m \in m}$ i macierz trójkątną górną $R \in M_{m \in n}$.
Jak można zauważyć, dolne $m-n$ wierszy macierzy R są zerowe, stosuje się więc
podział macierzy Q i R na $R_1 \in M_{n \times n}$ i diw macierze posiadające
ortogonalne kolumny, mianowicie $Q_1 \in M_{m \times n}$ i $Q_2 \in M_{m, m-n}$.
Zachodzi równość:
$$A=QR=(Q_1, Q_2)=\begin{pmatrix} R_1 \\ 0 \end{pmatrix}=Q_1R_1.$$
W dalszych rozważąniac ograniczymy się jednak do macierz kwadratowych.
\subsubsection{Metoda Grama-Schmidta}
Jest to pierwsza rozważana przez nas metoda dekompozycji QR. Definiujemy
operator projekcji jako:
$$proj_u(v)=\frac{<v,u>}{<u,u>}u,$$
gdzie $<x,y>$ oznacza iloczyn skalarny wektorów x i y. Innymi słowy, projekcja
to rzut wektora v na prostą rozpinaną przez wektor u. Danymi wejściowymi
metody Grama-Schmidta są kolejne kolumny macierzy $A=[a_1,...,a_n]$.
Oznaczamy przez ${u_k}$ i ${e_k}, 1 \leq k \leq n,$ następujące wektory:
$$ u_k = a_k-\sum_{j=1}^{k-1} proj_{e_j}a_k, \hspace{2cm}
e_k = \frac{u_k}{\|u_k\|}.$$
Przekształcająć odpowiednio podane równości i kożystając z faktu, że
$<e_k, a_k> = \|u_k\|$ i wektory $e_k$ są wektorami jednostkowymi dla
$1 \leq k \leq n$ otrzymujemy, że:
$$ a_k = \sum_{i=1}^{k} <e_i,a_k>e_i.$$
Powyższą równość możemu zapisać w postaci macierzowej $A=QR$, które są
postaci jakiej oczekiwaliśmy. Macierz Q bowiem jest macierzą ortonormalną,
zaś R górnotrójkątną i możemy je zapisać w postaci:
$$ Q = [e_1,...,e_n], \hspace{1cm} R = \begin{pmatrix} \langle\ e_1,a_1
\rangle & \langle e_1,a_2\rangle & \langle e_1, a_3\rangle & \ldots \\ 0
& \langle e_2, a_2\rangle & \langle e_2, a_3\rangle & \ldots \\ 0 & 0 &
\langle e_3, a_3\rangle & \ldots \\ \vdots & \vdots & \vdots & \ddots
\end{pmatrix}.$$
\subsubsection{Metoda Householdera}
\subsection{Rozkład LU}
