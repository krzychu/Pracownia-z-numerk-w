\section{Znajdowanie macierzy odwrotnej}
\subsection{Rozkład QR}
Załóżmy, że mamy już obliczony rozkład macierzy $A=QR$. Wobec tego
odwrotność danej macierzy, możamy obliczyć wykorzystująć tożsamość:
$$A^{-1}=(QR)^{-1}= R^{-1}Q^{-1}.$$
Wystarczy więc, że obliczymy osobno odwrotnośći macierzy ortogonalnej $Q$ i
górnotrójkątnej $R$. Znaleznienie odwrotności macierzy $Q$ jest zadaniem
trywialnym. Wynika to z tego że, jest to transpozycja macierzy
$Q^{-1}=Q^{T}$, zadanie to sprowadza się więc do zamiany wierszy z
kolumnami.

Więcej, ale wciąż niewiele, trudnośći przysparza nam znalezienie odwrotności
macierzy górnotrójkątej $R$. Oznaczmy przez $r_{ij}$ elementy macierzy $R$,
zaś $x_ij$ elementy jej odwrotnośći, $1 \leq i,j \leq n$. Wówczas elementy
macierzy $R^{-1}$ możemy obliczyć w następujący sposób:
$$ x_{ij}=\left\{\begin{matrix} 0 & \mbox{jeśli } j<i \\
\frac{1}{r_{ij}} & \mbox{jeśli } i=j \\
-\frac{\sum_{k=0}^{j-i}a_{i,i+k}x_{i+k,j}}{a_{ii}} & \mbox{jeśli } j>i \end{matrix}\right.$$
co wynika z rozwiązania układu równań $RR^{-1}=I$.
\subsection{Rozkład LU}
Jeżeli mamy rozkład LU macierzy to, podobnie jak dla dekompozycji QR,
wykorzystujemy wzór na odwrotność iloczynu macierzy. Stąd $A^{-1}=U^{-1}L^{-1}$,
wystarczy więc umieć odwracać macierze trójkątne. To już jednak umiemy, odwróciliśmy
bowiem macierz R z poprzedniego podrozdziału. Odwrotność macierzy dolnotrójkątnej
uzyskuje się analognicznie do górnotrójkątnej.
