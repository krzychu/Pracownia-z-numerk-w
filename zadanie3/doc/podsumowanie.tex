\section{Podsumowanie}
Wkonane przez nas zadanie pozwoliło nam zapoznać się z metodami wykonywania rozkładu
$QR$, który wykorzystaliśmy do odwracania macierzy kwadratowych. Ponadto zaobserwowaliśmy,
że nie nie zawsze otrzymany wynik przy pomocy badanych metod jest bliski wynikowi oczekiwanemu.

Zbierzmy teraz zebrane przez nas obserwacje i wnioski. Przede wszystkim przypomnijmy, że wszystkie
badane perzez nas metody miały podobny czas działania, który nie zależy od struktury macierzy.
Wynika to z prostego faktu, zarówno metody \textit{HH}, \textit{GS}, jak i \textit{LU} mają taką
samą złożoność obliczeniową równą $O(n^3)$. W rozdziale \S5 opisaliśmy przeprowadzone przez nas
badania nad dokładnością metod, która okazuje się zależeć nie tylko od wielkości macierze, ale
także od jej struktury. Jeśli chodzi o rozkład $QR$, metoda \textit{HH} okazała się lepsza od
metody \textit{GS}. Wobec tego to właśnie tę metodę zestawiliśmy z algorytmam wykorzystującym
rozkłas $LU$ i z programem \textit{octave}. Dokładności uzyskane za pomocą metod \textit{HH} i
\textit{LU} okazały się zbliożone dla wszystkich badanych rodzajów macierzy, oprócz tych z zerującym
się monorem głównym. W tym przypadku ewidentnie dokładniejsza okazała się być metoda Householdera.
Zauważyliśmy ponadto, że obydwa algorytmy nie radzą sobie z odwracaniem macierzy Hilberta, błędy
uzyskane były katastrofalnie duże. Podobne obserwacje uzyskaliśmy prz porównywaniu naszych metod z
programem \textit{octave}. Satysfakcjonujące wyniki otrzymaliśmy dla macierzy losowych, z dominującą
przekątną, o wyznaczniku bliskim zeru i trójdiagonalnych. Ponownie metoda \textit{HH} dała lepsze wyniki dla
macierzy typu \textbf{6}. Liczony wskażnik był dla nich około $10^{12}$ razy mniejszy.
