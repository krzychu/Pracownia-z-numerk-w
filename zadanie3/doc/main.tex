\documentclass[12pt,a4paper,wide,authortitle]{mwart}

\usepackage[polish]{babel}%\usepackage{polski}
\usepackage[utf8]{inputenc}
\usepackage[OT4]{fontenc}
\usepackage{amsmath} %amssymb
\usepackage{amsthm}
\numberwithin{figure}{section}  
\numberwithin{table}{section}  
\linespread{1.3}
\usepackage{multirow}
\usepackage[small, bf]{caption}
\newtheorem{lm}{Lemat}
\usepackage{graphicx}
\usepackage{wrapfig}
\usepackage{subfigure}
\newcounter{myenum}
\newenvironment{myenumerate}{\setcounter{myenum}{1}\def\item{\par{\bf
      \arabic{myenum} - \ }\stepcounter{myenum}}}{\newline}

\begin{document}
\title{\LARGE \textbf{Pracownia z Analizy Numerycznej}\\Rozwiązanie zadania P3.14}
\author{Krzysztof Chrobak\thanks{\textit{E-mail}:
\texttt{krzychuch@gmail.com}} \and Aleksandra Spyra\thanks{\textit{E-mail}:
\texttt{aleksandra.spyra180@gmail.com}}}
\date {Wrocław, \today}
\maketitle
\tableofcontents
\newpage
\section{Wstęp}
Macierze to sposób reprezentacji danych i przekształceń, z którego w dzisiejszych
czasach nie korzystają wyłącznie matematycy. Szeroko stosowane są one
w informatyce, przykładowo w gafice komputerowej. Ważne jest aby wszystkie
obliczenia macierzowe realizować w sposób możliwie najszybszy i najdokładniejszy.
Wiele operacji znacznie upraszcza się, jeśli macierz przedstawimy jako
iloczyn macierzy prostszej postaci, dla których znamy algorytmy pozwalające
nam w efektywny sposób dokonywać obliczenia. Taki rozkład macierzy stanowi
dekompozycja $QR$, która przedstawia macierz jako iloczyn macierzy ortogonalnej
i macierzy trójkątnej górnej. Jak się okazuje, istnieją różne algorytmy,
wyznaczające taki rozkład, który jest cenny, ze względu na szerokie zastosowanie.

Głównym celem naszego zadania jest opracowanie metod służących do
znajdowania dekompozycji $QR$. Skupiamy naszą uwagę na dwóch algorytmach: \textit{metodzie Grama-Schmidta},
oznaczanej jako \textit{GS}
i \textit{metodzie Householdera},
którą będziemy skrótowo określać jako \textit{HH}. Stawiamy przed sobą problem, jak wyznaczyć
macierz odwrotną mając dany rozkład $QR$ macierzy. Dodatkowo postaramy się
porównać pod względem dokładności i szybkości oba algorytmy,
zestawiając je także z algorytmem wyznaczania odwrotności przy użyciu
dekompozycji LU, którą poznaliśmy podczas tegorocznych zajęć.

Rozdział \S2 zawiera definicje poszczególnych dekompozycji. Również w jego
obrębie znajduje się opis wykorzystanych przez nas metod Grama-Schmidta i Householdera.
Kolejna część zawiera opis sposobu, w jaki mając daną dekompozycję $QR$
uzyskać odwrotność wejściowej macierzy. Jak się okaże, jest to bardzo proste
w realizacji. Rozdział \S4 stanowi porównanie szybkości znajdowania
odwrotności za pomocą rozkładu LU oraz QR z wykorzystaniem obydwu
wymienionych powyżej metod. W następnym rozdziale zaś, zamieściliśmy
relację z doświadczeń, które pozwoliły nam zbadać dokładność wszystkich
trzech wspomnianych sposobów odwracania macierzy. Aby zrealizować
tę część zadania, wykorzystaliśmy nie tylko macierze generowane losowo, ale
także zrobiliśmy próby dla macierzy o szczególnych właściwościach,
przykładowo dla macierzy źle uwarunkowanych, czy macierzy trójdiagonalnych.
Porównaliśmy nasze algorytmy także z wynikami jakie generuje biblioteka \textit{numpy}.
Dane, które zostały zgromadzone podczas badań, postaraliśmy się
umiejętnie przedstawić przy pomocy wykresów i tabel.
Ostatnia część sprawozdania to swoiste zebranie wniosków i obserwacji
nagromadzonych podczas realizacji naszego zadania.

Implementację przedstawionych metod i algorytmów można odnaleźć w katalogu \textit{prog}.





\section{Opis metod}
\subsection{Rozkład QR}
\subsubsection{Metoda Grama-Schmidta}
\subsubsection{Metoda Householdera}
\subsection{Rozkład LU}

\section{Znajdowanie macierzy odwrotnej}
\subsection{Rozkład QR}
Załóżmy, że mamy już obliczony rozkład macierzy $A=QR$. Wobec tego
odwrotność danej macierzy, możamy obliczyć wykorzystująć tożsamość:
$$A^{-1}=(QR)^{-1}= R^{-1}Q^{-1}.$$
Wystarczy więc, że obliczymy osobno odwrotnośći macierzy ortogonalnej $Q$ i
górnotrójkątnej $R$. Znaleznienie odwrotności macierzy $Q$ jest zadaniem
trywialnym. Wynika to z tego że, jest to transpozycja macierzy
$Q^{-1}=Q^{T}$, zadanie to sprowadza się więc do zamiany wierszy z
kolumnami.

Więcej, ale wciąż niewiele, trudnośći przysparza nam znalezienie odwrotności
macierzy górnotrójkątej $R$. Oznaczmy przez $r_{ij}$ elementy macierzy $R$,
zaś $x_ij$ elementy jej odwrotnośći, $1 \leq i,j \leq n$. Wówczas elementy
macierzy $R^{-1}$ możemy obliczyć w następujący sposób:
$$ x_{ij}=\left\{\begin{matrix} 0 & \mbox{jeśli } j<i \\
\frac{1}{r_{ij}} & \mbox{jeśli } i=j \\
-\frac{\sum_{k=0}^{j-i}a_{i,i+k}x_{i+k,j}}{a_{ii}} & \mbox{jeśli } j>i \end{matrix}\right.$$
co wynika z rozwiązania układu równań $RR^{-1}=I$.
\subsection{Rozkład LU}
Jeżeli mamy rozkład LU macierzy to, podobnie jak dla dekompozycji QR,
wykorzystujemy wzór na odwrotność iloczynu macierzy. Stąd $A^{-1}=U^{-1}L^{-1}$,
wystarczy więc umieć odwracać macierze trójkątne. To już jednak umiemy, odwróciliśmy
bowiem macierz R z poprzedniego podrozdziału. Odwrotność macierzy dolnotrójkątnej
uzyskuje się analognicznie do górnotrójkątnej.
\section{Porównanie szybkości działania metod}
Aby zbadać, która medota spośród wspomnianych trzech (Grama-Schmidta, Householdera, czy
opierająca się o rozkład $LU$), najszybciej odwraca podaną jej na wejściu
macierz $A$, zmierzyliśmy czas ich działania dla macierzy losowych. Przedział
wielkośći losowanych macierzy ustaliliśmy na $[10, 200]$ z krokiem $5$. Mogliśmy
oprzeć się wyłącznie o wyniki dla losowo wygenerowanych danych, ze względu na to,
że czas działania każdego z badanych algorytmów nie zależy od postaci i
wewnętrznej struktury macierzy. Wyniki naszego doświadczenia obrazuje wykres z
rysunku \textbf{4.1}.
\begin{figure}[h!tb]
\begin{center}
% GNUPLOT: LaTeX picture
\setlength{\unitlength}{0.240900pt}
\ifx\plotpoint\undefined\newsavebox{\plotpoint}\fi
\sbox{\plotpoint}{\rule[-0.200pt]{0.400pt}{0.400pt}}%
\begin{picture}(1500,900)(0,0)
\sbox{\plotpoint}{\rule[-0.200pt]{0.400pt}{0.400pt}}%
\put(190.0,82.0){\rule[-0.200pt]{4.818pt}{0.400pt}}
\put(170,82){\makebox(0,0)[r]{ 0}}
\put(1419.0,82.0){\rule[-0.200pt]{4.818pt}{0.400pt}}
\put(190.0,211.0){\rule[-0.200pt]{4.818pt}{0.400pt}}
\put(170,211){\makebox(0,0)[r]{ 100000}}
\put(1419.0,211.0){\rule[-0.200pt]{4.818pt}{0.400pt}}
\put(190.0,341.0){\rule[-0.200pt]{4.818pt}{0.400pt}}
\put(170,341){\makebox(0,0)[r]{ 200000}}
\put(1419.0,341.0){\rule[-0.200pt]{4.818pt}{0.400pt}}
\put(190.0,470.0){\rule[-0.200pt]{4.818pt}{0.400pt}}
\put(170,470){\makebox(0,0)[r]{ 300000}}
\put(1419.0,470.0){\rule[-0.200pt]{4.818pt}{0.400pt}}
\put(190.0,600.0){\rule[-0.200pt]{4.818pt}{0.400pt}}
\put(170,600){\makebox(0,0)[r]{ 400000}}
\put(1419.0,600.0){\rule[-0.200pt]{4.818pt}{0.400pt}}
\put(190.0,729.0){\rule[-0.200pt]{4.818pt}{0.400pt}}
\put(170,729){\makebox(0,0)[r]{ 500000}}
\put(1419.0,729.0){\rule[-0.200pt]{4.818pt}{0.400pt}}
\put(190.0,859.0){\rule[-0.200pt]{4.818pt}{0.400pt}}
\put(170,859){\makebox(0,0)[r]{ 600000}}
\put(1419.0,859.0){\rule[-0.200pt]{4.818pt}{0.400pt}}
\put(190.0,82.0){\rule[-0.200pt]{0.400pt}{4.818pt}}
\put(190,41){\makebox(0,0){ 0}}
\put(190.0,839.0){\rule[-0.200pt]{0.400pt}{4.818pt}}
\put(502.0,82.0){\rule[-0.200pt]{0.400pt}{4.818pt}}
\put(502,41){\makebox(0,0){ 50}}
\put(502.0,839.0){\rule[-0.200pt]{0.400pt}{4.818pt}}
\put(815.0,82.0){\rule[-0.200pt]{0.400pt}{4.818pt}}
\put(815,41){\makebox(0,0){ 100}}
\put(815.0,839.0){\rule[-0.200pt]{0.400pt}{4.818pt}}
\put(1127.0,82.0){\rule[-0.200pt]{0.400pt}{4.818pt}}
\put(1127,41){\makebox(0,0){ 150}}
\put(1127.0,839.0){\rule[-0.200pt]{0.400pt}{4.818pt}}
\put(1439.0,82.0){\rule[-0.200pt]{0.400pt}{4.818pt}}
\put(1439,41){\makebox(0,0){ 200}}
\put(1439.0,839.0){\rule[-0.200pt]{0.400pt}{4.818pt}}
\put(190.0,82.0){\rule[-0.200pt]{0.400pt}{187.179pt}}
\put(190.0,82.0){\rule[-0.200pt]{300.884pt}{0.400pt}}
\put(1439.0,82.0){\rule[-0.200pt]{0.400pt}{187.179pt}}
\put(190.0,859.0){\rule[-0.200pt]{300.884pt}{0.400pt}}
\put(1279,205){\makebox(0,0)[r]{LU}}
\put(1299.0,205.0){\rule[-0.200pt]{24.090pt}{0.400pt}}
\put(252,82){\usebox{\plotpoint}}
\put(276,81.67){\rule{2.891pt}{0.400pt}}
\multiput(276.00,81.17)(6.000,1.000){2}{\rule{1.445pt}{0.400pt}}
\put(252.0,82.0){\rule[-0.200pt]{5.782pt}{0.400pt}}
\put(324,82.67){\rule{2.891pt}{0.400pt}}
\multiput(324.00,82.17)(6.000,1.000){2}{\rule{1.445pt}{0.400pt}}
\put(288.0,83.0){\rule[-0.200pt]{8.672pt}{0.400pt}}
\put(360,83.67){\rule{2.891pt}{0.400pt}}
\multiput(360.00,83.17)(6.000,1.000){2}{\rule{1.445pt}{0.400pt}}
\put(336.0,84.0){\rule[-0.200pt]{5.782pt}{0.400pt}}
\put(384,84.67){\rule{2.891pt}{0.400pt}}
\multiput(384.00,84.17)(6.000,1.000){2}{\rule{1.445pt}{0.400pt}}
\put(396,85.67){\rule{2.891pt}{0.400pt}}
\multiput(396.00,85.17)(6.000,1.000){2}{\rule{1.445pt}{0.400pt}}
\put(372.0,85.0){\rule[-0.200pt]{2.891pt}{0.400pt}}
\put(420,86.67){\rule{2.891pt}{0.400pt}}
\multiput(420.00,86.17)(6.000,1.000){2}{\rule{1.445pt}{0.400pt}}
\put(432,87.67){\rule{2.891pt}{0.400pt}}
\multiput(432.00,87.17)(6.000,1.000){2}{\rule{1.445pt}{0.400pt}}
\put(444,88.67){\rule{2.891pt}{0.400pt}}
\multiput(444.00,88.17)(6.000,1.000){2}{\rule{1.445pt}{0.400pt}}
\put(456,89.67){\rule{2.891pt}{0.400pt}}
\multiput(456.00,89.17)(6.000,1.000){2}{\rule{1.445pt}{0.400pt}}
\put(468,90.67){\rule{2.891pt}{0.400pt}}
\multiput(468.00,90.17)(6.000,1.000){2}{\rule{1.445pt}{0.400pt}}
\put(480,91.67){\rule{2.891pt}{0.400pt}}
\multiput(480.00,91.17)(6.000,1.000){2}{\rule{1.445pt}{0.400pt}}
\put(492,92.67){\rule{2.891pt}{0.400pt}}
\multiput(492.00,92.17)(6.000,1.000){2}{\rule{1.445pt}{0.400pt}}
\put(504,93.67){\rule{2.891pt}{0.400pt}}
\multiput(504.00,93.17)(6.000,1.000){2}{\rule{1.445pt}{0.400pt}}
\put(516,94.67){\rule{2.891pt}{0.400pt}}
\multiput(516.00,94.17)(6.000,1.000){2}{\rule{1.445pt}{0.400pt}}
\put(528,96.17){\rule{2.500pt}{0.400pt}}
\multiput(528.00,95.17)(6.811,2.000){2}{\rule{1.250pt}{0.400pt}}
\put(540,97.67){\rule{2.891pt}{0.400pt}}
\multiput(540.00,97.17)(6.000,1.000){2}{\rule{1.445pt}{0.400pt}}
\put(552,99.17){\rule{2.500pt}{0.400pt}}
\multiput(552.00,98.17)(6.811,2.000){2}{\rule{1.250pt}{0.400pt}}
\put(564,101.17){\rule{2.500pt}{0.400pt}}
\multiput(564.00,100.17)(6.811,2.000){2}{\rule{1.250pt}{0.400pt}}
\put(576,103.17){\rule{2.500pt}{0.400pt}}
\multiput(576.00,102.17)(6.811,2.000){2}{\rule{1.250pt}{0.400pt}}
\put(588,105.17){\rule{2.500pt}{0.400pt}}
\multiput(588.00,104.17)(6.811,2.000){2}{\rule{1.250pt}{0.400pt}}
\put(600,107.17){\rule{2.500pt}{0.400pt}}
\multiput(600.00,106.17)(6.811,2.000){2}{\rule{1.250pt}{0.400pt}}
\put(612,109.17){\rule{2.500pt}{0.400pt}}
\multiput(612.00,108.17)(6.811,2.000){2}{\rule{1.250pt}{0.400pt}}
\put(624,111.17){\rule{2.500pt}{0.400pt}}
\multiput(624.00,110.17)(6.811,2.000){2}{\rule{1.250pt}{0.400pt}}
\put(636,113.17){\rule{2.500pt}{0.400pt}}
\multiput(636.00,112.17)(6.811,2.000){2}{\rule{1.250pt}{0.400pt}}
\multiput(648.00,115.61)(2.472,0.447){3}{\rule{1.700pt}{0.108pt}}
\multiput(648.00,114.17)(8.472,3.000){2}{\rule{0.850pt}{0.400pt}}
\multiput(660.00,118.61)(2.472,0.447){3}{\rule{1.700pt}{0.108pt}}
\multiput(660.00,117.17)(8.472,3.000){2}{\rule{0.850pt}{0.400pt}}
\put(672,121.17){\rule{2.500pt}{0.400pt}}
\multiput(672.00,120.17)(6.811,2.000){2}{\rule{1.250pt}{0.400pt}}
\multiput(684.00,123.61)(2.472,0.447){3}{\rule{1.700pt}{0.108pt}}
\multiput(684.00,122.17)(8.472,3.000){2}{\rule{0.850pt}{0.400pt}}
\multiput(696.00,126.61)(2.472,0.447){3}{\rule{1.700pt}{0.108pt}}
\multiput(696.00,125.17)(8.472,3.000){2}{\rule{0.850pt}{0.400pt}}
\multiput(708.00,129.60)(1.651,0.468){5}{\rule{1.300pt}{0.113pt}}
\multiput(708.00,128.17)(9.302,4.000){2}{\rule{0.650pt}{0.400pt}}
\multiput(720.00,133.61)(2.472,0.447){3}{\rule{1.700pt}{0.108pt}}
\multiput(720.00,132.17)(8.472,3.000){2}{\rule{0.850pt}{0.400pt}}
\multiput(732.00,136.61)(2.472,0.447){3}{\rule{1.700pt}{0.108pt}}
\multiput(732.00,135.17)(8.472,3.000){2}{\rule{0.850pt}{0.400pt}}
\multiput(744.00,139.60)(1.651,0.468){5}{\rule{1.300pt}{0.113pt}}
\multiput(744.00,138.17)(9.302,4.000){2}{\rule{0.650pt}{0.400pt}}
\multiput(756.00,143.60)(1.651,0.468){5}{\rule{1.300pt}{0.113pt}}
\multiput(756.00,142.17)(9.302,4.000){2}{\rule{0.650pt}{0.400pt}}
\multiput(768.00,147.60)(1.651,0.468){5}{\rule{1.300pt}{0.113pt}}
\multiput(768.00,146.17)(9.302,4.000){2}{\rule{0.650pt}{0.400pt}}
\multiput(780.00,151.60)(1.651,0.468){5}{\rule{1.300pt}{0.113pt}}
\multiput(780.00,150.17)(9.302,4.000){2}{\rule{0.650pt}{0.400pt}}
\multiput(792.00,155.60)(1.651,0.468){5}{\rule{1.300pt}{0.113pt}}
\multiput(792.00,154.17)(9.302,4.000){2}{\rule{0.650pt}{0.400pt}}
\multiput(804.00,159.59)(1.267,0.477){7}{\rule{1.060pt}{0.115pt}}
\multiput(804.00,158.17)(9.800,5.000){2}{\rule{0.530pt}{0.400pt}}
\multiput(816.00,164.60)(1.651,0.468){5}{\rule{1.300pt}{0.113pt}}
\multiput(816.00,163.17)(9.302,4.000){2}{\rule{0.650pt}{0.400pt}}
\multiput(828.00,168.59)(1.267,0.477){7}{\rule{1.060pt}{0.115pt}}
\multiput(828.00,167.17)(9.800,5.000){2}{\rule{0.530pt}{0.400pt}}
\multiput(840.00,173.59)(1.267,0.477){7}{\rule{1.060pt}{0.115pt}}
\multiput(840.00,172.17)(9.800,5.000){2}{\rule{0.530pt}{0.400pt}}
\multiput(852.00,178.59)(1.267,0.477){7}{\rule{1.060pt}{0.115pt}}
\multiput(852.00,177.17)(9.800,5.000){2}{\rule{0.530pt}{0.400pt}}
\multiput(864.00,183.59)(1.267,0.477){7}{\rule{1.060pt}{0.115pt}}
\multiput(864.00,182.17)(9.800,5.000){2}{\rule{0.530pt}{0.400pt}}
\multiput(876.00,188.59)(1.033,0.482){9}{\rule{0.900pt}{0.116pt}}
\multiput(876.00,187.17)(10.132,6.000){2}{\rule{0.450pt}{0.400pt}}
\multiput(888.00,194.59)(1.267,0.477){7}{\rule{1.060pt}{0.115pt}}
\multiput(888.00,193.17)(9.800,5.000){2}{\rule{0.530pt}{0.400pt}}
\multiput(900.00,199.59)(1.033,0.482){9}{\rule{0.900pt}{0.116pt}}
\multiput(900.00,198.17)(10.132,6.000){2}{\rule{0.450pt}{0.400pt}}
\multiput(912.00,205.59)(1.033,0.482){9}{\rule{0.900pt}{0.116pt}}
\multiput(912.00,204.17)(10.132,6.000){2}{\rule{0.450pt}{0.400pt}}
\multiput(924.00,211.59)(1.033,0.482){9}{\rule{0.900pt}{0.116pt}}
\multiput(924.00,210.17)(10.132,6.000){2}{\rule{0.450pt}{0.400pt}}
\multiput(936.00,217.59)(1.033,0.482){9}{\rule{0.900pt}{0.116pt}}
\multiput(936.00,216.17)(10.132,6.000){2}{\rule{0.450pt}{0.400pt}}
\multiput(948.00,223.59)(0.874,0.485){11}{\rule{0.786pt}{0.117pt}}
\multiput(948.00,222.17)(10.369,7.000){2}{\rule{0.393pt}{0.400pt}}
\multiput(960.00,230.59)(1.033,0.482){9}{\rule{0.900pt}{0.116pt}}
\multiput(960.00,229.17)(10.132,6.000){2}{\rule{0.450pt}{0.400pt}}
\multiput(972.00,236.59)(0.874,0.485){11}{\rule{0.786pt}{0.117pt}}
\multiput(972.00,235.17)(10.369,7.000){2}{\rule{0.393pt}{0.400pt}}
\multiput(984.00,243.59)(0.874,0.485){11}{\rule{0.786pt}{0.117pt}}
\multiput(984.00,242.17)(10.369,7.000){2}{\rule{0.393pt}{0.400pt}}
\multiput(996.00,250.59)(0.874,0.485){11}{\rule{0.786pt}{0.117pt}}
\multiput(996.00,249.17)(10.369,7.000){2}{\rule{0.393pt}{0.400pt}}
\multiput(1008.00,257.59)(0.874,0.485){11}{\rule{0.786pt}{0.117pt}}
\multiput(1008.00,256.17)(10.369,7.000){2}{\rule{0.393pt}{0.400pt}}
\multiput(1020.00,264.59)(0.692,0.488){13}{\rule{0.650pt}{0.117pt}}
\multiput(1020.00,263.17)(9.651,8.000){2}{\rule{0.325pt}{0.400pt}}
\multiput(1031.00,272.59)(0.758,0.488){13}{\rule{0.700pt}{0.117pt}}
\multiput(1031.00,271.17)(10.547,8.000){2}{\rule{0.350pt}{0.400pt}}
\multiput(1043.00,280.59)(0.874,0.485){11}{\rule{0.786pt}{0.117pt}}
\multiput(1043.00,279.17)(10.369,7.000){2}{\rule{0.393pt}{0.400pt}}
\multiput(1055.00,287.59)(0.758,0.488){13}{\rule{0.700pt}{0.117pt}}
\multiput(1055.00,286.17)(10.547,8.000){2}{\rule{0.350pt}{0.400pt}}
\multiput(1067.00,295.59)(0.669,0.489){15}{\rule{0.633pt}{0.118pt}}
\multiput(1067.00,294.17)(10.685,9.000){2}{\rule{0.317pt}{0.400pt}}
\multiput(1079.00,304.59)(0.758,0.488){13}{\rule{0.700pt}{0.117pt}}
\multiput(1079.00,303.17)(10.547,8.000){2}{\rule{0.350pt}{0.400pt}}
\multiput(1091.00,312.59)(0.669,0.489){15}{\rule{0.633pt}{0.118pt}}
\multiput(1091.00,311.17)(10.685,9.000){2}{\rule{0.317pt}{0.400pt}}
\multiput(1103.00,321.59)(0.669,0.489){15}{\rule{0.633pt}{0.118pt}}
\multiput(1103.00,320.17)(10.685,9.000){2}{\rule{0.317pt}{0.400pt}}
\multiput(1115.00,330.59)(0.669,0.489){15}{\rule{0.633pt}{0.118pt}}
\multiput(1115.00,329.17)(10.685,9.000){2}{\rule{0.317pt}{0.400pt}}
\multiput(1127.00,339.58)(0.600,0.491){17}{\rule{0.580pt}{0.118pt}}
\multiput(1127.00,338.17)(10.796,10.000){2}{\rule{0.290pt}{0.400pt}}
\multiput(1139.00,349.58)(0.600,0.491){17}{\rule{0.580pt}{0.118pt}}
\multiput(1139.00,348.17)(10.796,10.000){2}{\rule{0.290pt}{0.400pt}}
\multiput(1151.00,359.58)(0.600,0.491){17}{\rule{0.580pt}{0.118pt}}
\multiput(1151.00,358.17)(10.796,10.000){2}{\rule{0.290pt}{0.400pt}}
\multiput(1163.00,369.58)(0.600,0.491){17}{\rule{0.580pt}{0.118pt}}
\multiput(1163.00,368.17)(10.796,10.000){2}{\rule{0.290pt}{0.400pt}}
\multiput(1175.00,379.58)(0.600,0.491){17}{\rule{0.580pt}{0.118pt}}
\multiput(1175.00,378.17)(10.796,10.000){2}{\rule{0.290pt}{0.400pt}}
\multiput(1187.00,389.58)(0.543,0.492){19}{\rule{0.536pt}{0.118pt}}
\multiput(1187.00,388.17)(10.887,11.000){2}{\rule{0.268pt}{0.400pt}}
\multiput(1199.00,400.58)(0.543,0.492){19}{\rule{0.536pt}{0.118pt}}
\multiput(1199.00,399.17)(10.887,11.000){2}{\rule{0.268pt}{0.400pt}}
\multiput(1211.00,411.58)(0.543,0.492){19}{\rule{0.536pt}{0.118pt}}
\multiput(1211.00,410.17)(10.887,11.000){2}{\rule{0.268pt}{0.400pt}}
\multiput(1223.00,422.58)(0.496,0.492){21}{\rule{0.500pt}{0.119pt}}
\multiput(1223.00,421.17)(10.962,12.000){2}{\rule{0.250pt}{0.400pt}}
\multiput(1235.00,434.58)(0.496,0.492){21}{\rule{0.500pt}{0.119pt}}
\multiput(1235.00,433.17)(10.962,12.000){2}{\rule{0.250pt}{0.400pt}}
\multiput(1247.00,446.58)(0.496,0.492){21}{\rule{0.500pt}{0.119pt}}
\multiput(1247.00,445.17)(10.962,12.000){2}{\rule{0.250pt}{0.400pt}}
\multiput(1259.00,458.58)(0.496,0.492){21}{\rule{0.500pt}{0.119pt}}
\multiput(1259.00,457.17)(10.962,12.000){2}{\rule{0.250pt}{0.400pt}}
\multiput(1271.00,470.58)(0.496,0.492){21}{\rule{0.500pt}{0.119pt}}
\multiput(1271.00,469.17)(10.962,12.000){2}{\rule{0.250pt}{0.400pt}}
\multiput(1283.58,482.00)(0.492,0.539){21}{\rule{0.119pt}{0.533pt}}
\multiput(1282.17,482.00)(12.000,11.893){2}{\rule{0.400pt}{0.267pt}}
\multiput(1295.58,495.00)(0.492,0.539){21}{\rule{0.119pt}{0.533pt}}
\multiput(1294.17,495.00)(12.000,11.893){2}{\rule{0.400pt}{0.267pt}}
\multiput(1307.58,508.00)(0.492,0.582){21}{\rule{0.119pt}{0.567pt}}
\multiput(1306.17,508.00)(12.000,12.824){2}{\rule{0.400pt}{0.283pt}}
\multiput(1319.58,522.00)(0.492,0.539){21}{\rule{0.119pt}{0.533pt}}
\multiput(1318.17,522.00)(12.000,11.893){2}{\rule{0.400pt}{0.267pt}}
\multiput(1331.58,535.00)(0.492,0.582){21}{\rule{0.119pt}{0.567pt}}
\multiput(1330.17,535.00)(12.000,12.824){2}{\rule{0.400pt}{0.283pt}}
\multiput(1343.58,549.00)(0.492,0.582){21}{\rule{0.119pt}{0.567pt}}
\multiput(1342.17,549.00)(12.000,12.824){2}{\rule{0.400pt}{0.283pt}}
\multiput(1355.58,563.00)(0.492,0.625){21}{\rule{0.119pt}{0.600pt}}
\multiput(1354.17,563.00)(12.000,13.755){2}{\rule{0.400pt}{0.300pt}}
\multiput(1367.58,578.00)(0.492,0.625){21}{\rule{0.119pt}{0.600pt}}
\multiput(1366.17,578.00)(12.000,13.755){2}{\rule{0.400pt}{0.300pt}}
\multiput(1379.58,593.00)(0.492,0.625){21}{\rule{0.119pt}{0.600pt}}
\multiput(1378.17,593.00)(12.000,13.755){2}{\rule{0.400pt}{0.300pt}}
\multiput(1391.58,608.00)(0.492,0.625){21}{\rule{0.119pt}{0.600pt}}
\multiput(1390.17,608.00)(12.000,13.755){2}{\rule{0.400pt}{0.300pt}}
\multiput(1403.58,623.00)(0.492,0.669){21}{\rule{0.119pt}{0.633pt}}
\multiput(1402.17,623.00)(12.000,14.685){2}{\rule{0.400pt}{0.317pt}}
\multiput(1415.58,639.00)(0.492,0.669){21}{\rule{0.119pt}{0.633pt}}
\multiput(1414.17,639.00)(12.000,14.685){2}{\rule{0.400pt}{0.317pt}}
\multiput(1427.58,655.00)(0.492,0.669){21}{\rule{0.119pt}{0.633pt}}
\multiput(1426.17,655.00)(12.000,14.685){2}{\rule{0.400pt}{0.317pt}}
\put(408.0,87.0){\rule[-0.200pt]{2.891pt}{0.400pt}}
\sbox{\plotpoint}{\rule[-0.500pt]{1.000pt}{1.000pt}}%
\sbox{\plotpoint}{\rule[-0.200pt]{0.400pt}{0.400pt}}%
\put(1279,164){\makebox(0,0)[r]{GS}}
\sbox{\plotpoint}{\rule[-0.500pt]{1.000pt}{1.000pt}}%
\multiput(1299,164)(20.756,0.000){5}{\usebox{\plotpoint}}
\put(1399,164){\usebox{\plotpoint}}
\put(252,82){\usebox{\plotpoint}}
\put(252.00,82.00){\usebox{\plotpoint}}
\put(272.76,82.00){\usebox{\plotpoint}}
\put(293.47,83.00){\usebox{\plotpoint}}
\put(314.22,83.00){\usebox{\plotpoint}}
\put(334.94,83.91){\usebox{\plotpoint}}
\put(355.67,84.64){\usebox{\plotpoint}}
\put(376.39,85.37){\usebox{\plotpoint}}
\put(397.12,86.09){\usebox{\plotpoint}}
\put(417.80,87.82){\usebox{\plotpoint}}
\put(438.49,89.54){\usebox{\plotpoint}}
\put(459.17,91.26){\usebox{\plotpoint}}
\put(479.85,92.99){\usebox{\plotpoint}}
\put(500.41,95.70){\usebox{\plotpoint}}
\put(520.92,98.82){\usebox{\plotpoint}}
\put(541.52,101.25){\usebox{\plotpoint}}
\put(561.99,104.67){\usebox{\plotpoint}}
\put(582.36,108.59){\usebox{\plotpoint}}
\put(602.74,112.46){\usebox{\plotpoint}}
\put(623.03,116.76){\usebox{\plotpoint}}
\put(643.16,121.79){\usebox{\plotpoint}}
\put(663.30,126.82){\usebox{\plotpoint}}
\put(683.43,131.86){\usebox{\plotpoint}}
\put(703.14,138.38){\usebox{\plotpoint}}
\put(723.09,144.03){\usebox{\plotpoint}}
\put(742.78,150.59){\usebox{\plotpoint}}
\put(762.14,158.05){\usebox{\plotpoint}}
\put(781.46,165.61){\usebox{\plotpoint}}
\put(800.62,173.59){\usebox{\plotpoint}}
\put(819.77,181.57){\usebox{\plotpoint}}
\put(838.59,190.30){\usebox{\plotpoint}}
\put(857.16,199.58){\usebox{\plotpoint}}
\put(875.72,208.86){\usebox{\plotpoint}}
\put(894.07,218.54){\usebox{\plotpoint}}
\put(912.00,229.00){\usebox{\plotpoint}}
\put(929.93,239.46){\usebox{\plotpoint}}
\put(947.86,249.92){\usebox{\plotpoint}}
\put(965.13,261.42){\usebox{\plotpoint}}
\put(982.40,272.93){\usebox{\plotpoint}}
\put(999.07,285.30){\usebox{\plotpoint}}
\put(1015.67,297.75){\usebox{\plotpoint}}
\put(1031.87,310.72){\usebox{\plotpoint}}
\put(1047.81,324.01){\usebox{\plotpoint}}
\put(1063.76,337.30){\usebox{\plotpoint}}
\put(1079.68,350.62){\usebox{\plotpoint}}
\put(1094.98,364.64){\usebox{\plotpoint}}
\put(1110.28,378.67){\usebox{\plotpoint}}
\put(1125.14,393.14){\usebox{\plotpoint}}
\put(1139.82,407.82){\usebox{\plotpoint}}
\put(1154.35,422.63){\usebox{\plotpoint}}
\put(1168.43,437.89){\usebox{\plotpoint}}
\put(1182.51,453.14){\usebox{\plotpoint}}
\put(1196.59,468.39){\usebox{\plotpoint}}
\put(1210.19,484.06){\usebox{\plotpoint}}
\put(1224.14,499.42){\usebox{\plotpoint}}
\put(1237.19,515.56){\usebox{\plotpoint}}
\put(1250.55,531.44){\usebox{\plotpoint}}
\put(1263.52,547.65){\usebox{\plotpoint}}
\put(1276.48,563.86){\usebox{\plotpoint}}
\put(1289.20,580.26){\usebox{\plotpoint}}
\put(1301.65,596.86){\usebox{\plotpoint}}
\put(1314.10,613.47){\usebox{\plotpoint}}
\put(1326.26,630.29){\usebox{\plotpoint}}
\put(1337.96,647.43){\usebox{\plotpoint}}
\put(1349.47,664.70){\usebox{\plotpoint}}
\put(1360.98,681.97){\usebox{\plotpoint}}
\put(1372.29,699.37){\usebox{\plotpoint}}
\put(1383.21,717.02){\usebox{\plotpoint}}
\put(1393.89,734.82){\usebox{\plotpoint}}
\put(1404.57,752.62){\usebox{\plotpoint}}
\multiput(1415,770)(11.083,17.549){2}{\usebox{\plotpoint}}
\put(1437.04,805.74){\usebox{\plotpoint}}
\put(1439,809){\usebox{\plotpoint}}
\sbox{\plotpoint}{\rule[-0.400pt]{0.800pt}{0.800pt}}%
\sbox{\plotpoint}{\rule[-0.200pt]{0.400pt}{0.400pt}}%
\put(1279,123){\makebox(0,0)[r]{HH}}
\sbox{\plotpoint}{\rule[-0.400pt]{0.800pt}{0.800pt}}%
\put(1299.0,123.0){\rule[-0.400pt]{24.090pt}{0.800pt}}
\put(252,82){\usebox{\plotpoint}}
\put(276,80.84){\rule{2.891pt}{0.800pt}}
\multiput(276.00,80.34)(6.000,1.000){2}{\rule{1.445pt}{0.800pt}}
\put(252.0,82.0){\rule[-0.400pt]{5.782pt}{0.800pt}}
\put(324,81.84){\rule{2.891pt}{0.800pt}}
\multiput(324.00,81.34)(6.000,1.000){2}{\rule{1.445pt}{0.800pt}}
\put(288.0,83.0){\rule[-0.400pt]{8.672pt}{0.800pt}}
\put(348,82.84){\rule{2.891pt}{0.800pt}}
\multiput(348.00,82.34)(6.000,1.000){2}{\rule{1.445pt}{0.800pt}}
\put(336.0,84.0){\rule[-0.400pt]{2.891pt}{0.800pt}}
\put(372,83.84){\rule{2.891pt}{0.800pt}}
\multiput(372.00,83.34)(6.000,1.000){2}{\rule{1.445pt}{0.800pt}}
\put(384,84.84){\rule{2.891pt}{0.800pt}}
\multiput(384.00,84.34)(6.000,1.000){2}{\rule{1.445pt}{0.800pt}}
\put(360.0,85.0){\rule[-0.400pt]{2.891pt}{0.800pt}}
\put(408,85.84){\rule{2.891pt}{0.800pt}}
\multiput(408.00,85.34)(6.000,1.000){2}{\rule{1.445pt}{0.800pt}}
\put(420,86.84){\rule{2.891pt}{0.800pt}}
\multiput(420.00,86.34)(6.000,1.000){2}{\rule{1.445pt}{0.800pt}}
\put(432,87.84){\rule{2.891pt}{0.800pt}}
\multiput(432.00,87.34)(6.000,1.000){2}{\rule{1.445pt}{0.800pt}}
\put(444,88.84){\rule{2.891pt}{0.800pt}}
\multiput(444.00,88.34)(6.000,1.000){2}{\rule{1.445pt}{0.800pt}}
\put(456,89.84){\rule{2.891pt}{0.800pt}}
\multiput(456.00,89.34)(6.000,1.000){2}{\rule{1.445pt}{0.800pt}}
\put(468,91.34){\rule{2.891pt}{0.800pt}}
\multiput(468.00,90.34)(6.000,2.000){2}{\rule{1.445pt}{0.800pt}}
\put(480,92.84){\rule{2.891pt}{0.800pt}}
\multiput(480.00,92.34)(6.000,1.000){2}{\rule{1.445pt}{0.800pt}}
\put(492,94.34){\rule{2.891pt}{0.800pt}}
\multiput(492.00,93.34)(6.000,2.000){2}{\rule{1.445pt}{0.800pt}}
\put(504,95.84){\rule{2.891pt}{0.800pt}}
\multiput(504.00,95.34)(6.000,1.000){2}{\rule{1.445pt}{0.800pt}}
\put(516,97.34){\rule{2.891pt}{0.800pt}}
\multiput(516.00,96.34)(6.000,2.000){2}{\rule{1.445pt}{0.800pt}}
\put(528,99.34){\rule{2.891pt}{0.800pt}}
\multiput(528.00,98.34)(6.000,2.000){2}{\rule{1.445pt}{0.800pt}}
\put(540,101.34){\rule{2.891pt}{0.800pt}}
\multiput(540.00,100.34)(6.000,2.000){2}{\rule{1.445pt}{0.800pt}}
\put(552,103.34){\rule{2.891pt}{0.800pt}}
\multiput(552.00,102.34)(6.000,2.000){2}{\rule{1.445pt}{0.800pt}}
\put(564,105.34){\rule{2.891pt}{0.800pt}}
\multiput(564.00,104.34)(6.000,2.000){2}{\rule{1.445pt}{0.800pt}}
\put(576,107.34){\rule{2.891pt}{0.800pt}}
\multiput(576.00,106.34)(6.000,2.000){2}{\rule{1.445pt}{0.800pt}}
\put(588,109.84){\rule{2.891pt}{0.800pt}}
\multiput(588.00,108.34)(6.000,3.000){2}{\rule{1.445pt}{0.800pt}}
\put(600,112.34){\rule{2.891pt}{0.800pt}}
\multiput(600.00,111.34)(6.000,2.000){2}{\rule{1.445pt}{0.800pt}}
\put(612,114.84){\rule{2.891pt}{0.800pt}}
\multiput(612.00,113.34)(6.000,3.000){2}{\rule{1.445pt}{0.800pt}}
\put(624,117.84){\rule{2.891pt}{0.800pt}}
\multiput(624.00,116.34)(6.000,3.000){2}{\rule{1.445pt}{0.800pt}}
\put(636,120.84){\rule{2.891pt}{0.800pt}}
\multiput(636.00,119.34)(6.000,3.000){2}{\rule{1.445pt}{0.800pt}}
\put(648,123.84){\rule{2.891pt}{0.800pt}}
\multiput(648.00,122.34)(6.000,3.000){2}{\rule{1.445pt}{0.800pt}}
\put(660,126.84){\rule{2.891pt}{0.800pt}}
\multiput(660.00,125.34)(6.000,3.000){2}{\rule{1.445pt}{0.800pt}}
\put(672,130.34){\rule{2.600pt}{0.800pt}}
\multiput(672.00,128.34)(6.604,4.000){2}{\rule{1.300pt}{0.800pt}}
\put(684,133.84){\rule{2.891pt}{0.800pt}}
\multiput(684.00,132.34)(6.000,3.000){2}{\rule{1.445pt}{0.800pt}}
\put(696,137.34){\rule{2.600pt}{0.800pt}}
\multiput(696.00,135.34)(6.604,4.000){2}{\rule{1.300pt}{0.800pt}}
\put(708,141.34){\rule{2.600pt}{0.800pt}}
\multiput(708.00,139.34)(6.604,4.000){2}{\rule{1.300pt}{0.800pt}}
\put(720,145.34){\rule{2.600pt}{0.800pt}}
\multiput(720.00,143.34)(6.604,4.000){2}{\rule{1.300pt}{0.800pt}}
\put(732,149.34){\rule{2.600pt}{0.800pt}}
\multiput(732.00,147.34)(6.604,4.000){2}{\rule{1.300pt}{0.800pt}}
\multiput(744.00,154.38)(1.600,0.560){3}{\rule{2.120pt}{0.135pt}}
\multiput(744.00,151.34)(7.600,5.000){2}{\rule{1.060pt}{0.800pt}}
\put(756,158.34){\rule{2.600pt}{0.800pt}}
\multiput(756.00,156.34)(6.604,4.000){2}{\rule{1.300pt}{0.800pt}}
\multiput(768.00,163.38)(1.600,0.560){3}{\rule{2.120pt}{0.135pt}}
\multiput(768.00,160.34)(7.600,5.000){2}{\rule{1.060pt}{0.800pt}}
\multiput(780.00,168.38)(1.600,0.560){3}{\rule{2.120pt}{0.135pt}}
\multiput(780.00,165.34)(7.600,5.000){2}{\rule{1.060pt}{0.800pt}}
\multiput(792.00,173.38)(1.600,0.560){3}{\rule{2.120pt}{0.135pt}}
\multiput(792.00,170.34)(7.600,5.000){2}{\rule{1.060pt}{0.800pt}}
\multiput(804.00,178.39)(1.132,0.536){5}{\rule{1.800pt}{0.129pt}}
\multiput(804.00,175.34)(8.264,6.000){2}{\rule{0.900pt}{0.800pt}}
\multiput(816.00,184.38)(1.600,0.560){3}{\rule{2.120pt}{0.135pt}}
\multiput(816.00,181.34)(7.600,5.000){2}{\rule{1.060pt}{0.800pt}}
\multiput(828.00,189.39)(1.132,0.536){5}{\rule{1.800pt}{0.129pt}}
\multiput(828.00,186.34)(8.264,6.000){2}{\rule{0.900pt}{0.800pt}}
\multiput(840.00,195.39)(1.132,0.536){5}{\rule{1.800pt}{0.129pt}}
\multiput(840.00,192.34)(8.264,6.000){2}{\rule{0.900pt}{0.800pt}}
\multiput(852.00,201.39)(1.132,0.536){5}{\rule{1.800pt}{0.129pt}}
\multiput(852.00,198.34)(8.264,6.000){2}{\rule{0.900pt}{0.800pt}}
\multiput(864.00,207.40)(0.913,0.526){7}{\rule{1.571pt}{0.127pt}}
\multiput(864.00,204.34)(8.738,7.000){2}{\rule{0.786pt}{0.800pt}}
\multiput(876.00,214.39)(1.132,0.536){5}{\rule{1.800pt}{0.129pt}}
\multiput(876.00,211.34)(8.264,6.000){2}{\rule{0.900pt}{0.800pt}}
\multiput(888.00,220.40)(0.913,0.526){7}{\rule{1.571pt}{0.127pt}}
\multiput(888.00,217.34)(8.738,7.000){2}{\rule{0.786pt}{0.800pt}}
\multiput(900.00,227.40)(0.913,0.526){7}{\rule{1.571pt}{0.127pt}}
\multiput(900.00,224.34)(8.738,7.000){2}{\rule{0.786pt}{0.800pt}}
\multiput(912.00,234.40)(0.913,0.526){7}{\rule{1.571pt}{0.127pt}}
\multiput(912.00,231.34)(8.738,7.000){2}{\rule{0.786pt}{0.800pt}}
\multiput(924.00,241.40)(0.774,0.520){9}{\rule{1.400pt}{0.125pt}}
\multiput(924.00,238.34)(9.094,8.000){2}{\rule{0.700pt}{0.800pt}}
\multiput(936.00,249.40)(0.913,0.526){7}{\rule{1.571pt}{0.127pt}}
\multiput(936.00,246.34)(8.738,7.000){2}{\rule{0.786pt}{0.800pt}}
\multiput(948.00,256.40)(0.774,0.520){9}{\rule{1.400pt}{0.125pt}}
\multiput(948.00,253.34)(9.094,8.000){2}{\rule{0.700pt}{0.800pt}}
\multiput(960.00,264.40)(0.774,0.520){9}{\rule{1.400pt}{0.125pt}}
\multiput(960.00,261.34)(9.094,8.000){2}{\rule{0.700pt}{0.800pt}}
\multiput(972.00,272.40)(0.674,0.516){11}{\rule{1.267pt}{0.124pt}}
\multiput(972.00,269.34)(9.371,9.000){2}{\rule{0.633pt}{0.800pt}}
\multiput(984.00,281.40)(0.674,0.516){11}{\rule{1.267pt}{0.124pt}}
\multiput(984.00,278.34)(9.371,9.000){2}{\rule{0.633pt}{0.800pt}}
\multiput(996.00,290.40)(0.674,0.516){11}{\rule{1.267pt}{0.124pt}}
\multiput(996.00,287.34)(9.371,9.000){2}{\rule{0.633pt}{0.800pt}}
\multiput(1008.00,299.40)(0.674,0.516){11}{\rule{1.267pt}{0.124pt}}
\multiput(1008.00,296.34)(9.371,9.000){2}{\rule{0.633pt}{0.800pt}}
\multiput(1020.00,308.40)(0.611,0.516){11}{\rule{1.178pt}{0.124pt}}
\multiput(1020.00,305.34)(8.555,9.000){2}{\rule{0.589pt}{0.800pt}}
\multiput(1031.00,317.40)(0.599,0.514){13}{\rule{1.160pt}{0.124pt}}
\multiput(1031.00,314.34)(9.592,10.000){2}{\rule{0.580pt}{0.800pt}}
\multiput(1043.00,327.40)(0.599,0.514){13}{\rule{1.160pt}{0.124pt}}
\multiput(1043.00,324.34)(9.592,10.000){2}{\rule{0.580pt}{0.800pt}}
\multiput(1055.00,337.40)(0.599,0.514){13}{\rule{1.160pt}{0.124pt}}
\multiput(1055.00,334.34)(9.592,10.000){2}{\rule{0.580pt}{0.800pt}}
\multiput(1067.00,347.40)(0.539,0.512){15}{\rule{1.073pt}{0.123pt}}
\multiput(1067.00,344.34)(9.774,11.000){2}{\rule{0.536pt}{0.800pt}}
\multiput(1079.00,358.40)(0.539,0.512){15}{\rule{1.073pt}{0.123pt}}
\multiput(1079.00,355.34)(9.774,11.000){2}{\rule{0.536pt}{0.800pt}}
\multiput(1091.00,369.40)(0.539,0.512){15}{\rule{1.073pt}{0.123pt}}
\multiput(1091.00,366.34)(9.774,11.000){2}{\rule{0.536pt}{0.800pt}}
\multiput(1103.00,380.40)(0.539,0.512){15}{\rule{1.073pt}{0.123pt}}
\multiput(1103.00,377.34)(9.774,11.000){2}{\rule{0.536pt}{0.800pt}}
\multiput(1115.00,391.41)(0.491,0.511){17}{\rule{1.000pt}{0.123pt}}
\multiput(1115.00,388.34)(9.924,12.000){2}{\rule{0.500pt}{0.800pt}}
\multiput(1127.00,403.41)(0.491,0.511){17}{\rule{1.000pt}{0.123pt}}
\multiput(1127.00,400.34)(9.924,12.000){2}{\rule{0.500pt}{0.800pt}}
\multiput(1139.00,415.41)(0.491,0.511){17}{\rule{1.000pt}{0.123pt}}
\multiput(1139.00,412.34)(9.924,12.000){2}{\rule{0.500pt}{0.800pt}}
\multiput(1152.41,426.00)(0.511,0.536){17}{\rule{0.123pt}{1.067pt}}
\multiput(1149.34,426.00)(12.000,10.786){2}{\rule{0.800pt}{0.533pt}}
\multiput(1164.41,439.00)(0.511,0.536){17}{\rule{0.123pt}{1.067pt}}
\multiput(1161.34,439.00)(12.000,10.786){2}{\rule{0.800pt}{0.533pt}}
\multiput(1176.41,452.00)(0.511,0.536){17}{\rule{0.123pt}{1.067pt}}
\multiput(1173.34,452.00)(12.000,10.786){2}{\rule{0.800pt}{0.533pt}}
\multiput(1188.41,465.00)(0.511,0.536){17}{\rule{0.123pt}{1.067pt}}
\multiput(1185.34,465.00)(12.000,10.786){2}{\rule{0.800pt}{0.533pt}}
\multiput(1200.41,478.00)(0.511,0.581){17}{\rule{0.123pt}{1.133pt}}
\multiput(1197.34,478.00)(12.000,11.648){2}{\rule{0.800pt}{0.567pt}}
\multiput(1212.41,492.00)(0.511,0.581){17}{\rule{0.123pt}{1.133pt}}
\multiput(1209.34,492.00)(12.000,11.648){2}{\rule{0.800pt}{0.567pt}}
\multiput(1224.41,506.00)(0.511,0.581){17}{\rule{0.123pt}{1.133pt}}
\multiput(1221.34,506.00)(12.000,11.648){2}{\rule{0.800pt}{0.567pt}}
\multiput(1236.41,520.00)(0.511,0.626){17}{\rule{0.123pt}{1.200pt}}
\multiput(1233.34,520.00)(12.000,12.509){2}{\rule{0.800pt}{0.600pt}}
\multiput(1248.41,535.00)(0.511,0.626){17}{\rule{0.123pt}{1.200pt}}
\multiput(1245.34,535.00)(12.000,12.509){2}{\rule{0.800pt}{0.600pt}}
\multiput(1260.41,550.00)(0.511,0.626){17}{\rule{0.123pt}{1.200pt}}
\multiput(1257.34,550.00)(12.000,12.509){2}{\rule{0.800pt}{0.600pt}}
\multiput(1272.41,565.00)(0.511,0.671){17}{\rule{0.123pt}{1.267pt}}
\multiput(1269.34,565.00)(12.000,13.371){2}{\rule{0.800pt}{0.633pt}}
\multiput(1284.41,581.00)(0.511,0.671){17}{\rule{0.123pt}{1.267pt}}
\multiput(1281.34,581.00)(12.000,13.371){2}{\rule{0.800pt}{0.633pt}}
\multiput(1296.41,597.00)(0.511,0.671){17}{\rule{0.123pt}{1.267pt}}
\multiput(1293.34,597.00)(12.000,13.371){2}{\rule{0.800pt}{0.633pt}}
\multiput(1308.41,613.00)(0.511,0.717){17}{\rule{0.123pt}{1.333pt}}
\multiput(1305.34,613.00)(12.000,14.233){2}{\rule{0.800pt}{0.667pt}}
\multiput(1320.41,630.00)(0.511,0.717){17}{\rule{0.123pt}{1.333pt}}
\multiput(1317.34,630.00)(12.000,14.233){2}{\rule{0.800pt}{0.667pt}}
\multiput(1332.41,647.00)(0.511,0.717){17}{\rule{0.123pt}{1.333pt}}
\multiput(1329.34,647.00)(12.000,14.233){2}{\rule{0.800pt}{0.667pt}}
\multiput(1344.41,664.00)(0.511,0.762){17}{\rule{0.123pt}{1.400pt}}
\multiput(1341.34,664.00)(12.000,15.094){2}{\rule{0.800pt}{0.700pt}}
\multiput(1356.41,682.00)(0.511,0.762){17}{\rule{0.123pt}{1.400pt}}
\multiput(1353.34,682.00)(12.000,15.094){2}{\rule{0.800pt}{0.700pt}}
\multiput(1368.41,700.00)(0.511,0.762){17}{\rule{0.123pt}{1.400pt}}
\multiput(1365.34,700.00)(12.000,15.094){2}{\rule{0.800pt}{0.700pt}}
\multiput(1380.41,718.00)(0.511,0.852){17}{\rule{0.123pt}{1.533pt}}
\multiput(1377.34,718.00)(12.000,16.817){2}{\rule{0.800pt}{0.767pt}}
\multiput(1392.41,738.00)(0.511,0.807){17}{\rule{0.123pt}{1.467pt}}
\multiput(1389.34,738.00)(12.000,15.956){2}{\rule{0.800pt}{0.733pt}}
\multiput(1404.41,757.00)(0.511,0.852){17}{\rule{0.123pt}{1.533pt}}
\multiput(1401.34,757.00)(12.000,16.817){2}{\rule{0.800pt}{0.767pt}}
\multiput(1416.41,777.00)(0.511,0.852){17}{\rule{0.123pt}{1.533pt}}
\multiput(1413.34,777.00)(12.000,16.817){2}{\rule{0.800pt}{0.767pt}}
\multiput(1428.41,797.00)(0.511,0.897){17}{\rule{0.123pt}{1.600pt}}
\multiput(1425.34,797.00)(12.000,17.679){2}{\rule{0.800pt}{0.800pt}}
\put(396.0,87.0){\rule[-0.400pt]{2.891pt}{0.800pt}}
\sbox{\plotpoint}{\rule[-0.200pt]{0.400pt}{0.400pt}}%
\put(190.0,82.0){\rule[-0.200pt]{0.400pt}{187.179pt}}
\put(190.0,82.0){\rule[-0.200pt]{300.884pt}{0.400pt}}
\put(1439.0,82.0){\rule[-0.200pt]{0.400pt}{187.179pt}}
\put(190.0,859.0){\rule[-0.200pt]{300.884pt}{0.400pt}}
\end{picture}

\caption{Wkres zależności czasu odwracania macierzy od wielkości macierzy dla trzech metod.}
\end{center}
\end{figure}

Patrząc na ten wykres można zaobserwować, że wszystkie metody działają w bardzo
zbiżonym czasie, wykresy dla $GS$ i $HH$ wręcz nachodzą na siebie. Algorytm
wykorzystujący rozkład $LU$ działa nieznacznie szybciej. Nie powinien nas dziwić
taki wynik tego eksperymentu. Wszstkie te trzy metody bowiem mają złożoność
$O(n^3)$.

\section{Porównanie dokładności metod}
Jeśli chodzi o badanie dokładności uzyskanych wyników sprawa nie jest już tak
prosta. W zależności od tego, jaką struktrę ma macierz, analizowane przez nas
algorytmy mogą zachowawać się różnie. Postanowiliśmy wiec przeprowadzić
eksperymenty dla różnych rodzajów macierzy z zakresu wielkośći $[10, 200]$. Przypożądkujemy numer indentyfikujący
dany typ macierzy, aby uprościć dalsze rozważania. Mianowicie przez kolejne cyfry
oznaczamy:
\begin{myenumerate}
\item macierz wygenerowaną losowo,
\item macierz trójkątną,
\item macierz Hilberta (jak wiemy jest to macierz źle uwarunkowana),
\item macierz z dominującą przekątną,
\item macierz, której wyznaczik jest bliski zeru,
\item macierz, dla której jeden z minorów jest zerowy,
\item macierz trójprzekątniową.
\end{myenumerate}

Pierwsze przeprowadzone przez nas badanie polegało na wykonaniu rozkładu $QR$
obydwiema metodami i obliczeniu dla uzyskanych wyników wartośći wyrażenia:
$$\mathrm{F}(QQ^T - I),$$
gdzie $\mathrm{F}$ oznacza \textit{normę Frobeniusa} macierzy, którą definujemy
dla danej macierzy $A=(a_{ij})$ jako:
$$F(A)=(\sum_{i=1}^{n} \sum_{j=1}^{n} a_{ij}^2)^\frac{1}{2}.$$
Innymi słowy sprawdzamy, w jakim stopniu wyznaczona macierz $Q$ jest ortogonalna. Im większy
współczynnik, tym obliczona macierz jest bardziej oddalona od porządanej. Częściowe wyniki zostały przedstwione
w tabeli na rysunku \textbf{5.1}. Pełne dane można znaleźć w katalogu \textit{prog/testy}. Pierwszy wiersz tabeli 
określa użytą metodę, drugi zaś wielkość macierzy. W pierwszej kolumnie znajdują się identyfikatory poszczególnych typów
macierzy.
\begin{figure}[h!tb]
\begin{center}
\begin{tabular}{|l||c|c|c|c||c|c|c|c|}
\hline
\multirow{2}{*}{} & \multicolumn{4}{|c||}{\textbf{\textit{GS}}} & \multicolumn{4}{|c|}{\textbf{\textit{HH}}}\\
\cline{2-9}
&\textbf{10} & \textbf{40} & \textbf{100} & \textbf{190} & \textbf{10} & \textbf{40} & \textbf{100} & \textbf{190} \\
\hline
\hline
\textbf{1} & $0.5824$ & $80.9396$ & $87.5263$ & $97.9383$ & $0.2926$ & $122.4385$ & $1.1238$ & $2.0456$ \\
\hline
\textbf{2} & $1.0041$ & $5.9157$ & $16.0520$ & $157.3206$ & $0.1491$ & $0.4351$ & $1.1557$ & $2.1509$ \\
\hline
\textbf{3} & $ 0.3179$ & $4.8791$ & $20.1218$ & $106.4283$ & $0.1105$ & $0.4917$ & $1.1883$ & $2.1245$ \\
\hline
\textbf{4} & $ 0.2962$ & $5.5866$ & $38.1047$ & $408.4966$ & $0.2899$ & $0.5077$ & $1.1076$ & $2.3334$ \\
\hline
\textbf{5} & $ 0.3169$ & $17.6216$ & $72.5043$ & $58.2336$ & $0.1160$ & $0.5325$ & $1.2233$ & $2.1119$ \\
\hline
\textbf{6} & $ 39.8250$ & $35.6775$ & $49.8930$ & $836.6792$ & $0.1480$ & $0.4992$ & $1.1016$ & $2.1021$ \\
\hline
\textbf{7} & $ 0.3563$ & $660.1143$ & $621.1397$ & $46.6918$ & $0.1622$ & $0.4681$ & $1.1303$ & $2.1675$ \\
\hline
\end{tabular}
\caption{Tabela dla wyrażenia $\mathrm{F}(QQ^T - I)$, jednostka: $10^{14}$.} 
\end{center}
\end{figure}

Analizując tę tabelę, można zaobserwować, że wartość badanego wspołczynnika nie zależy od typu macierzy, zależy
zaś od jej wielkości. Dotyczy to zarówno algorytmu $HH$ jak i $GS$. Jednocześnie widzimy, że algorytm Householdera
w tym przypadku sprawuje się lepiej, niekiedy nawet wartość współczynnika jest dla niego dwa rzędy niższa. 

Kolejny eksperymet również dotyczy wyznaczania rozkładu $QR$. Tym razem przyjrzymy się dla obydwu metod
i danej macierzy $A$ wartości wyrażenia:
$$\mathrm{F}(A-QT).$$
Tym samym chcemy się przekonać, jak bardzo rozkład $QR$ deformuje macierz wejściową $A$ w przypadku obu alorytmów
i która z metod jest mniej dokładna. Częściowe wyniki uzyskane przy wykonywaniu tego zadania znajdują się w tabeli
\textbf{5.2}.
\begin{figure}[h!tb]
\begin{center}
\begin{tabular}{|l||c|c|c|c||c|c|c|c|}
\hline
\multirow{2}{*}{} & \multicolumn{4}{|c||}{\textbf{\textit{GS}}} & \multicolumn{4}{|c|}{\textbf{\textit{HH}}}\\
\cline{2-9}
&\textbf{10} & \textbf{40} & \textbf{100} & \textbf{190} & \textbf{10} & \textbf{40} & \textbf{100} & \textbf{190} \\
\hline
\hline
\textbf{1} & $0.1013$ & $17.1612$ & $4.6160$ & $16.2619$ & $0.1616$ & $5.6800$ & $2.7392$ & $6.7934$ \\
\hline
\textbf{2} & $0.0689$ & $0.6461$ & $3.6861$ & $45.8629$ & $0.1253$ & $0.7209$ & $ 2.8090$ & $7.0702$ \\
\hline
\textbf{3} & $0.0739$ & $0.5330$ & $3.1111$ & $23.2040$ & $0.1268$ & $0.7758$ & $2.7903$ & $7.1237$ \\
\hline
\textbf{4} & $0.1038$ & $0.8578$ & $6.5664$ & $49.8840$ & $0.1986$ & $0.7444$ & $2.7274$ & $7.3635$ \\
\hline
\textbf{5} & $0.0764$ & $6.8850$ & $23.1308$ & $13.7058$ & $0.0979$ & $0.7653$ & $2.7367$ & $6.9651$ \\
\hline
\textbf{6} & $9.9341$ & $5.9521$ & $9.1611$ & $262.4094$ & $0.0850$ & $0.7019$ & $2.7706$ & $6.9461$ \\
\hline
\textbf{7} & $0.0492$ & $260.6983$ & $31.3226$ & $21.3792$ & $0.1278$ & $0.7038$ & $2.7631$ & $7.1066$ \\
\hline
\end{tabular}
\caption{Tabela dla wyrażenia $\mathrm{F}(A-QT)$, jednostka: $10^{-12}$.} 
\end{center}
\end{figure}
Tak jak w poprzednim przypadku, tak i tutaj wartość wyznacznika dla obydwu metod w niewielkim stopniu
zależy od typu badanej macierzy, zależy zaś w podobny sposób od wielkośći macierzy. Na tym nie kończą
się zaobserwowane podobieństwa. Ponownie bardziej efektowny okazuje się być algorytm Householdera, przy
czym odpowiednie wielkości normy różnią się o jeden bądź dwa rzęd wielkości dla większych macierzy. Skupimy
się więc w dalszych rozważaniach na metodzie \textit{HH}.
 



\section{Podsumowanie}
Wkonane przez nas zadanie pozwoliło nam zapoznać się z metodami wykonywania rozkładu
$QR$, który wykorzystaliśmy do odwracania macierzy kwadratowych. Ponadto zaobserwowaliśmy,
że nie nie zawsze otrzymany wynik przy pomocy badanych metod jest bliski wynikowi oczekiwanemu.

Zbierzmy teraz zebrane przez nas obserwacje i wnioski. Przede wszystkim przypomnijmy, że wszystkie
badane perzez nas metody miały podobny czas działania, który nie zależy od struktury macierzy.
Wynika to z prostego faktu, zarówno metody \textit{HH}, \textit{GS}, jak i \textit{LU} mają taką
samą złożoność obliczeniową równą $O(n^3)$. W rozdziale \S5 opisaliśmy przeprowadzone przez nas
badania nad dokładnością metod, która okazuje się zależeć nie tylko od wielkości macierze, ale
także od jej struktury. Jeśli chodzi o rozkład $QR$, metoda \textit{HH} okazała się lepsza od
metody \textit{GS}. Wobec tego to właśnie tę metodę zestawiliśmy z algorytmam wykorzystującym
rozkłas $LU$ i z programem \textit{octave}. Dokładności uzyskane za pomocą metod \textit{HH} i
\textit{LU} okazały się zbliożone dla wszystkich badanych rodzajów macierzy, oprócz tych z zerującym
się monorem głównym. W tym przypadku ewidentnie dokładniejsza okazała się być metoda Householdera.
Zauważyliśmy ponadto, że obydwa algorytmy nie radzą sobie z odwracaniem macierzy Hilberta, błędy
uzyskane były katastrofalnie duże. Podobne obserwacje uzyskaliśmy prz porównywaniu naszych metod z
programem \textit{octave}. Satysfakcjonujące wyniki otrzymaliśmy dla macierzy losowych, z dominującą
przekątną, o wyznaczniku bliskim zeru i trójdiagonalnych. Ponownie metoda \textit{HH} dała lepsze wyniki dla
macierzy typu \textbf{6}. Liczony wskażnik był dla nich około $10^{12}$ razy mniejszy.

\end{document}

