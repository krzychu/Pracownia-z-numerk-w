\section{Porównanie dokładności metod}
Jeśli chodzi o badanie dokładności uzyskanych wyników sprawa nie jest już tak
prosta. W zależności od tego, jaką struktrę ma macierz, analizowane przez nas
algorytmy mogą zachowawać się różnie. Postanowiliśmy wiec przeprowadzić
eksperymenty dla różnych rodzajów macierzy z zakresu wielkośći $[10, 200]$. Przypożądkujemy numer indentyfikujący
dany typ macierzy, aby uprościć dalsze rozważania. Mianowicie przez kolejne cyfry
oznaczamy:
\begin{myenumerate}
\item macierz wygenerowaną losowo,
\item macierz trójkątną,
\item macierz Hilberta (jak wiemy jest to macierz źle uwarunkowana),
\item macierz z dominującą przekątną,
\item macierz, której wyznaczik jest bliski zeru,
\item macierz, dla której jeden z minorów jest zerowy,
\item macierz trójprzekątniową.
\end{myenumerate}

Pierwsze przeprowadzone przez nas badanie polegało na wykonaniu rozkładu $QR$
obydwiema metodami i obliczeniu dla uzyskanych wyników wartośći wyrażenia:
$$\mathrm{F}(QQ^T - I),$$
gdzie $\mathrm{F}$ oznacza \textit{normę Frobeniusa} macierzy, którą definujemy
dla danej macierzy $A=(a_{ij})$ jako:
$$F(A)=(\sum_{i=1}^{n} \sum_{j=1}^{n} a_{ij}^2)^\frac{1}{2}.$$
Innymi słowy sprawdzamy, w jakim stopniu wyznaczona macierz $Q$ jest ortogonalna. Im większy
współczynnik, tym obliczona macierz jest bardziej oddalona od porządanej. Częściowe wyniki zostały przedstwione
w tabeli na rysunku \textbf{5.1}. Pełne dane można znaleźć w katalogu \textit{prog/testy}. Pierwszy wiersz tabeli 
określa użytą metodę, drugi zaś wielkość macierzy. W pierwszej kolumnie znajdują się identyfikatory poszczególnych typów
macierzy.
\begin{figure}[h!tb]
\begin{center}
\begin{tabular}{|l||c|c|c|c||c|c|c|c|}
\hline
\multirow{2}{*}{} & \multicolumn{4}{|c||}{\textbf{\textit{GS}}} & \multicolumn{4}{|c|}{\textbf{\textit{HH}}}\\
\cline{2-9}
&\textbf{10} & \textbf{40} & \textbf{100} & \textbf{190} & \textbf{10} & \textbf{40} & \textbf{100} & \textbf{190} \\
\hline
\hline
\textbf{1} & $0.5824$ & $80.9396$ & $87.5263$ & $97.9383$ & $0.2926$ & $122.4385$ & $1.1238$ & $2.0456$ \\
\hline
\textbf{2} & $1.0041$ & $5.9157$ & $16.0520$ & $157.3206$ & $0.1491$ & $0.4351$ & $1.1557$ & $2.1509$ \\
\hline
\textbf{3} & $ 0.3179$ & $4.8791$ & $20.1218$ & $106.4283$ & $0.1105$ & $0.4917$ & $1.1883$ & $2.1245$ \\
\hline
\textbf{4} & $ 0.2962$ & $5.5866$ & $38.1047$ & $408.4966$ & $0.2899$ & $0.5077$ & $1.1076$ & $2.3334$ \\
\hline
\textbf{5} & $ 0.3169$ & $17.6216$ & $72.5043$ & $58.2336$ & $0.1160$ & $0.5325$ & $1.2233$ & $2.1119$ \\
\hline
\textbf{6} & $ 39.8250$ & $35.6775$ & $49.8930$ & $836.6792$ & $0.1480$ & $0.4992$ & $1.1016$ & $2.1021$ \\
\hline
\textbf{7} & $ 0.3563$ & $660.1143$ & $621.1397$ & $46.6918$ & $0.1622$ & $0.4681$ & $1.1303$ & $2.1675$ \\
\hline
\end{tabular}
\caption{Tabela dla wyrażenia $\mathrm{F}(QQ^T - I)$, jednostka: $10^{14}$.} 
\end{center}
\end{figure}

Analizując tę tabelę, można zaobserwować, że wartość badanego wspołczynnika nie zależy od typu macierzy, zależy
zaś od jej wielkości. Dotyczy to zarówno algorytmu $HH$ jak i $GS$. Jednocześnie widzimy, że algorytm Householdera
w tym przypadku sprawuje się lepiej, niekiedy nawet wartość współczynnika jest dla niego dwa rzędy niższa. 

Kolejny eksperymet również dotyczy wyznaczania rozkładu $QR$. Tym razem przyjrzymy się dla obydwu metod
i danej macierzy $A$ wartości wyrażenia:
$$\mathrm{F}(A-QT).$$
Tym samym chcemy się przekonać, jak bardzo rozkład $QR$ deformuje macierz wejściową $A$ w przypadku obu alorytmów
i która z metod jest mniej dokładna. Częściowe wyniki uzyskane przy wykonywaniu tego zadania znajdują się w tabeli
\textbf{5.2}.
\begin{figure}[h!tb]
\begin{center}
\begin{tabular}{|l||c|c|c|c||c|c|c|c|}
\hline
\multirow{2}{*}{} & \multicolumn{4}{|c||}{\textbf{\textit{GS}}} & \multicolumn{4}{|c|}{\textbf{\textit{HH}}}\\
\cline{2-9}
&\textbf{10} & \textbf{40} & \textbf{100} & \textbf{190} & \textbf{10} & \textbf{40} & \textbf{100} & \textbf{190} \\
\hline
\hline
\textbf{1} & $0.1013$ & $17.1612$ & $4.6160$ & $16.2619$ & $0.1616$ & $5.6800$ & $2.7392$ & $6.7934$ \\
\hline
\textbf{2} & $0.0689$ & $0.6461$ & $3.6861$ & $45.8629$ & $0.1253$ & $0.7209$ & $ 2.8090$ & $7.0702$ \\
\hline
\textbf{3} & $0.0739$ & $0.5330$ & $3.1111$ & $23.2040$ & $0.1268$ & $0.7758$ & $2.7903$ & $7.1237$ \\
\hline
\textbf{4} & $0.1038$ & $0.8578$ & $6.5664$ & $49.8840$ & $0.1986$ & $0.7444$ & $2.7274$ & $7.3635$ \\
\hline
\textbf{5} & $0.0764$ & $6.8850$ & $23.1308$ & $13.7058$ & $0.0979$ & $0.7653$ & $2.7367$ & $6.9651$ \\
\hline
\textbf{6} & $9.9341$ & $5.9521$ & $9.1611$ & $262.4094$ & $0.0850$ & $0.7019$ & $2.7706$ & $6.9461$ \\
\hline
\textbf{7} & $0.0492$ & $260.6983$ & $31.3226$ & $21.3792$ & $0.1278$ & $0.7038$ & $2.7631$ & $7.1066$ \\
\hline
\end{tabular}
\caption{Tabela dla wyrażenia $\mathrm{F}(A-QT)$, jednostka: $10^{-12}$.} 
\end{center}
\end{figure}
Tak jak w poprzednim przypadku, tak i tutaj wartość wyznacznika dla obydwu metod w niewielkim stopniu
zależy od typu badanej macierzy, zależy zaś w podobny sposób od wielkośći macierzy. Na tym nie kończą
się zaobserwowane podobieństwa. Ponownie bardziej efektowny okazuje się być algorytm Householdera, przy
czym odpowiednie wielkości normy różnią się o jeden bądź dwa rzęd wielkości dla większych macierzy. Skupimy
się więc w dalszych rozważaniach na metodzie \textit{HH}.

Zajmijmy się teraz porównaniem algorytmu \textit{HH} z \textit{LU} pod względem dokładności wyznaczanie dla
danej macierzy $A$ jej odwrotności. W tym celu będziemy obserwować, jak dla każdej z metod przy przyjęciu kolejnych
podanych wcześniej typów macierzy zachowuje się wartość wyrażenia:
$$\mathrm{F}(AA^{-1} - I).$$
Rysunek \textbf{5.3} przedstawia tabele z częścią wyników przeprowadzonego eksperymentu.
\begin{figure}[h!tb]
\begin{center}
\begin{tabular}{|l||c|c|c||c|c|c|}
\hline
\multirow{2}{*}{} & \multicolumn{3}{|c||}{\textbf{\textit{LU}}} & \multicolumn{3}{|c|}{\textbf{\textit{HH}}}\\
\cline{2-7}
& \textbf{40} & \textbf{100} & \textbf{190} & \textbf{40} & \textbf{100} & \textbf{190} \\
\hline
\hline
\textbf{1} & $0.12 \cdot 10^{-12}$ & $19.79 \cdot 10^{-12}$ & $61.11 \cdot 10^{-12}$ & $0.02 \cdot 10^{-12}$ & $0.58 \cdot 10^{-12}$ & $0.26 \cdot 10^{-12}$ \\
\hline
\textbf{2} & $2.90 \cdot 10^{-6}$ & $0.84 \cdot 10^{12}$ & $2.60 \cdot 10^{30}$ & $1.32 \cdot 10^{-6}$ & $0.53 \cdot 10^{12} $ & $11.55 \cdot 10^{30}$ \\
\hline
\textbf{3} & $0.11 \cdot 10^{9}$ & $8.90\cdot 10^{9}$ & $12.94\cdot 10^{9}$ & $0.19\cdot 10^{9}$ & $2.75\cdot 10^{12}$ & $86.71\cdot 10^{12}$ \\
\hline
\textbf{4} & $2.12 \cdot 10^{-15}$ & $5.96 \cdot 10^{-15}$ & $11.04 \cdot 10^{-15}$ & $9.12 \cdot 10^{-15}$ & $22.94 \cdot 10^{-15}$ & $51.09 \cdot 10^{-15}$ \\
\hline
\textbf{5} & $0.0764$ & $6.8850$ & $23.1308$ & $13.7058$ & $0.0979$ & $0.7653$ \\
\hline
\textbf{6} & $9.9341$ & $5.9521$ & $9.1611$ & $262.4094$ & $0.0850$ & $0.7019$ \\
\hline
\textbf{7} & $0.0492$ & $260.6983$ & $31.3226$ & $21.3792$ & $0.1278$ & $0.7038$ \\
\hline
\end{tabular}
\caption{Tabela dla wyrażenia $\mathrm{F}(AA^{-1}-I)$.} 
\end{center}
\end{figure}

 



