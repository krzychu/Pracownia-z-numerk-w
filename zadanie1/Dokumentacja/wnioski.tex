\section{Wnioski}
Przeprowadzenie doświadczeń nad algorytmami naturalnym, Strassena i stworzonym
przez nas algorytmem łąćzącym obydwie metody doprowadziło nas do kilku ciekawych obserwacji
i wniosków z nich płynących. Na pierwszą z nich natknęliśmy się przy badaniu czsów działania
programów będących realizacją wszystkich metod. Zaskoczyło nas bowiem, że mimo, iż
algorytm Strassena ma mniejszą złożoność, a mimo to działa wielokrotnie wolniej od algorytmu
naturalnego dla badanego zakresu danych. Macierze o większych rozmiarach są mniej przydatne,
prowadzi nas to więc do wniosku, że algorytm Strassena nie jest przydatny, gdy zależ nam
na szybkości obliczeń, co daje nam swoisty dowód na to, jak ważne są doświadczenia numeryczne.
Również wyniki badań dokładności algorytmów opowiadają się na niekorzyść algorytmu Strassena.
Można także wspomnieć, że poprzez wielokrotne odwołania rekurencyjne algorytm Strassena
wymaga wielokrotnie więcej pamięci. Właśnie z tych powodów przeprowadziliśmy eksperymenty nad połączeniem
obydwu metod. Wyniki okazały się bardzo zadowalające. Dla badanego przez nas zakresu algorytm połączony dał przybliżone
a nawet dla większych macierzy lepsze wyniki jeśli chodzi o czas działania, a także bliskie algorytmowi
naturalnemu współczynniki dokładności.

Podsumowując, doświadczenie to pozwoliło nam zapoznać się ze znanymi powszechnie algorytmami
mnożenia macierzy oraz nauczyło nas, że nikiedy teoretyczne rozważania warto jest przebadać
eksperymentalnie i zobaczyć, czy mają one znaczenie dla praktycznych zastosowań. Algorytm Strassena,
choć daje tak duże nadzieje patrząc na jego złożoność, okazuje się być niekoniecznym w obliczeniach dla
macierzy z badanego przez nas zakresu.
