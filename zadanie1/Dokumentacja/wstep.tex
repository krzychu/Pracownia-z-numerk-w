\section{Wstęp}
Macierze stosuje się do reprezentacji danych i obliczeń w wielu dziedzinach naukowych i technicznych m. in. w matematyce, fizyce czy grafice komputerowej. Często istotne jest również przprowadzanie działań na macierzach, takich jak mnożenie, dodatkowo w jak najkrótszym czasie. Nie dziwi nas więc fakt, że powstało wiele algorytmów obliczjących iloczyn macierzy. W naszej pracy przedstawimy i zbadamy doświadczalnie dwa spośród nich.

Głównym celem naszego zadania jest porównanie pod względem dokładności i szybkości dwuch algorytmów mnożenia macierzy rozmiaru $n \in [4, 500]$: algorytmu naturalnego i algorytmu Strassena. Stawiamy przed sobą dodatkowe zadanie jakim jest rozszerzenie zakresu wielkości danych oraz znalezienie przedziałów wielkości macierzy dla których określony algorytm daje dokładniejsze wyniki czy też jest wydajniejszy czasowo.

W rozdziale \S2 znajduje się krótki opis algorytmów, następny rozdział przedstawia realizację zadania. Tam znajduje się raport z przeprowadzonych doświadczeń, jak również przegląd otrzymanych wyników w postaci tabel i wykresów. Rozdział \S 5 zawiera nie tylko wnioski oraz obserwacje zebrane przez nas podczas rozwiązywania problemu, ale także został wzbogacony o rozważania teoretyczne, które pozwoliły nam uzyskać szersze spojrzenie na obydwa algorytmy.

Do obliczeń wykorzystaliśmy napisany przez nas program w języku C++, implementujący dwie wspomniane powyżej metody liczenia iloczynu macierzy. Pliki z wszstkimi testami oraz wynikami zostały umieszczone w katalogu \textit{testy}.   
