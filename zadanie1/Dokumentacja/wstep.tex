\section{Wstęp}
Macierze stosuje się do reprezentacji danych i obliczeń w wielu dziedzinach
naukowych i technicznych m. in. w matematyce, fizyce czy grafice komputerowej.
Często istotne jest również przprowadzanie działań na macierzach, takich jak
mnożenie, dodatkowo w jak najkrótszym czasie i jak największej dokładności. Nie dziwi nas więc fakt, że
powstało wiele algorytmów obliczjących iloczyny macierzy. W naszej pracy
przedstawimy i zbadamy doświadczalnie dwa spośród nich.

Głównym celem naszego zadania jest porównanie pod względem dokładności i
szybkości dwuch algorytmów mnożenia macierzy rozmiaru $n \in [4, 500]$:
algorytmu naturalnego i algorytmu Strassena. Stawiamy przed sobą dodatkowe
zadanie jakim jest rozszerzenie zakresu wielkości macierzy oraz znalezienie
przedziałów wielkości macierzy dla których określony algorytm daje
dokładniejsze wyniki czy też jest wydajniejszy czasowo.

W rozdziale \S2 znajduje się krótki opis algorytmów, następny rozdział przedstawia realizację postawionego przed nami zadania. Tam znajduje się między innymi raport
z przeprowadzonych doświadczeń, jak i również przegląd otrzymanych wyników w
postaci wykresów. Rozdział \S 4 zawiera wnioski oraz obserwacje zebrane przez nas podczas rozwiązywania problemu, postaraliśmy się w nim także przedsawić nasze hipotezy, przemyślenia odnośnie przyczyn, jakie mogły wystąpić i spowodować takie, a nie inne wyniki badań.
Do obliczeń wykorzystaliśmy napisany przez nas program w języku C++, implementujący
dwie wspomniane powyżej metody obliczania iloczynu macierzy. Pliki z wszstkimi testami
oraz wynikami, które zostały przez nas wykorzystane do wygenerowania wykresów, znajdują
się w folderze {\textit wykresy} pod nazwami \textit{assoc\_double.dat, assoc\_float.dat, time\_all.dat,
time\_best.dat, inverse\_double.dat, inverse\_float.dat}.
W plikach tych podana została również krótka informacja o danych w nich zawartych.
