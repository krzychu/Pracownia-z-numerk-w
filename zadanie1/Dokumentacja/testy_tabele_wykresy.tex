\section{Przebieg doświadczenia numerycznego}
\subsection{Porównanie szybkości algorytmów}
Dla porównania szybkości obydwu algorytmów opisanych powyżej wygenerowaliśmy
losowo macierze z zakresu wielkośći $n \in [4, 2000]$ i zmierzyliśmy czasy ich
działania. Otrzmane wyniki wykorzystaliśmy do wygenerowania wykresu zależności
czasu działania od wielkosci macierzy dla algrymu Strassena i algorytmu
naturalnego.
%tu będzie wykres
Powyższy wykres pozwolił nam na wyciągnięcie wniosku, że dla badanego przez nas
zakresu wielkości macierzy, mimo że algorytm Strassena ma mniejszą złożoność
obliczeniową, jest wolniejszy od algorytmu naturalnego. W związku z tym
postanowiliśmy zbadać, czy połączenie obydwu algorytmów nie dałoby znacznie
lepszych wyników. Na podstawie sporządzonych przez nas wykresów wyznaczyliśmy
próg na $n=128$. Oznacza to, że dla mniejszych macierzy (z zakresu
$n \in [4, 128]$)do obliczeń wykorzystujemy naturalną metodę mnożenia macierzy,
dla reszty zaś metodę Strassena.
\subsection{Porównanie dokładności algorytmów}
W celu zbadania dokładności metod, obliczyliśmy wartości odpowiednich
współczynników. Pierwszy z nich zadany jest wzorem:
$$\Delta(XX^{-1}-I),$$
gdzie $I$ jest macierzą jednostkową, natomiast $\Delta(X) = sum_{i=1}^{n}
sum_{j=1}^{n} x_{ij}^2.$
