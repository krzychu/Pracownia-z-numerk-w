\section{Przebieg doświadczenia numerycznego}
\subsection{Porównanie szybkości algorytmów}
Dla porównania szybkości obydwu algorytmów opisanych powyżej wygenerowaliśmy
losowo macierze z zakresu wielkośći $n \in [2, 500]$ i zmierzyliśmy czas ich
mnożenia przez obydwie metody. Otrzmane wyniki wykorzystaliśmy do wygenerowania wykresu zależności
czasu działania od wielkosci macierzy dla algrymu Strassena i algorytmu
naturalnego.
\begin{figure}[hb]
\begin{center}
% GNUPLOT: LaTeX picture
\setlength{\unitlength}{0.240900pt}
\ifx\plotpoint\undefined\newsavebox{\plotpoint}\fi
\sbox{\plotpoint}{\rule[-0.200pt]{0.400pt}{0.400pt}}%
\begin{picture}(1500,900)(0,0)
\sbox{\plotpoint}{\rule[-0.200pt]{0.400pt}{0.400pt}}%
\put(170.0,82.0){\rule[-0.200pt]{4.818pt}{0.400pt}}
\put(150,82){\makebox(0,0)[r]{ 1}}
\put(1419.0,82.0){\rule[-0.200pt]{4.818pt}{0.400pt}}
\put(170.0,108.0){\rule[-0.200pt]{2.409pt}{0.400pt}}
\put(1429.0,108.0){\rule[-0.200pt]{2.409pt}{0.400pt}}
\put(170.0,123.0){\rule[-0.200pt]{2.409pt}{0.400pt}}
\put(1429.0,123.0){\rule[-0.200pt]{2.409pt}{0.400pt}}
\put(170.0,134.0){\rule[-0.200pt]{2.409pt}{0.400pt}}
\put(1429.0,134.0){\rule[-0.200pt]{2.409pt}{0.400pt}}
\put(170.0,142.0){\rule[-0.200pt]{2.409pt}{0.400pt}}
\put(1429.0,142.0){\rule[-0.200pt]{2.409pt}{0.400pt}}
\put(170.0,149.0){\rule[-0.200pt]{2.409pt}{0.400pt}}
\put(1429.0,149.0){\rule[-0.200pt]{2.409pt}{0.400pt}}
\put(170.0,155.0){\rule[-0.200pt]{2.409pt}{0.400pt}}
\put(1429.0,155.0){\rule[-0.200pt]{2.409pt}{0.400pt}}
\put(170.0,160.0){\rule[-0.200pt]{2.409pt}{0.400pt}}
\put(1429.0,160.0){\rule[-0.200pt]{2.409pt}{0.400pt}}
\put(170.0,164.0){\rule[-0.200pt]{2.409pt}{0.400pt}}
\put(1429.0,164.0){\rule[-0.200pt]{2.409pt}{0.400pt}}
\put(170.0,168.0){\rule[-0.200pt]{2.409pt}{0.400pt}}
\put(1429.0,168.0){\rule[-0.200pt]{2.409pt}{0.400pt}}
\put(170.0,172.0){\rule[-0.200pt]{2.409pt}{0.400pt}}
\put(1429.0,172.0){\rule[-0.200pt]{2.409pt}{0.400pt}}
\put(170.0,175.0){\rule[-0.200pt]{2.409pt}{0.400pt}}
\put(1429.0,175.0){\rule[-0.200pt]{2.409pt}{0.400pt}}
\put(170.0,178.0){\rule[-0.200pt]{2.409pt}{0.400pt}}
\put(1429.0,178.0){\rule[-0.200pt]{2.409pt}{0.400pt}}
\put(170.0,181.0){\rule[-0.200pt]{2.409pt}{0.400pt}}
\put(1429.0,181.0){\rule[-0.200pt]{2.409pt}{0.400pt}}
\put(170.0,184.0){\rule[-0.200pt]{2.409pt}{0.400pt}}
\put(1429.0,184.0){\rule[-0.200pt]{2.409pt}{0.400pt}}
\put(170.0,186.0){\rule[-0.200pt]{2.409pt}{0.400pt}}
\put(1429.0,186.0){\rule[-0.200pt]{2.409pt}{0.400pt}}
\put(170.0,188.0){\rule[-0.200pt]{2.409pt}{0.400pt}}
\put(1429.0,188.0){\rule[-0.200pt]{2.409pt}{0.400pt}}
\put(170.0,190.0){\rule[-0.200pt]{2.409pt}{0.400pt}}
\put(1429.0,190.0){\rule[-0.200pt]{2.409pt}{0.400pt}}
\put(170.0,192.0){\rule[-0.200pt]{2.409pt}{0.400pt}}
\put(1429.0,192.0){\rule[-0.200pt]{2.409pt}{0.400pt}}
\put(170.0,194.0){\rule[-0.200pt]{2.409pt}{0.400pt}}
\put(1429.0,194.0){\rule[-0.200pt]{2.409pt}{0.400pt}}
\put(170.0,196.0){\rule[-0.200pt]{2.409pt}{0.400pt}}
\put(1429.0,196.0){\rule[-0.200pt]{2.409pt}{0.400pt}}
\put(170.0,198.0){\rule[-0.200pt]{2.409pt}{0.400pt}}
\put(1429.0,198.0){\rule[-0.200pt]{2.409pt}{0.400pt}}
\put(170.0,200.0){\rule[-0.200pt]{2.409pt}{0.400pt}}
\put(1429.0,200.0){\rule[-0.200pt]{2.409pt}{0.400pt}}
\put(170.0,201.0){\rule[-0.200pt]{2.409pt}{0.400pt}}
\put(1429.0,201.0){\rule[-0.200pt]{2.409pt}{0.400pt}}
\put(170.0,203.0){\rule[-0.200pt]{2.409pt}{0.400pt}}
\put(1429.0,203.0){\rule[-0.200pt]{2.409pt}{0.400pt}}
\put(170.0,204.0){\rule[-0.200pt]{2.409pt}{0.400pt}}
\put(1429.0,204.0){\rule[-0.200pt]{2.409pt}{0.400pt}}
\put(170.0,206.0){\rule[-0.200pt]{2.409pt}{0.400pt}}
\put(1429.0,206.0){\rule[-0.200pt]{2.409pt}{0.400pt}}
\put(170.0,207.0){\rule[-0.200pt]{2.409pt}{0.400pt}}
\put(1429.0,207.0){\rule[-0.200pt]{2.409pt}{0.400pt}}
\put(170.0,208.0){\rule[-0.200pt]{2.409pt}{0.400pt}}
\put(1429.0,208.0){\rule[-0.200pt]{2.409pt}{0.400pt}}
\put(170.0,210.0){\rule[-0.200pt]{2.409pt}{0.400pt}}
\put(1429.0,210.0){\rule[-0.200pt]{2.409pt}{0.400pt}}
\put(170.0,211.0){\rule[-0.200pt]{2.409pt}{0.400pt}}
\put(1429.0,211.0){\rule[-0.200pt]{2.409pt}{0.400pt}}
\put(170.0,212.0){\rule[-0.200pt]{2.409pt}{0.400pt}}
\put(1429.0,212.0){\rule[-0.200pt]{2.409pt}{0.400pt}}
\put(170.0,213.0){\rule[-0.200pt]{2.409pt}{0.400pt}}
\put(1429.0,213.0){\rule[-0.200pt]{2.409pt}{0.400pt}}
\put(170.0,214.0){\rule[-0.200pt]{2.409pt}{0.400pt}}
\put(1429.0,214.0){\rule[-0.200pt]{2.409pt}{0.400pt}}
\put(170.0,215.0){\rule[-0.200pt]{2.409pt}{0.400pt}}
\put(1429.0,215.0){\rule[-0.200pt]{2.409pt}{0.400pt}}
\put(170.0,216.0){\rule[-0.200pt]{2.409pt}{0.400pt}}
\put(1429.0,216.0){\rule[-0.200pt]{2.409pt}{0.400pt}}
\put(170.0,217.0){\rule[-0.200pt]{2.409pt}{0.400pt}}
\put(1429.0,217.0){\rule[-0.200pt]{2.409pt}{0.400pt}}
\put(170.0,218.0){\rule[-0.200pt]{2.409pt}{0.400pt}}
\put(1429.0,218.0){\rule[-0.200pt]{2.409pt}{0.400pt}}
\put(170.0,219.0){\rule[-0.200pt]{2.409pt}{0.400pt}}
\put(1429.0,219.0){\rule[-0.200pt]{2.409pt}{0.400pt}}
\put(170.0,220.0){\rule[-0.200pt]{2.409pt}{0.400pt}}
\put(1429.0,220.0){\rule[-0.200pt]{2.409pt}{0.400pt}}
\put(170.0,221.0){\rule[-0.200pt]{2.409pt}{0.400pt}}
\put(1429.0,221.0){\rule[-0.200pt]{2.409pt}{0.400pt}}
\put(170.0,222.0){\rule[-0.200pt]{2.409pt}{0.400pt}}
\put(1429.0,222.0){\rule[-0.200pt]{2.409pt}{0.400pt}}
\put(170.0,223.0){\rule[-0.200pt]{2.409pt}{0.400pt}}
\put(1429.0,223.0){\rule[-0.200pt]{2.409pt}{0.400pt}}
\put(170.0,224.0){\rule[-0.200pt]{2.409pt}{0.400pt}}
\put(1429.0,224.0){\rule[-0.200pt]{2.409pt}{0.400pt}}
\put(170.0,225.0){\rule[-0.200pt]{2.409pt}{0.400pt}}
\put(1429.0,225.0){\rule[-0.200pt]{2.409pt}{0.400pt}}
\put(170.0,226.0){\rule[-0.200pt]{2.409pt}{0.400pt}}
\put(1429.0,226.0){\rule[-0.200pt]{2.409pt}{0.400pt}}
\put(170.0,226.0){\rule[-0.200pt]{2.409pt}{0.400pt}}
\put(1429.0,226.0){\rule[-0.200pt]{2.409pt}{0.400pt}}
\put(170.0,227.0){\rule[-0.200pt]{2.409pt}{0.400pt}}
\put(1429.0,227.0){\rule[-0.200pt]{2.409pt}{0.400pt}}
\put(170.0,228.0){\rule[-0.200pt]{2.409pt}{0.400pt}}
\put(1429.0,228.0){\rule[-0.200pt]{2.409pt}{0.400pt}}
\put(170.0,229.0){\rule[-0.200pt]{2.409pt}{0.400pt}}
\put(1429.0,229.0){\rule[-0.200pt]{2.409pt}{0.400pt}}
\put(170.0,229.0){\rule[-0.200pt]{2.409pt}{0.400pt}}
\put(1429.0,229.0){\rule[-0.200pt]{2.409pt}{0.400pt}}
\put(170.0,230.0){\rule[-0.200pt]{2.409pt}{0.400pt}}
\put(1429.0,230.0){\rule[-0.200pt]{2.409pt}{0.400pt}}
\put(170.0,231.0){\rule[-0.200pt]{2.409pt}{0.400pt}}
\put(1429.0,231.0){\rule[-0.200pt]{2.409pt}{0.400pt}}
\put(170.0,232.0){\rule[-0.200pt]{2.409pt}{0.400pt}}
\put(1429.0,232.0){\rule[-0.200pt]{2.409pt}{0.400pt}}
\put(170.0,232.0){\rule[-0.200pt]{2.409pt}{0.400pt}}
\put(1429.0,232.0){\rule[-0.200pt]{2.409pt}{0.400pt}}
\put(170.0,233.0){\rule[-0.200pt]{2.409pt}{0.400pt}}
\put(1429.0,233.0){\rule[-0.200pt]{2.409pt}{0.400pt}}
\put(170.0,234.0){\rule[-0.200pt]{2.409pt}{0.400pt}}
\put(1429.0,234.0){\rule[-0.200pt]{2.409pt}{0.400pt}}
\put(170.0,234.0){\rule[-0.200pt]{2.409pt}{0.400pt}}
\put(1429.0,234.0){\rule[-0.200pt]{2.409pt}{0.400pt}}
\put(170.0,235.0){\rule[-0.200pt]{2.409pt}{0.400pt}}
\put(1429.0,235.0){\rule[-0.200pt]{2.409pt}{0.400pt}}
\put(170.0,236.0){\rule[-0.200pt]{2.409pt}{0.400pt}}
\put(1429.0,236.0){\rule[-0.200pt]{2.409pt}{0.400pt}}
\put(170.0,236.0){\rule[-0.200pt]{2.409pt}{0.400pt}}
\put(1429.0,236.0){\rule[-0.200pt]{2.409pt}{0.400pt}}
\put(170.0,237.0){\rule[-0.200pt]{2.409pt}{0.400pt}}
\put(1429.0,237.0){\rule[-0.200pt]{2.409pt}{0.400pt}}
\put(170.0,237.0){\rule[-0.200pt]{2.409pt}{0.400pt}}
\put(1429.0,237.0){\rule[-0.200pt]{2.409pt}{0.400pt}}
\put(170.0,238.0){\rule[-0.200pt]{2.409pt}{0.400pt}}
\put(1429.0,238.0){\rule[-0.200pt]{2.409pt}{0.400pt}}
\put(170.0,239.0){\rule[-0.200pt]{2.409pt}{0.400pt}}
\put(1429.0,239.0){\rule[-0.200pt]{2.409pt}{0.400pt}}
\put(170.0,239.0){\rule[-0.200pt]{2.409pt}{0.400pt}}
\put(1429.0,239.0){\rule[-0.200pt]{2.409pt}{0.400pt}}
\put(170.0,240.0){\rule[-0.200pt]{2.409pt}{0.400pt}}
\put(1429.0,240.0){\rule[-0.200pt]{2.409pt}{0.400pt}}
\put(170.0,240.0){\rule[-0.200pt]{2.409pt}{0.400pt}}
\put(1429.0,240.0){\rule[-0.200pt]{2.409pt}{0.400pt}}
\put(170.0,241.0){\rule[-0.200pt]{2.409pt}{0.400pt}}
\put(1429.0,241.0){\rule[-0.200pt]{2.409pt}{0.400pt}}
\put(170.0,241.0){\rule[-0.200pt]{2.409pt}{0.400pt}}
\put(1429.0,241.0){\rule[-0.200pt]{2.409pt}{0.400pt}}
\put(170.0,242.0){\rule[-0.200pt]{2.409pt}{0.400pt}}
\put(1429.0,242.0){\rule[-0.200pt]{2.409pt}{0.400pt}}
\put(170.0,242.0){\rule[-0.200pt]{2.409pt}{0.400pt}}
\put(1429.0,242.0){\rule[-0.200pt]{2.409pt}{0.400pt}}
\put(170.0,243.0){\rule[-0.200pt]{2.409pt}{0.400pt}}
\put(1429.0,243.0){\rule[-0.200pt]{2.409pt}{0.400pt}}
\put(170.0,243.0){\rule[-0.200pt]{2.409pt}{0.400pt}}
\put(1429.0,243.0){\rule[-0.200pt]{2.409pt}{0.400pt}}
\put(170.0,244.0){\rule[-0.200pt]{2.409pt}{0.400pt}}
\put(1429.0,244.0){\rule[-0.200pt]{2.409pt}{0.400pt}}
\put(170.0,244.0){\rule[-0.200pt]{2.409pt}{0.400pt}}
\put(1429.0,244.0){\rule[-0.200pt]{2.409pt}{0.400pt}}
\put(170.0,245.0){\rule[-0.200pt]{2.409pt}{0.400pt}}
\put(1429.0,245.0){\rule[-0.200pt]{2.409pt}{0.400pt}}
\put(170.0,245.0){\rule[-0.200pt]{2.409pt}{0.400pt}}
\put(1429.0,245.0){\rule[-0.200pt]{2.409pt}{0.400pt}}
\put(170.0,246.0){\rule[-0.200pt]{2.409pt}{0.400pt}}
\put(1429.0,246.0){\rule[-0.200pt]{2.409pt}{0.400pt}}
\put(170.0,246.0){\rule[-0.200pt]{2.409pt}{0.400pt}}
\put(1429.0,246.0){\rule[-0.200pt]{2.409pt}{0.400pt}}
\put(170.0,247.0){\rule[-0.200pt]{2.409pt}{0.400pt}}
\put(1429.0,247.0){\rule[-0.200pt]{2.409pt}{0.400pt}}
\put(170.0,247.0){\rule[-0.200pt]{2.409pt}{0.400pt}}
\put(1429.0,247.0){\rule[-0.200pt]{2.409pt}{0.400pt}}
\put(170.0,248.0){\rule[-0.200pt]{2.409pt}{0.400pt}}
\put(1429.0,248.0){\rule[-0.200pt]{2.409pt}{0.400pt}}
\put(170.0,248.0){\rule[-0.200pt]{2.409pt}{0.400pt}}
\put(1429.0,248.0){\rule[-0.200pt]{2.409pt}{0.400pt}}
\put(170.0,249.0){\rule[-0.200pt]{2.409pt}{0.400pt}}
\put(1429.0,249.0){\rule[-0.200pt]{2.409pt}{0.400pt}}
\put(170.0,249.0){\rule[-0.200pt]{2.409pt}{0.400pt}}
\put(1429.0,249.0){\rule[-0.200pt]{2.409pt}{0.400pt}}
\put(170.0,249.0){\rule[-0.200pt]{2.409pt}{0.400pt}}
\put(1429.0,249.0){\rule[-0.200pt]{2.409pt}{0.400pt}}
\put(170.0,250.0){\rule[-0.200pt]{2.409pt}{0.400pt}}
\put(1429.0,250.0){\rule[-0.200pt]{2.409pt}{0.400pt}}
\put(170.0,250.0){\rule[-0.200pt]{2.409pt}{0.400pt}}
\put(1429.0,250.0){\rule[-0.200pt]{2.409pt}{0.400pt}}
\put(170.0,251.0){\rule[-0.200pt]{2.409pt}{0.400pt}}
\put(1429.0,251.0){\rule[-0.200pt]{2.409pt}{0.400pt}}
\put(170.0,251.0){\rule[-0.200pt]{2.409pt}{0.400pt}}
\put(1429.0,251.0){\rule[-0.200pt]{2.409pt}{0.400pt}}
\put(170.0,252.0){\rule[-0.200pt]{2.409pt}{0.400pt}}
\put(1429.0,252.0){\rule[-0.200pt]{2.409pt}{0.400pt}}
\put(170.0,252.0){\rule[-0.200pt]{2.409pt}{0.400pt}}
\put(1429.0,252.0){\rule[-0.200pt]{2.409pt}{0.400pt}}
\put(170.0,252.0){\rule[-0.200pt]{2.409pt}{0.400pt}}
\put(1429.0,252.0){\rule[-0.200pt]{2.409pt}{0.400pt}}
\put(170.0,253.0){\rule[-0.200pt]{2.409pt}{0.400pt}}
\put(1429.0,253.0){\rule[-0.200pt]{2.409pt}{0.400pt}}
\put(170.0,253.0){\rule[-0.200pt]{2.409pt}{0.400pt}}
\put(1429.0,253.0){\rule[-0.200pt]{2.409pt}{0.400pt}}
\put(170.0,254.0){\rule[-0.200pt]{2.409pt}{0.400pt}}
\put(1429.0,254.0){\rule[-0.200pt]{2.409pt}{0.400pt}}
\put(170.0,254.0){\rule[-0.200pt]{2.409pt}{0.400pt}}
\put(1429.0,254.0){\rule[-0.200pt]{2.409pt}{0.400pt}}
\put(170.0,254.0){\rule[-0.200pt]{2.409pt}{0.400pt}}
\put(1429.0,254.0){\rule[-0.200pt]{2.409pt}{0.400pt}}
\put(170.0,255.0){\rule[-0.200pt]{2.409pt}{0.400pt}}
\put(1429.0,255.0){\rule[-0.200pt]{2.409pt}{0.400pt}}
\put(170.0,255.0){\rule[-0.200pt]{2.409pt}{0.400pt}}
\put(1429.0,255.0){\rule[-0.200pt]{2.409pt}{0.400pt}}
\put(170.0,255.0){\rule[-0.200pt]{2.409pt}{0.400pt}}
\put(1429.0,255.0){\rule[-0.200pt]{2.409pt}{0.400pt}}
\put(170.0,256.0){\rule[-0.200pt]{2.409pt}{0.400pt}}
\put(1429.0,256.0){\rule[-0.200pt]{2.409pt}{0.400pt}}
\put(170.0,256.0){\rule[-0.200pt]{2.409pt}{0.400pt}}
\put(1429.0,256.0){\rule[-0.200pt]{2.409pt}{0.400pt}}
\put(170.0,256.0){\rule[-0.200pt]{2.409pt}{0.400pt}}
\put(1429.0,256.0){\rule[-0.200pt]{2.409pt}{0.400pt}}
\put(170.0,257.0){\rule[-0.200pt]{2.409pt}{0.400pt}}
\put(1429.0,257.0){\rule[-0.200pt]{2.409pt}{0.400pt}}
\put(170.0,257.0){\rule[-0.200pt]{2.409pt}{0.400pt}}
\put(1429.0,257.0){\rule[-0.200pt]{2.409pt}{0.400pt}}
\put(170.0,258.0){\rule[-0.200pt]{2.409pt}{0.400pt}}
\put(1429.0,258.0){\rule[-0.200pt]{2.409pt}{0.400pt}}
\put(170.0,258.0){\rule[-0.200pt]{2.409pt}{0.400pt}}
\put(1429.0,258.0){\rule[-0.200pt]{2.409pt}{0.400pt}}
\put(170.0,258.0){\rule[-0.200pt]{2.409pt}{0.400pt}}
\put(1429.0,258.0){\rule[-0.200pt]{2.409pt}{0.400pt}}
\put(170.0,259.0){\rule[-0.200pt]{2.409pt}{0.400pt}}
\put(1429.0,259.0){\rule[-0.200pt]{2.409pt}{0.400pt}}
\put(170.0,259.0){\rule[-0.200pt]{2.409pt}{0.400pt}}
\put(1429.0,259.0){\rule[-0.200pt]{2.409pt}{0.400pt}}
\put(170.0,259.0){\rule[-0.200pt]{2.409pt}{0.400pt}}
\put(1429.0,259.0){\rule[-0.200pt]{2.409pt}{0.400pt}}
\put(170.0,260.0){\rule[-0.200pt]{2.409pt}{0.400pt}}
\put(1429.0,260.0){\rule[-0.200pt]{2.409pt}{0.400pt}}
\put(170.0,260.0){\rule[-0.200pt]{2.409pt}{0.400pt}}
\put(1429.0,260.0){\rule[-0.200pt]{2.409pt}{0.400pt}}
\put(170.0,260.0){\rule[-0.200pt]{2.409pt}{0.400pt}}
\put(1429.0,260.0){\rule[-0.200pt]{2.409pt}{0.400pt}}
\put(170.0,261.0){\rule[-0.200pt]{2.409pt}{0.400pt}}
\put(1429.0,261.0){\rule[-0.200pt]{2.409pt}{0.400pt}}
\put(170.0,261.0){\rule[-0.200pt]{2.409pt}{0.400pt}}
\put(1429.0,261.0){\rule[-0.200pt]{2.409pt}{0.400pt}}
\put(170.0,261.0){\rule[-0.200pt]{2.409pt}{0.400pt}}
\put(1429.0,261.0){\rule[-0.200pt]{2.409pt}{0.400pt}}
\put(170.0,262.0){\rule[-0.200pt]{2.409pt}{0.400pt}}
\put(1429.0,262.0){\rule[-0.200pt]{2.409pt}{0.400pt}}
\put(170.0,262.0){\rule[-0.200pt]{2.409pt}{0.400pt}}
\put(1429.0,262.0){\rule[-0.200pt]{2.409pt}{0.400pt}}
\put(170.0,262.0){\rule[-0.200pt]{2.409pt}{0.400pt}}
\put(1429.0,262.0){\rule[-0.200pt]{2.409pt}{0.400pt}}
\put(170.0,262.0){\rule[-0.200pt]{2.409pt}{0.400pt}}
\put(1429.0,262.0){\rule[-0.200pt]{2.409pt}{0.400pt}}
\put(170.0,263.0){\rule[-0.200pt]{2.409pt}{0.400pt}}
\put(1429.0,263.0){\rule[-0.200pt]{2.409pt}{0.400pt}}
\put(170.0,263.0){\rule[-0.200pt]{2.409pt}{0.400pt}}
\put(1429.0,263.0){\rule[-0.200pt]{2.409pt}{0.400pt}}
\put(170.0,263.0){\rule[-0.200pt]{2.409pt}{0.400pt}}
\put(1429.0,263.0){\rule[-0.200pt]{2.409pt}{0.400pt}}
\put(170.0,264.0){\rule[-0.200pt]{2.409pt}{0.400pt}}
\put(1429.0,264.0){\rule[-0.200pt]{2.409pt}{0.400pt}}
\put(170.0,264.0){\rule[-0.200pt]{2.409pt}{0.400pt}}
\put(1429.0,264.0){\rule[-0.200pt]{2.409pt}{0.400pt}}
\put(170.0,264.0){\rule[-0.200pt]{2.409pt}{0.400pt}}
\put(1429.0,264.0){\rule[-0.200pt]{2.409pt}{0.400pt}}
\put(170.0,265.0){\rule[-0.200pt]{2.409pt}{0.400pt}}
\put(1429.0,265.0){\rule[-0.200pt]{2.409pt}{0.400pt}}
\put(170.0,265.0){\rule[-0.200pt]{2.409pt}{0.400pt}}
\put(1429.0,265.0){\rule[-0.200pt]{2.409pt}{0.400pt}}
\put(170.0,265.0){\rule[-0.200pt]{2.409pt}{0.400pt}}
\put(1429.0,265.0){\rule[-0.200pt]{2.409pt}{0.400pt}}
\put(170.0,265.0){\rule[-0.200pt]{2.409pt}{0.400pt}}
\put(1429.0,265.0){\rule[-0.200pt]{2.409pt}{0.400pt}}
\put(170.0,266.0){\rule[-0.200pt]{2.409pt}{0.400pt}}
\put(1429.0,266.0){\rule[-0.200pt]{2.409pt}{0.400pt}}
\put(170.0,266.0){\rule[-0.200pt]{2.409pt}{0.400pt}}
\put(1429.0,266.0){\rule[-0.200pt]{2.409pt}{0.400pt}}
\put(170.0,266.0){\rule[-0.200pt]{2.409pt}{0.400pt}}
\put(1429.0,266.0){\rule[-0.200pt]{2.409pt}{0.400pt}}
\put(170.0,266.0){\rule[-0.200pt]{2.409pt}{0.400pt}}
\put(1429.0,266.0){\rule[-0.200pt]{2.409pt}{0.400pt}}
\put(170.0,267.0){\rule[-0.200pt]{2.409pt}{0.400pt}}
\put(1429.0,267.0){\rule[-0.200pt]{2.409pt}{0.400pt}}
\put(170.0,267.0){\rule[-0.200pt]{2.409pt}{0.400pt}}
\put(1429.0,267.0){\rule[-0.200pt]{2.409pt}{0.400pt}}
\put(170.0,267.0){\rule[-0.200pt]{2.409pt}{0.400pt}}
\put(1429.0,267.0){\rule[-0.200pt]{2.409pt}{0.400pt}}
\put(170.0,268.0){\rule[-0.200pt]{2.409pt}{0.400pt}}
\put(1429.0,268.0){\rule[-0.200pt]{2.409pt}{0.400pt}}
\put(170.0,268.0){\rule[-0.200pt]{2.409pt}{0.400pt}}
\put(1429.0,268.0){\rule[-0.200pt]{2.409pt}{0.400pt}}
\put(170.0,268.0){\rule[-0.200pt]{2.409pt}{0.400pt}}
\put(1429.0,268.0){\rule[-0.200pt]{2.409pt}{0.400pt}}
\put(170.0,268.0){\rule[-0.200pt]{2.409pt}{0.400pt}}
\put(1429.0,268.0){\rule[-0.200pt]{2.409pt}{0.400pt}}
\put(170.0,269.0){\rule[-0.200pt]{2.409pt}{0.400pt}}
\put(1429.0,269.0){\rule[-0.200pt]{2.409pt}{0.400pt}}
\put(170.0,269.0){\rule[-0.200pt]{2.409pt}{0.400pt}}
\put(1429.0,269.0){\rule[-0.200pt]{2.409pt}{0.400pt}}
\put(170.0,269.0){\rule[-0.200pt]{2.409pt}{0.400pt}}
\put(1429.0,269.0){\rule[-0.200pt]{2.409pt}{0.400pt}}
\put(170.0,269.0){\rule[-0.200pt]{2.409pt}{0.400pt}}
\put(1429.0,269.0){\rule[-0.200pt]{2.409pt}{0.400pt}}
\put(170.0,270.0){\rule[-0.200pt]{2.409pt}{0.400pt}}
\put(1429.0,270.0){\rule[-0.200pt]{2.409pt}{0.400pt}}
\put(170.0,270.0){\rule[-0.200pt]{2.409pt}{0.400pt}}
\put(1429.0,270.0){\rule[-0.200pt]{2.409pt}{0.400pt}}
\put(170.0,270.0){\rule[-0.200pt]{2.409pt}{0.400pt}}
\put(1429.0,270.0){\rule[-0.200pt]{2.409pt}{0.400pt}}
\put(170.0,270.0){\rule[-0.200pt]{2.409pt}{0.400pt}}
\put(1429.0,270.0){\rule[-0.200pt]{2.409pt}{0.400pt}}
\put(170.0,271.0){\rule[-0.200pt]{2.409pt}{0.400pt}}
\put(1429.0,271.0){\rule[-0.200pt]{2.409pt}{0.400pt}}
\put(170.0,271.0){\rule[-0.200pt]{2.409pt}{0.400pt}}
\put(1429.0,271.0){\rule[-0.200pt]{2.409pt}{0.400pt}}
\put(170.0,271.0){\rule[-0.200pt]{2.409pt}{0.400pt}}
\put(1429.0,271.0){\rule[-0.200pt]{2.409pt}{0.400pt}}
\put(170.0,271.0){\rule[-0.200pt]{2.409pt}{0.400pt}}
\put(1429.0,271.0){\rule[-0.200pt]{2.409pt}{0.400pt}}
\put(170.0,272.0){\rule[-0.200pt]{2.409pt}{0.400pt}}
\put(1429.0,272.0){\rule[-0.200pt]{2.409pt}{0.400pt}}
\put(170.0,272.0){\rule[-0.200pt]{2.409pt}{0.400pt}}
\put(1429.0,272.0){\rule[-0.200pt]{2.409pt}{0.400pt}}
\put(170.0,272.0){\rule[-0.200pt]{2.409pt}{0.400pt}}
\put(1429.0,272.0){\rule[-0.200pt]{2.409pt}{0.400pt}}
\put(170.0,272.0){\rule[-0.200pt]{2.409pt}{0.400pt}}
\put(1429.0,272.0){\rule[-0.200pt]{2.409pt}{0.400pt}}
\put(170.0,273.0){\rule[-0.200pt]{2.409pt}{0.400pt}}
\put(1429.0,273.0){\rule[-0.200pt]{2.409pt}{0.400pt}}
\put(170.0,273.0){\rule[-0.200pt]{2.409pt}{0.400pt}}
\put(1429.0,273.0){\rule[-0.200pt]{2.409pt}{0.400pt}}
\put(170.0,273.0){\rule[-0.200pt]{2.409pt}{0.400pt}}
\put(1429.0,273.0){\rule[-0.200pt]{2.409pt}{0.400pt}}
\put(170.0,273.0){\rule[-0.200pt]{2.409pt}{0.400pt}}
\put(1429.0,273.0){\rule[-0.200pt]{2.409pt}{0.400pt}}
\put(170.0,273.0){\rule[-0.200pt]{2.409pt}{0.400pt}}
\put(1429.0,273.0){\rule[-0.200pt]{2.409pt}{0.400pt}}
\put(170.0,274.0){\rule[-0.200pt]{2.409pt}{0.400pt}}
\put(1429.0,274.0){\rule[-0.200pt]{2.409pt}{0.400pt}}
\put(170.0,274.0){\rule[-0.200pt]{2.409pt}{0.400pt}}
\put(1429.0,274.0){\rule[-0.200pt]{2.409pt}{0.400pt}}
\put(170.0,274.0){\rule[-0.200pt]{2.409pt}{0.400pt}}
\put(1429.0,274.0){\rule[-0.200pt]{2.409pt}{0.400pt}}
\put(170.0,274.0){\rule[-0.200pt]{2.409pt}{0.400pt}}
\put(1429.0,274.0){\rule[-0.200pt]{2.409pt}{0.400pt}}
\put(170.0,275.0){\rule[-0.200pt]{2.409pt}{0.400pt}}
\put(1429.0,275.0){\rule[-0.200pt]{2.409pt}{0.400pt}}
\put(170.0,275.0){\rule[-0.200pt]{2.409pt}{0.400pt}}
\put(1429.0,275.0){\rule[-0.200pt]{2.409pt}{0.400pt}}
\put(170.0,275.0){\rule[-0.200pt]{2.409pt}{0.400pt}}
\put(1429.0,275.0){\rule[-0.200pt]{2.409pt}{0.400pt}}
\put(170.0,275.0){\rule[-0.200pt]{2.409pt}{0.400pt}}
\put(1429.0,275.0){\rule[-0.200pt]{2.409pt}{0.400pt}}
\put(170.0,275.0){\rule[-0.200pt]{2.409pt}{0.400pt}}
\put(1429.0,275.0){\rule[-0.200pt]{2.409pt}{0.400pt}}
\put(170.0,276.0){\rule[-0.200pt]{2.409pt}{0.400pt}}
\put(1429.0,276.0){\rule[-0.200pt]{2.409pt}{0.400pt}}
\put(170.0,276.0){\rule[-0.200pt]{2.409pt}{0.400pt}}
\put(1429.0,276.0){\rule[-0.200pt]{2.409pt}{0.400pt}}
\put(170.0,276.0){\rule[-0.200pt]{2.409pt}{0.400pt}}
\put(1429.0,276.0){\rule[-0.200pt]{2.409pt}{0.400pt}}
\put(170.0,276.0){\rule[-0.200pt]{2.409pt}{0.400pt}}
\put(1429.0,276.0){\rule[-0.200pt]{2.409pt}{0.400pt}}
\put(170.0,276.0){\rule[-0.200pt]{2.409pt}{0.400pt}}
\put(1429.0,276.0){\rule[-0.200pt]{2.409pt}{0.400pt}}
\put(170.0,277.0){\rule[-0.200pt]{2.409pt}{0.400pt}}
\put(1429.0,277.0){\rule[-0.200pt]{2.409pt}{0.400pt}}
\put(170.0,277.0){\rule[-0.200pt]{2.409pt}{0.400pt}}
\put(1429.0,277.0){\rule[-0.200pt]{2.409pt}{0.400pt}}
\put(170.0,277.0){\rule[-0.200pt]{2.409pt}{0.400pt}}
\put(1429.0,277.0){\rule[-0.200pt]{2.409pt}{0.400pt}}
\put(170.0,277.0){\rule[-0.200pt]{2.409pt}{0.400pt}}
\put(1429.0,277.0){\rule[-0.200pt]{2.409pt}{0.400pt}}
\put(170.0,278.0){\rule[-0.200pt]{2.409pt}{0.400pt}}
\put(1429.0,278.0){\rule[-0.200pt]{2.409pt}{0.400pt}}
\put(170.0,278.0){\rule[-0.200pt]{2.409pt}{0.400pt}}
\put(1429.0,278.0){\rule[-0.200pt]{2.409pt}{0.400pt}}
\put(170.0,278.0){\rule[-0.200pt]{2.409pt}{0.400pt}}
\put(1429.0,278.0){\rule[-0.200pt]{2.409pt}{0.400pt}}
\put(170.0,278.0){\rule[-0.200pt]{2.409pt}{0.400pt}}
\put(1429.0,278.0){\rule[-0.200pt]{2.409pt}{0.400pt}}
\put(170.0,278.0){\rule[-0.200pt]{2.409pt}{0.400pt}}
\put(1429.0,278.0){\rule[-0.200pt]{2.409pt}{0.400pt}}
\put(170.0,279.0){\rule[-0.200pt]{2.409pt}{0.400pt}}
\put(1429.0,279.0){\rule[-0.200pt]{2.409pt}{0.400pt}}
\put(170.0,279.0){\rule[-0.200pt]{2.409pt}{0.400pt}}
\put(1429.0,279.0){\rule[-0.200pt]{2.409pt}{0.400pt}}
\put(170.0,279.0){\rule[-0.200pt]{2.409pt}{0.400pt}}
\put(1429.0,279.0){\rule[-0.200pt]{2.409pt}{0.400pt}}
\put(170.0,279.0){\rule[-0.200pt]{2.409pt}{0.400pt}}
\put(1429.0,279.0){\rule[-0.200pt]{2.409pt}{0.400pt}}
\put(170.0,279.0){\rule[-0.200pt]{2.409pt}{0.400pt}}
\put(1429.0,279.0){\rule[-0.200pt]{2.409pt}{0.400pt}}
\put(170.0,280.0){\rule[-0.200pt]{2.409pt}{0.400pt}}
\put(1429.0,280.0){\rule[-0.200pt]{2.409pt}{0.400pt}}
\put(170.0,280.0){\rule[-0.200pt]{2.409pt}{0.400pt}}
\put(1429.0,280.0){\rule[-0.200pt]{2.409pt}{0.400pt}}
\put(170.0,280.0){\rule[-0.200pt]{2.409pt}{0.400pt}}
\put(1429.0,280.0){\rule[-0.200pt]{2.409pt}{0.400pt}}
\put(170.0,280.0){\rule[-0.200pt]{2.409pt}{0.400pt}}
\put(1429.0,280.0){\rule[-0.200pt]{2.409pt}{0.400pt}}
\put(170.0,280.0){\rule[-0.200pt]{2.409pt}{0.400pt}}
\put(1429.0,280.0){\rule[-0.200pt]{2.409pt}{0.400pt}}
\put(170.0,280.0){\rule[-0.200pt]{2.409pt}{0.400pt}}
\put(1429.0,280.0){\rule[-0.200pt]{2.409pt}{0.400pt}}
\put(170.0,281.0){\rule[-0.200pt]{2.409pt}{0.400pt}}
\put(1429.0,281.0){\rule[-0.200pt]{2.409pt}{0.400pt}}
\put(170.0,281.0){\rule[-0.200pt]{2.409pt}{0.400pt}}
\put(1429.0,281.0){\rule[-0.200pt]{2.409pt}{0.400pt}}
\put(170.0,281.0){\rule[-0.200pt]{2.409pt}{0.400pt}}
\put(1429.0,281.0){\rule[-0.200pt]{2.409pt}{0.400pt}}
\put(170.0,281.0){\rule[-0.200pt]{2.409pt}{0.400pt}}
\put(1429.0,281.0){\rule[-0.200pt]{2.409pt}{0.400pt}}
\put(170.0,281.0){\rule[-0.200pt]{2.409pt}{0.400pt}}
\put(1429.0,281.0){\rule[-0.200pt]{2.409pt}{0.400pt}}
\put(170.0,282.0){\rule[-0.200pt]{2.409pt}{0.400pt}}
\put(1429.0,282.0){\rule[-0.200pt]{2.409pt}{0.400pt}}
\put(170.0,282.0){\rule[-0.200pt]{2.409pt}{0.400pt}}
\put(1429.0,282.0){\rule[-0.200pt]{2.409pt}{0.400pt}}
\put(170.0,282.0){\rule[-0.200pt]{2.409pt}{0.400pt}}
\put(1429.0,282.0){\rule[-0.200pt]{2.409pt}{0.400pt}}
\put(170.0,282.0){\rule[-0.200pt]{2.409pt}{0.400pt}}
\put(1429.0,282.0){\rule[-0.200pt]{2.409pt}{0.400pt}}
\put(170.0,282.0){\rule[-0.200pt]{2.409pt}{0.400pt}}
\put(1429.0,282.0){\rule[-0.200pt]{2.409pt}{0.400pt}}
\put(170.0,282.0){\rule[-0.200pt]{2.409pt}{0.400pt}}
\put(1429.0,282.0){\rule[-0.200pt]{2.409pt}{0.400pt}}
\put(170.0,283.0){\rule[-0.200pt]{2.409pt}{0.400pt}}
\put(1429.0,283.0){\rule[-0.200pt]{2.409pt}{0.400pt}}
\put(170.0,283.0){\rule[-0.200pt]{2.409pt}{0.400pt}}
\put(1429.0,283.0){\rule[-0.200pt]{2.409pt}{0.400pt}}
\put(170.0,283.0){\rule[-0.200pt]{2.409pt}{0.400pt}}
\put(1429.0,283.0){\rule[-0.200pt]{2.409pt}{0.400pt}}
\put(170.0,283.0){\rule[-0.200pt]{2.409pt}{0.400pt}}
\put(1429.0,283.0){\rule[-0.200pt]{2.409pt}{0.400pt}}
\put(170.0,283.0){\rule[-0.200pt]{2.409pt}{0.400pt}}
\put(1429.0,283.0){\rule[-0.200pt]{2.409pt}{0.400pt}}
\put(170.0,284.0){\rule[-0.200pt]{2.409pt}{0.400pt}}
\put(1429.0,284.0){\rule[-0.200pt]{2.409pt}{0.400pt}}
\put(170.0,284.0){\rule[-0.200pt]{2.409pt}{0.400pt}}
\put(1429.0,284.0){\rule[-0.200pt]{2.409pt}{0.400pt}}
\put(170.0,284.0){\rule[-0.200pt]{2.409pt}{0.400pt}}
\put(1429.0,284.0){\rule[-0.200pt]{2.409pt}{0.400pt}}
\put(170.0,284.0){\rule[-0.200pt]{2.409pt}{0.400pt}}
\put(1429.0,284.0){\rule[-0.200pt]{2.409pt}{0.400pt}}
\put(170.0,284.0){\rule[-0.200pt]{2.409pt}{0.400pt}}
\put(1429.0,284.0){\rule[-0.200pt]{2.409pt}{0.400pt}}
\put(170.0,284.0){\rule[-0.200pt]{2.409pt}{0.400pt}}
\put(1429.0,284.0){\rule[-0.200pt]{2.409pt}{0.400pt}}
\put(170.0,285.0){\rule[-0.200pt]{2.409pt}{0.400pt}}
\put(1429.0,285.0){\rule[-0.200pt]{2.409pt}{0.400pt}}
\put(170.0,285.0){\rule[-0.200pt]{2.409pt}{0.400pt}}
\put(1429.0,285.0){\rule[-0.200pt]{2.409pt}{0.400pt}}
\put(170.0,285.0){\rule[-0.200pt]{2.409pt}{0.400pt}}
\put(1429.0,285.0){\rule[-0.200pt]{2.409pt}{0.400pt}}
\put(170.0,285.0){\rule[-0.200pt]{2.409pt}{0.400pt}}
\put(1429.0,285.0){\rule[-0.200pt]{2.409pt}{0.400pt}}
\put(170.0,285.0){\rule[-0.200pt]{2.409pt}{0.400pt}}
\put(1429.0,285.0){\rule[-0.200pt]{2.409pt}{0.400pt}}
\put(170.0,285.0){\rule[-0.200pt]{2.409pt}{0.400pt}}
\put(1429.0,285.0){\rule[-0.200pt]{2.409pt}{0.400pt}}
\put(170.0,286.0){\rule[-0.200pt]{2.409pt}{0.400pt}}
\put(1429.0,286.0){\rule[-0.200pt]{2.409pt}{0.400pt}}
\put(170.0,286.0){\rule[-0.200pt]{2.409pt}{0.400pt}}
\put(1429.0,286.0){\rule[-0.200pt]{2.409pt}{0.400pt}}
\put(170.0,286.0){\rule[-0.200pt]{2.409pt}{0.400pt}}
\put(1429.0,286.0){\rule[-0.200pt]{2.409pt}{0.400pt}}
\put(170.0,286.0){\rule[-0.200pt]{2.409pt}{0.400pt}}
\put(1429.0,286.0){\rule[-0.200pt]{2.409pt}{0.400pt}}
\put(170.0,286.0){\rule[-0.200pt]{2.409pt}{0.400pt}}
\put(1429.0,286.0){\rule[-0.200pt]{2.409pt}{0.400pt}}
\put(170.0,286.0){\rule[-0.200pt]{2.409pt}{0.400pt}}
\put(1429.0,286.0){\rule[-0.200pt]{2.409pt}{0.400pt}}
\put(170.0,287.0){\rule[-0.200pt]{2.409pt}{0.400pt}}
\put(1429.0,287.0){\rule[-0.200pt]{2.409pt}{0.400pt}}
\put(170.0,287.0){\rule[-0.200pt]{2.409pt}{0.400pt}}
\put(1429.0,287.0){\rule[-0.200pt]{2.409pt}{0.400pt}}
\put(170.0,287.0){\rule[-0.200pt]{2.409pt}{0.400pt}}
\put(1429.0,287.0){\rule[-0.200pt]{2.409pt}{0.400pt}}
\put(170.0,287.0){\rule[-0.200pt]{2.409pt}{0.400pt}}
\put(1429.0,287.0){\rule[-0.200pt]{2.409pt}{0.400pt}}
\put(170.0,287.0){\rule[-0.200pt]{2.409pt}{0.400pt}}
\put(1429.0,287.0){\rule[-0.200pt]{2.409pt}{0.400pt}}
\put(170.0,287.0){\rule[-0.200pt]{2.409pt}{0.400pt}}
\put(1429.0,287.0){\rule[-0.200pt]{2.409pt}{0.400pt}}
\put(170.0,287.0){\rule[-0.200pt]{2.409pt}{0.400pt}}
\put(1429.0,287.0){\rule[-0.200pt]{2.409pt}{0.400pt}}
\put(170.0,288.0){\rule[-0.200pt]{2.409pt}{0.400pt}}
\put(1429.0,288.0){\rule[-0.200pt]{2.409pt}{0.400pt}}
\put(170.0,288.0){\rule[-0.200pt]{2.409pt}{0.400pt}}
\put(1429.0,288.0){\rule[-0.200pt]{2.409pt}{0.400pt}}
\put(170.0,288.0){\rule[-0.200pt]{2.409pt}{0.400pt}}
\put(1429.0,288.0){\rule[-0.200pt]{2.409pt}{0.400pt}}
\put(170.0,288.0){\rule[-0.200pt]{2.409pt}{0.400pt}}
\put(1429.0,288.0){\rule[-0.200pt]{2.409pt}{0.400pt}}
\put(170.0,288.0){\rule[-0.200pt]{2.409pt}{0.400pt}}
\put(1429.0,288.0){\rule[-0.200pt]{2.409pt}{0.400pt}}
\put(170.0,288.0){\rule[-0.200pt]{2.409pt}{0.400pt}}
\put(1429.0,288.0){\rule[-0.200pt]{2.409pt}{0.400pt}}
\put(170.0,289.0){\rule[-0.200pt]{2.409pt}{0.400pt}}
\put(1429.0,289.0){\rule[-0.200pt]{2.409pt}{0.400pt}}
\put(170.0,289.0){\rule[-0.200pt]{2.409pt}{0.400pt}}
\put(1429.0,289.0){\rule[-0.200pt]{2.409pt}{0.400pt}}
\put(170.0,289.0){\rule[-0.200pt]{2.409pt}{0.400pt}}
\put(1429.0,289.0){\rule[-0.200pt]{2.409pt}{0.400pt}}
\put(170.0,289.0){\rule[-0.200pt]{2.409pt}{0.400pt}}
\put(1429.0,289.0){\rule[-0.200pt]{2.409pt}{0.400pt}}
\put(170.0,289.0){\rule[-0.200pt]{2.409pt}{0.400pt}}
\put(1429.0,289.0){\rule[-0.200pt]{2.409pt}{0.400pt}}
\put(170.0,289.0){\rule[-0.200pt]{2.409pt}{0.400pt}}
\put(1429.0,289.0){\rule[-0.200pt]{2.409pt}{0.400pt}}
\put(170.0,289.0){\rule[-0.200pt]{2.409pt}{0.400pt}}
\put(1429.0,289.0){\rule[-0.200pt]{2.409pt}{0.400pt}}
\put(170.0,290.0){\rule[-0.200pt]{2.409pt}{0.400pt}}
\put(1429.0,290.0){\rule[-0.200pt]{2.409pt}{0.400pt}}
\put(170.0,290.0){\rule[-0.200pt]{2.409pt}{0.400pt}}
\put(1429.0,290.0){\rule[-0.200pt]{2.409pt}{0.400pt}}
\put(170.0,290.0){\rule[-0.200pt]{2.409pt}{0.400pt}}
\put(1429.0,290.0){\rule[-0.200pt]{2.409pt}{0.400pt}}
\put(170.0,290.0){\rule[-0.200pt]{2.409pt}{0.400pt}}
\put(1429.0,290.0){\rule[-0.200pt]{2.409pt}{0.400pt}}
\put(170.0,290.0){\rule[-0.200pt]{2.409pt}{0.400pt}}
\put(1429.0,290.0){\rule[-0.200pt]{2.409pt}{0.400pt}}
\put(170.0,290.0){\rule[-0.200pt]{2.409pt}{0.400pt}}
\put(1429.0,290.0){\rule[-0.200pt]{2.409pt}{0.400pt}}
\put(170.0,290.0){\rule[-0.200pt]{2.409pt}{0.400pt}}
\put(1429.0,290.0){\rule[-0.200pt]{2.409pt}{0.400pt}}
\put(170.0,291.0){\rule[-0.200pt]{2.409pt}{0.400pt}}
\put(1429.0,291.0){\rule[-0.200pt]{2.409pt}{0.400pt}}
\put(170.0,291.0){\rule[-0.200pt]{2.409pt}{0.400pt}}
\put(1429.0,291.0){\rule[-0.200pt]{2.409pt}{0.400pt}}
\put(170.0,291.0){\rule[-0.200pt]{2.409pt}{0.400pt}}
\put(1429.0,291.0){\rule[-0.200pt]{2.409pt}{0.400pt}}
\put(170.0,291.0){\rule[-0.200pt]{2.409pt}{0.400pt}}
\put(1429.0,291.0){\rule[-0.200pt]{2.409pt}{0.400pt}}
\put(170.0,291.0){\rule[-0.200pt]{2.409pt}{0.400pt}}
\put(1429.0,291.0){\rule[-0.200pt]{2.409pt}{0.400pt}}
\put(170.0,291.0){\rule[-0.200pt]{2.409pt}{0.400pt}}
\put(1429.0,291.0){\rule[-0.200pt]{2.409pt}{0.400pt}}
\put(170.0,291.0){\rule[-0.200pt]{2.409pt}{0.400pt}}
\put(1429.0,291.0){\rule[-0.200pt]{2.409pt}{0.400pt}}
\put(170.0,292.0){\rule[-0.200pt]{2.409pt}{0.400pt}}
\put(1429.0,292.0){\rule[-0.200pt]{2.409pt}{0.400pt}}
\put(170.0,292.0){\rule[-0.200pt]{2.409pt}{0.400pt}}
\put(1429.0,292.0){\rule[-0.200pt]{2.409pt}{0.400pt}}
\put(170.0,292.0){\rule[-0.200pt]{2.409pt}{0.400pt}}
\put(1429.0,292.0){\rule[-0.200pt]{2.409pt}{0.400pt}}
\put(170.0,292.0){\rule[-0.200pt]{2.409pt}{0.400pt}}
\put(1429.0,292.0){\rule[-0.200pt]{2.409pt}{0.400pt}}
\put(170.0,292.0){\rule[-0.200pt]{2.409pt}{0.400pt}}
\put(1429.0,292.0){\rule[-0.200pt]{2.409pt}{0.400pt}}
\put(170.0,292.0){\rule[-0.200pt]{2.409pt}{0.400pt}}
\put(1429.0,292.0){\rule[-0.200pt]{2.409pt}{0.400pt}}
\put(170.0,292.0){\rule[-0.200pt]{2.409pt}{0.400pt}}
\put(1429.0,292.0){\rule[-0.200pt]{2.409pt}{0.400pt}}
\put(170.0,293.0){\rule[-0.200pt]{2.409pt}{0.400pt}}
\put(1429.0,293.0){\rule[-0.200pt]{2.409pt}{0.400pt}}
\put(170.0,293.0){\rule[-0.200pt]{2.409pt}{0.400pt}}
\put(1429.0,293.0){\rule[-0.200pt]{2.409pt}{0.400pt}}
\put(170.0,293.0){\rule[-0.200pt]{2.409pt}{0.400pt}}
\put(1429.0,293.0){\rule[-0.200pt]{2.409pt}{0.400pt}}
\put(170.0,293.0){\rule[-0.200pt]{2.409pt}{0.400pt}}
\put(1429.0,293.0){\rule[-0.200pt]{2.409pt}{0.400pt}}
\put(170.0,293.0){\rule[-0.200pt]{2.409pt}{0.400pt}}
\put(1429.0,293.0){\rule[-0.200pt]{2.409pt}{0.400pt}}
\put(170.0,293.0){\rule[-0.200pt]{2.409pt}{0.400pt}}
\put(1429.0,293.0){\rule[-0.200pt]{2.409pt}{0.400pt}}
\put(170.0,293.0){\rule[-0.200pt]{2.409pt}{0.400pt}}
\put(1429.0,293.0){\rule[-0.200pt]{2.409pt}{0.400pt}}
\put(170.0,294.0){\rule[-0.200pt]{2.409pt}{0.400pt}}
\put(1429.0,294.0){\rule[-0.200pt]{2.409pt}{0.400pt}}
\put(170.0,294.0){\rule[-0.200pt]{2.409pt}{0.400pt}}
\put(1429.0,294.0){\rule[-0.200pt]{2.409pt}{0.400pt}}
\put(170.0,294.0){\rule[-0.200pt]{2.409pt}{0.400pt}}
\put(1429.0,294.0){\rule[-0.200pt]{2.409pt}{0.400pt}}
\put(170.0,294.0){\rule[-0.200pt]{2.409pt}{0.400pt}}
\put(1429.0,294.0){\rule[-0.200pt]{2.409pt}{0.400pt}}
\put(170.0,294.0){\rule[-0.200pt]{2.409pt}{0.400pt}}
\put(1429.0,294.0){\rule[-0.200pt]{2.409pt}{0.400pt}}
\put(170.0,294.0){\rule[-0.200pt]{2.409pt}{0.400pt}}
\put(1429.0,294.0){\rule[-0.200pt]{2.409pt}{0.400pt}}
\put(170.0,294.0){\rule[-0.200pt]{2.409pt}{0.400pt}}
\put(1429.0,294.0){\rule[-0.200pt]{2.409pt}{0.400pt}}
\put(170.0,294.0){\rule[-0.200pt]{2.409pt}{0.400pt}}
\put(1429.0,294.0){\rule[-0.200pt]{2.409pt}{0.400pt}}
\put(170.0,295.0){\rule[-0.200pt]{2.409pt}{0.400pt}}
\put(1429.0,295.0){\rule[-0.200pt]{2.409pt}{0.400pt}}
\put(170.0,295.0){\rule[-0.200pt]{2.409pt}{0.400pt}}
\put(1429.0,295.0){\rule[-0.200pt]{2.409pt}{0.400pt}}
\put(170.0,295.0){\rule[-0.200pt]{2.409pt}{0.400pt}}
\put(1429.0,295.0){\rule[-0.200pt]{2.409pt}{0.400pt}}
\put(170.0,295.0){\rule[-0.200pt]{2.409pt}{0.400pt}}
\put(1429.0,295.0){\rule[-0.200pt]{2.409pt}{0.400pt}}
\put(170.0,295.0){\rule[-0.200pt]{2.409pt}{0.400pt}}
\put(1429.0,295.0){\rule[-0.200pt]{2.409pt}{0.400pt}}
\put(170.0,295.0){\rule[-0.200pt]{2.409pt}{0.400pt}}
\put(1429.0,295.0){\rule[-0.200pt]{2.409pt}{0.400pt}}
\put(170.0,295.0){\rule[-0.200pt]{2.409pt}{0.400pt}}
\put(1429.0,295.0){\rule[-0.200pt]{2.409pt}{0.400pt}}
\put(170.0,295.0){\rule[-0.200pt]{2.409pt}{0.400pt}}
\put(1429.0,295.0){\rule[-0.200pt]{2.409pt}{0.400pt}}
\put(170.0,296.0){\rule[-0.200pt]{2.409pt}{0.400pt}}
\put(1429.0,296.0){\rule[-0.200pt]{2.409pt}{0.400pt}}
\put(170.0,296.0){\rule[-0.200pt]{2.409pt}{0.400pt}}
\put(1429.0,296.0){\rule[-0.200pt]{2.409pt}{0.400pt}}
\put(170.0,296.0){\rule[-0.200pt]{2.409pt}{0.400pt}}
\put(1429.0,296.0){\rule[-0.200pt]{2.409pt}{0.400pt}}
\put(170.0,296.0){\rule[-0.200pt]{2.409pt}{0.400pt}}
\put(1429.0,296.0){\rule[-0.200pt]{2.409pt}{0.400pt}}
\put(170.0,296.0){\rule[-0.200pt]{2.409pt}{0.400pt}}
\put(1429.0,296.0){\rule[-0.200pt]{2.409pt}{0.400pt}}
\put(170.0,296.0){\rule[-0.200pt]{2.409pt}{0.400pt}}
\put(1429.0,296.0){\rule[-0.200pt]{2.409pt}{0.400pt}}
\put(170.0,296.0){\rule[-0.200pt]{2.409pt}{0.400pt}}
\put(1429.0,296.0){\rule[-0.200pt]{2.409pt}{0.400pt}}
\put(170.0,296.0){\rule[-0.200pt]{2.409pt}{0.400pt}}
\put(1429.0,296.0){\rule[-0.200pt]{2.409pt}{0.400pt}}
\put(170.0,297.0){\rule[-0.200pt]{2.409pt}{0.400pt}}
\put(1429.0,297.0){\rule[-0.200pt]{2.409pt}{0.400pt}}
\put(170.0,297.0){\rule[-0.200pt]{2.409pt}{0.400pt}}
\put(1429.0,297.0){\rule[-0.200pt]{2.409pt}{0.400pt}}
\put(170.0,297.0){\rule[-0.200pt]{2.409pt}{0.400pt}}
\put(1429.0,297.0){\rule[-0.200pt]{2.409pt}{0.400pt}}
\put(170.0,297.0){\rule[-0.200pt]{2.409pt}{0.400pt}}
\put(1429.0,297.0){\rule[-0.200pt]{2.409pt}{0.400pt}}
\put(170.0,297.0){\rule[-0.200pt]{2.409pt}{0.400pt}}
\put(1429.0,297.0){\rule[-0.200pt]{2.409pt}{0.400pt}}
\put(170.0,297.0){\rule[-0.200pt]{2.409pt}{0.400pt}}
\put(1429.0,297.0){\rule[-0.200pt]{2.409pt}{0.400pt}}
\put(170.0,297.0){\rule[-0.200pt]{2.409pt}{0.400pt}}
\put(1429.0,297.0){\rule[-0.200pt]{2.409pt}{0.400pt}}
\put(170.0,297.0){\rule[-0.200pt]{2.409pt}{0.400pt}}
\put(1429.0,297.0){\rule[-0.200pt]{2.409pt}{0.400pt}}
\put(170.0,298.0){\rule[-0.200pt]{2.409pt}{0.400pt}}
\put(1429.0,298.0){\rule[-0.200pt]{2.409pt}{0.400pt}}
\put(170.0,298.0){\rule[-0.200pt]{2.409pt}{0.400pt}}
\put(1429.0,298.0){\rule[-0.200pt]{2.409pt}{0.400pt}}
\put(170.0,298.0){\rule[-0.200pt]{2.409pt}{0.400pt}}
\put(1429.0,298.0){\rule[-0.200pt]{2.409pt}{0.400pt}}
\put(170.0,298.0){\rule[-0.200pt]{2.409pt}{0.400pt}}
\put(1429.0,298.0){\rule[-0.200pt]{2.409pt}{0.400pt}}
\put(170.0,298.0){\rule[-0.200pt]{2.409pt}{0.400pt}}
\put(1429.0,298.0){\rule[-0.200pt]{2.409pt}{0.400pt}}
\put(170.0,298.0){\rule[-0.200pt]{2.409pt}{0.400pt}}
\put(1429.0,298.0){\rule[-0.200pt]{2.409pt}{0.400pt}}
\put(170.0,298.0){\rule[-0.200pt]{2.409pt}{0.400pt}}
\put(1429.0,298.0){\rule[-0.200pt]{2.409pt}{0.400pt}}
\put(170.0,298.0){\rule[-0.200pt]{2.409pt}{0.400pt}}
\put(1429.0,298.0){\rule[-0.200pt]{2.409pt}{0.400pt}}
\put(170.0,299.0){\rule[-0.200pt]{2.409pt}{0.400pt}}
\put(1429.0,299.0){\rule[-0.200pt]{2.409pt}{0.400pt}}
\put(170.0,299.0){\rule[-0.200pt]{2.409pt}{0.400pt}}
\put(1429.0,299.0){\rule[-0.200pt]{2.409pt}{0.400pt}}
\put(170.0,299.0){\rule[-0.200pt]{2.409pt}{0.400pt}}
\put(1429.0,299.0){\rule[-0.200pt]{2.409pt}{0.400pt}}
\put(170.0,299.0){\rule[-0.200pt]{2.409pt}{0.400pt}}
\put(1429.0,299.0){\rule[-0.200pt]{2.409pt}{0.400pt}}
\put(170.0,299.0){\rule[-0.200pt]{2.409pt}{0.400pt}}
\put(1429.0,299.0){\rule[-0.200pt]{2.409pt}{0.400pt}}
\put(170.0,299.0){\rule[-0.200pt]{2.409pt}{0.400pt}}
\put(1429.0,299.0){\rule[-0.200pt]{2.409pt}{0.400pt}}
\put(170.0,299.0){\rule[-0.200pt]{2.409pt}{0.400pt}}
\put(1429.0,299.0){\rule[-0.200pt]{2.409pt}{0.400pt}}
\put(170.0,299.0){\rule[-0.200pt]{2.409pt}{0.400pt}}
\put(1429.0,299.0){\rule[-0.200pt]{2.409pt}{0.400pt}}
\put(170.0,299.0){\rule[-0.200pt]{2.409pt}{0.400pt}}
\put(1429.0,299.0){\rule[-0.200pt]{2.409pt}{0.400pt}}
\put(170.0,300.0){\rule[-0.200pt]{2.409pt}{0.400pt}}
\put(1429.0,300.0){\rule[-0.200pt]{2.409pt}{0.400pt}}
\put(170.0,300.0){\rule[-0.200pt]{2.409pt}{0.400pt}}
\put(1429.0,300.0){\rule[-0.200pt]{2.409pt}{0.400pt}}
\put(170.0,300.0){\rule[-0.200pt]{2.409pt}{0.400pt}}
\put(1429.0,300.0){\rule[-0.200pt]{2.409pt}{0.400pt}}
\put(170.0,300.0){\rule[-0.200pt]{2.409pt}{0.400pt}}
\put(1429.0,300.0){\rule[-0.200pt]{2.409pt}{0.400pt}}
\put(170.0,300.0){\rule[-0.200pt]{2.409pt}{0.400pt}}
\put(1429.0,300.0){\rule[-0.200pt]{2.409pt}{0.400pt}}
\put(170.0,300.0){\rule[-0.200pt]{2.409pt}{0.400pt}}
\put(1429.0,300.0){\rule[-0.200pt]{2.409pt}{0.400pt}}
\put(170.0,300.0){\rule[-0.200pt]{2.409pt}{0.400pt}}
\put(1429.0,300.0){\rule[-0.200pt]{2.409pt}{0.400pt}}
\put(170.0,300.0){\rule[-0.200pt]{2.409pt}{0.400pt}}
\put(1429.0,300.0){\rule[-0.200pt]{2.409pt}{0.400pt}}
\put(170.0,300.0){\rule[-0.200pt]{2.409pt}{0.400pt}}
\put(1429.0,300.0){\rule[-0.200pt]{2.409pt}{0.400pt}}
\put(170.0,301.0){\rule[-0.200pt]{2.409pt}{0.400pt}}
\put(1429.0,301.0){\rule[-0.200pt]{2.409pt}{0.400pt}}
\put(170.0,301.0){\rule[-0.200pt]{2.409pt}{0.400pt}}
\put(1429.0,301.0){\rule[-0.200pt]{2.409pt}{0.400pt}}
\put(170.0,301.0){\rule[-0.200pt]{2.409pt}{0.400pt}}
\put(1429.0,301.0){\rule[-0.200pt]{2.409pt}{0.400pt}}
\put(170.0,301.0){\rule[-0.200pt]{2.409pt}{0.400pt}}
\put(1429.0,301.0){\rule[-0.200pt]{2.409pt}{0.400pt}}
\put(170.0,301.0){\rule[-0.200pt]{2.409pt}{0.400pt}}
\put(1429.0,301.0){\rule[-0.200pt]{2.409pt}{0.400pt}}
\put(170.0,301.0){\rule[-0.200pt]{2.409pt}{0.400pt}}
\put(1429.0,301.0){\rule[-0.200pt]{2.409pt}{0.400pt}}
\put(170.0,301.0){\rule[-0.200pt]{2.409pt}{0.400pt}}
\put(1429.0,301.0){\rule[-0.200pt]{2.409pt}{0.400pt}}
\put(170.0,301.0){\rule[-0.200pt]{2.409pt}{0.400pt}}
\put(1429.0,301.0){\rule[-0.200pt]{2.409pt}{0.400pt}}
\put(170.0,301.0){\rule[-0.200pt]{2.409pt}{0.400pt}}
\put(1429.0,301.0){\rule[-0.200pt]{2.409pt}{0.400pt}}
\put(170.0,302.0){\rule[-0.200pt]{2.409pt}{0.400pt}}
\put(1429.0,302.0){\rule[-0.200pt]{2.409pt}{0.400pt}}
\put(170.0,302.0){\rule[-0.200pt]{2.409pt}{0.400pt}}
\put(1429.0,302.0){\rule[-0.200pt]{2.409pt}{0.400pt}}
\put(170.0,302.0){\rule[-0.200pt]{2.409pt}{0.400pt}}
\put(1429.0,302.0){\rule[-0.200pt]{2.409pt}{0.400pt}}
\put(170.0,302.0){\rule[-0.200pt]{2.409pt}{0.400pt}}
\put(1429.0,302.0){\rule[-0.200pt]{2.409pt}{0.400pt}}
\put(170.0,302.0){\rule[-0.200pt]{2.409pt}{0.400pt}}
\put(1429.0,302.0){\rule[-0.200pt]{2.409pt}{0.400pt}}
\put(170.0,302.0){\rule[-0.200pt]{2.409pt}{0.400pt}}
\put(1429.0,302.0){\rule[-0.200pt]{2.409pt}{0.400pt}}
\put(170.0,302.0){\rule[-0.200pt]{2.409pt}{0.400pt}}
\put(1429.0,302.0){\rule[-0.200pt]{2.409pt}{0.400pt}}
\put(170.0,302.0){\rule[-0.200pt]{2.409pt}{0.400pt}}
\put(1429.0,302.0){\rule[-0.200pt]{2.409pt}{0.400pt}}
\put(170.0,302.0){\rule[-0.200pt]{2.409pt}{0.400pt}}
\put(1429.0,302.0){\rule[-0.200pt]{2.409pt}{0.400pt}}
\put(170.0,302.0){\rule[-0.200pt]{2.409pt}{0.400pt}}
\put(1429.0,302.0){\rule[-0.200pt]{2.409pt}{0.400pt}}
\put(170.0,303.0){\rule[-0.200pt]{2.409pt}{0.400pt}}
\put(1429.0,303.0){\rule[-0.200pt]{2.409pt}{0.400pt}}
\put(170.0,303.0){\rule[-0.200pt]{2.409pt}{0.400pt}}
\put(1429.0,303.0){\rule[-0.200pt]{2.409pt}{0.400pt}}
\put(170.0,303.0){\rule[-0.200pt]{2.409pt}{0.400pt}}
\put(1429.0,303.0){\rule[-0.200pt]{2.409pt}{0.400pt}}
\put(170.0,303.0){\rule[-0.200pt]{2.409pt}{0.400pt}}
\put(1429.0,303.0){\rule[-0.200pt]{2.409pt}{0.400pt}}
\put(170.0,303.0){\rule[-0.200pt]{2.409pt}{0.400pt}}
\put(1429.0,303.0){\rule[-0.200pt]{2.409pt}{0.400pt}}
\put(170.0,303.0){\rule[-0.200pt]{2.409pt}{0.400pt}}
\put(1429.0,303.0){\rule[-0.200pt]{2.409pt}{0.400pt}}
\put(170.0,303.0){\rule[-0.200pt]{2.409pt}{0.400pt}}
\put(1429.0,303.0){\rule[-0.200pt]{2.409pt}{0.400pt}}
\put(170.0,303.0){\rule[-0.200pt]{2.409pt}{0.400pt}}
\put(1429.0,303.0){\rule[-0.200pt]{2.409pt}{0.400pt}}
\put(170.0,303.0){\rule[-0.200pt]{2.409pt}{0.400pt}}
\put(1429.0,303.0){\rule[-0.200pt]{2.409pt}{0.400pt}}
\put(170.0,304.0){\rule[-0.200pt]{2.409pt}{0.400pt}}
\put(1429.0,304.0){\rule[-0.200pt]{2.409pt}{0.400pt}}
\put(170.0,304.0){\rule[-0.200pt]{2.409pt}{0.400pt}}
\put(1429.0,304.0){\rule[-0.200pt]{2.409pt}{0.400pt}}
\put(170.0,304.0){\rule[-0.200pt]{2.409pt}{0.400pt}}
\put(1429.0,304.0){\rule[-0.200pt]{2.409pt}{0.400pt}}
\put(170.0,304.0){\rule[-0.200pt]{2.409pt}{0.400pt}}
\put(1429.0,304.0){\rule[-0.200pt]{2.409pt}{0.400pt}}
\put(170.0,304.0){\rule[-0.200pt]{2.409pt}{0.400pt}}
\put(1429.0,304.0){\rule[-0.200pt]{2.409pt}{0.400pt}}
\put(170.0,304.0){\rule[-0.200pt]{2.409pt}{0.400pt}}
\put(1429.0,304.0){\rule[-0.200pt]{2.409pt}{0.400pt}}
\put(170.0,304.0){\rule[-0.200pt]{2.409pt}{0.400pt}}
\put(1429.0,304.0){\rule[-0.200pt]{2.409pt}{0.400pt}}
\put(170.0,304.0){\rule[-0.200pt]{2.409pt}{0.400pt}}
\put(1429.0,304.0){\rule[-0.200pt]{2.409pt}{0.400pt}}
\put(170.0,304.0){\rule[-0.200pt]{2.409pt}{0.400pt}}
\put(1429.0,304.0){\rule[-0.200pt]{2.409pt}{0.400pt}}
\put(170.0,304.0){\rule[-0.200pt]{2.409pt}{0.400pt}}
\put(1429.0,304.0){\rule[-0.200pt]{2.409pt}{0.400pt}}
\put(170.0,305.0){\rule[-0.200pt]{2.409pt}{0.400pt}}
\put(1429.0,305.0){\rule[-0.200pt]{2.409pt}{0.400pt}}
\put(170.0,305.0){\rule[-0.200pt]{2.409pt}{0.400pt}}
\put(1429.0,305.0){\rule[-0.200pt]{2.409pt}{0.400pt}}
\put(170.0,305.0){\rule[-0.200pt]{2.409pt}{0.400pt}}
\put(1429.0,305.0){\rule[-0.200pt]{2.409pt}{0.400pt}}
\put(170.0,305.0){\rule[-0.200pt]{2.409pt}{0.400pt}}
\put(1429.0,305.0){\rule[-0.200pt]{2.409pt}{0.400pt}}
\put(170.0,305.0){\rule[-0.200pt]{2.409pt}{0.400pt}}
\put(1429.0,305.0){\rule[-0.200pt]{2.409pt}{0.400pt}}
\put(170.0,305.0){\rule[-0.200pt]{2.409pt}{0.400pt}}
\put(1429.0,305.0){\rule[-0.200pt]{2.409pt}{0.400pt}}
\put(170.0,305.0){\rule[-0.200pt]{2.409pt}{0.400pt}}
\put(1429.0,305.0){\rule[-0.200pt]{2.409pt}{0.400pt}}
\put(170.0,305.0){\rule[-0.200pt]{2.409pt}{0.400pt}}
\put(1429.0,305.0){\rule[-0.200pt]{2.409pt}{0.400pt}}
\put(170.0,305.0){\rule[-0.200pt]{2.409pt}{0.400pt}}
\put(1429.0,305.0){\rule[-0.200pt]{2.409pt}{0.400pt}}
\put(170.0,305.0){\rule[-0.200pt]{2.409pt}{0.400pt}}
\put(1429.0,305.0){\rule[-0.200pt]{2.409pt}{0.400pt}}
\put(170.0,306.0){\rule[-0.200pt]{2.409pt}{0.400pt}}
\put(1429.0,306.0){\rule[-0.200pt]{2.409pt}{0.400pt}}
\put(170.0,306.0){\rule[-0.200pt]{2.409pt}{0.400pt}}
\put(1429.0,306.0){\rule[-0.200pt]{2.409pt}{0.400pt}}
\put(170.0,306.0){\rule[-0.200pt]{2.409pt}{0.400pt}}
\put(1429.0,306.0){\rule[-0.200pt]{2.409pt}{0.400pt}}
\put(170.0,306.0){\rule[-0.200pt]{2.409pt}{0.400pt}}
\put(1429.0,306.0){\rule[-0.200pt]{2.409pt}{0.400pt}}
\put(170.0,306.0){\rule[-0.200pt]{2.409pt}{0.400pt}}
\put(1429.0,306.0){\rule[-0.200pt]{2.409pt}{0.400pt}}
\put(170.0,306.0){\rule[-0.200pt]{2.409pt}{0.400pt}}
\put(1429.0,306.0){\rule[-0.200pt]{2.409pt}{0.400pt}}
\put(170.0,306.0){\rule[-0.200pt]{2.409pt}{0.400pt}}
\put(1429.0,306.0){\rule[-0.200pt]{2.409pt}{0.400pt}}
\put(170.0,306.0){\rule[-0.200pt]{2.409pt}{0.400pt}}
\put(1429.0,306.0){\rule[-0.200pt]{2.409pt}{0.400pt}}
\put(170.0,306.0){\rule[-0.200pt]{2.409pt}{0.400pt}}
\put(1429.0,306.0){\rule[-0.200pt]{2.409pt}{0.400pt}}
\put(170.0,306.0){\rule[-0.200pt]{2.409pt}{0.400pt}}
\put(1429.0,306.0){\rule[-0.200pt]{2.409pt}{0.400pt}}
\put(170.0,306.0){\rule[-0.200pt]{2.409pt}{0.400pt}}
\put(1429.0,306.0){\rule[-0.200pt]{2.409pt}{0.400pt}}
\put(170.0,307.0){\rule[-0.200pt]{2.409pt}{0.400pt}}
\put(1429.0,307.0){\rule[-0.200pt]{2.409pt}{0.400pt}}
\put(170.0,307.0){\rule[-0.200pt]{2.409pt}{0.400pt}}
\put(1429.0,307.0){\rule[-0.200pt]{2.409pt}{0.400pt}}
\put(170.0,307.0){\rule[-0.200pt]{2.409pt}{0.400pt}}
\put(1429.0,307.0){\rule[-0.200pt]{2.409pt}{0.400pt}}
\put(170.0,307.0){\rule[-0.200pt]{2.409pt}{0.400pt}}
\put(1429.0,307.0){\rule[-0.200pt]{2.409pt}{0.400pt}}
\put(170.0,307.0){\rule[-0.200pt]{2.409pt}{0.400pt}}
\put(1429.0,307.0){\rule[-0.200pt]{2.409pt}{0.400pt}}
\put(170.0,307.0){\rule[-0.200pt]{2.409pt}{0.400pt}}
\put(1429.0,307.0){\rule[-0.200pt]{2.409pt}{0.400pt}}
\put(170.0,307.0){\rule[-0.200pt]{2.409pt}{0.400pt}}
\put(1429.0,307.0){\rule[-0.200pt]{2.409pt}{0.400pt}}
\put(170.0,307.0){\rule[-0.200pt]{2.409pt}{0.400pt}}
\put(1429.0,307.0){\rule[-0.200pt]{2.409pt}{0.400pt}}
\put(170.0,307.0){\rule[-0.200pt]{2.409pt}{0.400pt}}
\put(1429.0,307.0){\rule[-0.200pt]{2.409pt}{0.400pt}}
\put(170.0,307.0){\rule[-0.200pt]{2.409pt}{0.400pt}}
\put(1429.0,307.0){\rule[-0.200pt]{2.409pt}{0.400pt}}
\put(170.0,307.0){\rule[-0.200pt]{2.409pt}{0.400pt}}
\put(1429.0,307.0){\rule[-0.200pt]{2.409pt}{0.400pt}}
\put(170.0,308.0){\rule[-0.200pt]{2.409pt}{0.400pt}}
\put(1429.0,308.0){\rule[-0.200pt]{2.409pt}{0.400pt}}
\put(170.0,308.0){\rule[-0.200pt]{2.409pt}{0.400pt}}
\put(1429.0,308.0){\rule[-0.200pt]{2.409pt}{0.400pt}}
\put(170.0,308.0){\rule[-0.200pt]{2.409pt}{0.400pt}}
\put(1429.0,308.0){\rule[-0.200pt]{2.409pt}{0.400pt}}
\put(170.0,308.0){\rule[-0.200pt]{2.409pt}{0.400pt}}
\put(1429.0,308.0){\rule[-0.200pt]{2.409pt}{0.400pt}}
\put(170.0,308.0){\rule[-0.200pt]{2.409pt}{0.400pt}}
\put(1429.0,308.0){\rule[-0.200pt]{2.409pt}{0.400pt}}
\put(170.0,308.0){\rule[-0.200pt]{2.409pt}{0.400pt}}
\put(1429.0,308.0){\rule[-0.200pt]{2.409pt}{0.400pt}}
\put(170.0,308.0){\rule[-0.200pt]{2.409pt}{0.400pt}}
\put(1429.0,308.0){\rule[-0.200pt]{2.409pt}{0.400pt}}
\put(170.0,308.0){\rule[-0.200pt]{2.409pt}{0.400pt}}
\put(1429.0,308.0){\rule[-0.200pt]{2.409pt}{0.400pt}}
\put(170.0,308.0){\rule[-0.200pt]{2.409pt}{0.400pt}}
\put(1429.0,308.0){\rule[-0.200pt]{2.409pt}{0.400pt}}
\put(170.0,308.0){\rule[-0.200pt]{2.409pt}{0.400pt}}
\put(1429.0,308.0){\rule[-0.200pt]{2.409pt}{0.400pt}}
\put(170.0,308.0){\rule[-0.200pt]{2.409pt}{0.400pt}}
\put(1429.0,308.0){\rule[-0.200pt]{2.409pt}{0.400pt}}
\put(170.0,309.0){\rule[-0.200pt]{2.409pt}{0.400pt}}
\put(1429.0,309.0){\rule[-0.200pt]{2.409pt}{0.400pt}}
\put(170.0,309.0){\rule[-0.200pt]{2.409pt}{0.400pt}}
\put(1429.0,309.0){\rule[-0.200pt]{2.409pt}{0.400pt}}
\put(170.0,309.0){\rule[-0.200pt]{2.409pt}{0.400pt}}
\put(1429.0,309.0){\rule[-0.200pt]{2.409pt}{0.400pt}}
\put(170.0,309.0){\rule[-0.200pt]{2.409pt}{0.400pt}}
\put(1429.0,309.0){\rule[-0.200pt]{2.409pt}{0.400pt}}
\put(170.0,309.0){\rule[-0.200pt]{2.409pt}{0.400pt}}
\put(1429.0,309.0){\rule[-0.200pt]{2.409pt}{0.400pt}}
\put(170.0,309.0){\rule[-0.200pt]{2.409pt}{0.400pt}}
\put(1429.0,309.0){\rule[-0.200pt]{2.409pt}{0.400pt}}
\put(170.0,309.0){\rule[-0.200pt]{2.409pt}{0.400pt}}
\put(1429.0,309.0){\rule[-0.200pt]{2.409pt}{0.400pt}}
\put(170.0,309.0){\rule[-0.200pt]{2.409pt}{0.400pt}}
\put(1429.0,309.0){\rule[-0.200pt]{2.409pt}{0.400pt}}
\put(170.0,309.0){\rule[-0.200pt]{2.409pt}{0.400pt}}
\put(1429.0,309.0){\rule[-0.200pt]{2.409pt}{0.400pt}}
\put(170.0,309.0){\rule[-0.200pt]{2.409pt}{0.400pt}}
\put(1429.0,309.0){\rule[-0.200pt]{2.409pt}{0.400pt}}
\put(170.0,309.0){\rule[-0.200pt]{2.409pt}{0.400pt}}
\put(1429.0,309.0){\rule[-0.200pt]{2.409pt}{0.400pt}}
\put(170.0,310.0){\rule[-0.200pt]{2.409pt}{0.400pt}}
\put(1429.0,310.0){\rule[-0.200pt]{2.409pt}{0.400pt}}
\put(170.0,310.0){\rule[-0.200pt]{2.409pt}{0.400pt}}
\put(1429.0,310.0){\rule[-0.200pt]{2.409pt}{0.400pt}}
\put(170.0,310.0){\rule[-0.200pt]{2.409pt}{0.400pt}}
\put(1429.0,310.0){\rule[-0.200pt]{2.409pt}{0.400pt}}
\put(170.0,310.0){\rule[-0.200pt]{2.409pt}{0.400pt}}
\put(1429.0,310.0){\rule[-0.200pt]{2.409pt}{0.400pt}}
\put(170.0,310.0){\rule[-0.200pt]{2.409pt}{0.400pt}}
\put(1429.0,310.0){\rule[-0.200pt]{2.409pt}{0.400pt}}
\put(170.0,310.0){\rule[-0.200pt]{2.409pt}{0.400pt}}
\put(1429.0,310.0){\rule[-0.200pt]{2.409pt}{0.400pt}}
\put(170.0,310.0){\rule[-0.200pt]{2.409pt}{0.400pt}}
\put(1429.0,310.0){\rule[-0.200pt]{2.409pt}{0.400pt}}
\put(170.0,310.0){\rule[-0.200pt]{2.409pt}{0.400pt}}
\put(1429.0,310.0){\rule[-0.200pt]{2.409pt}{0.400pt}}
\put(170.0,310.0){\rule[-0.200pt]{2.409pt}{0.400pt}}
\put(1429.0,310.0){\rule[-0.200pt]{2.409pt}{0.400pt}}
\put(170.0,310.0){\rule[-0.200pt]{2.409pt}{0.400pt}}
\put(1429.0,310.0){\rule[-0.200pt]{2.409pt}{0.400pt}}
\put(170.0,310.0){\rule[-0.200pt]{2.409pt}{0.400pt}}
\put(1429.0,310.0){\rule[-0.200pt]{2.409pt}{0.400pt}}
\put(170.0,310.0){\rule[-0.200pt]{2.409pt}{0.400pt}}
\put(1429.0,310.0){\rule[-0.200pt]{2.409pt}{0.400pt}}
\put(170.0,311.0){\rule[-0.200pt]{2.409pt}{0.400pt}}
\put(1429.0,311.0){\rule[-0.200pt]{2.409pt}{0.400pt}}
\put(170.0,311.0){\rule[-0.200pt]{2.409pt}{0.400pt}}
\put(1429.0,311.0){\rule[-0.200pt]{2.409pt}{0.400pt}}
\put(170.0,311.0){\rule[-0.200pt]{2.409pt}{0.400pt}}
\put(1429.0,311.0){\rule[-0.200pt]{2.409pt}{0.400pt}}
\put(170.0,311.0){\rule[-0.200pt]{2.409pt}{0.400pt}}
\put(1429.0,311.0){\rule[-0.200pt]{2.409pt}{0.400pt}}
\put(170.0,311.0){\rule[-0.200pt]{2.409pt}{0.400pt}}
\put(1429.0,311.0){\rule[-0.200pt]{2.409pt}{0.400pt}}
\put(170.0,311.0){\rule[-0.200pt]{2.409pt}{0.400pt}}
\put(1429.0,311.0){\rule[-0.200pt]{2.409pt}{0.400pt}}
\put(170.0,311.0){\rule[-0.200pt]{2.409pt}{0.400pt}}
\put(1429.0,311.0){\rule[-0.200pt]{2.409pt}{0.400pt}}
\put(170.0,311.0){\rule[-0.200pt]{2.409pt}{0.400pt}}
\put(1429.0,311.0){\rule[-0.200pt]{2.409pt}{0.400pt}}
\put(170.0,311.0){\rule[-0.200pt]{2.409pt}{0.400pt}}
\put(1429.0,311.0){\rule[-0.200pt]{2.409pt}{0.400pt}}
\put(170.0,311.0){\rule[-0.200pt]{2.409pt}{0.400pt}}
\put(1429.0,311.0){\rule[-0.200pt]{2.409pt}{0.400pt}}
\put(170.0,311.0){\rule[-0.200pt]{2.409pt}{0.400pt}}
\put(1429.0,311.0){\rule[-0.200pt]{2.409pt}{0.400pt}}
\put(170.0,311.0){\rule[-0.200pt]{2.409pt}{0.400pt}}
\put(1429.0,311.0){\rule[-0.200pt]{2.409pt}{0.400pt}}
\put(170.0,312.0){\rule[-0.200pt]{2.409pt}{0.400pt}}
\put(1429.0,312.0){\rule[-0.200pt]{2.409pt}{0.400pt}}
\put(170.0,312.0){\rule[-0.200pt]{2.409pt}{0.400pt}}
\put(1429.0,312.0){\rule[-0.200pt]{2.409pt}{0.400pt}}
\put(170.0,312.0){\rule[-0.200pt]{2.409pt}{0.400pt}}
\put(1429.0,312.0){\rule[-0.200pt]{2.409pt}{0.400pt}}
\put(170.0,312.0){\rule[-0.200pt]{2.409pt}{0.400pt}}
\put(1429.0,312.0){\rule[-0.200pt]{2.409pt}{0.400pt}}
\put(170.0,312.0){\rule[-0.200pt]{2.409pt}{0.400pt}}
\put(1429.0,312.0){\rule[-0.200pt]{2.409pt}{0.400pt}}
\put(170.0,312.0){\rule[-0.200pt]{2.409pt}{0.400pt}}
\put(1429.0,312.0){\rule[-0.200pt]{2.409pt}{0.400pt}}
\put(170.0,312.0){\rule[-0.200pt]{2.409pt}{0.400pt}}
\put(1429.0,312.0){\rule[-0.200pt]{2.409pt}{0.400pt}}
\put(170.0,312.0){\rule[-0.200pt]{2.409pt}{0.400pt}}
\put(1429.0,312.0){\rule[-0.200pt]{2.409pt}{0.400pt}}
\put(170.0,312.0){\rule[-0.200pt]{2.409pt}{0.400pt}}
\put(1429.0,312.0){\rule[-0.200pt]{2.409pt}{0.400pt}}
\put(170.0,312.0){\rule[-0.200pt]{2.409pt}{0.400pt}}
\put(1429.0,312.0){\rule[-0.200pt]{2.409pt}{0.400pt}}
\put(170.0,312.0){\rule[-0.200pt]{2.409pt}{0.400pt}}
\put(1429.0,312.0){\rule[-0.200pt]{2.409pt}{0.400pt}}
\put(170.0,312.0){\rule[-0.200pt]{2.409pt}{0.400pt}}
\put(1429.0,312.0){\rule[-0.200pt]{2.409pt}{0.400pt}}
\put(170.0,313.0){\rule[-0.200pt]{2.409pt}{0.400pt}}
\put(1429.0,313.0){\rule[-0.200pt]{2.409pt}{0.400pt}}
\put(170.0,313.0){\rule[-0.200pt]{2.409pt}{0.400pt}}
\put(1429.0,313.0){\rule[-0.200pt]{2.409pt}{0.400pt}}
\put(170.0,313.0){\rule[-0.200pt]{2.409pt}{0.400pt}}
\put(1429.0,313.0){\rule[-0.200pt]{2.409pt}{0.400pt}}
\put(170.0,313.0){\rule[-0.200pt]{2.409pt}{0.400pt}}
\put(1429.0,313.0){\rule[-0.200pt]{2.409pt}{0.400pt}}
\put(170.0,313.0){\rule[-0.200pt]{2.409pt}{0.400pt}}
\put(1429.0,313.0){\rule[-0.200pt]{2.409pt}{0.400pt}}
\put(170.0,313.0){\rule[-0.200pt]{2.409pt}{0.400pt}}
\put(1429.0,313.0){\rule[-0.200pt]{2.409pt}{0.400pt}}
\put(170.0,313.0){\rule[-0.200pt]{2.409pt}{0.400pt}}
\put(1429.0,313.0){\rule[-0.200pt]{2.409pt}{0.400pt}}
\put(170.0,313.0){\rule[-0.200pt]{2.409pt}{0.400pt}}
\put(1429.0,313.0){\rule[-0.200pt]{2.409pt}{0.400pt}}
\put(170.0,313.0){\rule[-0.200pt]{2.409pt}{0.400pt}}
\put(1429.0,313.0){\rule[-0.200pt]{2.409pt}{0.400pt}}
\put(170.0,313.0){\rule[-0.200pt]{2.409pt}{0.400pt}}
\put(1429.0,313.0){\rule[-0.200pt]{2.409pt}{0.400pt}}
\put(170.0,313.0){\rule[-0.200pt]{2.409pt}{0.400pt}}
\put(1429.0,313.0){\rule[-0.200pt]{2.409pt}{0.400pt}}
\put(170.0,313.0){\rule[-0.200pt]{2.409pt}{0.400pt}}
\put(1429.0,313.0){\rule[-0.200pt]{2.409pt}{0.400pt}}
\put(170.0,313.0){\rule[-0.200pt]{2.409pt}{0.400pt}}
\put(1429.0,313.0){\rule[-0.200pt]{2.409pt}{0.400pt}}
\put(170.0,314.0){\rule[-0.200pt]{2.409pt}{0.400pt}}
\put(1429.0,314.0){\rule[-0.200pt]{2.409pt}{0.400pt}}
\put(170.0,314.0){\rule[-0.200pt]{2.409pt}{0.400pt}}
\put(1429.0,314.0){\rule[-0.200pt]{2.409pt}{0.400pt}}
\put(170.0,314.0){\rule[-0.200pt]{2.409pt}{0.400pt}}
\put(1429.0,314.0){\rule[-0.200pt]{2.409pt}{0.400pt}}
\put(170.0,314.0){\rule[-0.200pt]{2.409pt}{0.400pt}}
\put(1429.0,314.0){\rule[-0.200pt]{2.409pt}{0.400pt}}
\put(170.0,314.0){\rule[-0.200pt]{2.409pt}{0.400pt}}
\put(1429.0,314.0){\rule[-0.200pt]{2.409pt}{0.400pt}}
\put(170.0,314.0){\rule[-0.200pt]{2.409pt}{0.400pt}}
\put(1429.0,314.0){\rule[-0.200pt]{2.409pt}{0.400pt}}
\put(170.0,314.0){\rule[-0.200pt]{2.409pt}{0.400pt}}
\put(1429.0,314.0){\rule[-0.200pt]{2.409pt}{0.400pt}}
\put(170.0,314.0){\rule[-0.200pt]{2.409pt}{0.400pt}}
\put(1429.0,314.0){\rule[-0.200pt]{2.409pt}{0.400pt}}
\put(170.0,314.0){\rule[-0.200pt]{2.409pt}{0.400pt}}
\put(1429.0,314.0){\rule[-0.200pt]{2.409pt}{0.400pt}}
\put(170.0,314.0){\rule[-0.200pt]{2.409pt}{0.400pt}}
\put(1429.0,314.0){\rule[-0.200pt]{2.409pt}{0.400pt}}
\put(170.0,314.0){\rule[-0.200pt]{2.409pt}{0.400pt}}
\put(1429.0,314.0){\rule[-0.200pt]{2.409pt}{0.400pt}}
\put(170.0,314.0){\rule[-0.200pt]{2.409pt}{0.400pt}}
\put(1429.0,314.0){\rule[-0.200pt]{2.409pt}{0.400pt}}
\put(170.0,314.0){\rule[-0.200pt]{2.409pt}{0.400pt}}
\put(1429.0,314.0){\rule[-0.200pt]{2.409pt}{0.400pt}}
\put(170.0,315.0){\rule[-0.200pt]{2.409pt}{0.400pt}}
\put(1429.0,315.0){\rule[-0.200pt]{2.409pt}{0.400pt}}
\put(170.0,315.0){\rule[-0.200pt]{2.409pt}{0.400pt}}
\put(1429.0,315.0){\rule[-0.200pt]{2.409pt}{0.400pt}}
\put(170.0,315.0){\rule[-0.200pt]{2.409pt}{0.400pt}}
\put(1429.0,315.0){\rule[-0.200pt]{2.409pt}{0.400pt}}
\put(170.0,315.0){\rule[-0.200pt]{2.409pt}{0.400pt}}
\put(1429.0,315.0){\rule[-0.200pt]{2.409pt}{0.400pt}}
\put(170.0,315.0){\rule[-0.200pt]{2.409pt}{0.400pt}}
\put(1429.0,315.0){\rule[-0.200pt]{2.409pt}{0.400pt}}
\put(170.0,315.0){\rule[-0.200pt]{2.409pt}{0.400pt}}
\put(1429.0,315.0){\rule[-0.200pt]{2.409pt}{0.400pt}}
\put(170.0,315.0){\rule[-0.200pt]{2.409pt}{0.400pt}}
\put(1429.0,315.0){\rule[-0.200pt]{2.409pt}{0.400pt}}
\put(170.0,315.0){\rule[-0.200pt]{2.409pt}{0.400pt}}
\put(1429.0,315.0){\rule[-0.200pt]{2.409pt}{0.400pt}}
\put(170.0,315.0){\rule[-0.200pt]{2.409pt}{0.400pt}}
\put(1429.0,315.0){\rule[-0.200pt]{2.409pt}{0.400pt}}
\put(170.0,315.0){\rule[-0.200pt]{2.409pt}{0.400pt}}
\put(1429.0,315.0){\rule[-0.200pt]{2.409pt}{0.400pt}}
\put(170.0,315.0){\rule[-0.200pt]{2.409pt}{0.400pt}}
\put(1429.0,315.0){\rule[-0.200pt]{2.409pt}{0.400pt}}
\put(170.0,315.0){\rule[-0.200pt]{2.409pt}{0.400pt}}
\put(1429.0,315.0){\rule[-0.200pt]{2.409pt}{0.400pt}}
\put(170.0,315.0){\rule[-0.200pt]{2.409pt}{0.400pt}}
\put(1429.0,315.0){\rule[-0.200pt]{2.409pt}{0.400pt}}
\put(170.0,316.0){\rule[-0.200pt]{2.409pt}{0.400pt}}
\put(1429.0,316.0){\rule[-0.200pt]{2.409pt}{0.400pt}}
\put(170.0,316.0){\rule[-0.200pt]{2.409pt}{0.400pt}}
\put(1429.0,316.0){\rule[-0.200pt]{2.409pt}{0.400pt}}
\put(170.0,316.0){\rule[-0.200pt]{2.409pt}{0.400pt}}
\put(1429.0,316.0){\rule[-0.200pt]{2.409pt}{0.400pt}}
\put(170.0,316.0){\rule[-0.200pt]{2.409pt}{0.400pt}}
\put(1429.0,316.0){\rule[-0.200pt]{2.409pt}{0.400pt}}
\put(170.0,316.0){\rule[-0.200pt]{2.409pt}{0.400pt}}
\put(1429.0,316.0){\rule[-0.200pt]{2.409pt}{0.400pt}}
\put(170.0,316.0){\rule[-0.200pt]{2.409pt}{0.400pt}}
\put(1429.0,316.0){\rule[-0.200pt]{2.409pt}{0.400pt}}
\put(170.0,316.0){\rule[-0.200pt]{2.409pt}{0.400pt}}
\put(1429.0,316.0){\rule[-0.200pt]{2.409pt}{0.400pt}}
\put(170.0,316.0){\rule[-0.200pt]{2.409pt}{0.400pt}}
\put(1429.0,316.0){\rule[-0.200pt]{2.409pt}{0.400pt}}
\put(170.0,316.0){\rule[-0.200pt]{2.409pt}{0.400pt}}
\put(1429.0,316.0){\rule[-0.200pt]{2.409pt}{0.400pt}}
\put(170.0,316.0){\rule[-0.200pt]{2.409pt}{0.400pt}}
\put(1429.0,316.0){\rule[-0.200pt]{2.409pt}{0.400pt}}
\put(170.0,316.0){\rule[-0.200pt]{2.409pt}{0.400pt}}
\put(1429.0,316.0){\rule[-0.200pt]{2.409pt}{0.400pt}}
\put(170.0,316.0){\rule[-0.200pt]{2.409pt}{0.400pt}}
\put(1429.0,316.0){\rule[-0.200pt]{2.409pt}{0.400pt}}
\put(170.0,316.0){\rule[-0.200pt]{2.409pt}{0.400pt}}
\put(1429.0,316.0){\rule[-0.200pt]{2.409pt}{0.400pt}}
\put(170.0,316.0){\rule[-0.200pt]{2.409pt}{0.400pt}}
\put(1429.0,316.0){\rule[-0.200pt]{2.409pt}{0.400pt}}
\put(170.0,317.0){\rule[-0.200pt]{2.409pt}{0.400pt}}
\put(1429.0,317.0){\rule[-0.200pt]{2.409pt}{0.400pt}}
\put(170.0,317.0){\rule[-0.200pt]{2.409pt}{0.400pt}}
\put(1429.0,317.0){\rule[-0.200pt]{2.409pt}{0.400pt}}
\put(170.0,317.0){\rule[-0.200pt]{2.409pt}{0.400pt}}
\put(1429.0,317.0){\rule[-0.200pt]{2.409pt}{0.400pt}}
\put(170.0,317.0){\rule[-0.200pt]{2.409pt}{0.400pt}}
\put(1429.0,317.0){\rule[-0.200pt]{2.409pt}{0.400pt}}
\put(170.0,317.0){\rule[-0.200pt]{2.409pt}{0.400pt}}
\put(1429.0,317.0){\rule[-0.200pt]{2.409pt}{0.400pt}}
\put(170.0,317.0){\rule[-0.200pt]{2.409pt}{0.400pt}}
\put(1429.0,317.0){\rule[-0.200pt]{2.409pt}{0.400pt}}
\put(170.0,317.0){\rule[-0.200pt]{2.409pt}{0.400pt}}
\put(1429.0,317.0){\rule[-0.200pt]{2.409pt}{0.400pt}}
\put(170.0,317.0){\rule[-0.200pt]{2.409pt}{0.400pt}}
\put(1429.0,317.0){\rule[-0.200pt]{2.409pt}{0.400pt}}
\put(170.0,317.0){\rule[-0.200pt]{2.409pt}{0.400pt}}
\put(1429.0,317.0){\rule[-0.200pt]{2.409pt}{0.400pt}}
\put(170.0,317.0){\rule[-0.200pt]{2.409pt}{0.400pt}}
\put(1429.0,317.0){\rule[-0.200pt]{2.409pt}{0.400pt}}
\put(170.0,317.0){\rule[-0.200pt]{2.409pt}{0.400pt}}
\put(1429.0,317.0){\rule[-0.200pt]{2.409pt}{0.400pt}}
\put(170.0,317.0){\rule[-0.200pt]{2.409pt}{0.400pt}}
\put(1429.0,317.0){\rule[-0.200pt]{2.409pt}{0.400pt}}
\put(170.0,317.0){\rule[-0.200pt]{2.409pt}{0.400pt}}
\put(1429.0,317.0){\rule[-0.200pt]{2.409pt}{0.400pt}}
\put(170.0,317.0){\rule[-0.200pt]{2.409pt}{0.400pt}}
\put(1429.0,317.0){\rule[-0.200pt]{2.409pt}{0.400pt}}
\put(170.0,318.0){\rule[-0.200pt]{2.409pt}{0.400pt}}
\put(1429.0,318.0){\rule[-0.200pt]{2.409pt}{0.400pt}}
\put(170.0,318.0){\rule[-0.200pt]{2.409pt}{0.400pt}}
\put(1429.0,318.0){\rule[-0.200pt]{2.409pt}{0.400pt}}
\put(170.0,318.0){\rule[-0.200pt]{2.409pt}{0.400pt}}
\put(1429.0,318.0){\rule[-0.200pt]{2.409pt}{0.400pt}}
\put(170.0,318.0){\rule[-0.200pt]{2.409pt}{0.400pt}}
\put(1429.0,318.0){\rule[-0.200pt]{2.409pt}{0.400pt}}
\put(170.0,318.0){\rule[-0.200pt]{2.409pt}{0.400pt}}
\put(1429.0,318.0){\rule[-0.200pt]{2.409pt}{0.400pt}}
\put(170.0,318.0){\rule[-0.200pt]{2.409pt}{0.400pt}}
\put(1429.0,318.0){\rule[-0.200pt]{2.409pt}{0.400pt}}
\put(170.0,318.0){\rule[-0.200pt]{2.409pt}{0.400pt}}
\put(1429.0,318.0){\rule[-0.200pt]{2.409pt}{0.400pt}}
\put(170.0,318.0){\rule[-0.200pt]{2.409pt}{0.400pt}}
\put(1429.0,318.0){\rule[-0.200pt]{2.409pt}{0.400pt}}
\put(170.0,318.0){\rule[-0.200pt]{2.409pt}{0.400pt}}
\put(1429.0,318.0){\rule[-0.200pt]{2.409pt}{0.400pt}}
\put(170.0,318.0){\rule[-0.200pt]{2.409pt}{0.400pt}}
\put(1429.0,318.0){\rule[-0.200pt]{2.409pt}{0.400pt}}
\put(170.0,318.0){\rule[-0.200pt]{2.409pt}{0.400pt}}
\put(1429.0,318.0){\rule[-0.200pt]{2.409pt}{0.400pt}}
\put(170.0,318.0){\rule[-0.200pt]{2.409pt}{0.400pt}}
\put(1429.0,318.0){\rule[-0.200pt]{2.409pt}{0.400pt}}
\put(170.0,318.0){\rule[-0.200pt]{2.409pt}{0.400pt}}
\put(1429.0,318.0){\rule[-0.200pt]{2.409pt}{0.400pt}}
\put(170.0,318.0){\rule[-0.200pt]{2.409pt}{0.400pt}}
\put(1429.0,318.0){\rule[-0.200pt]{2.409pt}{0.400pt}}
\put(170.0,319.0){\rule[-0.200pt]{2.409pt}{0.400pt}}
\put(1429.0,319.0){\rule[-0.200pt]{2.409pt}{0.400pt}}
\put(170.0,319.0){\rule[-0.200pt]{2.409pt}{0.400pt}}
\put(1429.0,319.0){\rule[-0.200pt]{2.409pt}{0.400pt}}
\put(170.0,319.0){\rule[-0.200pt]{2.409pt}{0.400pt}}
\put(1429.0,319.0){\rule[-0.200pt]{2.409pt}{0.400pt}}
\put(170.0,319.0){\rule[-0.200pt]{2.409pt}{0.400pt}}
\put(1429.0,319.0){\rule[-0.200pt]{2.409pt}{0.400pt}}
\put(170.0,319.0){\rule[-0.200pt]{2.409pt}{0.400pt}}
\put(1429.0,319.0){\rule[-0.200pt]{2.409pt}{0.400pt}}
\put(170.0,319.0){\rule[-0.200pt]{2.409pt}{0.400pt}}
\put(1429.0,319.0){\rule[-0.200pt]{2.409pt}{0.400pt}}
\put(170.0,319.0){\rule[-0.200pt]{2.409pt}{0.400pt}}
\put(1429.0,319.0){\rule[-0.200pt]{2.409pt}{0.400pt}}
\put(170.0,319.0){\rule[-0.200pt]{2.409pt}{0.400pt}}
\put(1429.0,319.0){\rule[-0.200pt]{2.409pt}{0.400pt}}
\put(170.0,319.0){\rule[-0.200pt]{2.409pt}{0.400pt}}
\put(1429.0,319.0){\rule[-0.200pt]{2.409pt}{0.400pt}}
\put(170.0,319.0){\rule[-0.200pt]{2.409pt}{0.400pt}}
\put(1429.0,319.0){\rule[-0.200pt]{2.409pt}{0.400pt}}
\put(170.0,319.0){\rule[-0.200pt]{2.409pt}{0.400pt}}
\put(1429.0,319.0){\rule[-0.200pt]{2.409pt}{0.400pt}}
\put(170.0,319.0){\rule[-0.200pt]{2.409pt}{0.400pt}}
\put(1429.0,319.0){\rule[-0.200pt]{2.409pt}{0.400pt}}
\put(170.0,319.0){\rule[-0.200pt]{2.409pt}{0.400pt}}
\put(1429.0,319.0){\rule[-0.200pt]{2.409pt}{0.400pt}}
\put(170.0,319.0){\rule[-0.200pt]{2.409pt}{0.400pt}}
\put(1429.0,319.0){\rule[-0.200pt]{2.409pt}{0.400pt}}
\put(170.0,319.0){\rule[-0.200pt]{2.409pt}{0.400pt}}
\put(1429.0,319.0){\rule[-0.200pt]{2.409pt}{0.400pt}}
\put(170.0,320.0){\rule[-0.200pt]{2.409pt}{0.400pt}}
\put(1429.0,320.0){\rule[-0.200pt]{2.409pt}{0.400pt}}
\put(170.0,320.0){\rule[-0.200pt]{2.409pt}{0.400pt}}
\put(1429.0,320.0){\rule[-0.200pt]{2.409pt}{0.400pt}}
\put(170.0,320.0){\rule[-0.200pt]{2.409pt}{0.400pt}}
\put(1429.0,320.0){\rule[-0.200pt]{2.409pt}{0.400pt}}
\put(170.0,320.0){\rule[-0.200pt]{2.409pt}{0.400pt}}
\put(1429.0,320.0){\rule[-0.200pt]{2.409pt}{0.400pt}}
\put(170.0,320.0){\rule[-0.200pt]{2.409pt}{0.400pt}}
\put(1429.0,320.0){\rule[-0.200pt]{2.409pt}{0.400pt}}
\put(170.0,320.0){\rule[-0.200pt]{2.409pt}{0.400pt}}
\put(1429.0,320.0){\rule[-0.200pt]{2.409pt}{0.400pt}}
\put(170.0,320.0){\rule[-0.200pt]{2.409pt}{0.400pt}}
\put(1429.0,320.0){\rule[-0.200pt]{2.409pt}{0.400pt}}
\put(170.0,320.0){\rule[-0.200pt]{2.409pt}{0.400pt}}
\put(1429.0,320.0){\rule[-0.200pt]{2.409pt}{0.400pt}}
\put(170.0,320.0){\rule[-0.200pt]{2.409pt}{0.400pt}}
\put(1429.0,320.0){\rule[-0.200pt]{2.409pt}{0.400pt}}
\put(170.0,320.0){\rule[-0.200pt]{2.409pt}{0.400pt}}
\put(1429.0,320.0){\rule[-0.200pt]{2.409pt}{0.400pt}}
\put(170.0,320.0){\rule[-0.200pt]{2.409pt}{0.400pt}}
\put(1429.0,320.0){\rule[-0.200pt]{2.409pt}{0.400pt}}
\put(170.0,320.0){\rule[-0.200pt]{2.409pt}{0.400pt}}
\put(1429.0,320.0){\rule[-0.200pt]{2.409pt}{0.400pt}}
\put(170.0,320.0){\rule[-0.200pt]{2.409pt}{0.400pt}}
\put(1429.0,320.0){\rule[-0.200pt]{2.409pt}{0.400pt}}
\put(170.0,320.0){\rule[-0.200pt]{2.409pt}{0.400pt}}
\put(1429.0,320.0){\rule[-0.200pt]{2.409pt}{0.400pt}}
\put(170.0,320.0){\rule[-0.200pt]{2.409pt}{0.400pt}}
\put(1429.0,320.0){\rule[-0.200pt]{2.409pt}{0.400pt}}
\put(170.0,321.0){\rule[-0.200pt]{2.409pt}{0.400pt}}
\put(1429.0,321.0){\rule[-0.200pt]{2.409pt}{0.400pt}}
\put(170.0,321.0){\rule[-0.200pt]{2.409pt}{0.400pt}}
\put(1429.0,321.0){\rule[-0.200pt]{2.409pt}{0.400pt}}
\put(170.0,321.0){\rule[-0.200pt]{2.409pt}{0.400pt}}
\put(1429.0,321.0){\rule[-0.200pt]{2.409pt}{0.400pt}}
\put(170.0,321.0){\rule[-0.200pt]{2.409pt}{0.400pt}}
\put(1429.0,321.0){\rule[-0.200pt]{2.409pt}{0.400pt}}
\put(170.0,321.0){\rule[-0.200pt]{2.409pt}{0.400pt}}
\put(1429.0,321.0){\rule[-0.200pt]{2.409pt}{0.400pt}}
\put(170.0,321.0){\rule[-0.200pt]{2.409pt}{0.400pt}}
\put(1429.0,321.0){\rule[-0.200pt]{2.409pt}{0.400pt}}
\put(170.0,321.0){\rule[-0.200pt]{2.409pt}{0.400pt}}
\put(1429.0,321.0){\rule[-0.200pt]{2.409pt}{0.400pt}}
\put(170.0,321.0){\rule[-0.200pt]{2.409pt}{0.400pt}}
\put(1429.0,321.0){\rule[-0.200pt]{2.409pt}{0.400pt}}
\put(170.0,321.0){\rule[-0.200pt]{2.409pt}{0.400pt}}
\put(1429.0,321.0){\rule[-0.200pt]{2.409pt}{0.400pt}}
\put(170.0,321.0){\rule[-0.200pt]{2.409pt}{0.400pt}}
\put(1429.0,321.0){\rule[-0.200pt]{2.409pt}{0.400pt}}
\put(170.0,321.0){\rule[-0.200pt]{2.409pt}{0.400pt}}
\put(1429.0,321.0){\rule[-0.200pt]{2.409pt}{0.400pt}}
\put(170.0,321.0){\rule[-0.200pt]{2.409pt}{0.400pt}}
\put(1429.0,321.0){\rule[-0.200pt]{2.409pt}{0.400pt}}
\put(170.0,321.0){\rule[-0.200pt]{2.409pt}{0.400pt}}
\put(1429.0,321.0){\rule[-0.200pt]{2.409pt}{0.400pt}}
\put(170.0,321.0){\rule[-0.200pt]{2.409pt}{0.400pt}}
\put(1429.0,321.0){\rule[-0.200pt]{2.409pt}{0.400pt}}
\put(170.0,321.0){\rule[-0.200pt]{2.409pt}{0.400pt}}
\put(1429.0,321.0){\rule[-0.200pt]{2.409pt}{0.400pt}}
\put(170.0,321.0){\rule[-0.200pt]{2.409pt}{0.400pt}}
\put(1429.0,321.0){\rule[-0.200pt]{2.409pt}{0.400pt}}
\put(170.0,322.0){\rule[-0.200pt]{2.409pt}{0.400pt}}
\put(1429.0,322.0){\rule[-0.200pt]{2.409pt}{0.400pt}}
\put(170.0,322.0){\rule[-0.200pt]{2.409pt}{0.400pt}}
\put(1429.0,322.0){\rule[-0.200pt]{2.409pt}{0.400pt}}
\put(170.0,322.0){\rule[-0.200pt]{2.409pt}{0.400pt}}
\put(1429.0,322.0){\rule[-0.200pt]{2.409pt}{0.400pt}}
\put(170.0,322.0){\rule[-0.200pt]{2.409pt}{0.400pt}}
\put(1429.0,322.0){\rule[-0.200pt]{2.409pt}{0.400pt}}
\put(170.0,322.0){\rule[-0.200pt]{2.409pt}{0.400pt}}
\put(1429.0,322.0){\rule[-0.200pt]{2.409pt}{0.400pt}}
\put(170.0,322.0){\rule[-0.200pt]{2.409pt}{0.400pt}}
\put(1429.0,322.0){\rule[-0.200pt]{2.409pt}{0.400pt}}
\put(170.0,322.0){\rule[-0.200pt]{2.409pt}{0.400pt}}
\put(1429.0,322.0){\rule[-0.200pt]{2.409pt}{0.400pt}}
\put(170.0,322.0){\rule[-0.200pt]{2.409pt}{0.400pt}}
\put(1429.0,322.0){\rule[-0.200pt]{2.409pt}{0.400pt}}
\put(170.0,322.0){\rule[-0.200pt]{2.409pt}{0.400pt}}
\put(1429.0,322.0){\rule[-0.200pt]{2.409pt}{0.400pt}}
\put(170.0,322.0){\rule[-0.200pt]{2.409pt}{0.400pt}}
\put(1429.0,322.0){\rule[-0.200pt]{2.409pt}{0.400pt}}
\put(170.0,322.0){\rule[-0.200pt]{2.409pt}{0.400pt}}
\put(1429.0,322.0){\rule[-0.200pt]{2.409pt}{0.400pt}}
\put(170.0,322.0){\rule[-0.200pt]{2.409pt}{0.400pt}}
\put(1429.0,322.0){\rule[-0.200pt]{2.409pt}{0.400pt}}
\put(170.0,322.0){\rule[-0.200pt]{2.409pt}{0.400pt}}
\put(1429.0,322.0){\rule[-0.200pt]{2.409pt}{0.400pt}}
\put(170.0,322.0){\rule[-0.200pt]{2.409pt}{0.400pt}}
\put(1429.0,322.0){\rule[-0.200pt]{2.409pt}{0.400pt}}
\put(170.0,322.0){\rule[-0.200pt]{2.409pt}{0.400pt}}
\put(1429.0,322.0){\rule[-0.200pt]{2.409pt}{0.400pt}}
\put(170.0,322.0){\rule[-0.200pt]{2.409pt}{0.400pt}}
\put(1429.0,322.0){\rule[-0.200pt]{2.409pt}{0.400pt}}
\put(170.0,323.0){\rule[-0.200pt]{2.409pt}{0.400pt}}
\put(1429.0,323.0){\rule[-0.200pt]{2.409pt}{0.400pt}}
\put(170.0,323.0){\rule[-0.200pt]{2.409pt}{0.400pt}}
\put(1429.0,323.0){\rule[-0.200pt]{2.409pt}{0.400pt}}
\put(170.0,323.0){\rule[-0.200pt]{2.409pt}{0.400pt}}
\put(1429.0,323.0){\rule[-0.200pt]{2.409pt}{0.400pt}}
\put(170.0,323.0){\rule[-0.200pt]{2.409pt}{0.400pt}}
\put(1429.0,323.0){\rule[-0.200pt]{2.409pt}{0.400pt}}
\put(170.0,323.0){\rule[-0.200pt]{2.409pt}{0.400pt}}
\put(1429.0,323.0){\rule[-0.200pt]{2.409pt}{0.400pt}}
\put(170.0,323.0){\rule[-0.200pt]{2.409pt}{0.400pt}}
\put(1429.0,323.0){\rule[-0.200pt]{2.409pt}{0.400pt}}
\put(170.0,323.0){\rule[-0.200pt]{2.409pt}{0.400pt}}
\put(1429.0,323.0){\rule[-0.200pt]{2.409pt}{0.400pt}}
\put(170.0,323.0){\rule[-0.200pt]{2.409pt}{0.400pt}}
\put(1429.0,323.0){\rule[-0.200pt]{2.409pt}{0.400pt}}
\put(170.0,323.0){\rule[-0.200pt]{2.409pt}{0.400pt}}
\put(1429.0,323.0){\rule[-0.200pt]{2.409pt}{0.400pt}}
\put(170.0,323.0){\rule[-0.200pt]{2.409pt}{0.400pt}}
\put(1429.0,323.0){\rule[-0.200pt]{2.409pt}{0.400pt}}
\put(170.0,323.0){\rule[-0.200pt]{2.409pt}{0.400pt}}
\put(1429.0,323.0){\rule[-0.200pt]{2.409pt}{0.400pt}}
\put(170.0,323.0){\rule[-0.200pt]{2.409pt}{0.400pt}}
\put(1429.0,323.0){\rule[-0.200pt]{2.409pt}{0.400pt}}
\put(170.0,323.0){\rule[-0.200pt]{2.409pt}{0.400pt}}
\put(1429.0,323.0){\rule[-0.200pt]{2.409pt}{0.400pt}}
\put(170.0,323.0){\rule[-0.200pt]{2.409pt}{0.400pt}}
\put(1429.0,323.0){\rule[-0.200pt]{2.409pt}{0.400pt}}
\put(170.0,323.0){\rule[-0.200pt]{2.409pt}{0.400pt}}
\put(1429.0,323.0){\rule[-0.200pt]{2.409pt}{0.400pt}}
\put(170.0,323.0){\rule[-0.200pt]{2.409pt}{0.400pt}}
\put(1429.0,323.0){\rule[-0.200pt]{2.409pt}{0.400pt}}
\put(170.0,323.0){\rule[-0.200pt]{2.409pt}{0.400pt}}
\put(1429.0,323.0){\rule[-0.200pt]{2.409pt}{0.400pt}}
\put(170.0,324.0){\rule[-0.200pt]{2.409pt}{0.400pt}}
\put(1429.0,324.0){\rule[-0.200pt]{2.409pt}{0.400pt}}
\put(170.0,324.0){\rule[-0.200pt]{2.409pt}{0.400pt}}
\put(1429.0,324.0){\rule[-0.200pt]{2.409pt}{0.400pt}}
\put(170.0,324.0){\rule[-0.200pt]{2.409pt}{0.400pt}}
\put(1429.0,324.0){\rule[-0.200pt]{2.409pt}{0.400pt}}
\put(170.0,324.0){\rule[-0.200pt]{2.409pt}{0.400pt}}
\put(1429.0,324.0){\rule[-0.200pt]{2.409pt}{0.400pt}}
\put(170.0,324.0){\rule[-0.200pt]{2.409pt}{0.400pt}}
\put(1429.0,324.0){\rule[-0.200pt]{2.409pt}{0.400pt}}
\put(170.0,324.0){\rule[-0.200pt]{2.409pt}{0.400pt}}
\put(1429.0,324.0){\rule[-0.200pt]{2.409pt}{0.400pt}}
\put(170.0,324.0){\rule[-0.200pt]{2.409pt}{0.400pt}}
\put(1429.0,324.0){\rule[-0.200pt]{2.409pt}{0.400pt}}
\put(170.0,324.0){\rule[-0.200pt]{2.409pt}{0.400pt}}
\put(1429.0,324.0){\rule[-0.200pt]{2.409pt}{0.400pt}}
\put(170.0,324.0){\rule[-0.200pt]{2.409pt}{0.400pt}}
\put(1429.0,324.0){\rule[-0.200pt]{2.409pt}{0.400pt}}
\put(170.0,324.0){\rule[-0.200pt]{2.409pt}{0.400pt}}
\put(1429.0,324.0){\rule[-0.200pt]{2.409pt}{0.400pt}}
\put(170.0,324.0){\rule[-0.200pt]{2.409pt}{0.400pt}}
\put(1429.0,324.0){\rule[-0.200pt]{2.409pt}{0.400pt}}
\put(170.0,324.0){\rule[-0.200pt]{2.409pt}{0.400pt}}
\put(1429.0,324.0){\rule[-0.200pt]{2.409pt}{0.400pt}}
\put(170.0,324.0){\rule[-0.200pt]{2.409pt}{0.400pt}}
\put(1429.0,324.0){\rule[-0.200pt]{2.409pt}{0.400pt}}
\put(170.0,324.0){\rule[-0.200pt]{2.409pt}{0.400pt}}
\put(1429.0,324.0){\rule[-0.200pt]{2.409pt}{0.400pt}}
\put(170.0,324.0){\rule[-0.200pt]{2.409pt}{0.400pt}}
\put(1429.0,324.0){\rule[-0.200pt]{2.409pt}{0.400pt}}
\put(170.0,324.0){\rule[-0.200pt]{2.409pt}{0.400pt}}
\put(1429.0,324.0){\rule[-0.200pt]{2.409pt}{0.400pt}}
\put(170.0,325.0){\rule[-0.200pt]{2.409pt}{0.400pt}}
\put(1429.0,325.0){\rule[-0.200pt]{2.409pt}{0.400pt}}
\put(170.0,325.0){\rule[-0.200pt]{2.409pt}{0.400pt}}
\put(1429.0,325.0){\rule[-0.200pt]{2.409pt}{0.400pt}}
\put(170.0,325.0){\rule[-0.200pt]{2.409pt}{0.400pt}}
\put(1429.0,325.0){\rule[-0.200pt]{2.409pt}{0.400pt}}
\put(170.0,325.0){\rule[-0.200pt]{2.409pt}{0.400pt}}
\put(1429.0,325.0){\rule[-0.200pt]{2.409pt}{0.400pt}}
\put(170.0,325.0){\rule[-0.200pt]{2.409pt}{0.400pt}}
\put(1429.0,325.0){\rule[-0.200pt]{2.409pt}{0.400pt}}
\put(170.0,325.0){\rule[-0.200pt]{2.409pt}{0.400pt}}
\put(1429.0,325.0){\rule[-0.200pt]{2.409pt}{0.400pt}}
\put(170.0,325.0){\rule[-0.200pt]{2.409pt}{0.400pt}}
\put(1429.0,325.0){\rule[-0.200pt]{2.409pt}{0.400pt}}
\put(170.0,325.0){\rule[-0.200pt]{2.409pt}{0.400pt}}
\put(1429.0,325.0){\rule[-0.200pt]{2.409pt}{0.400pt}}
\put(170.0,325.0){\rule[-0.200pt]{2.409pt}{0.400pt}}
\put(1429.0,325.0){\rule[-0.200pt]{2.409pt}{0.400pt}}
\put(170.0,325.0){\rule[-0.200pt]{2.409pt}{0.400pt}}
\put(1429.0,325.0){\rule[-0.200pt]{2.409pt}{0.400pt}}
\put(170.0,325.0){\rule[-0.200pt]{2.409pt}{0.400pt}}
\put(1429.0,325.0){\rule[-0.200pt]{2.409pt}{0.400pt}}
\put(170.0,325.0){\rule[-0.200pt]{2.409pt}{0.400pt}}
\put(1429.0,325.0){\rule[-0.200pt]{2.409pt}{0.400pt}}
\put(170.0,325.0){\rule[-0.200pt]{2.409pt}{0.400pt}}
\put(1429.0,325.0){\rule[-0.200pt]{2.409pt}{0.400pt}}
\put(170.0,325.0){\rule[-0.200pt]{2.409pt}{0.400pt}}
\put(1429.0,325.0){\rule[-0.200pt]{2.409pt}{0.400pt}}
\put(170.0,325.0){\rule[-0.200pt]{2.409pt}{0.400pt}}
\put(1429.0,325.0){\rule[-0.200pt]{2.409pt}{0.400pt}}
\put(170.0,325.0){\rule[-0.200pt]{2.409pt}{0.400pt}}
\put(1429.0,325.0){\rule[-0.200pt]{2.409pt}{0.400pt}}
\put(170.0,325.0){\rule[-0.200pt]{2.409pt}{0.400pt}}
\put(1429.0,325.0){\rule[-0.200pt]{2.409pt}{0.400pt}}
\put(170.0,325.0){\rule[-0.200pt]{2.409pt}{0.400pt}}
\put(1429.0,325.0){\rule[-0.200pt]{2.409pt}{0.400pt}}
\put(170.0,326.0){\rule[-0.200pt]{2.409pt}{0.400pt}}
\put(1429.0,326.0){\rule[-0.200pt]{2.409pt}{0.400pt}}
\put(170.0,326.0){\rule[-0.200pt]{2.409pt}{0.400pt}}
\put(1429.0,326.0){\rule[-0.200pt]{2.409pt}{0.400pt}}
\put(170.0,326.0){\rule[-0.200pt]{2.409pt}{0.400pt}}
\put(1429.0,326.0){\rule[-0.200pt]{2.409pt}{0.400pt}}
\put(170.0,326.0){\rule[-0.200pt]{2.409pt}{0.400pt}}
\put(1429.0,326.0){\rule[-0.200pt]{2.409pt}{0.400pt}}
\put(170.0,326.0){\rule[-0.200pt]{2.409pt}{0.400pt}}
\put(1429.0,326.0){\rule[-0.200pt]{2.409pt}{0.400pt}}
\put(170.0,326.0){\rule[-0.200pt]{2.409pt}{0.400pt}}
\put(1429.0,326.0){\rule[-0.200pt]{2.409pt}{0.400pt}}
\put(170.0,326.0){\rule[-0.200pt]{2.409pt}{0.400pt}}
\put(1429.0,326.0){\rule[-0.200pt]{2.409pt}{0.400pt}}
\put(170.0,326.0){\rule[-0.200pt]{2.409pt}{0.400pt}}
\put(1429.0,326.0){\rule[-0.200pt]{2.409pt}{0.400pt}}
\put(170.0,326.0){\rule[-0.200pt]{2.409pt}{0.400pt}}
\put(1429.0,326.0){\rule[-0.200pt]{2.409pt}{0.400pt}}
\put(170.0,326.0){\rule[-0.200pt]{2.409pt}{0.400pt}}
\put(1429.0,326.0){\rule[-0.200pt]{2.409pt}{0.400pt}}
\put(170.0,326.0){\rule[-0.200pt]{2.409pt}{0.400pt}}
\put(1429.0,326.0){\rule[-0.200pt]{2.409pt}{0.400pt}}
\put(170.0,326.0){\rule[-0.200pt]{2.409pt}{0.400pt}}
\put(1429.0,326.0){\rule[-0.200pt]{2.409pt}{0.400pt}}
\put(170.0,326.0){\rule[-0.200pt]{2.409pt}{0.400pt}}
\put(1429.0,326.0){\rule[-0.200pt]{2.409pt}{0.400pt}}
\put(170.0,326.0){\rule[-0.200pt]{2.409pt}{0.400pt}}
\put(1429.0,326.0){\rule[-0.200pt]{2.409pt}{0.400pt}}
\put(170.0,326.0){\rule[-0.200pt]{2.409pt}{0.400pt}}
\put(1429.0,326.0){\rule[-0.200pt]{2.409pt}{0.400pt}}
\put(170.0,326.0){\rule[-0.200pt]{2.409pt}{0.400pt}}
\put(1429.0,326.0){\rule[-0.200pt]{2.409pt}{0.400pt}}
\put(170.0,326.0){\rule[-0.200pt]{2.409pt}{0.400pt}}
\put(1429.0,326.0){\rule[-0.200pt]{2.409pt}{0.400pt}}
\put(170.0,326.0){\rule[-0.200pt]{2.409pt}{0.400pt}}
\put(1429.0,326.0){\rule[-0.200pt]{2.409pt}{0.400pt}}
\put(170.0,327.0){\rule[-0.200pt]{2.409pt}{0.400pt}}
\put(1429.0,327.0){\rule[-0.200pt]{2.409pt}{0.400pt}}
\put(170.0,327.0){\rule[-0.200pt]{2.409pt}{0.400pt}}
\put(1429.0,327.0){\rule[-0.200pt]{2.409pt}{0.400pt}}
\put(170.0,327.0){\rule[-0.200pt]{2.409pt}{0.400pt}}
\put(1429.0,327.0){\rule[-0.200pt]{2.409pt}{0.400pt}}
\put(170.0,327.0){\rule[-0.200pt]{2.409pt}{0.400pt}}
\put(1429.0,327.0){\rule[-0.200pt]{2.409pt}{0.400pt}}
\put(170.0,327.0){\rule[-0.200pt]{2.409pt}{0.400pt}}
\put(1429.0,327.0){\rule[-0.200pt]{2.409pt}{0.400pt}}
\put(170.0,327.0){\rule[-0.200pt]{2.409pt}{0.400pt}}
\put(1429.0,327.0){\rule[-0.200pt]{2.409pt}{0.400pt}}
\put(170.0,327.0){\rule[-0.200pt]{2.409pt}{0.400pt}}
\put(1429.0,327.0){\rule[-0.200pt]{2.409pt}{0.400pt}}
\put(170.0,327.0){\rule[-0.200pt]{2.409pt}{0.400pt}}
\put(1429.0,327.0){\rule[-0.200pt]{2.409pt}{0.400pt}}
\put(170.0,327.0){\rule[-0.200pt]{2.409pt}{0.400pt}}
\put(1429.0,327.0){\rule[-0.200pt]{2.409pt}{0.400pt}}
\put(170.0,327.0){\rule[-0.200pt]{2.409pt}{0.400pt}}
\put(1429.0,327.0){\rule[-0.200pt]{2.409pt}{0.400pt}}
\put(170.0,327.0){\rule[-0.200pt]{2.409pt}{0.400pt}}
\put(1429.0,327.0){\rule[-0.200pt]{2.409pt}{0.400pt}}
\put(170.0,327.0){\rule[-0.200pt]{2.409pt}{0.400pt}}
\put(1429.0,327.0){\rule[-0.200pt]{2.409pt}{0.400pt}}
\put(170.0,327.0){\rule[-0.200pt]{2.409pt}{0.400pt}}
\put(1429.0,327.0){\rule[-0.200pt]{2.409pt}{0.400pt}}
\put(170.0,327.0){\rule[-0.200pt]{2.409pt}{0.400pt}}
\put(1429.0,327.0){\rule[-0.200pt]{2.409pt}{0.400pt}}
\put(170.0,327.0){\rule[-0.200pt]{2.409pt}{0.400pt}}
\put(1429.0,327.0){\rule[-0.200pt]{2.409pt}{0.400pt}}
\put(170.0,327.0){\rule[-0.200pt]{2.409pt}{0.400pt}}
\put(1429.0,327.0){\rule[-0.200pt]{2.409pt}{0.400pt}}
\put(170.0,327.0){\rule[-0.200pt]{2.409pt}{0.400pt}}
\put(1429.0,327.0){\rule[-0.200pt]{2.409pt}{0.400pt}}
\put(170.0,327.0){\rule[-0.200pt]{2.409pt}{0.400pt}}
\put(1429.0,327.0){\rule[-0.200pt]{2.409pt}{0.400pt}}
\put(170.0,328.0){\rule[-0.200pt]{2.409pt}{0.400pt}}
\put(1429.0,328.0){\rule[-0.200pt]{2.409pt}{0.400pt}}
\put(170.0,328.0){\rule[-0.200pt]{2.409pt}{0.400pt}}
\put(1429.0,328.0){\rule[-0.200pt]{2.409pt}{0.400pt}}
\put(170.0,328.0){\rule[-0.200pt]{2.409pt}{0.400pt}}
\put(1429.0,328.0){\rule[-0.200pt]{2.409pt}{0.400pt}}
\put(170.0,328.0){\rule[-0.200pt]{2.409pt}{0.400pt}}
\put(1429.0,328.0){\rule[-0.200pt]{2.409pt}{0.400pt}}
\put(170.0,328.0){\rule[-0.200pt]{2.409pt}{0.400pt}}
\put(1429.0,328.0){\rule[-0.200pt]{2.409pt}{0.400pt}}
\put(170.0,328.0){\rule[-0.200pt]{2.409pt}{0.400pt}}
\put(1429.0,328.0){\rule[-0.200pt]{2.409pt}{0.400pt}}
\put(170.0,328.0){\rule[-0.200pt]{2.409pt}{0.400pt}}
\put(1429.0,328.0){\rule[-0.200pt]{2.409pt}{0.400pt}}
\put(170.0,328.0){\rule[-0.200pt]{2.409pt}{0.400pt}}
\put(1429.0,328.0){\rule[-0.200pt]{2.409pt}{0.400pt}}
\put(170.0,328.0){\rule[-0.200pt]{2.409pt}{0.400pt}}
\put(1429.0,328.0){\rule[-0.200pt]{2.409pt}{0.400pt}}
\put(170.0,328.0){\rule[-0.200pt]{2.409pt}{0.400pt}}
\put(1429.0,328.0){\rule[-0.200pt]{2.409pt}{0.400pt}}
\put(170.0,328.0){\rule[-0.200pt]{2.409pt}{0.400pt}}
\put(1429.0,328.0){\rule[-0.200pt]{2.409pt}{0.400pt}}
\put(170.0,328.0){\rule[-0.200pt]{2.409pt}{0.400pt}}
\put(1429.0,328.0){\rule[-0.200pt]{2.409pt}{0.400pt}}
\put(170.0,328.0){\rule[-0.200pt]{2.409pt}{0.400pt}}
\put(1429.0,328.0){\rule[-0.200pt]{2.409pt}{0.400pt}}
\put(170.0,328.0){\rule[-0.200pt]{2.409pt}{0.400pt}}
\put(1429.0,328.0){\rule[-0.200pt]{2.409pt}{0.400pt}}
\put(170.0,328.0){\rule[-0.200pt]{2.409pt}{0.400pt}}
\put(1429.0,328.0){\rule[-0.200pt]{2.409pt}{0.400pt}}
\put(170.0,328.0){\rule[-0.200pt]{2.409pt}{0.400pt}}
\put(1429.0,328.0){\rule[-0.200pt]{2.409pt}{0.400pt}}
\put(170.0,328.0){\rule[-0.200pt]{2.409pt}{0.400pt}}
\put(1429.0,328.0){\rule[-0.200pt]{2.409pt}{0.400pt}}
\put(170.0,328.0){\rule[-0.200pt]{2.409pt}{0.400pt}}
\put(1429.0,328.0){\rule[-0.200pt]{2.409pt}{0.400pt}}
\put(170.0,328.0){\rule[-0.200pt]{2.409pt}{0.400pt}}
\put(1429.0,328.0){\rule[-0.200pt]{2.409pt}{0.400pt}}
\put(170.0,329.0){\rule[-0.200pt]{2.409pt}{0.400pt}}
\put(1429.0,329.0){\rule[-0.200pt]{2.409pt}{0.400pt}}
\put(170.0,329.0){\rule[-0.200pt]{2.409pt}{0.400pt}}
\put(1429.0,329.0){\rule[-0.200pt]{2.409pt}{0.400pt}}
\put(170.0,329.0){\rule[-0.200pt]{2.409pt}{0.400pt}}
\put(1429.0,329.0){\rule[-0.200pt]{2.409pt}{0.400pt}}
\put(170.0,329.0){\rule[-0.200pt]{2.409pt}{0.400pt}}
\put(1429.0,329.0){\rule[-0.200pt]{2.409pt}{0.400pt}}
\put(170.0,329.0){\rule[-0.200pt]{2.409pt}{0.400pt}}
\put(1429.0,329.0){\rule[-0.200pt]{2.409pt}{0.400pt}}
\put(170.0,329.0){\rule[-0.200pt]{2.409pt}{0.400pt}}
\put(1429.0,329.0){\rule[-0.200pt]{2.409pt}{0.400pt}}
\put(170.0,329.0){\rule[-0.200pt]{2.409pt}{0.400pt}}
\put(1429.0,329.0){\rule[-0.200pt]{2.409pt}{0.400pt}}
\put(170.0,329.0){\rule[-0.200pt]{2.409pt}{0.400pt}}
\put(1429.0,329.0){\rule[-0.200pt]{2.409pt}{0.400pt}}
\put(170.0,329.0){\rule[-0.200pt]{2.409pt}{0.400pt}}
\put(1429.0,329.0){\rule[-0.200pt]{2.409pt}{0.400pt}}
\put(170.0,329.0){\rule[-0.200pt]{2.409pt}{0.400pt}}
\put(1429.0,329.0){\rule[-0.200pt]{2.409pt}{0.400pt}}
\put(170.0,329.0){\rule[-0.200pt]{2.409pt}{0.400pt}}
\put(1429.0,329.0){\rule[-0.200pt]{2.409pt}{0.400pt}}
\put(170.0,329.0){\rule[-0.200pt]{2.409pt}{0.400pt}}
\put(1429.0,329.0){\rule[-0.200pt]{2.409pt}{0.400pt}}
\put(170.0,329.0){\rule[-0.200pt]{2.409pt}{0.400pt}}
\put(1429.0,329.0){\rule[-0.200pt]{2.409pt}{0.400pt}}
\put(170.0,329.0){\rule[-0.200pt]{2.409pt}{0.400pt}}
\put(1429.0,329.0){\rule[-0.200pt]{2.409pt}{0.400pt}}
\put(170.0,329.0){\rule[-0.200pt]{2.409pt}{0.400pt}}
\put(1429.0,329.0){\rule[-0.200pt]{2.409pt}{0.400pt}}
\put(170.0,329.0){\rule[-0.200pt]{2.409pt}{0.400pt}}
\put(1429.0,329.0){\rule[-0.200pt]{2.409pt}{0.400pt}}
\put(170.0,329.0){\rule[-0.200pt]{2.409pt}{0.400pt}}
\put(1429.0,329.0){\rule[-0.200pt]{2.409pt}{0.400pt}}
\put(170.0,329.0){\rule[-0.200pt]{2.409pt}{0.400pt}}
\put(1429.0,329.0){\rule[-0.200pt]{2.409pt}{0.400pt}}
\put(170.0,329.0){\rule[-0.200pt]{2.409pt}{0.400pt}}
\put(1429.0,329.0){\rule[-0.200pt]{2.409pt}{0.400pt}}
\put(170.0,330.0){\rule[-0.200pt]{2.409pt}{0.400pt}}
\put(1429.0,330.0){\rule[-0.200pt]{2.409pt}{0.400pt}}
\put(170.0,330.0){\rule[-0.200pt]{2.409pt}{0.400pt}}
\put(1429.0,330.0){\rule[-0.200pt]{2.409pt}{0.400pt}}
\put(170.0,330.0){\rule[-0.200pt]{2.409pt}{0.400pt}}
\put(1429.0,330.0){\rule[-0.200pt]{2.409pt}{0.400pt}}
\put(170.0,330.0){\rule[-0.200pt]{2.409pt}{0.400pt}}
\put(1429.0,330.0){\rule[-0.200pt]{2.409pt}{0.400pt}}
\put(170.0,330.0){\rule[-0.200pt]{2.409pt}{0.400pt}}
\put(1429.0,330.0){\rule[-0.200pt]{2.409pt}{0.400pt}}
\put(170.0,330.0){\rule[-0.200pt]{2.409pt}{0.400pt}}
\put(1429.0,330.0){\rule[-0.200pt]{2.409pt}{0.400pt}}
\put(170.0,330.0){\rule[-0.200pt]{2.409pt}{0.400pt}}
\put(1429.0,330.0){\rule[-0.200pt]{2.409pt}{0.400pt}}
\put(170.0,330.0){\rule[-0.200pt]{2.409pt}{0.400pt}}
\put(1429.0,330.0){\rule[-0.200pt]{2.409pt}{0.400pt}}
\put(170.0,330.0){\rule[-0.200pt]{2.409pt}{0.400pt}}
\put(1429.0,330.0){\rule[-0.200pt]{2.409pt}{0.400pt}}
\put(170.0,330.0){\rule[-0.200pt]{2.409pt}{0.400pt}}
\put(1429.0,330.0){\rule[-0.200pt]{2.409pt}{0.400pt}}
\put(170.0,330.0){\rule[-0.200pt]{2.409pt}{0.400pt}}
\put(1429.0,330.0){\rule[-0.200pt]{2.409pt}{0.400pt}}
\put(170.0,330.0){\rule[-0.200pt]{2.409pt}{0.400pt}}
\put(1429.0,330.0){\rule[-0.200pt]{2.409pt}{0.400pt}}
\put(170.0,330.0){\rule[-0.200pt]{2.409pt}{0.400pt}}
\put(1429.0,330.0){\rule[-0.200pt]{2.409pt}{0.400pt}}
\put(170.0,330.0){\rule[-0.200pt]{2.409pt}{0.400pt}}
\put(1429.0,330.0){\rule[-0.200pt]{2.409pt}{0.400pt}}
\put(170.0,330.0){\rule[-0.200pt]{2.409pt}{0.400pt}}
\put(1429.0,330.0){\rule[-0.200pt]{2.409pt}{0.400pt}}
\put(170.0,330.0){\rule[-0.200pt]{2.409pt}{0.400pt}}
\put(1429.0,330.0){\rule[-0.200pt]{2.409pt}{0.400pt}}
\put(170.0,330.0){\rule[-0.200pt]{2.409pt}{0.400pt}}
\put(1429.0,330.0){\rule[-0.200pt]{2.409pt}{0.400pt}}
\put(170.0,330.0){\rule[-0.200pt]{2.409pt}{0.400pt}}
\put(1429.0,330.0){\rule[-0.200pt]{2.409pt}{0.400pt}}
\put(170.0,330.0){\rule[-0.200pt]{2.409pt}{0.400pt}}
\put(1429.0,330.0){\rule[-0.200pt]{2.409pt}{0.400pt}}
\put(170.0,330.0){\rule[-0.200pt]{2.409pt}{0.400pt}}
\put(1429.0,330.0){\rule[-0.200pt]{2.409pt}{0.400pt}}
\put(170.0,331.0){\rule[-0.200pt]{2.409pt}{0.400pt}}
\put(1429.0,331.0){\rule[-0.200pt]{2.409pt}{0.400pt}}
\put(170.0,331.0){\rule[-0.200pt]{2.409pt}{0.400pt}}
\put(1429.0,331.0){\rule[-0.200pt]{2.409pt}{0.400pt}}
\put(170.0,331.0){\rule[-0.200pt]{2.409pt}{0.400pt}}
\put(1429.0,331.0){\rule[-0.200pt]{2.409pt}{0.400pt}}
\put(170.0,331.0){\rule[-0.200pt]{2.409pt}{0.400pt}}
\put(1429.0,331.0){\rule[-0.200pt]{2.409pt}{0.400pt}}
\put(170.0,331.0){\rule[-0.200pt]{2.409pt}{0.400pt}}
\put(1429.0,331.0){\rule[-0.200pt]{2.409pt}{0.400pt}}
\put(170.0,331.0){\rule[-0.200pt]{2.409pt}{0.400pt}}
\put(1429.0,331.0){\rule[-0.200pt]{2.409pt}{0.400pt}}
\put(170.0,331.0){\rule[-0.200pt]{2.409pt}{0.400pt}}
\put(1429.0,331.0){\rule[-0.200pt]{2.409pt}{0.400pt}}
\put(170.0,331.0){\rule[-0.200pt]{2.409pt}{0.400pt}}
\put(1429.0,331.0){\rule[-0.200pt]{2.409pt}{0.400pt}}
\put(170.0,331.0){\rule[-0.200pt]{2.409pt}{0.400pt}}
\put(1429.0,331.0){\rule[-0.200pt]{2.409pt}{0.400pt}}
\put(170.0,331.0){\rule[-0.200pt]{2.409pt}{0.400pt}}
\put(1429.0,331.0){\rule[-0.200pt]{2.409pt}{0.400pt}}
\put(170.0,331.0){\rule[-0.200pt]{2.409pt}{0.400pt}}
\put(1429.0,331.0){\rule[-0.200pt]{2.409pt}{0.400pt}}
\put(170.0,331.0){\rule[-0.200pt]{2.409pt}{0.400pt}}
\put(1429.0,331.0){\rule[-0.200pt]{2.409pt}{0.400pt}}
\put(170.0,331.0){\rule[-0.200pt]{2.409pt}{0.400pt}}
\put(1429.0,331.0){\rule[-0.200pt]{2.409pt}{0.400pt}}
\put(170.0,331.0){\rule[-0.200pt]{2.409pt}{0.400pt}}
\put(1429.0,331.0){\rule[-0.200pt]{2.409pt}{0.400pt}}
\put(170.0,331.0){\rule[-0.200pt]{2.409pt}{0.400pt}}
\put(1429.0,331.0){\rule[-0.200pt]{2.409pt}{0.400pt}}
\put(170.0,331.0){\rule[-0.200pt]{2.409pt}{0.400pt}}
\put(1429.0,331.0){\rule[-0.200pt]{2.409pt}{0.400pt}}
\put(170.0,331.0){\rule[-0.200pt]{2.409pt}{0.400pt}}
\put(1429.0,331.0){\rule[-0.200pt]{2.409pt}{0.400pt}}
\put(170.0,331.0){\rule[-0.200pt]{2.409pt}{0.400pt}}
\put(1429.0,331.0){\rule[-0.200pt]{2.409pt}{0.400pt}}
\put(170.0,331.0){\rule[-0.200pt]{2.409pt}{0.400pt}}
\put(1429.0,331.0){\rule[-0.200pt]{2.409pt}{0.400pt}}
\put(170.0,331.0){\rule[-0.200pt]{2.409pt}{0.400pt}}
\put(1429.0,331.0){\rule[-0.200pt]{2.409pt}{0.400pt}}
\put(170.0,331.0){\rule[-0.200pt]{2.409pt}{0.400pt}}
\put(1429.0,331.0){\rule[-0.200pt]{2.409pt}{0.400pt}}
\put(170.0,332.0){\rule[-0.200pt]{2.409pt}{0.400pt}}
\put(1429.0,332.0){\rule[-0.200pt]{2.409pt}{0.400pt}}
\put(170.0,332.0){\rule[-0.200pt]{2.409pt}{0.400pt}}
\put(1429.0,332.0){\rule[-0.200pt]{2.409pt}{0.400pt}}
\put(170.0,332.0){\rule[-0.200pt]{2.409pt}{0.400pt}}
\put(1429.0,332.0){\rule[-0.200pt]{2.409pt}{0.400pt}}
\put(170.0,332.0){\rule[-0.200pt]{2.409pt}{0.400pt}}
\put(1429.0,332.0){\rule[-0.200pt]{2.409pt}{0.400pt}}
\put(170.0,332.0){\rule[-0.200pt]{2.409pt}{0.400pt}}
\put(1429.0,332.0){\rule[-0.200pt]{2.409pt}{0.400pt}}
\put(170.0,332.0){\rule[-0.200pt]{2.409pt}{0.400pt}}
\put(1429.0,332.0){\rule[-0.200pt]{2.409pt}{0.400pt}}
\put(170.0,332.0){\rule[-0.200pt]{2.409pt}{0.400pt}}
\put(1429.0,332.0){\rule[-0.200pt]{2.409pt}{0.400pt}}
\put(170.0,332.0){\rule[-0.200pt]{2.409pt}{0.400pt}}
\put(1429.0,332.0){\rule[-0.200pt]{2.409pt}{0.400pt}}
\put(170.0,332.0){\rule[-0.200pt]{2.409pt}{0.400pt}}
\put(1429.0,332.0){\rule[-0.200pt]{2.409pt}{0.400pt}}
\put(170.0,332.0){\rule[-0.200pt]{2.409pt}{0.400pt}}
\put(1429.0,332.0){\rule[-0.200pt]{2.409pt}{0.400pt}}
\put(170.0,332.0){\rule[-0.200pt]{2.409pt}{0.400pt}}
\put(1429.0,332.0){\rule[-0.200pt]{2.409pt}{0.400pt}}
\put(170.0,332.0){\rule[-0.200pt]{2.409pt}{0.400pt}}
\put(1429.0,332.0){\rule[-0.200pt]{2.409pt}{0.400pt}}
\put(170.0,332.0){\rule[-0.200pt]{2.409pt}{0.400pt}}
\put(1429.0,332.0){\rule[-0.200pt]{2.409pt}{0.400pt}}
\put(170.0,332.0){\rule[-0.200pt]{2.409pt}{0.400pt}}
\put(1429.0,332.0){\rule[-0.200pt]{2.409pt}{0.400pt}}
\put(170.0,332.0){\rule[-0.200pt]{2.409pt}{0.400pt}}
\put(1429.0,332.0){\rule[-0.200pt]{2.409pt}{0.400pt}}
\put(170.0,332.0){\rule[-0.200pt]{2.409pt}{0.400pt}}
\put(1429.0,332.0){\rule[-0.200pt]{2.409pt}{0.400pt}}
\put(170.0,332.0){\rule[-0.200pt]{2.409pt}{0.400pt}}
\put(1429.0,332.0){\rule[-0.200pt]{2.409pt}{0.400pt}}
\put(170.0,332.0){\rule[-0.200pt]{2.409pt}{0.400pt}}
\put(1429.0,332.0){\rule[-0.200pt]{2.409pt}{0.400pt}}
\put(170.0,332.0){\rule[-0.200pt]{2.409pt}{0.400pt}}
\put(1429.0,332.0){\rule[-0.200pt]{2.409pt}{0.400pt}}
\put(170.0,332.0){\rule[-0.200pt]{2.409pt}{0.400pt}}
\put(1429.0,332.0){\rule[-0.200pt]{2.409pt}{0.400pt}}
\put(170.0,332.0){\rule[-0.200pt]{2.409pt}{0.400pt}}
\put(1429.0,332.0){\rule[-0.200pt]{2.409pt}{0.400pt}}
\put(170.0,333.0){\rule[-0.200pt]{2.409pt}{0.400pt}}
\put(1429.0,333.0){\rule[-0.200pt]{2.409pt}{0.400pt}}
\put(170.0,333.0){\rule[-0.200pt]{2.409pt}{0.400pt}}
\put(1429.0,333.0){\rule[-0.200pt]{2.409pt}{0.400pt}}
\put(170.0,333.0){\rule[-0.200pt]{2.409pt}{0.400pt}}
\put(1429.0,333.0){\rule[-0.200pt]{2.409pt}{0.400pt}}
\put(170.0,333.0){\rule[-0.200pt]{2.409pt}{0.400pt}}
\put(1429.0,333.0){\rule[-0.200pt]{2.409pt}{0.400pt}}
\put(170.0,333.0){\rule[-0.200pt]{2.409pt}{0.400pt}}
\put(1429.0,333.0){\rule[-0.200pt]{2.409pt}{0.400pt}}
\put(170.0,333.0){\rule[-0.200pt]{2.409pt}{0.400pt}}
\put(1429.0,333.0){\rule[-0.200pt]{2.409pt}{0.400pt}}
\put(170.0,333.0){\rule[-0.200pt]{2.409pt}{0.400pt}}
\put(1429.0,333.0){\rule[-0.200pt]{2.409pt}{0.400pt}}
\put(170.0,333.0){\rule[-0.200pt]{2.409pt}{0.400pt}}
\put(1429.0,333.0){\rule[-0.200pt]{2.409pt}{0.400pt}}
\put(170.0,333.0){\rule[-0.200pt]{2.409pt}{0.400pt}}
\put(1429.0,333.0){\rule[-0.200pt]{2.409pt}{0.400pt}}
\put(170.0,333.0){\rule[-0.200pt]{2.409pt}{0.400pt}}
\put(1429.0,333.0){\rule[-0.200pt]{2.409pt}{0.400pt}}
\put(170.0,333.0){\rule[-0.200pt]{2.409pt}{0.400pt}}
\put(1429.0,333.0){\rule[-0.200pt]{2.409pt}{0.400pt}}
\put(170.0,333.0){\rule[-0.200pt]{2.409pt}{0.400pt}}
\put(1429.0,333.0){\rule[-0.200pt]{2.409pt}{0.400pt}}
\put(170.0,333.0){\rule[-0.200pt]{2.409pt}{0.400pt}}
\put(1429.0,333.0){\rule[-0.200pt]{2.409pt}{0.400pt}}
\put(170.0,333.0){\rule[-0.200pt]{2.409pt}{0.400pt}}
\put(1429.0,333.0){\rule[-0.200pt]{2.409pt}{0.400pt}}
\put(170.0,333.0){\rule[-0.200pt]{2.409pt}{0.400pt}}
\put(1429.0,333.0){\rule[-0.200pt]{2.409pt}{0.400pt}}
\put(170.0,333.0){\rule[-0.200pt]{2.409pt}{0.400pt}}
\put(1429.0,333.0){\rule[-0.200pt]{2.409pt}{0.400pt}}
\put(170.0,333.0){\rule[-0.200pt]{2.409pt}{0.400pt}}
\put(1429.0,333.0){\rule[-0.200pt]{2.409pt}{0.400pt}}
\put(170.0,333.0){\rule[-0.200pt]{2.409pt}{0.400pt}}
\put(1429.0,333.0){\rule[-0.200pt]{2.409pt}{0.400pt}}
\put(170.0,333.0){\rule[-0.200pt]{2.409pt}{0.400pt}}
\put(1429.0,333.0){\rule[-0.200pt]{2.409pt}{0.400pt}}
\put(170.0,333.0){\rule[-0.200pt]{2.409pt}{0.400pt}}
\put(1429.0,333.0){\rule[-0.200pt]{2.409pt}{0.400pt}}
\put(170.0,333.0){\rule[-0.200pt]{2.409pt}{0.400pt}}
\put(1429.0,333.0){\rule[-0.200pt]{2.409pt}{0.400pt}}
\put(170.0,334.0){\rule[-0.200pt]{2.409pt}{0.400pt}}
\put(1429.0,334.0){\rule[-0.200pt]{2.409pt}{0.400pt}}
\put(170.0,334.0){\rule[-0.200pt]{2.409pt}{0.400pt}}
\put(1429.0,334.0){\rule[-0.200pt]{2.409pt}{0.400pt}}
\put(170.0,334.0){\rule[-0.200pt]{2.409pt}{0.400pt}}
\put(1429.0,334.0){\rule[-0.200pt]{2.409pt}{0.400pt}}
\put(170.0,334.0){\rule[-0.200pt]{2.409pt}{0.400pt}}
\put(1429.0,334.0){\rule[-0.200pt]{2.409pt}{0.400pt}}
\put(170.0,334.0){\rule[-0.200pt]{2.409pt}{0.400pt}}
\put(1429.0,334.0){\rule[-0.200pt]{2.409pt}{0.400pt}}
\put(170.0,334.0){\rule[-0.200pt]{2.409pt}{0.400pt}}
\put(1429.0,334.0){\rule[-0.200pt]{2.409pt}{0.400pt}}
\put(170.0,334.0){\rule[-0.200pt]{2.409pt}{0.400pt}}
\put(1429.0,334.0){\rule[-0.200pt]{2.409pt}{0.400pt}}
\put(170.0,334.0){\rule[-0.200pt]{2.409pt}{0.400pt}}
\put(1429.0,334.0){\rule[-0.200pt]{2.409pt}{0.400pt}}
\put(170.0,334.0){\rule[-0.200pt]{2.409pt}{0.400pt}}
\put(1429.0,334.0){\rule[-0.200pt]{2.409pt}{0.400pt}}
\put(170.0,334.0){\rule[-0.200pt]{2.409pt}{0.400pt}}
\put(1429.0,334.0){\rule[-0.200pt]{2.409pt}{0.400pt}}
\put(170.0,334.0){\rule[-0.200pt]{2.409pt}{0.400pt}}
\put(1429.0,334.0){\rule[-0.200pt]{2.409pt}{0.400pt}}
\put(170.0,334.0){\rule[-0.200pt]{2.409pt}{0.400pt}}
\put(1429.0,334.0){\rule[-0.200pt]{2.409pt}{0.400pt}}
\put(170.0,334.0){\rule[-0.200pt]{2.409pt}{0.400pt}}
\put(1429.0,334.0){\rule[-0.200pt]{2.409pt}{0.400pt}}
\put(170.0,334.0){\rule[-0.200pt]{2.409pt}{0.400pt}}
\put(1429.0,334.0){\rule[-0.200pt]{2.409pt}{0.400pt}}
\put(170.0,334.0){\rule[-0.200pt]{2.409pt}{0.400pt}}
\put(1429.0,334.0){\rule[-0.200pt]{2.409pt}{0.400pt}}
\put(170.0,334.0){\rule[-0.200pt]{2.409pt}{0.400pt}}
\put(1429.0,334.0){\rule[-0.200pt]{2.409pt}{0.400pt}}
\put(170.0,334.0){\rule[-0.200pt]{2.409pt}{0.400pt}}
\put(1429.0,334.0){\rule[-0.200pt]{2.409pt}{0.400pt}}
\put(170.0,334.0){\rule[-0.200pt]{2.409pt}{0.400pt}}
\put(1429.0,334.0){\rule[-0.200pt]{2.409pt}{0.400pt}}
\put(170.0,334.0){\rule[-0.200pt]{2.409pt}{0.400pt}}
\put(1429.0,334.0){\rule[-0.200pt]{2.409pt}{0.400pt}}
\put(170.0,334.0){\rule[-0.200pt]{2.409pt}{0.400pt}}
\put(1429.0,334.0){\rule[-0.200pt]{2.409pt}{0.400pt}}
\put(170.0,334.0){\rule[-0.200pt]{2.409pt}{0.400pt}}
\put(1429.0,334.0){\rule[-0.200pt]{2.409pt}{0.400pt}}
\put(170.0,334.0){\rule[-0.200pt]{2.409pt}{0.400pt}}
\put(1429.0,334.0){\rule[-0.200pt]{2.409pt}{0.400pt}}
\put(170.0,335.0){\rule[-0.200pt]{2.409pt}{0.400pt}}
\put(1429.0,335.0){\rule[-0.200pt]{2.409pt}{0.400pt}}
\put(170.0,335.0){\rule[-0.200pt]{2.409pt}{0.400pt}}
\put(1429.0,335.0){\rule[-0.200pt]{2.409pt}{0.400pt}}
\put(170.0,335.0){\rule[-0.200pt]{2.409pt}{0.400pt}}
\put(1429.0,335.0){\rule[-0.200pt]{2.409pt}{0.400pt}}
\put(170.0,335.0){\rule[-0.200pt]{2.409pt}{0.400pt}}
\put(1429.0,335.0){\rule[-0.200pt]{2.409pt}{0.400pt}}
\put(170.0,335.0){\rule[-0.200pt]{2.409pt}{0.400pt}}
\put(1429.0,335.0){\rule[-0.200pt]{2.409pt}{0.400pt}}
\put(170.0,335.0){\rule[-0.200pt]{2.409pt}{0.400pt}}
\put(1429.0,335.0){\rule[-0.200pt]{2.409pt}{0.400pt}}
\put(170.0,335.0){\rule[-0.200pt]{2.409pt}{0.400pt}}
\put(1429.0,335.0){\rule[-0.200pt]{2.409pt}{0.400pt}}
\put(170.0,335.0){\rule[-0.200pt]{2.409pt}{0.400pt}}
\put(1429.0,335.0){\rule[-0.200pt]{2.409pt}{0.400pt}}
\put(170.0,335.0){\rule[-0.200pt]{2.409pt}{0.400pt}}
\put(1429.0,335.0){\rule[-0.200pt]{2.409pt}{0.400pt}}
\put(170.0,335.0){\rule[-0.200pt]{2.409pt}{0.400pt}}
\put(1429.0,335.0){\rule[-0.200pt]{2.409pt}{0.400pt}}
\put(170.0,335.0){\rule[-0.200pt]{2.409pt}{0.400pt}}
\put(1429.0,335.0){\rule[-0.200pt]{2.409pt}{0.400pt}}
\put(170.0,335.0){\rule[-0.200pt]{2.409pt}{0.400pt}}
\put(1429.0,335.0){\rule[-0.200pt]{2.409pt}{0.400pt}}
\put(170.0,335.0){\rule[-0.200pt]{2.409pt}{0.400pt}}
\put(1429.0,335.0){\rule[-0.200pt]{2.409pt}{0.400pt}}
\put(170.0,335.0){\rule[-0.200pt]{2.409pt}{0.400pt}}
\put(1429.0,335.0){\rule[-0.200pt]{2.409pt}{0.400pt}}
\put(170.0,335.0){\rule[-0.200pt]{2.409pt}{0.400pt}}
\put(1429.0,335.0){\rule[-0.200pt]{2.409pt}{0.400pt}}
\put(170.0,335.0){\rule[-0.200pt]{2.409pt}{0.400pt}}
\put(1429.0,335.0){\rule[-0.200pt]{2.409pt}{0.400pt}}
\put(170.0,335.0){\rule[-0.200pt]{2.409pt}{0.400pt}}
\put(1429.0,335.0){\rule[-0.200pt]{2.409pt}{0.400pt}}
\put(170.0,335.0){\rule[-0.200pt]{2.409pt}{0.400pt}}
\put(1429.0,335.0){\rule[-0.200pt]{2.409pt}{0.400pt}}
\put(170.0,335.0){\rule[-0.200pt]{2.409pt}{0.400pt}}
\put(1429.0,335.0){\rule[-0.200pt]{2.409pt}{0.400pt}}
\put(170.0,335.0){\rule[-0.200pt]{2.409pt}{0.400pt}}
\put(1429.0,335.0){\rule[-0.200pt]{2.409pt}{0.400pt}}
\put(170.0,335.0){\rule[-0.200pt]{2.409pt}{0.400pt}}
\put(1429.0,335.0){\rule[-0.200pt]{2.409pt}{0.400pt}}
\put(170.0,335.0){\rule[-0.200pt]{2.409pt}{0.400pt}}
\put(1429.0,335.0){\rule[-0.200pt]{2.409pt}{0.400pt}}
\put(170.0,335.0){\rule[-0.200pt]{2.409pt}{0.400pt}}
\put(1429.0,335.0){\rule[-0.200pt]{2.409pt}{0.400pt}}
\put(170.0,336.0){\rule[-0.200pt]{2.409pt}{0.400pt}}
\put(1429.0,336.0){\rule[-0.200pt]{2.409pt}{0.400pt}}
\put(170.0,336.0){\rule[-0.200pt]{2.409pt}{0.400pt}}
\put(1429.0,336.0){\rule[-0.200pt]{2.409pt}{0.400pt}}
\put(170.0,336.0){\rule[-0.200pt]{2.409pt}{0.400pt}}
\put(1429.0,336.0){\rule[-0.200pt]{2.409pt}{0.400pt}}
\put(170.0,336.0){\rule[-0.200pt]{2.409pt}{0.400pt}}
\put(1429.0,336.0){\rule[-0.200pt]{2.409pt}{0.400pt}}
\put(170.0,336.0){\rule[-0.200pt]{2.409pt}{0.400pt}}
\put(1429.0,336.0){\rule[-0.200pt]{2.409pt}{0.400pt}}
\put(170.0,336.0){\rule[-0.200pt]{2.409pt}{0.400pt}}
\put(1429.0,336.0){\rule[-0.200pt]{2.409pt}{0.400pt}}
\put(170.0,336.0){\rule[-0.200pt]{2.409pt}{0.400pt}}
\put(1429.0,336.0){\rule[-0.200pt]{2.409pt}{0.400pt}}
\put(170.0,336.0){\rule[-0.200pt]{2.409pt}{0.400pt}}
\put(1429.0,336.0){\rule[-0.200pt]{2.409pt}{0.400pt}}
\put(170.0,336.0){\rule[-0.200pt]{2.409pt}{0.400pt}}
\put(1429.0,336.0){\rule[-0.200pt]{2.409pt}{0.400pt}}
\put(170.0,336.0){\rule[-0.200pt]{2.409pt}{0.400pt}}
\put(1429.0,336.0){\rule[-0.200pt]{2.409pt}{0.400pt}}
\put(170.0,336.0){\rule[-0.200pt]{2.409pt}{0.400pt}}
\put(1429.0,336.0){\rule[-0.200pt]{2.409pt}{0.400pt}}
\put(170.0,336.0){\rule[-0.200pt]{2.409pt}{0.400pt}}
\put(1429.0,336.0){\rule[-0.200pt]{2.409pt}{0.400pt}}
\put(170.0,336.0){\rule[-0.200pt]{2.409pt}{0.400pt}}
\put(1429.0,336.0){\rule[-0.200pt]{2.409pt}{0.400pt}}
\put(170.0,336.0){\rule[-0.200pt]{2.409pt}{0.400pt}}
\put(1429.0,336.0){\rule[-0.200pt]{2.409pt}{0.400pt}}
\put(170.0,336.0){\rule[-0.200pt]{2.409pt}{0.400pt}}
\put(1429.0,336.0){\rule[-0.200pt]{2.409pt}{0.400pt}}
\put(170.0,336.0){\rule[-0.200pt]{2.409pt}{0.400pt}}
\put(1429.0,336.0){\rule[-0.200pt]{2.409pt}{0.400pt}}
\put(170.0,336.0){\rule[-0.200pt]{2.409pt}{0.400pt}}
\put(1429.0,336.0){\rule[-0.200pt]{2.409pt}{0.400pt}}
\put(170.0,336.0){\rule[-0.200pt]{2.409pt}{0.400pt}}
\put(1429.0,336.0){\rule[-0.200pt]{2.409pt}{0.400pt}}
\put(170.0,336.0){\rule[-0.200pt]{2.409pt}{0.400pt}}
\put(1429.0,336.0){\rule[-0.200pt]{2.409pt}{0.400pt}}
\put(170.0,336.0){\rule[-0.200pt]{2.409pt}{0.400pt}}
\put(1429.0,336.0){\rule[-0.200pt]{2.409pt}{0.400pt}}
\put(170.0,336.0){\rule[-0.200pt]{2.409pt}{0.400pt}}
\put(1429.0,336.0){\rule[-0.200pt]{2.409pt}{0.400pt}}
\put(170.0,336.0){\rule[-0.200pt]{2.409pt}{0.400pt}}
\put(1429.0,336.0){\rule[-0.200pt]{2.409pt}{0.400pt}}
\put(170.0,336.0){\rule[-0.200pt]{2.409pt}{0.400pt}}
\put(1429.0,336.0){\rule[-0.200pt]{2.409pt}{0.400pt}}
\put(170.0,337.0){\rule[-0.200pt]{2.409pt}{0.400pt}}
\put(1429.0,337.0){\rule[-0.200pt]{2.409pt}{0.400pt}}
\put(170.0,337.0){\rule[-0.200pt]{2.409pt}{0.400pt}}
\put(1429.0,337.0){\rule[-0.200pt]{2.409pt}{0.400pt}}
\put(170.0,337.0){\rule[-0.200pt]{2.409pt}{0.400pt}}
\put(1429.0,337.0){\rule[-0.200pt]{2.409pt}{0.400pt}}
\put(170.0,337.0){\rule[-0.200pt]{2.409pt}{0.400pt}}
\put(1429.0,337.0){\rule[-0.200pt]{2.409pt}{0.400pt}}
\put(170.0,337.0){\rule[-0.200pt]{2.409pt}{0.400pt}}
\put(1429.0,337.0){\rule[-0.200pt]{2.409pt}{0.400pt}}
\put(170.0,337.0){\rule[-0.200pt]{2.409pt}{0.400pt}}
\put(1429.0,337.0){\rule[-0.200pt]{2.409pt}{0.400pt}}
\put(170.0,337.0){\rule[-0.200pt]{2.409pt}{0.400pt}}
\put(1429.0,337.0){\rule[-0.200pt]{2.409pt}{0.400pt}}
\put(170.0,337.0){\rule[-0.200pt]{2.409pt}{0.400pt}}
\put(1429.0,337.0){\rule[-0.200pt]{2.409pt}{0.400pt}}
\put(170.0,337.0){\rule[-0.200pt]{2.409pt}{0.400pt}}
\put(1429.0,337.0){\rule[-0.200pt]{2.409pt}{0.400pt}}
\put(170.0,337.0){\rule[-0.200pt]{2.409pt}{0.400pt}}
\put(1429.0,337.0){\rule[-0.200pt]{2.409pt}{0.400pt}}
\put(170.0,337.0){\rule[-0.200pt]{2.409pt}{0.400pt}}
\put(1429.0,337.0){\rule[-0.200pt]{2.409pt}{0.400pt}}
\put(170.0,337.0){\rule[-0.200pt]{2.409pt}{0.400pt}}
\put(1429.0,337.0){\rule[-0.200pt]{2.409pt}{0.400pt}}
\put(170.0,337.0){\rule[-0.200pt]{2.409pt}{0.400pt}}
\put(1429.0,337.0){\rule[-0.200pt]{2.409pt}{0.400pt}}
\put(170.0,337.0){\rule[-0.200pt]{2.409pt}{0.400pt}}
\put(1429.0,337.0){\rule[-0.200pt]{2.409pt}{0.400pt}}
\put(170.0,337.0){\rule[-0.200pt]{2.409pt}{0.400pt}}
\put(1429.0,337.0){\rule[-0.200pt]{2.409pt}{0.400pt}}
\put(170.0,337.0){\rule[-0.200pt]{2.409pt}{0.400pt}}
\put(1429.0,337.0){\rule[-0.200pt]{2.409pt}{0.400pt}}
\put(170.0,337.0){\rule[-0.200pt]{2.409pt}{0.400pt}}
\put(1429.0,337.0){\rule[-0.200pt]{2.409pt}{0.400pt}}
\put(170.0,337.0){\rule[-0.200pt]{2.409pt}{0.400pt}}
\put(1429.0,337.0){\rule[-0.200pt]{2.409pt}{0.400pt}}
\put(170.0,337.0){\rule[-0.200pt]{2.409pt}{0.400pt}}
\put(1429.0,337.0){\rule[-0.200pt]{2.409pt}{0.400pt}}
\put(170.0,337.0){\rule[-0.200pt]{2.409pt}{0.400pt}}
\put(1429.0,337.0){\rule[-0.200pt]{2.409pt}{0.400pt}}
\put(170.0,337.0){\rule[-0.200pt]{2.409pt}{0.400pt}}
\put(1429.0,337.0){\rule[-0.200pt]{2.409pt}{0.400pt}}
\put(170.0,337.0){\rule[-0.200pt]{2.409pt}{0.400pt}}
\put(1429.0,337.0){\rule[-0.200pt]{2.409pt}{0.400pt}}
\put(170.0,337.0){\rule[-0.200pt]{2.409pt}{0.400pt}}
\put(1429.0,337.0){\rule[-0.200pt]{2.409pt}{0.400pt}}
\put(170.0,337.0){\rule[-0.200pt]{2.409pt}{0.400pt}}
\put(1429.0,337.0){\rule[-0.200pt]{2.409pt}{0.400pt}}
\put(170.0,338.0){\rule[-0.200pt]{2.409pt}{0.400pt}}
\put(1429.0,338.0){\rule[-0.200pt]{2.409pt}{0.400pt}}
\put(170.0,338.0){\rule[-0.200pt]{2.409pt}{0.400pt}}
\put(1429.0,338.0){\rule[-0.200pt]{2.409pt}{0.400pt}}
\put(170.0,338.0){\rule[-0.200pt]{2.409pt}{0.400pt}}
\put(1429.0,338.0){\rule[-0.200pt]{2.409pt}{0.400pt}}
\put(170.0,338.0){\rule[-0.200pt]{2.409pt}{0.400pt}}
\put(1429.0,338.0){\rule[-0.200pt]{2.409pt}{0.400pt}}
\put(170.0,338.0){\rule[-0.200pt]{2.409pt}{0.400pt}}
\put(1429.0,338.0){\rule[-0.200pt]{2.409pt}{0.400pt}}
\put(170.0,338.0){\rule[-0.200pt]{2.409pt}{0.400pt}}
\put(1429.0,338.0){\rule[-0.200pt]{2.409pt}{0.400pt}}
\put(170.0,338.0){\rule[-0.200pt]{2.409pt}{0.400pt}}
\put(1429.0,338.0){\rule[-0.200pt]{2.409pt}{0.400pt}}
\put(170.0,338.0){\rule[-0.200pt]{2.409pt}{0.400pt}}
\put(1429.0,338.0){\rule[-0.200pt]{2.409pt}{0.400pt}}
\put(170.0,338.0){\rule[-0.200pt]{2.409pt}{0.400pt}}
\put(1429.0,338.0){\rule[-0.200pt]{2.409pt}{0.400pt}}
\put(170.0,338.0){\rule[-0.200pt]{2.409pt}{0.400pt}}
\put(1429.0,338.0){\rule[-0.200pt]{2.409pt}{0.400pt}}
\put(170.0,338.0){\rule[-0.200pt]{2.409pt}{0.400pt}}
\put(1429.0,338.0){\rule[-0.200pt]{2.409pt}{0.400pt}}
\put(170.0,338.0){\rule[-0.200pt]{2.409pt}{0.400pt}}
\put(1429.0,338.0){\rule[-0.200pt]{2.409pt}{0.400pt}}
\put(170.0,338.0){\rule[-0.200pt]{2.409pt}{0.400pt}}
\put(1429.0,338.0){\rule[-0.200pt]{2.409pt}{0.400pt}}
\put(170.0,338.0){\rule[-0.200pt]{2.409pt}{0.400pt}}
\put(1429.0,338.0){\rule[-0.200pt]{2.409pt}{0.400pt}}
\put(170.0,338.0){\rule[-0.200pt]{2.409pt}{0.400pt}}
\put(1429.0,338.0){\rule[-0.200pt]{2.409pt}{0.400pt}}
\put(170.0,338.0){\rule[-0.200pt]{2.409pt}{0.400pt}}
\put(1429.0,338.0){\rule[-0.200pt]{2.409pt}{0.400pt}}
\put(170.0,338.0){\rule[-0.200pt]{2.409pt}{0.400pt}}
\put(1429.0,338.0){\rule[-0.200pt]{2.409pt}{0.400pt}}
\put(170.0,338.0){\rule[-0.200pt]{2.409pt}{0.400pt}}
\put(1429.0,338.0){\rule[-0.200pt]{2.409pt}{0.400pt}}
\put(170.0,338.0){\rule[-0.200pt]{2.409pt}{0.400pt}}
\put(1429.0,338.0){\rule[-0.200pt]{2.409pt}{0.400pt}}
\put(170.0,338.0){\rule[-0.200pt]{2.409pt}{0.400pt}}
\put(1429.0,338.0){\rule[-0.200pt]{2.409pt}{0.400pt}}
\put(170.0,338.0){\rule[-0.200pt]{2.409pt}{0.400pt}}
\put(1429.0,338.0){\rule[-0.200pt]{2.409pt}{0.400pt}}
\put(170.0,338.0){\rule[-0.200pt]{2.409pt}{0.400pt}}
\put(1429.0,338.0){\rule[-0.200pt]{2.409pt}{0.400pt}}
\put(170.0,338.0){\rule[-0.200pt]{2.409pt}{0.400pt}}
\put(1429.0,338.0){\rule[-0.200pt]{2.409pt}{0.400pt}}
\put(170.0,338.0){\rule[-0.200pt]{2.409pt}{0.400pt}}
\put(1429.0,338.0){\rule[-0.200pt]{2.409pt}{0.400pt}}
\put(170.0,338.0){\rule[-0.200pt]{2.409pt}{0.400pt}}
\put(1429.0,338.0){\rule[-0.200pt]{2.409pt}{0.400pt}}
\put(170.0,339.0){\rule[-0.200pt]{2.409pt}{0.400pt}}
\put(1429.0,339.0){\rule[-0.200pt]{2.409pt}{0.400pt}}
\put(170.0,339.0){\rule[-0.200pt]{2.409pt}{0.400pt}}
\put(1429.0,339.0){\rule[-0.200pt]{2.409pt}{0.400pt}}
\put(170.0,339.0){\rule[-0.200pt]{2.409pt}{0.400pt}}
\put(1429.0,339.0){\rule[-0.200pt]{2.409pt}{0.400pt}}
\put(170.0,339.0){\rule[-0.200pt]{2.409pt}{0.400pt}}
\put(1429.0,339.0){\rule[-0.200pt]{2.409pt}{0.400pt}}
\put(170.0,339.0){\rule[-0.200pt]{2.409pt}{0.400pt}}
\put(1429.0,339.0){\rule[-0.200pt]{2.409pt}{0.400pt}}
\put(170.0,339.0){\rule[-0.200pt]{2.409pt}{0.400pt}}
\put(1429.0,339.0){\rule[-0.200pt]{2.409pt}{0.400pt}}
\put(170.0,339.0){\rule[-0.200pt]{2.409pt}{0.400pt}}
\put(1429.0,339.0){\rule[-0.200pt]{2.409pt}{0.400pt}}
\put(170.0,339.0){\rule[-0.200pt]{2.409pt}{0.400pt}}
\put(1429.0,339.0){\rule[-0.200pt]{2.409pt}{0.400pt}}
\put(170.0,339.0){\rule[-0.200pt]{2.409pt}{0.400pt}}
\put(1429.0,339.0){\rule[-0.200pt]{2.409pt}{0.400pt}}
\put(170.0,339.0){\rule[-0.200pt]{2.409pt}{0.400pt}}
\put(1429.0,339.0){\rule[-0.200pt]{2.409pt}{0.400pt}}
\put(170.0,339.0){\rule[-0.200pt]{2.409pt}{0.400pt}}
\put(1429.0,339.0){\rule[-0.200pt]{2.409pt}{0.400pt}}
\put(170.0,339.0){\rule[-0.200pt]{2.409pt}{0.400pt}}
\put(1429.0,339.0){\rule[-0.200pt]{2.409pt}{0.400pt}}
\put(170.0,339.0){\rule[-0.200pt]{2.409pt}{0.400pt}}
\put(1429.0,339.0){\rule[-0.200pt]{2.409pt}{0.400pt}}
\put(170.0,339.0){\rule[-0.200pt]{2.409pt}{0.400pt}}
\put(1429.0,339.0){\rule[-0.200pt]{2.409pt}{0.400pt}}
\put(170.0,339.0){\rule[-0.200pt]{2.409pt}{0.400pt}}
\put(1429.0,339.0){\rule[-0.200pt]{2.409pt}{0.400pt}}
\put(170.0,339.0){\rule[-0.200pt]{2.409pt}{0.400pt}}
\put(1429.0,339.0){\rule[-0.200pt]{2.409pt}{0.400pt}}
\put(170.0,339.0){\rule[-0.200pt]{2.409pt}{0.400pt}}
\put(1429.0,339.0){\rule[-0.200pt]{2.409pt}{0.400pt}}
\put(170.0,339.0){\rule[-0.200pt]{2.409pt}{0.400pt}}
\put(1429.0,339.0){\rule[-0.200pt]{2.409pt}{0.400pt}}
\put(170.0,339.0){\rule[-0.200pt]{2.409pt}{0.400pt}}
\put(1429.0,339.0){\rule[-0.200pt]{2.409pt}{0.400pt}}
\put(170.0,339.0){\rule[-0.200pt]{2.409pt}{0.400pt}}
\put(1429.0,339.0){\rule[-0.200pt]{2.409pt}{0.400pt}}
\put(170.0,339.0){\rule[-0.200pt]{2.409pt}{0.400pt}}
\put(1429.0,339.0){\rule[-0.200pt]{2.409pt}{0.400pt}}
\put(170.0,339.0){\rule[-0.200pt]{2.409pt}{0.400pt}}
\put(1429.0,339.0){\rule[-0.200pt]{2.409pt}{0.400pt}}
\put(170.0,339.0){\rule[-0.200pt]{2.409pt}{0.400pt}}
\put(1429.0,339.0){\rule[-0.200pt]{2.409pt}{0.400pt}}
\put(170.0,339.0){\rule[-0.200pt]{2.409pt}{0.400pt}}
\put(1429.0,339.0){\rule[-0.200pt]{2.409pt}{0.400pt}}
\put(170.0,339.0){\rule[-0.200pt]{2.409pt}{0.400pt}}
\put(1429.0,339.0){\rule[-0.200pt]{2.409pt}{0.400pt}}
\put(170.0,340.0){\rule[-0.200pt]{2.409pt}{0.400pt}}
\put(1429.0,340.0){\rule[-0.200pt]{2.409pt}{0.400pt}}
\put(170.0,340.0){\rule[-0.200pt]{2.409pt}{0.400pt}}
\put(1429.0,340.0){\rule[-0.200pt]{2.409pt}{0.400pt}}
\put(170.0,340.0){\rule[-0.200pt]{2.409pt}{0.400pt}}
\put(1429.0,340.0){\rule[-0.200pt]{2.409pt}{0.400pt}}
\put(170.0,340.0){\rule[-0.200pt]{2.409pt}{0.400pt}}
\put(1429.0,340.0){\rule[-0.200pt]{2.409pt}{0.400pt}}
\put(170.0,340.0){\rule[-0.200pt]{2.409pt}{0.400pt}}
\put(1429.0,340.0){\rule[-0.200pt]{2.409pt}{0.400pt}}
\put(170.0,340.0){\rule[-0.200pt]{2.409pt}{0.400pt}}
\put(1429.0,340.0){\rule[-0.200pt]{2.409pt}{0.400pt}}
\put(170.0,340.0){\rule[-0.200pt]{2.409pt}{0.400pt}}
\put(1429.0,340.0){\rule[-0.200pt]{2.409pt}{0.400pt}}
\put(170.0,340.0){\rule[-0.200pt]{2.409pt}{0.400pt}}
\put(1429.0,340.0){\rule[-0.200pt]{2.409pt}{0.400pt}}
\put(170.0,340.0){\rule[-0.200pt]{2.409pt}{0.400pt}}
\put(1429.0,340.0){\rule[-0.200pt]{2.409pt}{0.400pt}}
\put(170.0,340.0){\rule[-0.200pt]{2.409pt}{0.400pt}}
\put(1429.0,340.0){\rule[-0.200pt]{2.409pt}{0.400pt}}
\put(170.0,340.0){\rule[-0.200pt]{2.409pt}{0.400pt}}
\put(1429.0,340.0){\rule[-0.200pt]{2.409pt}{0.400pt}}
\put(170.0,340.0){\rule[-0.200pt]{2.409pt}{0.400pt}}
\put(1429.0,340.0){\rule[-0.200pt]{2.409pt}{0.400pt}}
\put(170.0,340.0){\rule[-0.200pt]{2.409pt}{0.400pt}}
\put(1429.0,340.0){\rule[-0.200pt]{2.409pt}{0.400pt}}
\put(170.0,340.0){\rule[-0.200pt]{2.409pt}{0.400pt}}
\put(1429.0,340.0){\rule[-0.200pt]{2.409pt}{0.400pt}}
\put(170.0,340.0){\rule[-0.200pt]{2.409pt}{0.400pt}}
\put(1429.0,340.0){\rule[-0.200pt]{2.409pt}{0.400pt}}
\put(170.0,340.0){\rule[-0.200pt]{2.409pt}{0.400pt}}
\put(1429.0,340.0){\rule[-0.200pt]{2.409pt}{0.400pt}}
\put(170.0,340.0){\rule[-0.200pt]{2.409pt}{0.400pt}}
\put(1429.0,340.0){\rule[-0.200pt]{2.409pt}{0.400pt}}
\put(170.0,340.0){\rule[-0.200pt]{2.409pt}{0.400pt}}
\put(1429.0,340.0){\rule[-0.200pt]{2.409pt}{0.400pt}}
\put(170.0,340.0){\rule[-0.200pt]{2.409pt}{0.400pt}}
\put(1429.0,340.0){\rule[-0.200pt]{2.409pt}{0.400pt}}
\put(170.0,340.0){\rule[-0.200pt]{2.409pt}{0.400pt}}
\put(1429.0,340.0){\rule[-0.200pt]{2.409pt}{0.400pt}}
\put(170.0,340.0){\rule[-0.200pt]{2.409pt}{0.400pt}}
\put(1429.0,340.0){\rule[-0.200pt]{2.409pt}{0.400pt}}
\put(170.0,340.0){\rule[-0.200pt]{2.409pt}{0.400pt}}
\put(1429.0,340.0){\rule[-0.200pt]{2.409pt}{0.400pt}}
\put(170.0,340.0){\rule[-0.200pt]{2.409pt}{0.400pt}}
\put(1429.0,340.0){\rule[-0.200pt]{2.409pt}{0.400pt}}
\put(170.0,340.0){\rule[-0.200pt]{2.409pt}{0.400pt}}
\put(1429.0,340.0){\rule[-0.200pt]{2.409pt}{0.400pt}}
\put(170.0,340.0){\rule[-0.200pt]{2.409pt}{0.400pt}}
\put(1429.0,340.0){\rule[-0.200pt]{2.409pt}{0.400pt}}
\put(170.0,340.0){\rule[-0.200pt]{2.409pt}{0.400pt}}
\put(1429.0,340.0){\rule[-0.200pt]{2.409pt}{0.400pt}}
\put(170.0,341.0){\rule[-0.200pt]{2.409pt}{0.400pt}}
\put(1429.0,341.0){\rule[-0.200pt]{2.409pt}{0.400pt}}
\put(170.0,341.0){\rule[-0.200pt]{2.409pt}{0.400pt}}
\put(1429.0,341.0){\rule[-0.200pt]{2.409pt}{0.400pt}}
\put(170.0,341.0){\rule[-0.200pt]{2.409pt}{0.400pt}}
\put(1429.0,341.0){\rule[-0.200pt]{2.409pt}{0.400pt}}
\put(170.0,341.0){\rule[-0.200pt]{2.409pt}{0.400pt}}
\put(1429.0,341.0){\rule[-0.200pt]{2.409pt}{0.400pt}}
\put(170.0,341.0){\rule[-0.200pt]{2.409pt}{0.400pt}}
\put(1429.0,341.0){\rule[-0.200pt]{2.409pt}{0.400pt}}
\put(170.0,341.0){\rule[-0.200pt]{2.409pt}{0.400pt}}
\put(1429.0,341.0){\rule[-0.200pt]{2.409pt}{0.400pt}}
\put(170.0,341.0){\rule[-0.200pt]{2.409pt}{0.400pt}}
\put(1429.0,341.0){\rule[-0.200pt]{2.409pt}{0.400pt}}
\put(170.0,341.0){\rule[-0.200pt]{2.409pt}{0.400pt}}
\put(1429.0,341.0){\rule[-0.200pt]{2.409pt}{0.400pt}}
\put(170.0,341.0){\rule[-0.200pt]{2.409pt}{0.400pt}}
\put(1429.0,341.0){\rule[-0.200pt]{2.409pt}{0.400pt}}
\put(170.0,341.0){\rule[-0.200pt]{2.409pt}{0.400pt}}
\put(1429.0,341.0){\rule[-0.200pt]{2.409pt}{0.400pt}}
\put(170.0,341.0){\rule[-0.200pt]{2.409pt}{0.400pt}}
\put(1429.0,341.0){\rule[-0.200pt]{2.409pt}{0.400pt}}
\put(170.0,341.0){\rule[-0.200pt]{2.409pt}{0.400pt}}
\put(1429.0,341.0){\rule[-0.200pt]{2.409pt}{0.400pt}}
\put(170.0,341.0){\rule[-0.200pt]{2.409pt}{0.400pt}}
\put(1429.0,341.0){\rule[-0.200pt]{2.409pt}{0.400pt}}
\put(170.0,341.0){\rule[-0.200pt]{4.818pt}{0.400pt}}
\put(150,341){\makebox(0,0)[r]{ 1000}}
\put(1419.0,341.0){\rule[-0.200pt]{4.818pt}{0.400pt}}
\put(170.0,367.0){\rule[-0.200pt]{2.409pt}{0.400pt}}
\put(1429.0,367.0){\rule[-0.200pt]{2.409pt}{0.400pt}}
\put(170.0,382.0){\rule[-0.200pt]{2.409pt}{0.400pt}}
\put(1429.0,382.0){\rule[-0.200pt]{2.409pt}{0.400pt}}
\put(170.0,393.0){\rule[-0.200pt]{2.409pt}{0.400pt}}
\put(1429.0,393.0){\rule[-0.200pt]{2.409pt}{0.400pt}}
\put(170.0,401.0){\rule[-0.200pt]{2.409pt}{0.400pt}}
\put(1429.0,401.0){\rule[-0.200pt]{2.409pt}{0.400pt}}
\put(170.0,408.0){\rule[-0.200pt]{2.409pt}{0.400pt}}
\put(1429.0,408.0){\rule[-0.200pt]{2.409pt}{0.400pt}}
\put(170.0,414.0){\rule[-0.200pt]{2.409pt}{0.400pt}}
\put(1429.0,414.0){\rule[-0.200pt]{2.409pt}{0.400pt}}
\put(170.0,419.0){\rule[-0.200pt]{2.409pt}{0.400pt}}
\put(1429.0,419.0){\rule[-0.200pt]{2.409pt}{0.400pt}}
\put(170.0,423.0){\rule[-0.200pt]{2.409pt}{0.400pt}}
\put(1429.0,423.0){\rule[-0.200pt]{2.409pt}{0.400pt}}
\put(170.0,427.0){\rule[-0.200pt]{2.409pt}{0.400pt}}
\put(1429.0,427.0){\rule[-0.200pt]{2.409pt}{0.400pt}}
\put(170.0,431.0){\rule[-0.200pt]{2.409pt}{0.400pt}}
\put(1429.0,431.0){\rule[-0.200pt]{2.409pt}{0.400pt}}
\put(170.0,434.0){\rule[-0.200pt]{2.409pt}{0.400pt}}
\put(1429.0,434.0){\rule[-0.200pt]{2.409pt}{0.400pt}}
\put(170.0,437.0){\rule[-0.200pt]{2.409pt}{0.400pt}}
\put(1429.0,437.0){\rule[-0.200pt]{2.409pt}{0.400pt}}
\put(170.0,440.0){\rule[-0.200pt]{2.409pt}{0.400pt}}
\put(1429.0,440.0){\rule[-0.200pt]{2.409pt}{0.400pt}}
\put(170.0,443.0){\rule[-0.200pt]{2.409pt}{0.400pt}}
\put(1429.0,443.0){\rule[-0.200pt]{2.409pt}{0.400pt}}
\put(170.0,445.0){\rule[-0.200pt]{2.409pt}{0.400pt}}
\put(1429.0,445.0){\rule[-0.200pt]{2.409pt}{0.400pt}}
\put(170.0,447.0){\rule[-0.200pt]{2.409pt}{0.400pt}}
\put(1429.0,447.0){\rule[-0.200pt]{2.409pt}{0.400pt}}
\put(170.0,449.0){\rule[-0.200pt]{2.409pt}{0.400pt}}
\put(1429.0,449.0){\rule[-0.200pt]{2.409pt}{0.400pt}}
\put(170.0,451.0){\rule[-0.200pt]{2.409pt}{0.400pt}}
\put(1429.0,451.0){\rule[-0.200pt]{2.409pt}{0.400pt}}
\put(170.0,453.0){\rule[-0.200pt]{2.409pt}{0.400pt}}
\put(1429.0,453.0){\rule[-0.200pt]{2.409pt}{0.400pt}}
\put(170.0,455.0){\rule[-0.200pt]{2.409pt}{0.400pt}}
\put(1429.0,455.0){\rule[-0.200pt]{2.409pt}{0.400pt}}
\put(170.0,457.0){\rule[-0.200pt]{2.409pt}{0.400pt}}
\put(1429.0,457.0){\rule[-0.200pt]{2.409pt}{0.400pt}}
\put(170.0,459.0){\rule[-0.200pt]{2.409pt}{0.400pt}}
\put(1429.0,459.0){\rule[-0.200pt]{2.409pt}{0.400pt}}
\put(170.0,460.0){\rule[-0.200pt]{2.409pt}{0.400pt}}
\put(1429.0,460.0){\rule[-0.200pt]{2.409pt}{0.400pt}}
\put(170.0,462.0){\rule[-0.200pt]{2.409pt}{0.400pt}}
\put(1429.0,462.0){\rule[-0.200pt]{2.409pt}{0.400pt}}
\put(170.0,463.0){\rule[-0.200pt]{2.409pt}{0.400pt}}
\put(1429.0,463.0){\rule[-0.200pt]{2.409pt}{0.400pt}}
\put(170.0,465.0){\rule[-0.200pt]{2.409pt}{0.400pt}}
\put(1429.0,465.0){\rule[-0.200pt]{2.409pt}{0.400pt}}
\put(170.0,466.0){\rule[-0.200pt]{2.409pt}{0.400pt}}
\put(1429.0,466.0){\rule[-0.200pt]{2.409pt}{0.400pt}}
\put(170.0,467.0){\rule[-0.200pt]{2.409pt}{0.400pt}}
\put(1429.0,467.0){\rule[-0.200pt]{2.409pt}{0.400pt}}
\put(170.0,469.0){\rule[-0.200pt]{2.409pt}{0.400pt}}
\put(1429.0,469.0){\rule[-0.200pt]{2.409pt}{0.400pt}}
\put(170.0,470.0){\rule[-0.200pt]{2.409pt}{0.400pt}}
\put(1429.0,470.0){\rule[-0.200pt]{2.409pt}{0.400pt}}
\put(170.0,471.0){\rule[-0.200pt]{2.409pt}{0.400pt}}
\put(1429.0,471.0){\rule[-0.200pt]{2.409pt}{0.400pt}}
\put(170.0,472.0){\rule[-0.200pt]{2.409pt}{0.400pt}}
\put(1429.0,472.0){\rule[-0.200pt]{2.409pt}{0.400pt}}
\put(170.0,473.0){\rule[-0.200pt]{2.409pt}{0.400pt}}
\put(1429.0,473.0){\rule[-0.200pt]{2.409pt}{0.400pt}}
\put(170.0,474.0){\rule[-0.200pt]{2.409pt}{0.400pt}}
\put(1429.0,474.0){\rule[-0.200pt]{2.409pt}{0.400pt}}
\put(170.0,475.0){\rule[-0.200pt]{2.409pt}{0.400pt}}
\put(1429.0,475.0){\rule[-0.200pt]{2.409pt}{0.400pt}}
\put(170.0,476.0){\rule[-0.200pt]{2.409pt}{0.400pt}}
\put(1429.0,476.0){\rule[-0.200pt]{2.409pt}{0.400pt}}
\put(170.0,477.0){\rule[-0.200pt]{2.409pt}{0.400pt}}
\put(1429.0,477.0){\rule[-0.200pt]{2.409pt}{0.400pt}}
\put(170.0,478.0){\rule[-0.200pt]{2.409pt}{0.400pt}}
\put(1429.0,478.0){\rule[-0.200pt]{2.409pt}{0.400pt}}
\put(170.0,479.0){\rule[-0.200pt]{2.409pt}{0.400pt}}
\put(1429.0,479.0){\rule[-0.200pt]{2.409pt}{0.400pt}}
\put(170.0,480.0){\rule[-0.200pt]{2.409pt}{0.400pt}}
\put(1429.0,480.0){\rule[-0.200pt]{2.409pt}{0.400pt}}
\put(170.0,481.0){\rule[-0.200pt]{2.409pt}{0.400pt}}
\put(1429.0,481.0){\rule[-0.200pt]{2.409pt}{0.400pt}}
\put(170.0,482.0){\rule[-0.200pt]{2.409pt}{0.400pt}}
\put(1429.0,482.0){\rule[-0.200pt]{2.409pt}{0.400pt}}
\put(170.0,483.0){\rule[-0.200pt]{2.409pt}{0.400pt}}
\put(1429.0,483.0){\rule[-0.200pt]{2.409pt}{0.400pt}}
\put(170.0,484.0){\rule[-0.200pt]{2.409pt}{0.400pt}}
\put(1429.0,484.0){\rule[-0.200pt]{2.409pt}{0.400pt}}
\put(170.0,485.0){\rule[-0.200pt]{2.409pt}{0.400pt}}
\put(1429.0,485.0){\rule[-0.200pt]{2.409pt}{0.400pt}}
\put(170.0,485.0){\rule[-0.200pt]{2.409pt}{0.400pt}}
\put(1429.0,485.0){\rule[-0.200pt]{2.409pt}{0.400pt}}
\put(170.0,486.0){\rule[-0.200pt]{2.409pt}{0.400pt}}
\put(1429.0,486.0){\rule[-0.200pt]{2.409pt}{0.400pt}}
\put(170.0,487.0){\rule[-0.200pt]{2.409pt}{0.400pt}}
\put(1429.0,487.0){\rule[-0.200pt]{2.409pt}{0.400pt}}
\put(170.0,488.0){\rule[-0.200pt]{2.409pt}{0.400pt}}
\put(1429.0,488.0){\rule[-0.200pt]{2.409pt}{0.400pt}}
\put(170.0,488.0){\rule[-0.200pt]{2.409pt}{0.400pt}}
\put(1429.0,488.0){\rule[-0.200pt]{2.409pt}{0.400pt}}
\put(170.0,489.0){\rule[-0.200pt]{2.409pt}{0.400pt}}
\put(1429.0,489.0){\rule[-0.200pt]{2.409pt}{0.400pt}}
\put(170.0,490.0){\rule[-0.200pt]{2.409pt}{0.400pt}}
\put(1429.0,490.0){\rule[-0.200pt]{2.409pt}{0.400pt}}
\put(170.0,491.0){\rule[-0.200pt]{2.409pt}{0.400pt}}
\put(1429.0,491.0){\rule[-0.200pt]{2.409pt}{0.400pt}}
\put(170.0,491.0){\rule[-0.200pt]{2.409pt}{0.400pt}}
\put(1429.0,491.0){\rule[-0.200pt]{2.409pt}{0.400pt}}
\put(170.0,492.0){\rule[-0.200pt]{2.409pt}{0.400pt}}
\put(1429.0,492.0){\rule[-0.200pt]{2.409pt}{0.400pt}}
\put(170.0,493.0){\rule[-0.200pt]{2.409pt}{0.400pt}}
\put(1429.0,493.0){\rule[-0.200pt]{2.409pt}{0.400pt}}
\put(170.0,493.0){\rule[-0.200pt]{2.409pt}{0.400pt}}
\put(1429.0,493.0){\rule[-0.200pt]{2.409pt}{0.400pt}}
\put(170.0,494.0){\rule[-0.200pt]{2.409pt}{0.400pt}}
\put(1429.0,494.0){\rule[-0.200pt]{2.409pt}{0.400pt}}
\put(170.0,495.0){\rule[-0.200pt]{2.409pt}{0.400pt}}
\put(1429.0,495.0){\rule[-0.200pt]{2.409pt}{0.400pt}}
\put(170.0,495.0){\rule[-0.200pt]{2.409pt}{0.400pt}}
\put(1429.0,495.0){\rule[-0.200pt]{2.409pt}{0.400pt}}
\put(170.0,496.0){\rule[-0.200pt]{2.409pt}{0.400pt}}
\put(1429.0,496.0){\rule[-0.200pt]{2.409pt}{0.400pt}}
\put(170.0,496.0){\rule[-0.200pt]{2.409pt}{0.400pt}}
\put(1429.0,496.0){\rule[-0.200pt]{2.409pt}{0.400pt}}
\put(170.0,497.0){\rule[-0.200pt]{2.409pt}{0.400pt}}
\put(1429.0,497.0){\rule[-0.200pt]{2.409pt}{0.400pt}}
\put(170.0,498.0){\rule[-0.200pt]{2.409pt}{0.400pt}}
\put(1429.0,498.0){\rule[-0.200pt]{2.409pt}{0.400pt}}
\put(170.0,498.0){\rule[-0.200pt]{2.409pt}{0.400pt}}
\put(1429.0,498.0){\rule[-0.200pt]{2.409pt}{0.400pt}}
\put(170.0,499.0){\rule[-0.200pt]{2.409pt}{0.400pt}}
\put(1429.0,499.0){\rule[-0.200pt]{2.409pt}{0.400pt}}
\put(170.0,499.0){\rule[-0.200pt]{2.409pt}{0.400pt}}
\put(1429.0,499.0){\rule[-0.200pt]{2.409pt}{0.400pt}}
\put(170.0,500.0){\rule[-0.200pt]{2.409pt}{0.400pt}}
\put(1429.0,500.0){\rule[-0.200pt]{2.409pt}{0.400pt}}
\put(170.0,500.0){\rule[-0.200pt]{2.409pt}{0.400pt}}
\put(1429.0,500.0){\rule[-0.200pt]{2.409pt}{0.400pt}}
\put(170.0,501.0){\rule[-0.200pt]{2.409pt}{0.400pt}}
\put(1429.0,501.0){\rule[-0.200pt]{2.409pt}{0.400pt}}
\put(170.0,501.0){\rule[-0.200pt]{2.409pt}{0.400pt}}
\put(1429.0,501.0){\rule[-0.200pt]{2.409pt}{0.400pt}}
\put(170.0,502.0){\rule[-0.200pt]{2.409pt}{0.400pt}}
\put(1429.0,502.0){\rule[-0.200pt]{2.409pt}{0.400pt}}
\put(170.0,502.0){\rule[-0.200pt]{2.409pt}{0.400pt}}
\put(1429.0,502.0){\rule[-0.200pt]{2.409pt}{0.400pt}}
\put(170.0,503.0){\rule[-0.200pt]{2.409pt}{0.400pt}}
\put(1429.0,503.0){\rule[-0.200pt]{2.409pt}{0.400pt}}
\put(170.0,503.0){\rule[-0.200pt]{2.409pt}{0.400pt}}
\put(1429.0,503.0){\rule[-0.200pt]{2.409pt}{0.400pt}}
\put(170.0,504.0){\rule[-0.200pt]{2.409pt}{0.400pt}}
\put(1429.0,504.0){\rule[-0.200pt]{2.409pt}{0.400pt}}
\put(170.0,504.0){\rule[-0.200pt]{2.409pt}{0.400pt}}
\put(1429.0,504.0){\rule[-0.200pt]{2.409pt}{0.400pt}}
\put(170.0,505.0){\rule[-0.200pt]{2.409pt}{0.400pt}}
\put(1429.0,505.0){\rule[-0.200pt]{2.409pt}{0.400pt}}
\put(170.0,505.0){\rule[-0.200pt]{2.409pt}{0.400pt}}
\put(1429.0,505.0){\rule[-0.200pt]{2.409pt}{0.400pt}}
\put(170.0,506.0){\rule[-0.200pt]{2.409pt}{0.400pt}}
\put(1429.0,506.0){\rule[-0.200pt]{2.409pt}{0.400pt}}
\put(170.0,506.0){\rule[-0.200pt]{2.409pt}{0.400pt}}
\put(1429.0,506.0){\rule[-0.200pt]{2.409pt}{0.400pt}}
\put(170.0,507.0){\rule[-0.200pt]{2.409pt}{0.400pt}}
\put(1429.0,507.0){\rule[-0.200pt]{2.409pt}{0.400pt}}
\put(170.0,507.0){\rule[-0.200pt]{2.409pt}{0.400pt}}
\put(1429.0,507.0){\rule[-0.200pt]{2.409pt}{0.400pt}}
\put(170.0,508.0){\rule[-0.200pt]{2.409pt}{0.400pt}}
\put(1429.0,508.0){\rule[-0.200pt]{2.409pt}{0.400pt}}
\put(170.0,508.0){\rule[-0.200pt]{2.409pt}{0.400pt}}
\put(1429.0,508.0){\rule[-0.200pt]{2.409pt}{0.400pt}}
\put(170.0,508.0){\rule[-0.200pt]{2.409pt}{0.400pt}}
\put(1429.0,508.0){\rule[-0.200pt]{2.409pt}{0.400pt}}
\put(170.0,509.0){\rule[-0.200pt]{2.409pt}{0.400pt}}
\put(1429.0,509.0){\rule[-0.200pt]{2.409pt}{0.400pt}}
\put(170.0,509.0){\rule[-0.200pt]{2.409pt}{0.400pt}}
\put(1429.0,509.0){\rule[-0.200pt]{2.409pt}{0.400pt}}
\put(170.0,510.0){\rule[-0.200pt]{2.409pt}{0.400pt}}
\put(1429.0,510.0){\rule[-0.200pt]{2.409pt}{0.400pt}}
\put(170.0,510.0){\rule[-0.200pt]{2.409pt}{0.400pt}}
\put(1429.0,510.0){\rule[-0.200pt]{2.409pt}{0.400pt}}
\put(170.0,511.0){\rule[-0.200pt]{2.409pt}{0.400pt}}
\put(1429.0,511.0){\rule[-0.200pt]{2.409pt}{0.400pt}}
\put(170.0,511.0){\rule[-0.200pt]{2.409pt}{0.400pt}}
\put(1429.0,511.0){\rule[-0.200pt]{2.409pt}{0.400pt}}
\put(170.0,511.0){\rule[-0.200pt]{2.409pt}{0.400pt}}
\put(1429.0,511.0){\rule[-0.200pt]{2.409pt}{0.400pt}}
\put(170.0,512.0){\rule[-0.200pt]{2.409pt}{0.400pt}}
\put(1429.0,512.0){\rule[-0.200pt]{2.409pt}{0.400pt}}
\put(170.0,512.0){\rule[-0.200pt]{2.409pt}{0.400pt}}
\put(1429.0,512.0){\rule[-0.200pt]{2.409pt}{0.400pt}}
\put(170.0,513.0){\rule[-0.200pt]{2.409pt}{0.400pt}}
\put(1429.0,513.0){\rule[-0.200pt]{2.409pt}{0.400pt}}
\put(170.0,513.0){\rule[-0.200pt]{2.409pt}{0.400pt}}
\put(1429.0,513.0){\rule[-0.200pt]{2.409pt}{0.400pt}}
\put(170.0,513.0){\rule[-0.200pt]{2.409pt}{0.400pt}}
\put(1429.0,513.0){\rule[-0.200pt]{2.409pt}{0.400pt}}
\put(170.0,514.0){\rule[-0.200pt]{2.409pt}{0.400pt}}
\put(1429.0,514.0){\rule[-0.200pt]{2.409pt}{0.400pt}}
\put(170.0,514.0){\rule[-0.200pt]{2.409pt}{0.400pt}}
\put(1429.0,514.0){\rule[-0.200pt]{2.409pt}{0.400pt}}
\put(170.0,514.0){\rule[-0.200pt]{2.409pt}{0.400pt}}
\put(1429.0,514.0){\rule[-0.200pt]{2.409pt}{0.400pt}}
\put(170.0,515.0){\rule[-0.200pt]{2.409pt}{0.400pt}}
\put(1429.0,515.0){\rule[-0.200pt]{2.409pt}{0.400pt}}
\put(170.0,515.0){\rule[-0.200pt]{2.409pt}{0.400pt}}
\put(1429.0,515.0){\rule[-0.200pt]{2.409pt}{0.400pt}}
\put(170.0,515.0){\rule[-0.200pt]{2.409pt}{0.400pt}}
\put(1429.0,515.0){\rule[-0.200pt]{2.409pt}{0.400pt}}
\put(170.0,516.0){\rule[-0.200pt]{2.409pt}{0.400pt}}
\put(1429.0,516.0){\rule[-0.200pt]{2.409pt}{0.400pt}}
\put(170.0,516.0){\rule[-0.200pt]{2.409pt}{0.400pt}}
\put(1429.0,516.0){\rule[-0.200pt]{2.409pt}{0.400pt}}
\put(170.0,517.0){\rule[-0.200pt]{2.409pt}{0.400pt}}
\put(1429.0,517.0){\rule[-0.200pt]{2.409pt}{0.400pt}}
\put(170.0,517.0){\rule[-0.200pt]{2.409pt}{0.400pt}}
\put(1429.0,517.0){\rule[-0.200pt]{2.409pt}{0.400pt}}
\put(170.0,517.0){\rule[-0.200pt]{2.409pt}{0.400pt}}
\put(1429.0,517.0){\rule[-0.200pt]{2.409pt}{0.400pt}}
\put(170.0,518.0){\rule[-0.200pt]{2.409pt}{0.400pt}}
\put(1429.0,518.0){\rule[-0.200pt]{2.409pt}{0.400pt}}
\put(170.0,518.0){\rule[-0.200pt]{2.409pt}{0.400pt}}
\put(1429.0,518.0){\rule[-0.200pt]{2.409pt}{0.400pt}}
\put(170.0,518.0){\rule[-0.200pt]{2.409pt}{0.400pt}}
\put(1429.0,518.0){\rule[-0.200pt]{2.409pt}{0.400pt}}
\put(170.0,519.0){\rule[-0.200pt]{2.409pt}{0.400pt}}
\put(1429.0,519.0){\rule[-0.200pt]{2.409pt}{0.400pt}}
\put(170.0,519.0){\rule[-0.200pt]{2.409pt}{0.400pt}}
\put(1429.0,519.0){\rule[-0.200pt]{2.409pt}{0.400pt}}
\put(170.0,519.0){\rule[-0.200pt]{2.409pt}{0.400pt}}
\put(1429.0,519.0){\rule[-0.200pt]{2.409pt}{0.400pt}}
\put(170.0,520.0){\rule[-0.200pt]{2.409pt}{0.400pt}}
\put(1429.0,520.0){\rule[-0.200pt]{2.409pt}{0.400pt}}
\put(170.0,520.0){\rule[-0.200pt]{2.409pt}{0.400pt}}
\put(1429.0,520.0){\rule[-0.200pt]{2.409pt}{0.400pt}}
\put(170.0,520.0){\rule[-0.200pt]{2.409pt}{0.400pt}}
\put(1429.0,520.0){\rule[-0.200pt]{2.409pt}{0.400pt}}
\put(170.0,521.0){\rule[-0.200pt]{2.409pt}{0.400pt}}
\put(1429.0,521.0){\rule[-0.200pt]{2.409pt}{0.400pt}}
\put(170.0,521.0){\rule[-0.200pt]{2.409pt}{0.400pt}}
\put(1429.0,521.0){\rule[-0.200pt]{2.409pt}{0.400pt}}
\put(170.0,521.0){\rule[-0.200pt]{2.409pt}{0.400pt}}
\put(1429.0,521.0){\rule[-0.200pt]{2.409pt}{0.400pt}}
\put(170.0,521.0){\rule[-0.200pt]{2.409pt}{0.400pt}}
\put(1429.0,521.0){\rule[-0.200pt]{2.409pt}{0.400pt}}
\put(170.0,522.0){\rule[-0.200pt]{2.409pt}{0.400pt}}
\put(1429.0,522.0){\rule[-0.200pt]{2.409pt}{0.400pt}}
\put(170.0,522.0){\rule[-0.200pt]{2.409pt}{0.400pt}}
\put(1429.0,522.0){\rule[-0.200pt]{2.409pt}{0.400pt}}
\put(170.0,522.0){\rule[-0.200pt]{2.409pt}{0.400pt}}
\put(1429.0,522.0){\rule[-0.200pt]{2.409pt}{0.400pt}}
\put(170.0,523.0){\rule[-0.200pt]{2.409pt}{0.400pt}}
\put(1429.0,523.0){\rule[-0.200pt]{2.409pt}{0.400pt}}
\put(170.0,523.0){\rule[-0.200pt]{2.409pt}{0.400pt}}
\put(1429.0,523.0){\rule[-0.200pt]{2.409pt}{0.400pt}}
\put(170.0,523.0){\rule[-0.200pt]{2.409pt}{0.400pt}}
\put(1429.0,523.0){\rule[-0.200pt]{2.409pt}{0.400pt}}
\put(170.0,524.0){\rule[-0.200pt]{2.409pt}{0.400pt}}
\put(1429.0,524.0){\rule[-0.200pt]{2.409pt}{0.400pt}}
\put(170.0,524.0){\rule[-0.200pt]{2.409pt}{0.400pt}}
\put(1429.0,524.0){\rule[-0.200pt]{2.409pt}{0.400pt}}
\put(170.0,524.0){\rule[-0.200pt]{2.409pt}{0.400pt}}
\put(1429.0,524.0){\rule[-0.200pt]{2.409pt}{0.400pt}}
\put(170.0,524.0){\rule[-0.200pt]{2.409pt}{0.400pt}}
\put(1429.0,524.0){\rule[-0.200pt]{2.409pt}{0.400pt}}
\put(170.0,525.0){\rule[-0.200pt]{2.409pt}{0.400pt}}
\put(1429.0,525.0){\rule[-0.200pt]{2.409pt}{0.400pt}}
\put(170.0,525.0){\rule[-0.200pt]{2.409pt}{0.400pt}}
\put(1429.0,525.0){\rule[-0.200pt]{2.409pt}{0.400pt}}
\put(170.0,525.0){\rule[-0.200pt]{2.409pt}{0.400pt}}
\put(1429.0,525.0){\rule[-0.200pt]{2.409pt}{0.400pt}}
\put(170.0,525.0){\rule[-0.200pt]{2.409pt}{0.400pt}}
\put(1429.0,525.0){\rule[-0.200pt]{2.409pt}{0.400pt}}
\put(170.0,526.0){\rule[-0.200pt]{2.409pt}{0.400pt}}
\put(1429.0,526.0){\rule[-0.200pt]{2.409pt}{0.400pt}}
\put(170.0,526.0){\rule[-0.200pt]{2.409pt}{0.400pt}}
\put(1429.0,526.0){\rule[-0.200pt]{2.409pt}{0.400pt}}
\put(170.0,526.0){\rule[-0.200pt]{2.409pt}{0.400pt}}
\put(1429.0,526.0){\rule[-0.200pt]{2.409pt}{0.400pt}}
\put(170.0,527.0){\rule[-0.200pt]{2.409pt}{0.400pt}}
\put(1429.0,527.0){\rule[-0.200pt]{2.409pt}{0.400pt}}
\put(170.0,527.0){\rule[-0.200pt]{2.409pt}{0.400pt}}
\put(1429.0,527.0){\rule[-0.200pt]{2.409pt}{0.400pt}}
\put(170.0,527.0){\rule[-0.200pt]{2.409pt}{0.400pt}}
\put(1429.0,527.0){\rule[-0.200pt]{2.409pt}{0.400pt}}
\put(170.0,527.0){\rule[-0.200pt]{2.409pt}{0.400pt}}
\put(1429.0,527.0){\rule[-0.200pt]{2.409pt}{0.400pt}}
\put(170.0,528.0){\rule[-0.200pt]{2.409pt}{0.400pt}}
\put(1429.0,528.0){\rule[-0.200pt]{2.409pt}{0.400pt}}
\put(170.0,528.0){\rule[-0.200pt]{2.409pt}{0.400pt}}
\put(1429.0,528.0){\rule[-0.200pt]{2.409pt}{0.400pt}}
\put(170.0,528.0){\rule[-0.200pt]{2.409pt}{0.400pt}}
\put(1429.0,528.0){\rule[-0.200pt]{2.409pt}{0.400pt}}
\put(170.0,528.0){\rule[-0.200pt]{2.409pt}{0.400pt}}
\put(1429.0,528.0){\rule[-0.200pt]{2.409pt}{0.400pt}}
\put(170.0,529.0){\rule[-0.200pt]{2.409pt}{0.400pt}}
\put(1429.0,529.0){\rule[-0.200pt]{2.409pt}{0.400pt}}
\put(170.0,529.0){\rule[-0.200pt]{2.409pt}{0.400pt}}
\put(1429.0,529.0){\rule[-0.200pt]{2.409pt}{0.400pt}}
\put(170.0,529.0){\rule[-0.200pt]{2.409pt}{0.400pt}}
\put(1429.0,529.0){\rule[-0.200pt]{2.409pt}{0.400pt}}
\put(170.0,529.0){\rule[-0.200pt]{2.409pt}{0.400pt}}
\put(1429.0,529.0){\rule[-0.200pt]{2.409pt}{0.400pt}}
\put(170.0,530.0){\rule[-0.200pt]{2.409pt}{0.400pt}}
\put(1429.0,530.0){\rule[-0.200pt]{2.409pt}{0.400pt}}
\put(170.0,530.0){\rule[-0.200pt]{2.409pt}{0.400pt}}
\put(1429.0,530.0){\rule[-0.200pt]{2.409pt}{0.400pt}}
\put(170.0,530.0){\rule[-0.200pt]{2.409pt}{0.400pt}}
\put(1429.0,530.0){\rule[-0.200pt]{2.409pt}{0.400pt}}
\put(170.0,530.0){\rule[-0.200pt]{2.409pt}{0.400pt}}
\put(1429.0,530.0){\rule[-0.200pt]{2.409pt}{0.400pt}}
\put(170.0,531.0){\rule[-0.200pt]{2.409pt}{0.400pt}}
\put(1429.0,531.0){\rule[-0.200pt]{2.409pt}{0.400pt}}
\put(170.0,531.0){\rule[-0.200pt]{2.409pt}{0.400pt}}
\put(1429.0,531.0){\rule[-0.200pt]{2.409pt}{0.400pt}}
\put(170.0,531.0){\rule[-0.200pt]{2.409pt}{0.400pt}}
\put(1429.0,531.0){\rule[-0.200pt]{2.409pt}{0.400pt}}
\put(170.0,531.0){\rule[-0.200pt]{2.409pt}{0.400pt}}
\put(1429.0,531.0){\rule[-0.200pt]{2.409pt}{0.400pt}}
\put(170.0,532.0){\rule[-0.200pt]{2.409pt}{0.400pt}}
\put(1429.0,532.0){\rule[-0.200pt]{2.409pt}{0.400pt}}
\put(170.0,532.0){\rule[-0.200pt]{2.409pt}{0.400pt}}
\put(1429.0,532.0){\rule[-0.200pt]{2.409pt}{0.400pt}}
\put(170.0,532.0){\rule[-0.200pt]{2.409pt}{0.400pt}}
\put(1429.0,532.0){\rule[-0.200pt]{2.409pt}{0.400pt}}
\put(170.0,532.0){\rule[-0.200pt]{2.409pt}{0.400pt}}
\put(1429.0,532.0){\rule[-0.200pt]{2.409pt}{0.400pt}}
\put(170.0,532.0){\rule[-0.200pt]{2.409pt}{0.400pt}}
\put(1429.0,532.0){\rule[-0.200pt]{2.409pt}{0.400pt}}
\put(170.0,533.0){\rule[-0.200pt]{2.409pt}{0.400pt}}
\put(1429.0,533.0){\rule[-0.200pt]{2.409pt}{0.400pt}}
\put(170.0,533.0){\rule[-0.200pt]{2.409pt}{0.400pt}}
\put(1429.0,533.0){\rule[-0.200pt]{2.409pt}{0.400pt}}
\put(170.0,533.0){\rule[-0.200pt]{2.409pt}{0.400pt}}
\put(1429.0,533.0){\rule[-0.200pt]{2.409pt}{0.400pt}}
\put(170.0,533.0){\rule[-0.200pt]{2.409pt}{0.400pt}}
\put(1429.0,533.0){\rule[-0.200pt]{2.409pt}{0.400pt}}
\put(170.0,534.0){\rule[-0.200pt]{2.409pt}{0.400pt}}
\put(1429.0,534.0){\rule[-0.200pt]{2.409pt}{0.400pt}}
\put(170.0,534.0){\rule[-0.200pt]{2.409pt}{0.400pt}}
\put(1429.0,534.0){\rule[-0.200pt]{2.409pt}{0.400pt}}
\put(170.0,534.0){\rule[-0.200pt]{2.409pt}{0.400pt}}
\put(1429.0,534.0){\rule[-0.200pt]{2.409pt}{0.400pt}}
\put(170.0,534.0){\rule[-0.200pt]{2.409pt}{0.400pt}}
\put(1429.0,534.0){\rule[-0.200pt]{2.409pt}{0.400pt}}
\put(170.0,534.0){\rule[-0.200pt]{2.409pt}{0.400pt}}
\put(1429.0,534.0){\rule[-0.200pt]{2.409pt}{0.400pt}}
\put(170.0,535.0){\rule[-0.200pt]{2.409pt}{0.400pt}}
\put(1429.0,535.0){\rule[-0.200pt]{2.409pt}{0.400pt}}
\put(170.0,535.0){\rule[-0.200pt]{2.409pt}{0.400pt}}
\put(1429.0,535.0){\rule[-0.200pt]{2.409pt}{0.400pt}}
\put(170.0,535.0){\rule[-0.200pt]{2.409pt}{0.400pt}}
\put(1429.0,535.0){\rule[-0.200pt]{2.409pt}{0.400pt}}
\put(170.0,535.0){\rule[-0.200pt]{2.409pt}{0.400pt}}
\put(1429.0,535.0){\rule[-0.200pt]{2.409pt}{0.400pt}}
\put(170.0,535.0){\rule[-0.200pt]{2.409pt}{0.400pt}}
\put(1429.0,535.0){\rule[-0.200pt]{2.409pt}{0.400pt}}
\put(170.0,536.0){\rule[-0.200pt]{2.409pt}{0.400pt}}
\put(1429.0,536.0){\rule[-0.200pt]{2.409pt}{0.400pt}}
\put(170.0,536.0){\rule[-0.200pt]{2.409pt}{0.400pt}}
\put(1429.0,536.0){\rule[-0.200pt]{2.409pt}{0.400pt}}
\put(170.0,536.0){\rule[-0.200pt]{2.409pt}{0.400pt}}
\put(1429.0,536.0){\rule[-0.200pt]{2.409pt}{0.400pt}}
\put(170.0,536.0){\rule[-0.200pt]{2.409pt}{0.400pt}}
\put(1429.0,536.0){\rule[-0.200pt]{2.409pt}{0.400pt}}
\put(170.0,537.0){\rule[-0.200pt]{2.409pt}{0.400pt}}
\put(1429.0,537.0){\rule[-0.200pt]{2.409pt}{0.400pt}}
\put(170.0,537.0){\rule[-0.200pt]{2.409pt}{0.400pt}}
\put(1429.0,537.0){\rule[-0.200pt]{2.409pt}{0.400pt}}
\put(170.0,537.0){\rule[-0.200pt]{2.409pt}{0.400pt}}
\put(1429.0,537.0){\rule[-0.200pt]{2.409pt}{0.400pt}}
\put(170.0,537.0){\rule[-0.200pt]{2.409pt}{0.400pt}}
\put(1429.0,537.0){\rule[-0.200pt]{2.409pt}{0.400pt}}
\put(170.0,537.0){\rule[-0.200pt]{2.409pt}{0.400pt}}
\put(1429.0,537.0){\rule[-0.200pt]{2.409pt}{0.400pt}}
\put(170.0,538.0){\rule[-0.200pt]{2.409pt}{0.400pt}}
\put(1429.0,538.0){\rule[-0.200pt]{2.409pt}{0.400pt}}
\put(170.0,538.0){\rule[-0.200pt]{2.409pt}{0.400pt}}
\put(1429.0,538.0){\rule[-0.200pt]{2.409pt}{0.400pt}}
\put(170.0,538.0){\rule[-0.200pt]{2.409pt}{0.400pt}}
\put(1429.0,538.0){\rule[-0.200pt]{2.409pt}{0.400pt}}
\put(170.0,538.0){\rule[-0.200pt]{2.409pt}{0.400pt}}
\put(1429.0,538.0){\rule[-0.200pt]{2.409pt}{0.400pt}}
\put(170.0,538.0){\rule[-0.200pt]{2.409pt}{0.400pt}}
\put(1429.0,538.0){\rule[-0.200pt]{2.409pt}{0.400pt}}
\put(170.0,539.0){\rule[-0.200pt]{2.409pt}{0.400pt}}
\put(1429.0,539.0){\rule[-0.200pt]{2.409pt}{0.400pt}}
\put(170.0,539.0){\rule[-0.200pt]{2.409pt}{0.400pt}}
\put(1429.0,539.0){\rule[-0.200pt]{2.409pt}{0.400pt}}
\put(170.0,539.0){\rule[-0.200pt]{2.409pt}{0.400pt}}
\put(1429.0,539.0){\rule[-0.200pt]{2.409pt}{0.400pt}}
\put(170.0,539.0){\rule[-0.200pt]{2.409pt}{0.400pt}}
\put(1429.0,539.0){\rule[-0.200pt]{2.409pt}{0.400pt}}
\put(170.0,539.0){\rule[-0.200pt]{2.409pt}{0.400pt}}
\put(1429.0,539.0){\rule[-0.200pt]{2.409pt}{0.400pt}}
\put(170.0,539.0){\rule[-0.200pt]{2.409pt}{0.400pt}}
\put(1429.0,539.0){\rule[-0.200pt]{2.409pt}{0.400pt}}
\put(170.0,540.0){\rule[-0.200pt]{2.409pt}{0.400pt}}
\put(1429.0,540.0){\rule[-0.200pt]{2.409pt}{0.400pt}}
\put(170.0,540.0){\rule[-0.200pt]{2.409pt}{0.400pt}}
\put(1429.0,540.0){\rule[-0.200pt]{2.409pt}{0.400pt}}
\put(170.0,540.0){\rule[-0.200pt]{2.409pt}{0.400pt}}
\put(1429.0,540.0){\rule[-0.200pt]{2.409pt}{0.400pt}}
\put(170.0,540.0){\rule[-0.200pt]{2.409pt}{0.400pt}}
\put(1429.0,540.0){\rule[-0.200pt]{2.409pt}{0.400pt}}
\put(170.0,540.0){\rule[-0.200pt]{2.409pt}{0.400pt}}
\put(1429.0,540.0){\rule[-0.200pt]{2.409pt}{0.400pt}}
\put(170.0,541.0){\rule[-0.200pt]{2.409pt}{0.400pt}}
\put(1429.0,541.0){\rule[-0.200pt]{2.409pt}{0.400pt}}
\put(170.0,541.0){\rule[-0.200pt]{2.409pt}{0.400pt}}
\put(1429.0,541.0){\rule[-0.200pt]{2.409pt}{0.400pt}}
\put(170.0,541.0){\rule[-0.200pt]{2.409pt}{0.400pt}}
\put(1429.0,541.0){\rule[-0.200pt]{2.409pt}{0.400pt}}
\put(170.0,541.0){\rule[-0.200pt]{2.409pt}{0.400pt}}
\put(1429.0,541.0){\rule[-0.200pt]{2.409pt}{0.400pt}}
\put(170.0,541.0){\rule[-0.200pt]{2.409pt}{0.400pt}}
\put(1429.0,541.0){\rule[-0.200pt]{2.409pt}{0.400pt}}
\put(170.0,541.0){\rule[-0.200pt]{2.409pt}{0.400pt}}
\put(1429.0,541.0){\rule[-0.200pt]{2.409pt}{0.400pt}}
\put(170.0,542.0){\rule[-0.200pt]{2.409pt}{0.400pt}}
\put(1429.0,542.0){\rule[-0.200pt]{2.409pt}{0.400pt}}
\put(170.0,542.0){\rule[-0.200pt]{2.409pt}{0.400pt}}
\put(1429.0,542.0){\rule[-0.200pt]{2.409pt}{0.400pt}}
\put(170.0,542.0){\rule[-0.200pt]{2.409pt}{0.400pt}}
\put(1429.0,542.0){\rule[-0.200pt]{2.409pt}{0.400pt}}
\put(170.0,542.0){\rule[-0.200pt]{2.409pt}{0.400pt}}
\put(1429.0,542.0){\rule[-0.200pt]{2.409pt}{0.400pt}}
\put(170.0,542.0){\rule[-0.200pt]{2.409pt}{0.400pt}}
\put(1429.0,542.0){\rule[-0.200pt]{2.409pt}{0.400pt}}
\put(170.0,543.0){\rule[-0.200pt]{2.409pt}{0.400pt}}
\put(1429.0,543.0){\rule[-0.200pt]{2.409pt}{0.400pt}}
\put(170.0,543.0){\rule[-0.200pt]{2.409pt}{0.400pt}}
\put(1429.0,543.0){\rule[-0.200pt]{2.409pt}{0.400pt}}
\put(170.0,543.0){\rule[-0.200pt]{2.409pt}{0.400pt}}
\put(1429.0,543.0){\rule[-0.200pt]{2.409pt}{0.400pt}}
\put(170.0,543.0){\rule[-0.200pt]{2.409pt}{0.400pt}}
\put(1429.0,543.0){\rule[-0.200pt]{2.409pt}{0.400pt}}
\put(170.0,543.0){\rule[-0.200pt]{2.409pt}{0.400pt}}
\put(1429.0,543.0){\rule[-0.200pt]{2.409pt}{0.400pt}}
\put(170.0,543.0){\rule[-0.200pt]{2.409pt}{0.400pt}}
\put(1429.0,543.0){\rule[-0.200pt]{2.409pt}{0.400pt}}
\put(170.0,544.0){\rule[-0.200pt]{2.409pt}{0.400pt}}
\put(1429.0,544.0){\rule[-0.200pt]{2.409pt}{0.400pt}}
\put(170.0,544.0){\rule[-0.200pt]{2.409pt}{0.400pt}}
\put(1429.0,544.0){\rule[-0.200pt]{2.409pt}{0.400pt}}
\put(170.0,544.0){\rule[-0.200pt]{2.409pt}{0.400pt}}
\put(1429.0,544.0){\rule[-0.200pt]{2.409pt}{0.400pt}}
\put(170.0,544.0){\rule[-0.200pt]{2.409pt}{0.400pt}}
\put(1429.0,544.0){\rule[-0.200pt]{2.409pt}{0.400pt}}
\put(170.0,544.0){\rule[-0.200pt]{2.409pt}{0.400pt}}
\put(1429.0,544.0){\rule[-0.200pt]{2.409pt}{0.400pt}}
\put(170.0,544.0){\rule[-0.200pt]{2.409pt}{0.400pt}}
\put(1429.0,544.0){\rule[-0.200pt]{2.409pt}{0.400pt}}
\put(170.0,545.0){\rule[-0.200pt]{2.409pt}{0.400pt}}
\put(1429.0,545.0){\rule[-0.200pt]{2.409pt}{0.400pt}}
\put(170.0,545.0){\rule[-0.200pt]{2.409pt}{0.400pt}}
\put(1429.0,545.0){\rule[-0.200pt]{2.409pt}{0.400pt}}
\put(170.0,545.0){\rule[-0.200pt]{2.409pt}{0.400pt}}
\put(1429.0,545.0){\rule[-0.200pt]{2.409pt}{0.400pt}}
\put(170.0,545.0){\rule[-0.200pt]{2.409pt}{0.400pt}}
\put(1429.0,545.0){\rule[-0.200pt]{2.409pt}{0.400pt}}
\put(170.0,545.0){\rule[-0.200pt]{2.409pt}{0.400pt}}
\put(1429.0,545.0){\rule[-0.200pt]{2.409pt}{0.400pt}}
\put(170.0,545.0){\rule[-0.200pt]{2.409pt}{0.400pt}}
\put(1429.0,545.0){\rule[-0.200pt]{2.409pt}{0.400pt}}
\put(170.0,546.0){\rule[-0.200pt]{2.409pt}{0.400pt}}
\put(1429.0,546.0){\rule[-0.200pt]{2.409pt}{0.400pt}}
\put(170.0,546.0){\rule[-0.200pt]{2.409pt}{0.400pt}}
\put(1429.0,546.0){\rule[-0.200pt]{2.409pt}{0.400pt}}
\put(170.0,546.0){\rule[-0.200pt]{2.409pt}{0.400pt}}
\put(1429.0,546.0){\rule[-0.200pt]{2.409pt}{0.400pt}}
\put(170.0,546.0){\rule[-0.200pt]{2.409pt}{0.400pt}}
\put(1429.0,546.0){\rule[-0.200pt]{2.409pt}{0.400pt}}
\put(170.0,546.0){\rule[-0.200pt]{2.409pt}{0.400pt}}
\put(1429.0,546.0){\rule[-0.200pt]{2.409pt}{0.400pt}}
\put(170.0,546.0){\rule[-0.200pt]{2.409pt}{0.400pt}}
\put(1429.0,546.0){\rule[-0.200pt]{2.409pt}{0.400pt}}
\put(170.0,546.0){\rule[-0.200pt]{2.409pt}{0.400pt}}
\put(1429.0,546.0){\rule[-0.200pt]{2.409pt}{0.400pt}}
\put(170.0,547.0){\rule[-0.200pt]{2.409pt}{0.400pt}}
\put(1429.0,547.0){\rule[-0.200pt]{2.409pt}{0.400pt}}
\put(170.0,547.0){\rule[-0.200pt]{2.409pt}{0.400pt}}
\put(1429.0,547.0){\rule[-0.200pt]{2.409pt}{0.400pt}}
\put(170.0,547.0){\rule[-0.200pt]{2.409pt}{0.400pt}}
\put(1429.0,547.0){\rule[-0.200pt]{2.409pt}{0.400pt}}
\put(170.0,547.0){\rule[-0.200pt]{2.409pt}{0.400pt}}
\put(1429.0,547.0){\rule[-0.200pt]{2.409pt}{0.400pt}}
\put(170.0,547.0){\rule[-0.200pt]{2.409pt}{0.400pt}}
\put(1429.0,547.0){\rule[-0.200pt]{2.409pt}{0.400pt}}
\put(170.0,547.0){\rule[-0.200pt]{2.409pt}{0.400pt}}
\put(1429.0,547.0){\rule[-0.200pt]{2.409pt}{0.400pt}}
\put(170.0,548.0){\rule[-0.200pt]{2.409pt}{0.400pt}}
\put(1429.0,548.0){\rule[-0.200pt]{2.409pt}{0.400pt}}
\put(170.0,548.0){\rule[-0.200pt]{2.409pt}{0.400pt}}
\put(1429.0,548.0){\rule[-0.200pt]{2.409pt}{0.400pt}}
\put(170.0,548.0){\rule[-0.200pt]{2.409pt}{0.400pt}}
\put(1429.0,548.0){\rule[-0.200pt]{2.409pt}{0.400pt}}
\put(170.0,548.0){\rule[-0.200pt]{2.409pt}{0.400pt}}
\put(1429.0,548.0){\rule[-0.200pt]{2.409pt}{0.400pt}}
\put(170.0,548.0){\rule[-0.200pt]{2.409pt}{0.400pt}}
\put(1429.0,548.0){\rule[-0.200pt]{2.409pt}{0.400pt}}
\put(170.0,548.0){\rule[-0.200pt]{2.409pt}{0.400pt}}
\put(1429.0,548.0){\rule[-0.200pt]{2.409pt}{0.400pt}}
\put(170.0,548.0){\rule[-0.200pt]{2.409pt}{0.400pt}}
\put(1429.0,548.0){\rule[-0.200pt]{2.409pt}{0.400pt}}
\put(170.0,549.0){\rule[-0.200pt]{2.409pt}{0.400pt}}
\put(1429.0,549.0){\rule[-0.200pt]{2.409pt}{0.400pt}}
\put(170.0,549.0){\rule[-0.200pt]{2.409pt}{0.400pt}}
\put(1429.0,549.0){\rule[-0.200pt]{2.409pt}{0.400pt}}
\put(170.0,549.0){\rule[-0.200pt]{2.409pt}{0.400pt}}
\put(1429.0,549.0){\rule[-0.200pt]{2.409pt}{0.400pt}}
\put(170.0,549.0){\rule[-0.200pt]{2.409pt}{0.400pt}}
\put(1429.0,549.0){\rule[-0.200pt]{2.409pt}{0.400pt}}
\put(170.0,549.0){\rule[-0.200pt]{2.409pt}{0.400pt}}
\put(1429.0,549.0){\rule[-0.200pt]{2.409pt}{0.400pt}}
\put(170.0,549.0){\rule[-0.200pt]{2.409pt}{0.400pt}}
\put(1429.0,549.0){\rule[-0.200pt]{2.409pt}{0.400pt}}
\put(170.0,549.0){\rule[-0.200pt]{2.409pt}{0.400pt}}
\put(1429.0,549.0){\rule[-0.200pt]{2.409pt}{0.400pt}}
\put(170.0,550.0){\rule[-0.200pt]{2.409pt}{0.400pt}}
\put(1429.0,550.0){\rule[-0.200pt]{2.409pt}{0.400pt}}
\put(170.0,550.0){\rule[-0.200pt]{2.409pt}{0.400pt}}
\put(1429.0,550.0){\rule[-0.200pt]{2.409pt}{0.400pt}}
\put(170.0,550.0){\rule[-0.200pt]{2.409pt}{0.400pt}}
\put(1429.0,550.0){\rule[-0.200pt]{2.409pt}{0.400pt}}
\put(170.0,550.0){\rule[-0.200pt]{2.409pt}{0.400pt}}
\put(1429.0,550.0){\rule[-0.200pt]{2.409pt}{0.400pt}}
\put(170.0,550.0){\rule[-0.200pt]{2.409pt}{0.400pt}}
\put(1429.0,550.0){\rule[-0.200pt]{2.409pt}{0.400pt}}
\put(170.0,550.0){\rule[-0.200pt]{2.409pt}{0.400pt}}
\put(1429.0,550.0){\rule[-0.200pt]{2.409pt}{0.400pt}}
\put(170.0,550.0){\rule[-0.200pt]{2.409pt}{0.400pt}}
\put(1429.0,550.0){\rule[-0.200pt]{2.409pt}{0.400pt}}
\put(170.0,551.0){\rule[-0.200pt]{2.409pt}{0.400pt}}
\put(1429.0,551.0){\rule[-0.200pt]{2.409pt}{0.400pt}}
\put(170.0,551.0){\rule[-0.200pt]{2.409pt}{0.400pt}}
\put(1429.0,551.0){\rule[-0.200pt]{2.409pt}{0.400pt}}
\put(170.0,551.0){\rule[-0.200pt]{2.409pt}{0.400pt}}
\put(1429.0,551.0){\rule[-0.200pt]{2.409pt}{0.400pt}}
\put(170.0,551.0){\rule[-0.200pt]{2.409pt}{0.400pt}}
\put(1429.0,551.0){\rule[-0.200pt]{2.409pt}{0.400pt}}
\put(170.0,551.0){\rule[-0.200pt]{2.409pt}{0.400pt}}
\put(1429.0,551.0){\rule[-0.200pt]{2.409pt}{0.400pt}}
\put(170.0,551.0){\rule[-0.200pt]{2.409pt}{0.400pt}}
\put(1429.0,551.0){\rule[-0.200pt]{2.409pt}{0.400pt}}
\put(170.0,551.0){\rule[-0.200pt]{2.409pt}{0.400pt}}
\put(1429.0,551.0){\rule[-0.200pt]{2.409pt}{0.400pt}}
\put(170.0,552.0){\rule[-0.200pt]{2.409pt}{0.400pt}}
\put(1429.0,552.0){\rule[-0.200pt]{2.409pt}{0.400pt}}
\put(170.0,552.0){\rule[-0.200pt]{2.409pt}{0.400pt}}
\put(1429.0,552.0){\rule[-0.200pt]{2.409pt}{0.400pt}}
\put(170.0,552.0){\rule[-0.200pt]{2.409pt}{0.400pt}}
\put(1429.0,552.0){\rule[-0.200pt]{2.409pt}{0.400pt}}
\put(170.0,552.0){\rule[-0.200pt]{2.409pt}{0.400pt}}
\put(1429.0,552.0){\rule[-0.200pt]{2.409pt}{0.400pt}}
\put(170.0,552.0){\rule[-0.200pt]{2.409pt}{0.400pt}}
\put(1429.0,552.0){\rule[-0.200pt]{2.409pt}{0.400pt}}
\put(170.0,552.0){\rule[-0.200pt]{2.409pt}{0.400pt}}
\put(1429.0,552.0){\rule[-0.200pt]{2.409pt}{0.400pt}}
\put(170.0,552.0){\rule[-0.200pt]{2.409pt}{0.400pt}}
\put(1429.0,552.0){\rule[-0.200pt]{2.409pt}{0.400pt}}
\put(170.0,553.0){\rule[-0.200pt]{2.409pt}{0.400pt}}
\put(1429.0,553.0){\rule[-0.200pt]{2.409pt}{0.400pt}}
\put(170.0,553.0){\rule[-0.200pt]{2.409pt}{0.400pt}}
\put(1429.0,553.0){\rule[-0.200pt]{2.409pt}{0.400pt}}
\put(170.0,553.0){\rule[-0.200pt]{2.409pt}{0.400pt}}
\put(1429.0,553.0){\rule[-0.200pt]{2.409pt}{0.400pt}}
\put(170.0,553.0){\rule[-0.200pt]{2.409pt}{0.400pt}}
\put(1429.0,553.0){\rule[-0.200pt]{2.409pt}{0.400pt}}
\put(170.0,553.0){\rule[-0.200pt]{2.409pt}{0.400pt}}
\put(1429.0,553.0){\rule[-0.200pt]{2.409pt}{0.400pt}}
\put(170.0,553.0){\rule[-0.200pt]{2.409pt}{0.400pt}}
\put(1429.0,553.0){\rule[-0.200pt]{2.409pt}{0.400pt}}
\put(170.0,553.0){\rule[-0.200pt]{2.409pt}{0.400pt}}
\put(1429.0,553.0){\rule[-0.200pt]{2.409pt}{0.400pt}}
\put(170.0,553.0){\rule[-0.200pt]{2.409pt}{0.400pt}}
\put(1429.0,553.0){\rule[-0.200pt]{2.409pt}{0.400pt}}
\put(170.0,554.0){\rule[-0.200pt]{2.409pt}{0.400pt}}
\put(1429.0,554.0){\rule[-0.200pt]{2.409pt}{0.400pt}}
\put(170.0,554.0){\rule[-0.200pt]{2.409pt}{0.400pt}}
\put(1429.0,554.0){\rule[-0.200pt]{2.409pt}{0.400pt}}
\put(170.0,554.0){\rule[-0.200pt]{2.409pt}{0.400pt}}
\put(1429.0,554.0){\rule[-0.200pt]{2.409pt}{0.400pt}}
\put(170.0,554.0){\rule[-0.200pt]{2.409pt}{0.400pt}}
\put(1429.0,554.0){\rule[-0.200pt]{2.409pt}{0.400pt}}
\put(170.0,554.0){\rule[-0.200pt]{2.409pt}{0.400pt}}
\put(1429.0,554.0){\rule[-0.200pt]{2.409pt}{0.400pt}}
\put(170.0,554.0){\rule[-0.200pt]{2.409pt}{0.400pt}}
\put(1429.0,554.0){\rule[-0.200pt]{2.409pt}{0.400pt}}
\put(170.0,554.0){\rule[-0.200pt]{2.409pt}{0.400pt}}
\put(1429.0,554.0){\rule[-0.200pt]{2.409pt}{0.400pt}}
\put(170.0,554.0){\rule[-0.200pt]{2.409pt}{0.400pt}}
\put(1429.0,554.0){\rule[-0.200pt]{2.409pt}{0.400pt}}
\put(170.0,555.0){\rule[-0.200pt]{2.409pt}{0.400pt}}
\put(1429.0,555.0){\rule[-0.200pt]{2.409pt}{0.400pt}}
\put(170.0,555.0){\rule[-0.200pt]{2.409pt}{0.400pt}}
\put(1429.0,555.0){\rule[-0.200pt]{2.409pt}{0.400pt}}
\put(170.0,555.0){\rule[-0.200pt]{2.409pt}{0.400pt}}
\put(1429.0,555.0){\rule[-0.200pt]{2.409pt}{0.400pt}}
\put(170.0,555.0){\rule[-0.200pt]{2.409pt}{0.400pt}}
\put(1429.0,555.0){\rule[-0.200pt]{2.409pt}{0.400pt}}
\put(170.0,555.0){\rule[-0.200pt]{2.409pt}{0.400pt}}
\put(1429.0,555.0){\rule[-0.200pt]{2.409pt}{0.400pt}}
\put(170.0,555.0){\rule[-0.200pt]{2.409pt}{0.400pt}}
\put(1429.0,555.0){\rule[-0.200pt]{2.409pt}{0.400pt}}
\put(170.0,555.0){\rule[-0.200pt]{2.409pt}{0.400pt}}
\put(1429.0,555.0){\rule[-0.200pt]{2.409pt}{0.400pt}}
\put(170.0,555.0){\rule[-0.200pt]{2.409pt}{0.400pt}}
\put(1429.0,555.0){\rule[-0.200pt]{2.409pt}{0.400pt}}
\put(170.0,556.0){\rule[-0.200pt]{2.409pt}{0.400pt}}
\put(1429.0,556.0){\rule[-0.200pt]{2.409pt}{0.400pt}}
\put(170.0,556.0){\rule[-0.200pt]{2.409pt}{0.400pt}}
\put(1429.0,556.0){\rule[-0.200pt]{2.409pt}{0.400pt}}
\put(170.0,556.0){\rule[-0.200pt]{2.409pt}{0.400pt}}
\put(1429.0,556.0){\rule[-0.200pt]{2.409pt}{0.400pt}}
\put(170.0,556.0){\rule[-0.200pt]{2.409pt}{0.400pt}}
\put(1429.0,556.0){\rule[-0.200pt]{2.409pt}{0.400pt}}
\put(170.0,556.0){\rule[-0.200pt]{2.409pt}{0.400pt}}
\put(1429.0,556.0){\rule[-0.200pt]{2.409pt}{0.400pt}}
\put(170.0,556.0){\rule[-0.200pt]{2.409pt}{0.400pt}}
\put(1429.0,556.0){\rule[-0.200pt]{2.409pt}{0.400pt}}
\put(170.0,556.0){\rule[-0.200pt]{2.409pt}{0.400pt}}
\put(1429.0,556.0){\rule[-0.200pt]{2.409pt}{0.400pt}}
\put(170.0,556.0){\rule[-0.200pt]{2.409pt}{0.400pt}}
\put(1429.0,556.0){\rule[-0.200pt]{2.409pt}{0.400pt}}
\put(170.0,557.0){\rule[-0.200pt]{2.409pt}{0.400pt}}
\put(1429.0,557.0){\rule[-0.200pt]{2.409pt}{0.400pt}}
\put(170.0,557.0){\rule[-0.200pt]{2.409pt}{0.400pt}}
\put(1429.0,557.0){\rule[-0.200pt]{2.409pt}{0.400pt}}
\put(170.0,557.0){\rule[-0.200pt]{2.409pt}{0.400pt}}
\put(1429.0,557.0){\rule[-0.200pt]{2.409pt}{0.400pt}}
\put(170.0,557.0){\rule[-0.200pt]{2.409pt}{0.400pt}}
\put(1429.0,557.0){\rule[-0.200pt]{2.409pt}{0.400pt}}
\put(170.0,557.0){\rule[-0.200pt]{2.409pt}{0.400pt}}
\put(1429.0,557.0){\rule[-0.200pt]{2.409pt}{0.400pt}}
\put(170.0,557.0){\rule[-0.200pt]{2.409pt}{0.400pt}}
\put(1429.0,557.0){\rule[-0.200pt]{2.409pt}{0.400pt}}
\put(170.0,557.0){\rule[-0.200pt]{2.409pt}{0.400pt}}
\put(1429.0,557.0){\rule[-0.200pt]{2.409pt}{0.400pt}}
\put(170.0,557.0){\rule[-0.200pt]{2.409pt}{0.400pt}}
\put(1429.0,557.0){\rule[-0.200pt]{2.409pt}{0.400pt}}
\put(170.0,558.0){\rule[-0.200pt]{2.409pt}{0.400pt}}
\put(1429.0,558.0){\rule[-0.200pt]{2.409pt}{0.400pt}}
\put(170.0,558.0){\rule[-0.200pt]{2.409pt}{0.400pt}}
\put(1429.0,558.0){\rule[-0.200pt]{2.409pt}{0.400pt}}
\put(170.0,558.0){\rule[-0.200pt]{2.409pt}{0.400pt}}
\put(1429.0,558.0){\rule[-0.200pt]{2.409pt}{0.400pt}}
\put(170.0,558.0){\rule[-0.200pt]{2.409pt}{0.400pt}}
\put(1429.0,558.0){\rule[-0.200pt]{2.409pt}{0.400pt}}
\put(170.0,558.0){\rule[-0.200pt]{2.409pt}{0.400pt}}
\put(1429.0,558.0){\rule[-0.200pt]{2.409pt}{0.400pt}}
\put(170.0,558.0){\rule[-0.200pt]{2.409pt}{0.400pt}}
\put(1429.0,558.0){\rule[-0.200pt]{2.409pt}{0.400pt}}
\put(170.0,558.0){\rule[-0.200pt]{2.409pt}{0.400pt}}
\put(1429.0,558.0){\rule[-0.200pt]{2.409pt}{0.400pt}}
\put(170.0,558.0){\rule[-0.200pt]{2.409pt}{0.400pt}}
\put(1429.0,558.0){\rule[-0.200pt]{2.409pt}{0.400pt}}
\put(170.0,558.0){\rule[-0.200pt]{2.409pt}{0.400pt}}
\put(1429.0,558.0){\rule[-0.200pt]{2.409pt}{0.400pt}}
\put(170.0,559.0){\rule[-0.200pt]{2.409pt}{0.400pt}}
\put(1429.0,559.0){\rule[-0.200pt]{2.409pt}{0.400pt}}
\put(170.0,559.0){\rule[-0.200pt]{2.409pt}{0.400pt}}
\put(1429.0,559.0){\rule[-0.200pt]{2.409pt}{0.400pt}}
\put(170.0,559.0){\rule[-0.200pt]{2.409pt}{0.400pt}}
\put(1429.0,559.0){\rule[-0.200pt]{2.409pt}{0.400pt}}
\put(170.0,559.0){\rule[-0.200pt]{2.409pt}{0.400pt}}
\put(1429.0,559.0){\rule[-0.200pt]{2.409pt}{0.400pt}}
\put(170.0,559.0){\rule[-0.200pt]{2.409pt}{0.400pt}}
\put(1429.0,559.0){\rule[-0.200pt]{2.409pt}{0.400pt}}
\put(170.0,559.0){\rule[-0.200pt]{2.409pt}{0.400pt}}
\put(1429.0,559.0){\rule[-0.200pt]{2.409pt}{0.400pt}}
\put(170.0,559.0){\rule[-0.200pt]{2.409pt}{0.400pt}}
\put(1429.0,559.0){\rule[-0.200pt]{2.409pt}{0.400pt}}
\put(170.0,559.0){\rule[-0.200pt]{2.409pt}{0.400pt}}
\put(1429.0,559.0){\rule[-0.200pt]{2.409pt}{0.400pt}}
\put(170.0,559.0){\rule[-0.200pt]{2.409pt}{0.400pt}}
\put(1429.0,559.0){\rule[-0.200pt]{2.409pt}{0.400pt}}
\put(170.0,560.0){\rule[-0.200pt]{2.409pt}{0.400pt}}
\put(1429.0,560.0){\rule[-0.200pt]{2.409pt}{0.400pt}}
\put(170.0,560.0){\rule[-0.200pt]{2.409pt}{0.400pt}}
\put(1429.0,560.0){\rule[-0.200pt]{2.409pt}{0.400pt}}
\put(170.0,560.0){\rule[-0.200pt]{2.409pt}{0.400pt}}
\put(1429.0,560.0){\rule[-0.200pt]{2.409pt}{0.400pt}}
\put(170.0,560.0){\rule[-0.200pt]{2.409pt}{0.400pt}}
\put(1429.0,560.0){\rule[-0.200pt]{2.409pt}{0.400pt}}
\put(170.0,560.0){\rule[-0.200pt]{2.409pt}{0.400pt}}
\put(1429.0,560.0){\rule[-0.200pt]{2.409pt}{0.400pt}}
\put(170.0,560.0){\rule[-0.200pt]{2.409pt}{0.400pt}}
\put(1429.0,560.0){\rule[-0.200pt]{2.409pt}{0.400pt}}
\put(170.0,560.0){\rule[-0.200pt]{2.409pt}{0.400pt}}
\put(1429.0,560.0){\rule[-0.200pt]{2.409pt}{0.400pt}}
\put(170.0,560.0){\rule[-0.200pt]{2.409pt}{0.400pt}}
\put(1429.0,560.0){\rule[-0.200pt]{2.409pt}{0.400pt}}
\put(170.0,560.0){\rule[-0.200pt]{2.409pt}{0.400pt}}
\put(1429.0,560.0){\rule[-0.200pt]{2.409pt}{0.400pt}}
\put(170.0,561.0){\rule[-0.200pt]{2.409pt}{0.400pt}}
\put(1429.0,561.0){\rule[-0.200pt]{2.409pt}{0.400pt}}
\put(170.0,561.0){\rule[-0.200pt]{2.409pt}{0.400pt}}
\put(1429.0,561.0){\rule[-0.200pt]{2.409pt}{0.400pt}}
\put(170.0,561.0){\rule[-0.200pt]{2.409pt}{0.400pt}}
\put(1429.0,561.0){\rule[-0.200pt]{2.409pt}{0.400pt}}
\put(170.0,561.0){\rule[-0.200pt]{2.409pt}{0.400pt}}
\put(1429.0,561.0){\rule[-0.200pt]{2.409pt}{0.400pt}}
\put(170.0,561.0){\rule[-0.200pt]{2.409pt}{0.400pt}}
\put(1429.0,561.0){\rule[-0.200pt]{2.409pt}{0.400pt}}
\put(170.0,561.0){\rule[-0.200pt]{2.409pt}{0.400pt}}
\put(1429.0,561.0){\rule[-0.200pt]{2.409pt}{0.400pt}}
\put(170.0,561.0){\rule[-0.200pt]{2.409pt}{0.400pt}}
\put(1429.0,561.0){\rule[-0.200pt]{2.409pt}{0.400pt}}
\put(170.0,561.0){\rule[-0.200pt]{2.409pt}{0.400pt}}
\put(1429.0,561.0){\rule[-0.200pt]{2.409pt}{0.400pt}}
\put(170.0,561.0){\rule[-0.200pt]{2.409pt}{0.400pt}}
\put(1429.0,561.0){\rule[-0.200pt]{2.409pt}{0.400pt}}
\put(170.0,561.0){\rule[-0.200pt]{2.409pt}{0.400pt}}
\put(1429.0,561.0){\rule[-0.200pt]{2.409pt}{0.400pt}}
\put(170.0,562.0){\rule[-0.200pt]{2.409pt}{0.400pt}}
\put(1429.0,562.0){\rule[-0.200pt]{2.409pt}{0.400pt}}
\put(170.0,562.0){\rule[-0.200pt]{2.409pt}{0.400pt}}
\put(1429.0,562.0){\rule[-0.200pt]{2.409pt}{0.400pt}}
\put(170.0,562.0){\rule[-0.200pt]{2.409pt}{0.400pt}}
\put(1429.0,562.0){\rule[-0.200pt]{2.409pt}{0.400pt}}
\put(170.0,562.0){\rule[-0.200pt]{2.409pt}{0.400pt}}
\put(1429.0,562.0){\rule[-0.200pt]{2.409pt}{0.400pt}}
\put(170.0,562.0){\rule[-0.200pt]{2.409pt}{0.400pt}}
\put(1429.0,562.0){\rule[-0.200pt]{2.409pt}{0.400pt}}
\put(170.0,562.0){\rule[-0.200pt]{2.409pt}{0.400pt}}
\put(1429.0,562.0){\rule[-0.200pt]{2.409pt}{0.400pt}}
\put(170.0,562.0){\rule[-0.200pt]{2.409pt}{0.400pt}}
\put(1429.0,562.0){\rule[-0.200pt]{2.409pt}{0.400pt}}
\put(170.0,562.0){\rule[-0.200pt]{2.409pt}{0.400pt}}
\put(1429.0,562.0){\rule[-0.200pt]{2.409pt}{0.400pt}}
\put(170.0,562.0){\rule[-0.200pt]{2.409pt}{0.400pt}}
\put(1429.0,562.0){\rule[-0.200pt]{2.409pt}{0.400pt}}
\put(170.0,563.0){\rule[-0.200pt]{2.409pt}{0.400pt}}
\put(1429.0,563.0){\rule[-0.200pt]{2.409pt}{0.400pt}}
\put(170.0,563.0){\rule[-0.200pt]{2.409pt}{0.400pt}}
\put(1429.0,563.0){\rule[-0.200pt]{2.409pt}{0.400pt}}
\put(170.0,563.0){\rule[-0.200pt]{2.409pt}{0.400pt}}
\put(1429.0,563.0){\rule[-0.200pt]{2.409pt}{0.400pt}}
\put(170.0,563.0){\rule[-0.200pt]{2.409pt}{0.400pt}}
\put(1429.0,563.0){\rule[-0.200pt]{2.409pt}{0.400pt}}
\put(170.0,563.0){\rule[-0.200pt]{2.409pt}{0.400pt}}
\put(1429.0,563.0){\rule[-0.200pt]{2.409pt}{0.400pt}}
\put(170.0,563.0){\rule[-0.200pt]{2.409pt}{0.400pt}}
\put(1429.0,563.0){\rule[-0.200pt]{2.409pt}{0.400pt}}
\put(170.0,563.0){\rule[-0.200pt]{2.409pt}{0.400pt}}
\put(1429.0,563.0){\rule[-0.200pt]{2.409pt}{0.400pt}}
\put(170.0,563.0){\rule[-0.200pt]{2.409pt}{0.400pt}}
\put(1429.0,563.0){\rule[-0.200pt]{2.409pt}{0.400pt}}
\put(170.0,563.0){\rule[-0.200pt]{2.409pt}{0.400pt}}
\put(1429.0,563.0){\rule[-0.200pt]{2.409pt}{0.400pt}}
\put(170.0,563.0){\rule[-0.200pt]{2.409pt}{0.400pt}}
\put(1429.0,563.0){\rule[-0.200pt]{2.409pt}{0.400pt}}
\put(170.0,564.0){\rule[-0.200pt]{2.409pt}{0.400pt}}
\put(1429.0,564.0){\rule[-0.200pt]{2.409pt}{0.400pt}}
\put(170.0,564.0){\rule[-0.200pt]{2.409pt}{0.400pt}}
\put(1429.0,564.0){\rule[-0.200pt]{2.409pt}{0.400pt}}
\put(170.0,564.0){\rule[-0.200pt]{2.409pt}{0.400pt}}
\put(1429.0,564.0){\rule[-0.200pt]{2.409pt}{0.400pt}}
\put(170.0,564.0){\rule[-0.200pt]{2.409pt}{0.400pt}}
\put(1429.0,564.0){\rule[-0.200pt]{2.409pt}{0.400pt}}
\put(170.0,564.0){\rule[-0.200pt]{2.409pt}{0.400pt}}
\put(1429.0,564.0){\rule[-0.200pt]{2.409pt}{0.400pt}}
\put(170.0,564.0){\rule[-0.200pt]{2.409pt}{0.400pt}}
\put(1429.0,564.0){\rule[-0.200pt]{2.409pt}{0.400pt}}
\put(170.0,564.0){\rule[-0.200pt]{2.409pt}{0.400pt}}
\put(1429.0,564.0){\rule[-0.200pt]{2.409pt}{0.400pt}}
\put(170.0,564.0){\rule[-0.200pt]{2.409pt}{0.400pt}}
\put(1429.0,564.0){\rule[-0.200pt]{2.409pt}{0.400pt}}
\put(170.0,564.0){\rule[-0.200pt]{2.409pt}{0.400pt}}
\put(1429.0,564.0){\rule[-0.200pt]{2.409pt}{0.400pt}}
\put(170.0,564.0){\rule[-0.200pt]{2.409pt}{0.400pt}}
\put(1429.0,564.0){\rule[-0.200pt]{2.409pt}{0.400pt}}
\put(170.0,565.0){\rule[-0.200pt]{2.409pt}{0.400pt}}
\put(1429.0,565.0){\rule[-0.200pt]{2.409pt}{0.400pt}}
\put(170.0,565.0){\rule[-0.200pt]{2.409pt}{0.400pt}}
\put(1429.0,565.0){\rule[-0.200pt]{2.409pt}{0.400pt}}
\put(170.0,565.0){\rule[-0.200pt]{2.409pt}{0.400pt}}
\put(1429.0,565.0){\rule[-0.200pt]{2.409pt}{0.400pt}}
\put(170.0,565.0){\rule[-0.200pt]{2.409pt}{0.400pt}}
\put(1429.0,565.0){\rule[-0.200pt]{2.409pt}{0.400pt}}
\put(170.0,565.0){\rule[-0.200pt]{2.409pt}{0.400pt}}
\put(1429.0,565.0){\rule[-0.200pt]{2.409pt}{0.400pt}}
\put(170.0,565.0){\rule[-0.200pt]{2.409pt}{0.400pt}}
\put(1429.0,565.0){\rule[-0.200pt]{2.409pt}{0.400pt}}
\put(170.0,565.0){\rule[-0.200pt]{2.409pt}{0.400pt}}
\put(1429.0,565.0){\rule[-0.200pt]{2.409pt}{0.400pt}}
\put(170.0,565.0){\rule[-0.200pt]{2.409pt}{0.400pt}}
\put(1429.0,565.0){\rule[-0.200pt]{2.409pt}{0.400pt}}
\put(170.0,565.0){\rule[-0.200pt]{2.409pt}{0.400pt}}
\put(1429.0,565.0){\rule[-0.200pt]{2.409pt}{0.400pt}}
\put(170.0,565.0){\rule[-0.200pt]{2.409pt}{0.400pt}}
\put(1429.0,565.0){\rule[-0.200pt]{2.409pt}{0.400pt}}
\put(170.0,565.0){\rule[-0.200pt]{2.409pt}{0.400pt}}
\put(1429.0,565.0){\rule[-0.200pt]{2.409pt}{0.400pt}}
\put(170.0,566.0){\rule[-0.200pt]{2.409pt}{0.400pt}}
\put(1429.0,566.0){\rule[-0.200pt]{2.409pt}{0.400pt}}
\put(170.0,566.0){\rule[-0.200pt]{2.409pt}{0.400pt}}
\put(1429.0,566.0){\rule[-0.200pt]{2.409pt}{0.400pt}}
\put(170.0,566.0){\rule[-0.200pt]{2.409pt}{0.400pt}}
\put(1429.0,566.0){\rule[-0.200pt]{2.409pt}{0.400pt}}
\put(170.0,566.0){\rule[-0.200pt]{2.409pt}{0.400pt}}
\put(1429.0,566.0){\rule[-0.200pt]{2.409pt}{0.400pt}}
\put(170.0,566.0){\rule[-0.200pt]{2.409pt}{0.400pt}}
\put(1429.0,566.0){\rule[-0.200pt]{2.409pt}{0.400pt}}
\put(170.0,566.0){\rule[-0.200pt]{2.409pt}{0.400pt}}
\put(1429.0,566.0){\rule[-0.200pt]{2.409pt}{0.400pt}}
\put(170.0,566.0){\rule[-0.200pt]{2.409pt}{0.400pt}}
\put(1429.0,566.0){\rule[-0.200pt]{2.409pt}{0.400pt}}
\put(170.0,566.0){\rule[-0.200pt]{2.409pt}{0.400pt}}
\put(1429.0,566.0){\rule[-0.200pt]{2.409pt}{0.400pt}}
\put(170.0,566.0){\rule[-0.200pt]{2.409pt}{0.400pt}}
\put(1429.0,566.0){\rule[-0.200pt]{2.409pt}{0.400pt}}
\put(170.0,566.0){\rule[-0.200pt]{2.409pt}{0.400pt}}
\put(1429.0,566.0){\rule[-0.200pt]{2.409pt}{0.400pt}}
\put(170.0,566.0){\rule[-0.200pt]{2.409pt}{0.400pt}}
\put(1429.0,566.0){\rule[-0.200pt]{2.409pt}{0.400pt}}
\put(170.0,567.0){\rule[-0.200pt]{2.409pt}{0.400pt}}
\put(1429.0,567.0){\rule[-0.200pt]{2.409pt}{0.400pt}}
\put(170.0,567.0){\rule[-0.200pt]{2.409pt}{0.400pt}}
\put(1429.0,567.0){\rule[-0.200pt]{2.409pt}{0.400pt}}
\put(170.0,567.0){\rule[-0.200pt]{2.409pt}{0.400pt}}
\put(1429.0,567.0){\rule[-0.200pt]{2.409pt}{0.400pt}}
\put(170.0,567.0){\rule[-0.200pt]{2.409pt}{0.400pt}}
\put(1429.0,567.0){\rule[-0.200pt]{2.409pt}{0.400pt}}
\put(170.0,567.0){\rule[-0.200pt]{2.409pt}{0.400pt}}
\put(1429.0,567.0){\rule[-0.200pt]{2.409pt}{0.400pt}}
\put(170.0,567.0){\rule[-0.200pt]{2.409pt}{0.400pt}}
\put(1429.0,567.0){\rule[-0.200pt]{2.409pt}{0.400pt}}
\put(170.0,567.0){\rule[-0.200pt]{2.409pt}{0.400pt}}
\put(1429.0,567.0){\rule[-0.200pt]{2.409pt}{0.400pt}}
\put(170.0,567.0){\rule[-0.200pt]{2.409pt}{0.400pt}}
\put(1429.0,567.0){\rule[-0.200pt]{2.409pt}{0.400pt}}
\put(170.0,567.0){\rule[-0.200pt]{2.409pt}{0.400pt}}
\put(1429.0,567.0){\rule[-0.200pt]{2.409pt}{0.400pt}}
\put(170.0,567.0){\rule[-0.200pt]{2.409pt}{0.400pt}}
\put(1429.0,567.0){\rule[-0.200pt]{2.409pt}{0.400pt}}
\put(170.0,567.0){\rule[-0.200pt]{2.409pt}{0.400pt}}
\put(1429.0,567.0){\rule[-0.200pt]{2.409pt}{0.400pt}}
\put(170.0,568.0){\rule[-0.200pt]{2.409pt}{0.400pt}}
\put(1429.0,568.0){\rule[-0.200pt]{2.409pt}{0.400pt}}
\put(170.0,568.0){\rule[-0.200pt]{2.409pt}{0.400pt}}
\put(1429.0,568.0){\rule[-0.200pt]{2.409pt}{0.400pt}}
\put(170.0,568.0){\rule[-0.200pt]{2.409pt}{0.400pt}}
\put(1429.0,568.0){\rule[-0.200pt]{2.409pt}{0.400pt}}
\put(170.0,568.0){\rule[-0.200pt]{2.409pt}{0.400pt}}
\put(1429.0,568.0){\rule[-0.200pt]{2.409pt}{0.400pt}}
\put(170.0,568.0){\rule[-0.200pt]{2.409pt}{0.400pt}}
\put(1429.0,568.0){\rule[-0.200pt]{2.409pt}{0.400pt}}
\put(170.0,568.0){\rule[-0.200pt]{2.409pt}{0.400pt}}
\put(1429.0,568.0){\rule[-0.200pt]{2.409pt}{0.400pt}}
\put(170.0,568.0){\rule[-0.200pt]{2.409pt}{0.400pt}}
\put(1429.0,568.0){\rule[-0.200pt]{2.409pt}{0.400pt}}
\put(170.0,568.0){\rule[-0.200pt]{2.409pt}{0.400pt}}
\put(1429.0,568.0){\rule[-0.200pt]{2.409pt}{0.400pt}}
\put(170.0,568.0){\rule[-0.200pt]{2.409pt}{0.400pt}}
\put(1429.0,568.0){\rule[-0.200pt]{2.409pt}{0.400pt}}
\put(170.0,568.0){\rule[-0.200pt]{2.409pt}{0.400pt}}
\put(1429.0,568.0){\rule[-0.200pt]{2.409pt}{0.400pt}}
\put(170.0,568.0){\rule[-0.200pt]{2.409pt}{0.400pt}}
\put(1429.0,568.0){\rule[-0.200pt]{2.409pt}{0.400pt}}
\put(170.0,569.0){\rule[-0.200pt]{2.409pt}{0.400pt}}
\put(1429.0,569.0){\rule[-0.200pt]{2.409pt}{0.400pt}}
\put(170.0,569.0){\rule[-0.200pt]{2.409pt}{0.400pt}}
\put(1429.0,569.0){\rule[-0.200pt]{2.409pt}{0.400pt}}
\put(170.0,569.0){\rule[-0.200pt]{2.409pt}{0.400pt}}
\put(1429.0,569.0){\rule[-0.200pt]{2.409pt}{0.400pt}}
\put(170.0,569.0){\rule[-0.200pt]{2.409pt}{0.400pt}}
\put(1429.0,569.0){\rule[-0.200pt]{2.409pt}{0.400pt}}
\put(170.0,569.0){\rule[-0.200pt]{2.409pt}{0.400pt}}
\put(1429.0,569.0){\rule[-0.200pt]{2.409pt}{0.400pt}}
\put(170.0,569.0){\rule[-0.200pt]{2.409pt}{0.400pt}}
\put(1429.0,569.0){\rule[-0.200pt]{2.409pt}{0.400pt}}
\put(170.0,569.0){\rule[-0.200pt]{2.409pt}{0.400pt}}
\put(1429.0,569.0){\rule[-0.200pt]{2.409pt}{0.400pt}}
\put(170.0,569.0){\rule[-0.200pt]{2.409pt}{0.400pt}}
\put(1429.0,569.0){\rule[-0.200pt]{2.409pt}{0.400pt}}
\put(170.0,569.0){\rule[-0.200pt]{2.409pt}{0.400pt}}
\put(1429.0,569.0){\rule[-0.200pt]{2.409pt}{0.400pt}}
\put(170.0,569.0){\rule[-0.200pt]{2.409pt}{0.400pt}}
\put(1429.0,569.0){\rule[-0.200pt]{2.409pt}{0.400pt}}
\put(170.0,569.0){\rule[-0.200pt]{2.409pt}{0.400pt}}
\put(1429.0,569.0){\rule[-0.200pt]{2.409pt}{0.400pt}}
\put(170.0,569.0){\rule[-0.200pt]{2.409pt}{0.400pt}}
\put(1429.0,569.0){\rule[-0.200pt]{2.409pt}{0.400pt}}
\put(170.0,570.0){\rule[-0.200pt]{2.409pt}{0.400pt}}
\put(1429.0,570.0){\rule[-0.200pt]{2.409pt}{0.400pt}}
\put(170.0,570.0){\rule[-0.200pt]{2.409pt}{0.400pt}}
\put(1429.0,570.0){\rule[-0.200pt]{2.409pt}{0.400pt}}
\put(170.0,570.0){\rule[-0.200pt]{2.409pt}{0.400pt}}
\put(1429.0,570.0){\rule[-0.200pt]{2.409pt}{0.400pt}}
\put(170.0,570.0){\rule[-0.200pt]{2.409pt}{0.400pt}}
\put(1429.0,570.0){\rule[-0.200pt]{2.409pt}{0.400pt}}
\put(170.0,570.0){\rule[-0.200pt]{2.409pt}{0.400pt}}
\put(1429.0,570.0){\rule[-0.200pt]{2.409pt}{0.400pt}}
\put(170.0,570.0){\rule[-0.200pt]{2.409pt}{0.400pt}}
\put(1429.0,570.0){\rule[-0.200pt]{2.409pt}{0.400pt}}
\put(170.0,570.0){\rule[-0.200pt]{2.409pt}{0.400pt}}
\put(1429.0,570.0){\rule[-0.200pt]{2.409pt}{0.400pt}}
\put(170.0,570.0){\rule[-0.200pt]{2.409pt}{0.400pt}}
\put(1429.0,570.0){\rule[-0.200pt]{2.409pt}{0.400pt}}
\put(170.0,570.0){\rule[-0.200pt]{2.409pt}{0.400pt}}
\put(1429.0,570.0){\rule[-0.200pt]{2.409pt}{0.400pt}}
\put(170.0,570.0){\rule[-0.200pt]{2.409pt}{0.400pt}}
\put(1429.0,570.0){\rule[-0.200pt]{2.409pt}{0.400pt}}
\put(170.0,570.0){\rule[-0.200pt]{2.409pt}{0.400pt}}
\put(1429.0,570.0){\rule[-0.200pt]{2.409pt}{0.400pt}}
\put(170.0,570.0){\rule[-0.200pt]{2.409pt}{0.400pt}}
\put(1429.0,570.0){\rule[-0.200pt]{2.409pt}{0.400pt}}
\put(170.0,571.0){\rule[-0.200pt]{2.409pt}{0.400pt}}
\put(1429.0,571.0){\rule[-0.200pt]{2.409pt}{0.400pt}}
\put(170.0,571.0){\rule[-0.200pt]{2.409pt}{0.400pt}}
\put(1429.0,571.0){\rule[-0.200pt]{2.409pt}{0.400pt}}
\put(170.0,571.0){\rule[-0.200pt]{2.409pt}{0.400pt}}
\put(1429.0,571.0){\rule[-0.200pt]{2.409pt}{0.400pt}}
\put(170.0,571.0){\rule[-0.200pt]{2.409pt}{0.400pt}}
\put(1429.0,571.0){\rule[-0.200pt]{2.409pt}{0.400pt}}
\put(170.0,571.0){\rule[-0.200pt]{2.409pt}{0.400pt}}
\put(1429.0,571.0){\rule[-0.200pt]{2.409pt}{0.400pt}}
\put(170.0,571.0){\rule[-0.200pt]{2.409pt}{0.400pt}}
\put(1429.0,571.0){\rule[-0.200pt]{2.409pt}{0.400pt}}
\put(170.0,571.0){\rule[-0.200pt]{2.409pt}{0.400pt}}
\put(1429.0,571.0){\rule[-0.200pt]{2.409pt}{0.400pt}}
\put(170.0,571.0){\rule[-0.200pt]{2.409pt}{0.400pt}}
\put(1429.0,571.0){\rule[-0.200pt]{2.409pt}{0.400pt}}
\put(170.0,571.0){\rule[-0.200pt]{2.409pt}{0.400pt}}
\put(1429.0,571.0){\rule[-0.200pt]{2.409pt}{0.400pt}}
\put(170.0,571.0){\rule[-0.200pt]{2.409pt}{0.400pt}}
\put(1429.0,571.0){\rule[-0.200pt]{2.409pt}{0.400pt}}
\put(170.0,571.0){\rule[-0.200pt]{2.409pt}{0.400pt}}
\put(1429.0,571.0){\rule[-0.200pt]{2.409pt}{0.400pt}}
\put(170.0,571.0){\rule[-0.200pt]{2.409pt}{0.400pt}}
\put(1429.0,571.0){\rule[-0.200pt]{2.409pt}{0.400pt}}
\put(170.0,572.0){\rule[-0.200pt]{2.409pt}{0.400pt}}
\put(1429.0,572.0){\rule[-0.200pt]{2.409pt}{0.400pt}}
\put(170.0,572.0){\rule[-0.200pt]{2.409pt}{0.400pt}}
\put(1429.0,572.0){\rule[-0.200pt]{2.409pt}{0.400pt}}
\put(170.0,572.0){\rule[-0.200pt]{2.409pt}{0.400pt}}
\put(1429.0,572.0){\rule[-0.200pt]{2.409pt}{0.400pt}}
\put(170.0,572.0){\rule[-0.200pt]{2.409pt}{0.400pt}}
\put(1429.0,572.0){\rule[-0.200pt]{2.409pt}{0.400pt}}
\put(170.0,572.0){\rule[-0.200pt]{2.409pt}{0.400pt}}
\put(1429.0,572.0){\rule[-0.200pt]{2.409pt}{0.400pt}}
\put(170.0,572.0){\rule[-0.200pt]{2.409pt}{0.400pt}}
\put(1429.0,572.0){\rule[-0.200pt]{2.409pt}{0.400pt}}
\put(170.0,572.0){\rule[-0.200pt]{2.409pt}{0.400pt}}
\put(1429.0,572.0){\rule[-0.200pt]{2.409pt}{0.400pt}}
\put(170.0,572.0){\rule[-0.200pt]{2.409pt}{0.400pt}}
\put(1429.0,572.0){\rule[-0.200pt]{2.409pt}{0.400pt}}
\put(170.0,572.0){\rule[-0.200pt]{2.409pt}{0.400pt}}
\put(1429.0,572.0){\rule[-0.200pt]{2.409pt}{0.400pt}}
\put(170.0,572.0){\rule[-0.200pt]{2.409pt}{0.400pt}}
\put(1429.0,572.0){\rule[-0.200pt]{2.409pt}{0.400pt}}
\put(170.0,572.0){\rule[-0.200pt]{2.409pt}{0.400pt}}
\put(1429.0,572.0){\rule[-0.200pt]{2.409pt}{0.400pt}}
\put(170.0,572.0){\rule[-0.200pt]{2.409pt}{0.400pt}}
\put(1429.0,572.0){\rule[-0.200pt]{2.409pt}{0.400pt}}
\put(170.0,572.0){\rule[-0.200pt]{2.409pt}{0.400pt}}
\put(1429.0,572.0){\rule[-0.200pt]{2.409pt}{0.400pt}}
\put(170.0,573.0){\rule[-0.200pt]{2.409pt}{0.400pt}}
\put(1429.0,573.0){\rule[-0.200pt]{2.409pt}{0.400pt}}
\put(170.0,573.0){\rule[-0.200pt]{2.409pt}{0.400pt}}
\put(1429.0,573.0){\rule[-0.200pt]{2.409pt}{0.400pt}}
\put(170.0,573.0){\rule[-0.200pt]{2.409pt}{0.400pt}}
\put(1429.0,573.0){\rule[-0.200pt]{2.409pt}{0.400pt}}
\put(170.0,573.0){\rule[-0.200pt]{2.409pt}{0.400pt}}
\put(1429.0,573.0){\rule[-0.200pt]{2.409pt}{0.400pt}}
\put(170.0,573.0){\rule[-0.200pt]{2.409pt}{0.400pt}}
\put(1429.0,573.0){\rule[-0.200pt]{2.409pt}{0.400pt}}
\put(170.0,573.0){\rule[-0.200pt]{2.409pt}{0.400pt}}
\put(1429.0,573.0){\rule[-0.200pt]{2.409pt}{0.400pt}}
\put(170.0,573.0){\rule[-0.200pt]{2.409pt}{0.400pt}}
\put(1429.0,573.0){\rule[-0.200pt]{2.409pt}{0.400pt}}
\put(170.0,573.0){\rule[-0.200pt]{2.409pt}{0.400pt}}
\put(1429.0,573.0){\rule[-0.200pt]{2.409pt}{0.400pt}}
\put(170.0,573.0){\rule[-0.200pt]{2.409pt}{0.400pt}}
\put(1429.0,573.0){\rule[-0.200pt]{2.409pt}{0.400pt}}
\put(170.0,573.0){\rule[-0.200pt]{2.409pt}{0.400pt}}
\put(1429.0,573.0){\rule[-0.200pt]{2.409pt}{0.400pt}}
\put(170.0,573.0){\rule[-0.200pt]{2.409pt}{0.400pt}}
\put(1429.0,573.0){\rule[-0.200pt]{2.409pt}{0.400pt}}
\put(170.0,573.0){\rule[-0.200pt]{2.409pt}{0.400pt}}
\put(1429.0,573.0){\rule[-0.200pt]{2.409pt}{0.400pt}}
\put(170.0,573.0){\rule[-0.200pt]{2.409pt}{0.400pt}}
\put(1429.0,573.0){\rule[-0.200pt]{2.409pt}{0.400pt}}
\put(170.0,574.0){\rule[-0.200pt]{2.409pt}{0.400pt}}
\put(1429.0,574.0){\rule[-0.200pt]{2.409pt}{0.400pt}}
\put(170.0,574.0){\rule[-0.200pt]{2.409pt}{0.400pt}}
\put(1429.0,574.0){\rule[-0.200pt]{2.409pt}{0.400pt}}
\put(170.0,574.0){\rule[-0.200pt]{2.409pt}{0.400pt}}
\put(1429.0,574.0){\rule[-0.200pt]{2.409pt}{0.400pt}}
\put(170.0,574.0){\rule[-0.200pt]{2.409pt}{0.400pt}}
\put(1429.0,574.0){\rule[-0.200pt]{2.409pt}{0.400pt}}
\put(170.0,574.0){\rule[-0.200pt]{2.409pt}{0.400pt}}
\put(1429.0,574.0){\rule[-0.200pt]{2.409pt}{0.400pt}}
\put(170.0,574.0){\rule[-0.200pt]{2.409pt}{0.400pt}}
\put(1429.0,574.0){\rule[-0.200pt]{2.409pt}{0.400pt}}
\put(170.0,574.0){\rule[-0.200pt]{2.409pt}{0.400pt}}
\put(1429.0,574.0){\rule[-0.200pt]{2.409pt}{0.400pt}}
\put(170.0,574.0){\rule[-0.200pt]{2.409pt}{0.400pt}}
\put(1429.0,574.0){\rule[-0.200pt]{2.409pt}{0.400pt}}
\put(170.0,574.0){\rule[-0.200pt]{2.409pt}{0.400pt}}
\put(1429.0,574.0){\rule[-0.200pt]{2.409pt}{0.400pt}}
\put(170.0,574.0){\rule[-0.200pt]{2.409pt}{0.400pt}}
\put(1429.0,574.0){\rule[-0.200pt]{2.409pt}{0.400pt}}
\put(170.0,574.0){\rule[-0.200pt]{2.409pt}{0.400pt}}
\put(1429.0,574.0){\rule[-0.200pt]{2.409pt}{0.400pt}}
\put(170.0,574.0){\rule[-0.200pt]{2.409pt}{0.400pt}}
\put(1429.0,574.0){\rule[-0.200pt]{2.409pt}{0.400pt}}
\put(170.0,574.0){\rule[-0.200pt]{2.409pt}{0.400pt}}
\put(1429.0,574.0){\rule[-0.200pt]{2.409pt}{0.400pt}}
\put(170.0,575.0){\rule[-0.200pt]{2.409pt}{0.400pt}}
\put(1429.0,575.0){\rule[-0.200pt]{2.409pt}{0.400pt}}
\put(170.0,575.0){\rule[-0.200pt]{2.409pt}{0.400pt}}
\put(1429.0,575.0){\rule[-0.200pt]{2.409pt}{0.400pt}}
\put(170.0,575.0){\rule[-0.200pt]{2.409pt}{0.400pt}}
\put(1429.0,575.0){\rule[-0.200pt]{2.409pt}{0.400pt}}
\put(170.0,575.0){\rule[-0.200pt]{2.409pt}{0.400pt}}
\put(1429.0,575.0){\rule[-0.200pt]{2.409pt}{0.400pt}}
\put(170.0,575.0){\rule[-0.200pt]{2.409pt}{0.400pt}}
\put(1429.0,575.0){\rule[-0.200pt]{2.409pt}{0.400pt}}
\put(170.0,575.0){\rule[-0.200pt]{2.409pt}{0.400pt}}
\put(1429.0,575.0){\rule[-0.200pt]{2.409pt}{0.400pt}}
\put(170.0,575.0){\rule[-0.200pt]{2.409pt}{0.400pt}}
\put(1429.0,575.0){\rule[-0.200pt]{2.409pt}{0.400pt}}
\put(170.0,575.0){\rule[-0.200pt]{2.409pt}{0.400pt}}
\put(1429.0,575.0){\rule[-0.200pt]{2.409pt}{0.400pt}}
\put(170.0,575.0){\rule[-0.200pt]{2.409pt}{0.400pt}}
\put(1429.0,575.0){\rule[-0.200pt]{2.409pt}{0.400pt}}
\put(170.0,575.0){\rule[-0.200pt]{2.409pt}{0.400pt}}
\put(1429.0,575.0){\rule[-0.200pt]{2.409pt}{0.400pt}}
\put(170.0,575.0){\rule[-0.200pt]{2.409pt}{0.400pt}}
\put(1429.0,575.0){\rule[-0.200pt]{2.409pt}{0.400pt}}
\put(170.0,575.0){\rule[-0.200pt]{2.409pt}{0.400pt}}
\put(1429.0,575.0){\rule[-0.200pt]{2.409pt}{0.400pt}}
\put(170.0,575.0){\rule[-0.200pt]{2.409pt}{0.400pt}}
\put(1429.0,575.0){\rule[-0.200pt]{2.409pt}{0.400pt}}
\put(170.0,575.0){\rule[-0.200pt]{2.409pt}{0.400pt}}
\put(1429.0,575.0){\rule[-0.200pt]{2.409pt}{0.400pt}}
\put(170.0,576.0){\rule[-0.200pt]{2.409pt}{0.400pt}}
\put(1429.0,576.0){\rule[-0.200pt]{2.409pt}{0.400pt}}
\put(170.0,576.0){\rule[-0.200pt]{2.409pt}{0.400pt}}
\put(1429.0,576.0){\rule[-0.200pt]{2.409pt}{0.400pt}}
\put(170.0,576.0){\rule[-0.200pt]{2.409pt}{0.400pt}}
\put(1429.0,576.0){\rule[-0.200pt]{2.409pt}{0.400pt}}
\put(170.0,576.0){\rule[-0.200pt]{2.409pt}{0.400pt}}
\put(1429.0,576.0){\rule[-0.200pt]{2.409pt}{0.400pt}}
\put(170.0,576.0){\rule[-0.200pt]{2.409pt}{0.400pt}}
\put(1429.0,576.0){\rule[-0.200pt]{2.409pt}{0.400pt}}
\put(170.0,576.0){\rule[-0.200pt]{2.409pt}{0.400pt}}
\put(1429.0,576.0){\rule[-0.200pt]{2.409pt}{0.400pt}}
\put(170.0,576.0){\rule[-0.200pt]{2.409pt}{0.400pt}}
\put(1429.0,576.0){\rule[-0.200pt]{2.409pt}{0.400pt}}
\put(170.0,576.0){\rule[-0.200pt]{2.409pt}{0.400pt}}
\put(1429.0,576.0){\rule[-0.200pt]{2.409pt}{0.400pt}}
\put(170.0,576.0){\rule[-0.200pt]{2.409pt}{0.400pt}}
\put(1429.0,576.0){\rule[-0.200pt]{2.409pt}{0.400pt}}
\put(170.0,576.0){\rule[-0.200pt]{2.409pt}{0.400pt}}
\put(1429.0,576.0){\rule[-0.200pt]{2.409pt}{0.400pt}}
\put(170.0,576.0){\rule[-0.200pt]{2.409pt}{0.400pt}}
\put(1429.0,576.0){\rule[-0.200pt]{2.409pt}{0.400pt}}
\put(170.0,576.0){\rule[-0.200pt]{2.409pt}{0.400pt}}
\put(1429.0,576.0){\rule[-0.200pt]{2.409pt}{0.400pt}}
\put(170.0,576.0){\rule[-0.200pt]{2.409pt}{0.400pt}}
\put(1429.0,576.0){\rule[-0.200pt]{2.409pt}{0.400pt}}
\put(170.0,576.0){\rule[-0.200pt]{2.409pt}{0.400pt}}
\put(1429.0,576.0){\rule[-0.200pt]{2.409pt}{0.400pt}}
\put(170.0,577.0){\rule[-0.200pt]{2.409pt}{0.400pt}}
\put(1429.0,577.0){\rule[-0.200pt]{2.409pt}{0.400pt}}
\put(170.0,577.0){\rule[-0.200pt]{2.409pt}{0.400pt}}
\put(1429.0,577.0){\rule[-0.200pt]{2.409pt}{0.400pt}}
\put(170.0,577.0){\rule[-0.200pt]{2.409pt}{0.400pt}}
\put(1429.0,577.0){\rule[-0.200pt]{2.409pt}{0.400pt}}
\put(170.0,577.0){\rule[-0.200pt]{2.409pt}{0.400pt}}
\put(1429.0,577.0){\rule[-0.200pt]{2.409pt}{0.400pt}}
\put(170.0,577.0){\rule[-0.200pt]{2.409pt}{0.400pt}}
\put(1429.0,577.0){\rule[-0.200pt]{2.409pt}{0.400pt}}
\put(170.0,577.0){\rule[-0.200pt]{2.409pt}{0.400pt}}
\put(1429.0,577.0){\rule[-0.200pt]{2.409pt}{0.400pt}}
\put(170.0,577.0){\rule[-0.200pt]{2.409pt}{0.400pt}}
\put(1429.0,577.0){\rule[-0.200pt]{2.409pt}{0.400pt}}
\put(170.0,577.0){\rule[-0.200pt]{2.409pt}{0.400pt}}
\put(1429.0,577.0){\rule[-0.200pt]{2.409pt}{0.400pt}}
\put(170.0,577.0){\rule[-0.200pt]{2.409pt}{0.400pt}}
\put(1429.0,577.0){\rule[-0.200pt]{2.409pt}{0.400pt}}
\put(170.0,577.0){\rule[-0.200pt]{2.409pt}{0.400pt}}
\put(1429.0,577.0){\rule[-0.200pt]{2.409pt}{0.400pt}}
\put(170.0,577.0){\rule[-0.200pt]{2.409pt}{0.400pt}}
\put(1429.0,577.0){\rule[-0.200pt]{2.409pt}{0.400pt}}
\put(170.0,577.0){\rule[-0.200pt]{2.409pt}{0.400pt}}
\put(1429.0,577.0){\rule[-0.200pt]{2.409pt}{0.400pt}}
\put(170.0,577.0){\rule[-0.200pt]{2.409pt}{0.400pt}}
\put(1429.0,577.0){\rule[-0.200pt]{2.409pt}{0.400pt}}
\put(170.0,577.0){\rule[-0.200pt]{2.409pt}{0.400pt}}
\put(1429.0,577.0){\rule[-0.200pt]{2.409pt}{0.400pt}}
\put(170.0,578.0){\rule[-0.200pt]{2.409pt}{0.400pt}}
\put(1429.0,578.0){\rule[-0.200pt]{2.409pt}{0.400pt}}
\put(170.0,578.0){\rule[-0.200pt]{2.409pt}{0.400pt}}
\put(1429.0,578.0){\rule[-0.200pt]{2.409pt}{0.400pt}}
\put(170.0,578.0){\rule[-0.200pt]{2.409pt}{0.400pt}}
\put(1429.0,578.0){\rule[-0.200pt]{2.409pt}{0.400pt}}
\put(170.0,578.0){\rule[-0.200pt]{2.409pt}{0.400pt}}
\put(1429.0,578.0){\rule[-0.200pt]{2.409pt}{0.400pt}}
\put(170.0,578.0){\rule[-0.200pt]{2.409pt}{0.400pt}}
\put(1429.0,578.0){\rule[-0.200pt]{2.409pt}{0.400pt}}
\put(170.0,578.0){\rule[-0.200pt]{2.409pt}{0.400pt}}
\put(1429.0,578.0){\rule[-0.200pt]{2.409pt}{0.400pt}}
\put(170.0,578.0){\rule[-0.200pt]{2.409pt}{0.400pt}}
\put(1429.0,578.0){\rule[-0.200pt]{2.409pt}{0.400pt}}
\put(170.0,578.0){\rule[-0.200pt]{2.409pt}{0.400pt}}
\put(1429.0,578.0){\rule[-0.200pt]{2.409pt}{0.400pt}}
\put(170.0,578.0){\rule[-0.200pt]{2.409pt}{0.400pt}}
\put(1429.0,578.0){\rule[-0.200pt]{2.409pt}{0.400pt}}
\put(170.0,578.0){\rule[-0.200pt]{2.409pt}{0.400pt}}
\put(1429.0,578.0){\rule[-0.200pt]{2.409pt}{0.400pt}}
\put(170.0,578.0){\rule[-0.200pt]{2.409pt}{0.400pt}}
\put(1429.0,578.0){\rule[-0.200pt]{2.409pt}{0.400pt}}
\put(170.0,578.0){\rule[-0.200pt]{2.409pt}{0.400pt}}
\put(1429.0,578.0){\rule[-0.200pt]{2.409pt}{0.400pt}}
\put(170.0,578.0){\rule[-0.200pt]{2.409pt}{0.400pt}}
\put(1429.0,578.0){\rule[-0.200pt]{2.409pt}{0.400pt}}
\put(170.0,578.0){\rule[-0.200pt]{2.409pt}{0.400pt}}
\put(1429.0,578.0){\rule[-0.200pt]{2.409pt}{0.400pt}}
\put(170.0,578.0){\rule[-0.200pt]{2.409pt}{0.400pt}}
\put(1429.0,578.0){\rule[-0.200pt]{2.409pt}{0.400pt}}
\put(170.0,579.0){\rule[-0.200pt]{2.409pt}{0.400pt}}
\put(1429.0,579.0){\rule[-0.200pt]{2.409pt}{0.400pt}}
\put(170.0,579.0){\rule[-0.200pt]{2.409pt}{0.400pt}}
\put(1429.0,579.0){\rule[-0.200pt]{2.409pt}{0.400pt}}
\put(170.0,579.0){\rule[-0.200pt]{2.409pt}{0.400pt}}
\put(1429.0,579.0){\rule[-0.200pt]{2.409pt}{0.400pt}}
\put(170.0,579.0){\rule[-0.200pt]{2.409pt}{0.400pt}}
\put(1429.0,579.0){\rule[-0.200pt]{2.409pt}{0.400pt}}
\put(170.0,579.0){\rule[-0.200pt]{2.409pt}{0.400pt}}
\put(1429.0,579.0){\rule[-0.200pt]{2.409pt}{0.400pt}}
\put(170.0,579.0){\rule[-0.200pt]{2.409pt}{0.400pt}}
\put(1429.0,579.0){\rule[-0.200pt]{2.409pt}{0.400pt}}
\put(170.0,579.0){\rule[-0.200pt]{2.409pt}{0.400pt}}
\put(1429.0,579.0){\rule[-0.200pt]{2.409pt}{0.400pt}}
\put(170.0,579.0){\rule[-0.200pt]{2.409pt}{0.400pt}}
\put(1429.0,579.0){\rule[-0.200pt]{2.409pt}{0.400pt}}
\put(170.0,579.0){\rule[-0.200pt]{2.409pt}{0.400pt}}
\put(1429.0,579.0){\rule[-0.200pt]{2.409pt}{0.400pt}}
\put(170.0,579.0){\rule[-0.200pt]{2.409pt}{0.400pt}}
\put(1429.0,579.0){\rule[-0.200pt]{2.409pt}{0.400pt}}
\put(170.0,579.0){\rule[-0.200pt]{2.409pt}{0.400pt}}
\put(1429.0,579.0){\rule[-0.200pt]{2.409pt}{0.400pt}}
\put(170.0,579.0){\rule[-0.200pt]{2.409pt}{0.400pt}}
\put(1429.0,579.0){\rule[-0.200pt]{2.409pt}{0.400pt}}
\put(170.0,579.0){\rule[-0.200pt]{2.409pt}{0.400pt}}
\put(1429.0,579.0){\rule[-0.200pt]{2.409pt}{0.400pt}}
\put(170.0,579.0){\rule[-0.200pt]{2.409pt}{0.400pt}}
\put(1429.0,579.0){\rule[-0.200pt]{2.409pt}{0.400pt}}
\put(170.0,579.0){\rule[-0.200pt]{2.409pt}{0.400pt}}
\put(1429.0,579.0){\rule[-0.200pt]{2.409pt}{0.400pt}}
\put(170.0,580.0){\rule[-0.200pt]{2.409pt}{0.400pt}}
\put(1429.0,580.0){\rule[-0.200pt]{2.409pt}{0.400pt}}
\put(170.0,580.0){\rule[-0.200pt]{2.409pt}{0.400pt}}
\put(1429.0,580.0){\rule[-0.200pt]{2.409pt}{0.400pt}}
\put(170.0,580.0){\rule[-0.200pt]{2.409pt}{0.400pt}}
\put(1429.0,580.0){\rule[-0.200pt]{2.409pt}{0.400pt}}
\put(170.0,580.0){\rule[-0.200pt]{2.409pt}{0.400pt}}
\put(1429.0,580.0){\rule[-0.200pt]{2.409pt}{0.400pt}}
\put(170.0,580.0){\rule[-0.200pt]{2.409pt}{0.400pt}}
\put(1429.0,580.0){\rule[-0.200pt]{2.409pt}{0.400pt}}
\put(170.0,580.0){\rule[-0.200pt]{2.409pt}{0.400pt}}
\put(1429.0,580.0){\rule[-0.200pt]{2.409pt}{0.400pt}}
\put(170.0,580.0){\rule[-0.200pt]{2.409pt}{0.400pt}}
\put(1429.0,580.0){\rule[-0.200pt]{2.409pt}{0.400pt}}
\put(170.0,580.0){\rule[-0.200pt]{2.409pt}{0.400pt}}
\put(1429.0,580.0){\rule[-0.200pt]{2.409pt}{0.400pt}}
\put(170.0,580.0){\rule[-0.200pt]{2.409pt}{0.400pt}}
\put(1429.0,580.0){\rule[-0.200pt]{2.409pt}{0.400pt}}
\put(170.0,580.0){\rule[-0.200pt]{2.409pt}{0.400pt}}
\put(1429.0,580.0){\rule[-0.200pt]{2.409pt}{0.400pt}}
\put(170.0,580.0){\rule[-0.200pt]{2.409pt}{0.400pt}}
\put(1429.0,580.0){\rule[-0.200pt]{2.409pt}{0.400pt}}
\put(170.0,580.0){\rule[-0.200pt]{2.409pt}{0.400pt}}
\put(1429.0,580.0){\rule[-0.200pt]{2.409pt}{0.400pt}}
\put(170.0,580.0){\rule[-0.200pt]{2.409pt}{0.400pt}}
\put(1429.0,580.0){\rule[-0.200pt]{2.409pt}{0.400pt}}
\put(170.0,580.0){\rule[-0.200pt]{2.409pt}{0.400pt}}
\put(1429.0,580.0){\rule[-0.200pt]{2.409pt}{0.400pt}}
\put(170.0,580.0){\rule[-0.200pt]{2.409pt}{0.400pt}}
\put(1429.0,580.0){\rule[-0.200pt]{2.409pt}{0.400pt}}
\put(170.0,580.0){\rule[-0.200pt]{2.409pt}{0.400pt}}
\put(1429.0,580.0){\rule[-0.200pt]{2.409pt}{0.400pt}}
\put(170.0,581.0){\rule[-0.200pt]{2.409pt}{0.400pt}}
\put(1429.0,581.0){\rule[-0.200pt]{2.409pt}{0.400pt}}
\put(170.0,581.0){\rule[-0.200pt]{2.409pt}{0.400pt}}
\put(1429.0,581.0){\rule[-0.200pt]{2.409pt}{0.400pt}}
\put(170.0,581.0){\rule[-0.200pt]{2.409pt}{0.400pt}}
\put(1429.0,581.0){\rule[-0.200pt]{2.409pt}{0.400pt}}
\put(170.0,581.0){\rule[-0.200pt]{2.409pt}{0.400pt}}
\put(1429.0,581.0){\rule[-0.200pt]{2.409pt}{0.400pt}}
\put(170.0,581.0){\rule[-0.200pt]{2.409pt}{0.400pt}}
\put(1429.0,581.0){\rule[-0.200pt]{2.409pt}{0.400pt}}
\put(170.0,581.0){\rule[-0.200pt]{2.409pt}{0.400pt}}
\put(1429.0,581.0){\rule[-0.200pt]{2.409pt}{0.400pt}}
\put(170.0,581.0){\rule[-0.200pt]{2.409pt}{0.400pt}}
\put(1429.0,581.0){\rule[-0.200pt]{2.409pt}{0.400pt}}
\put(170.0,581.0){\rule[-0.200pt]{2.409pt}{0.400pt}}
\put(1429.0,581.0){\rule[-0.200pt]{2.409pt}{0.400pt}}
\put(170.0,581.0){\rule[-0.200pt]{2.409pt}{0.400pt}}
\put(1429.0,581.0){\rule[-0.200pt]{2.409pt}{0.400pt}}
\put(170.0,581.0){\rule[-0.200pt]{2.409pt}{0.400pt}}
\put(1429.0,581.0){\rule[-0.200pt]{2.409pt}{0.400pt}}
\put(170.0,581.0){\rule[-0.200pt]{2.409pt}{0.400pt}}
\put(1429.0,581.0){\rule[-0.200pt]{2.409pt}{0.400pt}}
\put(170.0,581.0){\rule[-0.200pt]{2.409pt}{0.400pt}}
\put(1429.0,581.0){\rule[-0.200pt]{2.409pt}{0.400pt}}
\put(170.0,581.0){\rule[-0.200pt]{2.409pt}{0.400pt}}
\put(1429.0,581.0){\rule[-0.200pt]{2.409pt}{0.400pt}}
\put(170.0,581.0){\rule[-0.200pt]{2.409pt}{0.400pt}}
\put(1429.0,581.0){\rule[-0.200pt]{2.409pt}{0.400pt}}
\put(170.0,581.0){\rule[-0.200pt]{2.409pt}{0.400pt}}
\put(1429.0,581.0){\rule[-0.200pt]{2.409pt}{0.400pt}}
\put(170.0,581.0){\rule[-0.200pt]{2.409pt}{0.400pt}}
\put(1429.0,581.0){\rule[-0.200pt]{2.409pt}{0.400pt}}
\put(170.0,582.0){\rule[-0.200pt]{2.409pt}{0.400pt}}
\put(1429.0,582.0){\rule[-0.200pt]{2.409pt}{0.400pt}}
\put(170.0,582.0){\rule[-0.200pt]{2.409pt}{0.400pt}}
\put(1429.0,582.0){\rule[-0.200pt]{2.409pt}{0.400pt}}
\put(170.0,582.0){\rule[-0.200pt]{2.409pt}{0.400pt}}
\put(1429.0,582.0){\rule[-0.200pt]{2.409pt}{0.400pt}}
\put(170.0,582.0){\rule[-0.200pt]{2.409pt}{0.400pt}}
\put(1429.0,582.0){\rule[-0.200pt]{2.409pt}{0.400pt}}
\put(170.0,582.0){\rule[-0.200pt]{2.409pt}{0.400pt}}
\put(1429.0,582.0){\rule[-0.200pt]{2.409pt}{0.400pt}}
\put(170.0,582.0){\rule[-0.200pt]{2.409pt}{0.400pt}}
\put(1429.0,582.0){\rule[-0.200pt]{2.409pt}{0.400pt}}
\put(170.0,582.0){\rule[-0.200pt]{2.409pt}{0.400pt}}
\put(1429.0,582.0){\rule[-0.200pt]{2.409pt}{0.400pt}}
\put(170.0,582.0){\rule[-0.200pt]{2.409pt}{0.400pt}}
\put(1429.0,582.0){\rule[-0.200pt]{2.409pt}{0.400pt}}
\put(170.0,582.0){\rule[-0.200pt]{2.409pt}{0.400pt}}
\put(1429.0,582.0){\rule[-0.200pt]{2.409pt}{0.400pt}}
\put(170.0,582.0){\rule[-0.200pt]{2.409pt}{0.400pt}}
\put(1429.0,582.0){\rule[-0.200pt]{2.409pt}{0.400pt}}
\put(170.0,582.0){\rule[-0.200pt]{2.409pt}{0.400pt}}
\put(1429.0,582.0){\rule[-0.200pt]{2.409pt}{0.400pt}}
\put(170.0,582.0){\rule[-0.200pt]{2.409pt}{0.400pt}}
\put(1429.0,582.0){\rule[-0.200pt]{2.409pt}{0.400pt}}
\put(170.0,582.0){\rule[-0.200pt]{2.409pt}{0.400pt}}
\put(1429.0,582.0){\rule[-0.200pt]{2.409pt}{0.400pt}}
\put(170.0,582.0){\rule[-0.200pt]{2.409pt}{0.400pt}}
\put(1429.0,582.0){\rule[-0.200pt]{2.409pt}{0.400pt}}
\put(170.0,582.0){\rule[-0.200pt]{2.409pt}{0.400pt}}
\put(1429.0,582.0){\rule[-0.200pt]{2.409pt}{0.400pt}}
\put(170.0,582.0){\rule[-0.200pt]{2.409pt}{0.400pt}}
\put(1429.0,582.0){\rule[-0.200pt]{2.409pt}{0.400pt}}
\put(170.0,582.0){\rule[-0.200pt]{2.409pt}{0.400pt}}
\put(1429.0,582.0){\rule[-0.200pt]{2.409pt}{0.400pt}}
\put(170.0,583.0){\rule[-0.200pt]{2.409pt}{0.400pt}}
\put(1429.0,583.0){\rule[-0.200pt]{2.409pt}{0.400pt}}
\put(170.0,583.0){\rule[-0.200pt]{2.409pt}{0.400pt}}
\put(1429.0,583.0){\rule[-0.200pt]{2.409pt}{0.400pt}}
\put(170.0,583.0){\rule[-0.200pt]{2.409pt}{0.400pt}}
\put(1429.0,583.0){\rule[-0.200pt]{2.409pt}{0.400pt}}
\put(170.0,583.0){\rule[-0.200pt]{2.409pt}{0.400pt}}
\put(1429.0,583.0){\rule[-0.200pt]{2.409pt}{0.400pt}}
\put(170.0,583.0){\rule[-0.200pt]{2.409pt}{0.400pt}}
\put(1429.0,583.0){\rule[-0.200pt]{2.409pt}{0.400pt}}
\put(170.0,583.0){\rule[-0.200pt]{2.409pt}{0.400pt}}
\put(1429.0,583.0){\rule[-0.200pt]{2.409pt}{0.400pt}}
\put(170.0,583.0){\rule[-0.200pt]{2.409pt}{0.400pt}}
\put(1429.0,583.0){\rule[-0.200pt]{2.409pt}{0.400pt}}
\put(170.0,583.0){\rule[-0.200pt]{2.409pt}{0.400pt}}
\put(1429.0,583.0){\rule[-0.200pt]{2.409pt}{0.400pt}}
\put(170.0,583.0){\rule[-0.200pt]{2.409pt}{0.400pt}}
\put(1429.0,583.0){\rule[-0.200pt]{2.409pt}{0.400pt}}
\put(170.0,583.0){\rule[-0.200pt]{2.409pt}{0.400pt}}
\put(1429.0,583.0){\rule[-0.200pt]{2.409pt}{0.400pt}}
\put(170.0,583.0){\rule[-0.200pt]{2.409pt}{0.400pt}}
\put(1429.0,583.0){\rule[-0.200pt]{2.409pt}{0.400pt}}
\put(170.0,583.0){\rule[-0.200pt]{2.409pt}{0.400pt}}
\put(1429.0,583.0){\rule[-0.200pt]{2.409pt}{0.400pt}}
\put(170.0,583.0){\rule[-0.200pt]{2.409pt}{0.400pt}}
\put(1429.0,583.0){\rule[-0.200pt]{2.409pt}{0.400pt}}
\put(170.0,583.0){\rule[-0.200pt]{2.409pt}{0.400pt}}
\put(1429.0,583.0){\rule[-0.200pt]{2.409pt}{0.400pt}}
\put(170.0,583.0){\rule[-0.200pt]{2.409pt}{0.400pt}}
\put(1429.0,583.0){\rule[-0.200pt]{2.409pt}{0.400pt}}
\put(170.0,583.0){\rule[-0.200pt]{2.409pt}{0.400pt}}
\put(1429.0,583.0){\rule[-0.200pt]{2.409pt}{0.400pt}}
\put(170.0,584.0){\rule[-0.200pt]{2.409pt}{0.400pt}}
\put(1429.0,584.0){\rule[-0.200pt]{2.409pt}{0.400pt}}
\put(170.0,584.0){\rule[-0.200pt]{2.409pt}{0.400pt}}
\put(1429.0,584.0){\rule[-0.200pt]{2.409pt}{0.400pt}}
\put(170.0,584.0){\rule[-0.200pt]{2.409pt}{0.400pt}}
\put(1429.0,584.0){\rule[-0.200pt]{2.409pt}{0.400pt}}
\put(170.0,584.0){\rule[-0.200pt]{2.409pt}{0.400pt}}
\put(1429.0,584.0){\rule[-0.200pt]{2.409pt}{0.400pt}}
\put(170.0,584.0){\rule[-0.200pt]{2.409pt}{0.400pt}}
\put(1429.0,584.0){\rule[-0.200pt]{2.409pt}{0.400pt}}
\put(170.0,584.0){\rule[-0.200pt]{2.409pt}{0.400pt}}
\put(1429.0,584.0){\rule[-0.200pt]{2.409pt}{0.400pt}}
\put(170.0,584.0){\rule[-0.200pt]{2.409pt}{0.400pt}}
\put(1429.0,584.0){\rule[-0.200pt]{2.409pt}{0.400pt}}
\put(170.0,584.0){\rule[-0.200pt]{2.409pt}{0.400pt}}
\put(1429.0,584.0){\rule[-0.200pt]{2.409pt}{0.400pt}}
\put(170.0,584.0){\rule[-0.200pt]{2.409pt}{0.400pt}}
\put(1429.0,584.0){\rule[-0.200pt]{2.409pt}{0.400pt}}
\put(170.0,584.0){\rule[-0.200pt]{2.409pt}{0.400pt}}
\put(1429.0,584.0){\rule[-0.200pt]{2.409pt}{0.400pt}}
\put(170.0,584.0){\rule[-0.200pt]{2.409pt}{0.400pt}}
\put(1429.0,584.0){\rule[-0.200pt]{2.409pt}{0.400pt}}
\put(170.0,584.0){\rule[-0.200pt]{2.409pt}{0.400pt}}
\put(1429.0,584.0){\rule[-0.200pt]{2.409pt}{0.400pt}}
\put(170.0,584.0){\rule[-0.200pt]{2.409pt}{0.400pt}}
\put(1429.0,584.0){\rule[-0.200pt]{2.409pt}{0.400pt}}
\put(170.0,584.0){\rule[-0.200pt]{2.409pt}{0.400pt}}
\put(1429.0,584.0){\rule[-0.200pt]{2.409pt}{0.400pt}}
\put(170.0,584.0){\rule[-0.200pt]{2.409pt}{0.400pt}}
\put(1429.0,584.0){\rule[-0.200pt]{2.409pt}{0.400pt}}
\put(170.0,584.0){\rule[-0.200pt]{2.409pt}{0.400pt}}
\put(1429.0,584.0){\rule[-0.200pt]{2.409pt}{0.400pt}}
\put(170.0,584.0){\rule[-0.200pt]{2.409pt}{0.400pt}}
\put(1429.0,584.0){\rule[-0.200pt]{2.409pt}{0.400pt}}
\put(170.0,584.0){\rule[-0.200pt]{2.409pt}{0.400pt}}
\put(1429.0,584.0){\rule[-0.200pt]{2.409pt}{0.400pt}}
\put(170.0,585.0){\rule[-0.200pt]{2.409pt}{0.400pt}}
\put(1429.0,585.0){\rule[-0.200pt]{2.409pt}{0.400pt}}
\put(170.0,585.0){\rule[-0.200pt]{2.409pt}{0.400pt}}
\put(1429.0,585.0){\rule[-0.200pt]{2.409pt}{0.400pt}}
\put(170.0,585.0){\rule[-0.200pt]{2.409pt}{0.400pt}}
\put(1429.0,585.0){\rule[-0.200pt]{2.409pt}{0.400pt}}
\put(170.0,585.0){\rule[-0.200pt]{2.409pt}{0.400pt}}
\put(1429.0,585.0){\rule[-0.200pt]{2.409pt}{0.400pt}}
\put(170.0,585.0){\rule[-0.200pt]{2.409pt}{0.400pt}}
\put(1429.0,585.0){\rule[-0.200pt]{2.409pt}{0.400pt}}
\put(170.0,585.0){\rule[-0.200pt]{2.409pt}{0.400pt}}
\put(1429.0,585.0){\rule[-0.200pt]{2.409pt}{0.400pt}}
\put(170.0,585.0){\rule[-0.200pt]{2.409pt}{0.400pt}}
\put(1429.0,585.0){\rule[-0.200pt]{2.409pt}{0.400pt}}
\put(170.0,585.0){\rule[-0.200pt]{2.409pt}{0.400pt}}
\put(1429.0,585.0){\rule[-0.200pt]{2.409pt}{0.400pt}}
\put(170.0,585.0){\rule[-0.200pt]{2.409pt}{0.400pt}}
\put(1429.0,585.0){\rule[-0.200pt]{2.409pt}{0.400pt}}
\put(170.0,585.0){\rule[-0.200pt]{2.409pt}{0.400pt}}
\put(1429.0,585.0){\rule[-0.200pt]{2.409pt}{0.400pt}}
\put(170.0,585.0){\rule[-0.200pt]{2.409pt}{0.400pt}}
\put(1429.0,585.0){\rule[-0.200pt]{2.409pt}{0.400pt}}
\put(170.0,585.0){\rule[-0.200pt]{2.409pt}{0.400pt}}
\put(1429.0,585.0){\rule[-0.200pt]{2.409pt}{0.400pt}}
\put(170.0,585.0){\rule[-0.200pt]{2.409pt}{0.400pt}}
\put(1429.0,585.0){\rule[-0.200pt]{2.409pt}{0.400pt}}
\put(170.0,585.0){\rule[-0.200pt]{2.409pt}{0.400pt}}
\put(1429.0,585.0){\rule[-0.200pt]{2.409pt}{0.400pt}}
\put(170.0,585.0){\rule[-0.200pt]{2.409pt}{0.400pt}}
\put(1429.0,585.0){\rule[-0.200pt]{2.409pt}{0.400pt}}
\put(170.0,585.0){\rule[-0.200pt]{2.409pt}{0.400pt}}
\put(1429.0,585.0){\rule[-0.200pt]{2.409pt}{0.400pt}}
\put(170.0,585.0){\rule[-0.200pt]{2.409pt}{0.400pt}}
\put(1429.0,585.0){\rule[-0.200pt]{2.409pt}{0.400pt}}
\put(170.0,585.0){\rule[-0.200pt]{2.409pt}{0.400pt}}
\put(1429.0,585.0){\rule[-0.200pt]{2.409pt}{0.400pt}}
\put(170.0,586.0){\rule[-0.200pt]{2.409pt}{0.400pt}}
\put(1429.0,586.0){\rule[-0.200pt]{2.409pt}{0.400pt}}
\put(170.0,586.0){\rule[-0.200pt]{2.409pt}{0.400pt}}
\put(1429.0,586.0){\rule[-0.200pt]{2.409pt}{0.400pt}}
\put(170.0,586.0){\rule[-0.200pt]{2.409pt}{0.400pt}}
\put(1429.0,586.0){\rule[-0.200pt]{2.409pt}{0.400pt}}
\put(170.0,586.0){\rule[-0.200pt]{2.409pt}{0.400pt}}
\put(1429.0,586.0){\rule[-0.200pt]{2.409pt}{0.400pt}}
\put(170.0,586.0){\rule[-0.200pt]{2.409pt}{0.400pt}}
\put(1429.0,586.0){\rule[-0.200pt]{2.409pt}{0.400pt}}
\put(170.0,586.0){\rule[-0.200pt]{2.409pt}{0.400pt}}
\put(1429.0,586.0){\rule[-0.200pt]{2.409pt}{0.400pt}}
\put(170.0,586.0){\rule[-0.200pt]{2.409pt}{0.400pt}}
\put(1429.0,586.0){\rule[-0.200pt]{2.409pt}{0.400pt}}
\put(170.0,586.0){\rule[-0.200pt]{2.409pt}{0.400pt}}
\put(1429.0,586.0){\rule[-0.200pt]{2.409pt}{0.400pt}}
\put(170.0,586.0){\rule[-0.200pt]{2.409pt}{0.400pt}}
\put(1429.0,586.0){\rule[-0.200pt]{2.409pt}{0.400pt}}
\put(170.0,586.0){\rule[-0.200pt]{2.409pt}{0.400pt}}
\put(1429.0,586.0){\rule[-0.200pt]{2.409pt}{0.400pt}}
\put(170.0,586.0){\rule[-0.200pt]{2.409pt}{0.400pt}}
\put(1429.0,586.0){\rule[-0.200pt]{2.409pt}{0.400pt}}
\put(170.0,586.0){\rule[-0.200pt]{2.409pt}{0.400pt}}
\put(1429.0,586.0){\rule[-0.200pt]{2.409pt}{0.400pt}}
\put(170.0,586.0){\rule[-0.200pt]{2.409pt}{0.400pt}}
\put(1429.0,586.0){\rule[-0.200pt]{2.409pt}{0.400pt}}
\put(170.0,586.0){\rule[-0.200pt]{2.409pt}{0.400pt}}
\put(1429.0,586.0){\rule[-0.200pt]{2.409pt}{0.400pt}}
\put(170.0,586.0){\rule[-0.200pt]{2.409pt}{0.400pt}}
\put(1429.0,586.0){\rule[-0.200pt]{2.409pt}{0.400pt}}
\put(170.0,586.0){\rule[-0.200pt]{2.409pt}{0.400pt}}
\put(1429.0,586.0){\rule[-0.200pt]{2.409pt}{0.400pt}}
\put(170.0,586.0){\rule[-0.200pt]{2.409pt}{0.400pt}}
\put(1429.0,586.0){\rule[-0.200pt]{2.409pt}{0.400pt}}
\put(170.0,586.0){\rule[-0.200pt]{2.409pt}{0.400pt}}
\put(1429.0,586.0){\rule[-0.200pt]{2.409pt}{0.400pt}}
\put(170.0,587.0){\rule[-0.200pt]{2.409pt}{0.400pt}}
\put(1429.0,587.0){\rule[-0.200pt]{2.409pt}{0.400pt}}
\put(170.0,587.0){\rule[-0.200pt]{2.409pt}{0.400pt}}
\put(1429.0,587.0){\rule[-0.200pt]{2.409pt}{0.400pt}}
\put(170.0,587.0){\rule[-0.200pt]{2.409pt}{0.400pt}}
\put(1429.0,587.0){\rule[-0.200pt]{2.409pt}{0.400pt}}
\put(170.0,587.0){\rule[-0.200pt]{2.409pt}{0.400pt}}
\put(1429.0,587.0){\rule[-0.200pt]{2.409pt}{0.400pt}}
\put(170.0,587.0){\rule[-0.200pt]{2.409pt}{0.400pt}}
\put(1429.0,587.0){\rule[-0.200pt]{2.409pt}{0.400pt}}
\put(170.0,587.0){\rule[-0.200pt]{2.409pt}{0.400pt}}
\put(1429.0,587.0){\rule[-0.200pt]{2.409pt}{0.400pt}}
\put(170.0,587.0){\rule[-0.200pt]{2.409pt}{0.400pt}}
\put(1429.0,587.0){\rule[-0.200pt]{2.409pt}{0.400pt}}
\put(170.0,587.0){\rule[-0.200pt]{2.409pt}{0.400pt}}
\put(1429.0,587.0){\rule[-0.200pt]{2.409pt}{0.400pt}}
\put(170.0,587.0){\rule[-0.200pt]{2.409pt}{0.400pt}}
\put(1429.0,587.0){\rule[-0.200pt]{2.409pt}{0.400pt}}
\put(170.0,587.0){\rule[-0.200pt]{2.409pt}{0.400pt}}
\put(1429.0,587.0){\rule[-0.200pt]{2.409pt}{0.400pt}}
\put(170.0,587.0){\rule[-0.200pt]{2.409pt}{0.400pt}}
\put(1429.0,587.0){\rule[-0.200pt]{2.409pt}{0.400pt}}
\put(170.0,587.0){\rule[-0.200pt]{2.409pt}{0.400pt}}
\put(1429.0,587.0){\rule[-0.200pt]{2.409pt}{0.400pt}}
\put(170.0,587.0){\rule[-0.200pt]{2.409pt}{0.400pt}}
\put(1429.0,587.0){\rule[-0.200pt]{2.409pt}{0.400pt}}
\put(170.0,587.0){\rule[-0.200pt]{2.409pt}{0.400pt}}
\put(1429.0,587.0){\rule[-0.200pt]{2.409pt}{0.400pt}}
\put(170.0,587.0){\rule[-0.200pt]{2.409pt}{0.400pt}}
\put(1429.0,587.0){\rule[-0.200pt]{2.409pt}{0.400pt}}
\put(170.0,587.0){\rule[-0.200pt]{2.409pt}{0.400pt}}
\put(1429.0,587.0){\rule[-0.200pt]{2.409pt}{0.400pt}}
\put(170.0,587.0){\rule[-0.200pt]{2.409pt}{0.400pt}}
\put(1429.0,587.0){\rule[-0.200pt]{2.409pt}{0.400pt}}
\put(170.0,587.0){\rule[-0.200pt]{2.409pt}{0.400pt}}
\put(1429.0,587.0){\rule[-0.200pt]{2.409pt}{0.400pt}}
\put(170.0,587.0){\rule[-0.200pt]{2.409pt}{0.400pt}}
\put(1429.0,587.0){\rule[-0.200pt]{2.409pt}{0.400pt}}
\put(170.0,588.0){\rule[-0.200pt]{2.409pt}{0.400pt}}
\put(1429.0,588.0){\rule[-0.200pt]{2.409pt}{0.400pt}}
\put(170.0,588.0){\rule[-0.200pt]{2.409pt}{0.400pt}}
\put(1429.0,588.0){\rule[-0.200pt]{2.409pt}{0.400pt}}
\put(170.0,588.0){\rule[-0.200pt]{2.409pt}{0.400pt}}
\put(1429.0,588.0){\rule[-0.200pt]{2.409pt}{0.400pt}}
\put(170.0,588.0){\rule[-0.200pt]{2.409pt}{0.400pt}}
\put(1429.0,588.0){\rule[-0.200pt]{2.409pt}{0.400pt}}
\put(170.0,588.0){\rule[-0.200pt]{2.409pt}{0.400pt}}
\put(1429.0,588.0){\rule[-0.200pt]{2.409pt}{0.400pt}}
\put(170.0,588.0){\rule[-0.200pt]{2.409pt}{0.400pt}}
\put(1429.0,588.0){\rule[-0.200pt]{2.409pt}{0.400pt}}
\put(170.0,588.0){\rule[-0.200pt]{2.409pt}{0.400pt}}
\put(1429.0,588.0){\rule[-0.200pt]{2.409pt}{0.400pt}}
\put(170.0,588.0){\rule[-0.200pt]{2.409pt}{0.400pt}}
\put(1429.0,588.0){\rule[-0.200pt]{2.409pt}{0.400pt}}
\put(170.0,588.0){\rule[-0.200pt]{2.409pt}{0.400pt}}
\put(1429.0,588.0){\rule[-0.200pt]{2.409pt}{0.400pt}}
\put(170.0,588.0){\rule[-0.200pt]{2.409pt}{0.400pt}}
\put(1429.0,588.0){\rule[-0.200pt]{2.409pt}{0.400pt}}
\put(170.0,588.0){\rule[-0.200pt]{2.409pt}{0.400pt}}
\put(1429.0,588.0){\rule[-0.200pt]{2.409pt}{0.400pt}}
\put(170.0,588.0){\rule[-0.200pt]{2.409pt}{0.400pt}}
\put(1429.0,588.0){\rule[-0.200pt]{2.409pt}{0.400pt}}
\put(170.0,588.0){\rule[-0.200pt]{2.409pt}{0.400pt}}
\put(1429.0,588.0){\rule[-0.200pt]{2.409pt}{0.400pt}}
\put(170.0,588.0){\rule[-0.200pt]{2.409pt}{0.400pt}}
\put(1429.0,588.0){\rule[-0.200pt]{2.409pt}{0.400pt}}
\put(170.0,588.0){\rule[-0.200pt]{2.409pt}{0.400pt}}
\put(1429.0,588.0){\rule[-0.200pt]{2.409pt}{0.400pt}}
\put(170.0,588.0){\rule[-0.200pt]{2.409pt}{0.400pt}}
\put(1429.0,588.0){\rule[-0.200pt]{2.409pt}{0.400pt}}
\put(170.0,588.0){\rule[-0.200pt]{2.409pt}{0.400pt}}
\put(1429.0,588.0){\rule[-0.200pt]{2.409pt}{0.400pt}}
\put(170.0,588.0){\rule[-0.200pt]{2.409pt}{0.400pt}}
\put(1429.0,588.0){\rule[-0.200pt]{2.409pt}{0.400pt}}
\put(170.0,588.0){\rule[-0.200pt]{2.409pt}{0.400pt}}
\put(1429.0,588.0){\rule[-0.200pt]{2.409pt}{0.400pt}}
\put(170.0,589.0){\rule[-0.200pt]{2.409pt}{0.400pt}}
\put(1429.0,589.0){\rule[-0.200pt]{2.409pt}{0.400pt}}
\put(170.0,589.0){\rule[-0.200pt]{2.409pt}{0.400pt}}
\put(1429.0,589.0){\rule[-0.200pt]{2.409pt}{0.400pt}}
\put(170.0,589.0){\rule[-0.200pt]{2.409pt}{0.400pt}}
\put(1429.0,589.0){\rule[-0.200pt]{2.409pt}{0.400pt}}
\put(170.0,589.0){\rule[-0.200pt]{2.409pt}{0.400pt}}
\put(1429.0,589.0){\rule[-0.200pt]{2.409pt}{0.400pt}}
\put(170.0,589.0){\rule[-0.200pt]{2.409pt}{0.400pt}}
\put(1429.0,589.0){\rule[-0.200pt]{2.409pt}{0.400pt}}
\put(170.0,589.0){\rule[-0.200pt]{2.409pt}{0.400pt}}
\put(1429.0,589.0){\rule[-0.200pt]{2.409pt}{0.400pt}}
\put(170.0,589.0){\rule[-0.200pt]{2.409pt}{0.400pt}}
\put(1429.0,589.0){\rule[-0.200pt]{2.409pt}{0.400pt}}
\put(170.0,589.0){\rule[-0.200pt]{2.409pt}{0.400pt}}
\put(1429.0,589.0){\rule[-0.200pt]{2.409pt}{0.400pt}}
\put(170.0,589.0){\rule[-0.200pt]{2.409pt}{0.400pt}}
\put(1429.0,589.0){\rule[-0.200pt]{2.409pt}{0.400pt}}
\put(170.0,589.0){\rule[-0.200pt]{2.409pt}{0.400pt}}
\put(1429.0,589.0){\rule[-0.200pt]{2.409pt}{0.400pt}}
\put(170.0,589.0){\rule[-0.200pt]{2.409pt}{0.400pt}}
\put(1429.0,589.0){\rule[-0.200pt]{2.409pt}{0.400pt}}
\put(170.0,589.0){\rule[-0.200pt]{2.409pt}{0.400pt}}
\put(1429.0,589.0){\rule[-0.200pt]{2.409pt}{0.400pt}}
\put(170.0,589.0){\rule[-0.200pt]{2.409pt}{0.400pt}}
\put(1429.0,589.0){\rule[-0.200pt]{2.409pt}{0.400pt}}
\put(170.0,589.0){\rule[-0.200pt]{2.409pt}{0.400pt}}
\put(1429.0,589.0){\rule[-0.200pt]{2.409pt}{0.400pt}}
\put(170.0,589.0){\rule[-0.200pt]{2.409pt}{0.400pt}}
\put(1429.0,589.0){\rule[-0.200pt]{2.409pt}{0.400pt}}
\put(170.0,589.0){\rule[-0.200pt]{2.409pt}{0.400pt}}
\put(1429.0,589.0){\rule[-0.200pt]{2.409pt}{0.400pt}}
\put(170.0,589.0){\rule[-0.200pt]{2.409pt}{0.400pt}}
\put(1429.0,589.0){\rule[-0.200pt]{2.409pt}{0.400pt}}
\put(170.0,589.0){\rule[-0.200pt]{2.409pt}{0.400pt}}
\put(1429.0,589.0){\rule[-0.200pt]{2.409pt}{0.400pt}}
\put(170.0,589.0){\rule[-0.200pt]{2.409pt}{0.400pt}}
\put(1429.0,589.0){\rule[-0.200pt]{2.409pt}{0.400pt}}
\put(170.0,589.0){\rule[-0.200pt]{2.409pt}{0.400pt}}
\put(1429.0,589.0){\rule[-0.200pt]{2.409pt}{0.400pt}}
\put(170.0,590.0){\rule[-0.200pt]{2.409pt}{0.400pt}}
\put(1429.0,590.0){\rule[-0.200pt]{2.409pt}{0.400pt}}
\put(170.0,590.0){\rule[-0.200pt]{2.409pt}{0.400pt}}
\put(1429.0,590.0){\rule[-0.200pt]{2.409pt}{0.400pt}}
\put(170.0,590.0){\rule[-0.200pt]{2.409pt}{0.400pt}}
\put(1429.0,590.0){\rule[-0.200pt]{2.409pt}{0.400pt}}
\put(170.0,590.0){\rule[-0.200pt]{2.409pt}{0.400pt}}
\put(1429.0,590.0){\rule[-0.200pt]{2.409pt}{0.400pt}}
\put(170.0,590.0){\rule[-0.200pt]{2.409pt}{0.400pt}}
\put(1429.0,590.0){\rule[-0.200pt]{2.409pt}{0.400pt}}
\put(170.0,590.0){\rule[-0.200pt]{2.409pt}{0.400pt}}
\put(1429.0,590.0){\rule[-0.200pt]{2.409pt}{0.400pt}}
\put(170.0,590.0){\rule[-0.200pt]{2.409pt}{0.400pt}}
\put(1429.0,590.0){\rule[-0.200pt]{2.409pt}{0.400pt}}
\put(170.0,590.0){\rule[-0.200pt]{2.409pt}{0.400pt}}
\put(1429.0,590.0){\rule[-0.200pt]{2.409pt}{0.400pt}}
\put(170.0,590.0){\rule[-0.200pt]{2.409pt}{0.400pt}}
\put(1429.0,590.0){\rule[-0.200pt]{2.409pt}{0.400pt}}
\put(170.0,590.0){\rule[-0.200pt]{2.409pt}{0.400pt}}
\put(1429.0,590.0){\rule[-0.200pt]{2.409pt}{0.400pt}}
\put(170.0,590.0){\rule[-0.200pt]{2.409pt}{0.400pt}}
\put(1429.0,590.0){\rule[-0.200pt]{2.409pt}{0.400pt}}
\put(170.0,590.0){\rule[-0.200pt]{2.409pt}{0.400pt}}
\put(1429.0,590.0){\rule[-0.200pt]{2.409pt}{0.400pt}}
\put(170.0,590.0){\rule[-0.200pt]{2.409pt}{0.400pt}}
\put(1429.0,590.0){\rule[-0.200pt]{2.409pt}{0.400pt}}
\put(170.0,590.0){\rule[-0.200pt]{2.409pt}{0.400pt}}
\put(1429.0,590.0){\rule[-0.200pt]{2.409pt}{0.400pt}}
\put(170.0,590.0){\rule[-0.200pt]{2.409pt}{0.400pt}}
\put(1429.0,590.0){\rule[-0.200pt]{2.409pt}{0.400pt}}
\put(170.0,590.0){\rule[-0.200pt]{2.409pt}{0.400pt}}
\put(1429.0,590.0){\rule[-0.200pt]{2.409pt}{0.400pt}}
\put(170.0,590.0){\rule[-0.200pt]{2.409pt}{0.400pt}}
\put(1429.0,590.0){\rule[-0.200pt]{2.409pt}{0.400pt}}
\put(170.0,590.0){\rule[-0.200pt]{2.409pt}{0.400pt}}
\put(1429.0,590.0){\rule[-0.200pt]{2.409pt}{0.400pt}}
\put(170.0,590.0){\rule[-0.200pt]{2.409pt}{0.400pt}}
\put(1429.0,590.0){\rule[-0.200pt]{2.409pt}{0.400pt}}
\put(170.0,590.0){\rule[-0.200pt]{2.409pt}{0.400pt}}
\put(1429.0,590.0){\rule[-0.200pt]{2.409pt}{0.400pt}}
\put(170.0,590.0){\rule[-0.200pt]{2.409pt}{0.400pt}}
\put(1429.0,590.0){\rule[-0.200pt]{2.409pt}{0.400pt}}
\put(170.0,591.0){\rule[-0.200pt]{2.409pt}{0.400pt}}
\put(1429.0,591.0){\rule[-0.200pt]{2.409pt}{0.400pt}}
\put(170.0,591.0){\rule[-0.200pt]{2.409pt}{0.400pt}}
\put(1429.0,591.0){\rule[-0.200pt]{2.409pt}{0.400pt}}
\put(170.0,591.0){\rule[-0.200pt]{2.409pt}{0.400pt}}
\put(1429.0,591.0){\rule[-0.200pt]{2.409pt}{0.400pt}}
\put(170.0,591.0){\rule[-0.200pt]{2.409pt}{0.400pt}}
\put(1429.0,591.0){\rule[-0.200pt]{2.409pt}{0.400pt}}
\put(170.0,591.0){\rule[-0.200pt]{2.409pt}{0.400pt}}
\put(1429.0,591.0){\rule[-0.200pt]{2.409pt}{0.400pt}}
\put(170.0,591.0){\rule[-0.200pt]{2.409pt}{0.400pt}}
\put(1429.0,591.0){\rule[-0.200pt]{2.409pt}{0.400pt}}
\put(170.0,591.0){\rule[-0.200pt]{2.409pt}{0.400pt}}
\put(1429.0,591.0){\rule[-0.200pt]{2.409pt}{0.400pt}}
\put(170.0,591.0){\rule[-0.200pt]{2.409pt}{0.400pt}}
\put(1429.0,591.0){\rule[-0.200pt]{2.409pt}{0.400pt}}
\put(170.0,591.0){\rule[-0.200pt]{2.409pt}{0.400pt}}
\put(1429.0,591.0){\rule[-0.200pt]{2.409pt}{0.400pt}}
\put(170.0,591.0){\rule[-0.200pt]{2.409pt}{0.400pt}}
\put(1429.0,591.0){\rule[-0.200pt]{2.409pt}{0.400pt}}
\put(170.0,591.0){\rule[-0.200pt]{2.409pt}{0.400pt}}
\put(1429.0,591.0){\rule[-0.200pt]{2.409pt}{0.400pt}}
\put(170.0,591.0){\rule[-0.200pt]{2.409pt}{0.400pt}}
\put(1429.0,591.0){\rule[-0.200pt]{2.409pt}{0.400pt}}
\put(170.0,591.0){\rule[-0.200pt]{2.409pt}{0.400pt}}
\put(1429.0,591.0){\rule[-0.200pt]{2.409pt}{0.400pt}}
\put(170.0,591.0){\rule[-0.200pt]{2.409pt}{0.400pt}}
\put(1429.0,591.0){\rule[-0.200pt]{2.409pt}{0.400pt}}
\put(170.0,591.0){\rule[-0.200pt]{2.409pt}{0.400pt}}
\put(1429.0,591.0){\rule[-0.200pt]{2.409pt}{0.400pt}}
\put(170.0,591.0){\rule[-0.200pt]{2.409pt}{0.400pt}}
\put(1429.0,591.0){\rule[-0.200pt]{2.409pt}{0.400pt}}
\put(170.0,591.0){\rule[-0.200pt]{2.409pt}{0.400pt}}
\put(1429.0,591.0){\rule[-0.200pt]{2.409pt}{0.400pt}}
\put(170.0,591.0){\rule[-0.200pt]{2.409pt}{0.400pt}}
\put(1429.0,591.0){\rule[-0.200pt]{2.409pt}{0.400pt}}
\put(170.0,591.0){\rule[-0.200pt]{2.409pt}{0.400pt}}
\put(1429.0,591.0){\rule[-0.200pt]{2.409pt}{0.400pt}}
\put(170.0,591.0){\rule[-0.200pt]{2.409pt}{0.400pt}}
\put(1429.0,591.0){\rule[-0.200pt]{2.409pt}{0.400pt}}
\put(170.0,591.0){\rule[-0.200pt]{2.409pt}{0.400pt}}
\put(1429.0,591.0){\rule[-0.200pt]{2.409pt}{0.400pt}}
\put(170.0,592.0){\rule[-0.200pt]{2.409pt}{0.400pt}}
\put(1429.0,592.0){\rule[-0.200pt]{2.409pt}{0.400pt}}
\put(170.0,592.0){\rule[-0.200pt]{2.409pt}{0.400pt}}
\put(1429.0,592.0){\rule[-0.200pt]{2.409pt}{0.400pt}}
\put(170.0,592.0){\rule[-0.200pt]{2.409pt}{0.400pt}}
\put(1429.0,592.0){\rule[-0.200pt]{2.409pt}{0.400pt}}
\put(170.0,592.0){\rule[-0.200pt]{2.409pt}{0.400pt}}
\put(1429.0,592.0){\rule[-0.200pt]{2.409pt}{0.400pt}}
\put(170.0,592.0){\rule[-0.200pt]{2.409pt}{0.400pt}}
\put(1429.0,592.0){\rule[-0.200pt]{2.409pt}{0.400pt}}
\put(170.0,592.0){\rule[-0.200pt]{2.409pt}{0.400pt}}
\put(1429.0,592.0){\rule[-0.200pt]{2.409pt}{0.400pt}}
\put(170.0,592.0){\rule[-0.200pt]{2.409pt}{0.400pt}}
\put(1429.0,592.0){\rule[-0.200pt]{2.409pt}{0.400pt}}
\put(170.0,592.0){\rule[-0.200pt]{2.409pt}{0.400pt}}
\put(1429.0,592.0){\rule[-0.200pt]{2.409pt}{0.400pt}}
\put(170.0,592.0){\rule[-0.200pt]{2.409pt}{0.400pt}}
\put(1429.0,592.0){\rule[-0.200pt]{2.409pt}{0.400pt}}
\put(170.0,592.0){\rule[-0.200pt]{2.409pt}{0.400pt}}
\put(1429.0,592.0){\rule[-0.200pt]{2.409pt}{0.400pt}}
\put(170.0,592.0){\rule[-0.200pt]{2.409pt}{0.400pt}}
\put(1429.0,592.0){\rule[-0.200pt]{2.409pt}{0.400pt}}
\put(170.0,592.0){\rule[-0.200pt]{2.409pt}{0.400pt}}
\put(1429.0,592.0){\rule[-0.200pt]{2.409pt}{0.400pt}}
\put(170.0,592.0){\rule[-0.200pt]{2.409pt}{0.400pt}}
\put(1429.0,592.0){\rule[-0.200pt]{2.409pt}{0.400pt}}
\put(170.0,592.0){\rule[-0.200pt]{2.409pt}{0.400pt}}
\put(1429.0,592.0){\rule[-0.200pt]{2.409pt}{0.400pt}}
\put(170.0,592.0){\rule[-0.200pt]{2.409pt}{0.400pt}}
\put(1429.0,592.0){\rule[-0.200pt]{2.409pt}{0.400pt}}
\put(170.0,592.0){\rule[-0.200pt]{2.409pt}{0.400pt}}
\put(1429.0,592.0){\rule[-0.200pt]{2.409pt}{0.400pt}}
\put(170.0,592.0){\rule[-0.200pt]{2.409pt}{0.400pt}}
\put(1429.0,592.0){\rule[-0.200pt]{2.409pt}{0.400pt}}
\put(170.0,592.0){\rule[-0.200pt]{2.409pt}{0.400pt}}
\put(1429.0,592.0){\rule[-0.200pt]{2.409pt}{0.400pt}}
\put(170.0,592.0){\rule[-0.200pt]{2.409pt}{0.400pt}}
\put(1429.0,592.0){\rule[-0.200pt]{2.409pt}{0.400pt}}
\put(170.0,592.0){\rule[-0.200pt]{2.409pt}{0.400pt}}
\put(1429.0,592.0){\rule[-0.200pt]{2.409pt}{0.400pt}}
\put(170.0,592.0){\rule[-0.200pt]{2.409pt}{0.400pt}}
\put(1429.0,592.0){\rule[-0.200pt]{2.409pt}{0.400pt}}
\put(170.0,593.0){\rule[-0.200pt]{2.409pt}{0.400pt}}
\put(1429.0,593.0){\rule[-0.200pt]{2.409pt}{0.400pt}}
\put(170.0,593.0){\rule[-0.200pt]{2.409pt}{0.400pt}}
\put(1429.0,593.0){\rule[-0.200pt]{2.409pt}{0.400pt}}
\put(170.0,593.0){\rule[-0.200pt]{2.409pt}{0.400pt}}
\put(1429.0,593.0){\rule[-0.200pt]{2.409pt}{0.400pt}}
\put(170.0,593.0){\rule[-0.200pt]{2.409pt}{0.400pt}}
\put(1429.0,593.0){\rule[-0.200pt]{2.409pt}{0.400pt}}
\put(170.0,593.0){\rule[-0.200pt]{2.409pt}{0.400pt}}
\put(1429.0,593.0){\rule[-0.200pt]{2.409pt}{0.400pt}}
\put(170.0,593.0){\rule[-0.200pt]{2.409pt}{0.400pt}}
\put(1429.0,593.0){\rule[-0.200pt]{2.409pt}{0.400pt}}
\put(170.0,593.0){\rule[-0.200pt]{2.409pt}{0.400pt}}
\put(1429.0,593.0){\rule[-0.200pt]{2.409pt}{0.400pt}}
\put(170.0,593.0){\rule[-0.200pt]{2.409pt}{0.400pt}}
\put(1429.0,593.0){\rule[-0.200pt]{2.409pt}{0.400pt}}
\put(170.0,593.0){\rule[-0.200pt]{2.409pt}{0.400pt}}
\put(1429.0,593.0){\rule[-0.200pt]{2.409pt}{0.400pt}}
\put(170.0,593.0){\rule[-0.200pt]{2.409pt}{0.400pt}}
\put(1429.0,593.0){\rule[-0.200pt]{2.409pt}{0.400pt}}
\put(170.0,593.0){\rule[-0.200pt]{2.409pt}{0.400pt}}
\put(1429.0,593.0){\rule[-0.200pt]{2.409pt}{0.400pt}}
\put(170.0,593.0){\rule[-0.200pt]{2.409pt}{0.400pt}}
\put(1429.0,593.0){\rule[-0.200pt]{2.409pt}{0.400pt}}
\put(170.0,593.0){\rule[-0.200pt]{2.409pt}{0.400pt}}
\put(1429.0,593.0){\rule[-0.200pt]{2.409pt}{0.400pt}}
\put(170.0,593.0){\rule[-0.200pt]{2.409pt}{0.400pt}}
\put(1429.0,593.0){\rule[-0.200pt]{2.409pt}{0.400pt}}
\put(170.0,593.0){\rule[-0.200pt]{2.409pt}{0.400pt}}
\put(1429.0,593.0){\rule[-0.200pt]{2.409pt}{0.400pt}}
\put(170.0,593.0){\rule[-0.200pt]{2.409pt}{0.400pt}}
\put(1429.0,593.0){\rule[-0.200pt]{2.409pt}{0.400pt}}
\put(170.0,593.0){\rule[-0.200pt]{2.409pt}{0.400pt}}
\put(1429.0,593.0){\rule[-0.200pt]{2.409pt}{0.400pt}}
\put(170.0,593.0){\rule[-0.200pt]{2.409pt}{0.400pt}}
\put(1429.0,593.0){\rule[-0.200pt]{2.409pt}{0.400pt}}
\put(170.0,593.0){\rule[-0.200pt]{2.409pt}{0.400pt}}
\put(1429.0,593.0){\rule[-0.200pt]{2.409pt}{0.400pt}}
\put(170.0,593.0){\rule[-0.200pt]{2.409pt}{0.400pt}}
\put(1429.0,593.0){\rule[-0.200pt]{2.409pt}{0.400pt}}
\put(170.0,593.0){\rule[-0.200pt]{2.409pt}{0.400pt}}
\put(1429.0,593.0){\rule[-0.200pt]{2.409pt}{0.400pt}}
\put(170.0,593.0){\rule[-0.200pt]{2.409pt}{0.400pt}}
\put(1429.0,593.0){\rule[-0.200pt]{2.409pt}{0.400pt}}
\put(170.0,594.0){\rule[-0.200pt]{2.409pt}{0.400pt}}
\put(1429.0,594.0){\rule[-0.200pt]{2.409pt}{0.400pt}}
\put(170.0,594.0){\rule[-0.200pt]{2.409pt}{0.400pt}}
\put(1429.0,594.0){\rule[-0.200pt]{2.409pt}{0.400pt}}
\put(170.0,594.0){\rule[-0.200pt]{2.409pt}{0.400pt}}
\put(1429.0,594.0){\rule[-0.200pt]{2.409pt}{0.400pt}}
\put(170.0,594.0){\rule[-0.200pt]{2.409pt}{0.400pt}}
\put(1429.0,594.0){\rule[-0.200pt]{2.409pt}{0.400pt}}
\put(170.0,594.0){\rule[-0.200pt]{2.409pt}{0.400pt}}
\put(1429.0,594.0){\rule[-0.200pt]{2.409pt}{0.400pt}}
\put(170.0,594.0){\rule[-0.200pt]{2.409pt}{0.400pt}}
\put(1429.0,594.0){\rule[-0.200pt]{2.409pt}{0.400pt}}
\put(170.0,594.0){\rule[-0.200pt]{2.409pt}{0.400pt}}
\put(1429.0,594.0){\rule[-0.200pt]{2.409pt}{0.400pt}}
\put(170.0,594.0){\rule[-0.200pt]{2.409pt}{0.400pt}}
\put(1429.0,594.0){\rule[-0.200pt]{2.409pt}{0.400pt}}
\put(170.0,594.0){\rule[-0.200pt]{2.409pt}{0.400pt}}
\put(1429.0,594.0){\rule[-0.200pt]{2.409pt}{0.400pt}}
\put(170.0,594.0){\rule[-0.200pt]{2.409pt}{0.400pt}}
\put(1429.0,594.0){\rule[-0.200pt]{2.409pt}{0.400pt}}
\put(170.0,594.0){\rule[-0.200pt]{2.409pt}{0.400pt}}
\put(1429.0,594.0){\rule[-0.200pt]{2.409pt}{0.400pt}}
\put(170.0,594.0){\rule[-0.200pt]{2.409pt}{0.400pt}}
\put(1429.0,594.0){\rule[-0.200pt]{2.409pt}{0.400pt}}
\put(170.0,594.0){\rule[-0.200pt]{2.409pt}{0.400pt}}
\put(1429.0,594.0){\rule[-0.200pt]{2.409pt}{0.400pt}}
\put(170.0,594.0){\rule[-0.200pt]{2.409pt}{0.400pt}}
\put(1429.0,594.0){\rule[-0.200pt]{2.409pt}{0.400pt}}
\put(170.0,594.0){\rule[-0.200pt]{2.409pt}{0.400pt}}
\put(1429.0,594.0){\rule[-0.200pt]{2.409pt}{0.400pt}}
\put(170.0,594.0){\rule[-0.200pt]{2.409pt}{0.400pt}}
\put(1429.0,594.0){\rule[-0.200pt]{2.409pt}{0.400pt}}
\put(170.0,594.0){\rule[-0.200pt]{2.409pt}{0.400pt}}
\put(1429.0,594.0){\rule[-0.200pt]{2.409pt}{0.400pt}}
\put(170.0,594.0){\rule[-0.200pt]{2.409pt}{0.400pt}}
\put(1429.0,594.0){\rule[-0.200pt]{2.409pt}{0.400pt}}
\put(170.0,594.0){\rule[-0.200pt]{2.409pt}{0.400pt}}
\put(1429.0,594.0){\rule[-0.200pt]{2.409pt}{0.400pt}}
\put(170.0,594.0){\rule[-0.200pt]{2.409pt}{0.400pt}}
\put(1429.0,594.0){\rule[-0.200pt]{2.409pt}{0.400pt}}
\put(170.0,594.0){\rule[-0.200pt]{2.409pt}{0.400pt}}
\put(1429.0,594.0){\rule[-0.200pt]{2.409pt}{0.400pt}}
\put(170.0,594.0){\rule[-0.200pt]{2.409pt}{0.400pt}}
\put(1429.0,594.0){\rule[-0.200pt]{2.409pt}{0.400pt}}
\put(170.0,594.0){\rule[-0.200pt]{2.409pt}{0.400pt}}
\put(1429.0,594.0){\rule[-0.200pt]{2.409pt}{0.400pt}}
\put(170.0,595.0){\rule[-0.200pt]{2.409pt}{0.400pt}}
\put(1429.0,595.0){\rule[-0.200pt]{2.409pt}{0.400pt}}
\put(170.0,595.0){\rule[-0.200pt]{2.409pt}{0.400pt}}
\put(1429.0,595.0){\rule[-0.200pt]{2.409pt}{0.400pt}}
\put(170.0,595.0){\rule[-0.200pt]{2.409pt}{0.400pt}}
\put(1429.0,595.0){\rule[-0.200pt]{2.409pt}{0.400pt}}
\put(170.0,595.0){\rule[-0.200pt]{2.409pt}{0.400pt}}
\put(1429.0,595.0){\rule[-0.200pt]{2.409pt}{0.400pt}}
\put(170.0,595.0){\rule[-0.200pt]{2.409pt}{0.400pt}}
\put(1429.0,595.0){\rule[-0.200pt]{2.409pt}{0.400pt}}
\put(170.0,595.0){\rule[-0.200pt]{2.409pt}{0.400pt}}
\put(1429.0,595.0){\rule[-0.200pt]{2.409pt}{0.400pt}}
\put(170.0,595.0){\rule[-0.200pt]{2.409pt}{0.400pt}}
\put(1429.0,595.0){\rule[-0.200pt]{2.409pt}{0.400pt}}
\put(170.0,595.0){\rule[-0.200pt]{2.409pt}{0.400pt}}
\put(1429.0,595.0){\rule[-0.200pt]{2.409pt}{0.400pt}}
\put(170.0,595.0){\rule[-0.200pt]{2.409pt}{0.400pt}}
\put(1429.0,595.0){\rule[-0.200pt]{2.409pt}{0.400pt}}
\put(170.0,595.0){\rule[-0.200pt]{2.409pt}{0.400pt}}
\put(1429.0,595.0){\rule[-0.200pt]{2.409pt}{0.400pt}}
\put(170.0,595.0){\rule[-0.200pt]{2.409pt}{0.400pt}}
\put(1429.0,595.0){\rule[-0.200pt]{2.409pt}{0.400pt}}
\put(170.0,595.0){\rule[-0.200pt]{2.409pt}{0.400pt}}
\put(1429.0,595.0){\rule[-0.200pt]{2.409pt}{0.400pt}}
\put(170.0,595.0){\rule[-0.200pt]{2.409pt}{0.400pt}}
\put(1429.0,595.0){\rule[-0.200pt]{2.409pt}{0.400pt}}
\put(170.0,595.0){\rule[-0.200pt]{2.409pt}{0.400pt}}
\put(1429.0,595.0){\rule[-0.200pt]{2.409pt}{0.400pt}}
\put(170.0,595.0){\rule[-0.200pt]{2.409pt}{0.400pt}}
\put(1429.0,595.0){\rule[-0.200pt]{2.409pt}{0.400pt}}
\put(170.0,595.0){\rule[-0.200pt]{2.409pt}{0.400pt}}
\put(1429.0,595.0){\rule[-0.200pt]{2.409pt}{0.400pt}}
\put(170.0,595.0){\rule[-0.200pt]{2.409pt}{0.400pt}}
\put(1429.0,595.0){\rule[-0.200pt]{2.409pt}{0.400pt}}
\put(170.0,595.0){\rule[-0.200pt]{2.409pt}{0.400pt}}
\put(1429.0,595.0){\rule[-0.200pt]{2.409pt}{0.400pt}}
\put(170.0,595.0){\rule[-0.200pt]{2.409pt}{0.400pt}}
\put(1429.0,595.0){\rule[-0.200pt]{2.409pt}{0.400pt}}
\put(170.0,595.0){\rule[-0.200pt]{2.409pt}{0.400pt}}
\put(1429.0,595.0){\rule[-0.200pt]{2.409pt}{0.400pt}}
\put(170.0,595.0){\rule[-0.200pt]{2.409pt}{0.400pt}}
\put(1429.0,595.0){\rule[-0.200pt]{2.409pt}{0.400pt}}
\put(170.0,595.0){\rule[-0.200pt]{2.409pt}{0.400pt}}
\put(1429.0,595.0){\rule[-0.200pt]{2.409pt}{0.400pt}}
\put(170.0,595.0){\rule[-0.200pt]{2.409pt}{0.400pt}}
\put(1429.0,595.0){\rule[-0.200pt]{2.409pt}{0.400pt}}
\put(170.0,596.0){\rule[-0.200pt]{2.409pt}{0.400pt}}
\put(1429.0,596.0){\rule[-0.200pt]{2.409pt}{0.400pt}}
\put(170.0,596.0){\rule[-0.200pt]{2.409pt}{0.400pt}}
\put(1429.0,596.0){\rule[-0.200pt]{2.409pt}{0.400pt}}
\put(170.0,596.0){\rule[-0.200pt]{2.409pt}{0.400pt}}
\put(1429.0,596.0){\rule[-0.200pt]{2.409pt}{0.400pt}}
\put(170.0,596.0){\rule[-0.200pt]{2.409pt}{0.400pt}}
\put(1429.0,596.0){\rule[-0.200pt]{2.409pt}{0.400pt}}
\put(170.0,596.0){\rule[-0.200pt]{2.409pt}{0.400pt}}
\put(1429.0,596.0){\rule[-0.200pt]{2.409pt}{0.400pt}}
\put(170.0,596.0){\rule[-0.200pt]{2.409pt}{0.400pt}}
\put(1429.0,596.0){\rule[-0.200pt]{2.409pt}{0.400pt}}
\put(170.0,596.0){\rule[-0.200pt]{2.409pt}{0.400pt}}
\put(1429.0,596.0){\rule[-0.200pt]{2.409pt}{0.400pt}}
\put(170.0,596.0){\rule[-0.200pt]{2.409pt}{0.400pt}}
\put(1429.0,596.0){\rule[-0.200pt]{2.409pt}{0.400pt}}
\put(170.0,596.0){\rule[-0.200pt]{2.409pt}{0.400pt}}
\put(1429.0,596.0){\rule[-0.200pt]{2.409pt}{0.400pt}}
\put(170.0,596.0){\rule[-0.200pt]{2.409pt}{0.400pt}}
\put(1429.0,596.0){\rule[-0.200pt]{2.409pt}{0.400pt}}
\put(170.0,596.0){\rule[-0.200pt]{2.409pt}{0.400pt}}
\put(1429.0,596.0){\rule[-0.200pt]{2.409pt}{0.400pt}}
\put(170.0,596.0){\rule[-0.200pt]{2.409pt}{0.400pt}}
\put(1429.0,596.0){\rule[-0.200pt]{2.409pt}{0.400pt}}
\put(170.0,596.0){\rule[-0.200pt]{2.409pt}{0.400pt}}
\put(1429.0,596.0){\rule[-0.200pt]{2.409pt}{0.400pt}}
\put(170.0,596.0){\rule[-0.200pt]{2.409pt}{0.400pt}}
\put(1429.0,596.0){\rule[-0.200pt]{2.409pt}{0.400pt}}
\put(170.0,596.0){\rule[-0.200pt]{2.409pt}{0.400pt}}
\put(1429.0,596.0){\rule[-0.200pt]{2.409pt}{0.400pt}}
\put(170.0,596.0){\rule[-0.200pt]{2.409pt}{0.400pt}}
\put(1429.0,596.0){\rule[-0.200pt]{2.409pt}{0.400pt}}
\put(170.0,596.0){\rule[-0.200pt]{2.409pt}{0.400pt}}
\put(1429.0,596.0){\rule[-0.200pt]{2.409pt}{0.400pt}}
\put(170.0,596.0){\rule[-0.200pt]{2.409pt}{0.400pt}}
\put(1429.0,596.0){\rule[-0.200pt]{2.409pt}{0.400pt}}
\put(170.0,596.0){\rule[-0.200pt]{2.409pt}{0.400pt}}
\put(1429.0,596.0){\rule[-0.200pt]{2.409pt}{0.400pt}}
\put(170.0,596.0){\rule[-0.200pt]{2.409pt}{0.400pt}}
\put(1429.0,596.0){\rule[-0.200pt]{2.409pt}{0.400pt}}
\put(170.0,596.0){\rule[-0.200pt]{2.409pt}{0.400pt}}
\put(1429.0,596.0){\rule[-0.200pt]{2.409pt}{0.400pt}}
\put(170.0,596.0){\rule[-0.200pt]{2.409pt}{0.400pt}}
\put(1429.0,596.0){\rule[-0.200pt]{2.409pt}{0.400pt}}
\put(170.0,596.0){\rule[-0.200pt]{2.409pt}{0.400pt}}
\put(1429.0,596.0){\rule[-0.200pt]{2.409pt}{0.400pt}}
\put(170.0,596.0){\rule[-0.200pt]{2.409pt}{0.400pt}}
\put(1429.0,596.0){\rule[-0.200pt]{2.409pt}{0.400pt}}
\put(170.0,597.0){\rule[-0.200pt]{2.409pt}{0.400pt}}
\put(1429.0,597.0){\rule[-0.200pt]{2.409pt}{0.400pt}}
\put(170.0,597.0){\rule[-0.200pt]{2.409pt}{0.400pt}}
\put(1429.0,597.0){\rule[-0.200pt]{2.409pt}{0.400pt}}
\put(170.0,597.0){\rule[-0.200pt]{2.409pt}{0.400pt}}
\put(1429.0,597.0){\rule[-0.200pt]{2.409pt}{0.400pt}}
\put(170.0,597.0){\rule[-0.200pt]{2.409pt}{0.400pt}}
\put(1429.0,597.0){\rule[-0.200pt]{2.409pt}{0.400pt}}
\put(170.0,597.0){\rule[-0.200pt]{2.409pt}{0.400pt}}
\put(1429.0,597.0){\rule[-0.200pt]{2.409pt}{0.400pt}}
\put(170.0,597.0){\rule[-0.200pt]{2.409pt}{0.400pt}}
\put(1429.0,597.0){\rule[-0.200pt]{2.409pt}{0.400pt}}
\put(170.0,597.0){\rule[-0.200pt]{2.409pt}{0.400pt}}
\put(1429.0,597.0){\rule[-0.200pt]{2.409pt}{0.400pt}}
\put(170.0,597.0){\rule[-0.200pt]{2.409pt}{0.400pt}}
\put(1429.0,597.0){\rule[-0.200pt]{2.409pt}{0.400pt}}
\put(170.0,597.0){\rule[-0.200pt]{2.409pt}{0.400pt}}
\put(1429.0,597.0){\rule[-0.200pt]{2.409pt}{0.400pt}}
\put(170.0,597.0){\rule[-0.200pt]{2.409pt}{0.400pt}}
\put(1429.0,597.0){\rule[-0.200pt]{2.409pt}{0.400pt}}
\put(170.0,597.0){\rule[-0.200pt]{2.409pt}{0.400pt}}
\put(1429.0,597.0){\rule[-0.200pt]{2.409pt}{0.400pt}}
\put(170.0,597.0){\rule[-0.200pt]{2.409pt}{0.400pt}}
\put(1429.0,597.0){\rule[-0.200pt]{2.409pt}{0.400pt}}
\put(170.0,597.0){\rule[-0.200pt]{2.409pt}{0.400pt}}
\put(1429.0,597.0){\rule[-0.200pt]{2.409pt}{0.400pt}}
\put(170.0,597.0){\rule[-0.200pt]{2.409pt}{0.400pt}}
\put(1429.0,597.0){\rule[-0.200pt]{2.409pt}{0.400pt}}
\put(170.0,597.0){\rule[-0.200pt]{2.409pt}{0.400pt}}
\put(1429.0,597.0){\rule[-0.200pt]{2.409pt}{0.400pt}}
\put(170.0,597.0){\rule[-0.200pt]{2.409pt}{0.400pt}}
\put(1429.0,597.0){\rule[-0.200pt]{2.409pt}{0.400pt}}
\put(170.0,597.0){\rule[-0.200pt]{2.409pt}{0.400pt}}
\put(1429.0,597.0){\rule[-0.200pt]{2.409pt}{0.400pt}}
\put(170.0,597.0){\rule[-0.200pt]{2.409pt}{0.400pt}}
\put(1429.0,597.0){\rule[-0.200pt]{2.409pt}{0.400pt}}
\put(170.0,597.0){\rule[-0.200pt]{2.409pt}{0.400pt}}
\put(1429.0,597.0){\rule[-0.200pt]{2.409pt}{0.400pt}}
\put(170.0,597.0){\rule[-0.200pt]{2.409pt}{0.400pt}}
\put(1429.0,597.0){\rule[-0.200pt]{2.409pt}{0.400pt}}
\put(170.0,597.0){\rule[-0.200pt]{2.409pt}{0.400pt}}
\put(1429.0,597.0){\rule[-0.200pt]{2.409pt}{0.400pt}}
\put(170.0,597.0){\rule[-0.200pt]{2.409pt}{0.400pt}}
\put(1429.0,597.0){\rule[-0.200pt]{2.409pt}{0.400pt}}
\put(170.0,597.0){\rule[-0.200pt]{2.409pt}{0.400pt}}
\put(1429.0,597.0){\rule[-0.200pt]{2.409pt}{0.400pt}}
\put(170.0,597.0){\rule[-0.200pt]{2.409pt}{0.400pt}}
\put(1429.0,597.0){\rule[-0.200pt]{2.409pt}{0.400pt}}
\put(170.0,597.0){\rule[-0.200pt]{2.409pt}{0.400pt}}
\put(1429.0,597.0){\rule[-0.200pt]{2.409pt}{0.400pt}}
\put(170.0,598.0){\rule[-0.200pt]{2.409pt}{0.400pt}}
\put(1429.0,598.0){\rule[-0.200pt]{2.409pt}{0.400pt}}
\put(170.0,598.0){\rule[-0.200pt]{2.409pt}{0.400pt}}
\put(1429.0,598.0){\rule[-0.200pt]{2.409pt}{0.400pt}}
\put(170.0,598.0){\rule[-0.200pt]{2.409pt}{0.400pt}}
\put(1429.0,598.0){\rule[-0.200pt]{2.409pt}{0.400pt}}
\put(170.0,598.0){\rule[-0.200pt]{2.409pt}{0.400pt}}
\put(1429.0,598.0){\rule[-0.200pt]{2.409pt}{0.400pt}}
\put(170.0,598.0){\rule[-0.200pt]{2.409pt}{0.400pt}}
\put(1429.0,598.0){\rule[-0.200pt]{2.409pt}{0.400pt}}
\put(170.0,598.0){\rule[-0.200pt]{2.409pt}{0.400pt}}
\put(1429.0,598.0){\rule[-0.200pt]{2.409pt}{0.400pt}}
\put(170.0,598.0){\rule[-0.200pt]{2.409pt}{0.400pt}}
\put(1429.0,598.0){\rule[-0.200pt]{2.409pt}{0.400pt}}
\put(170.0,598.0){\rule[-0.200pt]{2.409pt}{0.400pt}}
\put(1429.0,598.0){\rule[-0.200pt]{2.409pt}{0.400pt}}
\put(170.0,598.0){\rule[-0.200pt]{2.409pt}{0.400pt}}
\put(1429.0,598.0){\rule[-0.200pt]{2.409pt}{0.400pt}}
\put(170.0,598.0){\rule[-0.200pt]{2.409pt}{0.400pt}}
\put(1429.0,598.0){\rule[-0.200pt]{2.409pt}{0.400pt}}
\put(170.0,598.0){\rule[-0.200pt]{2.409pt}{0.400pt}}
\put(1429.0,598.0){\rule[-0.200pt]{2.409pt}{0.400pt}}
\put(170.0,598.0){\rule[-0.200pt]{2.409pt}{0.400pt}}
\put(1429.0,598.0){\rule[-0.200pt]{2.409pt}{0.400pt}}
\put(170.0,598.0){\rule[-0.200pt]{2.409pt}{0.400pt}}
\put(1429.0,598.0){\rule[-0.200pt]{2.409pt}{0.400pt}}
\put(170.0,598.0){\rule[-0.200pt]{2.409pt}{0.400pt}}
\put(1429.0,598.0){\rule[-0.200pt]{2.409pt}{0.400pt}}
\put(170.0,598.0){\rule[-0.200pt]{2.409pt}{0.400pt}}
\put(1429.0,598.0){\rule[-0.200pt]{2.409pt}{0.400pt}}
\put(170.0,598.0){\rule[-0.200pt]{2.409pt}{0.400pt}}
\put(1429.0,598.0){\rule[-0.200pt]{2.409pt}{0.400pt}}
\put(170.0,598.0){\rule[-0.200pt]{2.409pt}{0.400pt}}
\put(1429.0,598.0){\rule[-0.200pt]{2.409pt}{0.400pt}}
\put(170.0,598.0){\rule[-0.200pt]{2.409pt}{0.400pt}}
\put(1429.0,598.0){\rule[-0.200pt]{2.409pt}{0.400pt}}
\put(170.0,598.0){\rule[-0.200pt]{2.409pt}{0.400pt}}
\put(1429.0,598.0){\rule[-0.200pt]{2.409pt}{0.400pt}}
\put(170.0,598.0){\rule[-0.200pt]{2.409pt}{0.400pt}}
\put(1429.0,598.0){\rule[-0.200pt]{2.409pt}{0.400pt}}
\put(170.0,598.0){\rule[-0.200pt]{2.409pt}{0.400pt}}
\put(1429.0,598.0){\rule[-0.200pt]{2.409pt}{0.400pt}}
\put(170.0,598.0){\rule[-0.200pt]{2.409pt}{0.400pt}}
\put(1429.0,598.0){\rule[-0.200pt]{2.409pt}{0.400pt}}
\put(170.0,598.0){\rule[-0.200pt]{2.409pt}{0.400pt}}
\put(1429.0,598.0){\rule[-0.200pt]{2.409pt}{0.400pt}}
\put(170.0,598.0){\rule[-0.200pt]{2.409pt}{0.400pt}}
\put(1429.0,598.0){\rule[-0.200pt]{2.409pt}{0.400pt}}
\put(170.0,598.0){\rule[-0.200pt]{2.409pt}{0.400pt}}
\put(1429.0,598.0){\rule[-0.200pt]{2.409pt}{0.400pt}}
\put(170.0,599.0){\rule[-0.200pt]{2.409pt}{0.400pt}}
\put(1429.0,599.0){\rule[-0.200pt]{2.409pt}{0.400pt}}
\put(170.0,599.0){\rule[-0.200pt]{2.409pt}{0.400pt}}
\put(1429.0,599.0){\rule[-0.200pt]{2.409pt}{0.400pt}}
\put(170.0,599.0){\rule[-0.200pt]{2.409pt}{0.400pt}}
\put(1429.0,599.0){\rule[-0.200pt]{2.409pt}{0.400pt}}
\put(170.0,599.0){\rule[-0.200pt]{2.409pt}{0.400pt}}
\put(1429.0,599.0){\rule[-0.200pt]{2.409pt}{0.400pt}}
\put(170.0,599.0){\rule[-0.200pt]{2.409pt}{0.400pt}}
\put(1429.0,599.0){\rule[-0.200pt]{2.409pt}{0.400pt}}
\put(170.0,599.0){\rule[-0.200pt]{2.409pt}{0.400pt}}
\put(1429.0,599.0){\rule[-0.200pt]{2.409pt}{0.400pt}}
\put(170.0,599.0){\rule[-0.200pt]{2.409pt}{0.400pt}}
\put(1429.0,599.0){\rule[-0.200pt]{2.409pt}{0.400pt}}
\put(170.0,599.0){\rule[-0.200pt]{2.409pt}{0.400pt}}
\put(1429.0,599.0){\rule[-0.200pt]{2.409pt}{0.400pt}}
\put(170.0,599.0){\rule[-0.200pt]{2.409pt}{0.400pt}}
\put(1429.0,599.0){\rule[-0.200pt]{2.409pt}{0.400pt}}
\put(170.0,599.0){\rule[-0.200pt]{2.409pt}{0.400pt}}
\put(1429.0,599.0){\rule[-0.200pt]{2.409pt}{0.400pt}}
\put(170.0,599.0){\rule[-0.200pt]{2.409pt}{0.400pt}}
\put(1429.0,599.0){\rule[-0.200pt]{2.409pt}{0.400pt}}
\put(170.0,599.0){\rule[-0.200pt]{2.409pt}{0.400pt}}
\put(1429.0,599.0){\rule[-0.200pt]{2.409pt}{0.400pt}}
\put(170.0,599.0){\rule[-0.200pt]{2.409pt}{0.400pt}}
\put(1429.0,599.0){\rule[-0.200pt]{2.409pt}{0.400pt}}
\put(170.0,599.0){\rule[-0.200pt]{2.409pt}{0.400pt}}
\put(1429.0,599.0){\rule[-0.200pt]{2.409pt}{0.400pt}}
\put(170.0,599.0){\rule[-0.200pt]{2.409pt}{0.400pt}}
\put(1429.0,599.0){\rule[-0.200pt]{2.409pt}{0.400pt}}
\put(170.0,599.0){\rule[-0.200pt]{2.409pt}{0.400pt}}
\put(1429.0,599.0){\rule[-0.200pt]{2.409pt}{0.400pt}}
\put(170.0,599.0){\rule[-0.200pt]{2.409pt}{0.400pt}}
\put(1429.0,599.0){\rule[-0.200pt]{2.409pt}{0.400pt}}
\put(170.0,599.0){\rule[-0.200pt]{2.409pt}{0.400pt}}
\put(1429.0,599.0){\rule[-0.200pt]{2.409pt}{0.400pt}}
\put(170.0,599.0){\rule[-0.200pt]{2.409pt}{0.400pt}}
\put(1429.0,599.0){\rule[-0.200pt]{2.409pt}{0.400pt}}
\put(170.0,599.0){\rule[-0.200pt]{2.409pt}{0.400pt}}
\put(1429.0,599.0){\rule[-0.200pt]{2.409pt}{0.400pt}}
\put(170.0,599.0){\rule[-0.200pt]{2.409pt}{0.400pt}}
\put(1429.0,599.0){\rule[-0.200pt]{2.409pt}{0.400pt}}
\put(170.0,599.0){\rule[-0.200pt]{2.409pt}{0.400pt}}
\put(1429.0,599.0){\rule[-0.200pt]{2.409pt}{0.400pt}}
\put(170.0,599.0){\rule[-0.200pt]{2.409pt}{0.400pt}}
\put(1429.0,599.0){\rule[-0.200pt]{2.409pt}{0.400pt}}
\put(170.0,599.0){\rule[-0.200pt]{2.409pt}{0.400pt}}
\put(1429.0,599.0){\rule[-0.200pt]{2.409pt}{0.400pt}}
\put(170.0,599.0){\rule[-0.200pt]{2.409pt}{0.400pt}}
\put(1429.0,599.0){\rule[-0.200pt]{2.409pt}{0.400pt}}
\put(170.0,599.0){\rule[-0.200pt]{2.409pt}{0.400pt}}
\put(1429.0,599.0){\rule[-0.200pt]{2.409pt}{0.400pt}}
\put(170.0,600.0){\rule[-0.200pt]{2.409pt}{0.400pt}}
\put(1429.0,600.0){\rule[-0.200pt]{2.409pt}{0.400pt}}
\put(170.0,600.0){\rule[-0.200pt]{2.409pt}{0.400pt}}
\put(1429.0,600.0){\rule[-0.200pt]{2.409pt}{0.400pt}}
\put(170.0,600.0){\rule[-0.200pt]{2.409pt}{0.400pt}}
\put(1429.0,600.0){\rule[-0.200pt]{2.409pt}{0.400pt}}
\put(170.0,600.0){\rule[-0.200pt]{2.409pt}{0.400pt}}
\put(1429.0,600.0){\rule[-0.200pt]{2.409pt}{0.400pt}}
\put(170.0,600.0){\rule[-0.200pt]{2.409pt}{0.400pt}}
\put(1429.0,600.0){\rule[-0.200pt]{2.409pt}{0.400pt}}
\put(170.0,600.0){\rule[-0.200pt]{2.409pt}{0.400pt}}
\put(1429.0,600.0){\rule[-0.200pt]{2.409pt}{0.400pt}}
\put(170.0,600.0){\rule[-0.200pt]{2.409pt}{0.400pt}}
\put(1429.0,600.0){\rule[-0.200pt]{2.409pt}{0.400pt}}
\put(170.0,600.0){\rule[-0.200pt]{2.409pt}{0.400pt}}
\put(1429.0,600.0){\rule[-0.200pt]{2.409pt}{0.400pt}}
\put(170.0,600.0){\rule[-0.200pt]{2.409pt}{0.400pt}}
\put(1429.0,600.0){\rule[-0.200pt]{2.409pt}{0.400pt}}
\put(170.0,600.0){\rule[-0.200pt]{2.409pt}{0.400pt}}
\put(1429.0,600.0){\rule[-0.200pt]{2.409pt}{0.400pt}}
\put(170.0,600.0){\rule[-0.200pt]{2.409pt}{0.400pt}}
\put(1429.0,600.0){\rule[-0.200pt]{2.409pt}{0.400pt}}
\put(170.0,600.0){\rule[-0.200pt]{2.409pt}{0.400pt}}
\put(1429.0,600.0){\rule[-0.200pt]{2.409pt}{0.400pt}}
\put(170.0,600.0){\rule[-0.200pt]{2.409pt}{0.400pt}}
\put(1429.0,600.0){\rule[-0.200pt]{2.409pt}{0.400pt}}
\put(170.0,600.0){\rule[-0.200pt]{4.818pt}{0.400pt}}
\put(150,600){\makebox(0,0)[r]{ 1e+06}}
\put(1419.0,600.0){\rule[-0.200pt]{4.818pt}{0.400pt}}
\put(170.0,626.0){\rule[-0.200pt]{2.409pt}{0.400pt}}
\put(1429.0,626.0){\rule[-0.200pt]{2.409pt}{0.400pt}}
\put(170.0,641.0){\rule[-0.200pt]{2.409pt}{0.400pt}}
\put(1429.0,641.0){\rule[-0.200pt]{2.409pt}{0.400pt}}
\put(170.0,652.0){\rule[-0.200pt]{2.409pt}{0.400pt}}
\put(1429.0,652.0){\rule[-0.200pt]{2.409pt}{0.400pt}}
\put(170.0,660.0){\rule[-0.200pt]{2.409pt}{0.400pt}}
\put(1429.0,660.0){\rule[-0.200pt]{2.409pt}{0.400pt}}
\put(170.0,667.0){\rule[-0.200pt]{2.409pt}{0.400pt}}
\put(1429.0,667.0){\rule[-0.200pt]{2.409pt}{0.400pt}}
\put(170.0,673.0){\rule[-0.200pt]{2.409pt}{0.400pt}}
\put(1429.0,673.0){\rule[-0.200pt]{2.409pt}{0.400pt}}
\put(170.0,678.0){\rule[-0.200pt]{2.409pt}{0.400pt}}
\put(1429.0,678.0){\rule[-0.200pt]{2.409pt}{0.400pt}}
\put(170.0,682.0){\rule[-0.200pt]{2.409pt}{0.400pt}}
\put(1429.0,682.0){\rule[-0.200pt]{2.409pt}{0.400pt}}
\put(170.0,686.0){\rule[-0.200pt]{2.409pt}{0.400pt}}
\put(1429.0,686.0){\rule[-0.200pt]{2.409pt}{0.400pt}}
\put(170.0,690.0){\rule[-0.200pt]{2.409pt}{0.400pt}}
\put(1429.0,690.0){\rule[-0.200pt]{2.409pt}{0.400pt}}
\put(170.0,693.0){\rule[-0.200pt]{2.409pt}{0.400pt}}
\put(1429.0,693.0){\rule[-0.200pt]{2.409pt}{0.400pt}}
\put(170.0,696.0){\rule[-0.200pt]{2.409pt}{0.400pt}}
\put(1429.0,696.0){\rule[-0.200pt]{2.409pt}{0.400pt}}
\put(170.0,699.0){\rule[-0.200pt]{2.409pt}{0.400pt}}
\put(1429.0,699.0){\rule[-0.200pt]{2.409pt}{0.400pt}}
\put(170.0,702.0){\rule[-0.200pt]{2.409pt}{0.400pt}}
\put(1429.0,702.0){\rule[-0.200pt]{2.409pt}{0.400pt}}
\put(170.0,704.0){\rule[-0.200pt]{2.409pt}{0.400pt}}
\put(1429.0,704.0){\rule[-0.200pt]{2.409pt}{0.400pt}}
\put(170.0,706.0){\rule[-0.200pt]{2.409pt}{0.400pt}}
\put(1429.0,706.0){\rule[-0.200pt]{2.409pt}{0.400pt}}
\put(170.0,708.0){\rule[-0.200pt]{2.409pt}{0.400pt}}
\put(1429.0,708.0){\rule[-0.200pt]{2.409pt}{0.400pt}}
\put(170.0,710.0){\rule[-0.200pt]{2.409pt}{0.400pt}}
\put(1429.0,710.0){\rule[-0.200pt]{2.409pt}{0.400pt}}
\put(170.0,712.0){\rule[-0.200pt]{2.409pt}{0.400pt}}
\put(1429.0,712.0){\rule[-0.200pt]{2.409pt}{0.400pt}}
\put(170.0,714.0){\rule[-0.200pt]{2.409pt}{0.400pt}}
\put(1429.0,714.0){\rule[-0.200pt]{2.409pt}{0.400pt}}
\put(170.0,716.0){\rule[-0.200pt]{2.409pt}{0.400pt}}
\put(1429.0,716.0){\rule[-0.200pt]{2.409pt}{0.400pt}}
\put(170.0,718.0){\rule[-0.200pt]{2.409pt}{0.400pt}}
\put(1429.0,718.0){\rule[-0.200pt]{2.409pt}{0.400pt}}
\put(170.0,719.0){\rule[-0.200pt]{2.409pt}{0.400pt}}
\put(1429.0,719.0){\rule[-0.200pt]{2.409pt}{0.400pt}}
\put(170.0,721.0){\rule[-0.200pt]{2.409pt}{0.400pt}}
\put(1429.0,721.0){\rule[-0.200pt]{2.409pt}{0.400pt}}
\put(170.0,722.0){\rule[-0.200pt]{2.409pt}{0.400pt}}
\put(1429.0,722.0){\rule[-0.200pt]{2.409pt}{0.400pt}}
\put(170.0,724.0){\rule[-0.200pt]{2.409pt}{0.400pt}}
\put(1429.0,724.0){\rule[-0.200pt]{2.409pt}{0.400pt}}
\put(170.0,725.0){\rule[-0.200pt]{2.409pt}{0.400pt}}
\put(1429.0,725.0){\rule[-0.200pt]{2.409pt}{0.400pt}}
\put(170.0,726.0){\rule[-0.200pt]{2.409pt}{0.400pt}}
\put(1429.0,726.0){\rule[-0.200pt]{2.409pt}{0.400pt}}
\put(170.0,728.0){\rule[-0.200pt]{2.409pt}{0.400pt}}
\put(1429.0,728.0){\rule[-0.200pt]{2.409pt}{0.400pt}}
\put(170.0,729.0){\rule[-0.200pt]{2.409pt}{0.400pt}}
\put(1429.0,729.0){\rule[-0.200pt]{2.409pt}{0.400pt}}
\put(170.0,730.0){\rule[-0.200pt]{2.409pt}{0.400pt}}
\put(1429.0,730.0){\rule[-0.200pt]{2.409pt}{0.400pt}}
\put(170.0,731.0){\rule[-0.200pt]{2.409pt}{0.400pt}}
\put(1429.0,731.0){\rule[-0.200pt]{2.409pt}{0.400pt}}
\put(170.0,732.0){\rule[-0.200pt]{2.409pt}{0.400pt}}
\put(1429.0,732.0){\rule[-0.200pt]{2.409pt}{0.400pt}}
\put(170.0,733.0){\rule[-0.200pt]{2.409pt}{0.400pt}}
\put(1429.0,733.0){\rule[-0.200pt]{2.409pt}{0.400pt}}
\put(170.0,734.0){\rule[-0.200pt]{2.409pt}{0.400pt}}
\put(1429.0,734.0){\rule[-0.200pt]{2.409pt}{0.400pt}}
\put(170.0,735.0){\rule[-0.200pt]{2.409pt}{0.400pt}}
\put(1429.0,735.0){\rule[-0.200pt]{2.409pt}{0.400pt}}
\put(170.0,736.0){\rule[-0.200pt]{2.409pt}{0.400pt}}
\put(1429.0,736.0){\rule[-0.200pt]{2.409pt}{0.400pt}}
\put(170.0,737.0){\rule[-0.200pt]{2.409pt}{0.400pt}}
\put(1429.0,737.0){\rule[-0.200pt]{2.409pt}{0.400pt}}
\put(170.0,738.0){\rule[-0.200pt]{2.409pt}{0.400pt}}
\put(1429.0,738.0){\rule[-0.200pt]{2.409pt}{0.400pt}}
\put(170.0,739.0){\rule[-0.200pt]{2.409pt}{0.400pt}}
\put(1429.0,739.0){\rule[-0.200pt]{2.409pt}{0.400pt}}
\put(170.0,740.0){\rule[-0.200pt]{2.409pt}{0.400pt}}
\put(1429.0,740.0){\rule[-0.200pt]{2.409pt}{0.400pt}}
\put(170.0,741.0){\rule[-0.200pt]{2.409pt}{0.400pt}}
\put(1429.0,741.0){\rule[-0.200pt]{2.409pt}{0.400pt}}
\put(170.0,742.0){\rule[-0.200pt]{2.409pt}{0.400pt}}
\put(1429.0,742.0){\rule[-0.200pt]{2.409pt}{0.400pt}}
\put(170.0,743.0){\rule[-0.200pt]{2.409pt}{0.400pt}}
\put(1429.0,743.0){\rule[-0.200pt]{2.409pt}{0.400pt}}
\put(170.0,744.0){\rule[-0.200pt]{2.409pt}{0.400pt}}
\put(1429.0,744.0){\rule[-0.200pt]{2.409pt}{0.400pt}}
\put(170.0,744.0){\rule[-0.200pt]{2.409pt}{0.400pt}}
\put(1429.0,744.0){\rule[-0.200pt]{2.409pt}{0.400pt}}
\put(170.0,745.0){\rule[-0.200pt]{2.409pt}{0.400pt}}
\put(1429.0,745.0){\rule[-0.200pt]{2.409pt}{0.400pt}}
\put(170.0,746.0){\rule[-0.200pt]{2.409pt}{0.400pt}}
\put(1429.0,746.0){\rule[-0.200pt]{2.409pt}{0.400pt}}
\put(170.0,747.0){\rule[-0.200pt]{2.409pt}{0.400pt}}
\put(1429.0,747.0){\rule[-0.200pt]{2.409pt}{0.400pt}}
\put(170.0,747.0){\rule[-0.200pt]{2.409pt}{0.400pt}}
\put(1429.0,747.0){\rule[-0.200pt]{2.409pt}{0.400pt}}
\put(170.0,748.0){\rule[-0.200pt]{2.409pt}{0.400pt}}
\put(1429.0,748.0){\rule[-0.200pt]{2.409pt}{0.400pt}}
\put(170.0,749.0){\rule[-0.200pt]{2.409pt}{0.400pt}}
\put(1429.0,749.0){\rule[-0.200pt]{2.409pt}{0.400pt}}
\put(170.0,750.0){\rule[-0.200pt]{2.409pt}{0.400pt}}
\put(1429.0,750.0){\rule[-0.200pt]{2.409pt}{0.400pt}}
\put(170.0,750.0){\rule[-0.200pt]{2.409pt}{0.400pt}}
\put(1429.0,750.0){\rule[-0.200pt]{2.409pt}{0.400pt}}
\put(170.0,751.0){\rule[-0.200pt]{2.409pt}{0.400pt}}
\put(1429.0,751.0){\rule[-0.200pt]{2.409pt}{0.400pt}}
\put(170.0,752.0){\rule[-0.200pt]{2.409pt}{0.400pt}}
\put(1429.0,752.0){\rule[-0.200pt]{2.409pt}{0.400pt}}
\put(170.0,752.0){\rule[-0.200pt]{2.409pt}{0.400pt}}
\put(1429.0,752.0){\rule[-0.200pt]{2.409pt}{0.400pt}}
\put(170.0,753.0){\rule[-0.200pt]{2.409pt}{0.400pt}}
\put(1429.0,753.0){\rule[-0.200pt]{2.409pt}{0.400pt}}
\put(170.0,754.0){\rule[-0.200pt]{2.409pt}{0.400pt}}
\put(1429.0,754.0){\rule[-0.200pt]{2.409pt}{0.400pt}}
\put(170.0,754.0){\rule[-0.200pt]{2.409pt}{0.400pt}}
\put(1429.0,754.0){\rule[-0.200pt]{2.409pt}{0.400pt}}
\put(170.0,755.0){\rule[-0.200pt]{2.409pt}{0.400pt}}
\put(1429.0,755.0){\rule[-0.200pt]{2.409pt}{0.400pt}}
\put(170.0,755.0){\rule[-0.200pt]{2.409pt}{0.400pt}}
\put(1429.0,755.0){\rule[-0.200pt]{2.409pt}{0.400pt}}
\put(170.0,756.0){\rule[-0.200pt]{2.409pt}{0.400pt}}
\put(1429.0,756.0){\rule[-0.200pt]{2.409pt}{0.400pt}}
\put(170.0,757.0){\rule[-0.200pt]{2.409pt}{0.400pt}}
\put(1429.0,757.0){\rule[-0.200pt]{2.409pt}{0.400pt}}
\put(170.0,757.0){\rule[-0.200pt]{2.409pt}{0.400pt}}
\put(1429.0,757.0){\rule[-0.200pt]{2.409pt}{0.400pt}}
\put(170.0,758.0){\rule[-0.200pt]{2.409pt}{0.400pt}}
\put(1429.0,758.0){\rule[-0.200pt]{2.409pt}{0.400pt}}
\put(170.0,758.0){\rule[-0.200pt]{2.409pt}{0.400pt}}
\put(1429.0,758.0){\rule[-0.200pt]{2.409pt}{0.400pt}}
\put(170.0,759.0){\rule[-0.200pt]{2.409pt}{0.400pt}}
\put(1429.0,759.0){\rule[-0.200pt]{2.409pt}{0.400pt}}
\put(170.0,759.0){\rule[-0.200pt]{2.409pt}{0.400pt}}
\put(1429.0,759.0){\rule[-0.200pt]{2.409pt}{0.400pt}}
\put(170.0,760.0){\rule[-0.200pt]{2.409pt}{0.400pt}}
\put(1429.0,760.0){\rule[-0.200pt]{2.409pt}{0.400pt}}
\put(170.0,760.0){\rule[-0.200pt]{2.409pt}{0.400pt}}
\put(1429.0,760.0){\rule[-0.200pt]{2.409pt}{0.400pt}}
\put(170.0,761.0){\rule[-0.200pt]{2.409pt}{0.400pt}}
\put(1429.0,761.0){\rule[-0.200pt]{2.409pt}{0.400pt}}
\put(170.0,761.0){\rule[-0.200pt]{2.409pt}{0.400pt}}
\put(1429.0,761.0){\rule[-0.200pt]{2.409pt}{0.400pt}}
\put(170.0,762.0){\rule[-0.200pt]{2.409pt}{0.400pt}}
\put(1429.0,762.0){\rule[-0.200pt]{2.409pt}{0.400pt}}
\put(170.0,762.0){\rule[-0.200pt]{2.409pt}{0.400pt}}
\put(1429.0,762.0){\rule[-0.200pt]{2.409pt}{0.400pt}}
\put(170.0,763.0){\rule[-0.200pt]{2.409pt}{0.400pt}}
\put(1429.0,763.0){\rule[-0.200pt]{2.409pt}{0.400pt}}
\put(170.0,763.0){\rule[-0.200pt]{2.409pt}{0.400pt}}
\put(1429.0,763.0){\rule[-0.200pt]{2.409pt}{0.400pt}}
\put(170.0,764.0){\rule[-0.200pt]{2.409pt}{0.400pt}}
\put(1429.0,764.0){\rule[-0.200pt]{2.409pt}{0.400pt}}
\put(170.0,764.0){\rule[-0.200pt]{2.409pt}{0.400pt}}
\put(1429.0,764.0){\rule[-0.200pt]{2.409pt}{0.400pt}}
\put(170.0,765.0){\rule[-0.200pt]{2.409pt}{0.400pt}}
\put(1429.0,765.0){\rule[-0.200pt]{2.409pt}{0.400pt}}
\put(170.0,765.0){\rule[-0.200pt]{2.409pt}{0.400pt}}
\put(1429.0,765.0){\rule[-0.200pt]{2.409pt}{0.400pt}}
\put(170.0,766.0){\rule[-0.200pt]{2.409pt}{0.400pt}}
\put(1429.0,766.0){\rule[-0.200pt]{2.409pt}{0.400pt}}
\put(170.0,766.0){\rule[-0.200pt]{2.409pt}{0.400pt}}
\put(1429.0,766.0){\rule[-0.200pt]{2.409pt}{0.400pt}}
\put(170.0,767.0){\rule[-0.200pt]{2.409pt}{0.400pt}}
\put(1429.0,767.0){\rule[-0.200pt]{2.409pt}{0.400pt}}
\put(170.0,767.0){\rule[-0.200pt]{2.409pt}{0.400pt}}
\put(1429.0,767.0){\rule[-0.200pt]{2.409pt}{0.400pt}}
\put(170.0,767.0){\rule[-0.200pt]{2.409pt}{0.400pt}}
\put(1429.0,767.0){\rule[-0.200pt]{2.409pt}{0.400pt}}
\put(170.0,768.0){\rule[-0.200pt]{2.409pt}{0.400pt}}
\put(1429.0,768.0){\rule[-0.200pt]{2.409pt}{0.400pt}}
\put(170.0,768.0){\rule[-0.200pt]{2.409pt}{0.400pt}}
\put(1429.0,768.0){\rule[-0.200pt]{2.409pt}{0.400pt}}
\put(170.0,769.0){\rule[-0.200pt]{2.409pt}{0.400pt}}
\put(1429.0,769.0){\rule[-0.200pt]{2.409pt}{0.400pt}}
\put(170.0,769.0){\rule[-0.200pt]{2.409pt}{0.400pt}}
\put(1429.0,769.0){\rule[-0.200pt]{2.409pt}{0.400pt}}
\put(170.0,770.0){\rule[-0.200pt]{2.409pt}{0.400pt}}
\put(1429.0,770.0){\rule[-0.200pt]{2.409pt}{0.400pt}}
\put(170.0,770.0){\rule[-0.200pt]{2.409pt}{0.400pt}}
\put(1429.0,770.0){\rule[-0.200pt]{2.409pt}{0.400pt}}
\put(170.0,770.0){\rule[-0.200pt]{2.409pt}{0.400pt}}
\put(1429.0,770.0){\rule[-0.200pt]{2.409pt}{0.400pt}}
\put(170.0,771.0){\rule[-0.200pt]{2.409pt}{0.400pt}}
\put(1429.0,771.0){\rule[-0.200pt]{2.409pt}{0.400pt}}
\put(170.0,771.0){\rule[-0.200pt]{2.409pt}{0.400pt}}
\put(1429.0,771.0){\rule[-0.200pt]{2.409pt}{0.400pt}}
\put(170.0,772.0){\rule[-0.200pt]{2.409pt}{0.400pt}}
\put(1429.0,772.0){\rule[-0.200pt]{2.409pt}{0.400pt}}
\put(170.0,772.0){\rule[-0.200pt]{2.409pt}{0.400pt}}
\put(1429.0,772.0){\rule[-0.200pt]{2.409pt}{0.400pt}}
\put(170.0,772.0){\rule[-0.200pt]{2.409pt}{0.400pt}}
\put(1429.0,772.0){\rule[-0.200pt]{2.409pt}{0.400pt}}
\put(170.0,773.0){\rule[-0.200pt]{2.409pt}{0.400pt}}
\put(1429.0,773.0){\rule[-0.200pt]{2.409pt}{0.400pt}}
\put(170.0,773.0){\rule[-0.200pt]{2.409pt}{0.400pt}}
\put(1429.0,773.0){\rule[-0.200pt]{2.409pt}{0.400pt}}
\put(170.0,773.0){\rule[-0.200pt]{2.409pt}{0.400pt}}
\put(1429.0,773.0){\rule[-0.200pt]{2.409pt}{0.400pt}}
\put(170.0,774.0){\rule[-0.200pt]{2.409pt}{0.400pt}}
\put(1429.0,774.0){\rule[-0.200pt]{2.409pt}{0.400pt}}
\put(170.0,774.0){\rule[-0.200pt]{2.409pt}{0.400pt}}
\put(1429.0,774.0){\rule[-0.200pt]{2.409pt}{0.400pt}}
\put(170.0,774.0){\rule[-0.200pt]{2.409pt}{0.400pt}}
\put(1429.0,774.0){\rule[-0.200pt]{2.409pt}{0.400pt}}
\put(170.0,775.0){\rule[-0.200pt]{2.409pt}{0.400pt}}
\put(1429.0,775.0){\rule[-0.200pt]{2.409pt}{0.400pt}}
\put(170.0,775.0){\rule[-0.200pt]{2.409pt}{0.400pt}}
\put(1429.0,775.0){\rule[-0.200pt]{2.409pt}{0.400pt}}
\put(170.0,776.0){\rule[-0.200pt]{2.409pt}{0.400pt}}
\put(1429.0,776.0){\rule[-0.200pt]{2.409pt}{0.400pt}}
\put(170.0,776.0){\rule[-0.200pt]{2.409pt}{0.400pt}}
\put(1429.0,776.0){\rule[-0.200pt]{2.409pt}{0.400pt}}
\put(170.0,776.0){\rule[-0.200pt]{2.409pt}{0.400pt}}
\put(1429.0,776.0){\rule[-0.200pt]{2.409pt}{0.400pt}}
\put(170.0,777.0){\rule[-0.200pt]{2.409pt}{0.400pt}}
\put(1429.0,777.0){\rule[-0.200pt]{2.409pt}{0.400pt}}
\put(170.0,777.0){\rule[-0.200pt]{2.409pt}{0.400pt}}
\put(1429.0,777.0){\rule[-0.200pt]{2.409pt}{0.400pt}}
\put(170.0,777.0){\rule[-0.200pt]{2.409pt}{0.400pt}}
\put(1429.0,777.0){\rule[-0.200pt]{2.409pt}{0.400pt}}
\put(170.0,778.0){\rule[-0.200pt]{2.409pt}{0.400pt}}
\put(1429.0,778.0){\rule[-0.200pt]{2.409pt}{0.400pt}}
\put(170.0,778.0){\rule[-0.200pt]{2.409pt}{0.400pt}}
\put(1429.0,778.0){\rule[-0.200pt]{2.409pt}{0.400pt}}
\put(170.0,778.0){\rule[-0.200pt]{2.409pt}{0.400pt}}
\put(1429.0,778.0){\rule[-0.200pt]{2.409pt}{0.400pt}}
\put(170.0,779.0){\rule[-0.200pt]{2.409pt}{0.400pt}}
\put(1429.0,779.0){\rule[-0.200pt]{2.409pt}{0.400pt}}
\put(170.0,779.0){\rule[-0.200pt]{2.409pt}{0.400pt}}
\put(1429.0,779.0){\rule[-0.200pt]{2.409pt}{0.400pt}}
\put(170.0,779.0){\rule[-0.200pt]{2.409pt}{0.400pt}}
\put(1429.0,779.0){\rule[-0.200pt]{2.409pt}{0.400pt}}
\put(170.0,780.0){\rule[-0.200pt]{2.409pt}{0.400pt}}
\put(1429.0,780.0){\rule[-0.200pt]{2.409pt}{0.400pt}}
\put(170.0,780.0){\rule[-0.200pt]{2.409pt}{0.400pt}}
\put(1429.0,780.0){\rule[-0.200pt]{2.409pt}{0.400pt}}
\put(170.0,780.0){\rule[-0.200pt]{2.409pt}{0.400pt}}
\put(1429.0,780.0){\rule[-0.200pt]{2.409pt}{0.400pt}}
\put(170.0,780.0){\rule[-0.200pt]{2.409pt}{0.400pt}}
\put(1429.0,780.0){\rule[-0.200pt]{2.409pt}{0.400pt}}
\put(170.0,781.0){\rule[-0.200pt]{2.409pt}{0.400pt}}
\put(1429.0,781.0){\rule[-0.200pt]{2.409pt}{0.400pt}}
\put(170.0,781.0){\rule[-0.200pt]{2.409pt}{0.400pt}}
\put(1429.0,781.0){\rule[-0.200pt]{2.409pt}{0.400pt}}
\put(170.0,781.0){\rule[-0.200pt]{2.409pt}{0.400pt}}
\put(1429.0,781.0){\rule[-0.200pt]{2.409pt}{0.400pt}}
\put(170.0,782.0){\rule[-0.200pt]{2.409pt}{0.400pt}}
\put(1429.0,782.0){\rule[-0.200pt]{2.409pt}{0.400pt}}
\put(170.0,782.0){\rule[-0.200pt]{2.409pt}{0.400pt}}
\put(1429.0,782.0){\rule[-0.200pt]{2.409pt}{0.400pt}}
\put(170.0,782.0){\rule[-0.200pt]{2.409pt}{0.400pt}}
\put(1429.0,782.0){\rule[-0.200pt]{2.409pt}{0.400pt}}
\put(170.0,783.0){\rule[-0.200pt]{2.409pt}{0.400pt}}
\put(1429.0,783.0){\rule[-0.200pt]{2.409pt}{0.400pt}}
\put(170.0,783.0){\rule[-0.200pt]{2.409pt}{0.400pt}}
\put(1429.0,783.0){\rule[-0.200pt]{2.409pt}{0.400pt}}
\put(170.0,783.0){\rule[-0.200pt]{2.409pt}{0.400pt}}
\put(1429.0,783.0){\rule[-0.200pt]{2.409pt}{0.400pt}}
\put(170.0,783.0){\rule[-0.200pt]{2.409pt}{0.400pt}}
\put(1429.0,783.0){\rule[-0.200pt]{2.409pt}{0.400pt}}
\put(170.0,784.0){\rule[-0.200pt]{2.409pt}{0.400pt}}
\put(1429.0,784.0){\rule[-0.200pt]{2.409pt}{0.400pt}}
\put(170.0,784.0){\rule[-0.200pt]{2.409pt}{0.400pt}}
\put(1429.0,784.0){\rule[-0.200pt]{2.409pt}{0.400pt}}
\put(170.0,784.0){\rule[-0.200pt]{2.409pt}{0.400pt}}
\put(1429.0,784.0){\rule[-0.200pt]{2.409pt}{0.400pt}}
\put(170.0,784.0){\rule[-0.200pt]{2.409pt}{0.400pt}}
\put(1429.0,784.0){\rule[-0.200pt]{2.409pt}{0.400pt}}
\put(170.0,785.0){\rule[-0.200pt]{2.409pt}{0.400pt}}
\put(1429.0,785.0){\rule[-0.200pt]{2.409pt}{0.400pt}}
\put(170.0,785.0){\rule[-0.200pt]{2.409pt}{0.400pt}}
\put(1429.0,785.0){\rule[-0.200pt]{2.409pt}{0.400pt}}
\put(170.0,785.0){\rule[-0.200pt]{2.409pt}{0.400pt}}
\put(1429.0,785.0){\rule[-0.200pt]{2.409pt}{0.400pt}}
\put(170.0,786.0){\rule[-0.200pt]{2.409pt}{0.400pt}}
\put(1429.0,786.0){\rule[-0.200pt]{2.409pt}{0.400pt}}
\put(170.0,786.0){\rule[-0.200pt]{2.409pt}{0.400pt}}
\put(1429.0,786.0){\rule[-0.200pt]{2.409pt}{0.400pt}}
\put(170.0,786.0){\rule[-0.200pt]{2.409pt}{0.400pt}}
\put(1429.0,786.0){\rule[-0.200pt]{2.409pt}{0.400pt}}
\put(170.0,786.0){\rule[-0.200pt]{2.409pt}{0.400pt}}
\put(1429.0,786.0){\rule[-0.200pt]{2.409pt}{0.400pt}}
\put(170.0,787.0){\rule[-0.200pt]{2.409pt}{0.400pt}}
\put(1429.0,787.0){\rule[-0.200pt]{2.409pt}{0.400pt}}
\put(170.0,787.0){\rule[-0.200pt]{2.409pt}{0.400pt}}
\put(1429.0,787.0){\rule[-0.200pt]{2.409pt}{0.400pt}}
\put(170.0,787.0){\rule[-0.200pt]{2.409pt}{0.400pt}}
\put(1429.0,787.0){\rule[-0.200pt]{2.409pt}{0.400pt}}
\put(170.0,787.0){\rule[-0.200pt]{2.409pt}{0.400pt}}
\put(1429.0,787.0){\rule[-0.200pt]{2.409pt}{0.400pt}}
\put(170.0,788.0){\rule[-0.200pt]{2.409pt}{0.400pt}}
\put(1429.0,788.0){\rule[-0.200pt]{2.409pt}{0.400pt}}
\put(170.0,788.0){\rule[-0.200pt]{2.409pt}{0.400pt}}
\put(1429.0,788.0){\rule[-0.200pt]{2.409pt}{0.400pt}}
\put(170.0,788.0){\rule[-0.200pt]{2.409pt}{0.400pt}}
\put(1429.0,788.0){\rule[-0.200pt]{2.409pt}{0.400pt}}
\put(170.0,788.0){\rule[-0.200pt]{2.409pt}{0.400pt}}
\put(1429.0,788.0){\rule[-0.200pt]{2.409pt}{0.400pt}}
\put(170.0,789.0){\rule[-0.200pt]{2.409pt}{0.400pt}}
\put(1429.0,789.0){\rule[-0.200pt]{2.409pt}{0.400pt}}
\put(170.0,789.0){\rule[-0.200pt]{2.409pt}{0.400pt}}
\put(1429.0,789.0){\rule[-0.200pt]{2.409pt}{0.400pt}}
\put(170.0,789.0){\rule[-0.200pt]{2.409pt}{0.400pt}}
\put(1429.0,789.0){\rule[-0.200pt]{2.409pt}{0.400pt}}
\put(170.0,789.0){\rule[-0.200pt]{2.409pt}{0.400pt}}
\put(1429.0,789.0){\rule[-0.200pt]{2.409pt}{0.400pt}}
\put(170.0,790.0){\rule[-0.200pt]{2.409pt}{0.400pt}}
\put(1429.0,790.0){\rule[-0.200pt]{2.409pt}{0.400pt}}
\put(170.0,790.0){\rule[-0.200pt]{2.409pt}{0.400pt}}
\put(1429.0,790.0){\rule[-0.200pt]{2.409pt}{0.400pt}}
\put(170.0,790.0){\rule[-0.200pt]{2.409pt}{0.400pt}}
\put(1429.0,790.0){\rule[-0.200pt]{2.409pt}{0.400pt}}
\put(170.0,790.0){\rule[-0.200pt]{2.409pt}{0.400pt}}
\put(1429.0,790.0){\rule[-0.200pt]{2.409pt}{0.400pt}}
\put(170.0,791.0){\rule[-0.200pt]{2.409pt}{0.400pt}}
\put(1429.0,791.0){\rule[-0.200pt]{2.409pt}{0.400pt}}
\put(170.0,791.0){\rule[-0.200pt]{2.409pt}{0.400pt}}
\put(1429.0,791.0){\rule[-0.200pt]{2.409pt}{0.400pt}}
\put(170.0,791.0){\rule[-0.200pt]{2.409pt}{0.400pt}}
\put(1429.0,791.0){\rule[-0.200pt]{2.409pt}{0.400pt}}
\put(170.0,791.0){\rule[-0.200pt]{2.409pt}{0.400pt}}
\put(1429.0,791.0){\rule[-0.200pt]{2.409pt}{0.400pt}}
\put(170.0,791.0){\rule[-0.200pt]{2.409pt}{0.400pt}}
\put(1429.0,791.0){\rule[-0.200pt]{2.409pt}{0.400pt}}
\put(170.0,792.0){\rule[-0.200pt]{2.409pt}{0.400pt}}
\put(1429.0,792.0){\rule[-0.200pt]{2.409pt}{0.400pt}}
\put(170.0,792.0){\rule[-0.200pt]{2.409pt}{0.400pt}}
\put(1429.0,792.0){\rule[-0.200pt]{2.409pt}{0.400pt}}
\put(170.0,792.0){\rule[-0.200pt]{2.409pt}{0.400pt}}
\put(1429.0,792.0){\rule[-0.200pt]{2.409pt}{0.400pt}}
\put(170.0,792.0){\rule[-0.200pt]{2.409pt}{0.400pt}}
\put(1429.0,792.0){\rule[-0.200pt]{2.409pt}{0.400pt}}
\put(170.0,793.0){\rule[-0.200pt]{2.409pt}{0.400pt}}
\put(1429.0,793.0){\rule[-0.200pt]{2.409pt}{0.400pt}}
\put(170.0,793.0){\rule[-0.200pt]{2.409pt}{0.400pt}}
\put(1429.0,793.0){\rule[-0.200pt]{2.409pt}{0.400pt}}
\put(170.0,793.0){\rule[-0.200pt]{2.409pt}{0.400pt}}
\put(1429.0,793.0){\rule[-0.200pt]{2.409pt}{0.400pt}}
\put(170.0,793.0){\rule[-0.200pt]{2.409pt}{0.400pt}}
\put(1429.0,793.0){\rule[-0.200pt]{2.409pt}{0.400pt}}
\put(170.0,793.0){\rule[-0.200pt]{2.409pt}{0.400pt}}
\put(1429.0,793.0){\rule[-0.200pt]{2.409pt}{0.400pt}}
\put(170.0,794.0){\rule[-0.200pt]{2.409pt}{0.400pt}}
\put(1429.0,794.0){\rule[-0.200pt]{2.409pt}{0.400pt}}
\put(170.0,794.0){\rule[-0.200pt]{2.409pt}{0.400pt}}
\put(1429.0,794.0){\rule[-0.200pt]{2.409pt}{0.400pt}}
\put(170.0,794.0){\rule[-0.200pt]{2.409pt}{0.400pt}}
\put(1429.0,794.0){\rule[-0.200pt]{2.409pt}{0.400pt}}
\put(170.0,794.0){\rule[-0.200pt]{2.409pt}{0.400pt}}
\put(1429.0,794.0){\rule[-0.200pt]{2.409pt}{0.400pt}}
\put(170.0,794.0){\rule[-0.200pt]{2.409pt}{0.400pt}}
\put(1429.0,794.0){\rule[-0.200pt]{2.409pt}{0.400pt}}
\put(170.0,795.0){\rule[-0.200pt]{2.409pt}{0.400pt}}
\put(1429.0,795.0){\rule[-0.200pt]{2.409pt}{0.400pt}}
\put(170.0,795.0){\rule[-0.200pt]{2.409pt}{0.400pt}}
\put(1429.0,795.0){\rule[-0.200pt]{2.409pt}{0.400pt}}
\put(170.0,795.0){\rule[-0.200pt]{2.409pt}{0.400pt}}
\put(1429.0,795.0){\rule[-0.200pt]{2.409pt}{0.400pt}}
\put(170.0,795.0){\rule[-0.200pt]{2.409pt}{0.400pt}}
\put(1429.0,795.0){\rule[-0.200pt]{2.409pt}{0.400pt}}
\put(170.0,796.0){\rule[-0.200pt]{2.409pt}{0.400pt}}
\put(1429.0,796.0){\rule[-0.200pt]{2.409pt}{0.400pt}}
\put(170.0,796.0){\rule[-0.200pt]{2.409pt}{0.400pt}}
\put(1429.0,796.0){\rule[-0.200pt]{2.409pt}{0.400pt}}
\put(170.0,796.0){\rule[-0.200pt]{2.409pt}{0.400pt}}
\put(1429.0,796.0){\rule[-0.200pt]{2.409pt}{0.400pt}}
\put(170.0,796.0){\rule[-0.200pt]{2.409pt}{0.400pt}}
\put(1429.0,796.0){\rule[-0.200pt]{2.409pt}{0.400pt}}
\put(170.0,796.0){\rule[-0.200pt]{2.409pt}{0.400pt}}
\put(1429.0,796.0){\rule[-0.200pt]{2.409pt}{0.400pt}}
\put(170.0,797.0){\rule[-0.200pt]{2.409pt}{0.400pt}}
\put(1429.0,797.0){\rule[-0.200pt]{2.409pt}{0.400pt}}
\put(170.0,797.0){\rule[-0.200pt]{2.409pt}{0.400pt}}
\put(1429.0,797.0){\rule[-0.200pt]{2.409pt}{0.400pt}}
\put(170.0,797.0){\rule[-0.200pt]{2.409pt}{0.400pt}}
\put(1429.0,797.0){\rule[-0.200pt]{2.409pt}{0.400pt}}
\put(170.0,797.0){\rule[-0.200pt]{2.409pt}{0.400pt}}
\put(1429.0,797.0){\rule[-0.200pt]{2.409pt}{0.400pt}}
\put(170.0,797.0){\rule[-0.200pt]{2.409pt}{0.400pt}}
\put(1429.0,797.0){\rule[-0.200pt]{2.409pt}{0.400pt}}
\put(170.0,798.0){\rule[-0.200pt]{2.409pt}{0.400pt}}
\put(1429.0,798.0){\rule[-0.200pt]{2.409pt}{0.400pt}}
\put(170.0,798.0){\rule[-0.200pt]{2.409pt}{0.400pt}}
\put(1429.0,798.0){\rule[-0.200pt]{2.409pt}{0.400pt}}
\put(170.0,798.0){\rule[-0.200pt]{2.409pt}{0.400pt}}
\put(1429.0,798.0){\rule[-0.200pt]{2.409pt}{0.400pt}}
\put(170.0,798.0){\rule[-0.200pt]{2.409pt}{0.400pt}}
\put(1429.0,798.0){\rule[-0.200pt]{2.409pt}{0.400pt}}
\put(170.0,798.0){\rule[-0.200pt]{2.409pt}{0.400pt}}
\put(1429.0,798.0){\rule[-0.200pt]{2.409pt}{0.400pt}}
\put(170.0,798.0){\rule[-0.200pt]{2.409pt}{0.400pt}}
\put(1429.0,798.0){\rule[-0.200pt]{2.409pt}{0.400pt}}
\put(170.0,799.0){\rule[-0.200pt]{2.409pt}{0.400pt}}
\put(1429.0,799.0){\rule[-0.200pt]{2.409pt}{0.400pt}}
\put(170.0,799.0){\rule[-0.200pt]{2.409pt}{0.400pt}}
\put(1429.0,799.0){\rule[-0.200pt]{2.409pt}{0.400pt}}
\put(170.0,799.0){\rule[-0.200pt]{2.409pt}{0.400pt}}
\put(1429.0,799.0){\rule[-0.200pt]{2.409pt}{0.400pt}}
\put(170.0,799.0){\rule[-0.200pt]{2.409pt}{0.400pt}}
\put(1429.0,799.0){\rule[-0.200pt]{2.409pt}{0.400pt}}
\put(170.0,799.0){\rule[-0.200pt]{2.409pt}{0.400pt}}
\put(1429.0,799.0){\rule[-0.200pt]{2.409pt}{0.400pt}}
\put(170.0,800.0){\rule[-0.200pt]{2.409pt}{0.400pt}}
\put(1429.0,800.0){\rule[-0.200pt]{2.409pt}{0.400pt}}
\put(170.0,800.0){\rule[-0.200pt]{2.409pt}{0.400pt}}
\put(1429.0,800.0){\rule[-0.200pt]{2.409pt}{0.400pt}}
\put(170.0,800.0){\rule[-0.200pt]{2.409pt}{0.400pt}}
\put(1429.0,800.0){\rule[-0.200pt]{2.409pt}{0.400pt}}
\put(170.0,800.0){\rule[-0.200pt]{2.409pt}{0.400pt}}
\put(1429.0,800.0){\rule[-0.200pt]{2.409pt}{0.400pt}}
\put(170.0,800.0){\rule[-0.200pt]{2.409pt}{0.400pt}}
\put(1429.0,800.0){\rule[-0.200pt]{2.409pt}{0.400pt}}
\put(170.0,800.0){\rule[-0.200pt]{2.409pt}{0.400pt}}
\put(1429.0,800.0){\rule[-0.200pt]{2.409pt}{0.400pt}}
\put(170.0,801.0){\rule[-0.200pt]{2.409pt}{0.400pt}}
\put(1429.0,801.0){\rule[-0.200pt]{2.409pt}{0.400pt}}
\put(170.0,801.0){\rule[-0.200pt]{2.409pt}{0.400pt}}
\put(1429.0,801.0){\rule[-0.200pt]{2.409pt}{0.400pt}}
\put(170.0,801.0){\rule[-0.200pt]{2.409pt}{0.400pt}}
\put(1429.0,801.0){\rule[-0.200pt]{2.409pt}{0.400pt}}
\put(170.0,801.0){\rule[-0.200pt]{2.409pt}{0.400pt}}
\put(1429.0,801.0){\rule[-0.200pt]{2.409pt}{0.400pt}}
\put(170.0,801.0){\rule[-0.200pt]{2.409pt}{0.400pt}}
\put(1429.0,801.0){\rule[-0.200pt]{2.409pt}{0.400pt}}
\put(170.0,802.0){\rule[-0.200pt]{2.409pt}{0.400pt}}
\put(1429.0,802.0){\rule[-0.200pt]{2.409pt}{0.400pt}}
\put(170.0,802.0){\rule[-0.200pt]{2.409pt}{0.400pt}}
\put(1429.0,802.0){\rule[-0.200pt]{2.409pt}{0.400pt}}
\put(170.0,802.0){\rule[-0.200pt]{2.409pt}{0.400pt}}
\put(1429.0,802.0){\rule[-0.200pt]{2.409pt}{0.400pt}}
\put(170.0,802.0){\rule[-0.200pt]{2.409pt}{0.400pt}}
\put(1429.0,802.0){\rule[-0.200pt]{2.409pt}{0.400pt}}
\put(170.0,802.0){\rule[-0.200pt]{2.409pt}{0.400pt}}
\put(1429.0,802.0){\rule[-0.200pt]{2.409pt}{0.400pt}}
\put(170.0,802.0){\rule[-0.200pt]{2.409pt}{0.400pt}}
\put(1429.0,802.0){\rule[-0.200pt]{2.409pt}{0.400pt}}
\put(170.0,803.0){\rule[-0.200pt]{2.409pt}{0.400pt}}
\put(1429.0,803.0){\rule[-0.200pt]{2.409pt}{0.400pt}}
\put(170.0,803.0){\rule[-0.200pt]{2.409pt}{0.400pt}}
\put(1429.0,803.0){\rule[-0.200pt]{2.409pt}{0.400pt}}
\put(170.0,803.0){\rule[-0.200pt]{2.409pt}{0.400pt}}
\put(1429.0,803.0){\rule[-0.200pt]{2.409pt}{0.400pt}}
\put(170.0,803.0){\rule[-0.200pt]{2.409pt}{0.400pt}}
\put(1429.0,803.0){\rule[-0.200pt]{2.409pt}{0.400pt}}
\put(170.0,803.0){\rule[-0.200pt]{2.409pt}{0.400pt}}
\put(1429.0,803.0){\rule[-0.200pt]{2.409pt}{0.400pt}}
\put(170.0,803.0){\rule[-0.200pt]{2.409pt}{0.400pt}}
\put(1429.0,803.0){\rule[-0.200pt]{2.409pt}{0.400pt}}
\put(170.0,804.0){\rule[-0.200pt]{2.409pt}{0.400pt}}
\put(1429.0,804.0){\rule[-0.200pt]{2.409pt}{0.400pt}}
\put(170.0,804.0){\rule[-0.200pt]{2.409pt}{0.400pt}}
\put(1429.0,804.0){\rule[-0.200pt]{2.409pt}{0.400pt}}
\put(170.0,804.0){\rule[-0.200pt]{2.409pt}{0.400pt}}
\put(1429.0,804.0){\rule[-0.200pt]{2.409pt}{0.400pt}}
\put(170.0,804.0){\rule[-0.200pt]{2.409pt}{0.400pt}}
\put(1429.0,804.0){\rule[-0.200pt]{2.409pt}{0.400pt}}
\put(170.0,804.0){\rule[-0.200pt]{2.409pt}{0.400pt}}
\put(1429.0,804.0){\rule[-0.200pt]{2.409pt}{0.400pt}}
\put(170.0,804.0){\rule[-0.200pt]{2.409pt}{0.400pt}}
\put(1429.0,804.0){\rule[-0.200pt]{2.409pt}{0.400pt}}
\put(170.0,805.0){\rule[-0.200pt]{2.409pt}{0.400pt}}
\put(1429.0,805.0){\rule[-0.200pt]{2.409pt}{0.400pt}}
\put(170.0,805.0){\rule[-0.200pt]{2.409pt}{0.400pt}}
\put(1429.0,805.0){\rule[-0.200pt]{2.409pt}{0.400pt}}
\put(170.0,805.0){\rule[-0.200pt]{2.409pt}{0.400pt}}
\put(1429.0,805.0){\rule[-0.200pt]{2.409pt}{0.400pt}}
\put(170.0,805.0){\rule[-0.200pt]{2.409pt}{0.400pt}}
\put(1429.0,805.0){\rule[-0.200pt]{2.409pt}{0.400pt}}
\put(170.0,805.0){\rule[-0.200pt]{2.409pt}{0.400pt}}
\put(1429.0,805.0){\rule[-0.200pt]{2.409pt}{0.400pt}}
\put(170.0,805.0){\rule[-0.200pt]{2.409pt}{0.400pt}}
\put(1429.0,805.0){\rule[-0.200pt]{2.409pt}{0.400pt}}
\put(170.0,805.0){\rule[-0.200pt]{2.409pt}{0.400pt}}
\put(1429.0,805.0){\rule[-0.200pt]{2.409pt}{0.400pt}}
\put(170.0,806.0){\rule[-0.200pt]{2.409pt}{0.400pt}}
\put(1429.0,806.0){\rule[-0.200pt]{2.409pt}{0.400pt}}
\put(170.0,806.0){\rule[-0.200pt]{2.409pt}{0.400pt}}
\put(1429.0,806.0){\rule[-0.200pt]{2.409pt}{0.400pt}}
\put(170.0,806.0){\rule[-0.200pt]{2.409pt}{0.400pt}}
\put(1429.0,806.0){\rule[-0.200pt]{2.409pt}{0.400pt}}
\put(170.0,806.0){\rule[-0.200pt]{2.409pt}{0.400pt}}
\put(1429.0,806.0){\rule[-0.200pt]{2.409pt}{0.400pt}}
\put(170.0,806.0){\rule[-0.200pt]{2.409pt}{0.400pt}}
\put(1429.0,806.0){\rule[-0.200pt]{2.409pt}{0.400pt}}
\put(170.0,806.0){\rule[-0.200pt]{2.409pt}{0.400pt}}
\put(1429.0,806.0){\rule[-0.200pt]{2.409pt}{0.400pt}}
\put(170.0,807.0){\rule[-0.200pt]{2.409pt}{0.400pt}}
\put(1429.0,807.0){\rule[-0.200pt]{2.409pt}{0.400pt}}
\put(170.0,807.0){\rule[-0.200pt]{2.409pt}{0.400pt}}
\put(1429.0,807.0){\rule[-0.200pt]{2.409pt}{0.400pt}}
\put(170.0,807.0){\rule[-0.200pt]{2.409pt}{0.400pt}}
\put(1429.0,807.0){\rule[-0.200pt]{2.409pt}{0.400pt}}
\put(170.0,807.0){\rule[-0.200pt]{2.409pt}{0.400pt}}
\put(1429.0,807.0){\rule[-0.200pt]{2.409pt}{0.400pt}}
\put(170.0,807.0){\rule[-0.200pt]{2.409pt}{0.400pt}}
\put(1429.0,807.0){\rule[-0.200pt]{2.409pt}{0.400pt}}
\put(170.0,807.0){\rule[-0.200pt]{2.409pt}{0.400pt}}
\put(1429.0,807.0){\rule[-0.200pt]{2.409pt}{0.400pt}}
\put(170.0,807.0){\rule[-0.200pt]{2.409pt}{0.400pt}}
\put(1429.0,807.0){\rule[-0.200pt]{2.409pt}{0.400pt}}
\put(170.0,808.0){\rule[-0.200pt]{2.409pt}{0.400pt}}
\put(1429.0,808.0){\rule[-0.200pt]{2.409pt}{0.400pt}}
\put(170.0,808.0){\rule[-0.200pt]{2.409pt}{0.400pt}}
\put(1429.0,808.0){\rule[-0.200pt]{2.409pt}{0.400pt}}
\put(170.0,808.0){\rule[-0.200pt]{2.409pt}{0.400pt}}
\put(1429.0,808.0){\rule[-0.200pt]{2.409pt}{0.400pt}}
\put(170.0,808.0){\rule[-0.200pt]{2.409pt}{0.400pt}}
\put(1429.0,808.0){\rule[-0.200pt]{2.409pt}{0.400pt}}
\put(170.0,808.0){\rule[-0.200pt]{2.409pt}{0.400pt}}
\put(1429.0,808.0){\rule[-0.200pt]{2.409pt}{0.400pt}}
\put(170.0,808.0){\rule[-0.200pt]{2.409pt}{0.400pt}}
\put(1429.0,808.0){\rule[-0.200pt]{2.409pt}{0.400pt}}
\put(170.0,808.0){\rule[-0.200pt]{2.409pt}{0.400pt}}
\put(1429.0,808.0){\rule[-0.200pt]{2.409pt}{0.400pt}}
\put(170.0,809.0){\rule[-0.200pt]{2.409pt}{0.400pt}}
\put(1429.0,809.0){\rule[-0.200pt]{2.409pt}{0.400pt}}
\put(170.0,809.0){\rule[-0.200pt]{2.409pt}{0.400pt}}
\put(1429.0,809.0){\rule[-0.200pt]{2.409pt}{0.400pt}}
\put(170.0,809.0){\rule[-0.200pt]{2.409pt}{0.400pt}}
\put(1429.0,809.0){\rule[-0.200pt]{2.409pt}{0.400pt}}
\put(170.0,809.0){\rule[-0.200pt]{2.409pt}{0.400pt}}
\put(1429.0,809.0){\rule[-0.200pt]{2.409pt}{0.400pt}}
\put(170.0,809.0){\rule[-0.200pt]{2.409pt}{0.400pt}}
\put(1429.0,809.0){\rule[-0.200pt]{2.409pt}{0.400pt}}
\put(170.0,809.0){\rule[-0.200pt]{2.409pt}{0.400pt}}
\put(1429.0,809.0){\rule[-0.200pt]{2.409pt}{0.400pt}}
\put(170.0,809.0){\rule[-0.200pt]{2.409pt}{0.400pt}}
\put(1429.0,809.0){\rule[-0.200pt]{2.409pt}{0.400pt}}
\put(170.0,810.0){\rule[-0.200pt]{2.409pt}{0.400pt}}
\put(1429.0,810.0){\rule[-0.200pt]{2.409pt}{0.400pt}}
\put(170.0,810.0){\rule[-0.200pt]{2.409pt}{0.400pt}}
\put(1429.0,810.0){\rule[-0.200pt]{2.409pt}{0.400pt}}
\put(170.0,810.0){\rule[-0.200pt]{2.409pt}{0.400pt}}
\put(1429.0,810.0){\rule[-0.200pt]{2.409pt}{0.400pt}}
\put(170.0,810.0){\rule[-0.200pt]{2.409pt}{0.400pt}}
\put(1429.0,810.0){\rule[-0.200pt]{2.409pt}{0.400pt}}
\put(170.0,810.0){\rule[-0.200pt]{2.409pt}{0.400pt}}
\put(1429.0,810.0){\rule[-0.200pt]{2.409pt}{0.400pt}}
\put(170.0,810.0){\rule[-0.200pt]{2.409pt}{0.400pt}}
\put(1429.0,810.0){\rule[-0.200pt]{2.409pt}{0.400pt}}
\put(170.0,810.0){\rule[-0.200pt]{2.409pt}{0.400pt}}
\put(1429.0,810.0){\rule[-0.200pt]{2.409pt}{0.400pt}}
\put(170.0,811.0){\rule[-0.200pt]{2.409pt}{0.400pt}}
\put(1429.0,811.0){\rule[-0.200pt]{2.409pt}{0.400pt}}
\put(170.0,811.0){\rule[-0.200pt]{2.409pt}{0.400pt}}
\put(1429.0,811.0){\rule[-0.200pt]{2.409pt}{0.400pt}}
\put(170.0,811.0){\rule[-0.200pt]{2.409pt}{0.400pt}}
\put(1429.0,811.0){\rule[-0.200pt]{2.409pt}{0.400pt}}
\put(170.0,811.0){\rule[-0.200pt]{2.409pt}{0.400pt}}
\put(1429.0,811.0){\rule[-0.200pt]{2.409pt}{0.400pt}}
\put(170.0,811.0){\rule[-0.200pt]{2.409pt}{0.400pt}}
\put(1429.0,811.0){\rule[-0.200pt]{2.409pt}{0.400pt}}
\put(170.0,811.0){\rule[-0.200pt]{2.409pt}{0.400pt}}
\put(1429.0,811.0){\rule[-0.200pt]{2.409pt}{0.400pt}}
\put(170.0,811.0){\rule[-0.200pt]{2.409pt}{0.400pt}}
\put(1429.0,811.0){\rule[-0.200pt]{2.409pt}{0.400pt}}
\put(170.0,812.0){\rule[-0.200pt]{2.409pt}{0.400pt}}
\put(1429.0,812.0){\rule[-0.200pt]{2.409pt}{0.400pt}}
\put(170.0,812.0){\rule[-0.200pt]{2.409pt}{0.400pt}}
\put(1429.0,812.0){\rule[-0.200pt]{2.409pt}{0.400pt}}
\put(170.0,812.0){\rule[-0.200pt]{2.409pt}{0.400pt}}
\put(1429.0,812.0){\rule[-0.200pt]{2.409pt}{0.400pt}}
\put(170.0,812.0){\rule[-0.200pt]{2.409pt}{0.400pt}}
\put(1429.0,812.0){\rule[-0.200pt]{2.409pt}{0.400pt}}
\put(170.0,812.0){\rule[-0.200pt]{2.409pt}{0.400pt}}
\put(1429.0,812.0){\rule[-0.200pt]{2.409pt}{0.400pt}}
\put(170.0,812.0){\rule[-0.200pt]{2.409pt}{0.400pt}}
\put(1429.0,812.0){\rule[-0.200pt]{2.409pt}{0.400pt}}
\put(170.0,812.0){\rule[-0.200pt]{2.409pt}{0.400pt}}
\put(1429.0,812.0){\rule[-0.200pt]{2.409pt}{0.400pt}}
\put(170.0,812.0){\rule[-0.200pt]{2.409pt}{0.400pt}}
\put(1429.0,812.0){\rule[-0.200pt]{2.409pt}{0.400pt}}
\put(170.0,813.0){\rule[-0.200pt]{2.409pt}{0.400pt}}
\put(1429.0,813.0){\rule[-0.200pt]{2.409pt}{0.400pt}}
\put(170.0,813.0){\rule[-0.200pt]{2.409pt}{0.400pt}}
\put(1429.0,813.0){\rule[-0.200pt]{2.409pt}{0.400pt}}
\put(170.0,813.0){\rule[-0.200pt]{2.409pt}{0.400pt}}
\put(1429.0,813.0){\rule[-0.200pt]{2.409pt}{0.400pt}}
\put(170.0,813.0){\rule[-0.200pt]{2.409pt}{0.400pt}}
\put(1429.0,813.0){\rule[-0.200pt]{2.409pt}{0.400pt}}
\put(170.0,813.0){\rule[-0.200pt]{2.409pt}{0.400pt}}
\put(1429.0,813.0){\rule[-0.200pt]{2.409pt}{0.400pt}}
\put(170.0,813.0){\rule[-0.200pt]{2.409pt}{0.400pt}}
\put(1429.0,813.0){\rule[-0.200pt]{2.409pt}{0.400pt}}
\put(170.0,813.0){\rule[-0.200pt]{2.409pt}{0.400pt}}
\put(1429.0,813.0){\rule[-0.200pt]{2.409pt}{0.400pt}}
\put(170.0,813.0){\rule[-0.200pt]{2.409pt}{0.400pt}}
\put(1429.0,813.0){\rule[-0.200pt]{2.409pt}{0.400pt}}
\put(170.0,814.0){\rule[-0.200pt]{2.409pt}{0.400pt}}
\put(1429.0,814.0){\rule[-0.200pt]{2.409pt}{0.400pt}}
\put(170.0,814.0){\rule[-0.200pt]{2.409pt}{0.400pt}}
\put(1429.0,814.0){\rule[-0.200pt]{2.409pt}{0.400pt}}
\put(170.0,814.0){\rule[-0.200pt]{2.409pt}{0.400pt}}
\put(1429.0,814.0){\rule[-0.200pt]{2.409pt}{0.400pt}}
\put(170.0,814.0){\rule[-0.200pt]{2.409pt}{0.400pt}}
\put(1429.0,814.0){\rule[-0.200pt]{2.409pt}{0.400pt}}
\put(170.0,814.0){\rule[-0.200pt]{2.409pt}{0.400pt}}
\put(1429.0,814.0){\rule[-0.200pt]{2.409pt}{0.400pt}}
\put(170.0,814.0){\rule[-0.200pt]{2.409pt}{0.400pt}}
\put(1429.0,814.0){\rule[-0.200pt]{2.409pt}{0.400pt}}
\put(170.0,814.0){\rule[-0.200pt]{2.409pt}{0.400pt}}
\put(1429.0,814.0){\rule[-0.200pt]{2.409pt}{0.400pt}}
\put(170.0,814.0){\rule[-0.200pt]{2.409pt}{0.400pt}}
\put(1429.0,814.0){\rule[-0.200pt]{2.409pt}{0.400pt}}
\put(170.0,815.0){\rule[-0.200pt]{2.409pt}{0.400pt}}
\put(1429.0,815.0){\rule[-0.200pt]{2.409pt}{0.400pt}}
\put(170.0,815.0){\rule[-0.200pt]{2.409pt}{0.400pt}}
\put(1429.0,815.0){\rule[-0.200pt]{2.409pt}{0.400pt}}
\put(170.0,815.0){\rule[-0.200pt]{2.409pt}{0.400pt}}
\put(1429.0,815.0){\rule[-0.200pt]{2.409pt}{0.400pt}}
\put(170.0,815.0){\rule[-0.200pt]{2.409pt}{0.400pt}}
\put(1429.0,815.0){\rule[-0.200pt]{2.409pt}{0.400pt}}
\put(170.0,815.0){\rule[-0.200pt]{2.409pt}{0.400pt}}
\put(1429.0,815.0){\rule[-0.200pt]{2.409pt}{0.400pt}}
\put(170.0,815.0){\rule[-0.200pt]{2.409pt}{0.400pt}}
\put(1429.0,815.0){\rule[-0.200pt]{2.409pt}{0.400pt}}
\put(170.0,815.0){\rule[-0.200pt]{2.409pt}{0.400pt}}
\put(1429.0,815.0){\rule[-0.200pt]{2.409pt}{0.400pt}}
\put(170.0,815.0){\rule[-0.200pt]{2.409pt}{0.400pt}}
\put(1429.0,815.0){\rule[-0.200pt]{2.409pt}{0.400pt}}
\put(170.0,816.0){\rule[-0.200pt]{2.409pt}{0.400pt}}
\put(1429.0,816.0){\rule[-0.200pt]{2.409pt}{0.400pt}}
\put(170.0,816.0){\rule[-0.200pt]{2.409pt}{0.400pt}}
\put(1429.0,816.0){\rule[-0.200pt]{2.409pt}{0.400pt}}
\put(170.0,816.0){\rule[-0.200pt]{2.409pt}{0.400pt}}
\put(1429.0,816.0){\rule[-0.200pt]{2.409pt}{0.400pt}}
\put(170.0,816.0){\rule[-0.200pt]{2.409pt}{0.400pt}}
\put(1429.0,816.0){\rule[-0.200pt]{2.409pt}{0.400pt}}
\put(170.0,816.0){\rule[-0.200pt]{2.409pt}{0.400pt}}
\put(1429.0,816.0){\rule[-0.200pt]{2.409pt}{0.400pt}}
\put(170.0,816.0){\rule[-0.200pt]{2.409pt}{0.400pt}}
\put(1429.0,816.0){\rule[-0.200pt]{2.409pt}{0.400pt}}
\put(170.0,816.0){\rule[-0.200pt]{2.409pt}{0.400pt}}
\put(1429.0,816.0){\rule[-0.200pt]{2.409pt}{0.400pt}}
\put(170.0,816.0){\rule[-0.200pt]{2.409pt}{0.400pt}}
\put(1429.0,816.0){\rule[-0.200pt]{2.409pt}{0.400pt}}
\put(170.0,817.0){\rule[-0.200pt]{2.409pt}{0.400pt}}
\put(1429.0,817.0){\rule[-0.200pt]{2.409pt}{0.400pt}}
\put(170.0,817.0){\rule[-0.200pt]{2.409pt}{0.400pt}}
\put(1429.0,817.0){\rule[-0.200pt]{2.409pt}{0.400pt}}
\put(170.0,817.0){\rule[-0.200pt]{2.409pt}{0.400pt}}
\put(1429.0,817.0){\rule[-0.200pt]{2.409pt}{0.400pt}}
\put(170.0,817.0){\rule[-0.200pt]{2.409pt}{0.400pt}}
\put(1429.0,817.0){\rule[-0.200pt]{2.409pt}{0.400pt}}
\put(170.0,817.0){\rule[-0.200pt]{2.409pt}{0.400pt}}
\put(1429.0,817.0){\rule[-0.200pt]{2.409pt}{0.400pt}}
\put(170.0,817.0){\rule[-0.200pt]{2.409pt}{0.400pt}}
\put(1429.0,817.0){\rule[-0.200pt]{2.409pt}{0.400pt}}
\put(170.0,817.0){\rule[-0.200pt]{2.409pt}{0.400pt}}
\put(1429.0,817.0){\rule[-0.200pt]{2.409pt}{0.400pt}}
\put(170.0,817.0){\rule[-0.200pt]{2.409pt}{0.400pt}}
\put(1429.0,817.0){\rule[-0.200pt]{2.409pt}{0.400pt}}
\put(170.0,817.0){\rule[-0.200pt]{2.409pt}{0.400pt}}
\put(1429.0,817.0){\rule[-0.200pt]{2.409pt}{0.400pt}}
\put(170.0,818.0){\rule[-0.200pt]{2.409pt}{0.400pt}}
\put(1429.0,818.0){\rule[-0.200pt]{2.409pt}{0.400pt}}
\put(170.0,818.0){\rule[-0.200pt]{2.409pt}{0.400pt}}
\put(1429.0,818.0){\rule[-0.200pt]{2.409pt}{0.400pt}}
\put(170.0,818.0){\rule[-0.200pt]{2.409pt}{0.400pt}}
\put(1429.0,818.0){\rule[-0.200pt]{2.409pt}{0.400pt}}
\put(170.0,818.0){\rule[-0.200pt]{2.409pt}{0.400pt}}
\put(1429.0,818.0){\rule[-0.200pt]{2.409pt}{0.400pt}}
\put(170.0,818.0){\rule[-0.200pt]{2.409pt}{0.400pt}}
\put(1429.0,818.0){\rule[-0.200pt]{2.409pt}{0.400pt}}
\put(170.0,818.0){\rule[-0.200pt]{2.409pt}{0.400pt}}
\put(1429.0,818.0){\rule[-0.200pt]{2.409pt}{0.400pt}}
\put(170.0,818.0){\rule[-0.200pt]{2.409pt}{0.400pt}}
\put(1429.0,818.0){\rule[-0.200pt]{2.409pt}{0.400pt}}
\put(170.0,818.0){\rule[-0.200pt]{2.409pt}{0.400pt}}
\put(1429.0,818.0){\rule[-0.200pt]{2.409pt}{0.400pt}}
\put(170.0,818.0){\rule[-0.200pt]{2.409pt}{0.400pt}}
\put(1429.0,818.0){\rule[-0.200pt]{2.409pt}{0.400pt}}
\put(170.0,819.0){\rule[-0.200pt]{2.409pt}{0.400pt}}
\put(1429.0,819.0){\rule[-0.200pt]{2.409pt}{0.400pt}}
\put(170.0,819.0){\rule[-0.200pt]{2.409pt}{0.400pt}}
\put(1429.0,819.0){\rule[-0.200pt]{2.409pt}{0.400pt}}
\put(170.0,819.0){\rule[-0.200pt]{2.409pt}{0.400pt}}
\put(1429.0,819.0){\rule[-0.200pt]{2.409pt}{0.400pt}}
\put(170.0,819.0){\rule[-0.200pt]{2.409pt}{0.400pt}}
\put(1429.0,819.0){\rule[-0.200pt]{2.409pt}{0.400pt}}
\put(170.0,819.0){\rule[-0.200pt]{2.409pt}{0.400pt}}
\put(1429.0,819.0){\rule[-0.200pt]{2.409pt}{0.400pt}}
\put(170.0,819.0){\rule[-0.200pt]{2.409pt}{0.400pt}}
\put(1429.0,819.0){\rule[-0.200pt]{2.409pt}{0.400pt}}
\put(170.0,819.0){\rule[-0.200pt]{2.409pt}{0.400pt}}
\put(1429.0,819.0){\rule[-0.200pt]{2.409pt}{0.400pt}}
\put(170.0,819.0){\rule[-0.200pt]{2.409pt}{0.400pt}}
\put(1429.0,819.0){\rule[-0.200pt]{2.409pt}{0.400pt}}
\put(170.0,819.0){\rule[-0.200pt]{2.409pt}{0.400pt}}
\put(1429.0,819.0){\rule[-0.200pt]{2.409pt}{0.400pt}}
\put(170.0,820.0){\rule[-0.200pt]{2.409pt}{0.400pt}}
\put(1429.0,820.0){\rule[-0.200pt]{2.409pt}{0.400pt}}
\put(170.0,820.0){\rule[-0.200pt]{2.409pt}{0.400pt}}
\put(1429.0,820.0){\rule[-0.200pt]{2.409pt}{0.400pt}}
\put(170.0,820.0){\rule[-0.200pt]{2.409pt}{0.400pt}}
\put(1429.0,820.0){\rule[-0.200pt]{2.409pt}{0.400pt}}
\put(170.0,820.0){\rule[-0.200pt]{2.409pt}{0.400pt}}
\put(1429.0,820.0){\rule[-0.200pt]{2.409pt}{0.400pt}}
\put(170.0,820.0){\rule[-0.200pt]{2.409pt}{0.400pt}}
\put(1429.0,820.0){\rule[-0.200pt]{2.409pt}{0.400pt}}
\put(170.0,820.0){\rule[-0.200pt]{2.409pt}{0.400pt}}
\put(1429.0,820.0){\rule[-0.200pt]{2.409pt}{0.400pt}}
\put(170.0,820.0){\rule[-0.200pt]{2.409pt}{0.400pt}}
\put(1429.0,820.0){\rule[-0.200pt]{2.409pt}{0.400pt}}
\put(170.0,820.0){\rule[-0.200pt]{2.409pt}{0.400pt}}
\put(1429.0,820.0){\rule[-0.200pt]{2.409pt}{0.400pt}}
\put(170.0,820.0){\rule[-0.200pt]{2.409pt}{0.400pt}}
\put(1429.0,820.0){\rule[-0.200pt]{2.409pt}{0.400pt}}
\put(170.0,820.0){\rule[-0.200pt]{2.409pt}{0.400pt}}
\put(1429.0,820.0){\rule[-0.200pt]{2.409pt}{0.400pt}}
\put(170.0,821.0){\rule[-0.200pt]{2.409pt}{0.400pt}}
\put(1429.0,821.0){\rule[-0.200pt]{2.409pt}{0.400pt}}
\put(170.0,821.0){\rule[-0.200pt]{2.409pt}{0.400pt}}
\put(1429.0,821.0){\rule[-0.200pt]{2.409pt}{0.400pt}}
\put(170.0,821.0){\rule[-0.200pt]{2.409pt}{0.400pt}}
\put(1429.0,821.0){\rule[-0.200pt]{2.409pt}{0.400pt}}
\put(170.0,821.0){\rule[-0.200pt]{2.409pt}{0.400pt}}
\put(1429.0,821.0){\rule[-0.200pt]{2.409pt}{0.400pt}}
\put(170.0,821.0){\rule[-0.200pt]{2.409pt}{0.400pt}}
\put(1429.0,821.0){\rule[-0.200pt]{2.409pt}{0.400pt}}
\put(170.0,821.0){\rule[-0.200pt]{2.409pt}{0.400pt}}
\put(1429.0,821.0){\rule[-0.200pt]{2.409pt}{0.400pt}}
\put(170.0,821.0){\rule[-0.200pt]{2.409pt}{0.400pt}}
\put(1429.0,821.0){\rule[-0.200pt]{2.409pt}{0.400pt}}
\put(170.0,821.0){\rule[-0.200pt]{2.409pt}{0.400pt}}
\put(1429.0,821.0){\rule[-0.200pt]{2.409pt}{0.400pt}}
\put(170.0,821.0){\rule[-0.200pt]{2.409pt}{0.400pt}}
\put(1429.0,821.0){\rule[-0.200pt]{2.409pt}{0.400pt}}
\put(170.0,822.0){\rule[-0.200pt]{2.409pt}{0.400pt}}
\put(1429.0,822.0){\rule[-0.200pt]{2.409pt}{0.400pt}}
\put(170.0,822.0){\rule[-0.200pt]{2.409pt}{0.400pt}}
\put(1429.0,822.0){\rule[-0.200pt]{2.409pt}{0.400pt}}
\put(170.0,822.0){\rule[-0.200pt]{2.409pt}{0.400pt}}
\put(1429.0,822.0){\rule[-0.200pt]{2.409pt}{0.400pt}}
\put(170.0,822.0){\rule[-0.200pt]{2.409pt}{0.400pt}}
\put(1429.0,822.0){\rule[-0.200pt]{2.409pt}{0.400pt}}
\put(170.0,822.0){\rule[-0.200pt]{2.409pt}{0.400pt}}
\put(1429.0,822.0){\rule[-0.200pt]{2.409pt}{0.400pt}}
\put(170.0,822.0){\rule[-0.200pt]{2.409pt}{0.400pt}}
\put(1429.0,822.0){\rule[-0.200pt]{2.409pt}{0.400pt}}
\put(170.0,822.0){\rule[-0.200pt]{2.409pt}{0.400pt}}
\put(1429.0,822.0){\rule[-0.200pt]{2.409pt}{0.400pt}}
\put(170.0,822.0){\rule[-0.200pt]{2.409pt}{0.400pt}}
\put(1429.0,822.0){\rule[-0.200pt]{2.409pt}{0.400pt}}
\put(170.0,822.0){\rule[-0.200pt]{2.409pt}{0.400pt}}
\put(1429.0,822.0){\rule[-0.200pt]{2.409pt}{0.400pt}}
\put(170.0,822.0){\rule[-0.200pt]{2.409pt}{0.400pt}}
\put(1429.0,822.0){\rule[-0.200pt]{2.409pt}{0.400pt}}
\put(170.0,823.0){\rule[-0.200pt]{2.409pt}{0.400pt}}
\put(1429.0,823.0){\rule[-0.200pt]{2.409pt}{0.400pt}}
\put(170.0,823.0){\rule[-0.200pt]{2.409pt}{0.400pt}}
\put(1429.0,823.0){\rule[-0.200pt]{2.409pt}{0.400pt}}
\put(170.0,823.0){\rule[-0.200pt]{2.409pt}{0.400pt}}
\put(1429.0,823.0){\rule[-0.200pt]{2.409pt}{0.400pt}}
\put(170.0,823.0){\rule[-0.200pt]{2.409pt}{0.400pt}}
\put(1429.0,823.0){\rule[-0.200pt]{2.409pt}{0.400pt}}
\put(170.0,823.0){\rule[-0.200pt]{2.409pt}{0.400pt}}
\put(1429.0,823.0){\rule[-0.200pt]{2.409pt}{0.400pt}}
\put(170.0,823.0){\rule[-0.200pt]{2.409pt}{0.400pt}}
\put(1429.0,823.0){\rule[-0.200pt]{2.409pt}{0.400pt}}
\put(170.0,823.0){\rule[-0.200pt]{2.409pt}{0.400pt}}
\put(1429.0,823.0){\rule[-0.200pt]{2.409pt}{0.400pt}}
\put(170.0,823.0){\rule[-0.200pt]{2.409pt}{0.400pt}}
\put(1429.0,823.0){\rule[-0.200pt]{2.409pt}{0.400pt}}
\put(170.0,823.0){\rule[-0.200pt]{2.409pt}{0.400pt}}
\put(1429.0,823.0){\rule[-0.200pt]{2.409pt}{0.400pt}}
\put(170.0,823.0){\rule[-0.200pt]{2.409pt}{0.400pt}}
\put(1429.0,823.0){\rule[-0.200pt]{2.409pt}{0.400pt}}
\put(170.0,824.0){\rule[-0.200pt]{2.409pt}{0.400pt}}
\put(1429.0,824.0){\rule[-0.200pt]{2.409pt}{0.400pt}}
\put(170.0,824.0){\rule[-0.200pt]{2.409pt}{0.400pt}}
\put(1429.0,824.0){\rule[-0.200pt]{2.409pt}{0.400pt}}
\put(170.0,824.0){\rule[-0.200pt]{2.409pt}{0.400pt}}
\put(1429.0,824.0){\rule[-0.200pt]{2.409pt}{0.400pt}}
\put(170.0,824.0){\rule[-0.200pt]{2.409pt}{0.400pt}}
\put(1429.0,824.0){\rule[-0.200pt]{2.409pt}{0.400pt}}
\put(170.0,824.0){\rule[-0.200pt]{2.409pt}{0.400pt}}
\put(1429.0,824.0){\rule[-0.200pt]{2.409pt}{0.400pt}}
\put(170.0,824.0){\rule[-0.200pt]{2.409pt}{0.400pt}}
\put(1429.0,824.0){\rule[-0.200pt]{2.409pt}{0.400pt}}
\put(170.0,824.0){\rule[-0.200pt]{2.409pt}{0.400pt}}
\put(1429.0,824.0){\rule[-0.200pt]{2.409pt}{0.400pt}}
\put(170.0,824.0){\rule[-0.200pt]{2.409pt}{0.400pt}}
\put(1429.0,824.0){\rule[-0.200pt]{2.409pt}{0.400pt}}
\put(170.0,824.0){\rule[-0.200pt]{2.409pt}{0.400pt}}
\put(1429.0,824.0){\rule[-0.200pt]{2.409pt}{0.400pt}}
\put(170.0,824.0){\rule[-0.200pt]{2.409pt}{0.400pt}}
\put(1429.0,824.0){\rule[-0.200pt]{2.409pt}{0.400pt}}
\put(170.0,824.0){\rule[-0.200pt]{2.409pt}{0.400pt}}
\put(1429.0,824.0){\rule[-0.200pt]{2.409pt}{0.400pt}}
\put(170.0,825.0){\rule[-0.200pt]{2.409pt}{0.400pt}}
\put(1429.0,825.0){\rule[-0.200pt]{2.409pt}{0.400pt}}
\put(170.0,825.0){\rule[-0.200pt]{2.409pt}{0.400pt}}
\put(1429.0,825.0){\rule[-0.200pt]{2.409pt}{0.400pt}}
\put(170.0,825.0){\rule[-0.200pt]{2.409pt}{0.400pt}}
\put(1429.0,825.0){\rule[-0.200pt]{2.409pt}{0.400pt}}
\put(170.0,825.0){\rule[-0.200pt]{2.409pt}{0.400pt}}
\put(1429.0,825.0){\rule[-0.200pt]{2.409pt}{0.400pt}}
\put(170.0,825.0){\rule[-0.200pt]{2.409pt}{0.400pt}}
\put(1429.0,825.0){\rule[-0.200pt]{2.409pt}{0.400pt}}
\put(170.0,825.0){\rule[-0.200pt]{2.409pt}{0.400pt}}
\put(1429.0,825.0){\rule[-0.200pt]{2.409pt}{0.400pt}}
\put(170.0,825.0){\rule[-0.200pt]{2.409pt}{0.400pt}}
\put(1429.0,825.0){\rule[-0.200pt]{2.409pt}{0.400pt}}
\put(170.0,825.0){\rule[-0.200pt]{2.409pt}{0.400pt}}
\put(1429.0,825.0){\rule[-0.200pt]{2.409pt}{0.400pt}}
\put(170.0,825.0){\rule[-0.200pt]{2.409pt}{0.400pt}}
\put(1429.0,825.0){\rule[-0.200pt]{2.409pt}{0.400pt}}
\put(170.0,825.0){\rule[-0.200pt]{2.409pt}{0.400pt}}
\put(1429.0,825.0){\rule[-0.200pt]{2.409pt}{0.400pt}}
\put(170.0,825.0){\rule[-0.200pt]{2.409pt}{0.400pt}}
\put(1429.0,825.0){\rule[-0.200pt]{2.409pt}{0.400pt}}
\put(170.0,826.0){\rule[-0.200pt]{2.409pt}{0.400pt}}
\put(1429.0,826.0){\rule[-0.200pt]{2.409pt}{0.400pt}}
\put(170.0,826.0){\rule[-0.200pt]{2.409pt}{0.400pt}}
\put(1429.0,826.0){\rule[-0.200pt]{2.409pt}{0.400pt}}
\put(170.0,826.0){\rule[-0.200pt]{2.409pt}{0.400pt}}
\put(1429.0,826.0){\rule[-0.200pt]{2.409pt}{0.400pt}}
\put(170.0,826.0){\rule[-0.200pt]{2.409pt}{0.400pt}}
\put(1429.0,826.0){\rule[-0.200pt]{2.409pt}{0.400pt}}
\put(170.0,826.0){\rule[-0.200pt]{2.409pt}{0.400pt}}
\put(1429.0,826.0){\rule[-0.200pt]{2.409pt}{0.400pt}}
\put(170.0,826.0){\rule[-0.200pt]{2.409pt}{0.400pt}}
\put(1429.0,826.0){\rule[-0.200pt]{2.409pt}{0.400pt}}
\put(170.0,826.0){\rule[-0.200pt]{2.409pt}{0.400pt}}
\put(1429.0,826.0){\rule[-0.200pt]{2.409pt}{0.400pt}}
\put(170.0,826.0){\rule[-0.200pt]{2.409pt}{0.400pt}}
\put(1429.0,826.0){\rule[-0.200pt]{2.409pt}{0.400pt}}
\put(170.0,826.0){\rule[-0.200pt]{2.409pt}{0.400pt}}
\put(1429.0,826.0){\rule[-0.200pt]{2.409pt}{0.400pt}}
\put(170.0,826.0){\rule[-0.200pt]{2.409pt}{0.400pt}}
\put(1429.0,826.0){\rule[-0.200pt]{2.409pt}{0.400pt}}
\put(170.0,826.0){\rule[-0.200pt]{2.409pt}{0.400pt}}
\put(1429.0,826.0){\rule[-0.200pt]{2.409pt}{0.400pt}}
\put(170.0,827.0){\rule[-0.200pt]{2.409pt}{0.400pt}}
\put(1429.0,827.0){\rule[-0.200pt]{2.409pt}{0.400pt}}
\put(170.0,827.0){\rule[-0.200pt]{2.409pt}{0.400pt}}
\put(1429.0,827.0){\rule[-0.200pt]{2.409pt}{0.400pt}}
\put(170.0,827.0){\rule[-0.200pt]{2.409pt}{0.400pt}}
\put(1429.0,827.0){\rule[-0.200pt]{2.409pt}{0.400pt}}
\put(170.0,827.0){\rule[-0.200pt]{2.409pt}{0.400pt}}
\put(1429.0,827.0){\rule[-0.200pt]{2.409pt}{0.400pt}}
\put(170.0,827.0){\rule[-0.200pt]{2.409pt}{0.400pt}}
\put(1429.0,827.0){\rule[-0.200pt]{2.409pt}{0.400pt}}
\put(170.0,827.0){\rule[-0.200pt]{2.409pt}{0.400pt}}
\put(1429.0,827.0){\rule[-0.200pt]{2.409pt}{0.400pt}}
\put(170.0,827.0){\rule[-0.200pt]{2.409pt}{0.400pt}}
\put(1429.0,827.0){\rule[-0.200pt]{2.409pt}{0.400pt}}
\put(170.0,827.0){\rule[-0.200pt]{2.409pt}{0.400pt}}
\put(1429.0,827.0){\rule[-0.200pt]{2.409pt}{0.400pt}}
\put(170.0,827.0){\rule[-0.200pt]{2.409pt}{0.400pt}}
\put(1429.0,827.0){\rule[-0.200pt]{2.409pt}{0.400pt}}
\put(170.0,827.0){\rule[-0.200pt]{2.409pt}{0.400pt}}
\put(1429.0,827.0){\rule[-0.200pt]{2.409pt}{0.400pt}}
\put(170.0,827.0){\rule[-0.200pt]{2.409pt}{0.400pt}}
\put(1429.0,827.0){\rule[-0.200pt]{2.409pt}{0.400pt}}
\put(170.0,828.0){\rule[-0.200pt]{2.409pt}{0.400pt}}
\put(1429.0,828.0){\rule[-0.200pt]{2.409pt}{0.400pt}}
\put(170.0,828.0){\rule[-0.200pt]{2.409pt}{0.400pt}}
\put(1429.0,828.0){\rule[-0.200pt]{2.409pt}{0.400pt}}
\put(170.0,828.0){\rule[-0.200pt]{2.409pt}{0.400pt}}
\put(1429.0,828.0){\rule[-0.200pt]{2.409pt}{0.400pt}}
\put(170.0,828.0){\rule[-0.200pt]{2.409pt}{0.400pt}}
\put(1429.0,828.0){\rule[-0.200pt]{2.409pt}{0.400pt}}
\put(170.0,828.0){\rule[-0.200pt]{2.409pt}{0.400pt}}
\put(1429.0,828.0){\rule[-0.200pt]{2.409pt}{0.400pt}}
\put(170.0,828.0){\rule[-0.200pt]{2.409pt}{0.400pt}}
\put(1429.0,828.0){\rule[-0.200pt]{2.409pt}{0.400pt}}
\put(170.0,828.0){\rule[-0.200pt]{2.409pt}{0.400pt}}
\put(1429.0,828.0){\rule[-0.200pt]{2.409pt}{0.400pt}}
\put(170.0,828.0){\rule[-0.200pt]{2.409pt}{0.400pt}}
\put(1429.0,828.0){\rule[-0.200pt]{2.409pt}{0.400pt}}
\put(170.0,828.0){\rule[-0.200pt]{2.409pt}{0.400pt}}
\put(1429.0,828.0){\rule[-0.200pt]{2.409pt}{0.400pt}}
\put(170.0,828.0){\rule[-0.200pt]{2.409pt}{0.400pt}}
\put(1429.0,828.0){\rule[-0.200pt]{2.409pt}{0.400pt}}
\put(170.0,828.0){\rule[-0.200pt]{2.409pt}{0.400pt}}
\put(1429.0,828.0){\rule[-0.200pt]{2.409pt}{0.400pt}}
\put(170.0,828.0){\rule[-0.200pt]{2.409pt}{0.400pt}}
\put(1429.0,828.0){\rule[-0.200pt]{2.409pt}{0.400pt}}
\put(170.0,829.0){\rule[-0.200pt]{2.409pt}{0.400pt}}
\put(1429.0,829.0){\rule[-0.200pt]{2.409pt}{0.400pt}}
\put(170.0,829.0){\rule[-0.200pt]{2.409pt}{0.400pt}}
\put(1429.0,829.0){\rule[-0.200pt]{2.409pt}{0.400pt}}
\put(170.0,829.0){\rule[-0.200pt]{2.409pt}{0.400pt}}
\put(1429.0,829.0){\rule[-0.200pt]{2.409pt}{0.400pt}}
\put(170.0,829.0){\rule[-0.200pt]{2.409pt}{0.400pt}}
\put(1429.0,829.0){\rule[-0.200pt]{2.409pt}{0.400pt}}
\put(170.0,829.0){\rule[-0.200pt]{2.409pt}{0.400pt}}
\put(1429.0,829.0){\rule[-0.200pt]{2.409pt}{0.400pt}}
\put(170.0,829.0){\rule[-0.200pt]{2.409pt}{0.400pt}}
\put(1429.0,829.0){\rule[-0.200pt]{2.409pt}{0.400pt}}
\put(170.0,829.0){\rule[-0.200pt]{2.409pt}{0.400pt}}
\put(1429.0,829.0){\rule[-0.200pt]{2.409pt}{0.400pt}}
\put(170.0,829.0){\rule[-0.200pt]{2.409pt}{0.400pt}}
\put(1429.0,829.0){\rule[-0.200pt]{2.409pt}{0.400pt}}
\put(170.0,829.0){\rule[-0.200pt]{2.409pt}{0.400pt}}
\put(1429.0,829.0){\rule[-0.200pt]{2.409pt}{0.400pt}}
\put(170.0,829.0){\rule[-0.200pt]{2.409pt}{0.400pt}}
\put(1429.0,829.0){\rule[-0.200pt]{2.409pt}{0.400pt}}
\put(170.0,829.0){\rule[-0.200pt]{2.409pt}{0.400pt}}
\put(1429.0,829.0){\rule[-0.200pt]{2.409pt}{0.400pt}}
\put(170.0,829.0){\rule[-0.200pt]{2.409pt}{0.400pt}}
\put(1429.0,829.0){\rule[-0.200pt]{2.409pt}{0.400pt}}
\put(170.0,830.0){\rule[-0.200pt]{2.409pt}{0.400pt}}
\put(1429.0,830.0){\rule[-0.200pt]{2.409pt}{0.400pt}}
\put(170.0,830.0){\rule[-0.200pt]{2.409pt}{0.400pt}}
\put(1429.0,830.0){\rule[-0.200pt]{2.409pt}{0.400pt}}
\put(170.0,830.0){\rule[-0.200pt]{2.409pt}{0.400pt}}
\put(1429.0,830.0){\rule[-0.200pt]{2.409pt}{0.400pt}}
\put(170.0,830.0){\rule[-0.200pt]{2.409pt}{0.400pt}}
\put(1429.0,830.0){\rule[-0.200pt]{2.409pt}{0.400pt}}
\put(170.0,830.0){\rule[-0.200pt]{2.409pt}{0.400pt}}
\put(1429.0,830.0){\rule[-0.200pt]{2.409pt}{0.400pt}}
\put(170.0,830.0){\rule[-0.200pt]{2.409pt}{0.400pt}}
\put(1429.0,830.0){\rule[-0.200pt]{2.409pt}{0.400pt}}
\put(170.0,830.0){\rule[-0.200pt]{2.409pt}{0.400pt}}
\put(1429.0,830.0){\rule[-0.200pt]{2.409pt}{0.400pt}}
\put(170.0,830.0){\rule[-0.200pt]{2.409pt}{0.400pt}}
\put(1429.0,830.0){\rule[-0.200pt]{2.409pt}{0.400pt}}
\put(170.0,830.0){\rule[-0.200pt]{2.409pt}{0.400pt}}
\put(1429.0,830.0){\rule[-0.200pt]{2.409pt}{0.400pt}}
\put(170.0,830.0){\rule[-0.200pt]{2.409pt}{0.400pt}}
\put(1429.0,830.0){\rule[-0.200pt]{2.409pt}{0.400pt}}
\put(170.0,830.0){\rule[-0.200pt]{2.409pt}{0.400pt}}
\put(1429.0,830.0){\rule[-0.200pt]{2.409pt}{0.400pt}}
\put(170.0,830.0){\rule[-0.200pt]{2.409pt}{0.400pt}}
\put(1429.0,830.0){\rule[-0.200pt]{2.409pt}{0.400pt}}
\put(170.0,831.0){\rule[-0.200pt]{2.409pt}{0.400pt}}
\put(1429.0,831.0){\rule[-0.200pt]{2.409pt}{0.400pt}}
\put(170.0,831.0){\rule[-0.200pt]{2.409pt}{0.400pt}}
\put(1429.0,831.0){\rule[-0.200pt]{2.409pt}{0.400pt}}
\put(170.0,831.0){\rule[-0.200pt]{2.409pt}{0.400pt}}
\put(1429.0,831.0){\rule[-0.200pt]{2.409pt}{0.400pt}}
\put(170.0,831.0){\rule[-0.200pt]{2.409pt}{0.400pt}}
\put(1429.0,831.0){\rule[-0.200pt]{2.409pt}{0.400pt}}
\put(170.0,831.0){\rule[-0.200pt]{2.409pt}{0.400pt}}
\put(1429.0,831.0){\rule[-0.200pt]{2.409pt}{0.400pt}}
\put(170.0,831.0){\rule[-0.200pt]{2.409pt}{0.400pt}}
\put(1429.0,831.0){\rule[-0.200pt]{2.409pt}{0.400pt}}
\put(170.0,831.0){\rule[-0.200pt]{2.409pt}{0.400pt}}
\put(1429.0,831.0){\rule[-0.200pt]{2.409pt}{0.400pt}}
\put(170.0,831.0){\rule[-0.200pt]{2.409pt}{0.400pt}}
\put(1429.0,831.0){\rule[-0.200pt]{2.409pt}{0.400pt}}
\put(170.0,831.0){\rule[-0.200pt]{2.409pt}{0.400pt}}
\put(1429.0,831.0){\rule[-0.200pt]{2.409pt}{0.400pt}}
\put(170.0,831.0){\rule[-0.200pt]{2.409pt}{0.400pt}}
\put(1429.0,831.0){\rule[-0.200pt]{2.409pt}{0.400pt}}
\put(170.0,831.0){\rule[-0.200pt]{2.409pt}{0.400pt}}
\put(1429.0,831.0){\rule[-0.200pt]{2.409pt}{0.400pt}}
\put(170.0,831.0){\rule[-0.200pt]{2.409pt}{0.400pt}}
\put(1429.0,831.0){\rule[-0.200pt]{2.409pt}{0.400pt}}
\put(170.0,831.0){\rule[-0.200pt]{2.409pt}{0.400pt}}
\put(1429.0,831.0){\rule[-0.200pt]{2.409pt}{0.400pt}}
\put(170.0,832.0){\rule[-0.200pt]{2.409pt}{0.400pt}}
\put(1429.0,832.0){\rule[-0.200pt]{2.409pt}{0.400pt}}
\put(170.0,832.0){\rule[-0.200pt]{2.409pt}{0.400pt}}
\put(1429.0,832.0){\rule[-0.200pt]{2.409pt}{0.400pt}}
\put(170.0,832.0){\rule[-0.200pt]{2.409pt}{0.400pt}}
\put(1429.0,832.0){\rule[-0.200pt]{2.409pt}{0.400pt}}
\put(170.0,832.0){\rule[-0.200pt]{2.409pt}{0.400pt}}
\put(1429.0,832.0){\rule[-0.200pt]{2.409pt}{0.400pt}}
\put(170.0,832.0){\rule[-0.200pt]{2.409pt}{0.400pt}}
\put(1429.0,832.0){\rule[-0.200pt]{2.409pt}{0.400pt}}
\put(170.0,832.0){\rule[-0.200pt]{2.409pt}{0.400pt}}
\put(1429.0,832.0){\rule[-0.200pt]{2.409pt}{0.400pt}}
\put(170.0,832.0){\rule[-0.200pt]{2.409pt}{0.400pt}}
\put(1429.0,832.0){\rule[-0.200pt]{2.409pt}{0.400pt}}
\put(170.0,832.0){\rule[-0.200pt]{2.409pt}{0.400pt}}
\put(1429.0,832.0){\rule[-0.200pt]{2.409pt}{0.400pt}}
\put(170.0,832.0){\rule[-0.200pt]{2.409pt}{0.400pt}}
\put(1429.0,832.0){\rule[-0.200pt]{2.409pt}{0.400pt}}
\put(170.0,832.0){\rule[-0.200pt]{2.409pt}{0.400pt}}
\put(1429.0,832.0){\rule[-0.200pt]{2.409pt}{0.400pt}}
\put(170.0,832.0){\rule[-0.200pt]{2.409pt}{0.400pt}}
\put(1429.0,832.0){\rule[-0.200pt]{2.409pt}{0.400pt}}
\put(170.0,832.0){\rule[-0.200pt]{2.409pt}{0.400pt}}
\put(1429.0,832.0){\rule[-0.200pt]{2.409pt}{0.400pt}}
\put(170.0,832.0){\rule[-0.200pt]{2.409pt}{0.400pt}}
\put(1429.0,832.0){\rule[-0.200pt]{2.409pt}{0.400pt}}
\put(170.0,833.0){\rule[-0.200pt]{2.409pt}{0.400pt}}
\put(1429.0,833.0){\rule[-0.200pt]{2.409pt}{0.400pt}}
\put(170.0,833.0){\rule[-0.200pt]{2.409pt}{0.400pt}}
\put(1429.0,833.0){\rule[-0.200pt]{2.409pt}{0.400pt}}
\put(170.0,833.0){\rule[-0.200pt]{2.409pt}{0.400pt}}
\put(1429.0,833.0){\rule[-0.200pt]{2.409pt}{0.400pt}}
\put(170.0,833.0){\rule[-0.200pt]{2.409pt}{0.400pt}}
\put(1429.0,833.0){\rule[-0.200pt]{2.409pt}{0.400pt}}
\put(170.0,833.0){\rule[-0.200pt]{2.409pt}{0.400pt}}
\put(1429.0,833.0){\rule[-0.200pt]{2.409pt}{0.400pt}}
\put(170.0,833.0){\rule[-0.200pt]{2.409pt}{0.400pt}}
\put(1429.0,833.0){\rule[-0.200pt]{2.409pt}{0.400pt}}
\put(170.0,833.0){\rule[-0.200pt]{2.409pt}{0.400pt}}
\put(1429.0,833.0){\rule[-0.200pt]{2.409pt}{0.400pt}}
\put(170.0,833.0){\rule[-0.200pt]{2.409pt}{0.400pt}}
\put(1429.0,833.0){\rule[-0.200pt]{2.409pt}{0.400pt}}
\put(170.0,833.0){\rule[-0.200pt]{2.409pt}{0.400pt}}
\put(1429.0,833.0){\rule[-0.200pt]{2.409pt}{0.400pt}}
\put(170.0,833.0){\rule[-0.200pt]{2.409pt}{0.400pt}}
\put(1429.0,833.0){\rule[-0.200pt]{2.409pt}{0.400pt}}
\put(170.0,833.0){\rule[-0.200pt]{2.409pt}{0.400pt}}
\put(1429.0,833.0){\rule[-0.200pt]{2.409pt}{0.400pt}}
\put(170.0,833.0){\rule[-0.200pt]{2.409pt}{0.400pt}}
\put(1429.0,833.0){\rule[-0.200pt]{2.409pt}{0.400pt}}
\put(170.0,833.0){\rule[-0.200pt]{2.409pt}{0.400pt}}
\put(1429.0,833.0){\rule[-0.200pt]{2.409pt}{0.400pt}}
\put(170.0,834.0){\rule[-0.200pt]{2.409pt}{0.400pt}}
\put(1429.0,834.0){\rule[-0.200pt]{2.409pt}{0.400pt}}
\put(170.0,834.0){\rule[-0.200pt]{2.409pt}{0.400pt}}
\put(1429.0,834.0){\rule[-0.200pt]{2.409pt}{0.400pt}}
\put(170.0,834.0){\rule[-0.200pt]{2.409pt}{0.400pt}}
\put(1429.0,834.0){\rule[-0.200pt]{2.409pt}{0.400pt}}
\put(170.0,834.0){\rule[-0.200pt]{2.409pt}{0.400pt}}
\put(1429.0,834.0){\rule[-0.200pt]{2.409pt}{0.400pt}}
\put(170.0,834.0){\rule[-0.200pt]{2.409pt}{0.400pt}}
\put(1429.0,834.0){\rule[-0.200pt]{2.409pt}{0.400pt}}
\put(170.0,834.0){\rule[-0.200pt]{2.409pt}{0.400pt}}
\put(1429.0,834.0){\rule[-0.200pt]{2.409pt}{0.400pt}}
\put(170.0,834.0){\rule[-0.200pt]{2.409pt}{0.400pt}}
\put(1429.0,834.0){\rule[-0.200pt]{2.409pt}{0.400pt}}
\put(170.0,834.0){\rule[-0.200pt]{2.409pt}{0.400pt}}
\put(1429.0,834.0){\rule[-0.200pt]{2.409pt}{0.400pt}}
\put(170.0,834.0){\rule[-0.200pt]{2.409pt}{0.400pt}}
\put(1429.0,834.0){\rule[-0.200pt]{2.409pt}{0.400pt}}
\put(170.0,834.0){\rule[-0.200pt]{2.409pt}{0.400pt}}
\put(1429.0,834.0){\rule[-0.200pt]{2.409pt}{0.400pt}}
\put(170.0,834.0){\rule[-0.200pt]{2.409pt}{0.400pt}}
\put(1429.0,834.0){\rule[-0.200pt]{2.409pt}{0.400pt}}
\put(170.0,834.0){\rule[-0.200pt]{2.409pt}{0.400pt}}
\put(1429.0,834.0){\rule[-0.200pt]{2.409pt}{0.400pt}}
\put(170.0,834.0){\rule[-0.200pt]{2.409pt}{0.400pt}}
\put(1429.0,834.0){\rule[-0.200pt]{2.409pt}{0.400pt}}
\put(170.0,834.0){\rule[-0.200pt]{2.409pt}{0.400pt}}
\put(1429.0,834.0){\rule[-0.200pt]{2.409pt}{0.400pt}}
\put(170.0,835.0){\rule[-0.200pt]{2.409pt}{0.400pt}}
\put(1429.0,835.0){\rule[-0.200pt]{2.409pt}{0.400pt}}
\put(170.0,835.0){\rule[-0.200pt]{2.409pt}{0.400pt}}
\put(1429.0,835.0){\rule[-0.200pt]{2.409pt}{0.400pt}}
\put(170.0,835.0){\rule[-0.200pt]{2.409pt}{0.400pt}}
\put(1429.0,835.0){\rule[-0.200pt]{2.409pt}{0.400pt}}
\put(170.0,835.0){\rule[-0.200pt]{2.409pt}{0.400pt}}
\put(1429.0,835.0){\rule[-0.200pt]{2.409pt}{0.400pt}}
\put(170.0,835.0){\rule[-0.200pt]{2.409pt}{0.400pt}}
\put(1429.0,835.0){\rule[-0.200pt]{2.409pt}{0.400pt}}
\put(170.0,835.0){\rule[-0.200pt]{2.409pt}{0.400pt}}
\put(1429.0,835.0){\rule[-0.200pt]{2.409pt}{0.400pt}}
\put(170.0,835.0){\rule[-0.200pt]{2.409pt}{0.400pt}}
\put(1429.0,835.0){\rule[-0.200pt]{2.409pt}{0.400pt}}
\put(170.0,835.0){\rule[-0.200pt]{2.409pt}{0.400pt}}
\put(1429.0,835.0){\rule[-0.200pt]{2.409pt}{0.400pt}}
\put(170.0,835.0){\rule[-0.200pt]{2.409pt}{0.400pt}}
\put(1429.0,835.0){\rule[-0.200pt]{2.409pt}{0.400pt}}
\put(170.0,835.0){\rule[-0.200pt]{2.409pt}{0.400pt}}
\put(1429.0,835.0){\rule[-0.200pt]{2.409pt}{0.400pt}}
\put(170.0,835.0){\rule[-0.200pt]{2.409pt}{0.400pt}}
\put(1429.0,835.0){\rule[-0.200pt]{2.409pt}{0.400pt}}
\put(170.0,835.0){\rule[-0.200pt]{2.409pt}{0.400pt}}
\put(1429.0,835.0){\rule[-0.200pt]{2.409pt}{0.400pt}}
\put(170.0,835.0){\rule[-0.200pt]{2.409pt}{0.400pt}}
\put(1429.0,835.0){\rule[-0.200pt]{2.409pt}{0.400pt}}
\put(170.0,835.0){\rule[-0.200pt]{2.409pt}{0.400pt}}
\put(1429.0,835.0){\rule[-0.200pt]{2.409pt}{0.400pt}}
\put(170.0,836.0){\rule[-0.200pt]{2.409pt}{0.400pt}}
\put(1429.0,836.0){\rule[-0.200pt]{2.409pt}{0.400pt}}
\put(170.0,836.0){\rule[-0.200pt]{2.409pt}{0.400pt}}
\put(1429.0,836.0){\rule[-0.200pt]{2.409pt}{0.400pt}}
\put(170.0,836.0){\rule[-0.200pt]{2.409pt}{0.400pt}}
\put(1429.0,836.0){\rule[-0.200pt]{2.409pt}{0.400pt}}
\put(170.0,836.0){\rule[-0.200pt]{2.409pt}{0.400pt}}
\put(1429.0,836.0){\rule[-0.200pt]{2.409pt}{0.400pt}}
\put(170.0,836.0){\rule[-0.200pt]{2.409pt}{0.400pt}}
\put(1429.0,836.0){\rule[-0.200pt]{2.409pt}{0.400pt}}
\put(170.0,836.0){\rule[-0.200pt]{2.409pt}{0.400pt}}
\put(1429.0,836.0){\rule[-0.200pt]{2.409pt}{0.400pt}}
\put(170.0,836.0){\rule[-0.200pt]{2.409pt}{0.400pt}}
\put(1429.0,836.0){\rule[-0.200pt]{2.409pt}{0.400pt}}
\put(170.0,836.0){\rule[-0.200pt]{2.409pt}{0.400pt}}
\put(1429.0,836.0){\rule[-0.200pt]{2.409pt}{0.400pt}}
\put(170.0,836.0){\rule[-0.200pt]{2.409pt}{0.400pt}}
\put(1429.0,836.0){\rule[-0.200pt]{2.409pt}{0.400pt}}
\put(170.0,836.0){\rule[-0.200pt]{2.409pt}{0.400pt}}
\put(1429.0,836.0){\rule[-0.200pt]{2.409pt}{0.400pt}}
\put(170.0,836.0){\rule[-0.200pt]{2.409pt}{0.400pt}}
\put(1429.0,836.0){\rule[-0.200pt]{2.409pt}{0.400pt}}
\put(170.0,836.0){\rule[-0.200pt]{2.409pt}{0.400pt}}
\put(1429.0,836.0){\rule[-0.200pt]{2.409pt}{0.400pt}}
\put(170.0,836.0){\rule[-0.200pt]{2.409pt}{0.400pt}}
\put(1429.0,836.0){\rule[-0.200pt]{2.409pt}{0.400pt}}
\put(170.0,836.0){\rule[-0.200pt]{2.409pt}{0.400pt}}
\put(1429.0,836.0){\rule[-0.200pt]{2.409pt}{0.400pt}}
\put(170.0,837.0){\rule[-0.200pt]{2.409pt}{0.400pt}}
\put(1429.0,837.0){\rule[-0.200pt]{2.409pt}{0.400pt}}
\put(170.0,837.0){\rule[-0.200pt]{2.409pt}{0.400pt}}
\put(1429.0,837.0){\rule[-0.200pt]{2.409pt}{0.400pt}}
\put(170.0,837.0){\rule[-0.200pt]{2.409pt}{0.400pt}}
\put(1429.0,837.0){\rule[-0.200pt]{2.409pt}{0.400pt}}
\put(170.0,837.0){\rule[-0.200pt]{2.409pt}{0.400pt}}
\put(1429.0,837.0){\rule[-0.200pt]{2.409pt}{0.400pt}}
\put(170.0,837.0){\rule[-0.200pt]{2.409pt}{0.400pt}}
\put(1429.0,837.0){\rule[-0.200pt]{2.409pt}{0.400pt}}
\put(170.0,837.0){\rule[-0.200pt]{2.409pt}{0.400pt}}
\put(1429.0,837.0){\rule[-0.200pt]{2.409pt}{0.400pt}}
\put(170.0,837.0){\rule[-0.200pt]{2.409pt}{0.400pt}}
\put(1429.0,837.0){\rule[-0.200pt]{2.409pt}{0.400pt}}
\put(170.0,837.0){\rule[-0.200pt]{2.409pt}{0.400pt}}
\put(1429.0,837.0){\rule[-0.200pt]{2.409pt}{0.400pt}}
\put(170.0,837.0){\rule[-0.200pt]{2.409pt}{0.400pt}}
\put(1429.0,837.0){\rule[-0.200pt]{2.409pt}{0.400pt}}
\put(170.0,837.0){\rule[-0.200pt]{2.409pt}{0.400pt}}
\put(1429.0,837.0){\rule[-0.200pt]{2.409pt}{0.400pt}}
\put(170.0,837.0){\rule[-0.200pt]{2.409pt}{0.400pt}}
\put(1429.0,837.0){\rule[-0.200pt]{2.409pt}{0.400pt}}
\put(170.0,837.0){\rule[-0.200pt]{2.409pt}{0.400pt}}
\put(1429.0,837.0){\rule[-0.200pt]{2.409pt}{0.400pt}}
\put(170.0,837.0){\rule[-0.200pt]{2.409pt}{0.400pt}}
\put(1429.0,837.0){\rule[-0.200pt]{2.409pt}{0.400pt}}
\put(170.0,837.0){\rule[-0.200pt]{2.409pt}{0.400pt}}
\put(1429.0,837.0){\rule[-0.200pt]{2.409pt}{0.400pt}}
\put(170.0,837.0){\rule[-0.200pt]{2.409pt}{0.400pt}}
\put(1429.0,837.0){\rule[-0.200pt]{2.409pt}{0.400pt}}
\put(170.0,838.0){\rule[-0.200pt]{2.409pt}{0.400pt}}
\put(1429.0,838.0){\rule[-0.200pt]{2.409pt}{0.400pt}}
\put(170.0,838.0){\rule[-0.200pt]{2.409pt}{0.400pt}}
\put(1429.0,838.0){\rule[-0.200pt]{2.409pt}{0.400pt}}
\put(170.0,838.0){\rule[-0.200pt]{2.409pt}{0.400pt}}
\put(1429.0,838.0){\rule[-0.200pt]{2.409pt}{0.400pt}}
\put(170.0,838.0){\rule[-0.200pt]{2.409pt}{0.400pt}}
\put(1429.0,838.0){\rule[-0.200pt]{2.409pt}{0.400pt}}
\put(170.0,838.0){\rule[-0.200pt]{2.409pt}{0.400pt}}
\put(1429.0,838.0){\rule[-0.200pt]{2.409pt}{0.400pt}}
\put(170.0,838.0){\rule[-0.200pt]{2.409pt}{0.400pt}}
\put(1429.0,838.0){\rule[-0.200pt]{2.409pt}{0.400pt}}
\put(170.0,838.0){\rule[-0.200pt]{2.409pt}{0.400pt}}
\put(1429.0,838.0){\rule[-0.200pt]{2.409pt}{0.400pt}}
\put(170.0,838.0){\rule[-0.200pt]{2.409pt}{0.400pt}}
\put(1429.0,838.0){\rule[-0.200pt]{2.409pt}{0.400pt}}
\put(170.0,838.0){\rule[-0.200pt]{2.409pt}{0.400pt}}
\put(1429.0,838.0){\rule[-0.200pt]{2.409pt}{0.400pt}}
\put(170.0,838.0){\rule[-0.200pt]{2.409pt}{0.400pt}}
\put(1429.0,838.0){\rule[-0.200pt]{2.409pt}{0.400pt}}
\put(170.0,838.0){\rule[-0.200pt]{2.409pt}{0.400pt}}
\put(1429.0,838.0){\rule[-0.200pt]{2.409pt}{0.400pt}}
\put(170.0,838.0){\rule[-0.200pt]{2.409pt}{0.400pt}}
\put(1429.0,838.0){\rule[-0.200pt]{2.409pt}{0.400pt}}
\put(170.0,838.0){\rule[-0.200pt]{2.409pt}{0.400pt}}
\put(1429.0,838.0){\rule[-0.200pt]{2.409pt}{0.400pt}}
\put(170.0,838.0){\rule[-0.200pt]{2.409pt}{0.400pt}}
\put(1429.0,838.0){\rule[-0.200pt]{2.409pt}{0.400pt}}
\put(170.0,838.0){\rule[-0.200pt]{2.409pt}{0.400pt}}
\put(1429.0,838.0){\rule[-0.200pt]{2.409pt}{0.400pt}}
\put(170.0,839.0){\rule[-0.200pt]{2.409pt}{0.400pt}}
\put(1429.0,839.0){\rule[-0.200pt]{2.409pt}{0.400pt}}
\put(170.0,839.0){\rule[-0.200pt]{2.409pt}{0.400pt}}
\put(1429.0,839.0){\rule[-0.200pt]{2.409pt}{0.400pt}}
\put(170.0,839.0){\rule[-0.200pt]{2.409pt}{0.400pt}}
\put(1429.0,839.0){\rule[-0.200pt]{2.409pt}{0.400pt}}
\put(170.0,839.0){\rule[-0.200pt]{2.409pt}{0.400pt}}
\put(1429.0,839.0){\rule[-0.200pt]{2.409pt}{0.400pt}}
\put(170.0,839.0){\rule[-0.200pt]{2.409pt}{0.400pt}}
\put(1429.0,839.0){\rule[-0.200pt]{2.409pt}{0.400pt}}
\put(170.0,839.0){\rule[-0.200pt]{2.409pt}{0.400pt}}
\put(1429.0,839.0){\rule[-0.200pt]{2.409pt}{0.400pt}}
\put(170.0,839.0){\rule[-0.200pt]{2.409pt}{0.400pt}}
\put(1429.0,839.0){\rule[-0.200pt]{2.409pt}{0.400pt}}
\put(170.0,839.0){\rule[-0.200pt]{2.409pt}{0.400pt}}
\put(1429.0,839.0){\rule[-0.200pt]{2.409pt}{0.400pt}}
\put(170.0,839.0){\rule[-0.200pt]{2.409pt}{0.400pt}}
\put(1429.0,839.0){\rule[-0.200pt]{2.409pt}{0.400pt}}
\put(170.0,839.0){\rule[-0.200pt]{2.409pt}{0.400pt}}
\put(1429.0,839.0){\rule[-0.200pt]{2.409pt}{0.400pt}}
\put(170.0,839.0){\rule[-0.200pt]{2.409pt}{0.400pt}}
\put(1429.0,839.0){\rule[-0.200pt]{2.409pt}{0.400pt}}
\put(170.0,839.0){\rule[-0.200pt]{2.409pt}{0.400pt}}
\put(1429.0,839.0){\rule[-0.200pt]{2.409pt}{0.400pt}}
\put(170.0,839.0){\rule[-0.200pt]{2.409pt}{0.400pt}}
\put(1429.0,839.0){\rule[-0.200pt]{2.409pt}{0.400pt}}
\put(170.0,839.0){\rule[-0.200pt]{2.409pt}{0.400pt}}
\put(1429.0,839.0){\rule[-0.200pt]{2.409pt}{0.400pt}}
\put(170.0,839.0){\rule[-0.200pt]{2.409pt}{0.400pt}}
\put(1429.0,839.0){\rule[-0.200pt]{2.409pt}{0.400pt}}
\put(170.0,839.0){\rule[-0.200pt]{2.409pt}{0.400pt}}
\put(1429.0,839.0){\rule[-0.200pt]{2.409pt}{0.400pt}}
\put(170.0,840.0){\rule[-0.200pt]{2.409pt}{0.400pt}}
\put(1429.0,840.0){\rule[-0.200pt]{2.409pt}{0.400pt}}
\put(170.0,840.0){\rule[-0.200pt]{2.409pt}{0.400pt}}
\put(1429.0,840.0){\rule[-0.200pt]{2.409pt}{0.400pt}}
\put(170.0,840.0){\rule[-0.200pt]{2.409pt}{0.400pt}}
\put(1429.0,840.0){\rule[-0.200pt]{2.409pt}{0.400pt}}
\put(170.0,840.0){\rule[-0.200pt]{2.409pt}{0.400pt}}
\put(1429.0,840.0){\rule[-0.200pt]{2.409pt}{0.400pt}}
\put(170.0,840.0){\rule[-0.200pt]{2.409pt}{0.400pt}}
\put(1429.0,840.0){\rule[-0.200pt]{2.409pt}{0.400pt}}
\put(170.0,840.0){\rule[-0.200pt]{2.409pt}{0.400pt}}
\put(1429.0,840.0){\rule[-0.200pt]{2.409pt}{0.400pt}}
\put(170.0,840.0){\rule[-0.200pt]{2.409pt}{0.400pt}}
\put(1429.0,840.0){\rule[-0.200pt]{2.409pt}{0.400pt}}
\put(170.0,840.0){\rule[-0.200pt]{2.409pt}{0.400pt}}
\put(1429.0,840.0){\rule[-0.200pt]{2.409pt}{0.400pt}}
\put(170.0,840.0){\rule[-0.200pt]{2.409pt}{0.400pt}}
\put(1429.0,840.0){\rule[-0.200pt]{2.409pt}{0.400pt}}
\put(170.0,840.0){\rule[-0.200pt]{2.409pt}{0.400pt}}
\put(1429.0,840.0){\rule[-0.200pt]{2.409pt}{0.400pt}}
\put(170.0,840.0){\rule[-0.200pt]{2.409pt}{0.400pt}}
\put(1429.0,840.0){\rule[-0.200pt]{2.409pt}{0.400pt}}
\put(170.0,840.0){\rule[-0.200pt]{2.409pt}{0.400pt}}
\put(1429.0,840.0){\rule[-0.200pt]{2.409pt}{0.400pt}}
\put(170.0,840.0){\rule[-0.200pt]{2.409pt}{0.400pt}}
\put(1429.0,840.0){\rule[-0.200pt]{2.409pt}{0.400pt}}
\put(170.0,840.0){\rule[-0.200pt]{2.409pt}{0.400pt}}
\put(1429.0,840.0){\rule[-0.200pt]{2.409pt}{0.400pt}}
\put(170.0,840.0){\rule[-0.200pt]{2.409pt}{0.400pt}}
\put(1429.0,840.0){\rule[-0.200pt]{2.409pt}{0.400pt}}
\put(170.0,840.0){\rule[-0.200pt]{2.409pt}{0.400pt}}
\put(1429.0,840.0){\rule[-0.200pt]{2.409pt}{0.400pt}}
\put(170.0,841.0){\rule[-0.200pt]{2.409pt}{0.400pt}}
\put(1429.0,841.0){\rule[-0.200pt]{2.409pt}{0.400pt}}
\put(170.0,841.0){\rule[-0.200pt]{2.409pt}{0.400pt}}
\put(1429.0,841.0){\rule[-0.200pt]{2.409pt}{0.400pt}}
\put(170.0,841.0){\rule[-0.200pt]{2.409pt}{0.400pt}}
\put(1429.0,841.0){\rule[-0.200pt]{2.409pt}{0.400pt}}
\put(170.0,841.0){\rule[-0.200pt]{2.409pt}{0.400pt}}
\put(1429.0,841.0){\rule[-0.200pt]{2.409pt}{0.400pt}}
\put(170.0,841.0){\rule[-0.200pt]{2.409pt}{0.400pt}}
\put(1429.0,841.0){\rule[-0.200pt]{2.409pt}{0.400pt}}
\put(170.0,841.0){\rule[-0.200pt]{2.409pt}{0.400pt}}
\put(1429.0,841.0){\rule[-0.200pt]{2.409pt}{0.400pt}}
\put(170.0,841.0){\rule[-0.200pt]{2.409pt}{0.400pt}}
\put(1429.0,841.0){\rule[-0.200pt]{2.409pt}{0.400pt}}
\put(170.0,841.0){\rule[-0.200pt]{2.409pt}{0.400pt}}
\put(1429.0,841.0){\rule[-0.200pt]{2.409pt}{0.400pt}}
\put(170.0,841.0){\rule[-0.200pt]{2.409pt}{0.400pt}}
\put(1429.0,841.0){\rule[-0.200pt]{2.409pt}{0.400pt}}
\put(170.0,841.0){\rule[-0.200pt]{2.409pt}{0.400pt}}
\put(1429.0,841.0){\rule[-0.200pt]{2.409pt}{0.400pt}}
\put(170.0,841.0){\rule[-0.200pt]{2.409pt}{0.400pt}}
\put(1429.0,841.0){\rule[-0.200pt]{2.409pt}{0.400pt}}
\put(170.0,841.0){\rule[-0.200pt]{2.409pt}{0.400pt}}
\put(1429.0,841.0){\rule[-0.200pt]{2.409pt}{0.400pt}}
\put(170.0,841.0){\rule[-0.200pt]{2.409pt}{0.400pt}}
\put(1429.0,841.0){\rule[-0.200pt]{2.409pt}{0.400pt}}
\put(170.0,841.0){\rule[-0.200pt]{2.409pt}{0.400pt}}
\put(1429.0,841.0){\rule[-0.200pt]{2.409pt}{0.400pt}}
\put(170.0,841.0){\rule[-0.200pt]{2.409pt}{0.400pt}}
\put(1429.0,841.0){\rule[-0.200pt]{2.409pt}{0.400pt}}
\put(170.0,841.0){\rule[-0.200pt]{2.409pt}{0.400pt}}
\put(1429.0,841.0){\rule[-0.200pt]{2.409pt}{0.400pt}}
\put(170.0,841.0){\rule[-0.200pt]{2.409pt}{0.400pt}}
\put(1429.0,841.0){\rule[-0.200pt]{2.409pt}{0.400pt}}
\put(170.0,842.0){\rule[-0.200pt]{2.409pt}{0.400pt}}
\put(1429.0,842.0){\rule[-0.200pt]{2.409pt}{0.400pt}}
\put(170.0,842.0){\rule[-0.200pt]{2.409pt}{0.400pt}}
\put(1429.0,842.0){\rule[-0.200pt]{2.409pt}{0.400pt}}
\put(170.0,842.0){\rule[-0.200pt]{2.409pt}{0.400pt}}
\put(1429.0,842.0){\rule[-0.200pt]{2.409pt}{0.400pt}}
\put(170.0,842.0){\rule[-0.200pt]{2.409pt}{0.400pt}}
\put(1429.0,842.0){\rule[-0.200pt]{2.409pt}{0.400pt}}
\put(170.0,842.0){\rule[-0.200pt]{2.409pt}{0.400pt}}
\put(1429.0,842.0){\rule[-0.200pt]{2.409pt}{0.400pt}}
\put(170.0,842.0){\rule[-0.200pt]{2.409pt}{0.400pt}}
\put(1429.0,842.0){\rule[-0.200pt]{2.409pt}{0.400pt}}
\put(170.0,842.0){\rule[-0.200pt]{2.409pt}{0.400pt}}
\put(1429.0,842.0){\rule[-0.200pt]{2.409pt}{0.400pt}}
\put(170.0,842.0){\rule[-0.200pt]{2.409pt}{0.400pt}}
\put(1429.0,842.0){\rule[-0.200pt]{2.409pt}{0.400pt}}
\put(170.0,842.0){\rule[-0.200pt]{2.409pt}{0.400pt}}
\put(1429.0,842.0){\rule[-0.200pt]{2.409pt}{0.400pt}}
\put(170.0,842.0){\rule[-0.200pt]{2.409pt}{0.400pt}}
\put(1429.0,842.0){\rule[-0.200pt]{2.409pt}{0.400pt}}
\put(170.0,842.0){\rule[-0.200pt]{2.409pt}{0.400pt}}
\put(1429.0,842.0){\rule[-0.200pt]{2.409pt}{0.400pt}}
\put(170.0,842.0){\rule[-0.200pt]{2.409pt}{0.400pt}}
\put(1429.0,842.0){\rule[-0.200pt]{2.409pt}{0.400pt}}
\put(170.0,842.0){\rule[-0.200pt]{2.409pt}{0.400pt}}
\put(1429.0,842.0){\rule[-0.200pt]{2.409pt}{0.400pt}}
\put(170.0,842.0){\rule[-0.200pt]{2.409pt}{0.400pt}}
\put(1429.0,842.0){\rule[-0.200pt]{2.409pt}{0.400pt}}
\put(170.0,842.0){\rule[-0.200pt]{2.409pt}{0.400pt}}
\put(1429.0,842.0){\rule[-0.200pt]{2.409pt}{0.400pt}}
\put(170.0,842.0){\rule[-0.200pt]{2.409pt}{0.400pt}}
\put(1429.0,842.0){\rule[-0.200pt]{2.409pt}{0.400pt}}
\put(170.0,843.0){\rule[-0.200pt]{2.409pt}{0.400pt}}
\put(1429.0,843.0){\rule[-0.200pt]{2.409pt}{0.400pt}}
\put(170.0,843.0){\rule[-0.200pt]{2.409pt}{0.400pt}}
\put(1429.0,843.0){\rule[-0.200pt]{2.409pt}{0.400pt}}
\put(170.0,843.0){\rule[-0.200pt]{2.409pt}{0.400pt}}
\put(1429.0,843.0){\rule[-0.200pt]{2.409pt}{0.400pt}}
\put(170.0,843.0){\rule[-0.200pt]{2.409pt}{0.400pt}}
\put(1429.0,843.0){\rule[-0.200pt]{2.409pt}{0.400pt}}
\put(170.0,843.0){\rule[-0.200pt]{2.409pt}{0.400pt}}
\put(1429.0,843.0){\rule[-0.200pt]{2.409pt}{0.400pt}}
\put(170.0,843.0){\rule[-0.200pt]{2.409pt}{0.400pt}}
\put(1429.0,843.0){\rule[-0.200pt]{2.409pt}{0.400pt}}
\put(170.0,843.0){\rule[-0.200pt]{2.409pt}{0.400pt}}
\put(1429.0,843.0){\rule[-0.200pt]{2.409pt}{0.400pt}}
\put(170.0,843.0){\rule[-0.200pt]{2.409pt}{0.400pt}}
\put(1429.0,843.0){\rule[-0.200pt]{2.409pt}{0.400pt}}
\put(170.0,843.0){\rule[-0.200pt]{2.409pt}{0.400pt}}
\put(1429.0,843.0){\rule[-0.200pt]{2.409pt}{0.400pt}}
\put(170.0,843.0){\rule[-0.200pt]{2.409pt}{0.400pt}}
\put(1429.0,843.0){\rule[-0.200pt]{2.409pt}{0.400pt}}
\put(170.0,843.0){\rule[-0.200pt]{2.409pt}{0.400pt}}
\put(1429.0,843.0){\rule[-0.200pt]{2.409pt}{0.400pt}}
\put(170.0,843.0){\rule[-0.200pt]{2.409pt}{0.400pt}}
\put(1429.0,843.0){\rule[-0.200pt]{2.409pt}{0.400pt}}
\put(170.0,843.0){\rule[-0.200pt]{2.409pt}{0.400pt}}
\put(1429.0,843.0){\rule[-0.200pt]{2.409pt}{0.400pt}}
\put(170.0,843.0){\rule[-0.200pt]{2.409pt}{0.400pt}}
\put(1429.0,843.0){\rule[-0.200pt]{2.409pt}{0.400pt}}
\put(170.0,843.0){\rule[-0.200pt]{2.409pt}{0.400pt}}
\put(1429.0,843.0){\rule[-0.200pt]{2.409pt}{0.400pt}}
\put(170.0,843.0){\rule[-0.200pt]{2.409pt}{0.400pt}}
\put(1429.0,843.0){\rule[-0.200pt]{2.409pt}{0.400pt}}
\put(170.0,843.0){\rule[-0.200pt]{2.409pt}{0.400pt}}
\put(1429.0,843.0){\rule[-0.200pt]{2.409pt}{0.400pt}}
\put(170.0,843.0){\rule[-0.200pt]{2.409pt}{0.400pt}}
\put(1429.0,843.0){\rule[-0.200pt]{2.409pt}{0.400pt}}
\put(170.0,844.0){\rule[-0.200pt]{2.409pt}{0.400pt}}
\put(1429.0,844.0){\rule[-0.200pt]{2.409pt}{0.400pt}}
\put(170.0,844.0){\rule[-0.200pt]{2.409pt}{0.400pt}}
\put(1429.0,844.0){\rule[-0.200pt]{2.409pt}{0.400pt}}
\put(170.0,844.0){\rule[-0.200pt]{2.409pt}{0.400pt}}
\put(1429.0,844.0){\rule[-0.200pt]{2.409pt}{0.400pt}}
\put(170.0,844.0){\rule[-0.200pt]{2.409pt}{0.400pt}}
\put(1429.0,844.0){\rule[-0.200pt]{2.409pt}{0.400pt}}
\put(170.0,844.0){\rule[-0.200pt]{2.409pt}{0.400pt}}
\put(1429.0,844.0){\rule[-0.200pt]{2.409pt}{0.400pt}}
\put(170.0,844.0){\rule[-0.200pt]{2.409pt}{0.400pt}}
\put(1429.0,844.0){\rule[-0.200pt]{2.409pt}{0.400pt}}
\put(170.0,844.0){\rule[-0.200pt]{2.409pt}{0.400pt}}
\put(1429.0,844.0){\rule[-0.200pt]{2.409pt}{0.400pt}}
\put(170.0,844.0){\rule[-0.200pt]{2.409pt}{0.400pt}}
\put(1429.0,844.0){\rule[-0.200pt]{2.409pt}{0.400pt}}
\put(170.0,844.0){\rule[-0.200pt]{2.409pt}{0.400pt}}
\put(1429.0,844.0){\rule[-0.200pt]{2.409pt}{0.400pt}}
\put(170.0,844.0){\rule[-0.200pt]{2.409pt}{0.400pt}}
\put(1429.0,844.0){\rule[-0.200pt]{2.409pt}{0.400pt}}
\put(170.0,844.0){\rule[-0.200pt]{2.409pt}{0.400pt}}
\put(1429.0,844.0){\rule[-0.200pt]{2.409pt}{0.400pt}}
\put(170.0,844.0){\rule[-0.200pt]{2.409pt}{0.400pt}}
\put(1429.0,844.0){\rule[-0.200pt]{2.409pt}{0.400pt}}
\put(170.0,844.0){\rule[-0.200pt]{2.409pt}{0.400pt}}
\put(1429.0,844.0){\rule[-0.200pt]{2.409pt}{0.400pt}}
\put(170.0,844.0){\rule[-0.200pt]{2.409pt}{0.400pt}}
\put(1429.0,844.0){\rule[-0.200pt]{2.409pt}{0.400pt}}
\put(170.0,844.0){\rule[-0.200pt]{2.409pt}{0.400pt}}
\put(1429.0,844.0){\rule[-0.200pt]{2.409pt}{0.400pt}}
\put(170.0,844.0){\rule[-0.200pt]{2.409pt}{0.400pt}}
\put(1429.0,844.0){\rule[-0.200pt]{2.409pt}{0.400pt}}
\put(170.0,844.0){\rule[-0.200pt]{2.409pt}{0.400pt}}
\put(1429.0,844.0){\rule[-0.200pt]{2.409pt}{0.400pt}}
\put(170.0,844.0){\rule[-0.200pt]{2.409pt}{0.400pt}}
\put(1429.0,844.0){\rule[-0.200pt]{2.409pt}{0.400pt}}
\put(170.0,845.0){\rule[-0.200pt]{2.409pt}{0.400pt}}
\put(1429.0,845.0){\rule[-0.200pt]{2.409pt}{0.400pt}}
\put(170.0,845.0){\rule[-0.200pt]{2.409pt}{0.400pt}}
\put(1429.0,845.0){\rule[-0.200pt]{2.409pt}{0.400pt}}
\put(170.0,845.0){\rule[-0.200pt]{2.409pt}{0.400pt}}
\put(1429.0,845.0){\rule[-0.200pt]{2.409pt}{0.400pt}}
\put(170.0,845.0){\rule[-0.200pt]{2.409pt}{0.400pt}}
\put(1429.0,845.0){\rule[-0.200pt]{2.409pt}{0.400pt}}
\put(170.0,845.0){\rule[-0.200pt]{2.409pt}{0.400pt}}
\put(1429.0,845.0){\rule[-0.200pt]{2.409pt}{0.400pt}}
\put(170.0,845.0){\rule[-0.200pt]{2.409pt}{0.400pt}}
\put(1429.0,845.0){\rule[-0.200pt]{2.409pt}{0.400pt}}
\put(170.0,845.0){\rule[-0.200pt]{2.409pt}{0.400pt}}
\put(1429.0,845.0){\rule[-0.200pt]{2.409pt}{0.400pt}}
\put(170.0,845.0){\rule[-0.200pt]{2.409pt}{0.400pt}}
\put(1429.0,845.0){\rule[-0.200pt]{2.409pt}{0.400pt}}
\put(170.0,845.0){\rule[-0.200pt]{2.409pt}{0.400pt}}
\put(1429.0,845.0){\rule[-0.200pt]{2.409pt}{0.400pt}}
\put(170.0,845.0){\rule[-0.200pt]{2.409pt}{0.400pt}}
\put(1429.0,845.0){\rule[-0.200pt]{2.409pt}{0.400pt}}
\put(170.0,845.0){\rule[-0.200pt]{2.409pt}{0.400pt}}
\put(1429.0,845.0){\rule[-0.200pt]{2.409pt}{0.400pt}}
\put(170.0,845.0){\rule[-0.200pt]{2.409pt}{0.400pt}}
\put(1429.0,845.0){\rule[-0.200pt]{2.409pt}{0.400pt}}
\put(170.0,845.0){\rule[-0.200pt]{2.409pt}{0.400pt}}
\put(1429.0,845.0){\rule[-0.200pt]{2.409pt}{0.400pt}}
\put(170.0,845.0){\rule[-0.200pt]{2.409pt}{0.400pt}}
\put(1429.0,845.0){\rule[-0.200pt]{2.409pt}{0.400pt}}
\put(170.0,845.0){\rule[-0.200pt]{2.409pt}{0.400pt}}
\put(1429.0,845.0){\rule[-0.200pt]{2.409pt}{0.400pt}}
\put(170.0,845.0){\rule[-0.200pt]{2.409pt}{0.400pt}}
\put(1429.0,845.0){\rule[-0.200pt]{2.409pt}{0.400pt}}
\put(170.0,845.0){\rule[-0.200pt]{2.409pt}{0.400pt}}
\put(1429.0,845.0){\rule[-0.200pt]{2.409pt}{0.400pt}}
\put(170.0,845.0){\rule[-0.200pt]{2.409pt}{0.400pt}}
\put(1429.0,845.0){\rule[-0.200pt]{2.409pt}{0.400pt}}
\put(170.0,846.0){\rule[-0.200pt]{2.409pt}{0.400pt}}
\put(1429.0,846.0){\rule[-0.200pt]{2.409pt}{0.400pt}}
\put(170.0,846.0){\rule[-0.200pt]{2.409pt}{0.400pt}}
\put(1429.0,846.0){\rule[-0.200pt]{2.409pt}{0.400pt}}
\put(170.0,846.0){\rule[-0.200pt]{2.409pt}{0.400pt}}
\put(1429.0,846.0){\rule[-0.200pt]{2.409pt}{0.400pt}}
\put(170.0,846.0){\rule[-0.200pt]{2.409pt}{0.400pt}}
\put(1429.0,846.0){\rule[-0.200pt]{2.409pt}{0.400pt}}
\put(170.0,846.0){\rule[-0.200pt]{2.409pt}{0.400pt}}
\put(1429.0,846.0){\rule[-0.200pt]{2.409pt}{0.400pt}}
\put(170.0,846.0){\rule[-0.200pt]{2.409pt}{0.400pt}}
\put(1429.0,846.0){\rule[-0.200pt]{2.409pt}{0.400pt}}
\put(170.0,846.0){\rule[-0.200pt]{2.409pt}{0.400pt}}
\put(1429.0,846.0){\rule[-0.200pt]{2.409pt}{0.400pt}}
\put(170.0,846.0){\rule[-0.200pt]{2.409pt}{0.400pt}}
\put(1429.0,846.0){\rule[-0.200pt]{2.409pt}{0.400pt}}
\put(170.0,846.0){\rule[-0.200pt]{2.409pt}{0.400pt}}
\put(1429.0,846.0){\rule[-0.200pt]{2.409pt}{0.400pt}}
\put(170.0,846.0){\rule[-0.200pt]{2.409pt}{0.400pt}}
\put(1429.0,846.0){\rule[-0.200pt]{2.409pt}{0.400pt}}
\put(170.0,846.0){\rule[-0.200pt]{2.409pt}{0.400pt}}
\put(1429.0,846.0){\rule[-0.200pt]{2.409pt}{0.400pt}}
\put(170.0,846.0){\rule[-0.200pt]{2.409pt}{0.400pt}}
\put(1429.0,846.0){\rule[-0.200pt]{2.409pt}{0.400pt}}
\put(170.0,846.0){\rule[-0.200pt]{2.409pt}{0.400pt}}
\put(1429.0,846.0){\rule[-0.200pt]{2.409pt}{0.400pt}}
\put(170.0,846.0){\rule[-0.200pt]{2.409pt}{0.400pt}}
\put(1429.0,846.0){\rule[-0.200pt]{2.409pt}{0.400pt}}
\put(170.0,846.0){\rule[-0.200pt]{2.409pt}{0.400pt}}
\put(1429.0,846.0){\rule[-0.200pt]{2.409pt}{0.400pt}}
\put(170.0,846.0){\rule[-0.200pt]{2.409pt}{0.400pt}}
\put(1429.0,846.0){\rule[-0.200pt]{2.409pt}{0.400pt}}
\put(170.0,846.0){\rule[-0.200pt]{2.409pt}{0.400pt}}
\put(1429.0,846.0){\rule[-0.200pt]{2.409pt}{0.400pt}}
\put(170.0,846.0){\rule[-0.200pt]{2.409pt}{0.400pt}}
\put(1429.0,846.0){\rule[-0.200pt]{2.409pt}{0.400pt}}
\put(170.0,846.0){\rule[-0.200pt]{2.409pt}{0.400pt}}
\put(1429.0,846.0){\rule[-0.200pt]{2.409pt}{0.400pt}}
\put(170.0,847.0){\rule[-0.200pt]{2.409pt}{0.400pt}}
\put(1429.0,847.0){\rule[-0.200pt]{2.409pt}{0.400pt}}
\put(170.0,847.0){\rule[-0.200pt]{2.409pt}{0.400pt}}
\put(1429.0,847.0){\rule[-0.200pt]{2.409pt}{0.400pt}}
\put(170.0,847.0){\rule[-0.200pt]{2.409pt}{0.400pt}}
\put(1429.0,847.0){\rule[-0.200pt]{2.409pt}{0.400pt}}
\put(170.0,847.0){\rule[-0.200pt]{2.409pt}{0.400pt}}
\put(1429.0,847.0){\rule[-0.200pt]{2.409pt}{0.400pt}}
\put(170.0,847.0){\rule[-0.200pt]{2.409pt}{0.400pt}}
\put(1429.0,847.0){\rule[-0.200pt]{2.409pt}{0.400pt}}
\put(170.0,847.0){\rule[-0.200pt]{2.409pt}{0.400pt}}
\put(1429.0,847.0){\rule[-0.200pt]{2.409pt}{0.400pt}}
\put(170.0,847.0){\rule[-0.200pt]{2.409pt}{0.400pt}}
\put(1429.0,847.0){\rule[-0.200pt]{2.409pt}{0.400pt}}
\put(170.0,847.0){\rule[-0.200pt]{2.409pt}{0.400pt}}
\put(1429.0,847.0){\rule[-0.200pt]{2.409pt}{0.400pt}}
\put(170.0,847.0){\rule[-0.200pt]{2.409pt}{0.400pt}}
\put(1429.0,847.0){\rule[-0.200pt]{2.409pt}{0.400pt}}
\put(170.0,847.0){\rule[-0.200pt]{2.409pt}{0.400pt}}
\put(1429.0,847.0){\rule[-0.200pt]{2.409pt}{0.400pt}}
\put(170.0,847.0){\rule[-0.200pt]{2.409pt}{0.400pt}}
\put(1429.0,847.0){\rule[-0.200pt]{2.409pt}{0.400pt}}
\put(170.0,847.0){\rule[-0.200pt]{2.409pt}{0.400pt}}
\put(1429.0,847.0){\rule[-0.200pt]{2.409pt}{0.400pt}}
\put(170.0,847.0){\rule[-0.200pt]{2.409pt}{0.400pt}}
\put(1429.0,847.0){\rule[-0.200pt]{2.409pt}{0.400pt}}
\put(170.0,847.0){\rule[-0.200pt]{2.409pt}{0.400pt}}
\put(1429.0,847.0){\rule[-0.200pt]{2.409pt}{0.400pt}}
\put(170.0,847.0){\rule[-0.200pt]{2.409pt}{0.400pt}}
\put(1429.0,847.0){\rule[-0.200pt]{2.409pt}{0.400pt}}
\put(170.0,847.0){\rule[-0.200pt]{2.409pt}{0.400pt}}
\put(1429.0,847.0){\rule[-0.200pt]{2.409pt}{0.400pt}}
\put(170.0,847.0){\rule[-0.200pt]{2.409pt}{0.400pt}}
\put(1429.0,847.0){\rule[-0.200pt]{2.409pt}{0.400pt}}
\put(170.0,847.0){\rule[-0.200pt]{2.409pt}{0.400pt}}
\put(1429.0,847.0){\rule[-0.200pt]{2.409pt}{0.400pt}}
\put(170.0,847.0){\rule[-0.200pt]{2.409pt}{0.400pt}}
\put(1429.0,847.0){\rule[-0.200pt]{2.409pt}{0.400pt}}
\put(170.0,848.0){\rule[-0.200pt]{2.409pt}{0.400pt}}
\put(1429.0,848.0){\rule[-0.200pt]{2.409pt}{0.400pt}}
\put(170.0,848.0){\rule[-0.200pt]{2.409pt}{0.400pt}}
\put(1429.0,848.0){\rule[-0.200pt]{2.409pt}{0.400pt}}
\put(170.0,848.0){\rule[-0.200pt]{2.409pt}{0.400pt}}
\put(1429.0,848.0){\rule[-0.200pt]{2.409pt}{0.400pt}}
\put(170.0,848.0){\rule[-0.200pt]{2.409pt}{0.400pt}}
\put(1429.0,848.0){\rule[-0.200pt]{2.409pt}{0.400pt}}
\put(170.0,848.0){\rule[-0.200pt]{2.409pt}{0.400pt}}
\put(1429.0,848.0){\rule[-0.200pt]{2.409pt}{0.400pt}}
\put(170.0,848.0){\rule[-0.200pt]{2.409pt}{0.400pt}}
\put(1429.0,848.0){\rule[-0.200pt]{2.409pt}{0.400pt}}
\put(170.0,848.0){\rule[-0.200pt]{2.409pt}{0.400pt}}
\put(1429.0,848.0){\rule[-0.200pt]{2.409pt}{0.400pt}}
\put(170.0,848.0){\rule[-0.200pt]{2.409pt}{0.400pt}}
\put(1429.0,848.0){\rule[-0.200pt]{2.409pt}{0.400pt}}
\put(170.0,848.0){\rule[-0.200pt]{2.409pt}{0.400pt}}
\put(1429.0,848.0){\rule[-0.200pt]{2.409pt}{0.400pt}}
\put(170.0,848.0){\rule[-0.200pt]{2.409pt}{0.400pt}}
\put(1429.0,848.0){\rule[-0.200pt]{2.409pt}{0.400pt}}
\put(170.0,848.0){\rule[-0.200pt]{2.409pt}{0.400pt}}
\put(1429.0,848.0){\rule[-0.200pt]{2.409pt}{0.400pt}}
\put(170.0,848.0){\rule[-0.200pt]{2.409pt}{0.400pt}}
\put(1429.0,848.0){\rule[-0.200pt]{2.409pt}{0.400pt}}
\put(170.0,848.0){\rule[-0.200pt]{2.409pt}{0.400pt}}
\put(1429.0,848.0){\rule[-0.200pt]{2.409pt}{0.400pt}}
\put(170.0,848.0){\rule[-0.200pt]{2.409pt}{0.400pt}}
\put(1429.0,848.0){\rule[-0.200pt]{2.409pt}{0.400pt}}
\put(170.0,848.0){\rule[-0.200pt]{2.409pt}{0.400pt}}
\put(1429.0,848.0){\rule[-0.200pt]{2.409pt}{0.400pt}}
\put(170.0,848.0){\rule[-0.200pt]{2.409pt}{0.400pt}}
\put(1429.0,848.0){\rule[-0.200pt]{2.409pt}{0.400pt}}
\put(170.0,848.0){\rule[-0.200pt]{2.409pt}{0.400pt}}
\put(1429.0,848.0){\rule[-0.200pt]{2.409pt}{0.400pt}}
\put(170.0,848.0){\rule[-0.200pt]{2.409pt}{0.400pt}}
\put(1429.0,848.0){\rule[-0.200pt]{2.409pt}{0.400pt}}
\put(170.0,848.0){\rule[-0.200pt]{2.409pt}{0.400pt}}
\put(1429.0,848.0){\rule[-0.200pt]{2.409pt}{0.400pt}}
\put(170.0,848.0){\rule[-0.200pt]{2.409pt}{0.400pt}}
\put(1429.0,848.0){\rule[-0.200pt]{2.409pt}{0.400pt}}
\put(170.0,849.0){\rule[-0.200pt]{2.409pt}{0.400pt}}
\put(1429.0,849.0){\rule[-0.200pt]{2.409pt}{0.400pt}}
\put(170.0,849.0){\rule[-0.200pt]{2.409pt}{0.400pt}}
\put(1429.0,849.0){\rule[-0.200pt]{2.409pt}{0.400pt}}
\put(170.0,849.0){\rule[-0.200pt]{2.409pt}{0.400pt}}
\put(1429.0,849.0){\rule[-0.200pt]{2.409pt}{0.400pt}}
\put(170.0,849.0){\rule[-0.200pt]{2.409pt}{0.400pt}}
\put(1429.0,849.0){\rule[-0.200pt]{2.409pt}{0.400pt}}
\put(170.0,849.0){\rule[-0.200pt]{2.409pt}{0.400pt}}
\put(1429.0,849.0){\rule[-0.200pt]{2.409pt}{0.400pt}}
\put(170.0,849.0){\rule[-0.200pt]{2.409pt}{0.400pt}}
\put(1429.0,849.0){\rule[-0.200pt]{2.409pt}{0.400pt}}
\put(170.0,849.0){\rule[-0.200pt]{2.409pt}{0.400pt}}
\put(1429.0,849.0){\rule[-0.200pt]{2.409pt}{0.400pt}}
\put(170.0,849.0){\rule[-0.200pt]{2.409pt}{0.400pt}}
\put(1429.0,849.0){\rule[-0.200pt]{2.409pt}{0.400pt}}
\put(170.0,849.0){\rule[-0.200pt]{2.409pt}{0.400pt}}
\put(1429.0,849.0){\rule[-0.200pt]{2.409pt}{0.400pt}}
\put(170.0,849.0){\rule[-0.200pt]{2.409pt}{0.400pt}}
\put(1429.0,849.0){\rule[-0.200pt]{2.409pt}{0.400pt}}
\put(170.0,849.0){\rule[-0.200pt]{2.409pt}{0.400pt}}
\put(1429.0,849.0){\rule[-0.200pt]{2.409pt}{0.400pt}}
\put(170.0,849.0){\rule[-0.200pt]{2.409pt}{0.400pt}}
\put(1429.0,849.0){\rule[-0.200pt]{2.409pt}{0.400pt}}
\put(170.0,849.0){\rule[-0.200pt]{2.409pt}{0.400pt}}
\put(1429.0,849.0){\rule[-0.200pt]{2.409pt}{0.400pt}}
\put(170.0,849.0){\rule[-0.200pt]{2.409pt}{0.400pt}}
\put(1429.0,849.0){\rule[-0.200pt]{2.409pt}{0.400pt}}
\put(170.0,849.0){\rule[-0.200pt]{2.409pt}{0.400pt}}
\put(1429.0,849.0){\rule[-0.200pt]{2.409pt}{0.400pt}}
\put(170.0,849.0){\rule[-0.200pt]{2.409pt}{0.400pt}}
\put(1429.0,849.0){\rule[-0.200pt]{2.409pt}{0.400pt}}
\put(170.0,849.0){\rule[-0.200pt]{2.409pt}{0.400pt}}
\put(1429.0,849.0){\rule[-0.200pt]{2.409pt}{0.400pt}}
\put(170.0,849.0){\rule[-0.200pt]{2.409pt}{0.400pt}}
\put(1429.0,849.0){\rule[-0.200pt]{2.409pt}{0.400pt}}
\put(170.0,849.0){\rule[-0.200pt]{2.409pt}{0.400pt}}
\put(1429.0,849.0){\rule[-0.200pt]{2.409pt}{0.400pt}}
\put(170.0,849.0){\rule[-0.200pt]{2.409pt}{0.400pt}}
\put(1429.0,849.0){\rule[-0.200pt]{2.409pt}{0.400pt}}
\put(170.0,849.0){\rule[-0.200pt]{2.409pt}{0.400pt}}
\put(1429.0,849.0){\rule[-0.200pt]{2.409pt}{0.400pt}}
\put(170.0,850.0){\rule[-0.200pt]{2.409pt}{0.400pt}}
\put(1429.0,850.0){\rule[-0.200pt]{2.409pt}{0.400pt}}
\put(170.0,850.0){\rule[-0.200pt]{2.409pt}{0.400pt}}
\put(1429.0,850.0){\rule[-0.200pt]{2.409pt}{0.400pt}}
\put(170.0,850.0){\rule[-0.200pt]{2.409pt}{0.400pt}}
\put(1429.0,850.0){\rule[-0.200pt]{2.409pt}{0.400pt}}
\put(170.0,850.0){\rule[-0.200pt]{2.409pt}{0.400pt}}
\put(1429.0,850.0){\rule[-0.200pt]{2.409pt}{0.400pt}}
\put(170.0,850.0){\rule[-0.200pt]{2.409pt}{0.400pt}}
\put(1429.0,850.0){\rule[-0.200pt]{2.409pt}{0.400pt}}
\put(170.0,850.0){\rule[-0.200pt]{2.409pt}{0.400pt}}
\put(1429.0,850.0){\rule[-0.200pt]{2.409pt}{0.400pt}}
\put(170.0,850.0){\rule[-0.200pt]{2.409pt}{0.400pt}}
\put(1429.0,850.0){\rule[-0.200pt]{2.409pt}{0.400pt}}
\put(170.0,850.0){\rule[-0.200pt]{2.409pt}{0.400pt}}
\put(1429.0,850.0){\rule[-0.200pt]{2.409pt}{0.400pt}}
\put(170.0,850.0){\rule[-0.200pt]{2.409pt}{0.400pt}}
\put(1429.0,850.0){\rule[-0.200pt]{2.409pt}{0.400pt}}
\put(170.0,850.0){\rule[-0.200pt]{2.409pt}{0.400pt}}
\put(1429.0,850.0){\rule[-0.200pt]{2.409pt}{0.400pt}}
\put(170.0,850.0){\rule[-0.200pt]{2.409pt}{0.400pt}}
\put(1429.0,850.0){\rule[-0.200pt]{2.409pt}{0.400pt}}
\put(170.0,850.0){\rule[-0.200pt]{2.409pt}{0.400pt}}
\put(1429.0,850.0){\rule[-0.200pt]{2.409pt}{0.400pt}}
\put(170.0,850.0){\rule[-0.200pt]{2.409pt}{0.400pt}}
\put(1429.0,850.0){\rule[-0.200pt]{2.409pt}{0.400pt}}
\put(170.0,850.0){\rule[-0.200pt]{2.409pt}{0.400pt}}
\put(1429.0,850.0){\rule[-0.200pt]{2.409pt}{0.400pt}}
\put(170.0,850.0){\rule[-0.200pt]{2.409pt}{0.400pt}}
\put(1429.0,850.0){\rule[-0.200pt]{2.409pt}{0.400pt}}
\put(170.0,850.0){\rule[-0.200pt]{2.409pt}{0.400pt}}
\put(1429.0,850.0){\rule[-0.200pt]{2.409pt}{0.400pt}}
\put(170.0,850.0){\rule[-0.200pt]{2.409pt}{0.400pt}}
\put(1429.0,850.0){\rule[-0.200pt]{2.409pt}{0.400pt}}
\put(170.0,850.0){\rule[-0.200pt]{2.409pt}{0.400pt}}
\put(1429.0,850.0){\rule[-0.200pt]{2.409pt}{0.400pt}}
\put(170.0,850.0){\rule[-0.200pt]{2.409pt}{0.400pt}}
\put(1429.0,850.0){\rule[-0.200pt]{2.409pt}{0.400pt}}
\put(170.0,850.0){\rule[-0.200pt]{2.409pt}{0.400pt}}
\put(1429.0,850.0){\rule[-0.200pt]{2.409pt}{0.400pt}}
\put(170.0,850.0){\rule[-0.200pt]{2.409pt}{0.400pt}}
\put(1429.0,850.0){\rule[-0.200pt]{2.409pt}{0.400pt}}
\put(170.0,851.0){\rule[-0.200pt]{2.409pt}{0.400pt}}
\put(1429.0,851.0){\rule[-0.200pt]{2.409pt}{0.400pt}}
\put(170.0,851.0){\rule[-0.200pt]{2.409pt}{0.400pt}}
\put(1429.0,851.0){\rule[-0.200pt]{2.409pt}{0.400pt}}
\put(170.0,851.0){\rule[-0.200pt]{2.409pt}{0.400pt}}
\put(1429.0,851.0){\rule[-0.200pt]{2.409pt}{0.400pt}}
\put(170.0,851.0){\rule[-0.200pt]{2.409pt}{0.400pt}}
\put(1429.0,851.0){\rule[-0.200pt]{2.409pt}{0.400pt}}
\put(170.0,851.0){\rule[-0.200pt]{2.409pt}{0.400pt}}
\put(1429.0,851.0){\rule[-0.200pt]{2.409pt}{0.400pt}}
\put(170.0,851.0){\rule[-0.200pt]{2.409pt}{0.400pt}}
\put(1429.0,851.0){\rule[-0.200pt]{2.409pt}{0.400pt}}
\put(170.0,851.0){\rule[-0.200pt]{2.409pt}{0.400pt}}
\put(1429.0,851.0){\rule[-0.200pt]{2.409pt}{0.400pt}}
\put(170.0,851.0){\rule[-0.200pt]{2.409pt}{0.400pt}}
\put(1429.0,851.0){\rule[-0.200pt]{2.409pt}{0.400pt}}
\put(170.0,851.0){\rule[-0.200pt]{2.409pt}{0.400pt}}
\put(1429.0,851.0){\rule[-0.200pt]{2.409pt}{0.400pt}}
\put(170.0,851.0){\rule[-0.200pt]{2.409pt}{0.400pt}}
\put(1429.0,851.0){\rule[-0.200pt]{2.409pt}{0.400pt}}
\put(170.0,851.0){\rule[-0.200pt]{2.409pt}{0.400pt}}
\put(1429.0,851.0){\rule[-0.200pt]{2.409pt}{0.400pt}}
\put(170.0,851.0){\rule[-0.200pt]{2.409pt}{0.400pt}}
\put(1429.0,851.0){\rule[-0.200pt]{2.409pt}{0.400pt}}
\put(170.0,851.0){\rule[-0.200pt]{2.409pt}{0.400pt}}
\put(1429.0,851.0){\rule[-0.200pt]{2.409pt}{0.400pt}}
\put(170.0,851.0){\rule[-0.200pt]{2.409pt}{0.400pt}}
\put(1429.0,851.0){\rule[-0.200pt]{2.409pt}{0.400pt}}
\put(170.0,851.0){\rule[-0.200pt]{2.409pt}{0.400pt}}
\put(1429.0,851.0){\rule[-0.200pt]{2.409pt}{0.400pt}}
\put(170.0,851.0){\rule[-0.200pt]{2.409pt}{0.400pt}}
\put(1429.0,851.0){\rule[-0.200pt]{2.409pt}{0.400pt}}
\put(170.0,851.0){\rule[-0.200pt]{2.409pt}{0.400pt}}
\put(1429.0,851.0){\rule[-0.200pt]{2.409pt}{0.400pt}}
\put(170.0,851.0){\rule[-0.200pt]{2.409pt}{0.400pt}}
\put(1429.0,851.0){\rule[-0.200pt]{2.409pt}{0.400pt}}
\put(170.0,851.0){\rule[-0.200pt]{2.409pt}{0.400pt}}
\put(1429.0,851.0){\rule[-0.200pt]{2.409pt}{0.400pt}}
\put(170.0,851.0){\rule[-0.200pt]{2.409pt}{0.400pt}}
\put(1429.0,851.0){\rule[-0.200pt]{2.409pt}{0.400pt}}
\put(170.0,851.0){\rule[-0.200pt]{2.409pt}{0.400pt}}
\put(1429.0,851.0){\rule[-0.200pt]{2.409pt}{0.400pt}}
\put(170.0,852.0){\rule[-0.200pt]{2.409pt}{0.400pt}}
\put(1429.0,852.0){\rule[-0.200pt]{2.409pt}{0.400pt}}
\put(170.0,852.0){\rule[-0.200pt]{2.409pt}{0.400pt}}
\put(1429.0,852.0){\rule[-0.200pt]{2.409pt}{0.400pt}}
\put(170.0,852.0){\rule[-0.200pt]{2.409pt}{0.400pt}}
\put(1429.0,852.0){\rule[-0.200pt]{2.409pt}{0.400pt}}
\put(170.0,852.0){\rule[-0.200pt]{2.409pt}{0.400pt}}
\put(1429.0,852.0){\rule[-0.200pt]{2.409pt}{0.400pt}}
\put(170.0,852.0){\rule[-0.200pt]{2.409pt}{0.400pt}}
\put(1429.0,852.0){\rule[-0.200pt]{2.409pt}{0.400pt}}
\put(170.0,852.0){\rule[-0.200pt]{2.409pt}{0.400pt}}
\put(1429.0,852.0){\rule[-0.200pt]{2.409pt}{0.400pt}}
\put(170.0,852.0){\rule[-0.200pt]{2.409pt}{0.400pt}}
\put(1429.0,852.0){\rule[-0.200pt]{2.409pt}{0.400pt}}
\put(170.0,852.0){\rule[-0.200pt]{2.409pt}{0.400pt}}
\put(1429.0,852.0){\rule[-0.200pt]{2.409pt}{0.400pt}}
\put(170.0,852.0){\rule[-0.200pt]{2.409pt}{0.400pt}}
\put(1429.0,852.0){\rule[-0.200pt]{2.409pt}{0.400pt}}
\put(170.0,852.0){\rule[-0.200pt]{2.409pt}{0.400pt}}
\put(1429.0,852.0){\rule[-0.200pt]{2.409pt}{0.400pt}}
\put(170.0,852.0){\rule[-0.200pt]{2.409pt}{0.400pt}}
\put(1429.0,852.0){\rule[-0.200pt]{2.409pt}{0.400pt}}
\put(170.0,852.0){\rule[-0.200pt]{2.409pt}{0.400pt}}
\put(1429.0,852.0){\rule[-0.200pt]{2.409pt}{0.400pt}}
\put(170.0,852.0){\rule[-0.200pt]{2.409pt}{0.400pt}}
\put(1429.0,852.0){\rule[-0.200pt]{2.409pt}{0.400pt}}
\put(170.0,852.0){\rule[-0.200pt]{2.409pt}{0.400pt}}
\put(1429.0,852.0){\rule[-0.200pt]{2.409pt}{0.400pt}}
\put(170.0,852.0){\rule[-0.200pt]{2.409pt}{0.400pt}}
\put(1429.0,852.0){\rule[-0.200pt]{2.409pt}{0.400pt}}
\put(170.0,852.0){\rule[-0.200pt]{2.409pt}{0.400pt}}
\put(1429.0,852.0){\rule[-0.200pt]{2.409pt}{0.400pt}}
\put(170.0,852.0){\rule[-0.200pt]{2.409pt}{0.400pt}}
\put(1429.0,852.0){\rule[-0.200pt]{2.409pt}{0.400pt}}
\put(170.0,852.0){\rule[-0.200pt]{2.409pt}{0.400pt}}
\put(1429.0,852.0){\rule[-0.200pt]{2.409pt}{0.400pt}}
\put(170.0,852.0){\rule[-0.200pt]{2.409pt}{0.400pt}}
\put(1429.0,852.0){\rule[-0.200pt]{2.409pt}{0.400pt}}
\put(170.0,852.0){\rule[-0.200pt]{2.409pt}{0.400pt}}
\put(1429.0,852.0){\rule[-0.200pt]{2.409pt}{0.400pt}}
\put(170.0,852.0){\rule[-0.200pt]{2.409pt}{0.400pt}}
\put(1429.0,852.0){\rule[-0.200pt]{2.409pt}{0.400pt}}
\put(170.0,852.0){\rule[-0.200pt]{2.409pt}{0.400pt}}
\put(1429.0,852.0){\rule[-0.200pt]{2.409pt}{0.400pt}}
\put(170.0,853.0){\rule[-0.200pt]{2.409pt}{0.400pt}}
\put(1429.0,853.0){\rule[-0.200pt]{2.409pt}{0.400pt}}
\put(170.0,853.0){\rule[-0.200pt]{2.409pt}{0.400pt}}
\put(1429.0,853.0){\rule[-0.200pt]{2.409pt}{0.400pt}}
\put(170.0,853.0){\rule[-0.200pt]{2.409pt}{0.400pt}}
\put(1429.0,853.0){\rule[-0.200pt]{2.409pt}{0.400pt}}
\put(170.0,853.0){\rule[-0.200pt]{2.409pt}{0.400pt}}
\put(1429.0,853.0){\rule[-0.200pt]{2.409pt}{0.400pt}}
\put(170.0,853.0){\rule[-0.200pt]{2.409pt}{0.400pt}}
\put(1429.0,853.0){\rule[-0.200pt]{2.409pt}{0.400pt}}
\put(170.0,853.0){\rule[-0.200pt]{2.409pt}{0.400pt}}
\put(1429.0,853.0){\rule[-0.200pt]{2.409pt}{0.400pt}}
\put(170.0,853.0){\rule[-0.200pt]{2.409pt}{0.400pt}}
\put(1429.0,853.0){\rule[-0.200pt]{2.409pt}{0.400pt}}
\put(170.0,853.0){\rule[-0.200pt]{2.409pt}{0.400pt}}
\put(1429.0,853.0){\rule[-0.200pt]{2.409pt}{0.400pt}}
\put(170.0,853.0){\rule[-0.200pt]{2.409pt}{0.400pt}}
\put(1429.0,853.0){\rule[-0.200pt]{2.409pt}{0.400pt}}
\put(170.0,853.0){\rule[-0.200pt]{2.409pt}{0.400pt}}
\put(1429.0,853.0){\rule[-0.200pt]{2.409pt}{0.400pt}}
\put(170.0,853.0){\rule[-0.200pt]{2.409pt}{0.400pt}}
\put(1429.0,853.0){\rule[-0.200pt]{2.409pt}{0.400pt}}
\put(170.0,853.0){\rule[-0.200pt]{2.409pt}{0.400pt}}
\put(1429.0,853.0){\rule[-0.200pt]{2.409pt}{0.400pt}}
\put(170.0,853.0){\rule[-0.200pt]{2.409pt}{0.400pt}}
\put(1429.0,853.0){\rule[-0.200pt]{2.409pt}{0.400pt}}
\put(170.0,853.0){\rule[-0.200pt]{2.409pt}{0.400pt}}
\put(1429.0,853.0){\rule[-0.200pt]{2.409pt}{0.400pt}}
\put(170.0,853.0){\rule[-0.200pt]{2.409pt}{0.400pt}}
\put(1429.0,853.0){\rule[-0.200pt]{2.409pt}{0.400pt}}
\put(170.0,853.0){\rule[-0.200pt]{2.409pt}{0.400pt}}
\put(1429.0,853.0){\rule[-0.200pt]{2.409pt}{0.400pt}}
\put(170.0,853.0){\rule[-0.200pt]{2.409pt}{0.400pt}}
\put(1429.0,853.0){\rule[-0.200pt]{2.409pt}{0.400pt}}
\put(170.0,853.0){\rule[-0.200pt]{2.409pt}{0.400pt}}
\put(1429.0,853.0){\rule[-0.200pt]{2.409pt}{0.400pt}}
\put(170.0,853.0){\rule[-0.200pt]{2.409pt}{0.400pt}}
\put(1429.0,853.0){\rule[-0.200pt]{2.409pt}{0.400pt}}
\put(170.0,853.0){\rule[-0.200pt]{2.409pt}{0.400pt}}
\put(1429.0,853.0){\rule[-0.200pt]{2.409pt}{0.400pt}}
\put(170.0,853.0){\rule[-0.200pt]{2.409pt}{0.400pt}}
\put(1429.0,853.0){\rule[-0.200pt]{2.409pt}{0.400pt}}
\put(170.0,853.0){\rule[-0.200pt]{2.409pt}{0.400pt}}
\put(1429.0,853.0){\rule[-0.200pt]{2.409pt}{0.400pt}}
\put(170.0,853.0){\rule[-0.200pt]{2.409pt}{0.400pt}}
\put(1429.0,853.0){\rule[-0.200pt]{2.409pt}{0.400pt}}
\put(170.0,854.0){\rule[-0.200pt]{2.409pt}{0.400pt}}
\put(1429.0,854.0){\rule[-0.200pt]{2.409pt}{0.400pt}}
\put(170.0,854.0){\rule[-0.200pt]{2.409pt}{0.400pt}}
\put(1429.0,854.0){\rule[-0.200pt]{2.409pt}{0.400pt}}
\put(170.0,854.0){\rule[-0.200pt]{2.409pt}{0.400pt}}
\put(1429.0,854.0){\rule[-0.200pt]{2.409pt}{0.400pt}}
\put(170.0,854.0){\rule[-0.200pt]{2.409pt}{0.400pt}}
\put(1429.0,854.0){\rule[-0.200pt]{2.409pt}{0.400pt}}
\put(170.0,854.0){\rule[-0.200pt]{2.409pt}{0.400pt}}
\put(1429.0,854.0){\rule[-0.200pt]{2.409pt}{0.400pt}}
\put(170.0,854.0){\rule[-0.200pt]{2.409pt}{0.400pt}}
\put(1429.0,854.0){\rule[-0.200pt]{2.409pt}{0.400pt}}
\put(170.0,854.0){\rule[-0.200pt]{2.409pt}{0.400pt}}
\put(1429.0,854.0){\rule[-0.200pt]{2.409pt}{0.400pt}}
\put(170.0,854.0){\rule[-0.200pt]{2.409pt}{0.400pt}}
\put(1429.0,854.0){\rule[-0.200pt]{2.409pt}{0.400pt}}
\put(170.0,854.0){\rule[-0.200pt]{2.409pt}{0.400pt}}
\put(1429.0,854.0){\rule[-0.200pt]{2.409pt}{0.400pt}}
\put(170.0,854.0){\rule[-0.200pt]{2.409pt}{0.400pt}}
\put(1429.0,854.0){\rule[-0.200pt]{2.409pt}{0.400pt}}
\put(170.0,854.0){\rule[-0.200pt]{2.409pt}{0.400pt}}
\put(1429.0,854.0){\rule[-0.200pt]{2.409pt}{0.400pt}}
\put(170.0,854.0){\rule[-0.200pt]{2.409pt}{0.400pt}}
\put(1429.0,854.0){\rule[-0.200pt]{2.409pt}{0.400pt}}
\put(170.0,854.0){\rule[-0.200pt]{2.409pt}{0.400pt}}
\put(1429.0,854.0){\rule[-0.200pt]{2.409pt}{0.400pt}}
\put(170.0,854.0){\rule[-0.200pt]{2.409pt}{0.400pt}}
\put(1429.0,854.0){\rule[-0.200pt]{2.409pt}{0.400pt}}
\put(170.0,854.0){\rule[-0.200pt]{2.409pt}{0.400pt}}
\put(1429.0,854.0){\rule[-0.200pt]{2.409pt}{0.400pt}}
\put(170.0,854.0){\rule[-0.200pt]{2.409pt}{0.400pt}}
\put(1429.0,854.0){\rule[-0.200pt]{2.409pt}{0.400pt}}
\put(170.0,854.0){\rule[-0.200pt]{2.409pt}{0.400pt}}
\put(1429.0,854.0){\rule[-0.200pt]{2.409pt}{0.400pt}}
\put(170.0,854.0){\rule[-0.200pt]{2.409pt}{0.400pt}}
\put(1429.0,854.0){\rule[-0.200pt]{2.409pt}{0.400pt}}
\put(170.0,854.0){\rule[-0.200pt]{2.409pt}{0.400pt}}
\put(1429.0,854.0){\rule[-0.200pt]{2.409pt}{0.400pt}}
\put(170.0,854.0){\rule[-0.200pt]{2.409pt}{0.400pt}}
\put(1429.0,854.0){\rule[-0.200pt]{2.409pt}{0.400pt}}
\put(170.0,854.0){\rule[-0.200pt]{2.409pt}{0.400pt}}
\put(1429.0,854.0){\rule[-0.200pt]{2.409pt}{0.400pt}}
\put(170.0,854.0){\rule[-0.200pt]{2.409pt}{0.400pt}}
\put(1429.0,854.0){\rule[-0.200pt]{2.409pt}{0.400pt}}
\put(170.0,854.0){\rule[-0.200pt]{2.409pt}{0.400pt}}
\put(1429.0,854.0){\rule[-0.200pt]{2.409pt}{0.400pt}}
\put(170.0,855.0){\rule[-0.200pt]{2.409pt}{0.400pt}}
\put(1429.0,855.0){\rule[-0.200pt]{2.409pt}{0.400pt}}
\put(170.0,855.0){\rule[-0.200pt]{2.409pt}{0.400pt}}
\put(1429.0,855.0){\rule[-0.200pt]{2.409pt}{0.400pt}}
\put(170.0,855.0){\rule[-0.200pt]{2.409pt}{0.400pt}}
\put(1429.0,855.0){\rule[-0.200pt]{2.409pt}{0.400pt}}
\put(170.0,855.0){\rule[-0.200pt]{2.409pt}{0.400pt}}
\put(1429.0,855.0){\rule[-0.200pt]{2.409pt}{0.400pt}}
\put(170.0,855.0){\rule[-0.200pt]{2.409pt}{0.400pt}}
\put(1429.0,855.0){\rule[-0.200pt]{2.409pt}{0.400pt}}
\put(170.0,855.0){\rule[-0.200pt]{2.409pt}{0.400pt}}
\put(1429.0,855.0){\rule[-0.200pt]{2.409pt}{0.400pt}}
\put(170.0,855.0){\rule[-0.200pt]{2.409pt}{0.400pt}}
\put(1429.0,855.0){\rule[-0.200pt]{2.409pt}{0.400pt}}
\put(170.0,855.0){\rule[-0.200pt]{2.409pt}{0.400pt}}
\put(1429.0,855.0){\rule[-0.200pt]{2.409pt}{0.400pt}}
\put(170.0,855.0){\rule[-0.200pt]{2.409pt}{0.400pt}}
\put(1429.0,855.0){\rule[-0.200pt]{2.409pt}{0.400pt}}
\put(170.0,855.0){\rule[-0.200pt]{2.409pt}{0.400pt}}
\put(1429.0,855.0){\rule[-0.200pt]{2.409pt}{0.400pt}}
\put(170.0,855.0){\rule[-0.200pt]{2.409pt}{0.400pt}}
\put(1429.0,855.0){\rule[-0.200pt]{2.409pt}{0.400pt}}
\put(170.0,855.0){\rule[-0.200pt]{2.409pt}{0.400pt}}
\put(1429.0,855.0){\rule[-0.200pt]{2.409pt}{0.400pt}}
\put(170.0,855.0){\rule[-0.200pt]{2.409pt}{0.400pt}}
\put(1429.0,855.0){\rule[-0.200pt]{2.409pt}{0.400pt}}
\put(170.0,855.0){\rule[-0.200pt]{2.409pt}{0.400pt}}
\put(1429.0,855.0){\rule[-0.200pt]{2.409pt}{0.400pt}}
\put(170.0,855.0){\rule[-0.200pt]{2.409pt}{0.400pt}}
\put(1429.0,855.0){\rule[-0.200pt]{2.409pt}{0.400pt}}
\put(170.0,855.0){\rule[-0.200pt]{2.409pt}{0.400pt}}
\put(1429.0,855.0){\rule[-0.200pt]{2.409pt}{0.400pt}}
\put(170.0,855.0){\rule[-0.200pt]{2.409pt}{0.400pt}}
\put(1429.0,855.0){\rule[-0.200pt]{2.409pt}{0.400pt}}
\put(170.0,855.0){\rule[-0.200pt]{2.409pt}{0.400pt}}
\put(1429.0,855.0){\rule[-0.200pt]{2.409pt}{0.400pt}}
\put(170.0,855.0){\rule[-0.200pt]{2.409pt}{0.400pt}}
\put(1429.0,855.0){\rule[-0.200pt]{2.409pt}{0.400pt}}
\put(170.0,855.0){\rule[-0.200pt]{2.409pt}{0.400pt}}
\put(1429.0,855.0){\rule[-0.200pt]{2.409pt}{0.400pt}}
\put(170.0,855.0){\rule[-0.200pt]{2.409pt}{0.400pt}}
\put(1429.0,855.0){\rule[-0.200pt]{2.409pt}{0.400pt}}
\put(170.0,855.0){\rule[-0.200pt]{2.409pt}{0.400pt}}
\put(1429.0,855.0){\rule[-0.200pt]{2.409pt}{0.400pt}}
\put(170.0,855.0){\rule[-0.200pt]{2.409pt}{0.400pt}}
\put(1429.0,855.0){\rule[-0.200pt]{2.409pt}{0.400pt}}
\put(170.0,855.0){\rule[-0.200pt]{2.409pt}{0.400pt}}
\put(1429.0,855.0){\rule[-0.200pt]{2.409pt}{0.400pt}}
\put(170.0,856.0){\rule[-0.200pt]{2.409pt}{0.400pt}}
\put(1429.0,856.0){\rule[-0.200pt]{2.409pt}{0.400pt}}
\put(170.0,856.0){\rule[-0.200pt]{2.409pt}{0.400pt}}
\put(1429.0,856.0){\rule[-0.200pt]{2.409pt}{0.400pt}}
\put(170.0,856.0){\rule[-0.200pt]{2.409pt}{0.400pt}}
\put(1429.0,856.0){\rule[-0.200pt]{2.409pt}{0.400pt}}
\put(170.0,856.0){\rule[-0.200pt]{2.409pt}{0.400pt}}
\put(1429.0,856.0){\rule[-0.200pt]{2.409pt}{0.400pt}}
\put(170.0,856.0){\rule[-0.200pt]{2.409pt}{0.400pt}}
\put(1429.0,856.0){\rule[-0.200pt]{2.409pt}{0.400pt}}
\put(170.0,856.0){\rule[-0.200pt]{2.409pt}{0.400pt}}
\put(1429.0,856.0){\rule[-0.200pt]{2.409pt}{0.400pt}}
\put(170.0,856.0){\rule[-0.200pt]{2.409pt}{0.400pt}}
\put(1429.0,856.0){\rule[-0.200pt]{2.409pt}{0.400pt}}
\put(170.0,856.0){\rule[-0.200pt]{2.409pt}{0.400pt}}
\put(1429.0,856.0){\rule[-0.200pt]{2.409pt}{0.400pt}}
\put(170.0,856.0){\rule[-0.200pt]{2.409pt}{0.400pt}}
\put(1429.0,856.0){\rule[-0.200pt]{2.409pt}{0.400pt}}
\put(170.0,856.0){\rule[-0.200pt]{2.409pt}{0.400pt}}
\put(1429.0,856.0){\rule[-0.200pt]{2.409pt}{0.400pt}}
\put(170.0,856.0){\rule[-0.200pt]{2.409pt}{0.400pt}}
\put(1429.0,856.0){\rule[-0.200pt]{2.409pt}{0.400pt}}
\put(170.0,856.0){\rule[-0.200pt]{2.409pt}{0.400pt}}
\put(1429.0,856.0){\rule[-0.200pt]{2.409pt}{0.400pt}}
\put(170.0,856.0){\rule[-0.200pt]{2.409pt}{0.400pt}}
\put(1429.0,856.0){\rule[-0.200pt]{2.409pt}{0.400pt}}
\put(170.0,856.0){\rule[-0.200pt]{2.409pt}{0.400pt}}
\put(1429.0,856.0){\rule[-0.200pt]{2.409pt}{0.400pt}}
\put(170.0,856.0){\rule[-0.200pt]{2.409pt}{0.400pt}}
\put(1429.0,856.0){\rule[-0.200pt]{2.409pt}{0.400pt}}
\put(170.0,856.0){\rule[-0.200pt]{2.409pt}{0.400pt}}
\put(1429.0,856.0){\rule[-0.200pt]{2.409pt}{0.400pt}}
\put(170.0,856.0){\rule[-0.200pt]{2.409pt}{0.400pt}}
\put(1429.0,856.0){\rule[-0.200pt]{2.409pt}{0.400pt}}
\put(170.0,856.0){\rule[-0.200pt]{2.409pt}{0.400pt}}
\put(1429.0,856.0){\rule[-0.200pt]{2.409pt}{0.400pt}}
\put(170.0,856.0){\rule[-0.200pt]{2.409pt}{0.400pt}}
\put(1429.0,856.0){\rule[-0.200pt]{2.409pt}{0.400pt}}
\put(170.0,856.0){\rule[-0.200pt]{2.409pt}{0.400pt}}
\put(1429.0,856.0){\rule[-0.200pt]{2.409pt}{0.400pt}}
\put(170.0,856.0){\rule[-0.200pt]{2.409pt}{0.400pt}}
\put(1429.0,856.0){\rule[-0.200pt]{2.409pt}{0.400pt}}
\put(170.0,856.0){\rule[-0.200pt]{2.409pt}{0.400pt}}
\put(1429.0,856.0){\rule[-0.200pt]{2.409pt}{0.400pt}}
\put(170.0,856.0){\rule[-0.200pt]{2.409pt}{0.400pt}}
\put(1429.0,856.0){\rule[-0.200pt]{2.409pt}{0.400pt}}
\put(170.0,856.0){\rule[-0.200pt]{2.409pt}{0.400pt}}
\put(1429.0,856.0){\rule[-0.200pt]{2.409pt}{0.400pt}}
\put(170.0,856.0){\rule[-0.200pt]{2.409pt}{0.400pt}}
\put(1429.0,856.0){\rule[-0.200pt]{2.409pt}{0.400pt}}
\put(170.0,857.0){\rule[-0.200pt]{2.409pt}{0.400pt}}
\put(1429.0,857.0){\rule[-0.200pt]{2.409pt}{0.400pt}}
\put(170.0,857.0){\rule[-0.200pt]{2.409pt}{0.400pt}}
\put(1429.0,857.0){\rule[-0.200pt]{2.409pt}{0.400pt}}
\put(170.0,857.0){\rule[-0.200pt]{2.409pt}{0.400pt}}
\put(1429.0,857.0){\rule[-0.200pt]{2.409pt}{0.400pt}}
\put(170.0,857.0){\rule[-0.200pt]{2.409pt}{0.400pt}}
\put(1429.0,857.0){\rule[-0.200pt]{2.409pt}{0.400pt}}
\put(170.0,857.0){\rule[-0.200pt]{2.409pt}{0.400pt}}
\put(1429.0,857.0){\rule[-0.200pt]{2.409pt}{0.400pt}}
\put(170.0,857.0){\rule[-0.200pt]{2.409pt}{0.400pt}}
\put(1429.0,857.0){\rule[-0.200pt]{2.409pt}{0.400pt}}
\put(170.0,857.0){\rule[-0.200pt]{2.409pt}{0.400pt}}
\put(1429.0,857.0){\rule[-0.200pt]{2.409pt}{0.400pt}}
\put(170.0,857.0){\rule[-0.200pt]{2.409pt}{0.400pt}}
\put(1429.0,857.0){\rule[-0.200pt]{2.409pt}{0.400pt}}
\put(170.0,857.0){\rule[-0.200pt]{2.409pt}{0.400pt}}
\put(1429.0,857.0){\rule[-0.200pt]{2.409pt}{0.400pt}}
\put(170.0,857.0){\rule[-0.200pt]{2.409pt}{0.400pt}}
\put(1429.0,857.0){\rule[-0.200pt]{2.409pt}{0.400pt}}
\put(170.0,857.0){\rule[-0.200pt]{2.409pt}{0.400pt}}
\put(1429.0,857.0){\rule[-0.200pt]{2.409pt}{0.400pt}}
\put(170.0,857.0){\rule[-0.200pt]{2.409pt}{0.400pt}}
\put(1429.0,857.0){\rule[-0.200pt]{2.409pt}{0.400pt}}
\put(170.0,857.0){\rule[-0.200pt]{2.409pt}{0.400pt}}
\put(1429.0,857.0){\rule[-0.200pt]{2.409pt}{0.400pt}}
\put(170.0,857.0){\rule[-0.200pt]{2.409pt}{0.400pt}}
\put(1429.0,857.0){\rule[-0.200pt]{2.409pt}{0.400pt}}
\put(170.0,857.0){\rule[-0.200pt]{2.409pt}{0.400pt}}
\put(1429.0,857.0){\rule[-0.200pt]{2.409pt}{0.400pt}}
\put(170.0,857.0){\rule[-0.200pt]{2.409pt}{0.400pt}}
\put(1429.0,857.0){\rule[-0.200pt]{2.409pt}{0.400pt}}
\put(170.0,857.0){\rule[-0.200pt]{2.409pt}{0.400pt}}
\put(1429.0,857.0){\rule[-0.200pt]{2.409pt}{0.400pt}}
\put(170.0,857.0){\rule[-0.200pt]{2.409pt}{0.400pt}}
\put(1429.0,857.0){\rule[-0.200pt]{2.409pt}{0.400pt}}
\put(170.0,857.0){\rule[-0.200pt]{2.409pt}{0.400pt}}
\put(1429.0,857.0){\rule[-0.200pt]{2.409pt}{0.400pt}}
\put(170.0,857.0){\rule[-0.200pt]{2.409pt}{0.400pt}}
\put(1429.0,857.0){\rule[-0.200pt]{2.409pt}{0.400pt}}
\put(170.0,857.0){\rule[-0.200pt]{2.409pt}{0.400pt}}
\put(1429.0,857.0){\rule[-0.200pt]{2.409pt}{0.400pt}}
\put(170.0,857.0){\rule[-0.200pt]{2.409pt}{0.400pt}}
\put(1429.0,857.0){\rule[-0.200pt]{2.409pt}{0.400pt}}
\put(170.0,857.0){\rule[-0.200pt]{2.409pt}{0.400pt}}
\put(1429.0,857.0){\rule[-0.200pt]{2.409pt}{0.400pt}}
\put(170.0,857.0){\rule[-0.200pt]{2.409pt}{0.400pt}}
\put(1429.0,857.0){\rule[-0.200pt]{2.409pt}{0.400pt}}
\put(170.0,857.0){\rule[-0.200pt]{2.409pt}{0.400pt}}
\put(1429.0,857.0){\rule[-0.200pt]{2.409pt}{0.400pt}}
\put(170.0,858.0){\rule[-0.200pt]{2.409pt}{0.400pt}}
\put(1429.0,858.0){\rule[-0.200pt]{2.409pt}{0.400pt}}
\put(170.0,858.0){\rule[-0.200pt]{2.409pt}{0.400pt}}
\put(1429.0,858.0){\rule[-0.200pt]{2.409pt}{0.400pt}}
\put(170.0,858.0){\rule[-0.200pt]{2.409pt}{0.400pt}}
\put(1429.0,858.0){\rule[-0.200pt]{2.409pt}{0.400pt}}
\put(170.0,858.0){\rule[-0.200pt]{2.409pt}{0.400pt}}
\put(1429.0,858.0){\rule[-0.200pt]{2.409pt}{0.400pt}}
\put(170.0,858.0){\rule[-0.200pt]{2.409pt}{0.400pt}}
\put(1429.0,858.0){\rule[-0.200pt]{2.409pt}{0.400pt}}
\put(170.0,858.0){\rule[-0.200pt]{2.409pt}{0.400pt}}
\put(1429.0,858.0){\rule[-0.200pt]{2.409pt}{0.400pt}}
\put(170.0,858.0){\rule[-0.200pt]{2.409pt}{0.400pt}}
\put(1429.0,858.0){\rule[-0.200pt]{2.409pt}{0.400pt}}
\put(170.0,858.0){\rule[-0.200pt]{2.409pt}{0.400pt}}
\put(1429.0,858.0){\rule[-0.200pt]{2.409pt}{0.400pt}}
\put(170.0,858.0){\rule[-0.200pt]{2.409pt}{0.400pt}}
\put(1429.0,858.0){\rule[-0.200pt]{2.409pt}{0.400pt}}
\put(170.0,858.0){\rule[-0.200pt]{2.409pt}{0.400pt}}
\put(1429.0,858.0){\rule[-0.200pt]{2.409pt}{0.400pt}}
\put(170.0,858.0){\rule[-0.200pt]{2.409pt}{0.400pt}}
\put(1429.0,858.0){\rule[-0.200pt]{2.409pt}{0.400pt}}
\put(170.0,858.0){\rule[-0.200pt]{2.409pt}{0.400pt}}
\put(1429.0,858.0){\rule[-0.200pt]{2.409pt}{0.400pt}}
\put(170.0,858.0){\rule[-0.200pt]{2.409pt}{0.400pt}}
\put(1429.0,858.0){\rule[-0.200pt]{2.409pt}{0.400pt}}
\put(170.0,858.0){\rule[-0.200pt]{2.409pt}{0.400pt}}
\put(1429.0,858.0){\rule[-0.200pt]{2.409pt}{0.400pt}}
\put(170.0,858.0){\rule[-0.200pt]{2.409pt}{0.400pt}}
\put(1429.0,858.0){\rule[-0.200pt]{2.409pt}{0.400pt}}
\put(170.0,858.0){\rule[-0.200pt]{2.409pt}{0.400pt}}
\put(1429.0,858.0){\rule[-0.200pt]{2.409pt}{0.400pt}}
\put(170.0,858.0){\rule[-0.200pt]{2.409pt}{0.400pt}}
\put(1429.0,858.0){\rule[-0.200pt]{2.409pt}{0.400pt}}
\put(170.0,858.0){\rule[-0.200pt]{2.409pt}{0.400pt}}
\put(1429.0,858.0){\rule[-0.200pt]{2.409pt}{0.400pt}}
\put(170.0,858.0){\rule[-0.200pt]{2.409pt}{0.400pt}}
\put(1429.0,858.0){\rule[-0.200pt]{2.409pt}{0.400pt}}
\put(170.0,858.0){\rule[-0.200pt]{2.409pt}{0.400pt}}
\put(1429.0,858.0){\rule[-0.200pt]{2.409pt}{0.400pt}}
\put(170.0,858.0){\rule[-0.200pt]{2.409pt}{0.400pt}}
\put(1429.0,858.0){\rule[-0.200pt]{2.409pt}{0.400pt}}
\put(170.0,858.0){\rule[-0.200pt]{2.409pt}{0.400pt}}
\put(1429.0,858.0){\rule[-0.200pt]{2.409pt}{0.400pt}}
\put(170.0,858.0){\rule[-0.200pt]{2.409pt}{0.400pt}}
\put(1429.0,858.0){\rule[-0.200pt]{2.409pt}{0.400pt}}
\put(170.0,858.0){\rule[-0.200pt]{2.409pt}{0.400pt}}
\put(1429.0,858.0){\rule[-0.200pt]{2.409pt}{0.400pt}}
\put(170.0,858.0){\rule[-0.200pt]{2.409pt}{0.400pt}}
\put(1429.0,858.0){\rule[-0.200pt]{2.409pt}{0.400pt}}
\put(170.0,858.0){\rule[-0.200pt]{2.409pt}{0.400pt}}
\put(1429.0,858.0){\rule[-0.200pt]{2.409pt}{0.400pt}}
\put(170.0,859.0){\rule[-0.200pt]{2.409pt}{0.400pt}}
\put(1429.0,859.0){\rule[-0.200pt]{2.409pt}{0.400pt}}
\put(170.0,859.0){\rule[-0.200pt]{2.409pt}{0.400pt}}
\put(1429.0,859.0){\rule[-0.200pt]{2.409pt}{0.400pt}}
\put(170.0,859.0){\rule[-0.200pt]{2.409pt}{0.400pt}}
\put(1429.0,859.0){\rule[-0.200pt]{2.409pt}{0.400pt}}
\put(170.0,859.0){\rule[-0.200pt]{2.409pt}{0.400pt}}
\put(1429.0,859.0){\rule[-0.200pt]{2.409pt}{0.400pt}}
\put(170.0,859.0){\rule[-0.200pt]{2.409pt}{0.400pt}}
\put(1429.0,859.0){\rule[-0.200pt]{2.409pt}{0.400pt}}
\put(170.0,859.0){\rule[-0.200pt]{2.409pt}{0.400pt}}
\put(1429.0,859.0){\rule[-0.200pt]{2.409pt}{0.400pt}}
\put(170.0,859.0){\rule[-0.200pt]{2.409pt}{0.400pt}}
\put(1429.0,859.0){\rule[-0.200pt]{2.409pt}{0.400pt}}
\put(170.0,859.0){\rule[-0.200pt]{2.409pt}{0.400pt}}
\put(1429.0,859.0){\rule[-0.200pt]{2.409pt}{0.400pt}}
\put(170.0,859.0){\rule[-0.200pt]{2.409pt}{0.400pt}}
\put(1429.0,859.0){\rule[-0.200pt]{2.409pt}{0.400pt}}
\put(170.0,859.0){\rule[-0.200pt]{2.409pt}{0.400pt}}
\put(1429.0,859.0){\rule[-0.200pt]{2.409pt}{0.400pt}}
\put(170.0,859.0){\rule[-0.200pt]{2.409pt}{0.400pt}}
\put(1429.0,859.0){\rule[-0.200pt]{2.409pt}{0.400pt}}
\put(170.0,859.0){\rule[-0.200pt]{2.409pt}{0.400pt}}
\put(1429.0,859.0){\rule[-0.200pt]{2.409pt}{0.400pt}}
\put(170.0,859.0){\rule[-0.200pt]{2.409pt}{0.400pt}}
\put(1429.0,859.0){\rule[-0.200pt]{2.409pt}{0.400pt}}
\put(170.0,859.0){\rule[-0.200pt]{4.818pt}{0.400pt}}
\put(150,859){\makebox(0,0)[r]{ 1e+09}}
\put(1419.0,859.0){\rule[-0.200pt]{4.818pt}{0.400pt}}
\put(170.0,82.0){\rule[-0.200pt]{0.400pt}{4.818pt}}
\put(170,41){\makebox(0,0){ 0}}
\put(170.0,839.0){\rule[-0.200pt]{0.400pt}{4.818pt}}
\put(424.0,82.0){\rule[-0.200pt]{0.400pt}{4.818pt}}
\put(424,41){\makebox(0,0){ 100}}
\put(424.0,839.0){\rule[-0.200pt]{0.400pt}{4.818pt}}
\put(678.0,82.0){\rule[-0.200pt]{0.400pt}{4.818pt}}
\put(678,41){\makebox(0,0){ 200}}
\put(678.0,839.0){\rule[-0.200pt]{0.400pt}{4.818pt}}
\put(931.0,82.0){\rule[-0.200pt]{0.400pt}{4.818pt}}
\put(931,41){\makebox(0,0){ 300}}
\put(931.0,839.0){\rule[-0.200pt]{0.400pt}{4.818pt}}
\put(1185.0,82.0){\rule[-0.200pt]{0.400pt}{4.818pt}}
\put(1185,41){\makebox(0,0){ 400}}
\put(1185.0,839.0){\rule[-0.200pt]{0.400pt}{4.818pt}}
\put(1439.0,82.0){\rule[-0.200pt]{0.400pt}{4.818pt}}
\put(1439,41){\makebox(0,0){ 500}}
\put(1439.0,839.0){\rule[-0.200pt]{0.400pt}{4.818pt}}
\put(170.0,82.0){\rule[-0.200pt]{0.400pt}{187.179pt}}
\put(170.0,82.0){\rule[-0.200pt]{305.702pt}{0.400pt}}
\put(1439.0,82.0){\rule[-0.200pt]{0.400pt}{187.179pt}}
\put(170.0,859.0){\rule[-0.200pt]{305.702pt}{0.400pt}}
\put(1279,164){\makebox(0,0)[r]{algorytm naturalny}}
\put(1299.0,164.0){\rule[-0.200pt]{24.090pt}{0.400pt}}
\put(175,82){\usebox{\plotpoint}}
\multiput(175.58,82.00)(0.493,1.091){23}{\rule{0.119pt}{0.962pt}}
\multiput(174.17,82.00)(13.000,26.004){2}{\rule{0.400pt}{0.481pt}}
\multiput(188.58,110.00)(0.493,0.853){23}{\rule{0.119pt}{0.777pt}}
\multiput(187.17,110.00)(13.000,20.387){2}{\rule{0.400pt}{0.388pt}}
\multiput(201.58,132.00)(0.493,0.734){23}{\rule{0.119pt}{0.685pt}}
\multiput(200.17,132.00)(13.000,17.579){2}{\rule{0.400pt}{0.342pt}}
\multiput(214.58,151.00)(0.493,0.734){23}{\rule{0.119pt}{0.685pt}}
\multiput(213.17,151.00)(13.000,17.579){2}{\rule{0.400pt}{0.342pt}}
\multiput(227.58,170.00)(0.493,0.734){23}{\rule{0.119pt}{0.685pt}}
\multiput(226.17,170.00)(13.000,17.579){2}{\rule{0.400pt}{0.342pt}}
\multiput(240.58,189.00)(0.493,0.695){23}{\rule{0.119pt}{0.654pt}}
\multiput(239.17,189.00)(13.000,16.643){2}{\rule{0.400pt}{0.327pt}}
\multiput(253.58,207.00)(0.493,0.655){23}{\rule{0.119pt}{0.623pt}}
\multiput(252.17,207.00)(13.000,15.707){2}{\rule{0.400pt}{0.312pt}}
\multiput(266.58,224.00)(0.493,0.536){23}{\rule{0.119pt}{0.531pt}}
\multiput(265.17,224.00)(13.000,12.898){2}{\rule{0.400pt}{0.265pt}}
\multiput(279.00,238.58)(0.539,0.492){21}{\rule{0.533pt}{0.119pt}}
\multiput(279.00,237.17)(11.893,12.000){2}{\rule{0.267pt}{0.400pt}}
\multiput(292.00,250.58)(0.539,0.492){21}{\rule{0.533pt}{0.119pt}}
\multiput(292.00,249.17)(11.893,12.000){2}{\rule{0.267pt}{0.400pt}}
\multiput(305.00,262.58)(0.652,0.491){17}{\rule{0.620pt}{0.118pt}}
\multiput(305.00,261.17)(11.713,10.000){2}{\rule{0.310pt}{0.400pt}}
\multiput(318.00,272.59)(0.728,0.489){15}{\rule{0.678pt}{0.118pt}}
\multiput(318.00,271.17)(11.593,9.000){2}{\rule{0.339pt}{0.400pt}}
\multiput(331.00,281.59)(0.728,0.489){15}{\rule{0.678pt}{0.118pt}}
\multiput(331.00,280.17)(11.593,9.000){2}{\rule{0.339pt}{0.400pt}}
\multiput(344.00,290.59)(0.824,0.488){13}{\rule{0.750pt}{0.117pt}}
\multiput(344.00,289.17)(11.443,8.000){2}{\rule{0.375pt}{0.400pt}}
\multiput(357.00,298.59)(0.824,0.488){13}{\rule{0.750pt}{0.117pt}}
\multiput(357.00,297.17)(11.443,8.000){2}{\rule{0.375pt}{0.400pt}}
\multiput(370.00,306.59)(0.950,0.485){11}{\rule{0.843pt}{0.117pt}}
\multiput(370.00,305.17)(11.251,7.000){2}{\rule{0.421pt}{0.400pt}}
\multiput(383.00,313.59)(1.123,0.482){9}{\rule{0.967pt}{0.116pt}}
\multiput(383.00,312.17)(10.994,6.000){2}{\rule{0.483pt}{0.400pt}}
\multiput(396.00,319.59)(0.950,0.485){11}{\rule{0.843pt}{0.117pt}}
\multiput(396.00,318.17)(11.251,7.000){2}{\rule{0.421pt}{0.400pt}}
\multiput(409.00,326.59)(1.214,0.482){9}{\rule{1.033pt}{0.116pt}}
\multiput(409.00,325.17)(11.855,6.000){2}{\rule{0.517pt}{0.400pt}}
\multiput(423.00,332.59)(1.378,0.477){7}{\rule{1.140pt}{0.115pt}}
\multiput(423.00,331.17)(10.634,5.000){2}{\rule{0.570pt}{0.400pt}}
\multiput(436.00,337.59)(1.123,0.482){9}{\rule{0.967pt}{0.116pt}}
\multiput(436.00,336.17)(10.994,6.000){2}{\rule{0.483pt}{0.400pt}}
\multiput(449.00,343.59)(1.378,0.477){7}{\rule{1.140pt}{0.115pt}}
\multiput(449.00,342.17)(10.634,5.000){2}{\rule{0.570pt}{0.400pt}}
\multiput(462.00,348.59)(1.378,0.477){7}{\rule{1.140pt}{0.115pt}}
\multiput(462.00,347.17)(10.634,5.000){2}{\rule{0.570pt}{0.400pt}}
\multiput(475.00,353.59)(1.378,0.477){7}{\rule{1.140pt}{0.115pt}}
\multiput(475.00,352.17)(10.634,5.000){2}{\rule{0.570pt}{0.400pt}}
\multiput(488.00,358.60)(1.797,0.468){5}{\rule{1.400pt}{0.113pt}}
\multiput(488.00,357.17)(10.094,4.000){2}{\rule{0.700pt}{0.400pt}}
\multiput(501.00,362.59)(1.378,0.477){7}{\rule{1.140pt}{0.115pt}}
\multiput(501.00,361.17)(10.634,5.000){2}{\rule{0.570pt}{0.400pt}}
\multiput(514.00,367.60)(1.797,0.468){5}{\rule{1.400pt}{0.113pt}}
\multiput(514.00,366.17)(10.094,4.000){2}{\rule{0.700pt}{0.400pt}}
\multiput(527.00,371.60)(1.797,0.468){5}{\rule{1.400pt}{0.113pt}}
\multiput(527.00,370.17)(10.094,4.000){2}{\rule{0.700pt}{0.400pt}}
\multiput(540.00,375.60)(1.797,0.468){5}{\rule{1.400pt}{0.113pt}}
\multiput(540.00,374.17)(10.094,4.000){2}{\rule{0.700pt}{0.400pt}}
\multiput(553.00,379.60)(1.797,0.468){5}{\rule{1.400pt}{0.113pt}}
\multiput(553.00,378.17)(10.094,4.000){2}{\rule{0.700pt}{0.400pt}}
\multiput(566.00,383.61)(2.695,0.447){3}{\rule{1.833pt}{0.108pt}}
\multiput(566.00,382.17)(9.195,3.000){2}{\rule{0.917pt}{0.400pt}}
\multiput(579.00,386.60)(1.797,0.468){5}{\rule{1.400pt}{0.113pt}}
\multiput(579.00,385.17)(10.094,4.000){2}{\rule{0.700pt}{0.400pt}}
\multiput(592.00,390.61)(2.695,0.447){3}{\rule{1.833pt}{0.108pt}}
\multiput(592.00,389.17)(9.195,3.000){2}{\rule{0.917pt}{0.400pt}}
\multiput(605.00,393.60)(1.797,0.468){5}{\rule{1.400pt}{0.113pt}}
\multiput(605.00,392.17)(10.094,4.000){2}{\rule{0.700pt}{0.400pt}}
\multiput(618.00,397.61)(2.695,0.447){3}{\rule{1.833pt}{0.108pt}}
\multiput(618.00,396.17)(9.195,3.000){2}{\rule{0.917pt}{0.400pt}}
\multiput(631.00,400.61)(2.695,0.447){3}{\rule{1.833pt}{0.108pt}}
\multiput(631.00,399.17)(9.195,3.000){2}{\rule{0.917pt}{0.400pt}}
\multiput(644.00,403.61)(2.695,0.447){3}{\rule{1.833pt}{0.108pt}}
\multiput(644.00,402.17)(9.195,3.000){2}{\rule{0.917pt}{0.400pt}}
\multiput(657.00,406.61)(2.695,0.447){3}{\rule{1.833pt}{0.108pt}}
\multiput(657.00,405.17)(9.195,3.000){2}{\rule{0.917pt}{0.400pt}}
\multiput(670.00,409.61)(2.695,0.447){3}{\rule{1.833pt}{0.108pt}}
\multiput(670.00,408.17)(9.195,3.000){2}{\rule{0.917pt}{0.400pt}}
\multiput(683.00,412.61)(2.695,0.447){3}{\rule{1.833pt}{0.108pt}}
\multiput(683.00,411.17)(9.195,3.000){2}{\rule{0.917pt}{0.400pt}}
\multiput(696.00,415.61)(2.695,0.447){3}{\rule{1.833pt}{0.108pt}}
\multiput(696.00,414.17)(9.195,3.000){2}{\rule{0.917pt}{0.400pt}}
\multiput(709.00,418.61)(2.695,0.447){3}{\rule{1.833pt}{0.108pt}}
\multiput(709.00,417.17)(9.195,3.000){2}{\rule{0.917pt}{0.400pt}}
\put(722,421.17){\rule{2.700pt}{0.400pt}}
\multiput(722.00,420.17)(7.396,2.000){2}{\rule{1.350pt}{0.400pt}}
\multiput(735.00,423.61)(2.695,0.447){3}{\rule{1.833pt}{0.108pt}}
\multiput(735.00,422.17)(9.195,3.000){2}{\rule{0.917pt}{0.400pt}}
\multiput(748.00,426.61)(2.695,0.447){3}{\rule{1.833pt}{0.108pt}}
\multiput(748.00,425.17)(9.195,3.000){2}{\rule{0.917pt}{0.400pt}}
\put(761,429.17){\rule{2.700pt}{0.400pt}}
\multiput(761.00,428.17)(7.396,2.000){2}{\rule{1.350pt}{0.400pt}}
\multiput(774.00,431.61)(2.695,0.447){3}{\rule{1.833pt}{0.108pt}}
\multiput(774.00,430.17)(9.195,3.000){2}{\rule{0.917pt}{0.400pt}}
\put(787,434.17){\rule{2.700pt}{0.400pt}}
\multiput(787.00,433.17)(7.396,2.000){2}{\rule{1.350pt}{0.400pt}}
\multiput(800.00,436.61)(2.695,0.447){3}{\rule{1.833pt}{0.108pt}}
\multiput(800.00,435.17)(9.195,3.000){2}{\rule{0.917pt}{0.400pt}}
\put(813,439.17){\rule{2.700pt}{0.400pt}}
\multiput(813.00,438.17)(7.396,2.000){2}{\rule{1.350pt}{0.400pt}}
\multiput(826.00,441.61)(2.695,0.447){3}{\rule{1.833pt}{0.108pt}}
\multiput(826.00,440.17)(9.195,3.000){2}{\rule{0.917pt}{0.400pt}}
\put(839,444.17){\rule{2.700pt}{0.400pt}}
\multiput(839.00,443.17)(7.396,2.000){2}{\rule{1.350pt}{0.400pt}}
\put(852,446.17){\rule{2.700pt}{0.400pt}}
\multiput(852.00,445.17)(7.396,2.000){2}{\rule{1.350pt}{0.400pt}}
\multiput(865.00,448.61)(2.695,0.447){3}{\rule{1.833pt}{0.108pt}}
\multiput(865.00,447.17)(9.195,3.000){2}{\rule{0.917pt}{0.400pt}}
\put(878,451.17){\rule{2.700pt}{0.400pt}}
\multiput(878.00,450.17)(7.396,2.000){2}{\rule{1.350pt}{0.400pt}}
\put(891,453.17){\rule{2.700pt}{0.400pt}}
\multiput(891.00,452.17)(7.396,2.000){2}{\rule{1.350pt}{0.400pt}}
\put(904,455.17){\rule{2.700pt}{0.400pt}}
\multiput(904.00,454.17)(7.396,2.000){2}{\rule{1.350pt}{0.400pt}}
\put(917,457.17){\rule{2.700pt}{0.400pt}}
\multiput(917.00,456.17)(7.396,2.000){2}{\rule{1.350pt}{0.400pt}}
\put(930,459.17){\rule{2.700pt}{0.400pt}}
\multiput(930.00,458.17)(7.396,2.000){2}{\rule{1.350pt}{0.400pt}}
\put(943,461.17){\rule{2.700pt}{0.400pt}}
\multiput(943.00,460.17)(7.396,2.000){2}{\rule{1.350pt}{0.400pt}}
\put(956,463.17){\rule{2.700pt}{0.400pt}}
\multiput(956.00,462.17)(7.396,2.000){2}{\rule{1.350pt}{0.400pt}}
\put(969,465.17){\rule{2.900pt}{0.400pt}}
\multiput(969.00,464.17)(7.981,2.000){2}{\rule{1.450pt}{0.400pt}}
\put(983,467.17){\rule{2.700pt}{0.400pt}}
\multiput(983.00,466.17)(7.396,2.000){2}{\rule{1.350pt}{0.400pt}}
\put(996,469.17){\rule{2.700pt}{0.400pt}}
\multiput(996.00,468.17)(7.396,2.000){2}{\rule{1.350pt}{0.400pt}}
\put(1009,471.17){\rule{2.700pt}{0.400pt}}
\multiput(1009.00,470.17)(7.396,2.000){2}{\rule{1.350pt}{0.400pt}}
\put(1022,473.17){\rule{2.700pt}{0.400pt}}
\multiput(1022.00,472.17)(7.396,2.000){2}{\rule{1.350pt}{0.400pt}}
\put(1035,475.17){\rule{2.700pt}{0.400pt}}
\multiput(1035.00,474.17)(7.396,2.000){2}{\rule{1.350pt}{0.400pt}}
\put(1048,476.67){\rule{3.132pt}{0.400pt}}
\multiput(1048.00,476.17)(6.500,1.000){2}{\rule{1.566pt}{0.400pt}}
\put(1061,478.17){\rule{2.700pt}{0.400pt}}
\multiput(1061.00,477.17)(7.396,2.000){2}{\rule{1.350pt}{0.400pt}}
\put(1074,480.17){\rule{2.700pt}{0.400pt}}
\multiput(1074.00,479.17)(7.396,2.000){2}{\rule{1.350pt}{0.400pt}}
\put(1087,481.67){\rule{3.132pt}{0.400pt}}
\multiput(1087.00,481.17)(6.500,1.000){2}{\rule{1.566pt}{0.400pt}}
\put(1100,483.17){\rule{2.700pt}{0.400pt}}
\multiput(1100.00,482.17)(7.396,2.000){2}{\rule{1.350pt}{0.400pt}}
\put(1113,485.17){\rule{2.700pt}{0.400pt}}
\multiput(1113.00,484.17)(7.396,2.000){2}{\rule{1.350pt}{0.400pt}}
\put(1126,486.67){\rule{3.132pt}{0.400pt}}
\multiput(1126.00,486.17)(6.500,1.000){2}{\rule{1.566pt}{0.400pt}}
\put(1139,488.17){\rule{2.700pt}{0.400pt}}
\multiput(1139.00,487.17)(7.396,2.000){2}{\rule{1.350pt}{0.400pt}}
\put(1152,490.17){\rule{2.700pt}{0.400pt}}
\multiput(1152.00,489.17)(7.396,2.000){2}{\rule{1.350pt}{0.400pt}}
\put(1165,491.67){\rule{3.132pt}{0.400pt}}
\multiput(1165.00,491.17)(6.500,1.000){2}{\rule{1.566pt}{0.400pt}}
\put(1178,493.17){\rule{2.700pt}{0.400pt}}
\multiput(1178.00,492.17)(7.396,2.000){2}{\rule{1.350pt}{0.400pt}}
\put(1191,494.67){\rule{3.132pt}{0.400pt}}
\multiput(1191.00,494.17)(6.500,1.000){2}{\rule{1.566pt}{0.400pt}}
\put(1204,496.17){\rule{2.700pt}{0.400pt}}
\multiput(1204.00,495.17)(7.396,2.000){2}{\rule{1.350pt}{0.400pt}}
\put(1217,497.67){\rule{3.132pt}{0.400pt}}
\multiput(1217.00,497.17)(6.500,1.000){2}{\rule{1.566pt}{0.400pt}}
\put(1230,498.67){\rule{3.132pt}{0.400pt}}
\multiput(1230.00,498.17)(6.500,1.000){2}{\rule{1.566pt}{0.400pt}}
\put(1243,500.17){\rule{2.700pt}{0.400pt}}
\multiput(1243.00,499.17)(7.396,2.000){2}{\rule{1.350pt}{0.400pt}}
\put(1256,501.67){\rule{3.132pt}{0.400pt}}
\multiput(1256.00,501.17)(6.500,1.000){2}{\rule{1.566pt}{0.400pt}}
\put(1269,502.67){\rule{3.132pt}{0.400pt}}
\multiput(1269.00,502.17)(6.500,1.000){2}{\rule{1.566pt}{0.400pt}}
\put(1282,504.17){\rule{2.700pt}{0.400pt}}
\multiput(1282.00,503.17)(7.396,2.000){2}{\rule{1.350pt}{0.400pt}}
\put(1295,505.67){\rule{3.132pt}{0.400pt}}
\multiput(1295.00,505.17)(6.500,1.000){2}{\rule{1.566pt}{0.400pt}}
\put(1308,507.17){\rule{2.700pt}{0.400pt}}
\multiput(1308.00,506.17)(7.396,2.000){2}{\rule{1.350pt}{0.400pt}}
\put(1321,508.67){\rule{3.132pt}{0.400pt}}
\multiput(1321.00,508.17)(6.500,1.000){2}{\rule{1.566pt}{0.400pt}}
\put(1334,509.67){\rule{3.132pt}{0.400pt}}
\multiput(1334.00,509.17)(6.500,1.000){2}{\rule{1.566pt}{0.400pt}}
\put(1347,511.17){\rule{2.700pt}{0.400pt}}
\multiput(1347.00,510.17)(7.396,2.000){2}{\rule{1.350pt}{0.400pt}}
\put(1360,513.17){\rule{2.700pt}{0.400pt}}
\multiput(1360.00,512.17)(7.396,2.000){2}{\rule{1.350pt}{0.400pt}}
\put(1373,514.67){\rule{3.132pt}{0.400pt}}
\multiput(1373.00,514.17)(6.500,1.000){2}{\rule{1.566pt}{0.400pt}}
\put(1386,515.67){\rule{3.132pt}{0.400pt}}
\multiput(1386.00,515.17)(6.500,1.000){2}{\rule{1.566pt}{0.400pt}}
\put(1399,516.67){\rule{3.132pt}{0.400pt}}
\multiput(1399.00,516.17)(6.500,1.000){2}{\rule{1.566pt}{0.400pt}}
\put(1412,518.17){\rule{2.700pt}{0.400pt}}
\multiput(1412.00,517.17)(7.396,2.000){2}{\rule{1.350pt}{0.400pt}}
\put(1425,519.67){\rule{3.132pt}{0.400pt}}
\multiput(1425.00,519.17)(6.500,1.000){2}{\rule{1.566pt}{0.400pt}}
\put(1438.0,521.0){\usebox{\plotpoint}}
\sbox{\plotpoint}{\rule[-0.500pt]{1.000pt}{1.000pt}}%
\sbox{\plotpoint}{\rule[-0.200pt]{0.400pt}{0.400pt}}%
\put(1279,123){\makebox(0,0)[r]{algorytm Strassena}}
\sbox{\plotpoint}{\rule[-0.500pt]{1.000pt}{1.000pt}}%
\multiput(1299,123)(20.756,0.000){5}{\usebox{\plotpoint}}
\put(1399,123){\usebox{\plotpoint}}
\put(175,236){\usebox{\plotpoint}}
\multiput(175,236)(2.235,20.635){6}{\usebox{\plotpoint}}
\multiput(188,356)(4.466,20.269){3}{\usebox{\plotpoint}}
\multiput(201,415)(7.413,19.387){2}{\usebox{\plotpoint}}
\put(218.83,458.65){\usebox{\plotpoint}}
\multiput(227,475)(10.925,17.648){2}{\usebox{\plotpoint}}
\put(250.89,511.92){\usebox{\plotpoint}}
\put(263.74,528.21){\usebox{\plotpoint}}
\put(278.14,543.14){\usebox{\plotpoint}}
\put(293.99,556.53){\usebox{\plotpoint}}
\put(310.04,569.65){\usebox{\plotpoint}}
\put(325.58,583.41){\usebox{\plotpoint}}
\put(341.82,596.32){\usebox{\plotpoint}}
\put(358.93,608.04){\usebox{\plotpoint}}
\put(377.64,616.94){\usebox{\plotpoint}}
\put(397.37,623.32){\usebox{\plotpoint}}
\put(417.48,628.42){\usebox{\plotpoint}}
\put(437.57,633.60){\usebox{\plotpoint}}
\put(456.72,641.56){\usebox{\plotpoint}}
\put(475.15,651.09){\usebox{\plotpoint}}
\put(492.99,661.69){\usebox{\plotpoint}}
\put(510.93,672.11){\usebox{\plotpoint}}
\put(529.23,681.86){\usebox{\plotpoint}}
\put(548.60,689.31){\usebox{\plotpoint}}
\put(568.67,694.41){\usebox{\plotpoint}}
\put(589.18,697.57){\usebox{\plotpoint}}
\put(609.85,699.37){\usebox{\plotpoint}}
\put(630.58,700.00){\usebox{\plotpoint}}
\put(651.34,700.00){\usebox{\plotpoint}}
\put(672.06,701.00){\usebox{\plotpoint}}
\put(692.81,701.00){\usebox{\plotpoint}}
\put(713.48,702.69){\usebox{\plotpoint}}
\put(734.09,704.93){\usebox{\plotpoint}}
\put(754.34,709.46){\usebox{\plotpoint}}
\put(773.99,716.00){\usebox{\plotpoint}}
\put(793.19,723.86){\usebox{\plotpoint}}
\put(812.04,732.55){\usebox{\plotpoint}}
\put(830.48,742.07){\usebox{\plotpoint}}
\put(849.32,750.76){\usebox{\plotpoint}}
\put(868.62,758.39){\usebox{\plotpoint}}
\put(888.43,764.41){\usebox{\plotpoint}}
\put(908.72,768.73){\usebox{\plotpoint}}
\put(929.34,770.95){\usebox{\plotpoint}}
\put(950.06,772.00){\usebox{\plotpoint}}
\put(970.81,772.00){\usebox{\plotpoint}}
\put(991.54,772.66){\usebox{\plotpoint}}
\put(1012.28,773.00){\usebox{\plotpoint}}
\put(1033.04,773.00){\usebox{\plotpoint}}
\put(1053.80,773.00){\usebox{\plotpoint}}
\put(1074.55,773.00){\usebox{\plotpoint}}
\put(1095.31,773.00){\usebox{\plotpoint}}
\put(1116.06,773.00){\usebox{\plotpoint}}
\put(1136.82,773.00){\usebox{\plotpoint}}
\put(1157.57,773.00){\usebox{\plotpoint}}
\put(1178.33,773.00){\usebox{\plotpoint}}
\put(1199.08,773.00){\usebox{\plotpoint}}
\put(1219.84,773.00){\usebox{\plotpoint}}
\put(1240.59,773.00){\usebox{\plotpoint}}
\put(1261.35,773.00){\usebox{\plotpoint}}
\put(1282.11,773.00){\usebox{\plotpoint}}
\put(1302.86,773.00){\usebox{\plotpoint}}
\put(1323.62,773.00){\usebox{\plotpoint}}
\put(1344.37,773.00){\usebox{\plotpoint}}
\put(1365.13,773.00){\usebox{\plotpoint}}
\put(1385.88,773.00){\usebox{\plotpoint}}
\put(1406.64,773.00){\usebox{\plotpoint}}
\put(1427.39,773.00){\usebox{\plotpoint}}
\put(1439,773){\usebox{\plotpoint}}
\sbox{\plotpoint}{\rule[-0.200pt]{0.400pt}{0.400pt}}%
\put(170.0,82.0){\rule[-0.200pt]{0.400pt}{187.179pt}}
\put(170.0,82.0){\rule[-0.200pt]{305.702pt}{0.400pt}}
\put(1439.0,82.0){\rule[-0.200pt]{0.400pt}{187.179pt}}
\put(170.0,859.0){\rule[-0.200pt]{305.702pt}{0.400pt}}
\end{picture}

\caption{Zleżność czasu działania od wielkości macierzy dla metody naturalnej i Strassena}
\end{center}
\end{figure}
\begin{figure}[hb]
\begin{center}
% GNUPLOT: LaTeX picture
\setlength{\unitlength}{0.240900pt}
\ifx\plotpoint\undefined\newsavebox{\plotpoint}\fi
\sbox{\plotpoint}{\rule[-0.200pt]{0.400pt}{0.400pt}}%
\begin{picture}(1500,900)(0,0)
\sbox{\plotpoint}{\rule[-0.200pt]{0.400pt}{0.400pt}}%
\put(170.0,82.0){\rule[-0.200pt]{4.818pt}{0.400pt}}
\put(150,82){\makebox(0,0)[r]{ 1}}
\put(1419.0,82.0){\rule[-0.200pt]{4.818pt}{0.400pt}}
\put(170.0,108.0){\rule[-0.200pt]{2.409pt}{0.400pt}}
\put(1429.0,108.0){\rule[-0.200pt]{2.409pt}{0.400pt}}
\put(170.0,123.0){\rule[-0.200pt]{2.409pt}{0.400pt}}
\put(1429.0,123.0){\rule[-0.200pt]{2.409pt}{0.400pt}}
\put(170.0,134.0){\rule[-0.200pt]{2.409pt}{0.400pt}}
\put(1429.0,134.0){\rule[-0.200pt]{2.409pt}{0.400pt}}
\put(170.0,142.0){\rule[-0.200pt]{2.409pt}{0.400pt}}
\put(1429.0,142.0){\rule[-0.200pt]{2.409pt}{0.400pt}}
\put(170.0,149.0){\rule[-0.200pt]{2.409pt}{0.400pt}}
\put(1429.0,149.0){\rule[-0.200pt]{2.409pt}{0.400pt}}
\put(170.0,155.0){\rule[-0.200pt]{2.409pt}{0.400pt}}
\put(1429.0,155.0){\rule[-0.200pt]{2.409pt}{0.400pt}}
\put(170.0,160.0){\rule[-0.200pt]{2.409pt}{0.400pt}}
\put(1429.0,160.0){\rule[-0.200pt]{2.409pt}{0.400pt}}
\put(170.0,164.0){\rule[-0.200pt]{2.409pt}{0.400pt}}
\put(1429.0,164.0){\rule[-0.200pt]{2.409pt}{0.400pt}}
\put(170.0,168.0){\rule[-0.200pt]{2.409pt}{0.400pt}}
\put(1429.0,168.0){\rule[-0.200pt]{2.409pt}{0.400pt}}
\put(170.0,172.0){\rule[-0.200pt]{2.409pt}{0.400pt}}
\put(1429.0,172.0){\rule[-0.200pt]{2.409pt}{0.400pt}}
\put(170.0,175.0){\rule[-0.200pt]{2.409pt}{0.400pt}}
\put(1429.0,175.0){\rule[-0.200pt]{2.409pt}{0.400pt}}
\put(170.0,178.0){\rule[-0.200pt]{2.409pt}{0.400pt}}
\put(1429.0,178.0){\rule[-0.200pt]{2.409pt}{0.400pt}}
\put(170.0,181.0){\rule[-0.200pt]{2.409pt}{0.400pt}}
\put(1429.0,181.0){\rule[-0.200pt]{2.409pt}{0.400pt}}
\put(170.0,184.0){\rule[-0.200pt]{2.409pt}{0.400pt}}
\put(1429.0,184.0){\rule[-0.200pt]{2.409pt}{0.400pt}}
\put(170.0,186.0){\rule[-0.200pt]{2.409pt}{0.400pt}}
\put(1429.0,186.0){\rule[-0.200pt]{2.409pt}{0.400pt}}
\put(170.0,188.0){\rule[-0.200pt]{2.409pt}{0.400pt}}
\put(1429.0,188.0){\rule[-0.200pt]{2.409pt}{0.400pt}}
\put(170.0,190.0){\rule[-0.200pt]{2.409pt}{0.400pt}}
\put(1429.0,190.0){\rule[-0.200pt]{2.409pt}{0.400pt}}
\put(170.0,192.0){\rule[-0.200pt]{2.409pt}{0.400pt}}
\put(1429.0,192.0){\rule[-0.200pt]{2.409pt}{0.400pt}}
\put(170.0,194.0){\rule[-0.200pt]{2.409pt}{0.400pt}}
\put(1429.0,194.0){\rule[-0.200pt]{2.409pt}{0.400pt}}
\put(170.0,196.0){\rule[-0.200pt]{2.409pt}{0.400pt}}
\put(1429.0,196.0){\rule[-0.200pt]{2.409pt}{0.400pt}}
\put(170.0,198.0){\rule[-0.200pt]{2.409pt}{0.400pt}}
\put(1429.0,198.0){\rule[-0.200pt]{2.409pt}{0.400pt}}
\put(170.0,200.0){\rule[-0.200pt]{2.409pt}{0.400pt}}
\put(1429.0,200.0){\rule[-0.200pt]{2.409pt}{0.400pt}}
\put(170.0,201.0){\rule[-0.200pt]{2.409pt}{0.400pt}}
\put(1429.0,201.0){\rule[-0.200pt]{2.409pt}{0.400pt}}
\put(170.0,203.0){\rule[-0.200pt]{2.409pt}{0.400pt}}
\put(1429.0,203.0){\rule[-0.200pt]{2.409pt}{0.400pt}}
\put(170.0,204.0){\rule[-0.200pt]{2.409pt}{0.400pt}}
\put(1429.0,204.0){\rule[-0.200pt]{2.409pt}{0.400pt}}
\put(170.0,206.0){\rule[-0.200pt]{2.409pt}{0.400pt}}
\put(1429.0,206.0){\rule[-0.200pt]{2.409pt}{0.400pt}}
\put(170.0,207.0){\rule[-0.200pt]{2.409pt}{0.400pt}}
\put(1429.0,207.0){\rule[-0.200pt]{2.409pt}{0.400pt}}
\put(170.0,208.0){\rule[-0.200pt]{2.409pt}{0.400pt}}
\put(1429.0,208.0){\rule[-0.200pt]{2.409pt}{0.400pt}}
\put(170.0,210.0){\rule[-0.200pt]{2.409pt}{0.400pt}}
\put(1429.0,210.0){\rule[-0.200pt]{2.409pt}{0.400pt}}
\put(170.0,211.0){\rule[-0.200pt]{2.409pt}{0.400pt}}
\put(1429.0,211.0){\rule[-0.200pt]{2.409pt}{0.400pt}}
\put(170.0,212.0){\rule[-0.200pt]{2.409pt}{0.400pt}}
\put(1429.0,212.0){\rule[-0.200pt]{2.409pt}{0.400pt}}
\put(170.0,213.0){\rule[-0.200pt]{2.409pt}{0.400pt}}
\put(1429.0,213.0){\rule[-0.200pt]{2.409pt}{0.400pt}}
\put(170.0,214.0){\rule[-0.200pt]{2.409pt}{0.400pt}}
\put(1429.0,214.0){\rule[-0.200pt]{2.409pt}{0.400pt}}
\put(170.0,215.0){\rule[-0.200pt]{2.409pt}{0.400pt}}
\put(1429.0,215.0){\rule[-0.200pt]{2.409pt}{0.400pt}}
\put(170.0,216.0){\rule[-0.200pt]{2.409pt}{0.400pt}}
\put(1429.0,216.0){\rule[-0.200pt]{2.409pt}{0.400pt}}
\put(170.0,217.0){\rule[-0.200pt]{2.409pt}{0.400pt}}
\put(1429.0,217.0){\rule[-0.200pt]{2.409pt}{0.400pt}}
\put(170.0,218.0){\rule[-0.200pt]{2.409pt}{0.400pt}}
\put(1429.0,218.0){\rule[-0.200pt]{2.409pt}{0.400pt}}
\put(170.0,219.0){\rule[-0.200pt]{2.409pt}{0.400pt}}
\put(1429.0,219.0){\rule[-0.200pt]{2.409pt}{0.400pt}}
\put(170.0,220.0){\rule[-0.200pt]{2.409pt}{0.400pt}}
\put(1429.0,220.0){\rule[-0.200pt]{2.409pt}{0.400pt}}
\put(170.0,221.0){\rule[-0.200pt]{2.409pt}{0.400pt}}
\put(1429.0,221.0){\rule[-0.200pt]{2.409pt}{0.400pt}}
\put(170.0,222.0){\rule[-0.200pt]{2.409pt}{0.400pt}}
\put(1429.0,222.0){\rule[-0.200pt]{2.409pt}{0.400pt}}
\put(170.0,223.0){\rule[-0.200pt]{2.409pt}{0.400pt}}
\put(1429.0,223.0){\rule[-0.200pt]{2.409pt}{0.400pt}}
\put(170.0,224.0){\rule[-0.200pt]{2.409pt}{0.400pt}}
\put(1429.0,224.0){\rule[-0.200pt]{2.409pt}{0.400pt}}
\put(170.0,225.0){\rule[-0.200pt]{2.409pt}{0.400pt}}
\put(1429.0,225.0){\rule[-0.200pt]{2.409pt}{0.400pt}}
\put(170.0,226.0){\rule[-0.200pt]{2.409pt}{0.400pt}}
\put(1429.0,226.0){\rule[-0.200pt]{2.409pt}{0.400pt}}
\put(170.0,226.0){\rule[-0.200pt]{2.409pt}{0.400pt}}
\put(1429.0,226.0){\rule[-0.200pt]{2.409pt}{0.400pt}}
\put(170.0,227.0){\rule[-0.200pt]{2.409pt}{0.400pt}}
\put(1429.0,227.0){\rule[-0.200pt]{2.409pt}{0.400pt}}
\put(170.0,228.0){\rule[-0.200pt]{2.409pt}{0.400pt}}
\put(1429.0,228.0){\rule[-0.200pt]{2.409pt}{0.400pt}}
\put(170.0,229.0){\rule[-0.200pt]{2.409pt}{0.400pt}}
\put(1429.0,229.0){\rule[-0.200pt]{2.409pt}{0.400pt}}
\put(170.0,229.0){\rule[-0.200pt]{2.409pt}{0.400pt}}
\put(1429.0,229.0){\rule[-0.200pt]{2.409pt}{0.400pt}}
\put(170.0,230.0){\rule[-0.200pt]{2.409pt}{0.400pt}}
\put(1429.0,230.0){\rule[-0.200pt]{2.409pt}{0.400pt}}
\put(170.0,231.0){\rule[-0.200pt]{2.409pt}{0.400pt}}
\put(1429.0,231.0){\rule[-0.200pt]{2.409pt}{0.400pt}}
\put(170.0,232.0){\rule[-0.200pt]{2.409pt}{0.400pt}}
\put(1429.0,232.0){\rule[-0.200pt]{2.409pt}{0.400pt}}
\put(170.0,232.0){\rule[-0.200pt]{2.409pt}{0.400pt}}
\put(1429.0,232.0){\rule[-0.200pt]{2.409pt}{0.400pt}}
\put(170.0,233.0){\rule[-0.200pt]{2.409pt}{0.400pt}}
\put(1429.0,233.0){\rule[-0.200pt]{2.409pt}{0.400pt}}
\put(170.0,234.0){\rule[-0.200pt]{2.409pt}{0.400pt}}
\put(1429.0,234.0){\rule[-0.200pt]{2.409pt}{0.400pt}}
\put(170.0,234.0){\rule[-0.200pt]{2.409pt}{0.400pt}}
\put(1429.0,234.0){\rule[-0.200pt]{2.409pt}{0.400pt}}
\put(170.0,235.0){\rule[-0.200pt]{2.409pt}{0.400pt}}
\put(1429.0,235.0){\rule[-0.200pt]{2.409pt}{0.400pt}}
\put(170.0,236.0){\rule[-0.200pt]{2.409pt}{0.400pt}}
\put(1429.0,236.0){\rule[-0.200pt]{2.409pt}{0.400pt}}
\put(170.0,236.0){\rule[-0.200pt]{2.409pt}{0.400pt}}
\put(1429.0,236.0){\rule[-0.200pt]{2.409pt}{0.400pt}}
\put(170.0,237.0){\rule[-0.200pt]{2.409pt}{0.400pt}}
\put(1429.0,237.0){\rule[-0.200pt]{2.409pt}{0.400pt}}
\put(170.0,237.0){\rule[-0.200pt]{2.409pt}{0.400pt}}
\put(1429.0,237.0){\rule[-0.200pt]{2.409pt}{0.400pt}}
\put(170.0,238.0){\rule[-0.200pt]{2.409pt}{0.400pt}}
\put(1429.0,238.0){\rule[-0.200pt]{2.409pt}{0.400pt}}
\put(170.0,239.0){\rule[-0.200pt]{2.409pt}{0.400pt}}
\put(1429.0,239.0){\rule[-0.200pt]{2.409pt}{0.400pt}}
\put(170.0,239.0){\rule[-0.200pt]{2.409pt}{0.400pt}}
\put(1429.0,239.0){\rule[-0.200pt]{2.409pt}{0.400pt}}
\put(170.0,240.0){\rule[-0.200pt]{2.409pt}{0.400pt}}
\put(1429.0,240.0){\rule[-0.200pt]{2.409pt}{0.400pt}}
\put(170.0,240.0){\rule[-0.200pt]{2.409pt}{0.400pt}}
\put(1429.0,240.0){\rule[-0.200pt]{2.409pt}{0.400pt}}
\put(170.0,241.0){\rule[-0.200pt]{2.409pt}{0.400pt}}
\put(1429.0,241.0){\rule[-0.200pt]{2.409pt}{0.400pt}}
\put(170.0,241.0){\rule[-0.200pt]{2.409pt}{0.400pt}}
\put(1429.0,241.0){\rule[-0.200pt]{2.409pt}{0.400pt}}
\put(170.0,242.0){\rule[-0.200pt]{2.409pt}{0.400pt}}
\put(1429.0,242.0){\rule[-0.200pt]{2.409pt}{0.400pt}}
\put(170.0,242.0){\rule[-0.200pt]{2.409pt}{0.400pt}}
\put(1429.0,242.0){\rule[-0.200pt]{2.409pt}{0.400pt}}
\put(170.0,243.0){\rule[-0.200pt]{2.409pt}{0.400pt}}
\put(1429.0,243.0){\rule[-0.200pt]{2.409pt}{0.400pt}}
\put(170.0,243.0){\rule[-0.200pt]{2.409pt}{0.400pt}}
\put(1429.0,243.0){\rule[-0.200pt]{2.409pt}{0.400pt}}
\put(170.0,244.0){\rule[-0.200pt]{2.409pt}{0.400pt}}
\put(1429.0,244.0){\rule[-0.200pt]{2.409pt}{0.400pt}}
\put(170.0,244.0){\rule[-0.200pt]{2.409pt}{0.400pt}}
\put(1429.0,244.0){\rule[-0.200pt]{2.409pt}{0.400pt}}
\put(170.0,245.0){\rule[-0.200pt]{2.409pt}{0.400pt}}
\put(1429.0,245.0){\rule[-0.200pt]{2.409pt}{0.400pt}}
\put(170.0,245.0){\rule[-0.200pt]{2.409pt}{0.400pt}}
\put(1429.0,245.0){\rule[-0.200pt]{2.409pt}{0.400pt}}
\put(170.0,246.0){\rule[-0.200pt]{2.409pt}{0.400pt}}
\put(1429.0,246.0){\rule[-0.200pt]{2.409pt}{0.400pt}}
\put(170.0,246.0){\rule[-0.200pt]{2.409pt}{0.400pt}}
\put(1429.0,246.0){\rule[-0.200pt]{2.409pt}{0.400pt}}
\put(170.0,247.0){\rule[-0.200pt]{2.409pt}{0.400pt}}
\put(1429.0,247.0){\rule[-0.200pt]{2.409pt}{0.400pt}}
\put(170.0,247.0){\rule[-0.200pt]{2.409pt}{0.400pt}}
\put(1429.0,247.0){\rule[-0.200pt]{2.409pt}{0.400pt}}
\put(170.0,248.0){\rule[-0.200pt]{2.409pt}{0.400pt}}
\put(1429.0,248.0){\rule[-0.200pt]{2.409pt}{0.400pt}}
\put(170.0,248.0){\rule[-0.200pt]{2.409pt}{0.400pt}}
\put(1429.0,248.0){\rule[-0.200pt]{2.409pt}{0.400pt}}
\put(170.0,249.0){\rule[-0.200pt]{2.409pt}{0.400pt}}
\put(1429.0,249.0){\rule[-0.200pt]{2.409pt}{0.400pt}}
\put(170.0,249.0){\rule[-0.200pt]{2.409pt}{0.400pt}}
\put(1429.0,249.0){\rule[-0.200pt]{2.409pt}{0.400pt}}
\put(170.0,249.0){\rule[-0.200pt]{2.409pt}{0.400pt}}
\put(1429.0,249.0){\rule[-0.200pt]{2.409pt}{0.400pt}}
\put(170.0,250.0){\rule[-0.200pt]{2.409pt}{0.400pt}}
\put(1429.0,250.0){\rule[-0.200pt]{2.409pt}{0.400pt}}
\put(170.0,250.0){\rule[-0.200pt]{2.409pt}{0.400pt}}
\put(1429.0,250.0){\rule[-0.200pt]{2.409pt}{0.400pt}}
\put(170.0,251.0){\rule[-0.200pt]{2.409pt}{0.400pt}}
\put(1429.0,251.0){\rule[-0.200pt]{2.409pt}{0.400pt}}
\put(170.0,251.0){\rule[-0.200pt]{2.409pt}{0.400pt}}
\put(1429.0,251.0){\rule[-0.200pt]{2.409pt}{0.400pt}}
\put(170.0,252.0){\rule[-0.200pt]{2.409pt}{0.400pt}}
\put(1429.0,252.0){\rule[-0.200pt]{2.409pt}{0.400pt}}
\put(170.0,252.0){\rule[-0.200pt]{2.409pt}{0.400pt}}
\put(1429.0,252.0){\rule[-0.200pt]{2.409pt}{0.400pt}}
\put(170.0,252.0){\rule[-0.200pt]{2.409pt}{0.400pt}}
\put(1429.0,252.0){\rule[-0.200pt]{2.409pt}{0.400pt}}
\put(170.0,253.0){\rule[-0.200pt]{2.409pt}{0.400pt}}
\put(1429.0,253.0){\rule[-0.200pt]{2.409pt}{0.400pt}}
\put(170.0,253.0){\rule[-0.200pt]{2.409pt}{0.400pt}}
\put(1429.0,253.0){\rule[-0.200pt]{2.409pt}{0.400pt}}
\put(170.0,254.0){\rule[-0.200pt]{2.409pt}{0.400pt}}
\put(1429.0,254.0){\rule[-0.200pt]{2.409pt}{0.400pt}}
\put(170.0,254.0){\rule[-0.200pt]{2.409pt}{0.400pt}}
\put(1429.0,254.0){\rule[-0.200pt]{2.409pt}{0.400pt}}
\put(170.0,254.0){\rule[-0.200pt]{2.409pt}{0.400pt}}
\put(1429.0,254.0){\rule[-0.200pt]{2.409pt}{0.400pt}}
\put(170.0,255.0){\rule[-0.200pt]{2.409pt}{0.400pt}}
\put(1429.0,255.0){\rule[-0.200pt]{2.409pt}{0.400pt}}
\put(170.0,255.0){\rule[-0.200pt]{2.409pt}{0.400pt}}
\put(1429.0,255.0){\rule[-0.200pt]{2.409pt}{0.400pt}}
\put(170.0,255.0){\rule[-0.200pt]{2.409pt}{0.400pt}}
\put(1429.0,255.0){\rule[-0.200pt]{2.409pt}{0.400pt}}
\put(170.0,256.0){\rule[-0.200pt]{2.409pt}{0.400pt}}
\put(1429.0,256.0){\rule[-0.200pt]{2.409pt}{0.400pt}}
\put(170.0,256.0){\rule[-0.200pt]{2.409pt}{0.400pt}}
\put(1429.0,256.0){\rule[-0.200pt]{2.409pt}{0.400pt}}
\put(170.0,256.0){\rule[-0.200pt]{2.409pt}{0.400pt}}
\put(1429.0,256.0){\rule[-0.200pt]{2.409pt}{0.400pt}}
\put(170.0,257.0){\rule[-0.200pt]{2.409pt}{0.400pt}}
\put(1429.0,257.0){\rule[-0.200pt]{2.409pt}{0.400pt}}
\put(170.0,257.0){\rule[-0.200pt]{2.409pt}{0.400pt}}
\put(1429.0,257.0){\rule[-0.200pt]{2.409pt}{0.400pt}}
\put(170.0,258.0){\rule[-0.200pt]{2.409pt}{0.400pt}}
\put(1429.0,258.0){\rule[-0.200pt]{2.409pt}{0.400pt}}
\put(170.0,258.0){\rule[-0.200pt]{2.409pt}{0.400pt}}
\put(1429.0,258.0){\rule[-0.200pt]{2.409pt}{0.400pt}}
\put(170.0,258.0){\rule[-0.200pt]{2.409pt}{0.400pt}}
\put(1429.0,258.0){\rule[-0.200pt]{2.409pt}{0.400pt}}
\put(170.0,259.0){\rule[-0.200pt]{2.409pt}{0.400pt}}
\put(1429.0,259.0){\rule[-0.200pt]{2.409pt}{0.400pt}}
\put(170.0,259.0){\rule[-0.200pt]{2.409pt}{0.400pt}}
\put(1429.0,259.0){\rule[-0.200pt]{2.409pt}{0.400pt}}
\put(170.0,259.0){\rule[-0.200pt]{2.409pt}{0.400pt}}
\put(1429.0,259.0){\rule[-0.200pt]{2.409pt}{0.400pt}}
\put(170.0,260.0){\rule[-0.200pt]{2.409pt}{0.400pt}}
\put(1429.0,260.0){\rule[-0.200pt]{2.409pt}{0.400pt}}
\put(170.0,260.0){\rule[-0.200pt]{2.409pt}{0.400pt}}
\put(1429.0,260.0){\rule[-0.200pt]{2.409pt}{0.400pt}}
\put(170.0,260.0){\rule[-0.200pt]{2.409pt}{0.400pt}}
\put(1429.0,260.0){\rule[-0.200pt]{2.409pt}{0.400pt}}
\put(170.0,261.0){\rule[-0.200pt]{2.409pt}{0.400pt}}
\put(1429.0,261.0){\rule[-0.200pt]{2.409pt}{0.400pt}}
\put(170.0,261.0){\rule[-0.200pt]{2.409pt}{0.400pt}}
\put(1429.0,261.0){\rule[-0.200pt]{2.409pt}{0.400pt}}
\put(170.0,261.0){\rule[-0.200pt]{2.409pt}{0.400pt}}
\put(1429.0,261.0){\rule[-0.200pt]{2.409pt}{0.400pt}}
\put(170.0,262.0){\rule[-0.200pt]{2.409pt}{0.400pt}}
\put(1429.0,262.0){\rule[-0.200pt]{2.409pt}{0.400pt}}
\put(170.0,262.0){\rule[-0.200pt]{2.409pt}{0.400pt}}
\put(1429.0,262.0){\rule[-0.200pt]{2.409pt}{0.400pt}}
\put(170.0,262.0){\rule[-0.200pt]{2.409pt}{0.400pt}}
\put(1429.0,262.0){\rule[-0.200pt]{2.409pt}{0.400pt}}
\put(170.0,262.0){\rule[-0.200pt]{2.409pt}{0.400pt}}
\put(1429.0,262.0){\rule[-0.200pt]{2.409pt}{0.400pt}}
\put(170.0,263.0){\rule[-0.200pt]{2.409pt}{0.400pt}}
\put(1429.0,263.0){\rule[-0.200pt]{2.409pt}{0.400pt}}
\put(170.0,263.0){\rule[-0.200pt]{2.409pt}{0.400pt}}
\put(1429.0,263.0){\rule[-0.200pt]{2.409pt}{0.400pt}}
\put(170.0,263.0){\rule[-0.200pt]{2.409pt}{0.400pt}}
\put(1429.0,263.0){\rule[-0.200pt]{2.409pt}{0.400pt}}
\put(170.0,264.0){\rule[-0.200pt]{2.409pt}{0.400pt}}
\put(1429.0,264.0){\rule[-0.200pt]{2.409pt}{0.400pt}}
\put(170.0,264.0){\rule[-0.200pt]{2.409pt}{0.400pt}}
\put(1429.0,264.0){\rule[-0.200pt]{2.409pt}{0.400pt}}
\put(170.0,264.0){\rule[-0.200pt]{2.409pt}{0.400pt}}
\put(1429.0,264.0){\rule[-0.200pt]{2.409pt}{0.400pt}}
\put(170.0,265.0){\rule[-0.200pt]{2.409pt}{0.400pt}}
\put(1429.0,265.0){\rule[-0.200pt]{2.409pt}{0.400pt}}
\put(170.0,265.0){\rule[-0.200pt]{2.409pt}{0.400pt}}
\put(1429.0,265.0){\rule[-0.200pt]{2.409pt}{0.400pt}}
\put(170.0,265.0){\rule[-0.200pt]{2.409pt}{0.400pt}}
\put(1429.0,265.0){\rule[-0.200pt]{2.409pt}{0.400pt}}
\put(170.0,265.0){\rule[-0.200pt]{2.409pt}{0.400pt}}
\put(1429.0,265.0){\rule[-0.200pt]{2.409pt}{0.400pt}}
\put(170.0,266.0){\rule[-0.200pt]{2.409pt}{0.400pt}}
\put(1429.0,266.0){\rule[-0.200pt]{2.409pt}{0.400pt}}
\put(170.0,266.0){\rule[-0.200pt]{2.409pt}{0.400pt}}
\put(1429.0,266.0){\rule[-0.200pt]{2.409pt}{0.400pt}}
\put(170.0,266.0){\rule[-0.200pt]{2.409pt}{0.400pt}}
\put(1429.0,266.0){\rule[-0.200pt]{2.409pt}{0.400pt}}
\put(170.0,266.0){\rule[-0.200pt]{2.409pt}{0.400pt}}
\put(1429.0,266.0){\rule[-0.200pt]{2.409pt}{0.400pt}}
\put(170.0,267.0){\rule[-0.200pt]{2.409pt}{0.400pt}}
\put(1429.0,267.0){\rule[-0.200pt]{2.409pt}{0.400pt}}
\put(170.0,267.0){\rule[-0.200pt]{2.409pt}{0.400pt}}
\put(1429.0,267.0){\rule[-0.200pt]{2.409pt}{0.400pt}}
\put(170.0,267.0){\rule[-0.200pt]{2.409pt}{0.400pt}}
\put(1429.0,267.0){\rule[-0.200pt]{2.409pt}{0.400pt}}
\put(170.0,268.0){\rule[-0.200pt]{2.409pt}{0.400pt}}
\put(1429.0,268.0){\rule[-0.200pt]{2.409pt}{0.400pt}}
\put(170.0,268.0){\rule[-0.200pt]{2.409pt}{0.400pt}}
\put(1429.0,268.0){\rule[-0.200pt]{2.409pt}{0.400pt}}
\put(170.0,268.0){\rule[-0.200pt]{2.409pt}{0.400pt}}
\put(1429.0,268.0){\rule[-0.200pt]{2.409pt}{0.400pt}}
\put(170.0,268.0){\rule[-0.200pt]{2.409pt}{0.400pt}}
\put(1429.0,268.0){\rule[-0.200pt]{2.409pt}{0.400pt}}
\put(170.0,269.0){\rule[-0.200pt]{2.409pt}{0.400pt}}
\put(1429.0,269.0){\rule[-0.200pt]{2.409pt}{0.400pt}}
\put(170.0,269.0){\rule[-0.200pt]{2.409pt}{0.400pt}}
\put(1429.0,269.0){\rule[-0.200pt]{2.409pt}{0.400pt}}
\put(170.0,269.0){\rule[-0.200pt]{2.409pt}{0.400pt}}
\put(1429.0,269.0){\rule[-0.200pt]{2.409pt}{0.400pt}}
\put(170.0,269.0){\rule[-0.200pt]{2.409pt}{0.400pt}}
\put(1429.0,269.0){\rule[-0.200pt]{2.409pt}{0.400pt}}
\put(170.0,270.0){\rule[-0.200pt]{2.409pt}{0.400pt}}
\put(1429.0,270.0){\rule[-0.200pt]{2.409pt}{0.400pt}}
\put(170.0,270.0){\rule[-0.200pt]{2.409pt}{0.400pt}}
\put(1429.0,270.0){\rule[-0.200pt]{2.409pt}{0.400pt}}
\put(170.0,270.0){\rule[-0.200pt]{2.409pt}{0.400pt}}
\put(1429.0,270.0){\rule[-0.200pt]{2.409pt}{0.400pt}}
\put(170.0,270.0){\rule[-0.200pt]{2.409pt}{0.400pt}}
\put(1429.0,270.0){\rule[-0.200pt]{2.409pt}{0.400pt}}
\put(170.0,271.0){\rule[-0.200pt]{2.409pt}{0.400pt}}
\put(1429.0,271.0){\rule[-0.200pt]{2.409pt}{0.400pt}}
\put(170.0,271.0){\rule[-0.200pt]{2.409pt}{0.400pt}}
\put(1429.0,271.0){\rule[-0.200pt]{2.409pt}{0.400pt}}
\put(170.0,271.0){\rule[-0.200pt]{2.409pt}{0.400pt}}
\put(1429.0,271.0){\rule[-0.200pt]{2.409pt}{0.400pt}}
\put(170.0,271.0){\rule[-0.200pt]{2.409pt}{0.400pt}}
\put(1429.0,271.0){\rule[-0.200pt]{2.409pt}{0.400pt}}
\put(170.0,272.0){\rule[-0.200pt]{2.409pt}{0.400pt}}
\put(1429.0,272.0){\rule[-0.200pt]{2.409pt}{0.400pt}}
\put(170.0,272.0){\rule[-0.200pt]{2.409pt}{0.400pt}}
\put(1429.0,272.0){\rule[-0.200pt]{2.409pt}{0.400pt}}
\put(170.0,272.0){\rule[-0.200pt]{2.409pt}{0.400pt}}
\put(1429.0,272.0){\rule[-0.200pt]{2.409pt}{0.400pt}}
\put(170.0,272.0){\rule[-0.200pt]{2.409pt}{0.400pt}}
\put(1429.0,272.0){\rule[-0.200pt]{2.409pt}{0.400pt}}
\put(170.0,273.0){\rule[-0.200pt]{2.409pt}{0.400pt}}
\put(1429.0,273.0){\rule[-0.200pt]{2.409pt}{0.400pt}}
\put(170.0,273.0){\rule[-0.200pt]{2.409pt}{0.400pt}}
\put(1429.0,273.0){\rule[-0.200pt]{2.409pt}{0.400pt}}
\put(170.0,273.0){\rule[-0.200pt]{2.409pt}{0.400pt}}
\put(1429.0,273.0){\rule[-0.200pt]{2.409pt}{0.400pt}}
\put(170.0,273.0){\rule[-0.200pt]{2.409pt}{0.400pt}}
\put(1429.0,273.0){\rule[-0.200pt]{2.409pt}{0.400pt}}
\put(170.0,273.0){\rule[-0.200pt]{2.409pt}{0.400pt}}
\put(1429.0,273.0){\rule[-0.200pt]{2.409pt}{0.400pt}}
\put(170.0,274.0){\rule[-0.200pt]{2.409pt}{0.400pt}}
\put(1429.0,274.0){\rule[-0.200pt]{2.409pt}{0.400pt}}
\put(170.0,274.0){\rule[-0.200pt]{2.409pt}{0.400pt}}
\put(1429.0,274.0){\rule[-0.200pt]{2.409pt}{0.400pt}}
\put(170.0,274.0){\rule[-0.200pt]{2.409pt}{0.400pt}}
\put(1429.0,274.0){\rule[-0.200pt]{2.409pt}{0.400pt}}
\put(170.0,274.0){\rule[-0.200pt]{2.409pt}{0.400pt}}
\put(1429.0,274.0){\rule[-0.200pt]{2.409pt}{0.400pt}}
\put(170.0,275.0){\rule[-0.200pt]{2.409pt}{0.400pt}}
\put(1429.0,275.0){\rule[-0.200pt]{2.409pt}{0.400pt}}
\put(170.0,275.0){\rule[-0.200pt]{2.409pt}{0.400pt}}
\put(1429.0,275.0){\rule[-0.200pt]{2.409pt}{0.400pt}}
\put(170.0,275.0){\rule[-0.200pt]{2.409pt}{0.400pt}}
\put(1429.0,275.0){\rule[-0.200pt]{2.409pt}{0.400pt}}
\put(170.0,275.0){\rule[-0.200pt]{2.409pt}{0.400pt}}
\put(1429.0,275.0){\rule[-0.200pt]{2.409pt}{0.400pt}}
\put(170.0,275.0){\rule[-0.200pt]{2.409pt}{0.400pt}}
\put(1429.0,275.0){\rule[-0.200pt]{2.409pt}{0.400pt}}
\put(170.0,276.0){\rule[-0.200pt]{2.409pt}{0.400pt}}
\put(1429.0,276.0){\rule[-0.200pt]{2.409pt}{0.400pt}}
\put(170.0,276.0){\rule[-0.200pt]{2.409pt}{0.400pt}}
\put(1429.0,276.0){\rule[-0.200pt]{2.409pt}{0.400pt}}
\put(170.0,276.0){\rule[-0.200pt]{2.409pt}{0.400pt}}
\put(1429.0,276.0){\rule[-0.200pt]{2.409pt}{0.400pt}}
\put(170.0,276.0){\rule[-0.200pt]{2.409pt}{0.400pt}}
\put(1429.0,276.0){\rule[-0.200pt]{2.409pt}{0.400pt}}
\put(170.0,276.0){\rule[-0.200pt]{2.409pt}{0.400pt}}
\put(1429.0,276.0){\rule[-0.200pt]{2.409pt}{0.400pt}}
\put(170.0,277.0){\rule[-0.200pt]{2.409pt}{0.400pt}}
\put(1429.0,277.0){\rule[-0.200pt]{2.409pt}{0.400pt}}
\put(170.0,277.0){\rule[-0.200pt]{2.409pt}{0.400pt}}
\put(1429.0,277.0){\rule[-0.200pt]{2.409pt}{0.400pt}}
\put(170.0,277.0){\rule[-0.200pt]{2.409pt}{0.400pt}}
\put(1429.0,277.0){\rule[-0.200pt]{2.409pt}{0.400pt}}
\put(170.0,277.0){\rule[-0.200pt]{2.409pt}{0.400pt}}
\put(1429.0,277.0){\rule[-0.200pt]{2.409pt}{0.400pt}}
\put(170.0,278.0){\rule[-0.200pt]{2.409pt}{0.400pt}}
\put(1429.0,278.0){\rule[-0.200pt]{2.409pt}{0.400pt}}
\put(170.0,278.0){\rule[-0.200pt]{2.409pt}{0.400pt}}
\put(1429.0,278.0){\rule[-0.200pt]{2.409pt}{0.400pt}}
\put(170.0,278.0){\rule[-0.200pt]{2.409pt}{0.400pt}}
\put(1429.0,278.0){\rule[-0.200pt]{2.409pt}{0.400pt}}
\put(170.0,278.0){\rule[-0.200pt]{2.409pt}{0.400pt}}
\put(1429.0,278.0){\rule[-0.200pt]{2.409pt}{0.400pt}}
\put(170.0,278.0){\rule[-0.200pt]{2.409pt}{0.400pt}}
\put(1429.0,278.0){\rule[-0.200pt]{2.409pt}{0.400pt}}
\put(170.0,279.0){\rule[-0.200pt]{2.409pt}{0.400pt}}
\put(1429.0,279.0){\rule[-0.200pt]{2.409pt}{0.400pt}}
\put(170.0,279.0){\rule[-0.200pt]{2.409pt}{0.400pt}}
\put(1429.0,279.0){\rule[-0.200pt]{2.409pt}{0.400pt}}
\put(170.0,279.0){\rule[-0.200pt]{2.409pt}{0.400pt}}
\put(1429.0,279.0){\rule[-0.200pt]{2.409pt}{0.400pt}}
\put(170.0,279.0){\rule[-0.200pt]{2.409pt}{0.400pt}}
\put(1429.0,279.0){\rule[-0.200pt]{2.409pt}{0.400pt}}
\put(170.0,279.0){\rule[-0.200pt]{2.409pt}{0.400pt}}
\put(1429.0,279.0){\rule[-0.200pt]{2.409pt}{0.400pt}}
\put(170.0,280.0){\rule[-0.200pt]{2.409pt}{0.400pt}}
\put(1429.0,280.0){\rule[-0.200pt]{2.409pt}{0.400pt}}
\put(170.0,280.0){\rule[-0.200pt]{2.409pt}{0.400pt}}
\put(1429.0,280.0){\rule[-0.200pt]{2.409pt}{0.400pt}}
\put(170.0,280.0){\rule[-0.200pt]{2.409pt}{0.400pt}}
\put(1429.0,280.0){\rule[-0.200pt]{2.409pt}{0.400pt}}
\put(170.0,280.0){\rule[-0.200pt]{2.409pt}{0.400pt}}
\put(1429.0,280.0){\rule[-0.200pt]{2.409pt}{0.400pt}}
\put(170.0,280.0){\rule[-0.200pt]{2.409pt}{0.400pt}}
\put(1429.0,280.0){\rule[-0.200pt]{2.409pt}{0.400pt}}
\put(170.0,280.0){\rule[-0.200pt]{2.409pt}{0.400pt}}
\put(1429.0,280.0){\rule[-0.200pt]{2.409pt}{0.400pt}}
\put(170.0,281.0){\rule[-0.200pt]{2.409pt}{0.400pt}}
\put(1429.0,281.0){\rule[-0.200pt]{2.409pt}{0.400pt}}
\put(170.0,281.0){\rule[-0.200pt]{2.409pt}{0.400pt}}
\put(1429.0,281.0){\rule[-0.200pt]{2.409pt}{0.400pt}}
\put(170.0,281.0){\rule[-0.200pt]{2.409pt}{0.400pt}}
\put(1429.0,281.0){\rule[-0.200pt]{2.409pt}{0.400pt}}
\put(170.0,281.0){\rule[-0.200pt]{2.409pt}{0.400pt}}
\put(1429.0,281.0){\rule[-0.200pt]{2.409pt}{0.400pt}}
\put(170.0,281.0){\rule[-0.200pt]{2.409pt}{0.400pt}}
\put(1429.0,281.0){\rule[-0.200pt]{2.409pt}{0.400pt}}
\put(170.0,282.0){\rule[-0.200pt]{2.409pt}{0.400pt}}
\put(1429.0,282.0){\rule[-0.200pt]{2.409pt}{0.400pt}}
\put(170.0,282.0){\rule[-0.200pt]{2.409pt}{0.400pt}}
\put(1429.0,282.0){\rule[-0.200pt]{2.409pt}{0.400pt}}
\put(170.0,282.0){\rule[-0.200pt]{2.409pt}{0.400pt}}
\put(1429.0,282.0){\rule[-0.200pt]{2.409pt}{0.400pt}}
\put(170.0,282.0){\rule[-0.200pt]{2.409pt}{0.400pt}}
\put(1429.0,282.0){\rule[-0.200pt]{2.409pt}{0.400pt}}
\put(170.0,282.0){\rule[-0.200pt]{2.409pt}{0.400pt}}
\put(1429.0,282.0){\rule[-0.200pt]{2.409pt}{0.400pt}}
\put(170.0,282.0){\rule[-0.200pt]{2.409pt}{0.400pt}}
\put(1429.0,282.0){\rule[-0.200pt]{2.409pt}{0.400pt}}
\put(170.0,283.0){\rule[-0.200pt]{2.409pt}{0.400pt}}
\put(1429.0,283.0){\rule[-0.200pt]{2.409pt}{0.400pt}}
\put(170.0,283.0){\rule[-0.200pt]{2.409pt}{0.400pt}}
\put(1429.0,283.0){\rule[-0.200pt]{2.409pt}{0.400pt}}
\put(170.0,283.0){\rule[-0.200pt]{2.409pt}{0.400pt}}
\put(1429.0,283.0){\rule[-0.200pt]{2.409pt}{0.400pt}}
\put(170.0,283.0){\rule[-0.200pt]{2.409pt}{0.400pt}}
\put(1429.0,283.0){\rule[-0.200pt]{2.409pt}{0.400pt}}
\put(170.0,283.0){\rule[-0.200pt]{2.409pt}{0.400pt}}
\put(1429.0,283.0){\rule[-0.200pt]{2.409pt}{0.400pt}}
\put(170.0,284.0){\rule[-0.200pt]{2.409pt}{0.400pt}}
\put(1429.0,284.0){\rule[-0.200pt]{2.409pt}{0.400pt}}
\put(170.0,284.0){\rule[-0.200pt]{2.409pt}{0.400pt}}
\put(1429.0,284.0){\rule[-0.200pt]{2.409pt}{0.400pt}}
\put(170.0,284.0){\rule[-0.200pt]{2.409pt}{0.400pt}}
\put(1429.0,284.0){\rule[-0.200pt]{2.409pt}{0.400pt}}
\put(170.0,284.0){\rule[-0.200pt]{2.409pt}{0.400pt}}
\put(1429.0,284.0){\rule[-0.200pt]{2.409pt}{0.400pt}}
\put(170.0,284.0){\rule[-0.200pt]{2.409pt}{0.400pt}}
\put(1429.0,284.0){\rule[-0.200pt]{2.409pt}{0.400pt}}
\put(170.0,284.0){\rule[-0.200pt]{2.409pt}{0.400pt}}
\put(1429.0,284.0){\rule[-0.200pt]{2.409pt}{0.400pt}}
\put(170.0,285.0){\rule[-0.200pt]{2.409pt}{0.400pt}}
\put(1429.0,285.0){\rule[-0.200pt]{2.409pt}{0.400pt}}
\put(170.0,285.0){\rule[-0.200pt]{2.409pt}{0.400pt}}
\put(1429.0,285.0){\rule[-0.200pt]{2.409pt}{0.400pt}}
\put(170.0,285.0){\rule[-0.200pt]{2.409pt}{0.400pt}}
\put(1429.0,285.0){\rule[-0.200pt]{2.409pt}{0.400pt}}
\put(170.0,285.0){\rule[-0.200pt]{2.409pt}{0.400pt}}
\put(1429.0,285.0){\rule[-0.200pt]{2.409pt}{0.400pt}}
\put(170.0,285.0){\rule[-0.200pt]{2.409pt}{0.400pt}}
\put(1429.0,285.0){\rule[-0.200pt]{2.409pt}{0.400pt}}
\put(170.0,285.0){\rule[-0.200pt]{2.409pt}{0.400pt}}
\put(1429.0,285.0){\rule[-0.200pt]{2.409pt}{0.400pt}}
\put(170.0,286.0){\rule[-0.200pt]{2.409pt}{0.400pt}}
\put(1429.0,286.0){\rule[-0.200pt]{2.409pt}{0.400pt}}
\put(170.0,286.0){\rule[-0.200pt]{2.409pt}{0.400pt}}
\put(1429.0,286.0){\rule[-0.200pt]{2.409pt}{0.400pt}}
\put(170.0,286.0){\rule[-0.200pt]{2.409pt}{0.400pt}}
\put(1429.0,286.0){\rule[-0.200pt]{2.409pt}{0.400pt}}
\put(170.0,286.0){\rule[-0.200pt]{2.409pt}{0.400pt}}
\put(1429.0,286.0){\rule[-0.200pt]{2.409pt}{0.400pt}}
\put(170.0,286.0){\rule[-0.200pt]{2.409pt}{0.400pt}}
\put(1429.0,286.0){\rule[-0.200pt]{2.409pt}{0.400pt}}
\put(170.0,286.0){\rule[-0.200pt]{2.409pt}{0.400pt}}
\put(1429.0,286.0){\rule[-0.200pt]{2.409pt}{0.400pt}}
\put(170.0,287.0){\rule[-0.200pt]{2.409pt}{0.400pt}}
\put(1429.0,287.0){\rule[-0.200pt]{2.409pt}{0.400pt}}
\put(170.0,287.0){\rule[-0.200pt]{2.409pt}{0.400pt}}
\put(1429.0,287.0){\rule[-0.200pt]{2.409pt}{0.400pt}}
\put(170.0,287.0){\rule[-0.200pt]{2.409pt}{0.400pt}}
\put(1429.0,287.0){\rule[-0.200pt]{2.409pt}{0.400pt}}
\put(170.0,287.0){\rule[-0.200pt]{2.409pt}{0.400pt}}
\put(1429.0,287.0){\rule[-0.200pt]{2.409pt}{0.400pt}}
\put(170.0,287.0){\rule[-0.200pt]{2.409pt}{0.400pt}}
\put(1429.0,287.0){\rule[-0.200pt]{2.409pt}{0.400pt}}
\put(170.0,287.0){\rule[-0.200pt]{2.409pt}{0.400pt}}
\put(1429.0,287.0){\rule[-0.200pt]{2.409pt}{0.400pt}}
\put(170.0,287.0){\rule[-0.200pt]{2.409pt}{0.400pt}}
\put(1429.0,287.0){\rule[-0.200pt]{2.409pt}{0.400pt}}
\put(170.0,288.0){\rule[-0.200pt]{2.409pt}{0.400pt}}
\put(1429.0,288.0){\rule[-0.200pt]{2.409pt}{0.400pt}}
\put(170.0,288.0){\rule[-0.200pt]{2.409pt}{0.400pt}}
\put(1429.0,288.0){\rule[-0.200pt]{2.409pt}{0.400pt}}
\put(170.0,288.0){\rule[-0.200pt]{2.409pt}{0.400pt}}
\put(1429.0,288.0){\rule[-0.200pt]{2.409pt}{0.400pt}}
\put(170.0,288.0){\rule[-0.200pt]{2.409pt}{0.400pt}}
\put(1429.0,288.0){\rule[-0.200pt]{2.409pt}{0.400pt}}
\put(170.0,288.0){\rule[-0.200pt]{2.409pt}{0.400pt}}
\put(1429.0,288.0){\rule[-0.200pt]{2.409pt}{0.400pt}}
\put(170.0,288.0){\rule[-0.200pt]{2.409pt}{0.400pt}}
\put(1429.0,288.0){\rule[-0.200pt]{2.409pt}{0.400pt}}
\put(170.0,289.0){\rule[-0.200pt]{2.409pt}{0.400pt}}
\put(1429.0,289.0){\rule[-0.200pt]{2.409pt}{0.400pt}}
\put(170.0,289.0){\rule[-0.200pt]{2.409pt}{0.400pt}}
\put(1429.0,289.0){\rule[-0.200pt]{2.409pt}{0.400pt}}
\put(170.0,289.0){\rule[-0.200pt]{2.409pt}{0.400pt}}
\put(1429.0,289.0){\rule[-0.200pt]{2.409pt}{0.400pt}}
\put(170.0,289.0){\rule[-0.200pt]{2.409pt}{0.400pt}}
\put(1429.0,289.0){\rule[-0.200pt]{2.409pt}{0.400pt}}
\put(170.0,289.0){\rule[-0.200pt]{2.409pt}{0.400pt}}
\put(1429.0,289.0){\rule[-0.200pt]{2.409pt}{0.400pt}}
\put(170.0,289.0){\rule[-0.200pt]{2.409pt}{0.400pt}}
\put(1429.0,289.0){\rule[-0.200pt]{2.409pt}{0.400pt}}
\put(170.0,289.0){\rule[-0.200pt]{2.409pt}{0.400pt}}
\put(1429.0,289.0){\rule[-0.200pt]{2.409pt}{0.400pt}}
\put(170.0,290.0){\rule[-0.200pt]{2.409pt}{0.400pt}}
\put(1429.0,290.0){\rule[-0.200pt]{2.409pt}{0.400pt}}
\put(170.0,290.0){\rule[-0.200pt]{2.409pt}{0.400pt}}
\put(1429.0,290.0){\rule[-0.200pt]{2.409pt}{0.400pt}}
\put(170.0,290.0){\rule[-0.200pt]{2.409pt}{0.400pt}}
\put(1429.0,290.0){\rule[-0.200pt]{2.409pt}{0.400pt}}
\put(170.0,290.0){\rule[-0.200pt]{2.409pt}{0.400pt}}
\put(1429.0,290.0){\rule[-0.200pt]{2.409pt}{0.400pt}}
\put(170.0,290.0){\rule[-0.200pt]{2.409pt}{0.400pt}}
\put(1429.0,290.0){\rule[-0.200pt]{2.409pt}{0.400pt}}
\put(170.0,290.0){\rule[-0.200pt]{2.409pt}{0.400pt}}
\put(1429.0,290.0){\rule[-0.200pt]{2.409pt}{0.400pt}}
\put(170.0,290.0){\rule[-0.200pt]{2.409pt}{0.400pt}}
\put(1429.0,290.0){\rule[-0.200pt]{2.409pt}{0.400pt}}
\put(170.0,291.0){\rule[-0.200pt]{2.409pt}{0.400pt}}
\put(1429.0,291.0){\rule[-0.200pt]{2.409pt}{0.400pt}}
\put(170.0,291.0){\rule[-0.200pt]{2.409pt}{0.400pt}}
\put(1429.0,291.0){\rule[-0.200pt]{2.409pt}{0.400pt}}
\put(170.0,291.0){\rule[-0.200pt]{2.409pt}{0.400pt}}
\put(1429.0,291.0){\rule[-0.200pt]{2.409pt}{0.400pt}}
\put(170.0,291.0){\rule[-0.200pt]{2.409pt}{0.400pt}}
\put(1429.0,291.0){\rule[-0.200pt]{2.409pt}{0.400pt}}
\put(170.0,291.0){\rule[-0.200pt]{2.409pt}{0.400pt}}
\put(1429.0,291.0){\rule[-0.200pt]{2.409pt}{0.400pt}}
\put(170.0,291.0){\rule[-0.200pt]{2.409pt}{0.400pt}}
\put(1429.0,291.0){\rule[-0.200pt]{2.409pt}{0.400pt}}
\put(170.0,291.0){\rule[-0.200pt]{2.409pt}{0.400pt}}
\put(1429.0,291.0){\rule[-0.200pt]{2.409pt}{0.400pt}}
\put(170.0,292.0){\rule[-0.200pt]{2.409pt}{0.400pt}}
\put(1429.0,292.0){\rule[-0.200pt]{2.409pt}{0.400pt}}
\put(170.0,292.0){\rule[-0.200pt]{2.409pt}{0.400pt}}
\put(1429.0,292.0){\rule[-0.200pt]{2.409pt}{0.400pt}}
\put(170.0,292.0){\rule[-0.200pt]{2.409pt}{0.400pt}}
\put(1429.0,292.0){\rule[-0.200pt]{2.409pt}{0.400pt}}
\put(170.0,292.0){\rule[-0.200pt]{2.409pt}{0.400pt}}
\put(1429.0,292.0){\rule[-0.200pt]{2.409pt}{0.400pt}}
\put(170.0,292.0){\rule[-0.200pt]{2.409pt}{0.400pt}}
\put(1429.0,292.0){\rule[-0.200pt]{2.409pt}{0.400pt}}
\put(170.0,292.0){\rule[-0.200pt]{2.409pt}{0.400pt}}
\put(1429.0,292.0){\rule[-0.200pt]{2.409pt}{0.400pt}}
\put(170.0,292.0){\rule[-0.200pt]{2.409pt}{0.400pt}}
\put(1429.0,292.0){\rule[-0.200pt]{2.409pt}{0.400pt}}
\put(170.0,293.0){\rule[-0.200pt]{2.409pt}{0.400pt}}
\put(1429.0,293.0){\rule[-0.200pt]{2.409pt}{0.400pt}}
\put(170.0,293.0){\rule[-0.200pt]{2.409pt}{0.400pt}}
\put(1429.0,293.0){\rule[-0.200pt]{2.409pt}{0.400pt}}
\put(170.0,293.0){\rule[-0.200pt]{2.409pt}{0.400pt}}
\put(1429.0,293.0){\rule[-0.200pt]{2.409pt}{0.400pt}}
\put(170.0,293.0){\rule[-0.200pt]{2.409pt}{0.400pt}}
\put(1429.0,293.0){\rule[-0.200pt]{2.409pt}{0.400pt}}
\put(170.0,293.0){\rule[-0.200pt]{2.409pt}{0.400pt}}
\put(1429.0,293.0){\rule[-0.200pt]{2.409pt}{0.400pt}}
\put(170.0,293.0){\rule[-0.200pt]{2.409pt}{0.400pt}}
\put(1429.0,293.0){\rule[-0.200pt]{2.409pt}{0.400pt}}
\put(170.0,293.0){\rule[-0.200pt]{2.409pt}{0.400pt}}
\put(1429.0,293.0){\rule[-0.200pt]{2.409pt}{0.400pt}}
\put(170.0,294.0){\rule[-0.200pt]{2.409pt}{0.400pt}}
\put(1429.0,294.0){\rule[-0.200pt]{2.409pt}{0.400pt}}
\put(170.0,294.0){\rule[-0.200pt]{2.409pt}{0.400pt}}
\put(1429.0,294.0){\rule[-0.200pt]{2.409pt}{0.400pt}}
\put(170.0,294.0){\rule[-0.200pt]{2.409pt}{0.400pt}}
\put(1429.0,294.0){\rule[-0.200pt]{2.409pt}{0.400pt}}
\put(170.0,294.0){\rule[-0.200pt]{2.409pt}{0.400pt}}
\put(1429.0,294.0){\rule[-0.200pt]{2.409pt}{0.400pt}}
\put(170.0,294.0){\rule[-0.200pt]{2.409pt}{0.400pt}}
\put(1429.0,294.0){\rule[-0.200pt]{2.409pt}{0.400pt}}
\put(170.0,294.0){\rule[-0.200pt]{2.409pt}{0.400pt}}
\put(1429.0,294.0){\rule[-0.200pt]{2.409pt}{0.400pt}}
\put(170.0,294.0){\rule[-0.200pt]{2.409pt}{0.400pt}}
\put(1429.0,294.0){\rule[-0.200pt]{2.409pt}{0.400pt}}
\put(170.0,294.0){\rule[-0.200pt]{2.409pt}{0.400pt}}
\put(1429.0,294.0){\rule[-0.200pt]{2.409pt}{0.400pt}}
\put(170.0,295.0){\rule[-0.200pt]{2.409pt}{0.400pt}}
\put(1429.0,295.0){\rule[-0.200pt]{2.409pt}{0.400pt}}
\put(170.0,295.0){\rule[-0.200pt]{2.409pt}{0.400pt}}
\put(1429.0,295.0){\rule[-0.200pt]{2.409pt}{0.400pt}}
\put(170.0,295.0){\rule[-0.200pt]{2.409pt}{0.400pt}}
\put(1429.0,295.0){\rule[-0.200pt]{2.409pt}{0.400pt}}
\put(170.0,295.0){\rule[-0.200pt]{2.409pt}{0.400pt}}
\put(1429.0,295.0){\rule[-0.200pt]{2.409pt}{0.400pt}}
\put(170.0,295.0){\rule[-0.200pt]{2.409pt}{0.400pt}}
\put(1429.0,295.0){\rule[-0.200pt]{2.409pt}{0.400pt}}
\put(170.0,295.0){\rule[-0.200pt]{2.409pt}{0.400pt}}
\put(1429.0,295.0){\rule[-0.200pt]{2.409pt}{0.400pt}}
\put(170.0,295.0){\rule[-0.200pt]{2.409pt}{0.400pt}}
\put(1429.0,295.0){\rule[-0.200pt]{2.409pt}{0.400pt}}
\put(170.0,295.0){\rule[-0.200pt]{2.409pt}{0.400pt}}
\put(1429.0,295.0){\rule[-0.200pt]{2.409pt}{0.400pt}}
\put(170.0,296.0){\rule[-0.200pt]{2.409pt}{0.400pt}}
\put(1429.0,296.0){\rule[-0.200pt]{2.409pt}{0.400pt}}
\put(170.0,296.0){\rule[-0.200pt]{2.409pt}{0.400pt}}
\put(1429.0,296.0){\rule[-0.200pt]{2.409pt}{0.400pt}}
\put(170.0,296.0){\rule[-0.200pt]{2.409pt}{0.400pt}}
\put(1429.0,296.0){\rule[-0.200pt]{2.409pt}{0.400pt}}
\put(170.0,296.0){\rule[-0.200pt]{2.409pt}{0.400pt}}
\put(1429.0,296.0){\rule[-0.200pt]{2.409pt}{0.400pt}}
\put(170.0,296.0){\rule[-0.200pt]{2.409pt}{0.400pt}}
\put(1429.0,296.0){\rule[-0.200pt]{2.409pt}{0.400pt}}
\put(170.0,296.0){\rule[-0.200pt]{2.409pt}{0.400pt}}
\put(1429.0,296.0){\rule[-0.200pt]{2.409pt}{0.400pt}}
\put(170.0,296.0){\rule[-0.200pt]{2.409pt}{0.400pt}}
\put(1429.0,296.0){\rule[-0.200pt]{2.409pt}{0.400pt}}
\put(170.0,296.0){\rule[-0.200pt]{2.409pt}{0.400pt}}
\put(1429.0,296.0){\rule[-0.200pt]{2.409pt}{0.400pt}}
\put(170.0,297.0){\rule[-0.200pt]{2.409pt}{0.400pt}}
\put(1429.0,297.0){\rule[-0.200pt]{2.409pt}{0.400pt}}
\put(170.0,297.0){\rule[-0.200pt]{2.409pt}{0.400pt}}
\put(1429.0,297.0){\rule[-0.200pt]{2.409pt}{0.400pt}}
\put(170.0,297.0){\rule[-0.200pt]{2.409pt}{0.400pt}}
\put(1429.0,297.0){\rule[-0.200pt]{2.409pt}{0.400pt}}
\put(170.0,297.0){\rule[-0.200pt]{2.409pt}{0.400pt}}
\put(1429.0,297.0){\rule[-0.200pt]{2.409pt}{0.400pt}}
\put(170.0,297.0){\rule[-0.200pt]{2.409pt}{0.400pt}}
\put(1429.0,297.0){\rule[-0.200pt]{2.409pt}{0.400pt}}
\put(170.0,297.0){\rule[-0.200pt]{2.409pt}{0.400pt}}
\put(1429.0,297.0){\rule[-0.200pt]{2.409pt}{0.400pt}}
\put(170.0,297.0){\rule[-0.200pt]{2.409pt}{0.400pt}}
\put(1429.0,297.0){\rule[-0.200pt]{2.409pt}{0.400pt}}
\put(170.0,297.0){\rule[-0.200pt]{2.409pt}{0.400pt}}
\put(1429.0,297.0){\rule[-0.200pt]{2.409pt}{0.400pt}}
\put(170.0,298.0){\rule[-0.200pt]{2.409pt}{0.400pt}}
\put(1429.0,298.0){\rule[-0.200pt]{2.409pt}{0.400pt}}
\put(170.0,298.0){\rule[-0.200pt]{2.409pt}{0.400pt}}
\put(1429.0,298.0){\rule[-0.200pt]{2.409pt}{0.400pt}}
\put(170.0,298.0){\rule[-0.200pt]{2.409pt}{0.400pt}}
\put(1429.0,298.0){\rule[-0.200pt]{2.409pt}{0.400pt}}
\put(170.0,298.0){\rule[-0.200pt]{2.409pt}{0.400pt}}
\put(1429.0,298.0){\rule[-0.200pt]{2.409pt}{0.400pt}}
\put(170.0,298.0){\rule[-0.200pt]{2.409pt}{0.400pt}}
\put(1429.0,298.0){\rule[-0.200pt]{2.409pt}{0.400pt}}
\put(170.0,298.0){\rule[-0.200pt]{2.409pt}{0.400pt}}
\put(1429.0,298.0){\rule[-0.200pt]{2.409pt}{0.400pt}}
\put(170.0,298.0){\rule[-0.200pt]{2.409pt}{0.400pt}}
\put(1429.0,298.0){\rule[-0.200pt]{2.409pt}{0.400pt}}
\put(170.0,298.0){\rule[-0.200pt]{2.409pt}{0.400pt}}
\put(1429.0,298.0){\rule[-0.200pt]{2.409pt}{0.400pt}}
\put(170.0,299.0){\rule[-0.200pt]{2.409pt}{0.400pt}}
\put(1429.0,299.0){\rule[-0.200pt]{2.409pt}{0.400pt}}
\put(170.0,299.0){\rule[-0.200pt]{2.409pt}{0.400pt}}
\put(1429.0,299.0){\rule[-0.200pt]{2.409pt}{0.400pt}}
\put(170.0,299.0){\rule[-0.200pt]{2.409pt}{0.400pt}}
\put(1429.0,299.0){\rule[-0.200pt]{2.409pt}{0.400pt}}
\put(170.0,299.0){\rule[-0.200pt]{2.409pt}{0.400pt}}
\put(1429.0,299.0){\rule[-0.200pt]{2.409pt}{0.400pt}}
\put(170.0,299.0){\rule[-0.200pt]{2.409pt}{0.400pt}}
\put(1429.0,299.0){\rule[-0.200pt]{2.409pt}{0.400pt}}
\put(170.0,299.0){\rule[-0.200pt]{2.409pt}{0.400pt}}
\put(1429.0,299.0){\rule[-0.200pt]{2.409pt}{0.400pt}}
\put(170.0,299.0){\rule[-0.200pt]{2.409pt}{0.400pt}}
\put(1429.0,299.0){\rule[-0.200pt]{2.409pt}{0.400pt}}
\put(170.0,299.0){\rule[-0.200pt]{2.409pt}{0.400pt}}
\put(1429.0,299.0){\rule[-0.200pt]{2.409pt}{0.400pt}}
\put(170.0,299.0){\rule[-0.200pt]{2.409pt}{0.400pt}}
\put(1429.0,299.0){\rule[-0.200pt]{2.409pt}{0.400pt}}
\put(170.0,300.0){\rule[-0.200pt]{2.409pt}{0.400pt}}
\put(1429.0,300.0){\rule[-0.200pt]{2.409pt}{0.400pt}}
\put(170.0,300.0){\rule[-0.200pt]{2.409pt}{0.400pt}}
\put(1429.0,300.0){\rule[-0.200pt]{2.409pt}{0.400pt}}
\put(170.0,300.0){\rule[-0.200pt]{2.409pt}{0.400pt}}
\put(1429.0,300.0){\rule[-0.200pt]{2.409pt}{0.400pt}}
\put(170.0,300.0){\rule[-0.200pt]{2.409pt}{0.400pt}}
\put(1429.0,300.0){\rule[-0.200pt]{2.409pt}{0.400pt}}
\put(170.0,300.0){\rule[-0.200pt]{2.409pt}{0.400pt}}
\put(1429.0,300.0){\rule[-0.200pt]{2.409pt}{0.400pt}}
\put(170.0,300.0){\rule[-0.200pt]{2.409pt}{0.400pt}}
\put(1429.0,300.0){\rule[-0.200pt]{2.409pt}{0.400pt}}
\put(170.0,300.0){\rule[-0.200pt]{2.409pt}{0.400pt}}
\put(1429.0,300.0){\rule[-0.200pt]{2.409pt}{0.400pt}}
\put(170.0,300.0){\rule[-0.200pt]{2.409pt}{0.400pt}}
\put(1429.0,300.0){\rule[-0.200pt]{2.409pt}{0.400pt}}
\put(170.0,300.0){\rule[-0.200pt]{2.409pt}{0.400pt}}
\put(1429.0,300.0){\rule[-0.200pt]{2.409pt}{0.400pt}}
\put(170.0,301.0){\rule[-0.200pt]{2.409pt}{0.400pt}}
\put(1429.0,301.0){\rule[-0.200pt]{2.409pt}{0.400pt}}
\put(170.0,301.0){\rule[-0.200pt]{2.409pt}{0.400pt}}
\put(1429.0,301.0){\rule[-0.200pt]{2.409pt}{0.400pt}}
\put(170.0,301.0){\rule[-0.200pt]{2.409pt}{0.400pt}}
\put(1429.0,301.0){\rule[-0.200pt]{2.409pt}{0.400pt}}
\put(170.0,301.0){\rule[-0.200pt]{2.409pt}{0.400pt}}
\put(1429.0,301.0){\rule[-0.200pt]{2.409pt}{0.400pt}}
\put(170.0,301.0){\rule[-0.200pt]{2.409pt}{0.400pt}}
\put(1429.0,301.0){\rule[-0.200pt]{2.409pt}{0.400pt}}
\put(170.0,301.0){\rule[-0.200pt]{2.409pt}{0.400pt}}
\put(1429.0,301.0){\rule[-0.200pt]{2.409pt}{0.400pt}}
\put(170.0,301.0){\rule[-0.200pt]{2.409pt}{0.400pt}}
\put(1429.0,301.0){\rule[-0.200pt]{2.409pt}{0.400pt}}
\put(170.0,301.0){\rule[-0.200pt]{2.409pt}{0.400pt}}
\put(1429.0,301.0){\rule[-0.200pt]{2.409pt}{0.400pt}}
\put(170.0,301.0){\rule[-0.200pt]{2.409pt}{0.400pt}}
\put(1429.0,301.0){\rule[-0.200pt]{2.409pt}{0.400pt}}
\put(170.0,302.0){\rule[-0.200pt]{2.409pt}{0.400pt}}
\put(1429.0,302.0){\rule[-0.200pt]{2.409pt}{0.400pt}}
\put(170.0,302.0){\rule[-0.200pt]{2.409pt}{0.400pt}}
\put(1429.0,302.0){\rule[-0.200pt]{2.409pt}{0.400pt}}
\put(170.0,302.0){\rule[-0.200pt]{2.409pt}{0.400pt}}
\put(1429.0,302.0){\rule[-0.200pt]{2.409pt}{0.400pt}}
\put(170.0,302.0){\rule[-0.200pt]{2.409pt}{0.400pt}}
\put(1429.0,302.0){\rule[-0.200pt]{2.409pt}{0.400pt}}
\put(170.0,302.0){\rule[-0.200pt]{2.409pt}{0.400pt}}
\put(1429.0,302.0){\rule[-0.200pt]{2.409pt}{0.400pt}}
\put(170.0,302.0){\rule[-0.200pt]{2.409pt}{0.400pt}}
\put(1429.0,302.0){\rule[-0.200pt]{2.409pt}{0.400pt}}
\put(170.0,302.0){\rule[-0.200pt]{2.409pt}{0.400pt}}
\put(1429.0,302.0){\rule[-0.200pt]{2.409pt}{0.400pt}}
\put(170.0,302.0){\rule[-0.200pt]{2.409pt}{0.400pt}}
\put(1429.0,302.0){\rule[-0.200pt]{2.409pt}{0.400pt}}
\put(170.0,302.0){\rule[-0.200pt]{2.409pt}{0.400pt}}
\put(1429.0,302.0){\rule[-0.200pt]{2.409pt}{0.400pt}}
\put(170.0,302.0){\rule[-0.200pt]{2.409pt}{0.400pt}}
\put(1429.0,302.0){\rule[-0.200pt]{2.409pt}{0.400pt}}
\put(170.0,303.0){\rule[-0.200pt]{2.409pt}{0.400pt}}
\put(1429.0,303.0){\rule[-0.200pt]{2.409pt}{0.400pt}}
\put(170.0,303.0){\rule[-0.200pt]{2.409pt}{0.400pt}}
\put(1429.0,303.0){\rule[-0.200pt]{2.409pt}{0.400pt}}
\put(170.0,303.0){\rule[-0.200pt]{2.409pt}{0.400pt}}
\put(1429.0,303.0){\rule[-0.200pt]{2.409pt}{0.400pt}}
\put(170.0,303.0){\rule[-0.200pt]{2.409pt}{0.400pt}}
\put(1429.0,303.0){\rule[-0.200pt]{2.409pt}{0.400pt}}
\put(170.0,303.0){\rule[-0.200pt]{2.409pt}{0.400pt}}
\put(1429.0,303.0){\rule[-0.200pt]{2.409pt}{0.400pt}}
\put(170.0,303.0){\rule[-0.200pt]{2.409pt}{0.400pt}}
\put(1429.0,303.0){\rule[-0.200pt]{2.409pt}{0.400pt}}
\put(170.0,303.0){\rule[-0.200pt]{2.409pt}{0.400pt}}
\put(1429.0,303.0){\rule[-0.200pt]{2.409pt}{0.400pt}}
\put(170.0,303.0){\rule[-0.200pt]{2.409pt}{0.400pt}}
\put(1429.0,303.0){\rule[-0.200pt]{2.409pt}{0.400pt}}
\put(170.0,303.0){\rule[-0.200pt]{2.409pt}{0.400pt}}
\put(1429.0,303.0){\rule[-0.200pt]{2.409pt}{0.400pt}}
\put(170.0,304.0){\rule[-0.200pt]{2.409pt}{0.400pt}}
\put(1429.0,304.0){\rule[-0.200pt]{2.409pt}{0.400pt}}
\put(170.0,304.0){\rule[-0.200pt]{2.409pt}{0.400pt}}
\put(1429.0,304.0){\rule[-0.200pt]{2.409pt}{0.400pt}}
\put(170.0,304.0){\rule[-0.200pt]{2.409pt}{0.400pt}}
\put(1429.0,304.0){\rule[-0.200pt]{2.409pt}{0.400pt}}
\put(170.0,304.0){\rule[-0.200pt]{2.409pt}{0.400pt}}
\put(1429.0,304.0){\rule[-0.200pt]{2.409pt}{0.400pt}}
\put(170.0,304.0){\rule[-0.200pt]{2.409pt}{0.400pt}}
\put(1429.0,304.0){\rule[-0.200pt]{2.409pt}{0.400pt}}
\put(170.0,304.0){\rule[-0.200pt]{2.409pt}{0.400pt}}
\put(1429.0,304.0){\rule[-0.200pt]{2.409pt}{0.400pt}}
\put(170.0,304.0){\rule[-0.200pt]{2.409pt}{0.400pt}}
\put(1429.0,304.0){\rule[-0.200pt]{2.409pt}{0.400pt}}
\put(170.0,304.0){\rule[-0.200pt]{2.409pt}{0.400pt}}
\put(1429.0,304.0){\rule[-0.200pt]{2.409pt}{0.400pt}}
\put(170.0,304.0){\rule[-0.200pt]{2.409pt}{0.400pt}}
\put(1429.0,304.0){\rule[-0.200pt]{2.409pt}{0.400pt}}
\put(170.0,304.0){\rule[-0.200pt]{2.409pt}{0.400pt}}
\put(1429.0,304.0){\rule[-0.200pt]{2.409pt}{0.400pt}}
\put(170.0,305.0){\rule[-0.200pt]{2.409pt}{0.400pt}}
\put(1429.0,305.0){\rule[-0.200pt]{2.409pt}{0.400pt}}
\put(170.0,305.0){\rule[-0.200pt]{2.409pt}{0.400pt}}
\put(1429.0,305.0){\rule[-0.200pt]{2.409pt}{0.400pt}}
\put(170.0,305.0){\rule[-0.200pt]{2.409pt}{0.400pt}}
\put(1429.0,305.0){\rule[-0.200pt]{2.409pt}{0.400pt}}
\put(170.0,305.0){\rule[-0.200pt]{2.409pt}{0.400pt}}
\put(1429.0,305.0){\rule[-0.200pt]{2.409pt}{0.400pt}}
\put(170.0,305.0){\rule[-0.200pt]{2.409pt}{0.400pt}}
\put(1429.0,305.0){\rule[-0.200pt]{2.409pt}{0.400pt}}
\put(170.0,305.0){\rule[-0.200pt]{2.409pt}{0.400pt}}
\put(1429.0,305.0){\rule[-0.200pt]{2.409pt}{0.400pt}}
\put(170.0,305.0){\rule[-0.200pt]{2.409pt}{0.400pt}}
\put(1429.0,305.0){\rule[-0.200pt]{2.409pt}{0.400pt}}
\put(170.0,305.0){\rule[-0.200pt]{2.409pt}{0.400pt}}
\put(1429.0,305.0){\rule[-0.200pt]{2.409pt}{0.400pt}}
\put(170.0,305.0){\rule[-0.200pt]{2.409pt}{0.400pt}}
\put(1429.0,305.0){\rule[-0.200pt]{2.409pt}{0.400pt}}
\put(170.0,305.0){\rule[-0.200pt]{2.409pt}{0.400pt}}
\put(1429.0,305.0){\rule[-0.200pt]{2.409pt}{0.400pt}}
\put(170.0,306.0){\rule[-0.200pt]{2.409pt}{0.400pt}}
\put(1429.0,306.0){\rule[-0.200pt]{2.409pt}{0.400pt}}
\put(170.0,306.0){\rule[-0.200pt]{2.409pt}{0.400pt}}
\put(1429.0,306.0){\rule[-0.200pt]{2.409pt}{0.400pt}}
\put(170.0,306.0){\rule[-0.200pt]{2.409pt}{0.400pt}}
\put(1429.0,306.0){\rule[-0.200pt]{2.409pt}{0.400pt}}
\put(170.0,306.0){\rule[-0.200pt]{2.409pt}{0.400pt}}
\put(1429.0,306.0){\rule[-0.200pt]{2.409pt}{0.400pt}}
\put(170.0,306.0){\rule[-0.200pt]{2.409pt}{0.400pt}}
\put(1429.0,306.0){\rule[-0.200pt]{2.409pt}{0.400pt}}
\put(170.0,306.0){\rule[-0.200pt]{2.409pt}{0.400pt}}
\put(1429.0,306.0){\rule[-0.200pt]{2.409pt}{0.400pt}}
\put(170.0,306.0){\rule[-0.200pt]{2.409pt}{0.400pt}}
\put(1429.0,306.0){\rule[-0.200pt]{2.409pt}{0.400pt}}
\put(170.0,306.0){\rule[-0.200pt]{2.409pt}{0.400pt}}
\put(1429.0,306.0){\rule[-0.200pt]{2.409pt}{0.400pt}}
\put(170.0,306.0){\rule[-0.200pt]{2.409pt}{0.400pt}}
\put(1429.0,306.0){\rule[-0.200pt]{2.409pt}{0.400pt}}
\put(170.0,306.0){\rule[-0.200pt]{2.409pt}{0.400pt}}
\put(1429.0,306.0){\rule[-0.200pt]{2.409pt}{0.400pt}}
\put(170.0,306.0){\rule[-0.200pt]{2.409pt}{0.400pt}}
\put(1429.0,306.0){\rule[-0.200pt]{2.409pt}{0.400pt}}
\put(170.0,307.0){\rule[-0.200pt]{2.409pt}{0.400pt}}
\put(1429.0,307.0){\rule[-0.200pt]{2.409pt}{0.400pt}}
\put(170.0,307.0){\rule[-0.200pt]{2.409pt}{0.400pt}}
\put(1429.0,307.0){\rule[-0.200pt]{2.409pt}{0.400pt}}
\put(170.0,307.0){\rule[-0.200pt]{2.409pt}{0.400pt}}
\put(1429.0,307.0){\rule[-0.200pt]{2.409pt}{0.400pt}}
\put(170.0,307.0){\rule[-0.200pt]{2.409pt}{0.400pt}}
\put(1429.0,307.0){\rule[-0.200pt]{2.409pt}{0.400pt}}
\put(170.0,307.0){\rule[-0.200pt]{2.409pt}{0.400pt}}
\put(1429.0,307.0){\rule[-0.200pt]{2.409pt}{0.400pt}}
\put(170.0,307.0){\rule[-0.200pt]{2.409pt}{0.400pt}}
\put(1429.0,307.0){\rule[-0.200pt]{2.409pt}{0.400pt}}
\put(170.0,307.0){\rule[-0.200pt]{2.409pt}{0.400pt}}
\put(1429.0,307.0){\rule[-0.200pt]{2.409pt}{0.400pt}}
\put(170.0,307.0){\rule[-0.200pt]{2.409pt}{0.400pt}}
\put(1429.0,307.0){\rule[-0.200pt]{2.409pt}{0.400pt}}
\put(170.0,307.0){\rule[-0.200pt]{2.409pt}{0.400pt}}
\put(1429.0,307.0){\rule[-0.200pt]{2.409pt}{0.400pt}}
\put(170.0,307.0){\rule[-0.200pt]{2.409pt}{0.400pt}}
\put(1429.0,307.0){\rule[-0.200pt]{2.409pt}{0.400pt}}
\put(170.0,307.0){\rule[-0.200pt]{2.409pt}{0.400pt}}
\put(1429.0,307.0){\rule[-0.200pt]{2.409pt}{0.400pt}}
\put(170.0,308.0){\rule[-0.200pt]{2.409pt}{0.400pt}}
\put(1429.0,308.0){\rule[-0.200pt]{2.409pt}{0.400pt}}
\put(170.0,308.0){\rule[-0.200pt]{2.409pt}{0.400pt}}
\put(1429.0,308.0){\rule[-0.200pt]{2.409pt}{0.400pt}}
\put(170.0,308.0){\rule[-0.200pt]{2.409pt}{0.400pt}}
\put(1429.0,308.0){\rule[-0.200pt]{2.409pt}{0.400pt}}
\put(170.0,308.0){\rule[-0.200pt]{2.409pt}{0.400pt}}
\put(1429.0,308.0){\rule[-0.200pt]{2.409pt}{0.400pt}}
\put(170.0,308.0){\rule[-0.200pt]{2.409pt}{0.400pt}}
\put(1429.0,308.0){\rule[-0.200pt]{2.409pt}{0.400pt}}
\put(170.0,308.0){\rule[-0.200pt]{2.409pt}{0.400pt}}
\put(1429.0,308.0){\rule[-0.200pt]{2.409pt}{0.400pt}}
\put(170.0,308.0){\rule[-0.200pt]{2.409pt}{0.400pt}}
\put(1429.0,308.0){\rule[-0.200pt]{2.409pt}{0.400pt}}
\put(170.0,308.0){\rule[-0.200pt]{2.409pt}{0.400pt}}
\put(1429.0,308.0){\rule[-0.200pt]{2.409pt}{0.400pt}}
\put(170.0,308.0){\rule[-0.200pt]{2.409pt}{0.400pt}}
\put(1429.0,308.0){\rule[-0.200pt]{2.409pt}{0.400pt}}
\put(170.0,308.0){\rule[-0.200pt]{2.409pt}{0.400pt}}
\put(1429.0,308.0){\rule[-0.200pt]{2.409pt}{0.400pt}}
\put(170.0,308.0){\rule[-0.200pt]{2.409pt}{0.400pt}}
\put(1429.0,308.0){\rule[-0.200pt]{2.409pt}{0.400pt}}
\put(170.0,309.0){\rule[-0.200pt]{2.409pt}{0.400pt}}
\put(1429.0,309.0){\rule[-0.200pt]{2.409pt}{0.400pt}}
\put(170.0,309.0){\rule[-0.200pt]{2.409pt}{0.400pt}}
\put(1429.0,309.0){\rule[-0.200pt]{2.409pt}{0.400pt}}
\put(170.0,309.0){\rule[-0.200pt]{2.409pt}{0.400pt}}
\put(1429.0,309.0){\rule[-0.200pt]{2.409pt}{0.400pt}}
\put(170.0,309.0){\rule[-0.200pt]{2.409pt}{0.400pt}}
\put(1429.0,309.0){\rule[-0.200pt]{2.409pt}{0.400pt}}
\put(170.0,309.0){\rule[-0.200pt]{2.409pt}{0.400pt}}
\put(1429.0,309.0){\rule[-0.200pt]{2.409pt}{0.400pt}}
\put(170.0,309.0){\rule[-0.200pt]{2.409pt}{0.400pt}}
\put(1429.0,309.0){\rule[-0.200pt]{2.409pt}{0.400pt}}
\put(170.0,309.0){\rule[-0.200pt]{2.409pt}{0.400pt}}
\put(1429.0,309.0){\rule[-0.200pt]{2.409pt}{0.400pt}}
\put(170.0,309.0){\rule[-0.200pt]{2.409pt}{0.400pt}}
\put(1429.0,309.0){\rule[-0.200pt]{2.409pt}{0.400pt}}
\put(170.0,309.0){\rule[-0.200pt]{2.409pt}{0.400pt}}
\put(1429.0,309.0){\rule[-0.200pt]{2.409pt}{0.400pt}}
\put(170.0,309.0){\rule[-0.200pt]{2.409pt}{0.400pt}}
\put(1429.0,309.0){\rule[-0.200pt]{2.409pt}{0.400pt}}
\put(170.0,309.0){\rule[-0.200pt]{2.409pt}{0.400pt}}
\put(1429.0,309.0){\rule[-0.200pt]{2.409pt}{0.400pt}}
\put(170.0,310.0){\rule[-0.200pt]{2.409pt}{0.400pt}}
\put(1429.0,310.0){\rule[-0.200pt]{2.409pt}{0.400pt}}
\put(170.0,310.0){\rule[-0.200pt]{2.409pt}{0.400pt}}
\put(1429.0,310.0){\rule[-0.200pt]{2.409pt}{0.400pt}}
\put(170.0,310.0){\rule[-0.200pt]{2.409pt}{0.400pt}}
\put(1429.0,310.0){\rule[-0.200pt]{2.409pt}{0.400pt}}
\put(170.0,310.0){\rule[-0.200pt]{2.409pt}{0.400pt}}
\put(1429.0,310.0){\rule[-0.200pt]{2.409pt}{0.400pt}}
\put(170.0,310.0){\rule[-0.200pt]{2.409pt}{0.400pt}}
\put(1429.0,310.0){\rule[-0.200pt]{2.409pt}{0.400pt}}
\put(170.0,310.0){\rule[-0.200pt]{2.409pt}{0.400pt}}
\put(1429.0,310.0){\rule[-0.200pt]{2.409pt}{0.400pt}}
\put(170.0,310.0){\rule[-0.200pt]{2.409pt}{0.400pt}}
\put(1429.0,310.0){\rule[-0.200pt]{2.409pt}{0.400pt}}
\put(170.0,310.0){\rule[-0.200pt]{2.409pt}{0.400pt}}
\put(1429.0,310.0){\rule[-0.200pt]{2.409pt}{0.400pt}}
\put(170.0,310.0){\rule[-0.200pt]{2.409pt}{0.400pt}}
\put(1429.0,310.0){\rule[-0.200pt]{2.409pt}{0.400pt}}
\put(170.0,310.0){\rule[-0.200pt]{2.409pt}{0.400pt}}
\put(1429.0,310.0){\rule[-0.200pt]{2.409pt}{0.400pt}}
\put(170.0,310.0){\rule[-0.200pt]{2.409pt}{0.400pt}}
\put(1429.0,310.0){\rule[-0.200pt]{2.409pt}{0.400pt}}
\put(170.0,310.0){\rule[-0.200pt]{2.409pt}{0.400pt}}
\put(1429.0,310.0){\rule[-0.200pt]{2.409pt}{0.400pt}}
\put(170.0,311.0){\rule[-0.200pt]{2.409pt}{0.400pt}}
\put(1429.0,311.0){\rule[-0.200pt]{2.409pt}{0.400pt}}
\put(170.0,311.0){\rule[-0.200pt]{2.409pt}{0.400pt}}
\put(1429.0,311.0){\rule[-0.200pt]{2.409pt}{0.400pt}}
\put(170.0,311.0){\rule[-0.200pt]{2.409pt}{0.400pt}}
\put(1429.0,311.0){\rule[-0.200pt]{2.409pt}{0.400pt}}
\put(170.0,311.0){\rule[-0.200pt]{2.409pt}{0.400pt}}
\put(1429.0,311.0){\rule[-0.200pt]{2.409pt}{0.400pt}}
\put(170.0,311.0){\rule[-0.200pt]{2.409pt}{0.400pt}}
\put(1429.0,311.0){\rule[-0.200pt]{2.409pt}{0.400pt}}
\put(170.0,311.0){\rule[-0.200pt]{2.409pt}{0.400pt}}
\put(1429.0,311.0){\rule[-0.200pt]{2.409pt}{0.400pt}}
\put(170.0,311.0){\rule[-0.200pt]{2.409pt}{0.400pt}}
\put(1429.0,311.0){\rule[-0.200pt]{2.409pt}{0.400pt}}
\put(170.0,311.0){\rule[-0.200pt]{2.409pt}{0.400pt}}
\put(1429.0,311.0){\rule[-0.200pt]{2.409pt}{0.400pt}}
\put(170.0,311.0){\rule[-0.200pt]{2.409pt}{0.400pt}}
\put(1429.0,311.0){\rule[-0.200pt]{2.409pt}{0.400pt}}
\put(170.0,311.0){\rule[-0.200pt]{2.409pt}{0.400pt}}
\put(1429.0,311.0){\rule[-0.200pt]{2.409pt}{0.400pt}}
\put(170.0,311.0){\rule[-0.200pt]{2.409pt}{0.400pt}}
\put(1429.0,311.0){\rule[-0.200pt]{2.409pt}{0.400pt}}
\put(170.0,311.0){\rule[-0.200pt]{2.409pt}{0.400pt}}
\put(1429.0,311.0){\rule[-0.200pt]{2.409pt}{0.400pt}}
\put(170.0,312.0){\rule[-0.200pt]{2.409pt}{0.400pt}}
\put(1429.0,312.0){\rule[-0.200pt]{2.409pt}{0.400pt}}
\put(170.0,312.0){\rule[-0.200pt]{2.409pt}{0.400pt}}
\put(1429.0,312.0){\rule[-0.200pt]{2.409pt}{0.400pt}}
\put(170.0,312.0){\rule[-0.200pt]{2.409pt}{0.400pt}}
\put(1429.0,312.0){\rule[-0.200pt]{2.409pt}{0.400pt}}
\put(170.0,312.0){\rule[-0.200pt]{2.409pt}{0.400pt}}
\put(1429.0,312.0){\rule[-0.200pt]{2.409pt}{0.400pt}}
\put(170.0,312.0){\rule[-0.200pt]{2.409pt}{0.400pt}}
\put(1429.0,312.0){\rule[-0.200pt]{2.409pt}{0.400pt}}
\put(170.0,312.0){\rule[-0.200pt]{2.409pt}{0.400pt}}
\put(1429.0,312.0){\rule[-0.200pt]{2.409pt}{0.400pt}}
\put(170.0,312.0){\rule[-0.200pt]{2.409pt}{0.400pt}}
\put(1429.0,312.0){\rule[-0.200pt]{2.409pt}{0.400pt}}
\put(170.0,312.0){\rule[-0.200pt]{2.409pt}{0.400pt}}
\put(1429.0,312.0){\rule[-0.200pt]{2.409pt}{0.400pt}}
\put(170.0,312.0){\rule[-0.200pt]{2.409pt}{0.400pt}}
\put(1429.0,312.0){\rule[-0.200pt]{2.409pt}{0.400pt}}
\put(170.0,312.0){\rule[-0.200pt]{2.409pt}{0.400pt}}
\put(1429.0,312.0){\rule[-0.200pt]{2.409pt}{0.400pt}}
\put(170.0,312.0){\rule[-0.200pt]{2.409pt}{0.400pt}}
\put(1429.0,312.0){\rule[-0.200pt]{2.409pt}{0.400pt}}
\put(170.0,312.0){\rule[-0.200pt]{2.409pt}{0.400pt}}
\put(1429.0,312.0){\rule[-0.200pt]{2.409pt}{0.400pt}}
\put(170.0,313.0){\rule[-0.200pt]{2.409pt}{0.400pt}}
\put(1429.0,313.0){\rule[-0.200pt]{2.409pt}{0.400pt}}
\put(170.0,313.0){\rule[-0.200pt]{2.409pt}{0.400pt}}
\put(1429.0,313.0){\rule[-0.200pt]{2.409pt}{0.400pt}}
\put(170.0,313.0){\rule[-0.200pt]{2.409pt}{0.400pt}}
\put(1429.0,313.0){\rule[-0.200pt]{2.409pt}{0.400pt}}
\put(170.0,313.0){\rule[-0.200pt]{2.409pt}{0.400pt}}
\put(1429.0,313.0){\rule[-0.200pt]{2.409pt}{0.400pt}}
\put(170.0,313.0){\rule[-0.200pt]{2.409pt}{0.400pt}}
\put(1429.0,313.0){\rule[-0.200pt]{2.409pt}{0.400pt}}
\put(170.0,313.0){\rule[-0.200pt]{2.409pt}{0.400pt}}
\put(1429.0,313.0){\rule[-0.200pt]{2.409pt}{0.400pt}}
\put(170.0,313.0){\rule[-0.200pt]{2.409pt}{0.400pt}}
\put(1429.0,313.0){\rule[-0.200pt]{2.409pt}{0.400pt}}
\put(170.0,313.0){\rule[-0.200pt]{2.409pt}{0.400pt}}
\put(1429.0,313.0){\rule[-0.200pt]{2.409pt}{0.400pt}}
\put(170.0,313.0){\rule[-0.200pt]{2.409pt}{0.400pt}}
\put(1429.0,313.0){\rule[-0.200pt]{2.409pt}{0.400pt}}
\put(170.0,313.0){\rule[-0.200pt]{2.409pt}{0.400pt}}
\put(1429.0,313.0){\rule[-0.200pt]{2.409pt}{0.400pt}}
\put(170.0,313.0){\rule[-0.200pt]{2.409pt}{0.400pt}}
\put(1429.0,313.0){\rule[-0.200pt]{2.409pt}{0.400pt}}
\put(170.0,313.0){\rule[-0.200pt]{2.409pt}{0.400pt}}
\put(1429.0,313.0){\rule[-0.200pt]{2.409pt}{0.400pt}}
\put(170.0,313.0){\rule[-0.200pt]{2.409pt}{0.400pt}}
\put(1429.0,313.0){\rule[-0.200pt]{2.409pt}{0.400pt}}
\put(170.0,314.0){\rule[-0.200pt]{2.409pt}{0.400pt}}
\put(1429.0,314.0){\rule[-0.200pt]{2.409pt}{0.400pt}}
\put(170.0,314.0){\rule[-0.200pt]{2.409pt}{0.400pt}}
\put(1429.0,314.0){\rule[-0.200pt]{2.409pt}{0.400pt}}
\put(170.0,314.0){\rule[-0.200pt]{2.409pt}{0.400pt}}
\put(1429.0,314.0){\rule[-0.200pt]{2.409pt}{0.400pt}}
\put(170.0,314.0){\rule[-0.200pt]{2.409pt}{0.400pt}}
\put(1429.0,314.0){\rule[-0.200pt]{2.409pt}{0.400pt}}
\put(170.0,314.0){\rule[-0.200pt]{2.409pt}{0.400pt}}
\put(1429.0,314.0){\rule[-0.200pt]{2.409pt}{0.400pt}}
\put(170.0,314.0){\rule[-0.200pt]{2.409pt}{0.400pt}}
\put(1429.0,314.0){\rule[-0.200pt]{2.409pt}{0.400pt}}
\put(170.0,314.0){\rule[-0.200pt]{2.409pt}{0.400pt}}
\put(1429.0,314.0){\rule[-0.200pt]{2.409pt}{0.400pt}}
\put(170.0,314.0){\rule[-0.200pt]{2.409pt}{0.400pt}}
\put(1429.0,314.0){\rule[-0.200pt]{2.409pt}{0.400pt}}
\put(170.0,314.0){\rule[-0.200pt]{2.409pt}{0.400pt}}
\put(1429.0,314.0){\rule[-0.200pt]{2.409pt}{0.400pt}}
\put(170.0,314.0){\rule[-0.200pt]{2.409pt}{0.400pt}}
\put(1429.0,314.0){\rule[-0.200pt]{2.409pt}{0.400pt}}
\put(170.0,314.0){\rule[-0.200pt]{2.409pt}{0.400pt}}
\put(1429.0,314.0){\rule[-0.200pt]{2.409pt}{0.400pt}}
\put(170.0,314.0){\rule[-0.200pt]{2.409pt}{0.400pt}}
\put(1429.0,314.0){\rule[-0.200pt]{2.409pt}{0.400pt}}
\put(170.0,314.0){\rule[-0.200pt]{2.409pt}{0.400pt}}
\put(1429.0,314.0){\rule[-0.200pt]{2.409pt}{0.400pt}}
\put(170.0,315.0){\rule[-0.200pt]{2.409pt}{0.400pt}}
\put(1429.0,315.0){\rule[-0.200pt]{2.409pt}{0.400pt}}
\put(170.0,315.0){\rule[-0.200pt]{2.409pt}{0.400pt}}
\put(1429.0,315.0){\rule[-0.200pt]{2.409pt}{0.400pt}}
\put(170.0,315.0){\rule[-0.200pt]{2.409pt}{0.400pt}}
\put(1429.0,315.0){\rule[-0.200pt]{2.409pt}{0.400pt}}
\put(170.0,315.0){\rule[-0.200pt]{2.409pt}{0.400pt}}
\put(1429.0,315.0){\rule[-0.200pt]{2.409pt}{0.400pt}}
\put(170.0,315.0){\rule[-0.200pt]{2.409pt}{0.400pt}}
\put(1429.0,315.0){\rule[-0.200pt]{2.409pt}{0.400pt}}
\put(170.0,315.0){\rule[-0.200pt]{2.409pt}{0.400pt}}
\put(1429.0,315.0){\rule[-0.200pt]{2.409pt}{0.400pt}}
\put(170.0,315.0){\rule[-0.200pt]{2.409pt}{0.400pt}}
\put(1429.0,315.0){\rule[-0.200pt]{2.409pt}{0.400pt}}
\put(170.0,315.0){\rule[-0.200pt]{2.409pt}{0.400pt}}
\put(1429.0,315.0){\rule[-0.200pt]{2.409pt}{0.400pt}}
\put(170.0,315.0){\rule[-0.200pt]{2.409pt}{0.400pt}}
\put(1429.0,315.0){\rule[-0.200pt]{2.409pt}{0.400pt}}
\put(170.0,315.0){\rule[-0.200pt]{2.409pt}{0.400pt}}
\put(1429.0,315.0){\rule[-0.200pt]{2.409pt}{0.400pt}}
\put(170.0,315.0){\rule[-0.200pt]{2.409pt}{0.400pt}}
\put(1429.0,315.0){\rule[-0.200pt]{2.409pt}{0.400pt}}
\put(170.0,315.0){\rule[-0.200pt]{2.409pt}{0.400pt}}
\put(1429.0,315.0){\rule[-0.200pt]{2.409pt}{0.400pt}}
\put(170.0,315.0){\rule[-0.200pt]{2.409pt}{0.400pt}}
\put(1429.0,315.0){\rule[-0.200pt]{2.409pt}{0.400pt}}
\put(170.0,316.0){\rule[-0.200pt]{2.409pt}{0.400pt}}
\put(1429.0,316.0){\rule[-0.200pt]{2.409pt}{0.400pt}}
\put(170.0,316.0){\rule[-0.200pt]{2.409pt}{0.400pt}}
\put(1429.0,316.0){\rule[-0.200pt]{2.409pt}{0.400pt}}
\put(170.0,316.0){\rule[-0.200pt]{2.409pt}{0.400pt}}
\put(1429.0,316.0){\rule[-0.200pt]{2.409pt}{0.400pt}}
\put(170.0,316.0){\rule[-0.200pt]{2.409pt}{0.400pt}}
\put(1429.0,316.0){\rule[-0.200pt]{2.409pt}{0.400pt}}
\put(170.0,316.0){\rule[-0.200pt]{2.409pt}{0.400pt}}
\put(1429.0,316.0){\rule[-0.200pt]{2.409pt}{0.400pt}}
\put(170.0,316.0){\rule[-0.200pt]{2.409pt}{0.400pt}}
\put(1429.0,316.0){\rule[-0.200pt]{2.409pt}{0.400pt}}
\put(170.0,316.0){\rule[-0.200pt]{2.409pt}{0.400pt}}
\put(1429.0,316.0){\rule[-0.200pt]{2.409pt}{0.400pt}}
\put(170.0,316.0){\rule[-0.200pt]{2.409pt}{0.400pt}}
\put(1429.0,316.0){\rule[-0.200pt]{2.409pt}{0.400pt}}
\put(170.0,316.0){\rule[-0.200pt]{2.409pt}{0.400pt}}
\put(1429.0,316.0){\rule[-0.200pt]{2.409pt}{0.400pt}}
\put(170.0,316.0){\rule[-0.200pt]{2.409pt}{0.400pt}}
\put(1429.0,316.0){\rule[-0.200pt]{2.409pt}{0.400pt}}
\put(170.0,316.0){\rule[-0.200pt]{2.409pt}{0.400pt}}
\put(1429.0,316.0){\rule[-0.200pt]{2.409pt}{0.400pt}}
\put(170.0,316.0){\rule[-0.200pt]{2.409pt}{0.400pt}}
\put(1429.0,316.0){\rule[-0.200pt]{2.409pt}{0.400pt}}
\put(170.0,316.0){\rule[-0.200pt]{2.409pt}{0.400pt}}
\put(1429.0,316.0){\rule[-0.200pt]{2.409pt}{0.400pt}}
\put(170.0,316.0){\rule[-0.200pt]{2.409pt}{0.400pt}}
\put(1429.0,316.0){\rule[-0.200pt]{2.409pt}{0.400pt}}
\put(170.0,317.0){\rule[-0.200pt]{2.409pt}{0.400pt}}
\put(1429.0,317.0){\rule[-0.200pt]{2.409pt}{0.400pt}}
\put(170.0,317.0){\rule[-0.200pt]{2.409pt}{0.400pt}}
\put(1429.0,317.0){\rule[-0.200pt]{2.409pt}{0.400pt}}
\put(170.0,317.0){\rule[-0.200pt]{2.409pt}{0.400pt}}
\put(1429.0,317.0){\rule[-0.200pt]{2.409pt}{0.400pt}}
\put(170.0,317.0){\rule[-0.200pt]{2.409pt}{0.400pt}}
\put(1429.0,317.0){\rule[-0.200pt]{2.409pt}{0.400pt}}
\put(170.0,317.0){\rule[-0.200pt]{2.409pt}{0.400pt}}
\put(1429.0,317.0){\rule[-0.200pt]{2.409pt}{0.400pt}}
\put(170.0,317.0){\rule[-0.200pt]{2.409pt}{0.400pt}}
\put(1429.0,317.0){\rule[-0.200pt]{2.409pt}{0.400pt}}
\put(170.0,317.0){\rule[-0.200pt]{2.409pt}{0.400pt}}
\put(1429.0,317.0){\rule[-0.200pt]{2.409pt}{0.400pt}}
\put(170.0,317.0){\rule[-0.200pt]{2.409pt}{0.400pt}}
\put(1429.0,317.0){\rule[-0.200pt]{2.409pt}{0.400pt}}
\put(170.0,317.0){\rule[-0.200pt]{2.409pt}{0.400pt}}
\put(1429.0,317.0){\rule[-0.200pt]{2.409pt}{0.400pt}}
\put(170.0,317.0){\rule[-0.200pt]{2.409pt}{0.400pt}}
\put(1429.0,317.0){\rule[-0.200pt]{2.409pt}{0.400pt}}
\put(170.0,317.0){\rule[-0.200pt]{2.409pt}{0.400pt}}
\put(1429.0,317.0){\rule[-0.200pt]{2.409pt}{0.400pt}}
\put(170.0,317.0){\rule[-0.200pt]{2.409pt}{0.400pt}}
\put(1429.0,317.0){\rule[-0.200pt]{2.409pt}{0.400pt}}
\put(170.0,317.0){\rule[-0.200pt]{2.409pt}{0.400pt}}
\put(1429.0,317.0){\rule[-0.200pt]{2.409pt}{0.400pt}}
\put(170.0,317.0){\rule[-0.200pt]{2.409pt}{0.400pt}}
\put(1429.0,317.0){\rule[-0.200pt]{2.409pt}{0.400pt}}
\put(170.0,318.0){\rule[-0.200pt]{2.409pt}{0.400pt}}
\put(1429.0,318.0){\rule[-0.200pt]{2.409pt}{0.400pt}}
\put(170.0,318.0){\rule[-0.200pt]{2.409pt}{0.400pt}}
\put(1429.0,318.0){\rule[-0.200pt]{2.409pt}{0.400pt}}
\put(170.0,318.0){\rule[-0.200pt]{2.409pt}{0.400pt}}
\put(1429.0,318.0){\rule[-0.200pt]{2.409pt}{0.400pt}}
\put(170.0,318.0){\rule[-0.200pt]{2.409pt}{0.400pt}}
\put(1429.0,318.0){\rule[-0.200pt]{2.409pt}{0.400pt}}
\put(170.0,318.0){\rule[-0.200pt]{2.409pt}{0.400pt}}
\put(1429.0,318.0){\rule[-0.200pt]{2.409pt}{0.400pt}}
\put(170.0,318.0){\rule[-0.200pt]{2.409pt}{0.400pt}}
\put(1429.0,318.0){\rule[-0.200pt]{2.409pt}{0.400pt}}
\put(170.0,318.0){\rule[-0.200pt]{2.409pt}{0.400pt}}
\put(1429.0,318.0){\rule[-0.200pt]{2.409pt}{0.400pt}}
\put(170.0,318.0){\rule[-0.200pt]{2.409pt}{0.400pt}}
\put(1429.0,318.0){\rule[-0.200pt]{2.409pt}{0.400pt}}
\put(170.0,318.0){\rule[-0.200pt]{2.409pt}{0.400pt}}
\put(1429.0,318.0){\rule[-0.200pt]{2.409pt}{0.400pt}}
\put(170.0,318.0){\rule[-0.200pt]{2.409pt}{0.400pt}}
\put(1429.0,318.0){\rule[-0.200pt]{2.409pt}{0.400pt}}
\put(170.0,318.0){\rule[-0.200pt]{2.409pt}{0.400pt}}
\put(1429.0,318.0){\rule[-0.200pt]{2.409pt}{0.400pt}}
\put(170.0,318.0){\rule[-0.200pt]{2.409pt}{0.400pt}}
\put(1429.0,318.0){\rule[-0.200pt]{2.409pt}{0.400pt}}
\put(170.0,318.0){\rule[-0.200pt]{2.409pt}{0.400pt}}
\put(1429.0,318.0){\rule[-0.200pt]{2.409pt}{0.400pt}}
\put(170.0,318.0){\rule[-0.200pt]{2.409pt}{0.400pt}}
\put(1429.0,318.0){\rule[-0.200pt]{2.409pt}{0.400pt}}
\put(170.0,319.0){\rule[-0.200pt]{2.409pt}{0.400pt}}
\put(1429.0,319.0){\rule[-0.200pt]{2.409pt}{0.400pt}}
\put(170.0,319.0){\rule[-0.200pt]{2.409pt}{0.400pt}}
\put(1429.0,319.0){\rule[-0.200pt]{2.409pt}{0.400pt}}
\put(170.0,319.0){\rule[-0.200pt]{2.409pt}{0.400pt}}
\put(1429.0,319.0){\rule[-0.200pt]{2.409pt}{0.400pt}}
\put(170.0,319.0){\rule[-0.200pt]{2.409pt}{0.400pt}}
\put(1429.0,319.0){\rule[-0.200pt]{2.409pt}{0.400pt}}
\put(170.0,319.0){\rule[-0.200pt]{2.409pt}{0.400pt}}
\put(1429.0,319.0){\rule[-0.200pt]{2.409pt}{0.400pt}}
\put(170.0,319.0){\rule[-0.200pt]{2.409pt}{0.400pt}}
\put(1429.0,319.0){\rule[-0.200pt]{2.409pt}{0.400pt}}
\put(170.0,319.0){\rule[-0.200pt]{2.409pt}{0.400pt}}
\put(1429.0,319.0){\rule[-0.200pt]{2.409pt}{0.400pt}}
\put(170.0,319.0){\rule[-0.200pt]{2.409pt}{0.400pt}}
\put(1429.0,319.0){\rule[-0.200pt]{2.409pt}{0.400pt}}
\put(170.0,319.0){\rule[-0.200pt]{2.409pt}{0.400pt}}
\put(1429.0,319.0){\rule[-0.200pt]{2.409pt}{0.400pt}}
\put(170.0,319.0){\rule[-0.200pt]{2.409pt}{0.400pt}}
\put(1429.0,319.0){\rule[-0.200pt]{2.409pt}{0.400pt}}
\put(170.0,319.0){\rule[-0.200pt]{2.409pt}{0.400pt}}
\put(1429.0,319.0){\rule[-0.200pt]{2.409pt}{0.400pt}}
\put(170.0,319.0){\rule[-0.200pt]{2.409pt}{0.400pt}}
\put(1429.0,319.0){\rule[-0.200pt]{2.409pt}{0.400pt}}
\put(170.0,319.0){\rule[-0.200pt]{2.409pt}{0.400pt}}
\put(1429.0,319.0){\rule[-0.200pt]{2.409pt}{0.400pt}}
\put(170.0,319.0){\rule[-0.200pt]{2.409pt}{0.400pt}}
\put(1429.0,319.0){\rule[-0.200pt]{2.409pt}{0.400pt}}
\put(170.0,319.0){\rule[-0.200pt]{2.409pt}{0.400pt}}
\put(1429.0,319.0){\rule[-0.200pt]{2.409pt}{0.400pt}}
\put(170.0,320.0){\rule[-0.200pt]{2.409pt}{0.400pt}}
\put(1429.0,320.0){\rule[-0.200pt]{2.409pt}{0.400pt}}
\put(170.0,320.0){\rule[-0.200pt]{2.409pt}{0.400pt}}
\put(1429.0,320.0){\rule[-0.200pt]{2.409pt}{0.400pt}}
\put(170.0,320.0){\rule[-0.200pt]{2.409pt}{0.400pt}}
\put(1429.0,320.0){\rule[-0.200pt]{2.409pt}{0.400pt}}
\put(170.0,320.0){\rule[-0.200pt]{2.409pt}{0.400pt}}
\put(1429.0,320.0){\rule[-0.200pt]{2.409pt}{0.400pt}}
\put(170.0,320.0){\rule[-0.200pt]{2.409pt}{0.400pt}}
\put(1429.0,320.0){\rule[-0.200pt]{2.409pt}{0.400pt}}
\put(170.0,320.0){\rule[-0.200pt]{2.409pt}{0.400pt}}
\put(1429.0,320.0){\rule[-0.200pt]{2.409pt}{0.400pt}}
\put(170.0,320.0){\rule[-0.200pt]{2.409pt}{0.400pt}}
\put(1429.0,320.0){\rule[-0.200pt]{2.409pt}{0.400pt}}
\put(170.0,320.0){\rule[-0.200pt]{2.409pt}{0.400pt}}
\put(1429.0,320.0){\rule[-0.200pt]{2.409pt}{0.400pt}}
\put(170.0,320.0){\rule[-0.200pt]{2.409pt}{0.400pt}}
\put(1429.0,320.0){\rule[-0.200pt]{2.409pt}{0.400pt}}
\put(170.0,320.0){\rule[-0.200pt]{2.409pt}{0.400pt}}
\put(1429.0,320.0){\rule[-0.200pt]{2.409pt}{0.400pt}}
\put(170.0,320.0){\rule[-0.200pt]{2.409pt}{0.400pt}}
\put(1429.0,320.0){\rule[-0.200pt]{2.409pt}{0.400pt}}
\put(170.0,320.0){\rule[-0.200pt]{2.409pt}{0.400pt}}
\put(1429.0,320.0){\rule[-0.200pt]{2.409pt}{0.400pt}}
\put(170.0,320.0){\rule[-0.200pt]{2.409pt}{0.400pt}}
\put(1429.0,320.0){\rule[-0.200pt]{2.409pt}{0.400pt}}
\put(170.0,320.0){\rule[-0.200pt]{2.409pt}{0.400pt}}
\put(1429.0,320.0){\rule[-0.200pt]{2.409pt}{0.400pt}}
\put(170.0,320.0){\rule[-0.200pt]{2.409pt}{0.400pt}}
\put(1429.0,320.0){\rule[-0.200pt]{2.409pt}{0.400pt}}
\put(170.0,321.0){\rule[-0.200pt]{2.409pt}{0.400pt}}
\put(1429.0,321.0){\rule[-0.200pt]{2.409pt}{0.400pt}}
\put(170.0,321.0){\rule[-0.200pt]{2.409pt}{0.400pt}}
\put(1429.0,321.0){\rule[-0.200pt]{2.409pt}{0.400pt}}
\put(170.0,321.0){\rule[-0.200pt]{2.409pt}{0.400pt}}
\put(1429.0,321.0){\rule[-0.200pt]{2.409pt}{0.400pt}}
\put(170.0,321.0){\rule[-0.200pt]{2.409pt}{0.400pt}}
\put(1429.0,321.0){\rule[-0.200pt]{2.409pt}{0.400pt}}
\put(170.0,321.0){\rule[-0.200pt]{2.409pt}{0.400pt}}
\put(1429.0,321.0){\rule[-0.200pt]{2.409pt}{0.400pt}}
\put(170.0,321.0){\rule[-0.200pt]{2.409pt}{0.400pt}}
\put(1429.0,321.0){\rule[-0.200pt]{2.409pt}{0.400pt}}
\put(170.0,321.0){\rule[-0.200pt]{2.409pt}{0.400pt}}
\put(1429.0,321.0){\rule[-0.200pt]{2.409pt}{0.400pt}}
\put(170.0,321.0){\rule[-0.200pt]{2.409pt}{0.400pt}}
\put(1429.0,321.0){\rule[-0.200pt]{2.409pt}{0.400pt}}
\put(170.0,321.0){\rule[-0.200pt]{2.409pt}{0.400pt}}
\put(1429.0,321.0){\rule[-0.200pt]{2.409pt}{0.400pt}}
\put(170.0,321.0){\rule[-0.200pt]{2.409pt}{0.400pt}}
\put(1429.0,321.0){\rule[-0.200pt]{2.409pt}{0.400pt}}
\put(170.0,321.0){\rule[-0.200pt]{2.409pt}{0.400pt}}
\put(1429.0,321.0){\rule[-0.200pt]{2.409pt}{0.400pt}}
\put(170.0,321.0){\rule[-0.200pt]{2.409pt}{0.400pt}}
\put(1429.0,321.0){\rule[-0.200pt]{2.409pt}{0.400pt}}
\put(170.0,321.0){\rule[-0.200pt]{2.409pt}{0.400pt}}
\put(1429.0,321.0){\rule[-0.200pt]{2.409pt}{0.400pt}}
\put(170.0,321.0){\rule[-0.200pt]{2.409pt}{0.400pt}}
\put(1429.0,321.0){\rule[-0.200pt]{2.409pt}{0.400pt}}
\put(170.0,321.0){\rule[-0.200pt]{2.409pt}{0.400pt}}
\put(1429.0,321.0){\rule[-0.200pt]{2.409pt}{0.400pt}}
\put(170.0,321.0){\rule[-0.200pt]{2.409pt}{0.400pt}}
\put(1429.0,321.0){\rule[-0.200pt]{2.409pt}{0.400pt}}
\put(170.0,322.0){\rule[-0.200pt]{2.409pt}{0.400pt}}
\put(1429.0,322.0){\rule[-0.200pt]{2.409pt}{0.400pt}}
\put(170.0,322.0){\rule[-0.200pt]{2.409pt}{0.400pt}}
\put(1429.0,322.0){\rule[-0.200pt]{2.409pt}{0.400pt}}
\put(170.0,322.0){\rule[-0.200pt]{2.409pt}{0.400pt}}
\put(1429.0,322.0){\rule[-0.200pt]{2.409pt}{0.400pt}}
\put(170.0,322.0){\rule[-0.200pt]{2.409pt}{0.400pt}}
\put(1429.0,322.0){\rule[-0.200pt]{2.409pt}{0.400pt}}
\put(170.0,322.0){\rule[-0.200pt]{2.409pt}{0.400pt}}
\put(1429.0,322.0){\rule[-0.200pt]{2.409pt}{0.400pt}}
\put(170.0,322.0){\rule[-0.200pt]{2.409pt}{0.400pt}}
\put(1429.0,322.0){\rule[-0.200pt]{2.409pt}{0.400pt}}
\put(170.0,322.0){\rule[-0.200pt]{2.409pt}{0.400pt}}
\put(1429.0,322.0){\rule[-0.200pt]{2.409pt}{0.400pt}}
\put(170.0,322.0){\rule[-0.200pt]{2.409pt}{0.400pt}}
\put(1429.0,322.0){\rule[-0.200pt]{2.409pt}{0.400pt}}
\put(170.0,322.0){\rule[-0.200pt]{2.409pt}{0.400pt}}
\put(1429.0,322.0){\rule[-0.200pt]{2.409pt}{0.400pt}}
\put(170.0,322.0){\rule[-0.200pt]{2.409pt}{0.400pt}}
\put(1429.0,322.0){\rule[-0.200pt]{2.409pt}{0.400pt}}
\put(170.0,322.0){\rule[-0.200pt]{2.409pt}{0.400pt}}
\put(1429.0,322.0){\rule[-0.200pt]{2.409pt}{0.400pt}}
\put(170.0,322.0){\rule[-0.200pt]{2.409pt}{0.400pt}}
\put(1429.0,322.0){\rule[-0.200pt]{2.409pt}{0.400pt}}
\put(170.0,322.0){\rule[-0.200pt]{2.409pt}{0.400pt}}
\put(1429.0,322.0){\rule[-0.200pt]{2.409pt}{0.400pt}}
\put(170.0,322.0){\rule[-0.200pt]{2.409pt}{0.400pt}}
\put(1429.0,322.0){\rule[-0.200pt]{2.409pt}{0.400pt}}
\put(170.0,322.0){\rule[-0.200pt]{2.409pt}{0.400pt}}
\put(1429.0,322.0){\rule[-0.200pt]{2.409pt}{0.400pt}}
\put(170.0,322.0){\rule[-0.200pt]{2.409pt}{0.400pt}}
\put(1429.0,322.0){\rule[-0.200pt]{2.409pt}{0.400pt}}
\put(170.0,323.0){\rule[-0.200pt]{2.409pt}{0.400pt}}
\put(1429.0,323.0){\rule[-0.200pt]{2.409pt}{0.400pt}}
\put(170.0,323.0){\rule[-0.200pt]{2.409pt}{0.400pt}}
\put(1429.0,323.0){\rule[-0.200pt]{2.409pt}{0.400pt}}
\put(170.0,323.0){\rule[-0.200pt]{2.409pt}{0.400pt}}
\put(1429.0,323.0){\rule[-0.200pt]{2.409pt}{0.400pt}}
\put(170.0,323.0){\rule[-0.200pt]{2.409pt}{0.400pt}}
\put(1429.0,323.0){\rule[-0.200pt]{2.409pt}{0.400pt}}
\put(170.0,323.0){\rule[-0.200pt]{2.409pt}{0.400pt}}
\put(1429.0,323.0){\rule[-0.200pt]{2.409pt}{0.400pt}}
\put(170.0,323.0){\rule[-0.200pt]{2.409pt}{0.400pt}}
\put(1429.0,323.0){\rule[-0.200pt]{2.409pt}{0.400pt}}
\put(170.0,323.0){\rule[-0.200pt]{2.409pt}{0.400pt}}
\put(1429.0,323.0){\rule[-0.200pt]{2.409pt}{0.400pt}}
\put(170.0,323.0){\rule[-0.200pt]{2.409pt}{0.400pt}}
\put(1429.0,323.0){\rule[-0.200pt]{2.409pt}{0.400pt}}
\put(170.0,323.0){\rule[-0.200pt]{2.409pt}{0.400pt}}
\put(1429.0,323.0){\rule[-0.200pt]{2.409pt}{0.400pt}}
\put(170.0,323.0){\rule[-0.200pt]{2.409pt}{0.400pt}}
\put(1429.0,323.0){\rule[-0.200pt]{2.409pt}{0.400pt}}
\put(170.0,323.0){\rule[-0.200pt]{2.409pt}{0.400pt}}
\put(1429.0,323.0){\rule[-0.200pt]{2.409pt}{0.400pt}}
\put(170.0,323.0){\rule[-0.200pt]{2.409pt}{0.400pt}}
\put(1429.0,323.0){\rule[-0.200pt]{2.409pt}{0.400pt}}
\put(170.0,323.0){\rule[-0.200pt]{2.409pt}{0.400pt}}
\put(1429.0,323.0){\rule[-0.200pt]{2.409pt}{0.400pt}}
\put(170.0,323.0){\rule[-0.200pt]{2.409pt}{0.400pt}}
\put(1429.0,323.0){\rule[-0.200pt]{2.409pt}{0.400pt}}
\put(170.0,323.0){\rule[-0.200pt]{2.409pt}{0.400pt}}
\put(1429.0,323.0){\rule[-0.200pt]{2.409pt}{0.400pt}}
\put(170.0,323.0){\rule[-0.200pt]{2.409pt}{0.400pt}}
\put(1429.0,323.0){\rule[-0.200pt]{2.409pt}{0.400pt}}
\put(170.0,323.0){\rule[-0.200pt]{2.409pt}{0.400pt}}
\put(1429.0,323.0){\rule[-0.200pt]{2.409pt}{0.400pt}}
\put(170.0,324.0){\rule[-0.200pt]{2.409pt}{0.400pt}}
\put(1429.0,324.0){\rule[-0.200pt]{2.409pt}{0.400pt}}
\put(170.0,324.0){\rule[-0.200pt]{2.409pt}{0.400pt}}
\put(1429.0,324.0){\rule[-0.200pt]{2.409pt}{0.400pt}}
\put(170.0,324.0){\rule[-0.200pt]{2.409pt}{0.400pt}}
\put(1429.0,324.0){\rule[-0.200pt]{2.409pt}{0.400pt}}
\put(170.0,324.0){\rule[-0.200pt]{2.409pt}{0.400pt}}
\put(1429.0,324.0){\rule[-0.200pt]{2.409pt}{0.400pt}}
\put(170.0,324.0){\rule[-0.200pt]{2.409pt}{0.400pt}}
\put(1429.0,324.0){\rule[-0.200pt]{2.409pt}{0.400pt}}
\put(170.0,324.0){\rule[-0.200pt]{2.409pt}{0.400pt}}
\put(1429.0,324.0){\rule[-0.200pt]{2.409pt}{0.400pt}}
\put(170.0,324.0){\rule[-0.200pt]{2.409pt}{0.400pt}}
\put(1429.0,324.0){\rule[-0.200pt]{2.409pt}{0.400pt}}
\put(170.0,324.0){\rule[-0.200pt]{2.409pt}{0.400pt}}
\put(1429.0,324.0){\rule[-0.200pt]{2.409pt}{0.400pt}}
\put(170.0,324.0){\rule[-0.200pt]{2.409pt}{0.400pt}}
\put(1429.0,324.0){\rule[-0.200pt]{2.409pt}{0.400pt}}
\put(170.0,324.0){\rule[-0.200pt]{2.409pt}{0.400pt}}
\put(1429.0,324.0){\rule[-0.200pt]{2.409pt}{0.400pt}}
\put(170.0,324.0){\rule[-0.200pt]{2.409pt}{0.400pt}}
\put(1429.0,324.0){\rule[-0.200pt]{2.409pt}{0.400pt}}
\put(170.0,324.0){\rule[-0.200pt]{2.409pt}{0.400pt}}
\put(1429.0,324.0){\rule[-0.200pt]{2.409pt}{0.400pt}}
\put(170.0,324.0){\rule[-0.200pt]{2.409pt}{0.400pt}}
\put(1429.0,324.0){\rule[-0.200pt]{2.409pt}{0.400pt}}
\put(170.0,324.0){\rule[-0.200pt]{2.409pt}{0.400pt}}
\put(1429.0,324.0){\rule[-0.200pt]{2.409pt}{0.400pt}}
\put(170.0,324.0){\rule[-0.200pt]{2.409pt}{0.400pt}}
\put(1429.0,324.0){\rule[-0.200pt]{2.409pt}{0.400pt}}
\put(170.0,324.0){\rule[-0.200pt]{2.409pt}{0.400pt}}
\put(1429.0,324.0){\rule[-0.200pt]{2.409pt}{0.400pt}}
\put(170.0,325.0){\rule[-0.200pt]{2.409pt}{0.400pt}}
\put(1429.0,325.0){\rule[-0.200pt]{2.409pt}{0.400pt}}
\put(170.0,325.0){\rule[-0.200pt]{2.409pt}{0.400pt}}
\put(1429.0,325.0){\rule[-0.200pt]{2.409pt}{0.400pt}}
\put(170.0,325.0){\rule[-0.200pt]{2.409pt}{0.400pt}}
\put(1429.0,325.0){\rule[-0.200pt]{2.409pt}{0.400pt}}
\put(170.0,325.0){\rule[-0.200pt]{2.409pt}{0.400pt}}
\put(1429.0,325.0){\rule[-0.200pt]{2.409pt}{0.400pt}}
\put(170.0,325.0){\rule[-0.200pt]{2.409pt}{0.400pt}}
\put(1429.0,325.0){\rule[-0.200pt]{2.409pt}{0.400pt}}
\put(170.0,325.0){\rule[-0.200pt]{2.409pt}{0.400pt}}
\put(1429.0,325.0){\rule[-0.200pt]{2.409pt}{0.400pt}}
\put(170.0,325.0){\rule[-0.200pt]{2.409pt}{0.400pt}}
\put(1429.0,325.0){\rule[-0.200pt]{2.409pt}{0.400pt}}
\put(170.0,325.0){\rule[-0.200pt]{2.409pt}{0.400pt}}
\put(1429.0,325.0){\rule[-0.200pt]{2.409pt}{0.400pt}}
\put(170.0,325.0){\rule[-0.200pt]{2.409pt}{0.400pt}}
\put(1429.0,325.0){\rule[-0.200pt]{2.409pt}{0.400pt}}
\put(170.0,325.0){\rule[-0.200pt]{2.409pt}{0.400pt}}
\put(1429.0,325.0){\rule[-0.200pt]{2.409pt}{0.400pt}}
\put(170.0,325.0){\rule[-0.200pt]{2.409pt}{0.400pt}}
\put(1429.0,325.0){\rule[-0.200pt]{2.409pt}{0.400pt}}
\put(170.0,325.0){\rule[-0.200pt]{2.409pt}{0.400pt}}
\put(1429.0,325.0){\rule[-0.200pt]{2.409pt}{0.400pt}}
\put(170.0,325.0){\rule[-0.200pt]{2.409pt}{0.400pt}}
\put(1429.0,325.0){\rule[-0.200pt]{2.409pt}{0.400pt}}
\put(170.0,325.0){\rule[-0.200pt]{2.409pt}{0.400pt}}
\put(1429.0,325.0){\rule[-0.200pt]{2.409pt}{0.400pt}}
\put(170.0,325.0){\rule[-0.200pt]{2.409pt}{0.400pt}}
\put(1429.0,325.0){\rule[-0.200pt]{2.409pt}{0.400pt}}
\put(170.0,325.0){\rule[-0.200pt]{2.409pt}{0.400pt}}
\put(1429.0,325.0){\rule[-0.200pt]{2.409pt}{0.400pt}}
\put(170.0,325.0){\rule[-0.200pt]{2.409pt}{0.400pt}}
\put(1429.0,325.0){\rule[-0.200pt]{2.409pt}{0.400pt}}
\put(170.0,325.0){\rule[-0.200pt]{2.409pt}{0.400pt}}
\put(1429.0,325.0){\rule[-0.200pt]{2.409pt}{0.400pt}}
\put(170.0,326.0){\rule[-0.200pt]{2.409pt}{0.400pt}}
\put(1429.0,326.0){\rule[-0.200pt]{2.409pt}{0.400pt}}
\put(170.0,326.0){\rule[-0.200pt]{2.409pt}{0.400pt}}
\put(1429.0,326.0){\rule[-0.200pt]{2.409pt}{0.400pt}}
\put(170.0,326.0){\rule[-0.200pt]{2.409pt}{0.400pt}}
\put(1429.0,326.0){\rule[-0.200pt]{2.409pt}{0.400pt}}
\put(170.0,326.0){\rule[-0.200pt]{2.409pt}{0.400pt}}
\put(1429.0,326.0){\rule[-0.200pt]{2.409pt}{0.400pt}}
\put(170.0,326.0){\rule[-0.200pt]{2.409pt}{0.400pt}}
\put(1429.0,326.0){\rule[-0.200pt]{2.409pt}{0.400pt}}
\put(170.0,326.0){\rule[-0.200pt]{2.409pt}{0.400pt}}
\put(1429.0,326.0){\rule[-0.200pt]{2.409pt}{0.400pt}}
\put(170.0,326.0){\rule[-0.200pt]{2.409pt}{0.400pt}}
\put(1429.0,326.0){\rule[-0.200pt]{2.409pt}{0.400pt}}
\put(170.0,326.0){\rule[-0.200pt]{2.409pt}{0.400pt}}
\put(1429.0,326.0){\rule[-0.200pt]{2.409pt}{0.400pt}}
\put(170.0,326.0){\rule[-0.200pt]{2.409pt}{0.400pt}}
\put(1429.0,326.0){\rule[-0.200pt]{2.409pt}{0.400pt}}
\put(170.0,326.0){\rule[-0.200pt]{2.409pt}{0.400pt}}
\put(1429.0,326.0){\rule[-0.200pt]{2.409pt}{0.400pt}}
\put(170.0,326.0){\rule[-0.200pt]{2.409pt}{0.400pt}}
\put(1429.0,326.0){\rule[-0.200pt]{2.409pt}{0.400pt}}
\put(170.0,326.0){\rule[-0.200pt]{2.409pt}{0.400pt}}
\put(1429.0,326.0){\rule[-0.200pt]{2.409pt}{0.400pt}}
\put(170.0,326.0){\rule[-0.200pt]{2.409pt}{0.400pt}}
\put(1429.0,326.0){\rule[-0.200pt]{2.409pt}{0.400pt}}
\put(170.0,326.0){\rule[-0.200pt]{2.409pt}{0.400pt}}
\put(1429.0,326.0){\rule[-0.200pt]{2.409pt}{0.400pt}}
\put(170.0,326.0){\rule[-0.200pt]{2.409pt}{0.400pt}}
\put(1429.0,326.0){\rule[-0.200pt]{2.409pt}{0.400pt}}
\put(170.0,326.0){\rule[-0.200pt]{2.409pt}{0.400pt}}
\put(1429.0,326.0){\rule[-0.200pt]{2.409pt}{0.400pt}}
\put(170.0,326.0){\rule[-0.200pt]{2.409pt}{0.400pt}}
\put(1429.0,326.0){\rule[-0.200pt]{2.409pt}{0.400pt}}
\put(170.0,326.0){\rule[-0.200pt]{2.409pt}{0.400pt}}
\put(1429.0,326.0){\rule[-0.200pt]{2.409pt}{0.400pt}}
\put(170.0,327.0){\rule[-0.200pt]{2.409pt}{0.400pt}}
\put(1429.0,327.0){\rule[-0.200pt]{2.409pt}{0.400pt}}
\put(170.0,327.0){\rule[-0.200pt]{2.409pt}{0.400pt}}
\put(1429.0,327.0){\rule[-0.200pt]{2.409pt}{0.400pt}}
\put(170.0,327.0){\rule[-0.200pt]{2.409pt}{0.400pt}}
\put(1429.0,327.0){\rule[-0.200pt]{2.409pt}{0.400pt}}
\put(170.0,327.0){\rule[-0.200pt]{2.409pt}{0.400pt}}
\put(1429.0,327.0){\rule[-0.200pt]{2.409pt}{0.400pt}}
\put(170.0,327.0){\rule[-0.200pt]{2.409pt}{0.400pt}}
\put(1429.0,327.0){\rule[-0.200pt]{2.409pt}{0.400pt}}
\put(170.0,327.0){\rule[-0.200pt]{2.409pt}{0.400pt}}
\put(1429.0,327.0){\rule[-0.200pt]{2.409pt}{0.400pt}}
\put(170.0,327.0){\rule[-0.200pt]{2.409pt}{0.400pt}}
\put(1429.0,327.0){\rule[-0.200pt]{2.409pt}{0.400pt}}
\put(170.0,327.0){\rule[-0.200pt]{2.409pt}{0.400pt}}
\put(1429.0,327.0){\rule[-0.200pt]{2.409pt}{0.400pt}}
\put(170.0,327.0){\rule[-0.200pt]{2.409pt}{0.400pt}}
\put(1429.0,327.0){\rule[-0.200pt]{2.409pt}{0.400pt}}
\put(170.0,327.0){\rule[-0.200pt]{2.409pt}{0.400pt}}
\put(1429.0,327.0){\rule[-0.200pt]{2.409pt}{0.400pt}}
\put(170.0,327.0){\rule[-0.200pt]{2.409pt}{0.400pt}}
\put(1429.0,327.0){\rule[-0.200pt]{2.409pt}{0.400pt}}
\put(170.0,327.0){\rule[-0.200pt]{2.409pt}{0.400pt}}
\put(1429.0,327.0){\rule[-0.200pt]{2.409pt}{0.400pt}}
\put(170.0,327.0){\rule[-0.200pt]{2.409pt}{0.400pt}}
\put(1429.0,327.0){\rule[-0.200pt]{2.409pt}{0.400pt}}
\put(170.0,327.0){\rule[-0.200pt]{2.409pt}{0.400pt}}
\put(1429.0,327.0){\rule[-0.200pt]{2.409pt}{0.400pt}}
\put(170.0,327.0){\rule[-0.200pt]{2.409pt}{0.400pt}}
\put(1429.0,327.0){\rule[-0.200pt]{2.409pt}{0.400pt}}
\put(170.0,327.0){\rule[-0.200pt]{2.409pt}{0.400pt}}
\put(1429.0,327.0){\rule[-0.200pt]{2.409pt}{0.400pt}}
\put(170.0,327.0){\rule[-0.200pt]{2.409pt}{0.400pt}}
\put(1429.0,327.0){\rule[-0.200pt]{2.409pt}{0.400pt}}
\put(170.0,327.0){\rule[-0.200pt]{2.409pt}{0.400pt}}
\put(1429.0,327.0){\rule[-0.200pt]{2.409pt}{0.400pt}}
\put(170.0,328.0){\rule[-0.200pt]{2.409pt}{0.400pt}}
\put(1429.0,328.0){\rule[-0.200pt]{2.409pt}{0.400pt}}
\put(170.0,328.0){\rule[-0.200pt]{2.409pt}{0.400pt}}
\put(1429.0,328.0){\rule[-0.200pt]{2.409pt}{0.400pt}}
\put(170.0,328.0){\rule[-0.200pt]{2.409pt}{0.400pt}}
\put(1429.0,328.0){\rule[-0.200pt]{2.409pt}{0.400pt}}
\put(170.0,328.0){\rule[-0.200pt]{2.409pt}{0.400pt}}
\put(1429.0,328.0){\rule[-0.200pt]{2.409pt}{0.400pt}}
\put(170.0,328.0){\rule[-0.200pt]{2.409pt}{0.400pt}}
\put(1429.0,328.0){\rule[-0.200pt]{2.409pt}{0.400pt}}
\put(170.0,328.0){\rule[-0.200pt]{2.409pt}{0.400pt}}
\put(1429.0,328.0){\rule[-0.200pt]{2.409pt}{0.400pt}}
\put(170.0,328.0){\rule[-0.200pt]{2.409pt}{0.400pt}}
\put(1429.0,328.0){\rule[-0.200pt]{2.409pt}{0.400pt}}
\put(170.0,328.0){\rule[-0.200pt]{2.409pt}{0.400pt}}
\put(1429.0,328.0){\rule[-0.200pt]{2.409pt}{0.400pt}}
\put(170.0,328.0){\rule[-0.200pt]{2.409pt}{0.400pt}}
\put(1429.0,328.0){\rule[-0.200pt]{2.409pt}{0.400pt}}
\put(170.0,328.0){\rule[-0.200pt]{2.409pt}{0.400pt}}
\put(1429.0,328.0){\rule[-0.200pt]{2.409pt}{0.400pt}}
\put(170.0,328.0){\rule[-0.200pt]{2.409pt}{0.400pt}}
\put(1429.0,328.0){\rule[-0.200pt]{2.409pt}{0.400pt}}
\put(170.0,328.0){\rule[-0.200pt]{2.409pt}{0.400pt}}
\put(1429.0,328.0){\rule[-0.200pt]{2.409pt}{0.400pt}}
\put(170.0,328.0){\rule[-0.200pt]{2.409pt}{0.400pt}}
\put(1429.0,328.0){\rule[-0.200pt]{2.409pt}{0.400pt}}
\put(170.0,328.0){\rule[-0.200pt]{2.409pt}{0.400pt}}
\put(1429.0,328.0){\rule[-0.200pt]{2.409pt}{0.400pt}}
\put(170.0,328.0){\rule[-0.200pt]{2.409pt}{0.400pt}}
\put(1429.0,328.0){\rule[-0.200pt]{2.409pt}{0.400pt}}
\put(170.0,328.0){\rule[-0.200pt]{2.409pt}{0.400pt}}
\put(1429.0,328.0){\rule[-0.200pt]{2.409pt}{0.400pt}}
\put(170.0,328.0){\rule[-0.200pt]{2.409pt}{0.400pt}}
\put(1429.0,328.0){\rule[-0.200pt]{2.409pt}{0.400pt}}
\put(170.0,328.0){\rule[-0.200pt]{2.409pt}{0.400pt}}
\put(1429.0,328.0){\rule[-0.200pt]{2.409pt}{0.400pt}}
\put(170.0,328.0){\rule[-0.200pt]{2.409pt}{0.400pt}}
\put(1429.0,328.0){\rule[-0.200pt]{2.409pt}{0.400pt}}
\put(170.0,329.0){\rule[-0.200pt]{2.409pt}{0.400pt}}
\put(1429.0,329.0){\rule[-0.200pt]{2.409pt}{0.400pt}}
\put(170.0,329.0){\rule[-0.200pt]{2.409pt}{0.400pt}}
\put(1429.0,329.0){\rule[-0.200pt]{2.409pt}{0.400pt}}
\put(170.0,329.0){\rule[-0.200pt]{2.409pt}{0.400pt}}
\put(1429.0,329.0){\rule[-0.200pt]{2.409pt}{0.400pt}}
\put(170.0,329.0){\rule[-0.200pt]{2.409pt}{0.400pt}}
\put(1429.0,329.0){\rule[-0.200pt]{2.409pt}{0.400pt}}
\put(170.0,329.0){\rule[-0.200pt]{2.409pt}{0.400pt}}
\put(1429.0,329.0){\rule[-0.200pt]{2.409pt}{0.400pt}}
\put(170.0,329.0){\rule[-0.200pt]{2.409pt}{0.400pt}}
\put(1429.0,329.0){\rule[-0.200pt]{2.409pt}{0.400pt}}
\put(170.0,329.0){\rule[-0.200pt]{2.409pt}{0.400pt}}
\put(1429.0,329.0){\rule[-0.200pt]{2.409pt}{0.400pt}}
\put(170.0,329.0){\rule[-0.200pt]{2.409pt}{0.400pt}}
\put(1429.0,329.0){\rule[-0.200pt]{2.409pt}{0.400pt}}
\put(170.0,329.0){\rule[-0.200pt]{2.409pt}{0.400pt}}
\put(1429.0,329.0){\rule[-0.200pt]{2.409pt}{0.400pt}}
\put(170.0,329.0){\rule[-0.200pt]{2.409pt}{0.400pt}}
\put(1429.0,329.0){\rule[-0.200pt]{2.409pt}{0.400pt}}
\put(170.0,329.0){\rule[-0.200pt]{2.409pt}{0.400pt}}
\put(1429.0,329.0){\rule[-0.200pt]{2.409pt}{0.400pt}}
\put(170.0,329.0){\rule[-0.200pt]{2.409pt}{0.400pt}}
\put(1429.0,329.0){\rule[-0.200pt]{2.409pt}{0.400pt}}
\put(170.0,329.0){\rule[-0.200pt]{2.409pt}{0.400pt}}
\put(1429.0,329.0){\rule[-0.200pt]{2.409pt}{0.400pt}}
\put(170.0,329.0){\rule[-0.200pt]{2.409pt}{0.400pt}}
\put(1429.0,329.0){\rule[-0.200pt]{2.409pt}{0.400pt}}
\put(170.0,329.0){\rule[-0.200pt]{2.409pt}{0.400pt}}
\put(1429.0,329.0){\rule[-0.200pt]{2.409pt}{0.400pt}}
\put(170.0,329.0){\rule[-0.200pt]{2.409pt}{0.400pt}}
\put(1429.0,329.0){\rule[-0.200pt]{2.409pt}{0.400pt}}
\put(170.0,329.0){\rule[-0.200pt]{2.409pt}{0.400pt}}
\put(1429.0,329.0){\rule[-0.200pt]{2.409pt}{0.400pt}}
\put(170.0,329.0){\rule[-0.200pt]{2.409pt}{0.400pt}}
\put(1429.0,329.0){\rule[-0.200pt]{2.409pt}{0.400pt}}
\put(170.0,329.0){\rule[-0.200pt]{2.409pt}{0.400pt}}
\put(1429.0,329.0){\rule[-0.200pt]{2.409pt}{0.400pt}}
\put(170.0,330.0){\rule[-0.200pt]{2.409pt}{0.400pt}}
\put(1429.0,330.0){\rule[-0.200pt]{2.409pt}{0.400pt}}
\put(170.0,330.0){\rule[-0.200pt]{2.409pt}{0.400pt}}
\put(1429.0,330.0){\rule[-0.200pt]{2.409pt}{0.400pt}}
\put(170.0,330.0){\rule[-0.200pt]{2.409pt}{0.400pt}}
\put(1429.0,330.0){\rule[-0.200pt]{2.409pt}{0.400pt}}
\put(170.0,330.0){\rule[-0.200pt]{2.409pt}{0.400pt}}
\put(1429.0,330.0){\rule[-0.200pt]{2.409pt}{0.400pt}}
\put(170.0,330.0){\rule[-0.200pt]{2.409pt}{0.400pt}}
\put(1429.0,330.0){\rule[-0.200pt]{2.409pt}{0.400pt}}
\put(170.0,330.0){\rule[-0.200pt]{2.409pt}{0.400pt}}
\put(1429.0,330.0){\rule[-0.200pt]{2.409pt}{0.400pt}}
\put(170.0,330.0){\rule[-0.200pt]{2.409pt}{0.400pt}}
\put(1429.0,330.0){\rule[-0.200pt]{2.409pt}{0.400pt}}
\put(170.0,330.0){\rule[-0.200pt]{2.409pt}{0.400pt}}
\put(1429.0,330.0){\rule[-0.200pt]{2.409pt}{0.400pt}}
\put(170.0,330.0){\rule[-0.200pt]{2.409pt}{0.400pt}}
\put(1429.0,330.0){\rule[-0.200pt]{2.409pt}{0.400pt}}
\put(170.0,330.0){\rule[-0.200pt]{2.409pt}{0.400pt}}
\put(1429.0,330.0){\rule[-0.200pt]{2.409pt}{0.400pt}}
\put(170.0,330.0){\rule[-0.200pt]{2.409pt}{0.400pt}}
\put(1429.0,330.0){\rule[-0.200pt]{2.409pt}{0.400pt}}
\put(170.0,330.0){\rule[-0.200pt]{2.409pt}{0.400pt}}
\put(1429.0,330.0){\rule[-0.200pt]{2.409pt}{0.400pt}}
\put(170.0,330.0){\rule[-0.200pt]{2.409pt}{0.400pt}}
\put(1429.0,330.0){\rule[-0.200pt]{2.409pt}{0.400pt}}
\put(170.0,330.0){\rule[-0.200pt]{2.409pt}{0.400pt}}
\put(1429.0,330.0){\rule[-0.200pt]{2.409pt}{0.400pt}}
\put(170.0,330.0){\rule[-0.200pt]{2.409pt}{0.400pt}}
\put(1429.0,330.0){\rule[-0.200pt]{2.409pt}{0.400pt}}
\put(170.0,330.0){\rule[-0.200pt]{2.409pt}{0.400pt}}
\put(1429.0,330.0){\rule[-0.200pt]{2.409pt}{0.400pt}}
\put(170.0,330.0){\rule[-0.200pt]{2.409pt}{0.400pt}}
\put(1429.0,330.0){\rule[-0.200pt]{2.409pt}{0.400pt}}
\put(170.0,330.0){\rule[-0.200pt]{2.409pt}{0.400pt}}
\put(1429.0,330.0){\rule[-0.200pt]{2.409pt}{0.400pt}}
\put(170.0,330.0){\rule[-0.200pt]{2.409pt}{0.400pt}}
\put(1429.0,330.0){\rule[-0.200pt]{2.409pt}{0.400pt}}
\put(170.0,330.0){\rule[-0.200pt]{2.409pt}{0.400pt}}
\put(1429.0,330.0){\rule[-0.200pt]{2.409pt}{0.400pt}}
\put(170.0,331.0){\rule[-0.200pt]{2.409pt}{0.400pt}}
\put(1429.0,331.0){\rule[-0.200pt]{2.409pt}{0.400pt}}
\put(170.0,331.0){\rule[-0.200pt]{2.409pt}{0.400pt}}
\put(1429.0,331.0){\rule[-0.200pt]{2.409pt}{0.400pt}}
\put(170.0,331.0){\rule[-0.200pt]{2.409pt}{0.400pt}}
\put(1429.0,331.0){\rule[-0.200pt]{2.409pt}{0.400pt}}
\put(170.0,331.0){\rule[-0.200pt]{2.409pt}{0.400pt}}
\put(1429.0,331.0){\rule[-0.200pt]{2.409pt}{0.400pt}}
\put(170.0,331.0){\rule[-0.200pt]{2.409pt}{0.400pt}}
\put(1429.0,331.0){\rule[-0.200pt]{2.409pt}{0.400pt}}
\put(170.0,331.0){\rule[-0.200pt]{2.409pt}{0.400pt}}
\put(1429.0,331.0){\rule[-0.200pt]{2.409pt}{0.400pt}}
\put(170.0,331.0){\rule[-0.200pt]{2.409pt}{0.400pt}}
\put(1429.0,331.0){\rule[-0.200pt]{2.409pt}{0.400pt}}
\put(170.0,331.0){\rule[-0.200pt]{2.409pt}{0.400pt}}
\put(1429.0,331.0){\rule[-0.200pt]{2.409pt}{0.400pt}}
\put(170.0,331.0){\rule[-0.200pt]{2.409pt}{0.400pt}}
\put(1429.0,331.0){\rule[-0.200pt]{2.409pt}{0.400pt}}
\put(170.0,331.0){\rule[-0.200pt]{2.409pt}{0.400pt}}
\put(1429.0,331.0){\rule[-0.200pt]{2.409pt}{0.400pt}}
\put(170.0,331.0){\rule[-0.200pt]{2.409pt}{0.400pt}}
\put(1429.0,331.0){\rule[-0.200pt]{2.409pt}{0.400pt}}
\put(170.0,331.0){\rule[-0.200pt]{2.409pt}{0.400pt}}
\put(1429.0,331.0){\rule[-0.200pt]{2.409pt}{0.400pt}}
\put(170.0,331.0){\rule[-0.200pt]{2.409pt}{0.400pt}}
\put(1429.0,331.0){\rule[-0.200pt]{2.409pt}{0.400pt}}
\put(170.0,331.0){\rule[-0.200pt]{2.409pt}{0.400pt}}
\put(1429.0,331.0){\rule[-0.200pt]{2.409pt}{0.400pt}}
\put(170.0,331.0){\rule[-0.200pt]{2.409pt}{0.400pt}}
\put(1429.0,331.0){\rule[-0.200pt]{2.409pt}{0.400pt}}
\put(170.0,331.0){\rule[-0.200pt]{2.409pt}{0.400pt}}
\put(1429.0,331.0){\rule[-0.200pt]{2.409pt}{0.400pt}}
\put(170.0,331.0){\rule[-0.200pt]{2.409pt}{0.400pt}}
\put(1429.0,331.0){\rule[-0.200pt]{2.409pt}{0.400pt}}
\put(170.0,331.0){\rule[-0.200pt]{2.409pt}{0.400pt}}
\put(1429.0,331.0){\rule[-0.200pt]{2.409pt}{0.400pt}}
\put(170.0,331.0){\rule[-0.200pt]{2.409pt}{0.400pt}}
\put(1429.0,331.0){\rule[-0.200pt]{2.409pt}{0.400pt}}
\put(170.0,331.0){\rule[-0.200pt]{2.409pt}{0.400pt}}
\put(1429.0,331.0){\rule[-0.200pt]{2.409pt}{0.400pt}}
\put(170.0,331.0){\rule[-0.200pt]{2.409pt}{0.400pt}}
\put(1429.0,331.0){\rule[-0.200pt]{2.409pt}{0.400pt}}
\put(170.0,332.0){\rule[-0.200pt]{2.409pt}{0.400pt}}
\put(1429.0,332.0){\rule[-0.200pt]{2.409pt}{0.400pt}}
\put(170.0,332.0){\rule[-0.200pt]{2.409pt}{0.400pt}}
\put(1429.0,332.0){\rule[-0.200pt]{2.409pt}{0.400pt}}
\put(170.0,332.0){\rule[-0.200pt]{2.409pt}{0.400pt}}
\put(1429.0,332.0){\rule[-0.200pt]{2.409pt}{0.400pt}}
\put(170.0,332.0){\rule[-0.200pt]{2.409pt}{0.400pt}}
\put(1429.0,332.0){\rule[-0.200pt]{2.409pt}{0.400pt}}
\put(170.0,332.0){\rule[-0.200pt]{2.409pt}{0.400pt}}
\put(1429.0,332.0){\rule[-0.200pt]{2.409pt}{0.400pt}}
\put(170.0,332.0){\rule[-0.200pt]{2.409pt}{0.400pt}}
\put(1429.0,332.0){\rule[-0.200pt]{2.409pt}{0.400pt}}
\put(170.0,332.0){\rule[-0.200pt]{2.409pt}{0.400pt}}
\put(1429.0,332.0){\rule[-0.200pt]{2.409pt}{0.400pt}}
\put(170.0,332.0){\rule[-0.200pt]{2.409pt}{0.400pt}}
\put(1429.0,332.0){\rule[-0.200pt]{2.409pt}{0.400pt}}
\put(170.0,332.0){\rule[-0.200pt]{2.409pt}{0.400pt}}
\put(1429.0,332.0){\rule[-0.200pt]{2.409pt}{0.400pt}}
\put(170.0,332.0){\rule[-0.200pt]{2.409pt}{0.400pt}}
\put(1429.0,332.0){\rule[-0.200pt]{2.409pt}{0.400pt}}
\put(170.0,332.0){\rule[-0.200pt]{2.409pt}{0.400pt}}
\put(1429.0,332.0){\rule[-0.200pt]{2.409pt}{0.400pt}}
\put(170.0,332.0){\rule[-0.200pt]{2.409pt}{0.400pt}}
\put(1429.0,332.0){\rule[-0.200pt]{2.409pt}{0.400pt}}
\put(170.0,332.0){\rule[-0.200pt]{2.409pt}{0.400pt}}
\put(1429.0,332.0){\rule[-0.200pt]{2.409pt}{0.400pt}}
\put(170.0,332.0){\rule[-0.200pt]{2.409pt}{0.400pt}}
\put(1429.0,332.0){\rule[-0.200pt]{2.409pt}{0.400pt}}
\put(170.0,332.0){\rule[-0.200pt]{2.409pt}{0.400pt}}
\put(1429.0,332.0){\rule[-0.200pt]{2.409pt}{0.400pt}}
\put(170.0,332.0){\rule[-0.200pt]{2.409pt}{0.400pt}}
\put(1429.0,332.0){\rule[-0.200pt]{2.409pt}{0.400pt}}
\put(170.0,332.0){\rule[-0.200pt]{2.409pt}{0.400pt}}
\put(1429.0,332.0){\rule[-0.200pt]{2.409pt}{0.400pt}}
\put(170.0,332.0){\rule[-0.200pt]{2.409pt}{0.400pt}}
\put(1429.0,332.0){\rule[-0.200pt]{2.409pt}{0.400pt}}
\put(170.0,332.0){\rule[-0.200pt]{2.409pt}{0.400pt}}
\put(1429.0,332.0){\rule[-0.200pt]{2.409pt}{0.400pt}}
\put(170.0,332.0){\rule[-0.200pt]{2.409pt}{0.400pt}}
\put(1429.0,332.0){\rule[-0.200pt]{2.409pt}{0.400pt}}
\put(170.0,332.0){\rule[-0.200pt]{2.409pt}{0.400pt}}
\put(1429.0,332.0){\rule[-0.200pt]{2.409pt}{0.400pt}}
\put(170.0,333.0){\rule[-0.200pt]{2.409pt}{0.400pt}}
\put(1429.0,333.0){\rule[-0.200pt]{2.409pt}{0.400pt}}
\put(170.0,333.0){\rule[-0.200pt]{2.409pt}{0.400pt}}
\put(1429.0,333.0){\rule[-0.200pt]{2.409pt}{0.400pt}}
\put(170.0,333.0){\rule[-0.200pt]{2.409pt}{0.400pt}}
\put(1429.0,333.0){\rule[-0.200pt]{2.409pt}{0.400pt}}
\put(170.0,333.0){\rule[-0.200pt]{2.409pt}{0.400pt}}
\put(1429.0,333.0){\rule[-0.200pt]{2.409pt}{0.400pt}}
\put(170.0,333.0){\rule[-0.200pt]{2.409pt}{0.400pt}}
\put(1429.0,333.0){\rule[-0.200pt]{2.409pt}{0.400pt}}
\put(170.0,333.0){\rule[-0.200pt]{2.409pt}{0.400pt}}
\put(1429.0,333.0){\rule[-0.200pt]{2.409pt}{0.400pt}}
\put(170.0,333.0){\rule[-0.200pt]{2.409pt}{0.400pt}}
\put(1429.0,333.0){\rule[-0.200pt]{2.409pt}{0.400pt}}
\put(170.0,333.0){\rule[-0.200pt]{2.409pt}{0.400pt}}
\put(1429.0,333.0){\rule[-0.200pt]{2.409pt}{0.400pt}}
\put(170.0,333.0){\rule[-0.200pt]{2.409pt}{0.400pt}}
\put(1429.0,333.0){\rule[-0.200pt]{2.409pt}{0.400pt}}
\put(170.0,333.0){\rule[-0.200pt]{2.409pt}{0.400pt}}
\put(1429.0,333.0){\rule[-0.200pt]{2.409pt}{0.400pt}}
\put(170.0,333.0){\rule[-0.200pt]{2.409pt}{0.400pt}}
\put(1429.0,333.0){\rule[-0.200pt]{2.409pt}{0.400pt}}
\put(170.0,333.0){\rule[-0.200pt]{2.409pt}{0.400pt}}
\put(1429.0,333.0){\rule[-0.200pt]{2.409pt}{0.400pt}}
\put(170.0,333.0){\rule[-0.200pt]{2.409pt}{0.400pt}}
\put(1429.0,333.0){\rule[-0.200pt]{2.409pt}{0.400pt}}
\put(170.0,333.0){\rule[-0.200pt]{2.409pt}{0.400pt}}
\put(1429.0,333.0){\rule[-0.200pt]{2.409pt}{0.400pt}}
\put(170.0,333.0){\rule[-0.200pt]{2.409pt}{0.400pt}}
\put(1429.0,333.0){\rule[-0.200pt]{2.409pt}{0.400pt}}
\put(170.0,333.0){\rule[-0.200pt]{2.409pt}{0.400pt}}
\put(1429.0,333.0){\rule[-0.200pt]{2.409pt}{0.400pt}}
\put(170.0,333.0){\rule[-0.200pt]{2.409pt}{0.400pt}}
\put(1429.0,333.0){\rule[-0.200pt]{2.409pt}{0.400pt}}
\put(170.0,333.0){\rule[-0.200pt]{2.409pt}{0.400pt}}
\put(1429.0,333.0){\rule[-0.200pt]{2.409pt}{0.400pt}}
\put(170.0,333.0){\rule[-0.200pt]{2.409pt}{0.400pt}}
\put(1429.0,333.0){\rule[-0.200pt]{2.409pt}{0.400pt}}
\put(170.0,333.0){\rule[-0.200pt]{2.409pt}{0.400pt}}
\put(1429.0,333.0){\rule[-0.200pt]{2.409pt}{0.400pt}}
\put(170.0,333.0){\rule[-0.200pt]{2.409pt}{0.400pt}}
\put(1429.0,333.0){\rule[-0.200pt]{2.409pt}{0.400pt}}
\put(170.0,334.0){\rule[-0.200pt]{2.409pt}{0.400pt}}
\put(1429.0,334.0){\rule[-0.200pt]{2.409pt}{0.400pt}}
\put(170.0,334.0){\rule[-0.200pt]{2.409pt}{0.400pt}}
\put(1429.0,334.0){\rule[-0.200pt]{2.409pt}{0.400pt}}
\put(170.0,334.0){\rule[-0.200pt]{2.409pt}{0.400pt}}
\put(1429.0,334.0){\rule[-0.200pt]{2.409pt}{0.400pt}}
\put(170.0,334.0){\rule[-0.200pt]{2.409pt}{0.400pt}}
\put(1429.0,334.0){\rule[-0.200pt]{2.409pt}{0.400pt}}
\put(170.0,334.0){\rule[-0.200pt]{2.409pt}{0.400pt}}
\put(1429.0,334.0){\rule[-0.200pt]{2.409pt}{0.400pt}}
\put(170.0,334.0){\rule[-0.200pt]{2.409pt}{0.400pt}}
\put(1429.0,334.0){\rule[-0.200pt]{2.409pt}{0.400pt}}
\put(170.0,334.0){\rule[-0.200pt]{2.409pt}{0.400pt}}
\put(1429.0,334.0){\rule[-0.200pt]{2.409pt}{0.400pt}}
\put(170.0,334.0){\rule[-0.200pt]{2.409pt}{0.400pt}}
\put(1429.0,334.0){\rule[-0.200pt]{2.409pt}{0.400pt}}
\put(170.0,334.0){\rule[-0.200pt]{2.409pt}{0.400pt}}
\put(1429.0,334.0){\rule[-0.200pt]{2.409pt}{0.400pt}}
\put(170.0,334.0){\rule[-0.200pt]{2.409pt}{0.400pt}}
\put(1429.0,334.0){\rule[-0.200pt]{2.409pt}{0.400pt}}
\put(170.0,334.0){\rule[-0.200pt]{2.409pt}{0.400pt}}
\put(1429.0,334.0){\rule[-0.200pt]{2.409pt}{0.400pt}}
\put(170.0,334.0){\rule[-0.200pt]{2.409pt}{0.400pt}}
\put(1429.0,334.0){\rule[-0.200pt]{2.409pt}{0.400pt}}
\put(170.0,334.0){\rule[-0.200pt]{2.409pt}{0.400pt}}
\put(1429.0,334.0){\rule[-0.200pt]{2.409pt}{0.400pt}}
\put(170.0,334.0){\rule[-0.200pt]{2.409pt}{0.400pt}}
\put(1429.0,334.0){\rule[-0.200pt]{2.409pt}{0.400pt}}
\put(170.0,334.0){\rule[-0.200pt]{2.409pt}{0.400pt}}
\put(1429.0,334.0){\rule[-0.200pt]{2.409pt}{0.400pt}}
\put(170.0,334.0){\rule[-0.200pt]{2.409pt}{0.400pt}}
\put(1429.0,334.0){\rule[-0.200pt]{2.409pt}{0.400pt}}
\put(170.0,334.0){\rule[-0.200pt]{2.409pt}{0.400pt}}
\put(1429.0,334.0){\rule[-0.200pt]{2.409pt}{0.400pt}}
\put(170.0,334.0){\rule[-0.200pt]{2.409pt}{0.400pt}}
\put(1429.0,334.0){\rule[-0.200pt]{2.409pt}{0.400pt}}
\put(170.0,334.0){\rule[-0.200pt]{2.409pt}{0.400pt}}
\put(1429.0,334.0){\rule[-0.200pt]{2.409pt}{0.400pt}}
\put(170.0,334.0){\rule[-0.200pt]{2.409pt}{0.400pt}}
\put(1429.0,334.0){\rule[-0.200pt]{2.409pt}{0.400pt}}
\put(170.0,334.0){\rule[-0.200pt]{2.409pt}{0.400pt}}
\put(1429.0,334.0){\rule[-0.200pt]{2.409pt}{0.400pt}}
\put(170.0,334.0){\rule[-0.200pt]{2.409pt}{0.400pt}}
\put(1429.0,334.0){\rule[-0.200pt]{2.409pt}{0.400pt}}
\put(170.0,335.0){\rule[-0.200pt]{2.409pt}{0.400pt}}
\put(1429.0,335.0){\rule[-0.200pt]{2.409pt}{0.400pt}}
\put(170.0,335.0){\rule[-0.200pt]{2.409pt}{0.400pt}}
\put(1429.0,335.0){\rule[-0.200pt]{2.409pt}{0.400pt}}
\put(170.0,335.0){\rule[-0.200pt]{2.409pt}{0.400pt}}
\put(1429.0,335.0){\rule[-0.200pt]{2.409pt}{0.400pt}}
\put(170.0,335.0){\rule[-0.200pt]{2.409pt}{0.400pt}}
\put(1429.0,335.0){\rule[-0.200pt]{2.409pt}{0.400pt}}
\put(170.0,335.0){\rule[-0.200pt]{2.409pt}{0.400pt}}
\put(1429.0,335.0){\rule[-0.200pt]{2.409pt}{0.400pt}}
\put(170.0,335.0){\rule[-0.200pt]{2.409pt}{0.400pt}}
\put(1429.0,335.0){\rule[-0.200pt]{2.409pt}{0.400pt}}
\put(170.0,335.0){\rule[-0.200pt]{2.409pt}{0.400pt}}
\put(1429.0,335.0){\rule[-0.200pt]{2.409pt}{0.400pt}}
\put(170.0,335.0){\rule[-0.200pt]{2.409pt}{0.400pt}}
\put(1429.0,335.0){\rule[-0.200pt]{2.409pt}{0.400pt}}
\put(170.0,335.0){\rule[-0.200pt]{2.409pt}{0.400pt}}
\put(1429.0,335.0){\rule[-0.200pt]{2.409pt}{0.400pt}}
\put(170.0,335.0){\rule[-0.200pt]{2.409pt}{0.400pt}}
\put(1429.0,335.0){\rule[-0.200pt]{2.409pt}{0.400pt}}
\put(170.0,335.0){\rule[-0.200pt]{2.409pt}{0.400pt}}
\put(1429.0,335.0){\rule[-0.200pt]{2.409pt}{0.400pt}}
\put(170.0,335.0){\rule[-0.200pt]{2.409pt}{0.400pt}}
\put(1429.0,335.0){\rule[-0.200pt]{2.409pt}{0.400pt}}
\put(170.0,335.0){\rule[-0.200pt]{2.409pt}{0.400pt}}
\put(1429.0,335.0){\rule[-0.200pt]{2.409pt}{0.400pt}}
\put(170.0,335.0){\rule[-0.200pt]{2.409pt}{0.400pt}}
\put(1429.0,335.0){\rule[-0.200pt]{2.409pt}{0.400pt}}
\put(170.0,335.0){\rule[-0.200pt]{2.409pt}{0.400pt}}
\put(1429.0,335.0){\rule[-0.200pt]{2.409pt}{0.400pt}}
\put(170.0,335.0){\rule[-0.200pt]{2.409pt}{0.400pt}}
\put(1429.0,335.0){\rule[-0.200pt]{2.409pt}{0.400pt}}
\put(170.0,335.0){\rule[-0.200pt]{2.409pt}{0.400pt}}
\put(1429.0,335.0){\rule[-0.200pt]{2.409pt}{0.400pt}}
\put(170.0,335.0){\rule[-0.200pt]{2.409pt}{0.400pt}}
\put(1429.0,335.0){\rule[-0.200pt]{2.409pt}{0.400pt}}
\put(170.0,335.0){\rule[-0.200pt]{2.409pt}{0.400pt}}
\put(1429.0,335.0){\rule[-0.200pt]{2.409pt}{0.400pt}}
\put(170.0,335.0){\rule[-0.200pt]{2.409pt}{0.400pt}}
\put(1429.0,335.0){\rule[-0.200pt]{2.409pt}{0.400pt}}
\put(170.0,335.0){\rule[-0.200pt]{2.409pt}{0.400pt}}
\put(1429.0,335.0){\rule[-0.200pt]{2.409pt}{0.400pt}}
\put(170.0,335.0){\rule[-0.200pt]{2.409pt}{0.400pt}}
\put(1429.0,335.0){\rule[-0.200pt]{2.409pt}{0.400pt}}
\put(170.0,335.0){\rule[-0.200pt]{2.409pt}{0.400pt}}
\put(1429.0,335.0){\rule[-0.200pt]{2.409pt}{0.400pt}}
\put(170.0,336.0){\rule[-0.200pt]{2.409pt}{0.400pt}}
\put(1429.0,336.0){\rule[-0.200pt]{2.409pt}{0.400pt}}
\put(170.0,336.0){\rule[-0.200pt]{2.409pt}{0.400pt}}
\put(1429.0,336.0){\rule[-0.200pt]{2.409pt}{0.400pt}}
\put(170.0,336.0){\rule[-0.200pt]{2.409pt}{0.400pt}}
\put(1429.0,336.0){\rule[-0.200pt]{2.409pt}{0.400pt}}
\put(170.0,336.0){\rule[-0.200pt]{2.409pt}{0.400pt}}
\put(1429.0,336.0){\rule[-0.200pt]{2.409pt}{0.400pt}}
\put(170.0,336.0){\rule[-0.200pt]{2.409pt}{0.400pt}}
\put(1429.0,336.0){\rule[-0.200pt]{2.409pt}{0.400pt}}
\put(170.0,336.0){\rule[-0.200pt]{2.409pt}{0.400pt}}
\put(1429.0,336.0){\rule[-0.200pt]{2.409pt}{0.400pt}}
\put(170.0,336.0){\rule[-0.200pt]{2.409pt}{0.400pt}}
\put(1429.0,336.0){\rule[-0.200pt]{2.409pt}{0.400pt}}
\put(170.0,336.0){\rule[-0.200pt]{2.409pt}{0.400pt}}
\put(1429.0,336.0){\rule[-0.200pt]{2.409pt}{0.400pt}}
\put(170.0,336.0){\rule[-0.200pt]{2.409pt}{0.400pt}}
\put(1429.0,336.0){\rule[-0.200pt]{2.409pt}{0.400pt}}
\put(170.0,336.0){\rule[-0.200pt]{2.409pt}{0.400pt}}
\put(1429.0,336.0){\rule[-0.200pt]{2.409pt}{0.400pt}}
\put(170.0,336.0){\rule[-0.200pt]{2.409pt}{0.400pt}}
\put(1429.0,336.0){\rule[-0.200pt]{2.409pt}{0.400pt}}
\put(170.0,336.0){\rule[-0.200pt]{2.409pt}{0.400pt}}
\put(1429.0,336.0){\rule[-0.200pt]{2.409pt}{0.400pt}}
\put(170.0,336.0){\rule[-0.200pt]{2.409pt}{0.400pt}}
\put(1429.0,336.0){\rule[-0.200pt]{2.409pt}{0.400pt}}
\put(170.0,336.0){\rule[-0.200pt]{2.409pt}{0.400pt}}
\put(1429.0,336.0){\rule[-0.200pt]{2.409pt}{0.400pt}}
\put(170.0,336.0){\rule[-0.200pt]{2.409pt}{0.400pt}}
\put(1429.0,336.0){\rule[-0.200pt]{2.409pt}{0.400pt}}
\put(170.0,336.0){\rule[-0.200pt]{2.409pt}{0.400pt}}
\put(1429.0,336.0){\rule[-0.200pt]{2.409pt}{0.400pt}}
\put(170.0,336.0){\rule[-0.200pt]{2.409pt}{0.400pt}}
\put(1429.0,336.0){\rule[-0.200pt]{2.409pt}{0.400pt}}
\put(170.0,336.0){\rule[-0.200pt]{2.409pt}{0.400pt}}
\put(1429.0,336.0){\rule[-0.200pt]{2.409pt}{0.400pt}}
\put(170.0,336.0){\rule[-0.200pt]{2.409pt}{0.400pt}}
\put(1429.0,336.0){\rule[-0.200pt]{2.409pt}{0.400pt}}
\put(170.0,336.0){\rule[-0.200pt]{2.409pt}{0.400pt}}
\put(1429.0,336.0){\rule[-0.200pt]{2.409pt}{0.400pt}}
\put(170.0,336.0){\rule[-0.200pt]{2.409pt}{0.400pt}}
\put(1429.0,336.0){\rule[-0.200pt]{2.409pt}{0.400pt}}
\put(170.0,336.0){\rule[-0.200pt]{2.409pt}{0.400pt}}
\put(1429.0,336.0){\rule[-0.200pt]{2.409pt}{0.400pt}}
\put(170.0,336.0){\rule[-0.200pt]{2.409pt}{0.400pt}}
\put(1429.0,336.0){\rule[-0.200pt]{2.409pt}{0.400pt}}
\put(170.0,337.0){\rule[-0.200pt]{2.409pt}{0.400pt}}
\put(1429.0,337.0){\rule[-0.200pt]{2.409pt}{0.400pt}}
\put(170.0,337.0){\rule[-0.200pt]{2.409pt}{0.400pt}}
\put(1429.0,337.0){\rule[-0.200pt]{2.409pt}{0.400pt}}
\put(170.0,337.0){\rule[-0.200pt]{2.409pt}{0.400pt}}
\put(1429.0,337.0){\rule[-0.200pt]{2.409pt}{0.400pt}}
\put(170.0,337.0){\rule[-0.200pt]{2.409pt}{0.400pt}}
\put(1429.0,337.0){\rule[-0.200pt]{2.409pt}{0.400pt}}
\put(170.0,337.0){\rule[-0.200pt]{2.409pt}{0.400pt}}
\put(1429.0,337.0){\rule[-0.200pt]{2.409pt}{0.400pt}}
\put(170.0,337.0){\rule[-0.200pt]{2.409pt}{0.400pt}}
\put(1429.0,337.0){\rule[-0.200pt]{2.409pt}{0.400pt}}
\put(170.0,337.0){\rule[-0.200pt]{2.409pt}{0.400pt}}
\put(1429.0,337.0){\rule[-0.200pt]{2.409pt}{0.400pt}}
\put(170.0,337.0){\rule[-0.200pt]{2.409pt}{0.400pt}}
\put(1429.0,337.0){\rule[-0.200pt]{2.409pt}{0.400pt}}
\put(170.0,337.0){\rule[-0.200pt]{2.409pt}{0.400pt}}
\put(1429.0,337.0){\rule[-0.200pt]{2.409pt}{0.400pt}}
\put(170.0,337.0){\rule[-0.200pt]{2.409pt}{0.400pt}}
\put(1429.0,337.0){\rule[-0.200pt]{2.409pt}{0.400pt}}
\put(170.0,337.0){\rule[-0.200pt]{2.409pt}{0.400pt}}
\put(1429.0,337.0){\rule[-0.200pt]{2.409pt}{0.400pt}}
\put(170.0,337.0){\rule[-0.200pt]{2.409pt}{0.400pt}}
\put(1429.0,337.0){\rule[-0.200pt]{2.409pt}{0.400pt}}
\put(170.0,337.0){\rule[-0.200pt]{2.409pt}{0.400pt}}
\put(1429.0,337.0){\rule[-0.200pt]{2.409pt}{0.400pt}}
\put(170.0,337.0){\rule[-0.200pt]{2.409pt}{0.400pt}}
\put(1429.0,337.0){\rule[-0.200pt]{2.409pt}{0.400pt}}
\put(170.0,337.0){\rule[-0.200pt]{2.409pt}{0.400pt}}
\put(1429.0,337.0){\rule[-0.200pt]{2.409pt}{0.400pt}}
\put(170.0,337.0){\rule[-0.200pt]{2.409pt}{0.400pt}}
\put(1429.0,337.0){\rule[-0.200pt]{2.409pt}{0.400pt}}
\put(170.0,337.0){\rule[-0.200pt]{2.409pt}{0.400pt}}
\put(1429.0,337.0){\rule[-0.200pt]{2.409pt}{0.400pt}}
\put(170.0,337.0){\rule[-0.200pt]{2.409pt}{0.400pt}}
\put(1429.0,337.0){\rule[-0.200pt]{2.409pt}{0.400pt}}
\put(170.0,337.0){\rule[-0.200pt]{2.409pt}{0.400pt}}
\put(1429.0,337.0){\rule[-0.200pt]{2.409pt}{0.400pt}}
\put(170.0,337.0){\rule[-0.200pt]{2.409pt}{0.400pt}}
\put(1429.0,337.0){\rule[-0.200pt]{2.409pt}{0.400pt}}
\put(170.0,337.0){\rule[-0.200pt]{2.409pt}{0.400pt}}
\put(1429.0,337.0){\rule[-0.200pt]{2.409pt}{0.400pt}}
\put(170.0,337.0){\rule[-0.200pt]{2.409pt}{0.400pt}}
\put(1429.0,337.0){\rule[-0.200pt]{2.409pt}{0.400pt}}
\put(170.0,337.0){\rule[-0.200pt]{2.409pt}{0.400pt}}
\put(1429.0,337.0){\rule[-0.200pt]{2.409pt}{0.400pt}}
\put(170.0,337.0){\rule[-0.200pt]{2.409pt}{0.400pt}}
\put(1429.0,337.0){\rule[-0.200pt]{2.409pt}{0.400pt}}
\put(170.0,338.0){\rule[-0.200pt]{2.409pt}{0.400pt}}
\put(1429.0,338.0){\rule[-0.200pt]{2.409pt}{0.400pt}}
\put(170.0,338.0){\rule[-0.200pt]{2.409pt}{0.400pt}}
\put(1429.0,338.0){\rule[-0.200pt]{2.409pt}{0.400pt}}
\put(170.0,338.0){\rule[-0.200pt]{2.409pt}{0.400pt}}
\put(1429.0,338.0){\rule[-0.200pt]{2.409pt}{0.400pt}}
\put(170.0,338.0){\rule[-0.200pt]{2.409pt}{0.400pt}}
\put(1429.0,338.0){\rule[-0.200pt]{2.409pt}{0.400pt}}
\put(170.0,338.0){\rule[-0.200pt]{2.409pt}{0.400pt}}
\put(1429.0,338.0){\rule[-0.200pt]{2.409pt}{0.400pt}}
\put(170.0,338.0){\rule[-0.200pt]{2.409pt}{0.400pt}}
\put(1429.0,338.0){\rule[-0.200pt]{2.409pt}{0.400pt}}
\put(170.0,338.0){\rule[-0.200pt]{2.409pt}{0.400pt}}
\put(1429.0,338.0){\rule[-0.200pt]{2.409pt}{0.400pt}}
\put(170.0,338.0){\rule[-0.200pt]{2.409pt}{0.400pt}}
\put(1429.0,338.0){\rule[-0.200pt]{2.409pt}{0.400pt}}
\put(170.0,338.0){\rule[-0.200pt]{2.409pt}{0.400pt}}
\put(1429.0,338.0){\rule[-0.200pt]{2.409pt}{0.400pt}}
\put(170.0,338.0){\rule[-0.200pt]{2.409pt}{0.400pt}}
\put(1429.0,338.0){\rule[-0.200pt]{2.409pt}{0.400pt}}
\put(170.0,338.0){\rule[-0.200pt]{2.409pt}{0.400pt}}
\put(1429.0,338.0){\rule[-0.200pt]{2.409pt}{0.400pt}}
\put(170.0,338.0){\rule[-0.200pt]{2.409pt}{0.400pt}}
\put(1429.0,338.0){\rule[-0.200pt]{2.409pt}{0.400pt}}
\put(170.0,338.0){\rule[-0.200pt]{2.409pt}{0.400pt}}
\put(1429.0,338.0){\rule[-0.200pt]{2.409pt}{0.400pt}}
\put(170.0,338.0){\rule[-0.200pt]{2.409pt}{0.400pt}}
\put(1429.0,338.0){\rule[-0.200pt]{2.409pt}{0.400pt}}
\put(170.0,338.0){\rule[-0.200pt]{2.409pt}{0.400pt}}
\put(1429.0,338.0){\rule[-0.200pt]{2.409pt}{0.400pt}}
\put(170.0,338.0){\rule[-0.200pt]{2.409pt}{0.400pt}}
\put(1429.0,338.0){\rule[-0.200pt]{2.409pt}{0.400pt}}
\put(170.0,338.0){\rule[-0.200pt]{2.409pt}{0.400pt}}
\put(1429.0,338.0){\rule[-0.200pt]{2.409pt}{0.400pt}}
\put(170.0,338.0){\rule[-0.200pt]{2.409pt}{0.400pt}}
\put(1429.0,338.0){\rule[-0.200pt]{2.409pt}{0.400pt}}
\put(170.0,338.0){\rule[-0.200pt]{2.409pt}{0.400pt}}
\put(1429.0,338.0){\rule[-0.200pt]{2.409pt}{0.400pt}}
\put(170.0,338.0){\rule[-0.200pt]{2.409pt}{0.400pt}}
\put(1429.0,338.0){\rule[-0.200pt]{2.409pt}{0.400pt}}
\put(170.0,338.0){\rule[-0.200pt]{2.409pt}{0.400pt}}
\put(1429.0,338.0){\rule[-0.200pt]{2.409pt}{0.400pt}}
\put(170.0,338.0){\rule[-0.200pt]{2.409pt}{0.400pt}}
\put(1429.0,338.0){\rule[-0.200pt]{2.409pt}{0.400pt}}
\put(170.0,338.0){\rule[-0.200pt]{2.409pt}{0.400pt}}
\put(1429.0,338.0){\rule[-0.200pt]{2.409pt}{0.400pt}}
\put(170.0,338.0){\rule[-0.200pt]{2.409pt}{0.400pt}}
\put(1429.0,338.0){\rule[-0.200pt]{2.409pt}{0.400pt}}
\put(170.0,338.0){\rule[-0.200pt]{2.409pt}{0.400pt}}
\put(1429.0,338.0){\rule[-0.200pt]{2.409pt}{0.400pt}}
\put(170.0,339.0){\rule[-0.200pt]{2.409pt}{0.400pt}}
\put(1429.0,339.0){\rule[-0.200pt]{2.409pt}{0.400pt}}
\put(170.0,339.0){\rule[-0.200pt]{2.409pt}{0.400pt}}
\put(1429.0,339.0){\rule[-0.200pt]{2.409pt}{0.400pt}}
\put(170.0,339.0){\rule[-0.200pt]{2.409pt}{0.400pt}}
\put(1429.0,339.0){\rule[-0.200pt]{2.409pt}{0.400pt}}
\put(170.0,339.0){\rule[-0.200pt]{2.409pt}{0.400pt}}
\put(1429.0,339.0){\rule[-0.200pt]{2.409pt}{0.400pt}}
\put(170.0,339.0){\rule[-0.200pt]{2.409pt}{0.400pt}}
\put(1429.0,339.0){\rule[-0.200pt]{2.409pt}{0.400pt}}
\put(170.0,339.0){\rule[-0.200pt]{2.409pt}{0.400pt}}
\put(1429.0,339.0){\rule[-0.200pt]{2.409pt}{0.400pt}}
\put(170.0,339.0){\rule[-0.200pt]{2.409pt}{0.400pt}}
\put(1429.0,339.0){\rule[-0.200pt]{2.409pt}{0.400pt}}
\put(170.0,339.0){\rule[-0.200pt]{2.409pt}{0.400pt}}
\put(1429.0,339.0){\rule[-0.200pt]{2.409pt}{0.400pt}}
\put(170.0,339.0){\rule[-0.200pt]{2.409pt}{0.400pt}}
\put(1429.0,339.0){\rule[-0.200pt]{2.409pt}{0.400pt}}
\put(170.0,339.0){\rule[-0.200pt]{2.409pt}{0.400pt}}
\put(1429.0,339.0){\rule[-0.200pt]{2.409pt}{0.400pt}}
\put(170.0,339.0){\rule[-0.200pt]{2.409pt}{0.400pt}}
\put(1429.0,339.0){\rule[-0.200pt]{2.409pt}{0.400pt}}
\put(170.0,339.0){\rule[-0.200pt]{2.409pt}{0.400pt}}
\put(1429.0,339.0){\rule[-0.200pt]{2.409pt}{0.400pt}}
\put(170.0,339.0){\rule[-0.200pt]{2.409pt}{0.400pt}}
\put(1429.0,339.0){\rule[-0.200pt]{2.409pt}{0.400pt}}
\put(170.0,339.0){\rule[-0.200pt]{2.409pt}{0.400pt}}
\put(1429.0,339.0){\rule[-0.200pt]{2.409pt}{0.400pt}}
\put(170.0,339.0){\rule[-0.200pt]{2.409pt}{0.400pt}}
\put(1429.0,339.0){\rule[-0.200pt]{2.409pt}{0.400pt}}
\put(170.0,339.0){\rule[-0.200pt]{2.409pt}{0.400pt}}
\put(1429.0,339.0){\rule[-0.200pt]{2.409pt}{0.400pt}}
\put(170.0,339.0){\rule[-0.200pt]{2.409pt}{0.400pt}}
\put(1429.0,339.0){\rule[-0.200pt]{2.409pt}{0.400pt}}
\put(170.0,339.0){\rule[-0.200pt]{2.409pt}{0.400pt}}
\put(1429.0,339.0){\rule[-0.200pt]{2.409pt}{0.400pt}}
\put(170.0,339.0){\rule[-0.200pt]{2.409pt}{0.400pt}}
\put(1429.0,339.0){\rule[-0.200pt]{2.409pt}{0.400pt}}
\put(170.0,339.0){\rule[-0.200pt]{2.409pt}{0.400pt}}
\put(1429.0,339.0){\rule[-0.200pt]{2.409pt}{0.400pt}}
\put(170.0,339.0){\rule[-0.200pt]{2.409pt}{0.400pt}}
\put(1429.0,339.0){\rule[-0.200pt]{2.409pt}{0.400pt}}
\put(170.0,339.0){\rule[-0.200pt]{2.409pt}{0.400pt}}
\put(1429.0,339.0){\rule[-0.200pt]{2.409pt}{0.400pt}}
\put(170.0,339.0){\rule[-0.200pt]{2.409pt}{0.400pt}}
\put(1429.0,339.0){\rule[-0.200pt]{2.409pt}{0.400pt}}
\put(170.0,339.0){\rule[-0.200pt]{2.409pt}{0.400pt}}
\put(1429.0,339.0){\rule[-0.200pt]{2.409pt}{0.400pt}}
\put(170.0,339.0){\rule[-0.200pt]{2.409pt}{0.400pt}}
\put(1429.0,339.0){\rule[-0.200pt]{2.409pt}{0.400pt}}
\put(170.0,340.0){\rule[-0.200pt]{2.409pt}{0.400pt}}
\put(1429.0,340.0){\rule[-0.200pt]{2.409pt}{0.400pt}}
\put(170.0,340.0){\rule[-0.200pt]{2.409pt}{0.400pt}}
\put(1429.0,340.0){\rule[-0.200pt]{2.409pt}{0.400pt}}
\put(170.0,340.0){\rule[-0.200pt]{2.409pt}{0.400pt}}
\put(1429.0,340.0){\rule[-0.200pt]{2.409pt}{0.400pt}}
\put(170.0,340.0){\rule[-0.200pt]{2.409pt}{0.400pt}}
\put(1429.0,340.0){\rule[-0.200pt]{2.409pt}{0.400pt}}
\put(170.0,340.0){\rule[-0.200pt]{2.409pt}{0.400pt}}
\put(1429.0,340.0){\rule[-0.200pt]{2.409pt}{0.400pt}}
\put(170.0,340.0){\rule[-0.200pt]{2.409pt}{0.400pt}}
\put(1429.0,340.0){\rule[-0.200pt]{2.409pt}{0.400pt}}
\put(170.0,340.0){\rule[-0.200pt]{2.409pt}{0.400pt}}
\put(1429.0,340.0){\rule[-0.200pt]{2.409pt}{0.400pt}}
\put(170.0,340.0){\rule[-0.200pt]{2.409pt}{0.400pt}}
\put(1429.0,340.0){\rule[-0.200pt]{2.409pt}{0.400pt}}
\put(170.0,340.0){\rule[-0.200pt]{2.409pt}{0.400pt}}
\put(1429.0,340.0){\rule[-0.200pt]{2.409pt}{0.400pt}}
\put(170.0,340.0){\rule[-0.200pt]{2.409pt}{0.400pt}}
\put(1429.0,340.0){\rule[-0.200pt]{2.409pt}{0.400pt}}
\put(170.0,340.0){\rule[-0.200pt]{2.409pt}{0.400pt}}
\put(1429.0,340.0){\rule[-0.200pt]{2.409pt}{0.400pt}}
\put(170.0,340.0){\rule[-0.200pt]{2.409pt}{0.400pt}}
\put(1429.0,340.0){\rule[-0.200pt]{2.409pt}{0.400pt}}
\put(170.0,340.0){\rule[-0.200pt]{2.409pt}{0.400pt}}
\put(1429.0,340.0){\rule[-0.200pt]{2.409pt}{0.400pt}}
\put(170.0,340.0){\rule[-0.200pt]{2.409pt}{0.400pt}}
\put(1429.0,340.0){\rule[-0.200pt]{2.409pt}{0.400pt}}
\put(170.0,340.0){\rule[-0.200pt]{2.409pt}{0.400pt}}
\put(1429.0,340.0){\rule[-0.200pt]{2.409pt}{0.400pt}}
\put(170.0,340.0){\rule[-0.200pt]{2.409pt}{0.400pt}}
\put(1429.0,340.0){\rule[-0.200pt]{2.409pt}{0.400pt}}
\put(170.0,340.0){\rule[-0.200pt]{2.409pt}{0.400pt}}
\put(1429.0,340.0){\rule[-0.200pt]{2.409pt}{0.400pt}}
\put(170.0,340.0){\rule[-0.200pt]{2.409pt}{0.400pt}}
\put(1429.0,340.0){\rule[-0.200pt]{2.409pt}{0.400pt}}
\put(170.0,340.0){\rule[-0.200pt]{2.409pt}{0.400pt}}
\put(1429.0,340.0){\rule[-0.200pt]{2.409pt}{0.400pt}}
\put(170.0,340.0){\rule[-0.200pt]{2.409pt}{0.400pt}}
\put(1429.0,340.0){\rule[-0.200pt]{2.409pt}{0.400pt}}
\put(170.0,340.0){\rule[-0.200pt]{2.409pt}{0.400pt}}
\put(1429.0,340.0){\rule[-0.200pt]{2.409pt}{0.400pt}}
\put(170.0,340.0){\rule[-0.200pt]{2.409pt}{0.400pt}}
\put(1429.0,340.0){\rule[-0.200pt]{2.409pt}{0.400pt}}
\put(170.0,340.0){\rule[-0.200pt]{2.409pt}{0.400pt}}
\put(1429.0,340.0){\rule[-0.200pt]{2.409pt}{0.400pt}}
\put(170.0,340.0){\rule[-0.200pt]{2.409pt}{0.400pt}}
\put(1429.0,340.0){\rule[-0.200pt]{2.409pt}{0.400pt}}
\put(170.0,340.0){\rule[-0.200pt]{2.409pt}{0.400pt}}
\put(1429.0,340.0){\rule[-0.200pt]{2.409pt}{0.400pt}}
\put(170.0,340.0){\rule[-0.200pt]{2.409pt}{0.400pt}}
\put(1429.0,340.0){\rule[-0.200pt]{2.409pt}{0.400pt}}
\put(170.0,341.0){\rule[-0.200pt]{2.409pt}{0.400pt}}
\put(1429.0,341.0){\rule[-0.200pt]{2.409pt}{0.400pt}}
\put(170.0,341.0){\rule[-0.200pt]{2.409pt}{0.400pt}}
\put(1429.0,341.0){\rule[-0.200pt]{2.409pt}{0.400pt}}
\put(170.0,341.0){\rule[-0.200pt]{2.409pt}{0.400pt}}
\put(1429.0,341.0){\rule[-0.200pt]{2.409pt}{0.400pt}}
\put(170.0,341.0){\rule[-0.200pt]{2.409pt}{0.400pt}}
\put(1429.0,341.0){\rule[-0.200pt]{2.409pt}{0.400pt}}
\put(170.0,341.0){\rule[-0.200pt]{2.409pt}{0.400pt}}
\put(1429.0,341.0){\rule[-0.200pt]{2.409pt}{0.400pt}}
\put(170.0,341.0){\rule[-0.200pt]{2.409pt}{0.400pt}}
\put(1429.0,341.0){\rule[-0.200pt]{2.409pt}{0.400pt}}
\put(170.0,341.0){\rule[-0.200pt]{2.409pt}{0.400pt}}
\put(1429.0,341.0){\rule[-0.200pt]{2.409pt}{0.400pt}}
\put(170.0,341.0){\rule[-0.200pt]{2.409pt}{0.400pt}}
\put(1429.0,341.0){\rule[-0.200pt]{2.409pt}{0.400pt}}
\put(170.0,341.0){\rule[-0.200pt]{2.409pt}{0.400pt}}
\put(1429.0,341.0){\rule[-0.200pt]{2.409pt}{0.400pt}}
\put(170.0,341.0){\rule[-0.200pt]{2.409pt}{0.400pt}}
\put(1429.0,341.0){\rule[-0.200pt]{2.409pt}{0.400pt}}
\put(170.0,341.0){\rule[-0.200pt]{2.409pt}{0.400pt}}
\put(1429.0,341.0){\rule[-0.200pt]{2.409pt}{0.400pt}}
\put(170.0,341.0){\rule[-0.200pt]{2.409pt}{0.400pt}}
\put(1429.0,341.0){\rule[-0.200pt]{2.409pt}{0.400pt}}
\put(170.0,341.0){\rule[-0.200pt]{2.409pt}{0.400pt}}
\put(1429.0,341.0){\rule[-0.200pt]{2.409pt}{0.400pt}}
\put(170.0,341.0){\rule[-0.200pt]{4.818pt}{0.400pt}}
\put(150,341){\makebox(0,0)[r]{ 1000}}
\put(1419.0,341.0){\rule[-0.200pt]{4.818pt}{0.400pt}}
\put(170.0,367.0){\rule[-0.200pt]{2.409pt}{0.400pt}}
\put(1429.0,367.0){\rule[-0.200pt]{2.409pt}{0.400pt}}
\put(170.0,382.0){\rule[-0.200pt]{2.409pt}{0.400pt}}
\put(1429.0,382.0){\rule[-0.200pt]{2.409pt}{0.400pt}}
\put(170.0,393.0){\rule[-0.200pt]{2.409pt}{0.400pt}}
\put(1429.0,393.0){\rule[-0.200pt]{2.409pt}{0.400pt}}
\put(170.0,401.0){\rule[-0.200pt]{2.409pt}{0.400pt}}
\put(1429.0,401.0){\rule[-0.200pt]{2.409pt}{0.400pt}}
\put(170.0,408.0){\rule[-0.200pt]{2.409pt}{0.400pt}}
\put(1429.0,408.0){\rule[-0.200pt]{2.409pt}{0.400pt}}
\put(170.0,414.0){\rule[-0.200pt]{2.409pt}{0.400pt}}
\put(1429.0,414.0){\rule[-0.200pt]{2.409pt}{0.400pt}}
\put(170.0,419.0){\rule[-0.200pt]{2.409pt}{0.400pt}}
\put(1429.0,419.0){\rule[-0.200pt]{2.409pt}{0.400pt}}
\put(170.0,423.0){\rule[-0.200pt]{2.409pt}{0.400pt}}
\put(1429.0,423.0){\rule[-0.200pt]{2.409pt}{0.400pt}}
\put(170.0,427.0){\rule[-0.200pt]{2.409pt}{0.400pt}}
\put(1429.0,427.0){\rule[-0.200pt]{2.409pt}{0.400pt}}
\put(170.0,431.0){\rule[-0.200pt]{2.409pt}{0.400pt}}
\put(1429.0,431.0){\rule[-0.200pt]{2.409pt}{0.400pt}}
\put(170.0,434.0){\rule[-0.200pt]{2.409pt}{0.400pt}}
\put(1429.0,434.0){\rule[-0.200pt]{2.409pt}{0.400pt}}
\put(170.0,437.0){\rule[-0.200pt]{2.409pt}{0.400pt}}
\put(1429.0,437.0){\rule[-0.200pt]{2.409pt}{0.400pt}}
\put(170.0,440.0){\rule[-0.200pt]{2.409pt}{0.400pt}}
\put(1429.0,440.0){\rule[-0.200pt]{2.409pt}{0.400pt}}
\put(170.0,443.0){\rule[-0.200pt]{2.409pt}{0.400pt}}
\put(1429.0,443.0){\rule[-0.200pt]{2.409pt}{0.400pt}}
\put(170.0,445.0){\rule[-0.200pt]{2.409pt}{0.400pt}}
\put(1429.0,445.0){\rule[-0.200pt]{2.409pt}{0.400pt}}
\put(170.0,447.0){\rule[-0.200pt]{2.409pt}{0.400pt}}
\put(1429.0,447.0){\rule[-0.200pt]{2.409pt}{0.400pt}}
\put(170.0,449.0){\rule[-0.200pt]{2.409pt}{0.400pt}}
\put(1429.0,449.0){\rule[-0.200pt]{2.409pt}{0.400pt}}
\put(170.0,451.0){\rule[-0.200pt]{2.409pt}{0.400pt}}
\put(1429.0,451.0){\rule[-0.200pt]{2.409pt}{0.400pt}}
\put(170.0,453.0){\rule[-0.200pt]{2.409pt}{0.400pt}}
\put(1429.0,453.0){\rule[-0.200pt]{2.409pt}{0.400pt}}
\put(170.0,455.0){\rule[-0.200pt]{2.409pt}{0.400pt}}
\put(1429.0,455.0){\rule[-0.200pt]{2.409pt}{0.400pt}}
\put(170.0,457.0){\rule[-0.200pt]{2.409pt}{0.400pt}}
\put(1429.0,457.0){\rule[-0.200pt]{2.409pt}{0.400pt}}
\put(170.0,459.0){\rule[-0.200pt]{2.409pt}{0.400pt}}
\put(1429.0,459.0){\rule[-0.200pt]{2.409pt}{0.400pt}}
\put(170.0,460.0){\rule[-0.200pt]{2.409pt}{0.400pt}}
\put(1429.0,460.0){\rule[-0.200pt]{2.409pt}{0.400pt}}
\put(170.0,462.0){\rule[-0.200pt]{2.409pt}{0.400pt}}
\put(1429.0,462.0){\rule[-0.200pt]{2.409pt}{0.400pt}}
\put(170.0,463.0){\rule[-0.200pt]{2.409pt}{0.400pt}}
\put(1429.0,463.0){\rule[-0.200pt]{2.409pt}{0.400pt}}
\put(170.0,465.0){\rule[-0.200pt]{2.409pt}{0.400pt}}
\put(1429.0,465.0){\rule[-0.200pt]{2.409pt}{0.400pt}}
\put(170.0,466.0){\rule[-0.200pt]{2.409pt}{0.400pt}}
\put(1429.0,466.0){\rule[-0.200pt]{2.409pt}{0.400pt}}
\put(170.0,467.0){\rule[-0.200pt]{2.409pt}{0.400pt}}
\put(1429.0,467.0){\rule[-0.200pt]{2.409pt}{0.400pt}}
\put(170.0,469.0){\rule[-0.200pt]{2.409pt}{0.400pt}}
\put(1429.0,469.0){\rule[-0.200pt]{2.409pt}{0.400pt}}
\put(170.0,470.0){\rule[-0.200pt]{2.409pt}{0.400pt}}
\put(1429.0,470.0){\rule[-0.200pt]{2.409pt}{0.400pt}}
\put(170.0,471.0){\rule[-0.200pt]{2.409pt}{0.400pt}}
\put(1429.0,471.0){\rule[-0.200pt]{2.409pt}{0.400pt}}
\put(170.0,472.0){\rule[-0.200pt]{2.409pt}{0.400pt}}
\put(1429.0,472.0){\rule[-0.200pt]{2.409pt}{0.400pt}}
\put(170.0,473.0){\rule[-0.200pt]{2.409pt}{0.400pt}}
\put(1429.0,473.0){\rule[-0.200pt]{2.409pt}{0.400pt}}
\put(170.0,474.0){\rule[-0.200pt]{2.409pt}{0.400pt}}
\put(1429.0,474.0){\rule[-0.200pt]{2.409pt}{0.400pt}}
\put(170.0,475.0){\rule[-0.200pt]{2.409pt}{0.400pt}}
\put(1429.0,475.0){\rule[-0.200pt]{2.409pt}{0.400pt}}
\put(170.0,476.0){\rule[-0.200pt]{2.409pt}{0.400pt}}
\put(1429.0,476.0){\rule[-0.200pt]{2.409pt}{0.400pt}}
\put(170.0,477.0){\rule[-0.200pt]{2.409pt}{0.400pt}}
\put(1429.0,477.0){\rule[-0.200pt]{2.409pt}{0.400pt}}
\put(170.0,478.0){\rule[-0.200pt]{2.409pt}{0.400pt}}
\put(1429.0,478.0){\rule[-0.200pt]{2.409pt}{0.400pt}}
\put(170.0,479.0){\rule[-0.200pt]{2.409pt}{0.400pt}}
\put(1429.0,479.0){\rule[-0.200pt]{2.409pt}{0.400pt}}
\put(170.0,480.0){\rule[-0.200pt]{2.409pt}{0.400pt}}
\put(1429.0,480.0){\rule[-0.200pt]{2.409pt}{0.400pt}}
\put(170.0,481.0){\rule[-0.200pt]{2.409pt}{0.400pt}}
\put(1429.0,481.0){\rule[-0.200pt]{2.409pt}{0.400pt}}
\put(170.0,482.0){\rule[-0.200pt]{2.409pt}{0.400pt}}
\put(1429.0,482.0){\rule[-0.200pt]{2.409pt}{0.400pt}}
\put(170.0,483.0){\rule[-0.200pt]{2.409pt}{0.400pt}}
\put(1429.0,483.0){\rule[-0.200pt]{2.409pt}{0.400pt}}
\put(170.0,484.0){\rule[-0.200pt]{2.409pt}{0.400pt}}
\put(1429.0,484.0){\rule[-0.200pt]{2.409pt}{0.400pt}}
\put(170.0,485.0){\rule[-0.200pt]{2.409pt}{0.400pt}}
\put(1429.0,485.0){\rule[-0.200pt]{2.409pt}{0.400pt}}
\put(170.0,485.0){\rule[-0.200pt]{2.409pt}{0.400pt}}
\put(1429.0,485.0){\rule[-0.200pt]{2.409pt}{0.400pt}}
\put(170.0,486.0){\rule[-0.200pt]{2.409pt}{0.400pt}}
\put(1429.0,486.0){\rule[-0.200pt]{2.409pt}{0.400pt}}
\put(170.0,487.0){\rule[-0.200pt]{2.409pt}{0.400pt}}
\put(1429.0,487.0){\rule[-0.200pt]{2.409pt}{0.400pt}}
\put(170.0,488.0){\rule[-0.200pt]{2.409pt}{0.400pt}}
\put(1429.0,488.0){\rule[-0.200pt]{2.409pt}{0.400pt}}
\put(170.0,488.0){\rule[-0.200pt]{2.409pt}{0.400pt}}
\put(1429.0,488.0){\rule[-0.200pt]{2.409pt}{0.400pt}}
\put(170.0,489.0){\rule[-0.200pt]{2.409pt}{0.400pt}}
\put(1429.0,489.0){\rule[-0.200pt]{2.409pt}{0.400pt}}
\put(170.0,490.0){\rule[-0.200pt]{2.409pt}{0.400pt}}
\put(1429.0,490.0){\rule[-0.200pt]{2.409pt}{0.400pt}}
\put(170.0,491.0){\rule[-0.200pt]{2.409pt}{0.400pt}}
\put(1429.0,491.0){\rule[-0.200pt]{2.409pt}{0.400pt}}
\put(170.0,491.0){\rule[-0.200pt]{2.409pt}{0.400pt}}
\put(1429.0,491.0){\rule[-0.200pt]{2.409pt}{0.400pt}}
\put(170.0,492.0){\rule[-0.200pt]{2.409pt}{0.400pt}}
\put(1429.0,492.0){\rule[-0.200pt]{2.409pt}{0.400pt}}
\put(170.0,493.0){\rule[-0.200pt]{2.409pt}{0.400pt}}
\put(1429.0,493.0){\rule[-0.200pt]{2.409pt}{0.400pt}}
\put(170.0,493.0){\rule[-0.200pt]{2.409pt}{0.400pt}}
\put(1429.0,493.0){\rule[-0.200pt]{2.409pt}{0.400pt}}
\put(170.0,494.0){\rule[-0.200pt]{2.409pt}{0.400pt}}
\put(1429.0,494.0){\rule[-0.200pt]{2.409pt}{0.400pt}}
\put(170.0,495.0){\rule[-0.200pt]{2.409pt}{0.400pt}}
\put(1429.0,495.0){\rule[-0.200pt]{2.409pt}{0.400pt}}
\put(170.0,495.0){\rule[-0.200pt]{2.409pt}{0.400pt}}
\put(1429.0,495.0){\rule[-0.200pt]{2.409pt}{0.400pt}}
\put(170.0,496.0){\rule[-0.200pt]{2.409pt}{0.400pt}}
\put(1429.0,496.0){\rule[-0.200pt]{2.409pt}{0.400pt}}
\put(170.0,496.0){\rule[-0.200pt]{2.409pt}{0.400pt}}
\put(1429.0,496.0){\rule[-0.200pt]{2.409pt}{0.400pt}}
\put(170.0,497.0){\rule[-0.200pt]{2.409pt}{0.400pt}}
\put(1429.0,497.0){\rule[-0.200pt]{2.409pt}{0.400pt}}
\put(170.0,498.0){\rule[-0.200pt]{2.409pt}{0.400pt}}
\put(1429.0,498.0){\rule[-0.200pt]{2.409pt}{0.400pt}}
\put(170.0,498.0){\rule[-0.200pt]{2.409pt}{0.400pt}}
\put(1429.0,498.0){\rule[-0.200pt]{2.409pt}{0.400pt}}
\put(170.0,499.0){\rule[-0.200pt]{2.409pt}{0.400pt}}
\put(1429.0,499.0){\rule[-0.200pt]{2.409pt}{0.400pt}}
\put(170.0,499.0){\rule[-0.200pt]{2.409pt}{0.400pt}}
\put(1429.0,499.0){\rule[-0.200pt]{2.409pt}{0.400pt}}
\put(170.0,500.0){\rule[-0.200pt]{2.409pt}{0.400pt}}
\put(1429.0,500.0){\rule[-0.200pt]{2.409pt}{0.400pt}}
\put(170.0,500.0){\rule[-0.200pt]{2.409pt}{0.400pt}}
\put(1429.0,500.0){\rule[-0.200pt]{2.409pt}{0.400pt}}
\put(170.0,501.0){\rule[-0.200pt]{2.409pt}{0.400pt}}
\put(1429.0,501.0){\rule[-0.200pt]{2.409pt}{0.400pt}}
\put(170.0,501.0){\rule[-0.200pt]{2.409pt}{0.400pt}}
\put(1429.0,501.0){\rule[-0.200pt]{2.409pt}{0.400pt}}
\put(170.0,502.0){\rule[-0.200pt]{2.409pt}{0.400pt}}
\put(1429.0,502.0){\rule[-0.200pt]{2.409pt}{0.400pt}}
\put(170.0,502.0){\rule[-0.200pt]{2.409pt}{0.400pt}}
\put(1429.0,502.0){\rule[-0.200pt]{2.409pt}{0.400pt}}
\put(170.0,503.0){\rule[-0.200pt]{2.409pt}{0.400pt}}
\put(1429.0,503.0){\rule[-0.200pt]{2.409pt}{0.400pt}}
\put(170.0,503.0){\rule[-0.200pt]{2.409pt}{0.400pt}}
\put(1429.0,503.0){\rule[-0.200pt]{2.409pt}{0.400pt}}
\put(170.0,504.0){\rule[-0.200pt]{2.409pt}{0.400pt}}
\put(1429.0,504.0){\rule[-0.200pt]{2.409pt}{0.400pt}}
\put(170.0,504.0){\rule[-0.200pt]{2.409pt}{0.400pt}}
\put(1429.0,504.0){\rule[-0.200pt]{2.409pt}{0.400pt}}
\put(170.0,505.0){\rule[-0.200pt]{2.409pt}{0.400pt}}
\put(1429.0,505.0){\rule[-0.200pt]{2.409pt}{0.400pt}}
\put(170.0,505.0){\rule[-0.200pt]{2.409pt}{0.400pt}}
\put(1429.0,505.0){\rule[-0.200pt]{2.409pt}{0.400pt}}
\put(170.0,506.0){\rule[-0.200pt]{2.409pt}{0.400pt}}
\put(1429.0,506.0){\rule[-0.200pt]{2.409pt}{0.400pt}}
\put(170.0,506.0){\rule[-0.200pt]{2.409pt}{0.400pt}}
\put(1429.0,506.0){\rule[-0.200pt]{2.409pt}{0.400pt}}
\put(170.0,507.0){\rule[-0.200pt]{2.409pt}{0.400pt}}
\put(1429.0,507.0){\rule[-0.200pt]{2.409pt}{0.400pt}}
\put(170.0,507.0){\rule[-0.200pt]{2.409pt}{0.400pt}}
\put(1429.0,507.0){\rule[-0.200pt]{2.409pt}{0.400pt}}
\put(170.0,508.0){\rule[-0.200pt]{2.409pt}{0.400pt}}
\put(1429.0,508.0){\rule[-0.200pt]{2.409pt}{0.400pt}}
\put(170.0,508.0){\rule[-0.200pt]{2.409pt}{0.400pt}}
\put(1429.0,508.0){\rule[-0.200pt]{2.409pt}{0.400pt}}
\put(170.0,508.0){\rule[-0.200pt]{2.409pt}{0.400pt}}
\put(1429.0,508.0){\rule[-0.200pt]{2.409pt}{0.400pt}}
\put(170.0,509.0){\rule[-0.200pt]{2.409pt}{0.400pt}}
\put(1429.0,509.0){\rule[-0.200pt]{2.409pt}{0.400pt}}
\put(170.0,509.0){\rule[-0.200pt]{2.409pt}{0.400pt}}
\put(1429.0,509.0){\rule[-0.200pt]{2.409pt}{0.400pt}}
\put(170.0,510.0){\rule[-0.200pt]{2.409pt}{0.400pt}}
\put(1429.0,510.0){\rule[-0.200pt]{2.409pt}{0.400pt}}
\put(170.0,510.0){\rule[-0.200pt]{2.409pt}{0.400pt}}
\put(1429.0,510.0){\rule[-0.200pt]{2.409pt}{0.400pt}}
\put(170.0,511.0){\rule[-0.200pt]{2.409pt}{0.400pt}}
\put(1429.0,511.0){\rule[-0.200pt]{2.409pt}{0.400pt}}
\put(170.0,511.0){\rule[-0.200pt]{2.409pt}{0.400pt}}
\put(1429.0,511.0){\rule[-0.200pt]{2.409pt}{0.400pt}}
\put(170.0,511.0){\rule[-0.200pt]{2.409pt}{0.400pt}}
\put(1429.0,511.0){\rule[-0.200pt]{2.409pt}{0.400pt}}
\put(170.0,512.0){\rule[-0.200pt]{2.409pt}{0.400pt}}
\put(1429.0,512.0){\rule[-0.200pt]{2.409pt}{0.400pt}}
\put(170.0,512.0){\rule[-0.200pt]{2.409pt}{0.400pt}}
\put(1429.0,512.0){\rule[-0.200pt]{2.409pt}{0.400pt}}
\put(170.0,513.0){\rule[-0.200pt]{2.409pt}{0.400pt}}
\put(1429.0,513.0){\rule[-0.200pt]{2.409pt}{0.400pt}}
\put(170.0,513.0){\rule[-0.200pt]{2.409pt}{0.400pt}}
\put(1429.0,513.0){\rule[-0.200pt]{2.409pt}{0.400pt}}
\put(170.0,513.0){\rule[-0.200pt]{2.409pt}{0.400pt}}
\put(1429.0,513.0){\rule[-0.200pt]{2.409pt}{0.400pt}}
\put(170.0,514.0){\rule[-0.200pt]{2.409pt}{0.400pt}}
\put(1429.0,514.0){\rule[-0.200pt]{2.409pt}{0.400pt}}
\put(170.0,514.0){\rule[-0.200pt]{2.409pt}{0.400pt}}
\put(1429.0,514.0){\rule[-0.200pt]{2.409pt}{0.400pt}}
\put(170.0,514.0){\rule[-0.200pt]{2.409pt}{0.400pt}}
\put(1429.0,514.0){\rule[-0.200pt]{2.409pt}{0.400pt}}
\put(170.0,515.0){\rule[-0.200pt]{2.409pt}{0.400pt}}
\put(1429.0,515.0){\rule[-0.200pt]{2.409pt}{0.400pt}}
\put(170.0,515.0){\rule[-0.200pt]{2.409pt}{0.400pt}}
\put(1429.0,515.0){\rule[-0.200pt]{2.409pt}{0.400pt}}
\put(170.0,515.0){\rule[-0.200pt]{2.409pt}{0.400pt}}
\put(1429.0,515.0){\rule[-0.200pt]{2.409pt}{0.400pt}}
\put(170.0,516.0){\rule[-0.200pt]{2.409pt}{0.400pt}}
\put(1429.0,516.0){\rule[-0.200pt]{2.409pt}{0.400pt}}
\put(170.0,516.0){\rule[-0.200pt]{2.409pt}{0.400pt}}
\put(1429.0,516.0){\rule[-0.200pt]{2.409pt}{0.400pt}}
\put(170.0,517.0){\rule[-0.200pt]{2.409pt}{0.400pt}}
\put(1429.0,517.0){\rule[-0.200pt]{2.409pt}{0.400pt}}
\put(170.0,517.0){\rule[-0.200pt]{2.409pt}{0.400pt}}
\put(1429.0,517.0){\rule[-0.200pt]{2.409pt}{0.400pt}}
\put(170.0,517.0){\rule[-0.200pt]{2.409pt}{0.400pt}}
\put(1429.0,517.0){\rule[-0.200pt]{2.409pt}{0.400pt}}
\put(170.0,518.0){\rule[-0.200pt]{2.409pt}{0.400pt}}
\put(1429.0,518.0){\rule[-0.200pt]{2.409pt}{0.400pt}}
\put(170.0,518.0){\rule[-0.200pt]{2.409pt}{0.400pt}}
\put(1429.0,518.0){\rule[-0.200pt]{2.409pt}{0.400pt}}
\put(170.0,518.0){\rule[-0.200pt]{2.409pt}{0.400pt}}
\put(1429.0,518.0){\rule[-0.200pt]{2.409pt}{0.400pt}}
\put(170.0,519.0){\rule[-0.200pt]{2.409pt}{0.400pt}}
\put(1429.0,519.0){\rule[-0.200pt]{2.409pt}{0.400pt}}
\put(170.0,519.0){\rule[-0.200pt]{2.409pt}{0.400pt}}
\put(1429.0,519.0){\rule[-0.200pt]{2.409pt}{0.400pt}}
\put(170.0,519.0){\rule[-0.200pt]{2.409pt}{0.400pt}}
\put(1429.0,519.0){\rule[-0.200pt]{2.409pt}{0.400pt}}
\put(170.0,520.0){\rule[-0.200pt]{2.409pt}{0.400pt}}
\put(1429.0,520.0){\rule[-0.200pt]{2.409pt}{0.400pt}}
\put(170.0,520.0){\rule[-0.200pt]{2.409pt}{0.400pt}}
\put(1429.0,520.0){\rule[-0.200pt]{2.409pt}{0.400pt}}
\put(170.0,520.0){\rule[-0.200pt]{2.409pt}{0.400pt}}
\put(1429.0,520.0){\rule[-0.200pt]{2.409pt}{0.400pt}}
\put(170.0,521.0){\rule[-0.200pt]{2.409pt}{0.400pt}}
\put(1429.0,521.0){\rule[-0.200pt]{2.409pt}{0.400pt}}
\put(170.0,521.0){\rule[-0.200pt]{2.409pt}{0.400pt}}
\put(1429.0,521.0){\rule[-0.200pt]{2.409pt}{0.400pt}}
\put(170.0,521.0){\rule[-0.200pt]{2.409pt}{0.400pt}}
\put(1429.0,521.0){\rule[-0.200pt]{2.409pt}{0.400pt}}
\put(170.0,521.0){\rule[-0.200pt]{2.409pt}{0.400pt}}
\put(1429.0,521.0){\rule[-0.200pt]{2.409pt}{0.400pt}}
\put(170.0,522.0){\rule[-0.200pt]{2.409pt}{0.400pt}}
\put(1429.0,522.0){\rule[-0.200pt]{2.409pt}{0.400pt}}
\put(170.0,522.0){\rule[-0.200pt]{2.409pt}{0.400pt}}
\put(1429.0,522.0){\rule[-0.200pt]{2.409pt}{0.400pt}}
\put(170.0,522.0){\rule[-0.200pt]{2.409pt}{0.400pt}}
\put(1429.0,522.0){\rule[-0.200pt]{2.409pt}{0.400pt}}
\put(170.0,523.0){\rule[-0.200pt]{2.409pt}{0.400pt}}
\put(1429.0,523.0){\rule[-0.200pt]{2.409pt}{0.400pt}}
\put(170.0,523.0){\rule[-0.200pt]{2.409pt}{0.400pt}}
\put(1429.0,523.0){\rule[-0.200pt]{2.409pt}{0.400pt}}
\put(170.0,523.0){\rule[-0.200pt]{2.409pt}{0.400pt}}
\put(1429.0,523.0){\rule[-0.200pt]{2.409pt}{0.400pt}}
\put(170.0,524.0){\rule[-0.200pt]{2.409pt}{0.400pt}}
\put(1429.0,524.0){\rule[-0.200pt]{2.409pt}{0.400pt}}
\put(170.0,524.0){\rule[-0.200pt]{2.409pt}{0.400pt}}
\put(1429.0,524.0){\rule[-0.200pt]{2.409pt}{0.400pt}}
\put(170.0,524.0){\rule[-0.200pt]{2.409pt}{0.400pt}}
\put(1429.0,524.0){\rule[-0.200pt]{2.409pt}{0.400pt}}
\put(170.0,524.0){\rule[-0.200pt]{2.409pt}{0.400pt}}
\put(1429.0,524.0){\rule[-0.200pt]{2.409pt}{0.400pt}}
\put(170.0,525.0){\rule[-0.200pt]{2.409pt}{0.400pt}}
\put(1429.0,525.0){\rule[-0.200pt]{2.409pt}{0.400pt}}
\put(170.0,525.0){\rule[-0.200pt]{2.409pt}{0.400pt}}
\put(1429.0,525.0){\rule[-0.200pt]{2.409pt}{0.400pt}}
\put(170.0,525.0){\rule[-0.200pt]{2.409pt}{0.400pt}}
\put(1429.0,525.0){\rule[-0.200pt]{2.409pt}{0.400pt}}
\put(170.0,525.0){\rule[-0.200pt]{2.409pt}{0.400pt}}
\put(1429.0,525.0){\rule[-0.200pt]{2.409pt}{0.400pt}}
\put(170.0,526.0){\rule[-0.200pt]{2.409pt}{0.400pt}}
\put(1429.0,526.0){\rule[-0.200pt]{2.409pt}{0.400pt}}
\put(170.0,526.0){\rule[-0.200pt]{2.409pt}{0.400pt}}
\put(1429.0,526.0){\rule[-0.200pt]{2.409pt}{0.400pt}}
\put(170.0,526.0){\rule[-0.200pt]{2.409pt}{0.400pt}}
\put(1429.0,526.0){\rule[-0.200pt]{2.409pt}{0.400pt}}
\put(170.0,527.0){\rule[-0.200pt]{2.409pt}{0.400pt}}
\put(1429.0,527.0){\rule[-0.200pt]{2.409pt}{0.400pt}}
\put(170.0,527.0){\rule[-0.200pt]{2.409pt}{0.400pt}}
\put(1429.0,527.0){\rule[-0.200pt]{2.409pt}{0.400pt}}
\put(170.0,527.0){\rule[-0.200pt]{2.409pt}{0.400pt}}
\put(1429.0,527.0){\rule[-0.200pt]{2.409pt}{0.400pt}}
\put(170.0,527.0){\rule[-0.200pt]{2.409pt}{0.400pt}}
\put(1429.0,527.0){\rule[-0.200pt]{2.409pt}{0.400pt}}
\put(170.0,528.0){\rule[-0.200pt]{2.409pt}{0.400pt}}
\put(1429.0,528.0){\rule[-0.200pt]{2.409pt}{0.400pt}}
\put(170.0,528.0){\rule[-0.200pt]{2.409pt}{0.400pt}}
\put(1429.0,528.0){\rule[-0.200pt]{2.409pt}{0.400pt}}
\put(170.0,528.0){\rule[-0.200pt]{2.409pt}{0.400pt}}
\put(1429.0,528.0){\rule[-0.200pt]{2.409pt}{0.400pt}}
\put(170.0,528.0){\rule[-0.200pt]{2.409pt}{0.400pt}}
\put(1429.0,528.0){\rule[-0.200pt]{2.409pt}{0.400pt}}
\put(170.0,529.0){\rule[-0.200pt]{2.409pt}{0.400pt}}
\put(1429.0,529.0){\rule[-0.200pt]{2.409pt}{0.400pt}}
\put(170.0,529.0){\rule[-0.200pt]{2.409pt}{0.400pt}}
\put(1429.0,529.0){\rule[-0.200pt]{2.409pt}{0.400pt}}
\put(170.0,529.0){\rule[-0.200pt]{2.409pt}{0.400pt}}
\put(1429.0,529.0){\rule[-0.200pt]{2.409pt}{0.400pt}}
\put(170.0,529.0){\rule[-0.200pt]{2.409pt}{0.400pt}}
\put(1429.0,529.0){\rule[-0.200pt]{2.409pt}{0.400pt}}
\put(170.0,530.0){\rule[-0.200pt]{2.409pt}{0.400pt}}
\put(1429.0,530.0){\rule[-0.200pt]{2.409pt}{0.400pt}}
\put(170.0,530.0){\rule[-0.200pt]{2.409pt}{0.400pt}}
\put(1429.0,530.0){\rule[-0.200pt]{2.409pt}{0.400pt}}
\put(170.0,530.0){\rule[-0.200pt]{2.409pt}{0.400pt}}
\put(1429.0,530.0){\rule[-0.200pt]{2.409pt}{0.400pt}}
\put(170.0,530.0){\rule[-0.200pt]{2.409pt}{0.400pt}}
\put(1429.0,530.0){\rule[-0.200pt]{2.409pt}{0.400pt}}
\put(170.0,531.0){\rule[-0.200pt]{2.409pt}{0.400pt}}
\put(1429.0,531.0){\rule[-0.200pt]{2.409pt}{0.400pt}}
\put(170.0,531.0){\rule[-0.200pt]{2.409pt}{0.400pt}}
\put(1429.0,531.0){\rule[-0.200pt]{2.409pt}{0.400pt}}
\put(170.0,531.0){\rule[-0.200pt]{2.409pt}{0.400pt}}
\put(1429.0,531.0){\rule[-0.200pt]{2.409pt}{0.400pt}}
\put(170.0,531.0){\rule[-0.200pt]{2.409pt}{0.400pt}}
\put(1429.0,531.0){\rule[-0.200pt]{2.409pt}{0.400pt}}
\put(170.0,532.0){\rule[-0.200pt]{2.409pt}{0.400pt}}
\put(1429.0,532.0){\rule[-0.200pt]{2.409pt}{0.400pt}}
\put(170.0,532.0){\rule[-0.200pt]{2.409pt}{0.400pt}}
\put(1429.0,532.0){\rule[-0.200pt]{2.409pt}{0.400pt}}
\put(170.0,532.0){\rule[-0.200pt]{2.409pt}{0.400pt}}
\put(1429.0,532.0){\rule[-0.200pt]{2.409pt}{0.400pt}}
\put(170.0,532.0){\rule[-0.200pt]{2.409pt}{0.400pt}}
\put(1429.0,532.0){\rule[-0.200pt]{2.409pt}{0.400pt}}
\put(170.0,532.0){\rule[-0.200pt]{2.409pt}{0.400pt}}
\put(1429.0,532.0){\rule[-0.200pt]{2.409pt}{0.400pt}}
\put(170.0,533.0){\rule[-0.200pt]{2.409pt}{0.400pt}}
\put(1429.0,533.0){\rule[-0.200pt]{2.409pt}{0.400pt}}
\put(170.0,533.0){\rule[-0.200pt]{2.409pt}{0.400pt}}
\put(1429.0,533.0){\rule[-0.200pt]{2.409pt}{0.400pt}}
\put(170.0,533.0){\rule[-0.200pt]{2.409pt}{0.400pt}}
\put(1429.0,533.0){\rule[-0.200pt]{2.409pt}{0.400pt}}
\put(170.0,533.0){\rule[-0.200pt]{2.409pt}{0.400pt}}
\put(1429.0,533.0){\rule[-0.200pt]{2.409pt}{0.400pt}}
\put(170.0,534.0){\rule[-0.200pt]{2.409pt}{0.400pt}}
\put(1429.0,534.0){\rule[-0.200pt]{2.409pt}{0.400pt}}
\put(170.0,534.0){\rule[-0.200pt]{2.409pt}{0.400pt}}
\put(1429.0,534.0){\rule[-0.200pt]{2.409pt}{0.400pt}}
\put(170.0,534.0){\rule[-0.200pt]{2.409pt}{0.400pt}}
\put(1429.0,534.0){\rule[-0.200pt]{2.409pt}{0.400pt}}
\put(170.0,534.0){\rule[-0.200pt]{2.409pt}{0.400pt}}
\put(1429.0,534.0){\rule[-0.200pt]{2.409pt}{0.400pt}}
\put(170.0,534.0){\rule[-0.200pt]{2.409pt}{0.400pt}}
\put(1429.0,534.0){\rule[-0.200pt]{2.409pt}{0.400pt}}
\put(170.0,535.0){\rule[-0.200pt]{2.409pt}{0.400pt}}
\put(1429.0,535.0){\rule[-0.200pt]{2.409pt}{0.400pt}}
\put(170.0,535.0){\rule[-0.200pt]{2.409pt}{0.400pt}}
\put(1429.0,535.0){\rule[-0.200pt]{2.409pt}{0.400pt}}
\put(170.0,535.0){\rule[-0.200pt]{2.409pt}{0.400pt}}
\put(1429.0,535.0){\rule[-0.200pt]{2.409pt}{0.400pt}}
\put(170.0,535.0){\rule[-0.200pt]{2.409pt}{0.400pt}}
\put(1429.0,535.0){\rule[-0.200pt]{2.409pt}{0.400pt}}
\put(170.0,535.0){\rule[-0.200pt]{2.409pt}{0.400pt}}
\put(1429.0,535.0){\rule[-0.200pt]{2.409pt}{0.400pt}}
\put(170.0,536.0){\rule[-0.200pt]{2.409pt}{0.400pt}}
\put(1429.0,536.0){\rule[-0.200pt]{2.409pt}{0.400pt}}
\put(170.0,536.0){\rule[-0.200pt]{2.409pt}{0.400pt}}
\put(1429.0,536.0){\rule[-0.200pt]{2.409pt}{0.400pt}}
\put(170.0,536.0){\rule[-0.200pt]{2.409pt}{0.400pt}}
\put(1429.0,536.0){\rule[-0.200pt]{2.409pt}{0.400pt}}
\put(170.0,536.0){\rule[-0.200pt]{2.409pt}{0.400pt}}
\put(1429.0,536.0){\rule[-0.200pt]{2.409pt}{0.400pt}}
\put(170.0,537.0){\rule[-0.200pt]{2.409pt}{0.400pt}}
\put(1429.0,537.0){\rule[-0.200pt]{2.409pt}{0.400pt}}
\put(170.0,537.0){\rule[-0.200pt]{2.409pt}{0.400pt}}
\put(1429.0,537.0){\rule[-0.200pt]{2.409pt}{0.400pt}}
\put(170.0,537.0){\rule[-0.200pt]{2.409pt}{0.400pt}}
\put(1429.0,537.0){\rule[-0.200pt]{2.409pt}{0.400pt}}
\put(170.0,537.0){\rule[-0.200pt]{2.409pt}{0.400pt}}
\put(1429.0,537.0){\rule[-0.200pt]{2.409pt}{0.400pt}}
\put(170.0,537.0){\rule[-0.200pt]{2.409pt}{0.400pt}}
\put(1429.0,537.0){\rule[-0.200pt]{2.409pt}{0.400pt}}
\put(170.0,538.0){\rule[-0.200pt]{2.409pt}{0.400pt}}
\put(1429.0,538.0){\rule[-0.200pt]{2.409pt}{0.400pt}}
\put(170.0,538.0){\rule[-0.200pt]{2.409pt}{0.400pt}}
\put(1429.0,538.0){\rule[-0.200pt]{2.409pt}{0.400pt}}
\put(170.0,538.0){\rule[-0.200pt]{2.409pt}{0.400pt}}
\put(1429.0,538.0){\rule[-0.200pt]{2.409pt}{0.400pt}}
\put(170.0,538.0){\rule[-0.200pt]{2.409pt}{0.400pt}}
\put(1429.0,538.0){\rule[-0.200pt]{2.409pt}{0.400pt}}
\put(170.0,538.0){\rule[-0.200pt]{2.409pt}{0.400pt}}
\put(1429.0,538.0){\rule[-0.200pt]{2.409pt}{0.400pt}}
\put(170.0,539.0){\rule[-0.200pt]{2.409pt}{0.400pt}}
\put(1429.0,539.0){\rule[-0.200pt]{2.409pt}{0.400pt}}
\put(170.0,539.0){\rule[-0.200pt]{2.409pt}{0.400pt}}
\put(1429.0,539.0){\rule[-0.200pt]{2.409pt}{0.400pt}}
\put(170.0,539.0){\rule[-0.200pt]{2.409pt}{0.400pt}}
\put(1429.0,539.0){\rule[-0.200pt]{2.409pt}{0.400pt}}
\put(170.0,539.0){\rule[-0.200pt]{2.409pt}{0.400pt}}
\put(1429.0,539.0){\rule[-0.200pt]{2.409pt}{0.400pt}}
\put(170.0,539.0){\rule[-0.200pt]{2.409pt}{0.400pt}}
\put(1429.0,539.0){\rule[-0.200pt]{2.409pt}{0.400pt}}
\put(170.0,539.0){\rule[-0.200pt]{2.409pt}{0.400pt}}
\put(1429.0,539.0){\rule[-0.200pt]{2.409pt}{0.400pt}}
\put(170.0,540.0){\rule[-0.200pt]{2.409pt}{0.400pt}}
\put(1429.0,540.0){\rule[-0.200pt]{2.409pt}{0.400pt}}
\put(170.0,540.0){\rule[-0.200pt]{2.409pt}{0.400pt}}
\put(1429.0,540.0){\rule[-0.200pt]{2.409pt}{0.400pt}}
\put(170.0,540.0){\rule[-0.200pt]{2.409pt}{0.400pt}}
\put(1429.0,540.0){\rule[-0.200pt]{2.409pt}{0.400pt}}
\put(170.0,540.0){\rule[-0.200pt]{2.409pt}{0.400pt}}
\put(1429.0,540.0){\rule[-0.200pt]{2.409pt}{0.400pt}}
\put(170.0,540.0){\rule[-0.200pt]{2.409pt}{0.400pt}}
\put(1429.0,540.0){\rule[-0.200pt]{2.409pt}{0.400pt}}
\put(170.0,541.0){\rule[-0.200pt]{2.409pt}{0.400pt}}
\put(1429.0,541.0){\rule[-0.200pt]{2.409pt}{0.400pt}}
\put(170.0,541.0){\rule[-0.200pt]{2.409pt}{0.400pt}}
\put(1429.0,541.0){\rule[-0.200pt]{2.409pt}{0.400pt}}
\put(170.0,541.0){\rule[-0.200pt]{2.409pt}{0.400pt}}
\put(1429.0,541.0){\rule[-0.200pt]{2.409pt}{0.400pt}}
\put(170.0,541.0){\rule[-0.200pt]{2.409pt}{0.400pt}}
\put(1429.0,541.0){\rule[-0.200pt]{2.409pt}{0.400pt}}
\put(170.0,541.0){\rule[-0.200pt]{2.409pt}{0.400pt}}
\put(1429.0,541.0){\rule[-0.200pt]{2.409pt}{0.400pt}}
\put(170.0,541.0){\rule[-0.200pt]{2.409pt}{0.400pt}}
\put(1429.0,541.0){\rule[-0.200pt]{2.409pt}{0.400pt}}
\put(170.0,542.0){\rule[-0.200pt]{2.409pt}{0.400pt}}
\put(1429.0,542.0){\rule[-0.200pt]{2.409pt}{0.400pt}}
\put(170.0,542.0){\rule[-0.200pt]{2.409pt}{0.400pt}}
\put(1429.0,542.0){\rule[-0.200pt]{2.409pt}{0.400pt}}
\put(170.0,542.0){\rule[-0.200pt]{2.409pt}{0.400pt}}
\put(1429.0,542.0){\rule[-0.200pt]{2.409pt}{0.400pt}}
\put(170.0,542.0){\rule[-0.200pt]{2.409pt}{0.400pt}}
\put(1429.0,542.0){\rule[-0.200pt]{2.409pt}{0.400pt}}
\put(170.0,542.0){\rule[-0.200pt]{2.409pt}{0.400pt}}
\put(1429.0,542.0){\rule[-0.200pt]{2.409pt}{0.400pt}}
\put(170.0,543.0){\rule[-0.200pt]{2.409pt}{0.400pt}}
\put(1429.0,543.0){\rule[-0.200pt]{2.409pt}{0.400pt}}
\put(170.0,543.0){\rule[-0.200pt]{2.409pt}{0.400pt}}
\put(1429.0,543.0){\rule[-0.200pt]{2.409pt}{0.400pt}}
\put(170.0,543.0){\rule[-0.200pt]{2.409pt}{0.400pt}}
\put(1429.0,543.0){\rule[-0.200pt]{2.409pt}{0.400pt}}
\put(170.0,543.0){\rule[-0.200pt]{2.409pt}{0.400pt}}
\put(1429.0,543.0){\rule[-0.200pt]{2.409pt}{0.400pt}}
\put(170.0,543.0){\rule[-0.200pt]{2.409pt}{0.400pt}}
\put(1429.0,543.0){\rule[-0.200pt]{2.409pt}{0.400pt}}
\put(170.0,543.0){\rule[-0.200pt]{2.409pt}{0.400pt}}
\put(1429.0,543.0){\rule[-0.200pt]{2.409pt}{0.400pt}}
\put(170.0,544.0){\rule[-0.200pt]{2.409pt}{0.400pt}}
\put(1429.0,544.0){\rule[-0.200pt]{2.409pt}{0.400pt}}
\put(170.0,544.0){\rule[-0.200pt]{2.409pt}{0.400pt}}
\put(1429.0,544.0){\rule[-0.200pt]{2.409pt}{0.400pt}}
\put(170.0,544.0){\rule[-0.200pt]{2.409pt}{0.400pt}}
\put(1429.0,544.0){\rule[-0.200pt]{2.409pt}{0.400pt}}
\put(170.0,544.0){\rule[-0.200pt]{2.409pt}{0.400pt}}
\put(1429.0,544.0){\rule[-0.200pt]{2.409pt}{0.400pt}}
\put(170.0,544.0){\rule[-0.200pt]{2.409pt}{0.400pt}}
\put(1429.0,544.0){\rule[-0.200pt]{2.409pt}{0.400pt}}
\put(170.0,544.0){\rule[-0.200pt]{2.409pt}{0.400pt}}
\put(1429.0,544.0){\rule[-0.200pt]{2.409pt}{0.400pt}}
\put(170.0,545.0){\rule[-0.200pt]{2.409pt}{0.400pt}}
\put(1429.0,545.0){\rule[-0.200pt]{2.409pt}{0.400pt}}
\put(170.0,545.0){\rule[-0.200pt]{2.409pt}{0.400pt}}
\put(1429.0,545.0){\rule[-0.200pt]{2.409pt}{0.400pt}}
\put(170.0,545.0){\rule[-0.200pt]{2.409pt}{0.400pt}}
\put(1429.0,545.0){\rule[-0.200pt]{2.409pt}{0.400pt}}
\put(170.0,545.0){\rule[-0.200pt]{2.409pt}{0.400pt}}
\put(1429.0,545.0){\rule[-0.200pt]{2.409pt}{0.400pt}}
\put(170.0,545.0){\rule[-0.200pt]{2.409pt}{0.400pt}}
\put(1429.0,545.0){\rule[-0.200pt]{2.409pt}{0.400pt}}
\put(170.0,545.0){\rule[-0.200pt]{2.409pt}{0.400pt}}
\put(1429.0,545.0){\rule[-0.200pt]{2.409pt}{0.400pt}}
\put(170.0,546.0){\rule[-0.200pt]{2.409pt}{0.400pt}}
\put(1429.0,546.0){\rule[-0.200pt]{2.409pt}{0.400pt}}
\put(170.0,546.0){\rule[-0.200pt]{2.409pt}{0.400pt}}
\put(1429.0,546.0){\rule[-0.200pt]{2.409pt}{0.400pt}}
\put(170.0,546.0){\rule[-0.200pt]{2.409pt}{0.400pt}}
\put(1429.0,546.0){\rule[-0.200pt]{2.409pt}{0.400pt}}
\put(170.0,546.0){\rule[-0.200pt]{2.409pt}{0.400pt}}
\put(1429.0,546.0){\rule[-0.200pt]{2.409pt}{0.400pt}}
\put(170.0,546.0){\rule[-0.200pt]{2.409pt}{0.400pt}}
\put(1429.0,546.0){\rule[-0.200pt]{2.409pt}{0.400pt}}
\put(170.0,546.0){\rule[-0.200pt]{2.409pt}{0.400pt}}
\put(1429.0,546.0){\rule[-0.200pt]{2.409pt}{0.400pt}}
\put(170.0,546.0){\rule[-0.200pt]{2.409pt}{0.400pt}}
\put(1429.0,546.0){\rule[-0.200pt]{2.409pt}{0.400pt}}
\put(170.0,547.0){\rule[-0.200pt]{2.409pt}{0.400pt}}
\put(1429.0,547.0){\rule[-0.200pt]{2.409pt}{0.400pt}}
\put(170.0,547.0){\rule[-0.200pt]{2.409pt}{0.400pt}}
\put(1429.0,547.0){\rule[-0.200pt]{2.409pt}{0.400pt}}
\put(170.0,547.0){\rule[-0.200pt]{2.409pt}{0.400pt}}
\put(1429.0,547.0){\rule[-0.200pt]{2.409pt}{0.400pt}}
\put(170.0,547.0){\rule[-0.200pt]{2.409pt}{0.400pt}}
\put(1429.0,547.0){\rule[-0.200pt]{2.409pt}{0.400pt}}
\put(170.0,547.0){\rule[-0.200pt]{2.409pt}{0.400pt}}
\put(1429.0,547.0){\rule[-0.200pt]{2.409pt}{0.400pt}}
\put(170.0,547.0){\rule[-0.200pt]{2.409pt}{0.400pt}}
\put(1429.0,547.0){\rule[-0.200pt]{2.409pt}{0.400pt}}
\put(170.0,548.0){\rule[-0.200pt]{2.409pt}{0.400pt}}
\put(1429.0,548.0){\rule[-0.200pt]{2.409pt}{0.400pt}}
\put(170.0,548.0){\rule[-0.200pt]{2.409pt}{0.400pt}}
\put(1429.0,548.0){\rule[-0.200pt]{2.409pt}{0.400pt}}
\put(170.0,548.0){\rule[-0.200pt]{2.409pt}{0.400pt}}
\put(1429.0,548.0){\rule[-0.200pt]{2.409pt}{0.400pt}}
\put(170.0,548.0){\rule[-0.200pt]{2.409pt}{0.400pt}}
\put(1429.0,548.0){\rule[-0.200pt]{2.409pt}{0.400pt}}
\put(170.0,548.0){\rule[-0.200pt]{2.409pt}{0.400pt}}
\put(1429.0,548.0){\rule[-0.200pt]{2.409pt}{0.400pt}}
\put(170.0,548.0){\rule[-0.200pt]{2.409pt}{0.400pt}}
\put(1429.0,548.0){\rule[-0.200pt]{2.409pt}{0.400pt}}
\put(170.0,548.0){\rule[-0.200pt]{2.409pt}{0.400pt}}
\put(1429.0,548.0){\rule[-0.200pt]{2.409pt}{0.400pt}}
\put(170.0,549.0){\rule[-0.200pt]{2.409pt}{0.400pt}}
\put(1429.0,549.0){\rule[-0.200pt]{2.409pt}{0.400pt}}
\put(170.0,549.0){\rule[-0.200pt]{2.409pt}{0.400pt}}
\put(1429.0,549.0){\rule[-0.200pt]{2.409pt}{0.400pt}}
\put(170.0,549.0){\rule[-0.200pt]{2.409pt}{0.400pt}}
\put(1429.0,549.0){\rule[-0.200pt]{2.409pt}{0.400pt}}
\put(170.0,549.0){\rule[-0.200pt]{2.409pt}{0.400pt}}
\put(1429.0,549.0){\rule[-0.200pt]{2.409pt}{0.400pt}}
\put(170.0,549.0){\rule[-0.200pt]{2.409pt}{0.400pt}}
\put(1429.0,549.0){\rule[-0.200pt]{2.409pt}{0.400pt}}
\put(170.0,549.0){\rule[-0.200pt]{2.409pt}{0.400pt}}
\put(1429.0,549.0){\rule[-0.200pt]{2.409pt}{0.400pt}}
\put(170.0,549.0){\rule[-0.200pt]{2.409pt}{0.400pt}}
\put(1429.0,549.0){\rule[-0.200pt]{2.409pt}{0.400pt}}
\put(170.0,550.0){\rule[-0.200pt]{2.409pt}{0.400pt}}
\put(1429.0,550.0){\rule[-0.200pt]{2.409pt}{0.400pt}}
\put(170.0,550.0){\rule[-0.200pt]{2.409pt}{0.400pt}}
\put(1429.0,550.0){\rule[-0.200pt]{2.409pt}{0.400pt}}
\put(170.0,550.0){\rule[-0.200pt]{2.409pt}{0.400pt}}
\put(1429.0,550.0){\rule[-0.200pt]{2.409pt}{0.400pt}}
\put(170.0,550.0){\rule[-0.200pt]{2.409pt}{0.400pt}}
\put(1429.0,550.0){\rule[-0.200pt]{2.409pt}{0.400pt}}
\put(170.0,550.0){\rule[-0.200pt]{2.409pt}{0.400pt}}
\put(1429.0,550.0){\rule[-0.200pt]{2.409pt}{0.400pt}}
\put(170.0,550.0){\rule[-0.200pt]{2.409pt}{0.400pt}}
\put(1429.0,550.0){\rule[-0.200pt]{2.409pt}{0.400pt}}
\put(170.0,550.0){\rule[-0.200pt]{2.409pt}{0.400pt}}
\put(1429.0,550.0){\rule[-0.200pt]{2.409pt}{0.400pt}}
\put(170.0,551.0){\rule[-0.200pt]{2.409pt}{0.400pt}}
\put(1429.0,551.0){\rule[-0.200pt]{2.409pt}{0.400pt}}
\put(170.0,551.0){\rule[-0.200pt]{2.409pt}{0.400pt}}
\put(1429.0,551.0){\rule[-0.200pt]{2.409pt}{0.400pt}}
\put(170.0,551.0){\rule[-0.200pt]{2.409pt}{0.400pt}}
\put(1429.0,551.0){\rule[-0.200pt]{2.409pt}{0.400pt}}
\put(170.0,551.0){\rule[-0.200pt]{2.409pt}{0.400pt}}
\put(1429.0,551.0){\rule[-0.200pt]{2.409pt}{0.400pt}}
\put(170.0,551.0){\rule[-0.200pt]{2.409pt}{0.400pt}}
\put(1429.0,551.0){\rule[-0.200pt]{2.409pt}{0.400pt}}
\put(170.0,551.0){\rule[-0.200pt]{2.409pt}{0.400pt}}
\put(1429.0,551.0){\rule[-0.200pt]{2.409pt}{0.400pt}}
\put(170.0,551.0){\rule[-0.200pt]{2.409pt}{0.400pt}}
\put(1429.0,551.0){\rule[-0.200pt]{2.409pt}{0.400pt}}
\put(170.0,552.0){\rule[-0.200pt]{2.409pt}{0.400pt}}
\put(1429.0,552.0){\rule[-0.200pt]{2.409pt}{0.400pt}}
\put(170.0,552.0){\rule[-0.200pt]{2.409pt}{0.400pt}}
\put(1429.0,552.0){\rule[-0.200pt]{2.409pt}{0.400pt}}
\put(170.0,552.0){\rule[-0.200pt]{2.409pt}{0.400pt}}
\put(1429.0,552.0){\rule[-0.200pt]{2.409pt}{0.400pt}}
\put(170.0,552.0){\rule[-0.200pt]{2.409pt}{0.400pt}}
\put(1429.0,552.0){\rule[-0.200pt]{2.409pt}{0.400pt}}
\put(170.0,552.0){\rule[-0.200pt]{2.409pt}{0.400pt}}
\put(1429.0,552.0){\rule[-0.200pt]{2.409pt}{0.400pt}}
\put(170.0,552.0){\rule[-0.200pt]{2.409pt}{0.400pt}}
\put(1429.0,552.0){\rule[-0.200pt]{2.409pt}{0.400pt}}
\put(170.0,552.0){\rule[-0.200pt]{2.409pt}{0.400pt}}
\put(1429.0,552.0){\rule[-0.200pt]{2.409pt}{0.400pt}}
\put(170.0,553.0){\rule[-0.200pt]{2.409pt}{0.400pt}}
\put(1429.0,553.0){\rule[-0.200pt]{2.409pt}{0.400pt}}
\put(170.0,553.0){\rule[-0.200pt]{2.409pt}{0.400pt}}
\put(1429.0,553.0){\rule[-0.200pt]{2.409pt}{0.400pt}}
\put(170.0,553.0){\rule[-0.200pt]{2.409pt}{0.400pt}}
\put(1429.0,553.0){\rule[-0.200pt]{2.409pt}{0.400pt}}
\put(170.0,553.0){\rule[-0.200pt]{2.409pt}{0.400pt}}
\put(1429.0,553.0){\rule[-0.200pt]{2.409pt}{0.400pt}}
\put(170.0,553.0){\rule[-0.200pt]{2.409pt}{0.400pt}}
\put(1429.0,553.0){\rule[-0.200pt]{2.409pt}{0.400pt}}
\put(170.0,553.0){\rule[-0.200pt]{2.409pt}{0.400pt}}
\put(1429.0,553.0){\rule[-0.200pt]{2.409pt}{0.400pt}}
\put(170.0,553.0){\rule[-0.200pt]{2.409pt}{0.400pt}}
\put(1429.0,553.0){\rule[-0.200pt]{2.409pt}{0.400pt}}
\put(170.0,553.0){\rule[-0.200pt]{2.409pt}{0.400pt}}
\put(1429.0,553.0){\rule[-0.200pt]{2.409pt}{0.400pt}}
\put(170.0,554.0){\rule[-0.200pt]{2.409pt}{0.400pt}}
\put(1429.0,554.0){\rule[-0.200pt]{2.409pt}{0.400pt}}
\put(170.0,554.0){\rule[-0.200pt]{2.409pt}{0.400pt}}
\put(1429.0,554.0){\rule[-0.200pt]{2.409pt}{0.400pt}}
\put(170.0,554.0){\rule[-0.200pt]{2.409pt}{0.400pt}}
\put(1429.0,554.0){\rule[-0.200pt]{2.409pt}{0.400pt}}
\put(170.0,554.0){\rule[-0.200pt]{2.409pt}{0.400pt}}
\put(1429.0,554.0){\rule[-0.200pt]{2.409pt}{0.400pt}}
\put(170.0,554.0){\rule[-0.200pt]{2.409pt}{0.400pt}}
\put(1429.0,554.0){\rule[-0.200pt]{2.409pt}{0.400pt}}
\put(170.0,554.0){\rule[-0.200pt]{2.409pt}{0.400pt}}
\put(1429.0,554.0){\rule[-0.200pt]{2.409pt}{0.400pt}}
\put(170.0,554.0){\rule[-0.200pt]{2.409pt}{0.400pt}}
\put(1429.0,554.0){\rule[-0.200pt]{2.409pt}{0.400pt}}
\put(170.0,554.0){\rule[-0.200pt]{2.409pt}{0.400pt}}
\put(1429.0,554.0){\rule[-0.200pt]{2.409pt}{0.400pt}}
\put(170.0,555.0){\rule[-0.200pt]{2.409pt}{0.400pt}}
\put(1429.0,555.0){\rule[-0.200pt]{2.409pt}{0.400pt}}
\put(170.0,555.0){\rule[-0.200pt]{2.409pt}{0.400pt}}
\put(1429.0,555.0){\rule[-0.200pt]{2.409pt}{0.400pt}}
\put(170.0,555.0){\rule[-0.200pt]{2.409pt}{0.400pt}}
\put(1429.0,555.0){\rule[-0.200pt]{2.409pt}{0.400pt}}
\put(170.0,555.0){\rule[-0.200pt]{2.409pt}{0.400pt}}
\put(1429.0,555.0){\rule[-0.200pt]{2.409pt}{0.400pt}}
\put(170.0,555.0){\rule[-0.200pt]{2.409pt}{0.400pt}}
\put(1429.0,555.0){\rule[-0.200pt]{2.409pt}{0.400pt}}
\put(170.0,555.0){\rule[-0.200pt]{2.409pt}{0.400pt}}
\put(1429.0,555.0){\rule[-0.200pt]{2.409pt}{0.400pt}}
\put(170.0,555.0){\rule[-0.200pt]{2.409pt}{0.400pt}}
\put(1429.0,555.0){\rule[-0.200pt]{2.409pt}{0.400pt}}
\put(170.0,555.0){\rule[-0.200pt]{2.409pt}{0.400pt}}
\put(1429.0,555.0){\rule[-0.200pt]{2.409pt}{0.400pt}}
\put(170.0,556.0){\rule[-0.200pt]{2.409pt}{0.400pt}}
\put(1429.0,556.0){\rule[-0.200pt]{2.409pt}{0.400pt}}
\put(170.0,556.0){\rule[-0.200pt]{2.409pt}{0.400pt}}
\put(1429.0,556.0){\rule[-0.200pt]{2.409pt}{0.400pt}}
\put(170.0,556.0){\rule[-0.200pt]{2.409pt}{0.400pt}}
\put(1429.0,556.0){\rule[-0.200pt]{2.409pt}{0.400pt}}
\put(170.0,556.0){\rule[-0.200pt]{2.409pt}{0.400pt}}
\put(1429.0,556.0){\rule[-0.200pt]{2.409pt}{0.400pt}}
\put(170.0,556.0){\rule[-0.200pt]{2.409pt}{0.400pt}}
\put(1429.0,556.0){\rule[-0.200pt]{2.409pt}{0.400pt}}
\put(170.0,556.0){\rule[-0.200pt]{2.409pt}{0.400pt}}
\put(1429.0,556.0){\rule[-0.200pt]{2.409pt}{0.400pt}}
\put(170.0,556.0){\rule[-0.200pt]{2.409pt}{0.400pt}}
\put(1429.0,556.0){\rule[-0.200pt]{2.409pt}{0.400pt}}
\put(170.0,556.0){\rule[-0.200pt]{2.409pt}{0.400pt}}
\put(1429.0,556.0){\rule[-0.200pt]{2.409pt}{0.400pt}}
\put(170.0,557.0){\rule[-0.200pt]{2.409pt}{0.400pt}}
\put(1429.0,557.0){\rule[-0.200pt]{2.409pt}{0.400pt}}
\put(170.0,557.0){\rule[-0.200pt]{2.409pt}{0.400pt}}
\put(1429.0,557.0){\rule[-0.200pt]{2.409pt}{0.400pt}}
\put(170.0,557.0){\rule[-0.200pt]{2.409pt}{0.400pt}}
\put(1429.0,557.0){\rule[-0.200pt]{2.409pt}{0.400pt}}
\put(170.0,557.0){\rule[-0.200pt]{2.409pt}{0.400pt}}
\put(1429.0,557.0){\rule[-0.200pt]{2.409pt}{0.400pt}}
\put(170.0,557.0){\rule[-0.200pt]{2.409pt}{0.400pt}}
\put(1429.0,557.0){\rule[-0.200pt]{2.409pt}{0.400pt}}
\put(170.0,557.0){\rule[-0.200pt]{2.409pt}{0.400pt}}
\put(1429.0,557.0){\rule[-0.200pt]{2.409pt}{0.400pt}}
\put(170.0,557.0){\rule[-0.200pt]{2.409pt}{0.400pt}}
\put(1429.0,557.0){\rule[-0.200pt]{2.409pt}{0.400pt}}
\put(170.0,557.0){\rule[-0.200pt]{2.409pt}{0.400pt}}
\put(1429.0,557.0){\rule[-0.200pt]{2.409pt}{0.400pt}}
\put(170.0,558.0){\rule[-0.200pt]{2.409pt}{0.400pt}}
\put(1429.0,558.0){\rule[-0.200pt]{2.409pt}{0.400pt}}
\put(170.0,558.0){\rule[-0.200pt]{2.409pt}{0.400pt}}
\put(1429.0,558.0){\rule[-0.200pt]{2.409pt}{0.400pt}}
\put(170.0,558.0){\rule[-0.200pt]{2.409pt}{0.400pt}}
\put(1429.0,558.0){\rule[-0.200pt]{2.409pt}{0.400pt}}
\put(170.0,558.0){\rule[-0.200pt]{2.409pt}{0.400pt}}
\put(1429.0,558.0){\rule[-0.200pt]{2.409pt}{0.400pt}}
\put(170.0,558.0){\rule[-0.200pt]{2.409pt}{0.400pt}}
\put(1429.0,558.0){\rule[-0.200pt]{2.409pt}{0.400pt}}
\put(170.0,558.0){\rule[-0.200pt]{2.409pt}{0.400pt}}
\put(1429.0,558.0){\rule[-0.200pt]{2.409pt}{0.400pt}}
\put(170.0,558.0){\rule[-0.200pt]{2.409pt}{0.400pt}}
\put(1429.0,558.0){\rule[-0.200pt]{2.409pt}{0.400pt}}
\put(170.0,558.0){\rule[-0.200pt]{2.409pt}{0.400pt}}
\put(1429.0,558.0){\rule[-0.200pt]{2.409pt}{0.400pt}}
\put(170.0,558.0){\rule[-0.200pt]{2.409pt}{0.400pt}}
\put(1429.0,558.0){\rule[-0.200pt]{2.409pt}{0.400pt}}
\put(170.0,559.0){\rule[-0.200pt]{2.409pt}{0.400pt}}
\put(1429.0,559.0){\rule[-0.200pt]{2.409pt}{0.400pt}}
\put(170.0,559.0){\rule[-0.200pt]{2.409pt}{0.400pt}}
\put(1429.0,559.0){\rule[-0.200pt]{2.409pt}{0.400pt}}
\put(170.0,559.0){\rule[-0.200pt]{2.409pt}{0.400pt}}
\put(1429.0,559.0){\rule[-0.200pt]{2.409pt}{0.400pt}}
\put(170.0,559.0){\rule[-0.200pt]{2.409pt}{0.400pt}}
\put(1429.0,559.0){\rule[-0.200pt]{2.409pt}{0.400pt}}
\put(170.0,559.0){\rule[-0.200pt]{2.409pt}{0.400pt}}
\put(1429.0,559.0){\rule[-0.200pt]{2.409pt}{0.400pt}}
\put(170.0,559.0){\rule[-0.200pt]{2.409pt}{0.400pt}}
\put(1429.0,559.0){\rule[-0.200pt]{2.409pt}{0.400pt}}
\put(170.0,559.0){\rule[-0.200pt]{2.409pt}{0.400pt}}
\put(1429.0,559.0){\rule[-0.200pt]{2.409pt}{0.400pt}}
\put(170.0,559.0){\rule[-0.200pt]{2.409pt}{0.400pt}}
\put(1429.0,559.0){\rule[-0.200pt]{2.409pt}{0.400pt}}
\put(170.0,559.0){\rule[-0.200pt]{2.409pt}{0.400pt}}
\put(1429.0,559.0){\rule[-0.200pt]{2.409pt}{0.400pt}}
\put(170.0,560.0){\rule[-0.200pt]{2.409pt}{0.400pt}}
\put(1429.0,560.0){\rule[-0.200pt]{2.409pt}{0.400pt}}
\put(170.0,560.0){\rule[-0.200pt]{2.409pt}{0.400pt}}
\put(1429.0,560.0){\rule[-0.200pt]{2.409pt}{0.400pt}}
\put(170.0,560.0){\rule[-0.200pt]{2.409pt}{0.400pt}}
\put(1429.0,560.0){\rule[-0.200pt]{2.409pt}{0.400pt}}
\put(170.0,560.0){\rule[-0.200pt]{2.409pt}{0.400pt}}
\put(1429.0,560.0){\rule[-0.200pt]{2.409pt}{0.400pt}}
\put(170.0,560.0){\rule[-0.200pt]{2.409pt}{0.400pt}}
\put(1429.0,560.0){\rule[-0.200pt]{2.409pt}{0.400pt}}
\put(170.0,560.0){\rule[-0.200pt]{2.409pt}{0.400pt}}
\put(1429.0,560.0){\rule[-0.200pt]{2.409pt}{0.400pt}}
\put(170.0,560.0){\rule[-0.200pt]{2.409pt}{0.400pt}}
\put(1429.0,560.0){\rule[-0.200pt]{2.409pt}{0.400pt}}
\put(170.0,560.0){\rule[-0.200pt]{2.409pt}{0.400pt}}
\put(1429.0,560.0){\rule[-0.200pt]{2.409pt}{0.400pt}}
\put(170.0,560.0){\rule[-0.200pt]{2.409pt}{0.400pt}}
\put(1429.0,560.0){\rule[-0.200pt]{2.409pt}{0.400pt}}
\put(170.0,561.0){\rule[-0.200pt]{2.409pt}{0.400pt}}
\put(1429.0,561.0){\rule[-0.200pt]{2.409pt}{0.400pt}}
\put(170.0,561.0){\rule[-0.200pt]{2.409pt}{0.400pt}}
\put(1429.0,561.0){\rule[-0.200pt]{2.409pt}{0.400pt}}
\put(170.0,561.0){\rule[-0.200pt]{2.409pt}{0.400pt}}
\put(1429.0,561.0){\rule[-0.200pt]{2.409pt}{0.400pt}}
\put(170.0,561.0){\rule[-0.200pt]{2.409pt}{0.400pt}}
\put(1429.0,561.0){\rule[-0.200pt]{2.409pt}{0.400pt}}
\put(170.0,561.0){\rule[-0.200pt]{2.409pt}{0.400pt}}
\put(1429.0,561.0){\rule[-0.200pt]{2.409pt}{0.400pt}}
\put(170.0,561.0){\rule[-0.200pt]{2.409pt}{0.400pt}}
\put(1429.0,561.0){\rule[-0.200pt]{2.409pt}{0.400pt}}
\put(170.0,561.0){\rule[-0.200pt]{2.409pt}{0.400pt}}
\put(1429.0,561.0){\rule[-0.200pt]{2.409pt}{0.400pt}}
\put(170.0,561.0){\rule[-0.200pt]{2.409pt}{0.400pt}}
\put(1429.0,561.0){\rule[-0.200pt]{2.409pt}{0.400pt}}
\put(170.0,561.0){\rule[-0.200pt]{2.409pt}{0.400pt}}
\put(1429.0,561.0){\rule[-0.200pt]{2.409pt}{0.400pt}}
\put(170.0,561.0){\rule[-0.200pt]{2.409pt}{0.400pt}}
\put(1429.0,561.0){\rule[-0.200pt]{2.409pt}{0.400pt}}
\put(170.0,562.0){\rule[-0.200pt]{2.409pt}{0.400pt}}
\put(1429.0,562.0){\rule[-0.200pt]{2.409pt}{0.400pt}}
\put(170.0,562.0){\rule[-0.200pt]{2.409pt}{0.400pt}}
\put(1429.0,562.0){\rule[-0.200pt]{2.409pt}{0.400pt}}
\put(170.0,562.0){\rule[-0.200pt]{2.409pt}{0.400pt}}
\put(1429.0,562.0){\rule[-0.200pt]{2.409pt}{0.400pt}}
\put(170.0,562.0){\rule[-0.200pt]{2.409pt}{0.400pt}}
\put(1429.0,562.0){\rule[-0.200pt]{2.409pt}{0.400pt}}
\put(170.0,562.0){\rule[-0.200pt]{2.409pt}{0.400pt}}
\put(1429.0,562.0){\rule[-0.200pt]{2.409pt}{0.400pt}}
\put(170.0,562.0){\rule[-0.200pt]{2.409pt}{0.400pt}}
\put(1429.0,562.0){\rule[-0.200pt]{2.409pt}{0.400pt}}
\put(170.0,562.0){\rule[-0.200pt]{2.409pt}{0.400pt}}
\put(1429.0,562.0){\rule[-0.200pt]{2.409pt}{0.400pt}}
\put(170.0,562.0){\rule[-0.200pt]{2.409pt}{0.400pt}}
\put(1429.0,562.0){\rule[-0.200pt]{2.409pt}{0.400pt}}
\put(170.0,562.0){\rule[-0.200pt]{2.409pt}{0.400pt}}
\put(1429.0,562.0){\rule[-0.200pt]{2.409pt}{0.400pt}}
\put(170.0,563.0){\rule[-0.200pt]{2.409pt}{0.400pt}}
\put(1429.0,563.0){\rule[-0.200pt]{2.409pt}{0.400pt}}
\put(170.0,563.0){\rule[-0.200pt]{2.409pt}{0.400pt}}
\put(1429.0,563.0){\rule[-0.200pt]{2.409pt}{0.400pt}}
\put(170.0,563.0){\rule[-0.200pt]{2.409pt}{0.400pt}}
\put(1429.0,563.0){\rule[-0.200pt]{2.409pt}{0.400pt}}
\put(170.0,563.0){\rule[-0.200pt]{2.409pt}{0.400pt}}
\put(1429.0,563.0){\rule[-0.200pt]{2.409pt}{0.400pt}}
\put(170.0,563.0){\rule[-0.200pt]{2.409pt}{0.400pt}}
\put(1429.0,563.0){\rule[-0.200pt]{2.409pt}{0.400pt}}
\put(170.0,563.0){\rule[-0.200pt]{2.409pt}{0.400pt}}
\put(1429.0,563.0){\rule[-0.200pt]{2.409pt}{0.400pt}}
\put(170.0,563.0){\rule[-0.200pt]{2.409pt}{0.400pt}}
\put(1429.0,563.0){\rule[-0.200pt]{2.409pt}{0.400pt}}
\put(170.0,563.0){\rule[-0.200pt]{2.409pt}{0.400pt}}
\put(1429.0,563.0){\rule[-0.200pt]{2.409pt}{0.400pt}}
\put(170.0,563.0){\rule[-0.200pt]{2.409pt}{0.400pt}}
\put(1429.0,563.0){\rule[-0.200pt]{2.409pt}{0.400pt}}
\put(170.0,563.0){\rule[-0.200pt]{2.409pt}{0.400pt}}
\put(1429.0,563.0){\rule[-0.200pt]{2.409pt}{0.400pt}}
\put(170.0,564.0){\rule[-0.200pt]{2.409pt}{0.400pt}}
\put(1429.0,564.0){\rule[-0.200pt]{2.409pt}{0.400pt}}
\put(170.0,564.0){\rule[-0.200pt]{2.409pt}{0.400pt}}
\put(1429.0,564.0){\rule[-0.200pt]{2.409pt}{0.400pt}}
\put(170.0,564.0){\rule[-0.200pt]{2.409pt}{0.400pt}}
\put(1429.0,564.0){\rule[-0.200pt]{2.409pt}{0.400pt}}
\put(170.0,564.0){\rule[-0.200pt]{2.409pt}{0.400pt}}
\put(1429.0,564.0){\rule[-0.200pt]{2.409pt}{0.400pt}}
\put(170.0,564.0){\rule[-0.200pt]{2.409pt}{0.400pt}}
\put(1429.0,564.0){\rule[-0.200pt]{2.409pt}{0.400pt}}
\put(170.0,564.0){\rule[-0.200pt]{2.409pt}{0.400pt}}
\put(1429.0,564.0){\rule[-0.200pt]{2.409pt}{0.400pt}}
\put(170.0,564.0){\rule[-0.200pt]{2.409pt}{0.400pt}}
\put(1429.0,564.0){\rule[-0.200pt]{2.409pt}{0.400pt}}
\put(170.0,564.0){\rule[-0.200pt]{2.409pt}{0.400pt}}
\put(1429.0,564.0){\rule[-0.200pt]{2.409pt}{0.400pt}}
\put(170.0,564.0){\rule[-0.200pt]{2.409pt}{0.400pt}}
\put(1429.0,564.0){\rule[-0.200pt]{2.409pt}{0.400pt}}
\put(170.0,564.0){\rule[-0.200pt]{2.409pt}{0.400pt}}
\put(1429.0,564.0){\rule[-0.200pt]{2.409pt}{0.400pt}}
\put(170.0,565.0){\rule[-0.200pt]{2.409pt}{0.400pt}}
\put(1429.0,565.0){\rule[-0.200pt]{2.409pt}{0.400pt}}
\put(170.0,565.0){\rule[-0.200pt]{2.409pt}{0.400pt}}
\put(1429.0,565.0){\rule[-0.200pt]{2.409pt}{0.400pt}}
\put(170.0,565.0){\rule[-0.200pt]{2.409pt}{0.400pt}}
\put(1429.0,565.0){\rule[-0.200pt]{2.409pt}{0.400pt}}
\put(170.0,565.0){\rule[-0.200pt]{2.409pt}{0.400pt}}
\put(1429.0,565.0){\rule[-0.200pt]{2.409pt}{0.400pt}}
\put(170.0,565.0){\rule[-0.200pt]{2.409pt}{0.400pt}}
\put(1429.0,565.0){\rule[-0.200pt]{2.409pt}{0.400pt}}
\put(170.0,565.0){\rule[-0.200pt]{2.409pt}{0.400pt}}
\put(1429.0,565.0){\rule[-0.200pt]{2.409pt}{0.400pt}}
\put(170.0,565.0){\rule[-0.200pt]{2.409pt}{0.400pt}}
\put(1429.0,565.0){\rule[-0.200pt]{2.409pt}{0.400pt}}
\put(170.0,565.0){\rule[-0.200pt]{2.409pt}{0.400pt}}
\put(1429.0,565.0){\rule[-0.200pt]{2.409pt}{0.400pt}}
\put(170.0,565.0){\rule[-0.200pt]{2.409pt}{0.400pt}}
\put(1429.0,565.0){\rule[-0.200pt]{2.409pt}{0.400pt}}
\put(170.0,565.0){\rule[-0.200pt]{2.409pt}{0.400pt}}
\put(1429.0,565.0){\rule[-0.200pt]{2.409pt}{0.400pt}}
\put(170.0,565.0){\rule[-0.200pt]{2.409pt}{0.400pt}}
\put(1429.0,565.0){\rule[-0.200pt]{2.409pt}{0.400pt}}
\put(170.0,566.0){\rule[-0.200pt]{2.409pt}{0.400pt}}
\put(1429.0,566.0){\rule[-0.200pt]{2.409pt}{0.400pt}}
\put(170.0,566.0){\rule[-0.200pt]{2.409pt}{0.400pt}}
\put(1429.0,566.0){\rule[-0.200pt]{2.409pt}{0.400pt}}
\put(170.0,566.0){\rule[-0.200pt]{2.409pt}{0.400pt}}
\put(1429.0,566.0){\rule[-0.200pt]{2.409pt}{0.400pt}}
\put(170.0,566.0){\rule[-0.200pt]{2.409pt}{0.400pt}}
\put(1429.0,566.0){\rule[-0.200pt]{2.409pt}{0.400pt}}
\put(170.0,566.0){\rule[-0.200pt]{2.409pt}{0.400pt}}
\put(1429.0,566.0){\rule[-0.200pt]{2.409pt}{0.400pt}}
\put(170.0,566.0){\rule[-0.200pt]{2.409pt}{0.400pt}}
\put(1429.0,566.0){\rule[-0.200pt]{2.409pt}{0.400pt}}
\put(170.0,566.0){\rule[-0.200pt]{2.409pt}{0.400pt}}
\put(1429.0,566.0){\rule[-0.200pt]{2.409pt}{0.400pt}}
\put(170.0,566.0){\rule[-0.200pt]{2.409pt}{0.400pt}}
\put(1429.0,566.0){\rule[-0.200pt]{2.409pt}{0.400pt}}
\put(170.0,566.0){\rule[-0.200pt]{2.409pt}{0.400pt}}
\put(1429.0,566.0){\rule[-0.200pt]{2.409pt}{0.400pt}}
\put(170.0,566.0){\rule[-0.200pt]{2.409pt}{0.400pt}}
\put(1429.0,566.0){\rule[-0.200pt]{2.409pt}{0.400pt}}
\put(170.0,566.0){\rule[-0.200pt]{2.409pt}{0.400pt}}
\put(1429.0,566.0){\rule[-0.200pt]{2.409pt}{0.400pt}}
\put(170.0,567.0){\rule[-0.200pt]{2.409pt}{0.400pt}}
\put(1429.0,567.0){\rule[-0.200pt]{2.409pt}{0.400pt}}
\put(170.0,567.0){\rule[-0.200pt]{2.409pt}{0.400pt}}
\put(1429.0,567.0){\rule[-0.200pt]{2.409pt}{0.400pt}}
\put(170.0,567.0){\rule[-0.200pt]{2.409pt}{0.400pt}}
\put(1429.0,567.0){\rule[-0.200pt]{2.409pt}{0.400pt}}
\put(170.0,567.0){\rule[-0.200pt]{2.409pt}{0.400pt}}
\put(1429.0,567.0){\rule[-0.200pt]{2.409pt}{0.400pt}}
\put(170.0,567.0){\rule[-0.200pt]{2.409pt}{0.400pt}}
\put(1429.0,567.0){\rule[-0.200pt]{2.409pt}{0.400pt}}
\put(170.0,567.0){\rule[-0.200pt]{2.409pt}{0.400pt}}
\put(1429.0,567.0){\rule[-0.200pt]{2.409pt}{0.400pt}}
\put(170.0,567.0){\rule[-0.200pt]{2.409pt}{0.400pt}}
\put(1429.0,567.0){\rule[-0.200pt]{2.409pt}{0.400pt}}
\put(170.0,567.0){\rule[-0.200pt]{2.409pt}{0.400pt}}
\put(1429.0,567.0){\rule[-0.200pt]{2.409pt}{0.400pt}}
\put(170.0,567.0){\rule[-0.200pt]{2.409pt}{0.400pt}}
\put(1429.0,567.0){\rule[-0.200pt]{2.409pt}{0.400pt}}
\put(170.0,567.0){\rule[-0.200pt]{2.409pt}{0.400pt}}
\put(1429.0,567.0){\rule[-0.200pt]{2.409pt}{0.400pt}}
\put(170.0,567.0){\rule[-0.200pt]{2.409pt}{0.400pt}}
\put(1429.0,567.0){\rule[-0.200pt]{2.409pt}{0.400pt}}
\put(170.0,568.0){\rule[-0.200pt]{2.409pt}{0.400pt}}
\put(1429.0,568.0){\rule[-0.200pt]{2.409pt}{0.400pt}}
\put(170.0,568.0){\rule[-0.200pt]{2.409pt}{0.400pt}}
\put(1429.0,568.0){\rule[-0.200pt]{2.409pt}{0.400pt}}
\put(170.0,568.0){\rule[-0.200pt]{2.409pt}{0.400pt}}
\put(1429.0,568.0){\rule[-0.200pt]{2.409pt}{0.400pt}}
\put(170.0,568.0){\rule[-0.200pt]{2.409pt}{0.400pt}}
\put(1429.0,568.0){\rule[-0.200pt]{2.409pt}{0.400pt}}
\put(170.0,568.0){\rule[-0.200pt]{2.409pt}{0.400pt}}
\put(1429.0,568.0){\rule[-0.200pt]{2.409pt}{0.400pt}}
\put(170.0,568.0){\rule[-0.200pt]{2.409pt}{0.400pt}}
\put(1429.0,568.0){\rule[-0.200pt]{2.409pt}{0.400pt}}
\put(170.0,568.0){\rule[-0.200pt]{2.409pt}{0.400pt}}
\put(1429.0,568.0){\rule[-0.200pt]{2.409pt}{0.400pt}}
\put(170.0,568.0){\rule[-0.200pt]{2.409pt}{0.400pt}}
\put(1429.0,568.0){\rule[-0.200pt]{2.409pt}{0.400pt}}
\put(170.0,568.0){\rule[-0.200pt]{2.409pt}{0.400pt}}
\put(1429.0,568.0){\rule[-0.200pt]{2.409pt}{0.400pt}}
\put(170.0,568.0){\rule[-0.200pt]{2.409pt}{0.400pt}}
\put(1429.0,568.0){\rule[-0.200pt]{2.409pt}{0.400pt}}
\put(170.0,568.0){\rule[-0.200pt]{2.409pt}{0.400pt}}
\put(1429.0,568.0){\rule[-0.200pt]{2.409pt}{0.400pt}}
\put(170.0,569.0){\rule[-0.200pt]{2.409pt}{0.400pt}}
\put(1429.0,569.0){\rule[-0.200pt]{2.409pt}{0.400pt}}
\put(170.0,569.0){\rule[-0.200pt]{2.409pt}{0.400pt}}
\put(1429.0,569.0){\rule[-0.200pt]{2.409pt}{0.400pt}}
\put(170.0,569.0){\rule[-0.200pt]{2.409pt}{0.400pt}}
\put(1429.0,569.0){\rule[-0.200pt]{2.409pt}{0.400pt}}
\put(170.0,569.0){\rule[-0.200pt]{2.409pt}{0.400pt}}
\put(1429.0,569.0){\rule[-0.200pt]{2.409pt}{0.400pt}}
\put(170.0,569.0){\rule[-0.200pt]{2.409pt}{0.400pt}}
\put(1429.0,569.0){\rule[-0.200pt]{2.409pt}{0.400pt}}
\put(170.0,569.0){\rule[-0.200pt]{2.409pt}{0.400pt}}
\put(1429.0,569.0){\rule[-0.200pt]{2.409pt}{0.400pt}}
\put(170.0,569.0){\rule[-0.200pt]{2.409pt}{0.400pt}}
\put(1429.0,569.0){\rule[-0.200pt]{2.409pt}{0.400pt}}
\put(170.0,569.0){\rule[-0.200pt]{2.409pt}{0.400pt}}
\put(1429.0,569.0){\rule[-0.200pt]{2.409pt}{0.400pt}}
\put(170.0,569.0){\rule[-0.200pt]{2.409pt}{0.400pt}}
\put(1429.0,569.0){\rule[-0.200pt]{2.409pt}{0.400pt}}
\put(170.0,569.0){\rule[-0.200pt]{2.409pt}{0.400pt}}
\put(1429.0,569.0){\rule[-0.200pt]{2.409pt}{0.400pt}}
\put(170.0,569.0){\rule[-0.200pt]{2.409pt}{0.400pt}}
\put(1429.0,569.0){\rule[-0.200pt]{2.409pt}{0.400pt}}
\put(170.0,569.0){\rule[-0.200pt]{2.409pt}{0.400pt}}
\put(1429.0,569.0){\rule[-0.200pt]{2.409pt}{0.400pt}}
\put(170.0,570.0){\rule[-0.200pt]{2.409pt}{0.400pt}}
\put(1429.0,570.0){\rule[-0.200pt]{2.409pt}{0.400pt}}
\put(170.0,570.0){\rule[-0.200pt]{2.409pt}{0.400pt}}
\put(1429.0,570.0){\rule[-0.200pt]{2.409pt}{0.400pt}}
\put(170.0,570.0){\rule[-0.200pt]{2.409pt}{0.400pt}}
\put(1429.0,570.0){\rule[-0.200pt]{2.409pt}{0.400pt}}
\put(170.0,570.0){\rule[-0.200pt]{2.409pt}{0.400pt}}
\put(1429.0,570.0){\rule[-0.200pt]{2.409pt}{0.400pt}}
\put(170.0,570.0){\rule[-0.200pt]{2.409pt}{0.400pt}}
\put(1429.0,570.0){\rule[-0.200pt]{2.409pt}{0.400pt}}
\put(170.0,570.0){\rule[-0.200pt]{2.409pt}{0.400pt}}
\put(1429.0,570.0){\rule[-0.200pt]{2.409pt}{0.400pt}}
\put(170.0,570.0){\rule[-0.200pt]{2.409pt}{0.400pt}}
\put(1429.0,570.0){\rule[-0.200pt]{2.409pt}{0.400pt}}
\put(170.0,570.0){\rule[-0.200pt]{2.409pt}{0.400pt}}
\put(1429.0,570.0){\rule[-0.200pt]{2.409pt}{0.400pt}}
\put(170.0,570.0){\rule[-0.200pt]{2.409pt}{0.400pt}}
\put(1429.0,570.0){\rule[-0.200pt]{2.409pt}{0.400pt}}
\put(170.0,570.0){\rule[-0.200pt]{2.409pt}{0.400pt}}
\put(1429.0,570.0){\rule[-0.200pt]{2.409pt}{0.400pt}}
\put(170.0,570.0){\rule[-0.200pt]{2.409pt}{0.400pt}}
\put(1429.0,570.0){\rule[-0.200pt]{2.409pt}{0.400pt}}
\put(170.0,570.0){\rule[-0.200pt]{2.409pt}{0.400pt}}
\put(1429.0,570.0){\rule[-0.200pt]{2.409pt}{0.400pt}}
\put(170.0,571.0){\rule[-0.200pt]{2.409pt}{0.400pt}}
\put(1429.0,571.0){\rule[-0.200pt]{2.409pt}{0.400pt}}
\put(170.0,571.0){\rule[-0.200pt]{2.409pt}{0.400pt}}
\put(1429.0,571.0){\rule[-0.200pt]{2.409pt}{0.400pt}}
\put(170.0,571.0){\rule[-0.200pt]{2.409pt}{0.400pt}}
\put(1429.0,571.0){\rule[-0.200pt]{2.409pt}{0.400pt}}
\put(170.0,571.0){\rule[-0.200pt]{2.409pt}{0.400pt}}
\put(1429.0,571.0){\rule[-0.200pt]{2.409pt}{0.400pt}}
\put(170.0,571.0){\rule[-0.200pt]{2.409pt}{0.400pt}}
\put(1429.0,571.0){\rule[-0.200pt]{2.409pt}{0.400pt}}
\put(170.0,571.0){\rule[-0.200pt]{2.409pt}{0.400pt}}
\put(1429.0,571.0){\rule[-0.200pt]{2.409pt}{0.400pt}}
\put(170.0,571.0){\rule[-0.200pt]{2.409pt}{0.400pt}}
\put(1429.0,571.0){\rule[-0.200pt]{2.409pt}{0.400pt}}
\put(170.0,571.0){\rule[-0.200pt]{2.409pt}{0.400pt}}
\put(1429.0,571.0){\rule[-0.200pt]{2.409pt}{0.400pt}}
\put(170.0,571.0){\rule[-0.200pt]{2.409pt}{0.400pt}}
\put(1429.0,571.0){\rule[-0.200pt]{2.409pt}{0.400pt}}
\put(170.0,571.0){\rule[-0.200pt]{2.409pt}{0.400pt}}
\put(1429.0,571.0){\rule[-0.200pt]{2.409pt}{0.400pt}}
\put(170.0,571.0){\rule[-0.200pt]{2.409pt}{0.400pt}}
\put(1429.0,571.0){\rule[-0.200pt]{2.409pt}{0.400pt}}
\put(170.0,571.0){\rule[-0.200pt]{2.409pt}{0.400pt}}
\put(1429.0,571.0){\rule[-0.200pt]{2.409pt}{0.400pt}}
\put(170.0,572.0){\rule[-0.200pt]{2.409pt}{0.400pt}}
\put(1429.0,572.0){\rule[-0.200pt]{2.409pt}{0.400pt}}
\put(170.0,572.0){\rule[-0.200pt]{2.409pt}{0.400pt}}
\put(1429.0,572.0){\rule[-0.200pt]{2.409pt}{0.400pt}}
\put(170.0,572.0){\rule[-0.200pt]{2.409pt}{0.400pt}}
\put(1429.0,572.0){\rule[-0.200pt]{2.409pt}{0.400pt}}
\put(170.0,572.0){\rule[-0.200pt]{2.409pt}{0.400pt}}
\put(1429.0,572.0){\rule[-0.200pt]{2.409pt}{0.400pt}}
\put(170.0,572.0){\rule[-0.200pt]{2.409pt}{0.400pt}}
\put(1429.0,572.0){\rule[-0.200pt]{2.409pt}{0.400pt}}
\put(170.0,572.0){\rule[-0.200pt]{2.409pt}{0.400pt}}
\put(1429.0,572.0){\rule[-0.200pt]{2.409pt}{0.400pt}}
\put(170.0,572.0){\rule[-0.200pt]{2.409pt}{0.400pt}}
\put(1429.0,572.0){\rule[-0.200pt]{2.409pt}{0.400pt}}
\put(170.0,572.0){\rule[-0.200pt]{2.409pt}{0.400pt}}
\put(1429.0,572.0){\rule[-0.200pt]{2.409pt}{0.400pt}}
\put(170.0,572.0){\rule[-0.200pt]{2.409pt}{0.400pt}}
\put(1429.0,572.0){\rule[-0.200pt]{2.409pt}{0.400pt}}
\put(170.0,572.0){\rule[-0.200pt]{2.409pt}{0.400pt}}
\put(1429.0,572.0){\rule[-0.200pt]{2.409pt}{0.400pt}}
\put(170.0,572.0){\rule[-0.200pt]{2.409pt}{0.400pt}}
\put(1429.0,572.0){\rule[-0.200pt]{2.409pt}{0.400pt}}
\put(170.0,572.0){\rule[-0.200pt]{2.409pt}{0.400pt}}
\put(1429.0,572.0){\rule[-0.200pt]{2.409pt}{0.400pt}}
\put(170.0,572.0){\rule[-0.200pt]{2.409pt}{0.400pt}}
\put(1429.0,572.0){\rule[-0.200pt]{2.409pt}{0.400pt}}
\put(170.0,573.0){\rule[-0.200pt]{2.409pt}{0.400pt}}
\put(1429.0,573.0){\rule[-0.200pt]{2.409pt}{0.400pt}}
\put(170.0,573.0){\rule[-0.200pt]{2.409pt}{0.400pt}}
\put(1429.0,573.0){\rule[-0.200pt]{2.409pt}{0.400pt}}
\put(170.0,573.0){\rule[-0.200pt]{2.409pt}{0.400pt}}
\put(1429.0,573.0){\rule[-0.200pt]{2.409pt}{0.400pt}}
\put(170.0,573.0){\rule[-0.200pt]{2.409pt}{0.400pt}}
\put(1429.0,573.0){\rule[-0.200pt]{2.409pt}{0.400pt}}
\put(170.0,573.0){\rule[-0.200pt]{2.409pt}{0.400pt}}
\put(1429.0,573.0){\rule[-0.200pt]{2.409pt}{0.400pt}}
\put(170.0,573.0){\rule[-0.200pt]{2.409pt}{0.400pt}}
\put(1429.0,573.0){\rule[-0.200pt]{2.409pt}{0.400pt}}
\put(170.0,573.0){\rule[-0.200pt]{2.409pt}{0.400pt}}
\put(1429.0,573.0){\rule[-0.200pt]{2.409pt}{0.400pt}}
\put(170.0,573.0){\rule[-0.200pt]{2.409pt}{0.400pt}}
\put(1429.0,573.0){\rule[-0.200pt]{2.409pt}{0.400pt}}
\put(170.0,573.0){\rule[-0.200pt]{2.409pt}{0.400pt}}
\put(1429.0,573.0){\rule[-0.200pt]{2.409pt}{0.400pt}}
\put(170.0,573.0){\rule[-0.200pt]{2.409pt}{0.400pt}}
\put(1429.0,573.0){\rule[-0.200pt]{2.409pt}{0.400pt}}
\put(170.0,573.0){\rule[-0.200pt]{2.409pt}{0.400pt}}
\put(1429.0,573.0){\rule[-0.200pt]{2.409pt}{0.400pt}}
\put(170.0,573.0){\rule[-0.200pt]{2.409pt}{0.400pt}}
\put(1429.0,573.0){\rule[-0.200pt]{2.409pt}{0.400pt}}
\put(170.0,573.0){\rule[-0.200pt]{2.409pt}{0.400pt}}
\put(1429.0,573.0){\rule[-0.200pt]{2.409pt}{0.400pt}}
\put(170.0,574.0){\rule[-0.200pt]{2.409pt}{0.400pt}}
\put(1429.0,574.0){\rule[-0.200pt]{2.409pt}{0.400pt}}
\put(170.0,574.0){\rule[-0.200pt]{2.409pt}{0.400pt}}
\put(1429.0,574.0){\rule[-0.200pt]{2.409pt}{0.400pt}}
\put(170.0,574.0){\rule[-0.200pt]{2.409pt}{0.400pt}}
\put(1429.0,574.0){\rule[-0.200pt]{2.409pt}{0.400pt}}
\put(170.0,574.0){\rule[-0.200pt]{2.409pt}{0.400pt}}
\put(1429.0,574.0){\rule[-0.200pt]{2.409pt}{0.400pt}}
\put(170.0,574.0){\rule[-0.200pt]{2.409pt}{0.400pt}}
\put(1429.0,574.0){\rule[-0.200pt]{2.409pt}{0.400pt}}
\put(170.0,574.0){\rule[-0.200pt]{2.409pt}{0.400pt}}
\put(1429.0,574.0){\rule[-0.200pt]{2.409pt}{0.400pt}}
\put(170.0,574.0){\rule[-0.200pt]{2.409pt}{0.400pt}}
\put(1429.0,574.0){\rule[-0.200pt]{2.409pt}{0.400pt}}
\put(170.0,574.0){\rule[-0.200pt]{2.409pt}{0.400pt}}
\put(1429.0,574.0){\rule[-0.200pt]{2.409pt}{0.400pt}}
\put(170.0,574.0){\rule[-0.200pt]{2.409pt}{0.400pt}}
\put(1429.0,574.0){\rule[-0.200pt]{2.409pt}{0.400pt}}
\put(170.0,574.0){\rule[-0.200pt]{2.409pt}{0.400pt}}
\put(1429.0,574.0){\rule[-0.200pt]{2.409pt}{0.400pt}}
\put(170.0,574.0){\rule[-0.200pt]{2.409pt}{0.400pt}}
\put(1429.0,574.0){\rule[-0.200pt]{2.409pt}{0.400pt}}
\put(170.0,574.0){\rule[-0.200pt]{2.409pt}{0.400pt}}
\put(1429.0,574.0){\rule[-0.200pt]{2.409pt}{0.400pt}}
\put(170.0,574.0){\rule[-0.200pt]{2.409pt}{0.400pt}}
\put(1429.0,574.0){\rule[-0.200pt]{2.409pt}{0.400pt}}
\put(170.0,575.0){\rule[-0.200pt]{2.409pt}{0.400pt}}
\put(1429.0,575.0){\rule[-0.200pt]{2.409pt}{0.400pt}}
\put(170.0,575.0){\rule[-0.200pt]{2.409pt}{0.400pt}}
\put(1429.0,575.0){\rule[-0.200pt]{2.409pt}{0.400pt}}
\put(170.0,575.0){\rule[-0.200pt]{2.409pt}{0.400pt}}
\put(1429.0,575.0){\rule[-0.200pt]{2.409pt}{0.400pt}}
\put(170.0,575.0){\rule[-0.200pt]{2.409pt}{0.400pt}}
\put(1429.0,575.0){\rule[-0.200pt]{2.409pt}{0.400pt}}
\put(170.0,575.0){\rule[-0.200pt]{2.409pt}{0.400pt}}
\put(1429.0,575.0){\rule[-0.200pt]{2.409pt}{0.400pt}}
\put(170.0,575.0){\rule[-0.200pt]{2.409pt}{0.400pt}}
\put(1429.0,575.0){\rule[-0.200pt]{2.409pt}{0.400pt}}
\put(170.0,575.0){\rule[-0.200pt]{2.409pt}{0.400pt}}
\put(1429.0,575.0){\rule[-0.200pt]{2.409pt}{0.400pt}}
\put(170.0,575.0){\rule[-0.200pt]{2.409pt}{0.400pt}}
\put(1429.0,575.0){\rule[-0.200pt]{2.409pt}{0.400pt}}
\put(170.0,575.0){\rule[-0.200pt]{2.409pt}{0.400pt}}
\put(1429.0,575.0){\rule[-0.200pt]{2.409pt}{0.400pt}}
\put(170.0,575.0){\rule[-0.200pt]{2.409pt}{0.400pt}}
\put(1429.0,575.0){\rule[-0.200pt]{2.409pt}{0.400pt}}
\put(170.0,575.0){\rule[-0.200pt]{2.409pt}{0.400pt}}
\put(1429.0,575.0){\rule[-0.200pt]{2.409pt}{0.400pt}}
\put(170.0,575.0){\rule[-0.200pt]{2.409pt}{0.400pt}}
\put(1429.0,575.0){\rule[-0.200pt]{2.409pt}{0.400pt}}
\put(170.0,575.0){\rule[-0.200pt]{2.409pt}{0.400pt}}
\put(1429.0,575.0){\rule[-0.200pt]{2.409pt}{0.400pt}}
\put(170.0,575.0){\rule[-0.200pt]{2.409pt}{0.400pt}}
\put(1429.0,575.0){\rule[-0.200pt]{2.409pt}{0.400pt}}
\put(170.0,576.0){\rule[-0.200pt]{2.409pt}{0.400pt}}
\put(1429.0,576.0){\rule[-0.200pt]{2.409pt}{0.400pt}}
\put(170.0,576.0){\rule[-0.200pt]{2.409pt}{0.400pt}}
\put(1429.0,576.0){\rule[-0.200pt]{2.409pt}{0.400pt}}
\put(170.0,576.0){\rule[-0.200pt]{2.409pt}{0.400pt}}
\put(1429.0,576.0){\rule[-0.200pt]{2.409pt}{0.400pt}}
\put(170.0,576.0){\rule[-0.200pt]{2.409pt}{0.400pt}}
\put(1429.0,576.0){\rule[-0.200pt]{2.409pt}{0.400pt}}
\put(170.0,576.0){\rule[-0.200pt]{2.409pt}{0.400pt}}
\put(1429.0,576.0){\rule[-0.200pt]{2.409pt}{0.400pt}}
\put(170.0,576.0){\rule[-0.200pt]{2.409pt}{0.400pt}}
\put(1429.0,576.0){\rule[-0.200pt]{2.409pt}{0.400pt}}
\put(170.0,576.0){\rule[-0.200pt]{2.409pt}{0.400pt}}
\put(1429.0,576.0){\rule[-0.200pt]{2.409pt}{0.400pt}}
\put(170.0,576.0){\rule[-0.200pt]{2.409pt}{0.400pt}}
\put(1429.0,576.0){\rule[-0.200pt]{2.409pt}{0.400pt}}
\put(170.0,576.0){\rule[-0.200pt]{2.409pt}{0.400pt}}
\put(1429.0,576.0){\rule[-0.200pt]{2.409pt}{0.400pt}}
\put(170.0,576.0){\rule[-0.200pt]{2.409pt}{0.400pt}}
\put(1429.0,576.0){\rule[-0.200pt]{2.409pt}{0.400pt}}
\put(170.0,576.0){\rule[-0.200pt]{2.409pt}{0.400pt}}
\put(1429.0,576.0){\rule[-0.200pt]{2.409pt}{0.400pt}}
\put(170.0,576.0){\rule[-0.200pt]{2.409pt}{0.400pt}}
\put(1429.0,576.0){\rule[-0.200pt]{2.409pt}{0.400pt}}
\put(170.0,576.0){\rule[-0.200pt]{2.409pt}{0.400pt}}
\put(1429.0,576.0){\rule[-0.200pt]{2.409pt}{0.400pt}}
\put(170.0,576.0){\rule[-0.200pt]{2.409pt}{0.400pt}}
\put(1429.0,576.0){\rule[-0.200pt]{2.409pt}{0.400pt}}
\put(170.0,577.0){\rule[-0.200pt]{2.409pt}{0.400pt}}
\put(1429.0,577.0){\rule[-0.200pt]{2.409pt}{0.400pt}}
\put(170.0,577.0){\rule[-0.200pt]{2.409pt}{0.400pt}}
\put(1429.0,577.0){\rule[-0.200pt]{2.409pt}{0.400pt}}
\put(170.0,577.0){\rule[-0.200pt]{2.409pt}{0.400pt}}
\put(1429.0,577.0){\rule[-0.200pt]{2.409pt}{0.400pt}}
\put(170.0,577.0){\rule[-0.200pt]{2.409pt}{0.400pt}}
\put(1429.0,577.0){\rule[-0.200pt]{2.409pt}{0.400pt}}
\put(170.0,577.0){\rule[-0.200pt]{2.409pt}{0.400pt}}
\put(1429.0,577.0){\rule[-0.200pt]{2.409pt}{0.400pt}}
\put(170.0,577.0){\rule[-0.200pt]{2.409pt}{0.400pt}}
\put(1429.0,577.0){\rule[-0.200pt]{2.409pt}{0.400pt}}
\put(170.0,577.0){\rule[-0.200pt]{2.409pt}{0.400pt}}
\put(1429.0,577.0){\rule[-0.200pt]{2.409pt}{0.400pt}}
\put(170.0,577.0){\rule[-0.200pt]{2.409pt}{0.400pt}}
\put(1429.0,577.0){\rule[-0.200pt]{2.409pt}{0.400pt}}
\put(170.0,577.0){\rule[-0.200pt]{2.409pt}{0.400pt}}
\put(1429.0,577.0){\rule[-0.200pt]{2.409pt}{0.400pt}}
\put(170.0,577.0){\rule[-0.200pt]{2.409pt}{0.400pt}}
\put(1429.0,577.0){\rule[-0.200pt]{2.409pt}{0.400pt}}
\put(170.0,577.0){\rule[-0.200pt]{2.409pt}{0.400pt}}
\put(1429.0,577.0){\rule[-0.200pt]{2.409pt}{0.400pt}}
\put(170.0,577.0){\rule[-0.200pt]{2.409pt}{0.400pt}}
\put(1429.0,577.0){\rule[-0.200pt]{2.409pt}{0.400pt}}
\put(170.0,577.0){\rule[-0.200pt]{2.409pt}{0.400pt}}
\put(1429.0,577.0){\rule[-0.200pt]{2.409pt}{0.400pt}}
\put(170.0,577.0){\rule[-0.200pt]{2.409pt}{0.400pt}}
\put(1429.0,577.0){\rule[-0.200pt]{2.409pt}{0.400pt}}
\put(170.0,578.0){\rule[-0.200pt]{2.409pt}{0.400pt}}
\put(1429.0,578.0){\rule[-0.200pt]{2.409pt}{0.400pt}}
\put(170.0,578.0){\rule[-0.200pt]{2.409pt}{0.400pt}}
\put(1429.0,578.0){\rule[-0.200pt]{2.409pt}{0.400pt}}
\put(170.0,578.0){\rule[-0.200pt]{2.409pt}{0.400pt}}
\put(1429.0,578.0){\rule[-0.200pt]{2.409pt}{0.400pt}}
\put(170.0,578.0){\rule[-0.200pt]{2.409pt}{0.400pt}}
\put(1429.0,578.0){\rule[-0.200pt]{2.409pt}{0.400pt}}
\put(170.0,578.0){\rule[-0.200pt]{2.409pt}{0.400pt}}
\put(1429.0,578.0){\rule[-0.200pt]{2.409pt}{0.400pt}}
\put(170.0,578.0){\rule[-0.200pt]{2.409pt}{0.400pt}}
\put(1429.0,578.0){\rule[-0.200pt]{2.409pt}{0.400pt}}
\put(170.0,578.0){\rule[-0.200pt]{2.409pt}{0.400pt}}
\put(1429.0,578.0){\rule[-0.200pt]{2.409pt}{0.400pt}}
\put(170.0,578.0){\rule[-0.200pt]{2.409pt}{0.400pt}}
\put(1429.0,578.0){\rule[-0.200pt]{2.409pt}{0.400pt}}
\put(170.0,578.0){\rule[-0.200pt]{2.409pt}{0.400pt}}
\put(1429.0,578.0){\rule[-0.200pt]{2.409pt}{0.400pt}}
\put(170.0,578.0){\rule[-0.200pt]{2.409pt}{0.400pt}}
\put(1429.0,578.0){\rule[-0.200pt]{2.409pt}{0.400pt}}
\put(170.0,578.0){\rule[-0.200pt]{2.409pt}{0.400pt}}
\put(1429.0,578.0){\rule[-0.200pt]{2.409pt}{0.400pt}}
\put(170.0,578.0){\rule[-0.200pt]{2.409pt}{0.400pt}}
\put(1429.0,578.0){\rule[-0.200pt]{2.409pt}{0.400pt}}
\put(170.0,578.0){\rule[-0.200pt]{2.409pt}{0.400pt}}
\put(1429.0,578.0){\rule[-0.200pt]{2.409pt}{0.400pt}}
\put(170.0,578.0){\rule[-0.200pt]{2.409pt}{0.400pt}}
\put(1429.0,578.0){\rule[-0.200pt]{2.409pt}{0.400pt}}
\put(170.0,578.0){\rule[-0.200pt]{2.409pt}{0.400pt}}
\put(1429.0,578.0){\rule[-0.200pt]{2.409pt}{0.400pt}}
\put(170.0,579.0){\rule[-0.200pt]{2.409pt}{0.400pt}}
\put(1429.0,579.0){\rule[-0.200pt]{2.409pt}{0.400pt}}
\put(170.0,579.0){\rule[-0.200pt]{2.409pt}{0.400pt}}
\put(1429.0,579.0){\rule[-0.200pt]{2.409pt}{0.400pt}}
\put(170.0,579.0){\rule[-0.200pt]{2.409pt}{0.400pt}}
\put(1429.0,579.0){\rule[-0.200pt]{2.409pt}{0.400pt}}
\put(170.0,579.0){\rule[-0.200pt]{2.409pt}{0.400pt}}
\put(1429.0,579.0){\rule[-0.200pt]{2.409pt}{0.400pt}}
\put(170.0,579.0){\rule[-0.200pt]{2.409pt}{0.400pt}}
\put(1429.0,579.0){\rule[-0.200pt]{2.409pt}{0.400pt}}
\put(170.0,579.0){\rule[-0.200pt]{2.409pt}{0.400pt}}
\put(1429.0,579.0){\rule[-0.200pt]{2.409pt}{0.400pt}}
\put(170.0,579.0){\rule[-0.200pt]{2.409pt}{0.400pt}}
\put(1429.0,579.0){\rule[-0.200pt]{2.409pt}{0.400pt}}
\put(170.0,579.0){\rule[-0.200pt]{2.409pt}{0.400pt}}
\put(1429.0,579.0){\rule[-0.200pt]{2.409pt}{0.400pt}}
\put(170.0,579.0){\rule[-0.200pt]{2.409pt}{0.400pt}}
\put(1429.0,579.0){\rule[-0.200pt]{2.409pt}{0.400pt}}
\put(170.0,579.0){\rule[-0.200pt]{2.409pt}{0.400pt}}
\put(1429.0,579.0){\rule[-0.200pt]{2.409pt}{0.400pt}}
\put(170.0,579.0){\rule[-0.200pt]{2.409pt}{0.400pt}}
\put(1429.0,579.0){\rule[-0.200pt]{2.409pt}{0.400pt}}
\put(170.0,579.0){\rule[-0.200pt]{2.409pt}{0.400pt}}
\put(1429.0,579.0){\rule[-0.200pt]{2.409pt}{0.400pt}}
\put(170.0,579.0){\rule[-0.200pt]{2.409pt}{0.400pt}}
\put(1429.0,579.0){\rule[-0.200pt]{2.409pt}{0.400pt}}
\put(170.0,579.0){\rule[-0.200pt]{2.409pt}{0.400pt}}
\put(1429.0,579.0){\rule[-0.200pt]{2.409pt}{0.400pt}}
\put(170.0,579.0){\rule[-0.200pt]{2.409pt}{0.400pt}}
\put(1429.0,579.0){\rule[-0.200pt]{2.409pt}{0.400pt}}
\put(170.0,580.0){\rule[-0.200pt]{2.409pt}{0.400pt}}
\put(1429.0,580.0){\rule[-0.200pt]{2.409pt}{0.400pt}}
\put(170.0,580.0){\rule[-0.200pt]{2.409pt}{0.400pt}}
\put(1429.0,580.0){\rule[-0.200pt]{2.409pt}{0.400pt}}
\put(170.0,580.0){\rule[-0.200pt]{2.409pt}{0.400pt}}
\put(1429.0,580.0){\rule[-0.200pt]{2.409pt}{0.400pt}}
\put(170.0,580.0){\rule[-0.200pt]{2.409pt}{0.400pt}}
\put(1429.0,580.0){\rule[-0.200pt]{2.409pt}{0.400pt}}
\put(170.0,580.0){\rule[-0.200pt]{2.409pt}{0.400pt}}
\put(1429.0,580.0){\rule[-0.200pt]{2.409pt}{0.400pt}}
\put(170.0,580.0){\rule[-0.200pt]{2.409pt}{0.400pt}}
\put(1429.0,580.0){\rule[-0.200pt]{2.409pt}{0.400pt}}
\put(170.0,580.0){\rule[-0.200pt]{2.409pt}{0.400pt}}
\put(1429.0,580.0){\rule[-0.200pt]{2.409pt}{0.400pt}}
\put(170.0,580.0){\rule[-0.200pt]{2.409pt}{0.400pt}}
\put(1429.0,580.0){\rule[-0.200pt]{2.409pt}{0.400pt}}
\put(170.0,580.0){\rule[-0.200pt]{2.409pt}{0.400pt}}
\put(1429.0,580.0){\rule[-0.200pt]{2.409pt}{0.400pt}}
\put(170.0,580.0){\rule[-0.200pt]{2.409pt}{0.400pt}}
\put(1429.0,580.0){\rule[-0.200pt]{2.409pt}{0.400pt}}
\put(170.0,580.0){\rule[-0.200pt]{2.409pt}{0.400pt}}
\put(1429.0,580.0){\rule[-0.200pt]{2.409pt}{0.400pt}}
\put(170.0,580.0){\rule[-0.200pt]{2.409pt}{0.400pt}}
\put(1429.0,580.0){\rule[-0.200pt]{2.409pt}{0.400pt}}
\put(170.0,580.0){\rule[-0.200pt]{2.409pt}{0.400pt}}
\put(1429.0,580.0){\rule[-0.200pt]{2.409pt}{0.400pt}}
\put(170.0,580.0){\rule[-0.200pt]{2.409pt}{0.400pt}}
\put(1429.0,580.0){\rule[-0.200pt]{2.409pt}{0.400pt}}
\put(170.0,580.0){\rule[-0.200pt]{2.409pt}{0.400pt}}
\put(1429.0,580.0){\rule[-0.200pt]{2.409pt}{0.400pt}}
\put(170.0,580.0){\rule[-0.200pt]{2.409pt}{0.400pt}}
\put(1429.0,580.0){\rule[-0.200pt]{2.409pt}{0.400pt}}
\put(170.0,581.0){\rule[-0.200pt]{2.409pt}{0.400pt}}
\put(1429.0,581.0){\rule[-0.200pt]{2.409pt}{0.400pt}}
\put(170.0,581.0){\rule[-0.200pt]{2.409pt}{0.400pt}}
\put(1429.0,581.0){\rule[-0.200pt]{2.409pt}{0.400pt}}
\put(170.0,581.0){\rule[-0.200pt]{2.409pt}{0.400pt}}
\put(1429.0,581.0){\rule[-0.200pt]{2.409pt}{0.400pt}}
\put(170.0,581.0){\rule[-0.200pt]{2.409pt}{0.400pt}}
\put(1429.0,581.0){\rule[-0.200pt]{2.409pt}{0.400pt}}
\put(170.0,581.0){\rule[-0.200pt]{2.409pt}{0.400pt}}
\put(1429.0,581.0){\rule[-0.200pt]{2.409pt}{0.400pt}}
\put(170.0,581.0){\rule[-0.200pt]{2.409pt}{0.400pt}}
\put(1429.0,581.0){\rule[-0.200pt]{2.409pt}{0.400pt}}
\put(170.0,581.0){\rule[-0.200pt]{2.409pt}{0.400pt}}
\put(1429.0,581.0){\rule[-0.200pt]{2.409pt}{0.400pt}}
\put(170.0,581.0){\rule[-0.200pt]{2.409pt}{0.400pt}}
\put(1429.0,581.0){\rule[-0.200pt]{2.409pt}{0.400pt}}
\put(170.0,581.0){\rule[-0.200pt]{2.409pt}{0.400pt}}
\put(1429.0,581.0){\rule[-0.200pt]{2.409pt}{0.400pt}}
\put(170.0,581.0){\rule[-0.200pt]{2.409pt}{0.400pt}}
\put(1429.0,581.0){\rule[-0.200pt]{2.409pt}{0.400pt}}
\put(170.0,581.0){\rule[-0.200pt]{2.409pt}{0.400pt}}
\put(1429.0,581.0){\rule[-0.200pt]{2.409pt}{0.400pt}}
\put(170.0,581.0){\rule[-0.200pt]{2.409pt}{0.400pt}}
\put(1429.0,581.0){\rule[-0.200pt]{2.409pt}{0.400pt}}
\put(170.0,581.0){\rule[-0.200pt]{2.409pt}{0.400pt}}
\put(1429.0,581.0){\rule[-0.200pt]{2.409pt}{0.400pt}}
\put(170.0,581.0){\rule[-0.200pt]{2.409pt}{0.400pt}}
\put(1429.0,581.0){\rule[-0.200pt]{2.409pt}{0.400pt}}
\put(170.0,581.0){\rule[-0.200pt]{2.409pt}{0.400pt}}
\put(1429.0,581.0){\rule[-0.200pt]{2.409pt}{0.400pt}}
\put(170.0,581.0){\rule[-0.200pt]{2.409pt}{0.400pt}}
\put(1429.0,581.0){\rule[-0.200pt]{2.409pt}{0.400pt}}
\put(170.0,582.0){\rule[-0.200pt]{2.409pt}{0.400pt}}
\put(1429.0,582.0){\rule[-0.200pt]{2.409pt}{0.400pt}}
\put(170.0,582.0){\rule[-0.200pt]{2.409pt}{0.400pt}}
\put(1429.0,582.0){\rule[-0.200pt]{2.409pt}{0.400pt}}
\put(170.0,582.0){\rule[-0.200pt]{2.409pt}{0.400pt}}
\put(1429.0,582.0){\rule[-0.200pt]{2.409pt}{0.400pt}}
\put(170.0,582.0){\rule[-0.200pt]{2.409pt}{0.400pt}}
\put(1429.0,582.0){\rule[-0.200pt]{2.409pt}{0.400pt}}
\put(170.0,582.0){\rule[-0.200pt]{2.409pt}{0.400pt}}
\put(1429.0,582.0){\rule[-0.200pt]{2.409pt}{0.400pt}}
\put(170.0,582.0){\rule[-0.200pt]{2.409pt}{0.400pt}}
\put(1429.0,582.0){\rule[-0.200pt]{2.409pt}{0.400pt}}
\put(170.0,582.0){\rule[-0.200pt]{2.409pt}{0.400pt}}
\put(1429.0,582.0){\rule[-0.200pt]{2.409pt}{0.400pt}}
\put(170.0,582.0){\rule[-0.200pt]{2.409pt}{0.400pt}}
\put(1429.0,582.0){\rule[-0.200pt]{2.409pt}{0.400pt}}
\put(170.0,582.0){\rule[-0.200pt]{2.409pt}{0.400pt}}
\put(1429.0,582.0){\rule[-0.200pt]{2.409pt}{0.400pt}}
\put(170.0,582.0){\rule[-0.200pt]{2.409pt}{0.400pt}}
\put(1429.0,582.0){\rule[-0.200pt]{2.409pt}{0.400pt}}
\put(170.0,582.0){\rule[-0.200pt]{2.409pt}{0.400pt}}
\put(1429.0,582.0){\rule[-0.200pt]{2.409pt}{0.400pt}}
\put(170.0,582.0){\rule[-0.200pt]{2.409pt}{0.400pt}}
\put(1429.0,582.0){\rule[-0.200pt]{2.409pt}{0.400pt}}
\put(170.0,582.0){\rule[-0.200pt]{2.409pt}{0.400pt}}
\put(1429.0,582.0){\rule[-0.200pt]{2.409pt}{0.400pt}}
\put(170.0,582.0){\rule[-0.200pt]{2.409pt}{0.400pt}}
\put(1429.0,582.0){\rule[-0.200pt]{2.409pt}{0.400pt}}
\put(170.0,582.0){\rule[-0.200pt]{2.409pt}{0.400pt}}
\put(1429.0,582.0){\rule[-0.200pt]{2.409pt}{0.400pt}}
\put(170.0,582.0){\rule[-0.200pt]{2.409pt}{0.400pt}}
\put(1429.0,582.0){\rule[-0.200pt]{2.409pt}{0.400pt}}
\put(170.0,582.0){\rule[-0.200pt]{2.409pt}{0.400pt}}
\put(1429.0,582.0){\rule[-0.200pt]{2.409pt}{0.400pt}}
\put(170.0,583.0){\rule[-0.200pt]{2.409pt}{0.400pt}}
\put(1429.0,583.0){\rule[-0.200pt]{2.409pt}{0.400pt}}
\put(170.0,583.0){\rule[-0.200pt]{2.409pt}{0.400pt}}
\put(1429.0,583.0){\rule[-0.200pt]{2.409pt}{0.400pt}}
\put(170.0,583.0){\rule[-0.200pt]{2.409pt}{0.400pt}}
\put(1429.0,583.0){\rule[-0.200pt]{2.409pt}{0.400pt}}
\put(170.0,583.0){\rule[-0.200pt]{2.409pt}{0.400pt}}
\put(1429.0,583.0){\rule[-0.200pt]{2.409pt}{0.400pt}}
\put(170.0,583.0){\rule[-0.200pt]{2.409pt}{0.400pt}}
\put(1429.0,583.0){\rule[-0.200pt]{2.409pt}{0.400pt}}
\put(170.0,583.0){\rule[-0.200pt]{2.409pt}{0.400pt}}
\put(1429.0,583.0){\rule[-0.200pt]{2.409pt}{0.400pt}}
\put(170.0,583.0){\rule[-0.200pt]{2.409pt}{0.400pt}}
\put(1429.0,583.0){\rule[-0.200pt]{2.409pt}{0.400pt}}
\put(170.0,583.0){\rule[-0.200pt]{2.409pt}{0.400pt}}
\put(1429.0,583.0){\rule[-0.200pt]{2.409pt}{0.400pt}}
\put(170.0,583.0){\rule[-0.200pt]{2.409pt}{0.400pt}}
\put(1429.0,583.0){\rule[-0.200pt]{2.409pt}{0.400pt}}
\put(170.0,583.0){\rule[-0.200pt]{2.409pt}{0.400pt}}
\put(1429.0,583.0){\rule[-0.200pt]{2.409pt}{0.400pt}}
\put(170.0,583.0){\rule[-0.200pt]{2.409pt}{0.400pt}}
\put(1429.0,583.0){\rule[-0.200pt]{2.409pt}{0.400pt}}
\put(170.0,583.0){\rule[-0.200pt]{2.409pt}{0.400pt}}
\put(1429.0,583.0){\rule[-0.200pt]{2.409pt}{0.400pt}}
\put(170.0,583.0){\rule[-0.200pt]{2.409pt}{0.400pt}}
\put(1429.0,583.0){\rule[-0.200pt]{2.409pt}{0.400pt}}
\put(170.0,583.0){\rule[-0.200pt]{2.409pt}{0.400pt}}
\put(1429.0,583.0){\rule[-0.200pt]{2.409pt}{0.400pt}}
\put(170.0,583.0){\rule[-0.200pt]{2.409pt}{0.400pt}}
\put(1429.0,583.0){\rule[-0.200pt]{2.409pt}{0.400pt}}
\put(170.0,583.0){\rule[-0.200pt]{2.409pt}{0.400pt}}
\put(1429.0,583.0){\rule[-0.200pt]{2.409pt}{0.400pt}}
\put(170.0,584.0){\rule[-0.200pt]{2.409pt}{0.400pt}}
\put(1429.0,584.0){\rule[-0.200pt]{2.409pt}{0.400pt}}
\put(170.0,584.0){\rule[-0.200pt]{2.409pt}{0.400pt}}
\put(1429.0,584.0){\rule[-0.200pt]{2.409pt}{0.400pt}}
\put(170.0,584.0){\rule[-0.200pt]{2.409pt}{0.400pt}}
\put(1429.0,584.0){\rule[-0.200pt]{2.409pt}{0.400pt}}
\put(170.0,584.0){\rule[-0.200pt]{2.409pt}{0.400pt}}
\put(1429.0,584.0){\rule[-0.200pt]{2.409pt}{0.400pt}}
\put(170.0,584.0){\rule[-0.200pt]{2.409pt}{0.400pt}}
\put(1429.0,584.0){\rule[-0.200pt]{2.409pt}{0.400pt}}
\put(170.0,584.0){\rule[-0.200pt]{2.409pt}{0.400pt}}
\put(1429.0,584.0){\rule[-0.200pt]{2.409pt}{0.400pt}}
\put(170.0,584.0){\rule[-0.200pt]{2.409pt}{0.400pt}}
\put(1429.0,584.0){\rule[-0.200pt]{2.409pt}{0.400pt}}
\put(170.0,584.0){\rule[-0.200pt]{2.409pt}{0.400pt}}
\put(1429.0,584.0){\rule[-0.200pt]{2.409pt}{0.400pt}}
\put(170.0,584.0){\rule[-0.200pt]{2.409pt}{0.400pt}}
\put(1429.0,584.0){\rule[-0.200pt]{2.409pt}{0.400pt}}
\put(170.0,584.0){\rule[-0.200pt]{2.409pt}{0.400pt}}
\put(1429.0,584.0){\rule[-0.200pt]{2.409pt}{0.400pt}}
\put(170.0,584.0){\rule[-0.200pt]{2.409pt}{0.400pt}}
\put(1429.0,584.0){\rule[-0.200pt]{2.409pt}{0.400pt}}
\put(170.0,584.0){\rule[-0.200pt]{2.409pt}{0.400pt}}
\put(1429.0,584.0){\rule[-0.200pt]{2.409pt}{0.400pt}}
\put(170.0,584.0){\rule[-0.200pt]{2.409pt}{0.400pt}}
\put(1429.0,584.0){\rule[-0.200pt]{2.409pt}{0.400pt}}
\put(170.0,584.0){\rule[-0.200pt]{2.409pt}{0.400pt}}
\put(1429.0,584.0){\rule[-0.200pt]{2.409pt}{0.400pt}}
\put(170.0,584.0){\rule[-0.200pt]{2.409pt}{0.400pt}}
\put(1429.0,584.0){\rule[-0.200pt]{2.409pt}{0.400pt}}
\put(170.0,584.0){\rule[-0.200pt]{2.409pt}{0.400pt}}
\put(1429.0,584.0){\rule[-0.200pt]{2.409pt}{0.400pt}}
\put(170.0,584.0){\rule[-0.200pt]{2.409pt}{0.400pt}}
\put(1429.0,584.0){\rule[-0.200pt]{2.409pt}{0.400pt}}
\put(170.0,584.0){\rule[-0.200pt]{2.409pt}{0.400pt}}
\put(1429.0,584.0){\rule[-0.200pt]{2.409pt}{0.400pt}}
\put(170.0,585.0){\rule[-0.200pt]{2.409pt}{0.400pt}}
\put(1429.0,585.0){\rule[-0.200pt]{2.409pt}{0.400pt}}
\put(170.0,585.0){\rule[-0.200pt]{2.409pt}{0.400pt}}
\put(1429.0,585.0){\rule[-0.200pt]{2.409pt}{0.400pt}}
\put(170.0,585.0){\rule[-0.200pt]{2.409pt}{0.400pt}}
\put(1429.0,585.0){\rule[-0.200pt]{2.409pt}{0.400pt}}
\put(170.0,585.0){\rule[-0.200pt]{2.409pt}{0.400pt}}
\put(1429.0,585.0){\rule[-0.200pt]{2.409pt}{0.400pt}}
\put(170.0,585.0){\rule[-0.200pt]{2.409pt}{0.400pt}}
\put(1429.0,585.0){\rule[-0.200pt]{2.409pt}{0.400pt}}
\put(170.0,585.0){\rule[-0.200pt]{2.409pt}{0.400pt}}
\put(1429.0,585.0){\rule[-0.200pt]{2.409pt}{0.400pt}}
\put(170.0,585.0){\rule[-0.200pt]{2.409pt}{0.400pt}}
\put(1429.0,585.0){\rule[-0.200pt]{2.409pt}{0.400pt}}
\put(170.0,585.0){\rule[-0.200pt]{2.409pt}{0.400pt}}
\put(1429.0,585.0){\rule[-0.200pt]{2.409pt}{0.400pt}}
\put(170.0,585.0){\rule[-0.200pt]{2.409pt}{0.400pt}}
\put(1429.0,585.0){\rule[-0.200pt]{2.409pt}{0.400pt}}
\put(170.0,585.0){\rule[-0.200pt]{2.409pt}{0.400pt}}
\put(1429.0,585.0){\rule[-0.200pt]{2.409pt}{0.400pt}}
\put(170.0,585.0){\rule[-0.200pt]{2.409pt}{0.400pt}}
\put(1429.0,585.0){\rule[-0.200pt]{2.409pt}{0.400pt}}
\put(170.0,585.0){\rule[-0.200pt]{2.409pt}{0.400pt}}
\put(1429.0,585.0){\rule[-0.200pt]{2.409pt}{0.400pt}}
\put(170.0,585.0){\rule[-0.200pt]{2.409pt}{0.400pt}}
\put(1429.0,585.0){\rule[-0.200pt]{2.409pt}{0.400pt}}
\put(170.0,585.0){\rule[-0.200pt]{2.409pt}{0.400pt}}
\put(1429.0,585.0){\rule[-0.200pt]{2.409pt}{0.400pt}}
\put(170.0,585.0){\rule[-0.200pt]{2.409pt}{0.400pt}}
\put(1429.0,585.0){\rule[-0.200pt]{2.409pt}{0.400pt}}
\put(170.0,585.0){\rule[-0.200pt]{2.409pt}{0.400pt}}
\put(1429.0,585.0){\rule[-0.200pt]{2.409pt}{0.400pt}}
\put(170.0,585.0){\rule[-0.200pt]{2.409pt}{0.400pt}}
\put(1429.0,585.0){\rule[-0.200pt]{2.409pt}{0.400pt}}
\put(170.0,585.0){\rule[-0.200pt]{2.409pt}{0.400pt}}
\put(1429.0,585.0){\rule[-0.200pt]{2.409pt}{0.400pt}}
\put(170.0,586.0){\rule[-0.200pt]{2.409pt}{0.400pt}}
\put(1429.0,586.0){\rule[-0.200pt]{2.409pt}{0.400pt}}
\put(170.0,586.0){\rule[-0.200pt]{2.409pt}{0.400pt}}
\put(1429.0,586.0){\rule[-0.200pt]{2.409pt}{0.400pt}}
\put(170.0,586.0){\rule[-0.200pt]{2.409pt}{0.400pt}}
\put(1429.0,586.0){\rule[-0.200pt]{2.409pt}{0.400pt}}
\put(170.0,586.0){\rule[-0.200pt]{2.409pt}{0.400pt}}
\put(1429.0,586.0){\rule[-0.200pt]{2.409pt}{0.400pt}}
\put(170.0,586.0){\rule[-0.200pt]{2.409pt}{0.400pt}}
\put(1429.0,586.0){\rule[-0.200pt]{2.409pt}{0.400pt}}
\put(170.0,586.0){\rule[-0.200pt]{2.409pt}{0.400pt}}
\put(1429.0,586.0){\rule[-0.200pt]{2.409pt}{0.400pt}}
\put(170.0,586.0){\rule[-0.200pt]{2.409pt}{0.400pt}}
\put(1429.0,586.0){\rule[-0.200pt]{2.409pt}{0.400pt}}
\put(170.0,586.0){\rule[-0.200pt]{2.409pt}{0.400pt}}
\put(1429.0,586.0){\rule[-0.200pt]{2.409pt}{0.400pt}}
\put(170.0,586.0){\rule[-0.200pt]{2.409pt}{0.400pt}}
\put(1429.0,586.0){\rule[-0.200pt]{2.409pt}{0.400pt}}
\put(170.0,586.0){\rule[-0.200pt]{2.409pt}{0.400pt}}
\put(1429.0,586.0){\rule[-0.200pt]{2.409pt}{0.400pt}}
\put(170.0,586.0){\rule[-0.200pt]{2.409pt}{0.400pt}}
\put(1429.0,586.0){\rule[-0.200pt]{2.409pt}{0.400pt}}
\put(170.0,586.0){\rule[-0.200pt]{2.409pt}{0.400pt}}
\put(1429.0,586.0){\rule[-0.200pt]{2.409pt}{0.400pt}}
\put(170.0,586.0){\rule[-0.200pt]{2.409pt}{0.400pt}}
\put(1429.0,586.0){\rule[-0.200pt]{2.409pt}{0.400pt}}
\put(170.0,586.0){\rule[-0.200pt]{2.409pt}{0.400pt}}
\put(1429.0,586.0){\rule[-0.200pt]{2.409pt}{0.400pt}}
\put(170.0,586.0){\rule[-0.200pt]{2.409pt}{0.400pt}}
\put(1429.0,586.0){\rule[-0.200pt]{2.409pt}{0.400pt}}
\put(170.0,586.0){\rule[-0.200pt]{2.409pt}{0.400pt}}
\put(1429.0,586.0){\rule[-0.200pt]{2.409pt}{0.400pt}}
\put(170.0,586.0){\rule[-0.200pt]{2.409pt}{0.400pt}}
\put(1429.0,586.0){\rule[-0.200pt]{2.409pt}{0.400pt}}
\put(170.0,586.0){\rule[-0.200pt]{2.409pt}{0.400pt}}
\put(1429.0,586.0){\rule[-0.200pt]{2.409pt}{0.400pt}}
\put(170.0,587.0){\rule[-0.200pt]{2.409pt}{0.400pt}}
\put(1429.0,587.0){\rule[-0.200pt]{2.409pt}{0.400pt}}
\put(170.0,587.0){\rule[-0.200pt]{2.409pt}{0.400pt}}
\put(1429.0,587.0){\rule[-0.200pt]{2.409pt}{0.400pt}}
\put(170.0,587.0){\rule[-0.200pt]{2.409pt}{0.400pt}}
\put(1429.0,587.0){\rule[-0.200pt]{2.409pt}{0.400pt}}
\put(170.0,587.0){\rule[-0.200pt]{2.409pt}{0.400pt}}
\put(1429.0,587.0){\rule[-0.200pt]{2.409pt}{0.400pt}}
\put(170.0,587.0){\rule[-0.200pt]{2.409pt}{0.400pt}}
\put(1429.0,587.0){\rule[-0.200pt]{2.409pt}{0.400pt}}
\put(170.0,587.0){\rule[-0.200pt]{2.409pt}{0.400pt}}
\put(1429.0,587.0){\rule[-0.200pt]{2.409pt}{0.400pt}}
\put(170.0,587.0){\rule[-0.200pt]{2.409pt}{0.400pt}}
\put(1429.0,587.0){\rule[-0.200pt]{2.409pt}{0.400pt}}
\put(170.0,587.0){\rule[-0.200pt]{2.409pt}{0.400pt}}
\put(1429.0,587.0){\rule[-0.200pt]{2.409pt}{0.400pt}}
\put(170.0,587.0){\rule[-0.200pt]{2.409pt}{0.400pt}}
\put(1429.0,587.0){\rule[-0.200pt]{2.409pt}{0.400pt}}
\put(170.0,587.0){\rule[-0.200pt]{2.409pt}{0.400pt}}
\put(1429.0,587.0){\rule[-0.200pt]{2.409pt}{0.400pt}}
\put(170.0,587.0){\rule[-0.200pt]{2.409pt}{0.400pt}}
\put(1429.0,587.0){\rule[-0.200pt]{2.409pt}{0.400pt}}
\put(170.0,587.0){\rule[-0.200pt]{2.409pt}{0.400pt}}
\put(1429.0,587.0){\rule[-0.200pt]{2.409pt}{0.400pt}}
\put(170.0,587.0){\rule[-0.200pt]{2.409pt}{0.400pt}}
\put(1429.0,587.0){\rule[-0.200pt]{2.409pt}{0.400pt}}
\put(170.0,587.0){\rule[-0.200pt]{2.409pt}{0.400pt}}
\put(1429.0,587.0){\rule[-0.200pt]{2.409pt}{0.400pt}}
\put(170.0,587.0){\rule[-0.200pt]{2.409pt}{0.400pt}}
\put(1429.0,587.0){\rule[-0.200pt]{2.409pt}{0.400pt}}
\put(170.0,587.0){\rule[-0.200pt]{2.409pt}{0.400pt}}
\put(1429.0,587.0){\rule[-0.200pt]{2.409pt}{0.400pt}}
\put(170.0,587.0){\rule[-0.200pt]{2.409pt}{0.400pt}}
\put(1429.0,587.0){\rule[-0.200pt]{2.409pt}{0.400pt}}
\put(170.0,587.0){\rule[-0.200pt]{2.409pt}{0.400pt}}
\put(1429.0,587.0){\rule[-0.200pt]{2.409pt}{0.400pt}}
\put(170.0,587.0){\rule[-0.200pt]{2.409pt}{0.400pt}}
\put(1429.0,587.0){\rule[-0.200pt]{2.409pt}{0.400pt}}
\put(170.0,588.0){\rule[-0.200pt]{2.409pt}{0.400pt}}
\put(1429.0,588.0){\rule[-0.200pt]{2.409pt}{0.400pt}}
\put(170.0,588.0){\rule[-0.200pt]{2.409pt}{0.400pt}}
\put(1429.0,588.0){\rule[-0.200pt]{2.409pt}{0.400pt}}
\put(170.0,588.0){\rule[-0.200pt]{2.409pt}{0.400pt}}
\put(1429.0,588.0){\rule[-0.200pt]{2.409pt}{0.400pt}}
\put(170.0,588.0){\rule[-0.200pt]{2.409pt}{0.400pt}}
\put(1429.0,588.0){\rule[-0.200pt]{2.409pt}{0.400pt}}
\put(170.0,588.0){\rule[-0.200pt]{2.409pt}{0.400pt}}
\put(1429.0,588.0){\rule[-0.200pt]{2.409pt}{0.400pt}}
\put(170.0,588.0){\rule[-0.200pt]{2.409pt}{0.400pt}}
\put(1429.0,588.0){\rule[-0.200pt]{2.409pt}{0.400pt}}
\put(170.0,588.0){\rule[-0.200pt]{2.409pt}{0.400pt}}
\put(1429.0,588.0){\rule[-0.200pt]{2.409pt}{0.400pt}}
\put(170.0,588.0){\rule[-0.200pt]{2.409pt}{0.400pt}}
\put(1429.0,588.0){\rule[-0.200pt]{2.409pt}{0.400pt}}
\put(170.0,588.0){\rule[-0.200pt]{2.409pt}{0.400pt}}
\put(1429.0,588.0){\rule[-0.200pt]{2.409pt}{0.400pt}}
\put(170.0,588.0){\rule[-0.200pt]{2.409pt}{0.400pt}}
\put(1429.0,588.0){\rule[-0.200pt]{2.409pt}{0.400pt}}
\put(170.0,588.0){\rule[-0.200pt]{2.409pt}{0.400pt}}
\put(1429.0,588.0){\rule[-0.200pt]{2.409pt}{0.400pt}}
\put(170.0,588.0){\rule[-0.200pt]{2.409pt}{0.400pt}}
\put(1429.0,588.0){\rule[-0.200pt]{2.409pt}{0.400pt}}
\put(170.0,588.0){\rule[-0.200pt]{2.409pt}{0.400pt}}
\put(1429.0,588.0){\rule[-0.200pt]{2.409pt}{0.400pt}}
\put(170.0,588.0){\rule[-0.200pt]{2.409pt}{0.400pt}}
\put(1429.0,588.0){\rule[-0.200pt]{2.409pt}{0.400pt}}
\put(170.0,588.0){\rule[-0.200pt]{2.409pt}{0.400pt}}
\put(1429.0,588.0){\rule[-0.200pt]{2.409pt}{0.400pt}}
\put(170.0,588.0){\rule[-0.200pt]{2.409pt}{0.400pt}}
\put(1429.0,588.0){\rule[-0.200pt]{2.409pt}{0.400pt}}
\put(170.0,588.0){\rule[-0.200pt]{2.409pt}{0.400pt}}
\put(1429.0,588.0){\rule[-0.200pt]{2.409pt}{0.400pt}}
\put(170.0,588.0){\rule[-0.200pt]{2.409pt}{0.400pt}}
\put(1429.0,588.0){\rule[-0.200pt]{2.409pt}{0.400pt}}
\put(170.0,588.0){\rule[-0.200pt]{2.409pt}{0.400pt}}
\put(1429.0,588.0){\rule[-0.200pt]{2.409pt}{0.400pt}}
\put(170.0,589.0){\rule[-0.200pt]{2.409pt}{0.400pt}}
\put(1429.0,589.0){\rule[-0.200pt]{2.409pt}{0.400pt}}
\put(170.0,589.0){\rule[-0.200pt]{2.409pt}{0.400pt}}
\put(1429.0,589.0){\rule[-0.200pt]{2.409pt}{0.400pt}}
\put(170.0,589.0){\rule[-0.200pt]{2.409pt}{0.400pt}}
\put(1429.0,589.0){\rule[-0.200pt]{2.409pt}{0.400pt}}
\put(170.0,589.0){\rule[-0.200pt]{2.409pt}{0.400pt}}
\put(1429.0,589.0){\rule[-0.200pt]{2.409pt}{0.400pt}}
\put(170.0,589.0){\rule[-0.200pt]{2.409pt}{0.400pt}}
\put(1429.0,589.0){\rule[-0.200pt]{2.409pt}{0.400pt}}
\put(170.0,589.0){\rule[-0.200pt]{2.409pt}{0.400pt}}
\put(1429.0,589.0){\rule[-0.200pt]{2.409pt}{0.400pt}}
\put(170.0,589.0){\rule[-0.200pt]{2.409pt}{0.400pt}}
\put(1429.0,589.0){\rule[-0.200pt]{2.409pt}{0.400pt}}
\put(170.0,589.0){\rule[-0.200pt]{2.409pt}{0.400pt}}
\put(1429.0,589.0){\rule[-0.200pt]{2.409pt}{0.400pt}}
\put(170.0,589.0){\rule[-0.200pt]{2.409pt}{0.400pt}}
\put(1429.0,589.0){\rule[-0.200pt]{2.409pt}{0.400pt}}
\put(170.0,589.0){\rule[-0.200pt]{2.409pt}{0.400pt}}
\put(1429.0,589.0){\rule[-0.200pt]{2.409pt}{0.400pt}}
\put(170.0,589.0){\rule[-0.200pt]{2.409pt}{0.400pt}}
\put(1429.0,589.0){\rule[-0.200pt]{2.409pt}{0.400pt}}
\put(170.0,589.0){\rule[-0.200pt]{2.409pt}{0.400pt}}
\put(1429.0,589.0){\rule[-0.200pt]{2.409pt}{0.400pt}}
\put(170.0,589.0){\rule[-0.200pt]{2.409pt}{0.400pt}}
\put(1429.0,589.0){\rule[-0.200pt]{2.409pt}{0.400pt}}
\put(170.0,589.0){\rule[-0.200pt]{2.409pt}{0.400pt}}
\put(1429.0,589.0){\rule[-0.200pt]{2.409pt}{0.400pt}}
\put(170.0,589.0){\rule[-0.200pt]{2.409pt}{0.400pt}}
\put(1429.0,589.0){\rule[-0.200pt]{2.409pt}{0.400pt}}
\put(170.0,589.0){\rule[-0.200pt]{2.409pt}{0.400pt}}
\put(1429.0,589.0){\rule[-0.200pt]{2.409pt}{0.400pt}}
\put(170.0,589.0){\rule[-0.200pt]{2.409pt}{0.400pt}}
\put(1429.0,589.0){\rule[-0.200pt]{2.409pt}{0.400pt}}
\put(170.0,589.0){\rule[-0.200pt]{2.409pt}{0.400pt}}
\put(1429.0,589.0){\rule[-0.200pt]{2.409pt}{0.400pt}}
\put(170.0,589.0){\rule[-0.200pt]{2.409pt}{0.400pt}}
\put(1429.0,589.0){\rule[-0.200pt]{2.409pt}{0.400pt}}
\put(170.0,589.0){\rule[-0.200pt]{2.409pt}{0.400pt}}
\put(1429.0,589.0){\rule[-0.200pt]{2.409pt}{0.400pt}}
\put(170.0,590.0){\rule[-0.200pt]{2.409pt}{0.400pt}}
\put(1429.0,590.0){\rule[-0.200pt]{2.409pt}{0.400pt}}
\put(170.0,590.0){\rule[-0.200pt]{2.409pt}{0.400pt}}
\put(1429.0,590.0){\rule[-0.200pt]{2.409pt}{0.400pt}}
\put(170.0,590.0){\rule[-0.200pt]{2.409pt}{0.400pt}}
\put(1429.0,590.0){\rule[-0.200pt]{2.409pt}{0.400pt}}
\put(170.0,590.0){\rule[-0.200pt]{2.409pt}{0.400pt}}
\put(1429.0,590.0){\rule[-0.200pt]{2.409pt}{0.400pt}}
\put(170.0,590.0){\rule[-0.200pt]{2.409pt}{0.400pt}}
\put(1429.0,590.0){\rule[-0.200pt]{2.409pt}{0.400pt}}
\put(170.0,590.0){\rule[-0.200pt]{2.409pt}{0.400pt}}
\put(1429.0,590.0){\rule[-0.200pt]{2.409pt}{0.400pt}}
\put(170.0,590.0){\rule[-0.200pt]{2.409pt}{0.400pt}}
\put(1429.0,590.0){\rule[-0.200pt]{2.409pt}{0.400pt}}
\put(170.0,590.0){\rule[-0.200pt]{2.409pt}{0.400pt}}
\put(1429.0,590.0){\rule[-0.200pt]{2.409pt}{0.400pt}}
\put(170.0,590.0){\rule[-0.200pt]{2.409pt}{0.400pt}}
\put(1429.0,590.0){\rule[-0.200pt]{2.409pt}{0.400pt}}
\put(170.0,590.0){\rule[-0.200pt]{2.409pt}{0.400pt}}
\put(1429.0,590.0){\rule[-0.200pt]{2.409pt}{0.400pt}}
\put(170.0,590.0){\rule[-0.200pt]{2.409pt}{0.400pt}}
\put(1429.0,590.0){\rule[-0.200pt]{2.409pt}{0.400pt}}
\put(170.0,590.0){\rule[-0.200pt]{2.409pt}{0.400pt}}
\put(1429.0,590.0){\rule[-0.200pt]{2.409pt}{0.400pt}}
\put(170.0,590.0){\rule[-0.200pt]{2.409pt}{0.400pt}}
\put(1429.0,590.0){\rule[-0.200pt]{2.409pt}{0.400pt}}
\put(170.0,590.0){\rule[-0.200pt]{2.409pt}{0.400pt}}
\put(1429.0,590.0){\rule[-0.200pt]{2.409pt}{0.400pt}}
\put(170.0,590.0){\rule[-0.200pt]{2.409pt}{0.400pt}}
\put(1429.0,590.0){\rule[-0.200pt]{2.409pt}{0.400pt}}
\put(170.0,590.0){\rule[-0.200pt]{2.409pt}{0.400pt}}
\put(1429.0,590.0){\rule[-0.200pt]{2.409pt}{0.400pt}}
\put(170.0,590.0){\rule[-0.200pt]{2.409pt}{0.400pt}}
\put(1429.0,590.0){\rule[-0.200pt]{2.409pt}{0.400pt}}
\put(170.0,590.0){\rule[-0.200pt]{2.409pt}{0.400pt}}
\put(1429.0,590.0){\rule[-0.200pt]{2.409pt}{0.400pt}}
\put(170.0,590.0){\rule[-0.200pt]{2.409pt}{0.400pt}}
\put(1429.0,590.0){\rule[-0.200pt]{2.409pt}{0.400pt}}
\put(170.0,590.0){\rule[-0.200pt]{2.409pt}{0.400pt}}
\put(1429.0,590.0){\rule[-0.200pt]{2.409pt}{0.400pt}}
\put(170.0,590.0){\rule[-0.200pt]{2.409pt}{0.400pt}}
\put(1429.0,590.0){\rule[-0.200pt]{2.409pt}{0.400pt}}
\put(170.0,591.0){\rule[-0.200pt]{2.409pt}{0.400pt}}
\put(1429.0,591.0){\rule[-0.200pt]{2.409pt}{0.400pt}}
\put(170.0,591.0){\rule[-0.200pt]{2.409pt}{0.400pt}}
\put(1429.0,591.0){\rule[-0.200pt]{2.409pt}{0.400pt}}
\put(170.0,591.0){\rule[-0.200pt]{2.409pt}{0.400pt}}
\put(1429.0,591.0){\rule[-0.200pt]{2.409pt}{0.400pt}}
\put(170.0,591.0){\rule[-0.200pt]{2.409pt}{0.400pt}}
\put(1429.0,591.0){\rule[-0.200pt]{2.409pt}{0.400pt}}
\put(170.0,591.0){\rule[-0.200pt]{2.409pt}{0.400pt}}
\put(1429.0,591.0){\rule[-0.200pt]{2.409pt}{0.400pt}}
\put(170.0,591.0){\rule[-0.200pt]{2.409pt}{0.400pt}}
\put(1429.0,591.0){\rule[-0.200pt]{2.409pt}{0.400pt}}
\put(170.0,591.0){\rule[-0.200pt]{2.409pt}{0.400pt}}
\put(1429.0,591.0){\rule[-0.200pt]{2.409pt}{0.400pt}}
\put(170.0,591.0){\rule[-0.200pt]{2.409pt}{0.400pt}}
\put(1429.0,591.0){\rule[-0.200pt]{2.409pt}{0.400pt}}
\put(170.0,591.0){\rule[-0.200pt]{2.409pt}{0.400pt}}
\put(1429.0,591.0){\rule[-0.200pt]{2.409pt}{0.400pt}}
\put(170.0,591.0){\rule[-0.200pt]{2.409pt}{0.400pt}}
\put(1429.0,591.0){\rule[-0.200pt]{2.409pt}{0.400pt}}
\put(170.0,591.0){\rule[-0.200pt]{2.409pt}{0.400pt}}
\put(1429.0,591.0){\rule[-0.200pt]{2.409pt}{0.400pt}}
\put(170.0,591.0){\rule[-0.200pt]{2.409pt}{0.400pt}}
\put(1429.0,591.0){\rule[-0.200pt]{2.409pt}{0.400pt}}
\put(170.0,591.0){\rule[-0.200pt]{2.409pt}{0.400pt}}
\put(1429.0,591.0){\rule[-0.200pt]{2.409pt}{0.400pt}}
\put(170.0,591.0){\rule[-0.200pt]{2.409pt}{0.400pt}}
\put(1429.0,591.0){\rule[-0.200pt]{2.409pt}{0.400pt}}
\put(170.0,591.0){\rule[-0.200pt]{2.409pt}{0.400pt}}
\put(1429.0,591.0){\rule[-0.200pt]{2.409pt}{0.400pt}}
\put(170.0,591.0){\rule[-0.200pt]{2.409pt}{0.400pt}}
\put(1429.0,591.0){\rule[-0.200pt]{2.409pt}{0.400pt}}
\put(170.0,591.0){\rule[-0.200pt]{2.409pt}{0.400pt}}
\put(1429.0,591.0){\rule[-0.200pt]{2.409pt}{0.400pt}}
\put(170.0,591.0){\rule[-0.200pt]{2.409pt}{0.400pt}}
\put(1429.0,591.0){\rule[-0.200pt]{2.409pt}{0.400pt}}
\put(170.0,591.0){\rule[-0.200pt]{2.409pt}{0.400pt}}
\put(1429.0,591.0){\rule[-0.200pt]{2.409pt}{0.400pt}}
\put(170.0,591.0){\rule[-0.200pt]{2.409pt}{0.400pt}}
\put(1429.0,591.0){\rule[-0.200pt]{2.409pt}{0.400pt}}
\put(170.0,591.0){\rule[-0.200pt]{2.409pt}{0.400pt}}
\put(1429.0,591.0){\rule[-0.200pt]{2.409pt}{0.400pt}}
\put(170.0,592.0){\rule[-0.200pt]{2.409pt}{0.400pt}}
\put(1429.0,592.0){\rule[-0.200pt]{2.409pt}{0.400pt}}
\put(170.0,592.0){\rule[-0.200pt]{2.409pt}{0.400pt}}
\put(1429.0,592.0){\rule[-0.200pt]{2.409pt}{0.400pt}}
\put(170.0,592.0){\rule[-0.200pt]{2.409pt}{0.400pt}}
\put(1429.0,592.0){\rule[-0.200pt]{2.409pt}{0.400pt}}
\put(170.0,592.0){\rule[-0.200pt]{2.409pt}{0.400pt}}
\put(1429.0,592.0){\rule[-0.200pt]{2.409pt}{0.400pt}}
\put(170.0,592.0){\rule[-0.200pt]{2.409pt}{0.400pt}}
\put(1429.0,592.0){\rule[-0.200pt]{2.409pt}{0.400pt}}
\put(170.0,592.0){\rule[-0.200pt]{2.409pt}{0.400pt}}
\put(1429.0,592.0){\rule[-0.200pt]{2.409pt}{0.400pt}}
\put(170.0,592.0){\rule[-0.200pt]{2.409pt}{0.400pt}}
\put(1429.0,592.0){\rule[-0.200pt]{2.409pt}{0.400pt}}
\put(170.0,592.0){\rule[-0.200pt]{2.409pt}{0.400pt}}
\put(1429.0,592.0){\rule[-0.200pt]{2.409pt}{0.400pt}}
\put(170.0,592.0){\rule[-0.200pt]{2.409pt}{0.400pt}}
\put(1429.0,592.0){\rule[-0.200pt]{2.409pt}{0.400pt}}
\put(170.0,592.0){\rule[-0.200pt]{2.409pt}{0.400pt}}
\put(1429.0,592.0){\rule[-0.200pt]{2.409pt}{0.400pt}}
\put(170.0,592.0){\rule[-0.200pt]{2.409pt}{0.400pt}}
\put(1429.0,592.0){\rule[-0.200pt]{2.409pt}{0.400pt}}
\put(170.0,592.0){\rule[-0.200pt]{2.409pt}{0.400pt}}
\put(1429.0,592.0){\rule[-0.200pt]{2.409pt}{0.400pt}}
\put(170.0,592.0){\rule[-0.200pt]{2.409pt}{0.400pt}}
\put(1429.0,592.0){\rule[-0.200pt]{2.409pt}{0.400pt}}
\put(170.0,592.0){\rule[-0.200pt]{2.409pt}{0.400pt}}
\put(1429.0,592.0){\rule[-0.200pt]{2.409pt}{0.400pt}}
\put(170.0,592.0){\rule[-0.200pt]{2.409pt}{0.400pt}}
\put(1429.0,592.0){\rule[-0.200pt]{2.409pt}{0.400pt}}
\put(170.0,592.0){\rule[-0.200pt]{2.409pt}{0.400pt}}
\put(1429.0,592.0){\rule[-0.200pt]{2.409pt}{0.400pt}}
\put(170.0,592.0){\rule[-0.200pt]{2.409pt}{0.400pt}}
\put(1429.0,592.0){\rule[-0.200pt]{2.409pt}{0.400pt}}
\put(170.0,592.0){\rule[-0.200pt]{2.409pt}{0.400pt}}
\put(1429.0,592.0){\rule[-0.200pt]{2.409pt}{0.400pt}}
\put(170.0,592.0){\rule[-0.200pt]{2.409pt}{0.400pt}}
\put(1429.0,592.0){\rule[-0.200pt]{2.409pt}{0.400pt}}
\put(170.0,592.0){\rule[-0.200pt]{2.409pt}{0.400pt}}
\put(1429.0,592.0){\rule[-0.200pt]{2.409pt}{0.400pt}}
\put(170.0,592.0){\rule[-0.200pt]{2.409pt}{0.400pt}}
\put(1429.0,592.0){\rule[-0.200pt]{2.409pt}{0.400pt}}
\put(170.0,593.0){\rule[-0.200pt]{2.409pt}{0.400pt}}
\put(1429.0,593.0){\rule[-0.200pt]{2.409pt}{0.400pt}}
\put(170.0,593.0){\rule[-0.200pt]{2.409pt}{0.400pt}}
\put(1429.0,593.0){\rule[-0.200pt]{2.409pt}{0.400pt}}
\put(170.0,593.0){\rule[-0.200pt]{2.409pt}{0.400pt}}
\put(1429.0,593.0){\rule[-0.200pt]{2.409pt}{0.400pt}}
\put(170.0,593.0){\rule[-0.200pt]{2.409pt}{0.400pt}}
\put(1429.0,593.0){\rule[-0.200pt]{2.409pt}{0.400pt}}
\put(170.0,593.0){\rule[-0.200pt]{2.409pt}{0.400pt}}
\put(1429.0,593.0){\rule[-0.200pt]{2.409pt}{0.400pt}}
\put(170.0,593.0){\rule[-0.200pt]{2.409pt}{0.400pt}}
\put(1429.0,593.0){\rule[-0.200pt]{2.409pt}{0.400pt}}
\put(170.0,593.0){\rule[-0.200pt]{2.409pt}{0.400pt}}
\put(1429.0,593.0){\rule[-0.200pt]{2.409pt}{0.400pt}}
\put(170.0,593.0){\rule[-0.200pt]{2.409pt}{0.400pt}}
\put(1429.0,593.0){\rule[-0.200pt]{2.409pt}{0.400pt}}
\put(170.0,593.0){\rule[-0.200pt]{2.409pt}{0.400pt}}
\put(1429.0,593.0){\rule[-0.200pt]{2.409pt}{0.400pt}}
\put(170.0,593.0){\rule[-0.200pt]{2.409pt}{0.400pt}}
\put(1429.0,593.0){\rule[-0.200pt]{2.409pt}{0.400pt}}
\put(170.0,593.0){\rule[-0.200pt]{2.409pt}{0.400pt}}
\put(1429.0,593.0){\rule[-0.200pt]{2.409pt}{0.400pt}}
\put(170.0,593.0){\rule[-0.200pt]{2.409pt}{0.400pt}}
\put(1429.0,593.0){\rule[-0.200pt]{2.409pt}{0.400pt}}
\put(170.0,593.0){\rule[-0.200pt]{2.409pt}{0.400pt}}
\put(1429.0,593.0){\rule[-0.200pt]{2.409pt}{0.400pt}}
\put(170.0,593.0){\rule[-0.200pt]{2.409pt}{0.400pt}}
\put(1429.0,593.0){\rule[-0.200pt]{2.409pt}{0.400pt}}
\put(170.0,593.0){\rule[-0.200pt]{2.409pt}{0.400pt}}
\put(1429.0,593.0){\rule[-0.200pt]{2.409pt}{0.400pt}}
\put(170.0,593.0){\rule[-0.200pt]{2.409pt}{0.400pt}}
\put(1429.0,593.0){\rule[-0.200pt]{2.409pt}{0.400pt}}
\put(170.0,593.0){\rule[-0.200pt]{2.409pt}{0.400pt}}
\put(1429.0,593.0){\rule[-0.200pt]{2.409pt}{0.400pt}}
\put(170.0,593.0){\rule[-0.200pt]{2.409pt}{0.400pt}}
\put(1429.0,593.0){\rule[-0.200pt]{2.409pt}{0.400pt}}
\put(170.0,593.0){\rule[-0.200pt]{2.409pt}{0.400pt}}
\put(1429.0,593.0){\rule[-0.200pt]{2.409pt}{0.400pt}}
\put(170.0,593.0){\rule[-0.200pt]{2.409pt}{0.400pt}}
\put(1429.0,593.0){\rule[-0.200pt]{2.409pt}{0.400pt}}
\put(170.0,593.0){\rule[-0.200pt]{2.409pt}{0.400pt}}
\put(1429.0,593.0){\rule[-0.200pt]{2.409pt}{0.400pt}}
\put(170.0,593.0){\rule[-0.200pt]{2.409pt}{0.400pt}}
\put(1429.0,593.0){\rule[-0.200pt]{2.409pt}{0.400pt}}
\put(170.0,594.0){\rule[-0.200pt]{2.409pt}{0.400pt}}
\put(1429.0,594.0){\rule[-0.200pt]{2.409pt}{0.400pt}}
\put(170.0,594.0){\rule[-0.200pt]{2.409pt}{0.400pt}}
\put(1429.0,594.0){\rule[-0.200pt]{2.409pt}{0.400pt}}
\put(170.0,594.0){\rule[-0.200pt]{2.409pt}{0.400pt}}
\put(1429.0,594.0){\rule[-0.200pt]{2.409pt}{0.400pt}}
\put(170.0,594.0){\rule[-0.200pt]{2.409pt}{0.400pt}}
\put(1429.0,594.0){\rule[-0.200pt]{2.409pt}{0.400pt}}
\put(170.0,594.0){\rule[-0.200pt]{2.409pt}{0.400pt}}
\put(1429.0,594.0){\rule[-0.200pt]{2.409pt}{0.400pt}}
\put(170.0,594.0){\rule[-0.200pt]{2.409pt}{0.400pt}}
\put(1429.0,594.0){\rule[-0.200pt]{2.409pt}{0.400pt}}
\put(170.0,594.0){\rule[-0.200pt]{2.409pt}{0.400pt}}
\put(1429.0,594.0){\rule[-0.200pt]{2.409pt}{0.400pt}}
\put(170.0,594.0){\rule[-0.200pt]{2.409pt}{0.400pt}}
\put(1429.0,594.0){\rule[-0.200pt]{2.409pt}{0.400pt}}
\put(170.0,594.0){\rule[-0.200pt]{2.409pt}{0.400pt}}
\put(1429.0,594.0){\rule[-0.200pt]{2.409pt}{0.400pt}}
\put(170.0,594.0){\rule[-0.200pt]{2.409pt}{0.400pt}}
\put(1429.0,594.0){\rule[-0.200pt]{2.409pt}{0.400pt}}
\put(170.0,594.0){\rule[-0.200pt]{2.409pt}{0.400pt}}
\put(1429.0,594.0){\rule[-0.200pt]{2.409pt}{0.400pt}}
\put(170.0,594.0){\rule[-0.200pt]{2.409pt}{0.400pt}}
\put(1429.0,594.0){\rule[-0.200pt]{2.409pt}{0.400pt}}
\put(170.0,594.0){\rule[-0.200pt]{2.409pt}{0.400pt}}
\put(1429.0,594.0){\rule[-0.200pt]{2.409pt}{0.400pt}}
\put(170.0,594.0){\rule[-0.200pt]{2.409pt}{0.400pt}}
\put(1429.0,594.0){\rule[-0.200pt]{2.409pt}{0.400pt}}
\put(170.0,594.0){\rule[-0.200pt]{2.409pt}{0.400pt}}
\put(1429.0,594.0){\rule[-0.200pt]{2.409pt}{0.400pt}}
\put(170.0,594.0){\rule[-0.200pt]{2.409pt}{0.400pt}}
\put(1429.0,594.0){\rule[-0.200pt]{2.409pt}{0.400pt}}
\put(170.0,594.0){\rule[-0.200pt]{2.409pt}{0.400pt}}
\put(1429.0,594.0){\rule[-0.200pt]{2.409pt}{0.400pt}}
\put(170.0,594.0){\rule[-0.200pt]{2.409pt}{0.400pt}}
\put(1429.0,594.0){\rule[-0.200pt]{2.409pt}{0.400pt}}
\put(170.0,594.0){\rule[-0.200pt]{2.409pt}{0.400pt}}
\put(1429.0,594.0){\rule[-0.200pt]{2.409pt}{0.400pt}}
\put(170.0,594.0){\rule[-0.200pt]{2.409pt}{0.400pt}}
\put(1429.0,594.0){\rule[-0.200pt]{2.409pt}{0.400pt}}
\put(170.0,594.0){\rule[-0.200pt]{2.409pt}{0.400pt}}
\put(1429.0,594.0){\rule[-0.200pt]{2.409pt}{0.400pt}}
\put(170.0,594.0){\rule[-0.200pt]{2.409pt}{0.400pt}}
\put(1429.0,594.0){\rule[-0.200pt]{2.409pt}{0.400pt}}
\put(170.0,594.0){\rule[-0.200pt]{2.409pt}{0.400pt}}
\put(1429.0,594.0){\rule[-0.200pt]{2.409pt}{0.400pt}}
\put(170.0,595.0){\rule[-0.200pt]{2.409pt}{0.400pt}}
\put(1429.0,595.0){\rule[-0.200pt]{2.409pt}{0.400pt}}
\put(170.0,595.0){\rule[-0.200pt]{2.409pt}{0.400pt}}
\put(1429.0,595.0){\rule[-0.200pt]{2.409pt}{0.400pt}}
\put(170.0,595.0){\rule[-0.200pt]{2.409pt}{0.400pt}}
\put(1429.0,595.0){\rule[-0.200pt]{2.409pt}{0.400pt}}
\put(170.0,595.0){\rule[-0.200pt]{2.409pt}{0.400pt}}
\put(1429.0,595.0){\rule[-0.200pt]{2.409pt}{0.400pt}}
\put(170.0,595.0){\rule[-0.200pt]{2.409pt}{0.400pt}}
\put(1429.0,595.0){\rule[-0.200pt]{2.409pt}{0.400pt}}
\put(170.0,595.0){\rule[-0.200pt]{2.409pt}{0.400pt}}
\put(1429.0,595.0){\rule[-0.200pt]{2.409pt}{0.400pt}}
\put(170.0,595.0){\rule[-0.200pt]{2.409pt}{0.400pt}}
\put(1429.0,595.0){\rule[-0.200pt]{2.409pt}{0.400pt}}
\put(170.0,595.0){\rule[-0.200pt]{2.409pt}{0.400pt}}
\put(1429.0,595.0){\rule[-0.200pt]{2.409pt}{0.400pt}}
\put(170.0,595.0){\rule[-0.200pt]{2.409pt}{0.400pt}}
\put(1429.0,595.0){\rule[-0.200pt]{2.409pt}{0.400pt}}
\put(170.0,595.0){\rule[-0.200pt]{2.409pt}{0.400pt}}
\put(1429.0,595.0){\rule[-0.200pt]{2.409pt}{0.400pt}}
\put(170.0,595.0){\rule[-0.200pt]{2.409pt}{0.400pt}}
\put(1429.0,595.0){\rule[-0.200pt]{2.409pt}{0.400pt}}
\put(170.0,595.0){\rule[-0.200pt]{2.409pt}{0.400pt}}
\put(1429.0,595.0){\rule[-0.200pt]{2.409pt}{0.400pt}}
\put(170.0,595.0){\rule[-0.200pt]{2.409pt}{0.400pt}}
\put(1429.0,595.0){\rule[-0.200pt]{2.409pt}{0.400pt}}
\put(170.0,595.0){\rule[-0.200pt]{2.409pt}{0.400pt}}
\put(1429.0,595.0){\rule[-0.200pt]{2.409pt}{0.400pt}}
\put(170.0,595.0){\rule[-0.200pt]{2.409pt}{0.400pt}}
\put(1429.0,595.0){\rule[-0.200pt]{2.409pt}{0.400pt}}
\put(170.0,595.0){\rule[-0.200pt]{2.409pt}{0.400pt}}
\put(1429.0,595.0){\rule[-0.200pt]{2.409pt}{0.400pt}}
\put(170.0,595.0){\rule[-0.200pt]{2.409pt}{0.400pt}}
\put(1429.0,595.0){\rule[-0.200pt]{2.409pt}{0.400pt}}
\put(170.0,595.0){\rule[-0.200pt]{2.409pt}{0.400pt}}
\put(1429.0,595.0){\rule[-0.200pt]{2.409pt}{0.400pt}}
\put(170.0,595.0){\rule[-0.200pt]{2.409pt}{0.400pt}}
\put(1429.0,595.0){\rule[-0.200pt]{2.409pt}{0.400pt}}
\put(170.0,595.0){\rule[-0.200pt]{2.409pt}{0.400pt}}
\put(1429.0,595.0){\rule[-0.200pt]{2.409pt}{0.400pt}}
\put(170.0,595.0){\rule[-0.200pt]{2.409pt}{0.400pt}}
\put(1429.0,595.0){\rule[-0.200pt]{2.409pt}{0.400pt}}
\put(170.0,595.0){\rule[-0.200pt]{2.409pt}{0.400pt}}
\put(1429.0,595.0){\rule[-0.200pt]{2.409pt}{0.400pt}}
\put(170.0,595.0){\rule[-0.200pt]{2.409pt}{0.400pt}}
\put(1429.0,595.0){\rule[-0.200pt]{2.409pt}{0.400pt}}
\put(170.0,596.0){\rule[-0.200pt]{2.409pt}{0.400pt}}
\put(1429.0,596.0){\rule[-0.200pt]{2.409pt}{0.400pt}}
\put(170.0,596.0){\rule[-0.200pt]{2.409pt}{0.400pt}}
\put(1429.0,596.0){\rule[-0.200pt]{2.409pt}{0.400pt}}
\put(170.0,596.0){\rule[-0.200pt]{2.409pt}{0.400pt}}
\put(1429.0,596.0){\rule[-0.200pt]{2.409pt}{0.400pt}}
\put(170.0,596.0){\rule[-0.200pt]{2.409pt}{0.400pt}}
\put(1429.0,596.0){\rule[-0.200pt]{2.409pt}{0.400pt}}
\put(170.0,596.0){\rule[-0.200pt]{2.409pt}{0.400pt}}
\put(1429.0,596.0){\rule[-0.200pt]{2.409pt}{0.400pt}}
\put(170.0,596.0){\rule[-0.200pt]{2.409pt}{0.400pt}}
\put(1429.0,596.0){\rule[-0.200pt]{2.409pt}{0.400pt}}
\put(170.0,596.0){\rule[-0.200pt]{2.409pt}{0.400pt}}
\put(1429.0,596.0){\rule[-0.200pt]{2.409pt}{0.400pt}}
\put(170.0,596.0){\rule[-0.200pt]{2.409pt}{0.400pt}}
\put(1429.0,596.0){\rule[-0.200pt]{2.409pt}{0.400pt}}
\put(170.0,596.0){\rule[-0.200pt]{2.409pt}{0.400pt}}
\put(1429.0,596.0){\rule[-0.200pt]{2.409pt}{0.400pt}}
\put(170.0,596.0){\rule[-0.200pt]{2.409pt}{0.400pt}}
\put(1429.0,596.0){\rule[-0.200pt]{2.409pt}{0.400pt}}
\put(170.0,596.0){\rule[-0.200pt]{2.409pt}{0.400pt}}
\put(1429.0,596.0){\rule[-0.200pt]{2.409pt}{0.400pt}}
\put(170.0,596.0){\rule[-0.200pt]{2.409pt}{0.400pt}}
\put(1429.0,596.0){\rule[-0.200pt]{2.409pt}{0.400pt}}
\put(170.0,596.0){\rule[-0.200pt]{2.409pt}{0.400pt}}
\put(1429.0,596.0){\rule[-0.200pt]{2.409pt}{0.400pt}}
\put(170.0,596.0){\rule[-0.200pt]{2.409pt}{0.400pt}}
\put(1429.0,596.0){\rule[-0.200pt]{2.409pt}{0.400pt}}
\put(170.0,596.0){\rule[-0.200pt]{2.409pt}{0.400pt}}
\put(1429.0,596.0){\rule[-0.200pt]{2.409pt}{0.400pt}}
\put(170.0,596.0){\rule[-0.200pt]{2.409pt}{0.400pt}}
\put(1429.0,596.0){\rule[-0.200pt]{2.409pt}{0.400pt}}
\put(170.0,596.0){\rule[-0.200pt]{2.409pt}{0.400pt}}
\put(1429.0,596.0){\rule[-0.200pt]{2.409pt}{0.400pt}}
\put(170.0,596.0){\rule[-0.200pt]{2.409pt}{0.400pt}}
\put(1429.0,596.0){\rule[-0.200pt]{2.409pt}{0.400pt}}
\put(170.0,596.0){\rule[-0.200pt]{2.409pt}{0.400pt}}
\put(1429.0,596.0){\rule[-0.200pt]{2.409pt}{0.400pt}}
\put(170.0,596.0){\rule[-0.200pt]{2.409pt}{0.400pt}}
\put(1429.0,596.0){\rule[-0.200pt]{2.409pt}{0.400pt}}
\put(170.0,596.0){\rule[-0.200pt]{2.409pt}{0.400pt}}
\put(1429.0,596.0){\rule[-0.200pt]{2.409pt}{0.400pt}}
\put(170.0,596.0){\rule[-0.200pt]{2.409pt}{0.400pt}}
\put(1429.0,596.0){\rule[-0.200pt]{2.409pt}{0.400pt}}
\put(170.0,596.0){\rule[-0.200pt]{2.409pt}{0.400pt}}
\put(1429.0,596.0){\rule[-0.200pt]{2.409pt}{0.400pt}}
\put(170.0,596.0){\rule[-0.200pt]{2.409pt}{0.400pt}}
\put(1429.0,596.0){\rule[-0.200pt]{2.409pt}{0.400pt}}
\put(170.0,597.0){\rule[-0.200pt]{2.409pt}{0.400pt}}
\put(1429.0,597.0){\rule[-0.200pt]{2.409pt}{0.400pt}}
\put(170.0,597.0){\rule[-0.200pt]{2.409pt}{0.400pt}}
\put(1429.0,597.0){\rule[-0.200pt]{2.409pt}{0.400pt}}
\put(170.0,597.0){\rule[-0.200pt]{2.409pt}{0.400pt}}
\put(1429.0,597.0){\rule[-0.200pt]{2.409pt}{0.400pt}}
\put(170.0,597.0){\rule[-0.200pt]{2.409pt}{0.400pt}}
\put(1429.0,597.0){\rule[-0.200pt]{2.409pt}{0.400pt}}
\put(170.0,597.0){\rule[-0.200pt]{2.409pt}{0.400pt}}
\put(1429.0,597.0){\rule[-0.200pt]{2.409pt}{0.400pt}}
\put(170.0,597.0){\rule[-0.200pt]{2.409pt}{0.400pt}}
\put(1429.0,597.0){\rule[-0.200pt]{2.409pt}{0.400pt}}
\put(170.0,597.0){\rule[-0.200pt]{2.409pt}{0.400pt}}
\put(1429.0,597.0){\rule[-0.200pt]{2.409pt}{0.400pt}}
\put(170.0,597.0){\rule[-0.200pt]{2.409pt}{0.400pt}}
\put(1429.0,597.0){\rule[-0.200pt]{2.409pt}{0.400pt}}
\put(170.0,597.0){\rule[-0.200pt]{2.409pt}{0.400pt}}
\put(1429.0,597.0){\rule[-0.200pt]{2.409pt}{0.400pt}}
\put(170.0,597.0){\rule[-0.200pt]{2.409pt}{0.400pt}}
\put(1429.0,597.0){\rule[-0.200pt]{2.409pt}{0.400pt}}
\put(170.0,597.0){\rule[-0.200pt]{2.409pt}{0.400pt}}
\put(1429.0,597.0){\rule[-0.200pt]{2.409pt}{0.400pt}}
\put(170.0,597.0){\rule[-0.200pt]{2.409pt}{0.400pt}}
\put(1429.0,597.0){\rule[-0.200pt]{2.409pt}{0.400pt}}
\put(170.0,597.0){\rule[-0.200pt]{2.409pt}{0.400pt}}
\put(1429.0,597.0){\rule[-0.200pt]{2.409pt}{0.400pt}}
\put(170.0,597.0){\rule[-0.200pt]{2.409pt}{0.400pt}}
\put(1429.0,597.0){\rule[-0.200pt]{2.409pt}{0.400pt}}
\put(170.0,597.0){\rule[-0.200pt]{2.409pt}{0.400pt}}
\put(1429.0,597.0){\rule[-0.200pt]{2.409pt}{0.400pt}}
\put(170.0,597.0){\rule[-0.200pt]{2.409pt}{0.400pt}}
\put(1429.0,597.0){\rule[-0.200pt]{2.409pt}{0.400pt}}
\put(170.0,597.0){\rule[-0.200pt]{2.409pt}{0.400pt}}
\put(1429.0,597.0){\rule[-0.200pt]{2.409pt}{0.400pt}}
\put(170.0,597.0){\rule[-0.200pt]{2.409pt}{0.400pt}}
\put(1429.0,597.0){\rule[-0.200pt]{2.409pt}{0.400pt}}
\put(170.0,597.0){\rule[-0.200pt]{2.409pt}{0.400pt}}
\put(1429.0,597.0){\rule[-0.200pt]{2.409pt}{0.400pt}}
\put(170.0,597.0){\rule[-0.200pt]{2.409pt}{0.400pt}}
\put(1429.0,597.0){\rule[-0.200pt]{2.409pt}{0.400pt}}
\put(170.0,597.0){\rule[-0.200pt]{2.409pt}{0.400pt}}
\put(1429.0,597.0){\rule[-0.200pt]{2.409pt}{0.400pt}}
\put(170.0,597.0){\rule[-0.200pt]{2.409pt}{0.400pt}}
\put(1429.0,597.0){\rule[-0.200pt]{2.409pt}{0.400pt}}
\put(170.0,597.0){\rule[-0.200pt]{2.409pt}{0.400pt}}
\put(1429.0,597.0){\rule[-0.200pt]{2.409pt}{0.400pt}}
\put(170.0,597.0){\rule[-0.200pt]{2.409pt}{0.400pt}}
\put(1429.0,597.0){\rule[-0.200pt]{2.409pt}{0.400pt}}
\put(170.0,597.0){\rule[-0.200pt]{2.409pt}{0.400pt}}
\put(1429.0,597.0){\rule[-0.200pt]{2.409pt}{0.400pt}}
\put(170.0,598.0){\rule[-0.200pt]{2.409pt}{0.400pt}}
\put(1429.0,598.0){\rule[-0.200pt]{2.409pt}{0.400pt}}
\put(170.0,598.0){\rule[-0.200pt]{2.409pt}{0.400pt}}
\put(1429.0,598.0){\rule[-0.200pt]{2.409pt}{0.400pt}}
\put(170.0,598.0){\rule[-0.200pt]{2.409pt}{0.400pt}}
\put(1429.0,598.0){\rule[-0.200pt]{2.409pt}{0.400pt}}
\put(170.0,598.0){\rule[-0.200pt]{2.409pt}{0.400pt}}
\put(1429.0,598.0){\rule[-0.200pt]{2.409pt}{0.400pt}}
\put(170.0,598.0){\rule[-0.200pt]{2.409pt}{0.400pt}}
\put(1429.0,598.0){\rule[-0.200pt]{2.409pt}{0.400pt}}
\put(170.0,598.0){\rule[-0.200pt]{2.409pt}{0.400pt}}
\put(1429.0,598.0){\rule[-0.200pt]{2.409pt}{0.400pt}}
\put(170.0,598.0){\rule[-0.200pt]{2.409pt}{0.400pt}}
\put(1429.0,598.0){\rule[-0.200pt]{2.409pt}{0.400pt}}
\put(170.0,598.0){\rule[-0.200pt]{2.409pt}{0.400pt}}
\put(1429.0,598.0){\rule[-0.200pt]{2.409pt}{0.400pt}}
\put(170.0,598.0){\rule[-0.200pt]{2.409pt}{0.400pt}}
\put(1429.0,598.0){\rule[-0.200pt]{2.409pt}{0.400pt}}
\put(170.0,598.0){\rule[-0.200pt]{2.409pt}{0.400pt}}
\put(1429.0,598.0){\rule[-0.200pt]{2.409pt}{0.400pt}}
\put(170.0,598.0){\rule[-0.200pt]{2.409pt}{0.400pt}}
\put(1429.0,598.0){\rule[-0.200pt]{2.409pt}{0.400pt}}
\put(170.0,598.0){\rule[-0.200pt]{2.409pt}{0.400pt}}
\put(1429.0,598.0){\rule[-0.200pt]{2.409pt}{0.400pt}}
\put(170.0,598.0){\rule[-0.200pt]{2.409pt}{0.400pt}}
\put(1429.0,598.0){\rule[-0.200pt]{2.409pt}{0.400pt}}
\put(170.0,598.0){\rule[-0.200pt]{2.409pt}{0.400pt}}
\put(1429.0,598.0){\rule[-0.200pt]{2.409pt}{0.400pt}}
\put(170.0,598.0){\rule[-0.200pt]{2.409pt}{0.400pt}}
\put(1429.0,598.0){\rule[-0.200pt]{2.409pt}{0.400pt}}
\put(170.0,598.0){\rule[-0.200pt]{2.409pt}{0.400pt}}
\put(1429.0,598.0){\rule[-0.200pt]{2.409pt}{0.400pt}}
\put(170.0,598.0){\rule[-0.200pt]{2.409pt}{0.400pt}}
\put(1429.0,598.0){\rule[-0.200pt]{2.409pt}{0.400pt}}
\put(170.0,598.0){\rule[-0.200pt]{2.409pt}{0.400pt}}
\put(1429.0,598.0){\rule[-0.200pt]{2.409pt}{0.400pt}}
\put(170.0,598.0){\rule[-0.200pt]{2.409pt}{0.400pt}}
\put(1429.0,598.0){\rule[-0.200pt]{2.409pt}{0.400pt}}
\put(170.0,598.0){\rule[-0.200pt]{2.409pt}{0.400pt}}
\put(1429.0,598.0){\rule[-0.200pt]{2.409pt}{0.400pt}}
\put(170.0,598.0){\rule[-0.200pt]{2.409pt}{0.400pt}}
\put(1429.0,598.0){\rule[-0.200pt]{2.409pt}{0.400pt}}
\put(170.0,598.0){\rule[-0.200pt]{2.409pt}{0.400pt}}
\put(1429.0,598.0){\rule[-0.200pt]{2.409pt}{0.400pt}}
\put(170.0,598.0){\rule[-0.200pt]{2.409pt}{0.400pt}}
\put(1429.0,598.0){\rule[-0.200pt]{2.409pt}{0.400pt}}
\put(170.0,598.0){\rule[-0.200pt]{2.409pt}{0.400pt}}
\put(1429.0,598.0){\rule[-0.200pt]{2.409pt}{0.400pt}}
\put(170.0,598.0){\rule[-0.200pt]{2.409pt}{0.400pt}}
\put(1429.0,598.0){\rule[-0.200pt]{2.409pt}{0.400pt}}
\put(170.0,599.0){\rule[-0.200pt]{2.409pt}{0.400pt}}
\put(1429.0,599.0){\rule[-0.200pt]{2.409pt}{0.400pt}}
\put(170.0,599.0){\rule[-0.200pt]{2.409pt}{0.400pt}}
\put(1429.0,599.0){\rule[-0.200pt]{2.409pt}{0.400pt}}
\put(170.0,599.0){\rule[-0.200pt]{2.409pt}{0.400pt}}
\put(1429.0,599.0){\rule[-0.200pt]{2.409pt}{0.400pt}}
\put(170.0,599.0){\rule[-0.200pt]{2.409pt}{0.400pt}}
\put(1429.0,599.0){\rule[-0.200pt]{2.409pt}{0.400pt}}
\put(170.0,599.0){\rule[-0.200pt]{2.409pt}{0.400pt}}
\put(1429.0,599.0){\rule[-0.200pt]{2.409pt}{0.400pt}}
\put(170.0,599.0){\rule[-0.200pt]{2.409pt}{0.400pt}}
\put(1429.0,599.0){\rule[-0.200pt]{2.409pt}{0.400pt}}
\put(170.0,599.0){\rule[-0.200pt]{2.409pt}{0.400pt}}
\put(1429.0,599.0){\rule[-0.200pt]{2.409pt}{0.400pt}}
\put(170.0,599.0){\rule[-0.200pt]{2.409pt}{0.400pt}}
\put(1429.0,599.0){\rule[-0.200pt]{2.409pt}{0.400pt}}
\put(170.0,599.0){\rule[-0.200pt]{2.409pt}{0.400pt}}
\put(1429.0,599.0){\rule[-0.200pt]{2.409pt}{0.400pt}}
\put(170.0,599.0){\rule[-0.200pt]{2.409pt}{0.400pt}}
\put(1429.0,599.0){\rule[-0.200pt]{2.409pt}{0.400pt}}
\put(170.0,599.0){\rule[-0.200pt]{2.409pt}{0.400pt}}
\put(1429.0,599.0){\rule[-0.200pt]{2.409pt}{0.400pt}}
\put(170.0,599.0){\rule[-0.200pt]{2.409pt}{0.400pt}}
\put(1429.0,599.0){\rule[-0.200pt]{2.409pt}{0.400pt}}
\put(170.0,599.0){\rule[-0.200pt]{2.409pt}{0.400pt}}
\put(1429.0,599.0){\rule[-0.200pt]{2.409pt}{0.400pt}}
\put(170.0,599.0){\rule[-0.200pt]{2.409pt}{0.400pt}}
\put(1429.0,599.0){\rule[-0.200pt]{2.409pt}{0.400pt}}
\put(170.0,599.0){\rule[-0.200pt]{2.409pt}{0.400pt}}
\put(1429.0,599.0){\rule[-0.200pt]{2.409pt}{0.400pt}}
\put(170.0,599.0){\rule[-0.200pt]{2.409pt}{0.400pt}}
\put(1429.0,599.0){\rule[-0.200pt]{2.409pt}{0.400pt}}
\put(170.0,599.0){\rule[-0.200pt]{2.409pt}{0.400pt}}
\put(1429.0,599.0){\rule[-0.200pt]{2.409pt}{0.400pt}}
\put(170.0,599.0){\rule[-0.200pt]{2.409pt}{0.400pt}}
\put(1429.0,599.0){\rule[-0.200pt]{2.409pt}{0.400pt}}
\put(170.0,599.0){\rule[-0.200pt]{2.409pt}{0.400pt}}
\put(1429.0,599.0){\rule[-0.200pt]{2.409pt}{0.400pt}}
\put(170.0,599.0){\rule[-0.200pt]{2.409pt}{0.400pt}}
\put(1429.0,599.0){\rule[-0.200pt]{2.409pt}{0.400pt}}
\put(170.0,599.0){\rule[-0.200pt]{2.409pt}{0.400pt}}
\put(1429.0,599.0){\rule[-0.200pt]{2.409pt}{0.400pt}}
\put(170.0,599.0){\rule[-0.200pt]{2.409pt}{0.400pt}}
\put(1429.0,599.0){\rule[-0.200pt]{2.409pt}{0.400pt}}
\put(170.0,599.0){\rule[-0.200pt]{2.409pt}{0.400pt}}
\put(1429.0,599.0){\rule[-0.200pt]{2.409pt}{0.400pt}}
\put(170.0,599.0){\rule[-0.200pt]{2.409pt}{0.400pt}}
\put(1429.0,599.0){\rule[-0.200pt]{2.409pt}{0.400pt}}
\put(170.0,599.0){\rule[-0.200pt]{2.409pt}{0.400pt}}
\put(1429.0,599.0){\rule[-0.200pt]{2.409pt}{0.400pt}}
\put(170.0,599.0){\rule[-0.200pt]{2.409pt}{0.400pt}}
\put(1429.0,599.0){\rule[-0.200pt]{2.409pt}{0.400pt}}
\put(170.0,600.0){\rule[-0.200pt]{2.409pt}{0.400pt}}
\put(1429.0,600.0){\rule[-0.200pt]{2.409pt}{0.400pt}}
\put(170.0,600.0){\rule[-0.200pt]{2.409pt}{0.400pt}}
\put(1429.0,600.0){\rule[-0.200pt]{2.409pt}{0.400pt}}
\put(170.0,600.0){\rule[-0.200pt]{2.409pt}{0.400pt}}
\put(1429.0,600.0){\rule[-0.200pt]{2.409pt}{0.400pt}}
\put(170.0,600.0){\rule[-0.200pt]{2.409pt}{0.400pt}}
\put(1429.0,600.0){\rule[-0.200pt]{2.409pt}{0.400pt}}
\put(170.0,600.0){\rule[-0.200pt]{2.409pt}{0.400pt}}
\put(1429.0,600.0){\rule[-0.200pt]{2.409pt}{0.400pt}}
\put(170.0,600.0){\rule[-0.200pt]{2.409pt}{0.400pt}}
\put(1429.0,600.0){\rule[-0.200pt]{2.409pt}{0.400pt}}
\put(170.0,600.0){\rule[-0.200pt]{2.409pt}{0.400pt}}
\put(1429.0,600.0){\rule[-0.200pt]{2.409pt}{0.400pt}}
\put(170.0,600.0){\rule[-0.200pt]{2.409pt}{0.400pt}}
\put(1429.0,600.0){\rule[-0.200pt]{2.409pt}{0.400pt}}
\put(170.0,600.0){\rule[-0.200pt]{2.409pt}{0.400pt}}
\put(1429.0,600.0){\rule[-0.200pt]{2.409pt}{0.400pt}}
\put(170.0,600.0){\rule[-0.200pt]{2.409pt}{0.400pt}}
\put(1429.0,600.0){\rule[-0.200pt]{2.409pt}{0.400pt}}
\put(170.0,600.0){\rule[-0.200pt]{2.409pt}{0.400pt}}
\put(1429.0,600.0){\rule[-0.200pt]{2.409pt}{0.400pt}}
\put(170.0,600.0){\rule[-0.200pt]{2.409pt}{0.400pt}}
\put(1429.0,600.0){\rule[-0.200pt]{2.409pt}{0.400pt}}
\put(170.0,600.0){\rule[-0.200pt]{2.409pt}{0.400pt}}
\put(1429.0,600.0){\rule[-0.200pt]{2.409pt}{0.400pt}}
\put(170.0,600.0){\rule[-0.200pt]{4.818pt}{0.400pt}}
\put(150,600){\makebox(0,0)[r]{ 1e+06}}
\put(1419.0,600.0){\rule[-0.200pt]{4.818pt}{0.400pt}}
\put(170.0,626.0){\rule[-0.200pt]{2.409pt}{0.400pt}}
\put(1429.0,626.0){\rule[-0.200pt]{2.409pt}{0.400pt}}
\put(170.0,641.0){\rule[-0.200pt]{2.409pt}{0.400pt}}
\put(1429.0,641.0){\rule[-0.200pt]{2.409pt}{0.400pt}}
\put(170.0,652.0){\rule[-0.200pt]{2.409pt}{0.400pt}}
\put(1429.0,652.0){\rule[-0.200pt]{2.409pt}{0.400pt}}
\put(170.0,660.0){\rule[-0.200pt]{2.409pt}{0.400pt}}
\put(1429.0,660.0){\rule[-0.200pt]{2.409pt}{0.400pt}}
\put(170.0,667.0){\rule[-0.200pt]{2.409pt}{0.400pt}}
\put(1429.0,667.0){\rule[-0.200pt]{2.409pt}{0.400pt}}
\put(170.0,673.0){\rule[-0.200pt]{2.409pt}{0.400pt}}
\put(1429.0,673.0){\rule[-0.200pt]{2.409pt}{0.400pt}}
\put(170.0,678.0){\rule[-0.200pt]{2.409pt}{0.400pt}}
\put(1429.0,678.0){\rule[-0.200pt]{2.409pt}{0.400pt}}
\put(170.0,682.0){\rule[-0.200pt]{2.409pt}{0.400pt}}
\put(1429.0,682.0){\rule[-0.200pt]{2.409pt}{0.400pt}}
\put(170.0,686.0){\rule[-0.200pt]{2.409pt}{0.400pt}}
\put(1429.0,686.0){\rule[-0.200pt]{2.409pt}{0.400pt}}
\put(170.0,690.0){\rule[-0.200pt]{2.409pt}{0.400pt}}
\put(1429.0,690.0){\rule[-0.200pt]{2.409pt}{0.400pt}}
\put(170.0,693.0){\rule[-0.200pt]{2.409pt}{0.400pt}}
\put(1429.0,693.0){\rule[-0.200pt]{2.409pt}{0.400pt}}
\put(170.0,696.0){\rule[-0.200pt]{2.409pt}{0.400pt}}
\put(1429.0,696.0){\rule[-0.200pt]{2.409pt}{0.400pt}}
\put(170.0,699.0){\rule[-0.200pt]{2.409pt}{0.400pt}}
\put(1429.0,699.0){\rule[-0.200pt]{2.409pt}{0.400pt}}
\put(170.0,702.0){\rule[-0.200pt]{2.409pt}{0.400pt}}
\put(1429.0,702.0){\rule[-0.200pt]{2.409pt}{0.400pt}}
\put(170.0,704.0){\rule[-0.200pt]{2.409pt}{0.400pt}}
\put(1429.0,704.0){\rule[-0.200pt]{2.409pt}{0.400pt}}
\put(170.0,706.0){\rule[-0.200pt]{2.409pt}{0.400pt}}
\put(1429.0,706.0){\rule[-0.200pt]{2.409pt}{0.400pt}}
\put(170.0,708.0){\rule[-0.200pt]{2.409pt}{0.400pt}}
\put(1429.0,708.0){\rule[-0.200pt]{2.409pt}{0.400pt}}
\put(170.0,710.0){\rule[-0.200pt]{2.409pt}{0.400pt}}
\put(1429.0,710.0){\rule[-0.200pt]{2.409pt}{0.400pt}}
\put(170.0,712.0){\rule[-0.200pt]{2.409pt}{0.400pt}}
\put(1429.0,712.0){\rule[-0.200pt]{2.409pt}{0.400pt}}
\put(170.0,714.0){\rule[-0.200pt]{2.409pt}{0.400pt}}
\put(1429.0,714.0){\rule[-0.200pt]{2.409pt}{0.400pt}}
\put(170.0,716.0){\rule[-0.200pt]{2.409pt}{0.400pt}}
\put(1429.0,716.0){\rule[-0.200pt]{2.409pt}{0.400pt}}
\put(170.0,718.0){\rule[-0.200pt]{2.409pt}{0.400pt}}
\put(1429.0,718.0){\rule[-0.200pt]{2.409pt}{0.400pt}}
\put(170.0,719.0){\rule[-0.200pt]{2.409pt}{0.400pt}}
\put(1429.0,719.0){\rule[-0.200pt]{2.409pt}{0.400pt}}
\put(170.0,721.0){\rule[-0.200pt]{2.409pt}{0.400pt}}
\put(1429.0,721.0){\rule[-0.200pt]{2.409pt}{0.400pt}}
\put(170.0,722.0){\rule[-0.200pt]{2.409pt}{0.400pt}}
\put(1429.0,722.0){\rule[-0.200pt]{2.409pt}{0.400pt}}
\put(170.0,724.0){\rule[-0.200pt]{2.409pt}{0.400pt}}
\put(1429.0,724.0){\rule[-0.200pt]{2.409pt}{0.400pt}}
\put(170.0,725.0){\rule[-0.200pt]{2.409pt}{0.400pt}}
\put(1429.0,725.0){\rule[-0.200pt]{2.409pt}{0.400pt}}
\put(170.0,726.0){\rule[-0.200pt]{2.409pt}{0.400pt}}
\put(1429.0,726.0){\rule[-0.200pt]{2.409pt}{0.400pt}}
\put(170.0,728.0){\rule[-0.200pt]{2.409pt}{0.400pt}}
\put(1429.0,728.0){\rule[-0.200pt]{2.409pt}{0.400pt}}
\put(170.0,729.0){\rule[-0.200pt]{2.409pt}{0.400pt}}
\put(1429.0,729.0){\rule[-0.200pt]{2.409pt}{0.400pt}}
\put(170.0,730.0){\rule[-0.200pt]{2.409pt}{0.400pt}}
\put(1429.0,730.0){\rule[-0.200pt]{2.409pt}{0.400pt}}
\put(170.0,731.0){\rule[-0.200pt]{2.409pt}{0.400pt}}
\put(1429.0,731.0){\rule[-0.200pt]{2.409pt}{0.400pt}}
\put(170.0,732.0){\rule[-0.200pt]{2.409pt}{0.400pt}}
\put(1429.0,732.0){\rule[-0.200pt]{2.409pt}{0.400pt}}
\put(170.0,733.0){\rule[-0.200pt]{2.409pt}{0.400pt}}
\put(1429.0,733.0){\rule[-0.200pt]{2.409pt}{0.400pt}}
\put(170.0,734.0){\rule[-0.200pt]{2.409pt}{0.400pt}}
\put(1429.0,734.0){\rule[-0.200pt]{2.409pt}{0.400pt}}
\put(170.0,735.0){\rule[-0.200pt]{2.409pt}{0.400pt}}
\put(1429.0,735.0){\rule[-0.200pt]{2.409pt}{0.400pt}}
\put(170.0,736.0){\rule[-0.200pt]{2.409pt}{0.400pt}}
\put(1429.0,736.0){\rule[-0.200pt]{2.409pt}{0.400pt}}
\put(170.0,737.0){\rule[-0.200pt]{2.409pt}{0.400pt}}
\put(1429.0,737.0){\rule[-0.200pt]{2.409pt}{0.400pt}}
\put(170.0,738.0){\rule[-0.200pt]{2.409pt}{0.400pt}}
\put(1429.0,738.0){\rule[-0.200pt]{2.409pt}{0.400pt}}
\put(170.0,739.0){\rule[-0.200pt]{2.409pt}{0.400pt}}
\put(1429.0,739.0){\rule[-0.200pt]{2.409pt}{0.400pt}}
\put(170.0,740.0){\rule[-0.200pt]{2.409pt}{0.400pt}}
\put(1429.0,740.0){\rule[-0.200pt]{2.409pt}{0.400pt}}
\put(170.0,741.0){\rule[-0.200pt]{2.409pt}{0.400pt}}
\put(1429.0,741.0){\rule[-0.200pt]{2.409pt}{0.400pt}}
\put(170.0,742.0){\rule[-0.200pt]{2.409pt}{0.400pt}}
\put(1429.0,742.0){\rule[-0.200pt]{2.409pt}{0.400pt}}
\put(170.0,743.0){\rule[-0.200pt]{2.409pt}{0.400pt}}
\put(1429.0,743.0){\rule[-0.200pt]{2.409pt}{0.400pt}}
\put(170.0,744.0){\rule[-0.200pt]{2.409pt}{0.400pt}}
\put(1429.0,744.0){\rule[-0.200pt]{2.409pt}{0.400pt}}
\put(170.0,744.0){\rule[-0.200pt]{2.409pt}{0.400pt}}
\put(1429.0,744.0){\rule[-0.200pt]{2.409pt}{0.400pt}}
\put(170.0,745.0){\rule[-0.200pt]{2.409pt}{0.400pt}}
\put(1429.0,745.0){\rule[-0.200pt]{2.409pt}{0.400pt}}
\put(170.0,746.0){\rule[-0.200pt]{2.409pt}{0.400pt}}
\put(1429.0,746.0){\rule[-0.200pt]{2.409pt}{0.400pt}}
\put(170.0,747.0){\rule[-0.200pt]{2.409pt}{0.400pt}}
\put(1429.0,747.0){\rule[-0.200pt]{2.409pt}{0.400pt}}
\put(170.0,747.0){\rule[-0.200pt]{2.409pt}{0.400pt}}
\put(1429.0,747.0){\rule[-0.200pt]{2.409pt}{0.400pt}}
\put(170.0,748.0){\rule[-0.200pt]{2.409pt}{0.400pt}}
\put(1429.0,748.0){\rule[-0.200pt]{2.409pt}{0.400pt}}
\put(170.0,749.0){\rule[-0.200pt]{2.409pt}{0.400pt}}
\put(1429.0,749.0){\rule[-0.200pt]{2.409pt}{0.400pt}}
\put(170.0,750.0){\rule[-0.200pt]{2.409pt}{0.400pt}}
\put(1429.0,750.0){\rule[-0.200pt]{2.409pt}{0.400pt}}
\put(170.0,750.0){\rule[-0.200pt]{2.409pt}{0.400pt}}
\put(1429.0,750.0){\rule[-0.200pt]{2.409pt}{0.400pt}}
\put(170.0,751.0){\rule[-0.200pt]{2.409pt}{0.400pt}}
\put(1429.0,751.0){\rule[-0.200pt]{2.409pt}{0.400pt}}
\put(170.0,752.0){\rule[-0.200pt]{2.409pt}{0.400pt}}
\put(1429.0,752.0){\rule[-0.200pt]{2.409pt}{0.400pt}}
\put(170.0,752.0){\rule[-0.200pt]{2.409pt}{0.400pt}}
\put(1429.0,752.0){\rule[-0.200pt]{2.409pt}{0.400pt}}
\put(170.0,753.0){\rule[-0.200pt]{2.409pt}{0.400pt}}
\put(1429.0,753.0){\rule[-0.200pt]{2.409pt}{0.400pt}}
\put(170.0,754.0){\rule[-0.200pt]{2.409pt}{0.400pt}}
\put(1429.0,754.0){\rule[-0.200pt]{2.409pt}{0.400pt}}
\put(170.0,754.0){\rule[-0.200pt]{2.409pt}{0.400pt}}
\put(1429.0,754.0){\rule[-0.200pt]{2.409pt}{0.400pt}}
\put(170.0,755.0){\rule[-0.200pt]{2.409pt}{0.400pt}}
\put(1429.0,755.0){\rule[-0.200pt]{2.409pt}{0.400pt}}
\put(170.0,755.0){\rule[-0.200pt]{2.409pt}{0.400pt}}
\put(1429.0,755.0){\rule[-0.200pt]{2.409pt}{0.400pt}}
\put(170.0,756.0){\rule[-0.200pt]{2.409pt}{0.400pt}}
\put(1429.0,756.0){\rule[-0.200pt]{2.409pt}{0.400pt}}
\put(170.0,757.0){\rule[-0.200pt]{2.409pt}{0.400pt}}
\put(1429.0,757.0){\rule[-0.200pt]{2.409pt}{0.400pt}}
\put(170.0,757.0){\rule[-0.200pt]{2.409pt}{0.400pt}}
\put(1429.0,757.0){\rule[-0.200pt]{2.409pt}{0.400pt}}
\put(170.0,758.0){\rule[-0.200pt]{2.409pt}{0.400pt}}
\put(1429.0,758.0){\rule[-0.200pt]{2.409pt}{0.400pt}}
\put(170.0,758.0){\rule[-0.200pt]{2.409pt}{0.400pt}}
\put(1429.0,758.0){\rule[-0.200pt]{2.409pt}{0.400pt}}
\put(170.0,759.0){\rule[-0.200pt]{2.409pt}{0.400pt}}
\put(1429.0,759.0){\rule[-0.200pt]{2.409pt}{0.400pt}}
\put(170.0,759.0){\rule[-0.200pt]{2.409pt}{0.400pt}}
\put(1429.0,759.0){\rule[-0.200pt]{2.409pt}{0.400pt}}
\put(170.0,760.0){\rule[-0.200pt]{2.409pt}{0.400pt}}
\put(1429.0,760.0){\rule[-0.200pt]{2.409pt}{0.400pt}}
\put(170.0,760.0){\rule[-0.200pt]{2.409pt}{0.400pt}}
\put(1429.0,760.0){\rule[-0.200pt]{2.409pt}{0.400pt}}
\put(170.0,761.0){\rule[-0.200pt]{2.409pt}{0.400pt}}
\put(1429.0,761.0){\rule[-0.200pt]{2.409pt}{0.400pt}}
\put(170.0,761.0){\rule[-0.200pt]{2.409pt}{0.400pt}}
\put(1429.0,761.0){\rule[-0.200pt]{2.409pt}{0.400pt}}
\put(170.0,762.0){\rule[-0.200pt]{2.409pt}{0.400pt}}
\put(1429.0,762.0){\rule[-0.200pt]{2.409pt}{0.400pt}}
\put(170.0,762.0){\rule[-0.200pt]{2.409pt}{0.400pt}}
\put(1429.0,762.0){\rule[-0.200pt]{2.409pt}{0.400pt}}
\put(170.0,763.0){\rule[-0.200pt]{2.409pt}{0.400pt}}
\put(1429.0,763.0){\rule[-0.200pt]{2.409pt}{0.400pt}}
\put(170.0,763.0){\rule[-0.200pt]{2.409pt}{0.400pt}}
\put(1429.0,763.0){\rule[-0.200pt]{2.409pt}{0.400pt}}
\put(170.0,764.0){\rule[-0.200pt]{2.409pt}{0.400pt}}
\put(1429.0,764.0){\rule[-0.200pt]{2.409pt}{0.400pt}}
\put(170.0,764.0){\rule[-0.200pt]{2.409pt}{0.400pt}}
\put(1429.0,764.0){\rule[-0.200pt]{2.409pt}{0.400pt}}
\put(170.0,765.0){\rule[-0.200pt]{2.409pt}{0.400pt}}
\put(1429.0,765.0){\rule[-0.200pt]{2.409pt}{0.400pt}}
\put(170.0,765.0){\rule[-0.200pt]{2.409pt}{0.400pt}}
\put(1429.0,765.0){\rule[-0.200pt]{2.409pt}{0.400pt}}
\put(170.0,766.0){\rule[-0.200pt]{2.409pt}{0.400pt}}
\put(1429.0,766.0){\rule[-0.200pt]{2.409pt}{0.400pt}}
\put(170.0,766.0){\rule[-0.200pt]{2.409pt}{0.400pt}}
\put(1429.0,766.0){\rule[-0.200pt]{2.409pt}{0.400pt}}
\put(170.0,767.0){\rule[-0.200pt]{2.409pt}{0.400pt}}
\put(1429.0,767.0){\rule[-0.200pt]{2.409pt}{0.400pt}}
\put(170.0,767.0){\rule[-0.200pt]{2.409pt}{0.400pt}}
\put(1429.0,767.0){\rule[-0.200pt]{2.409pt}{0.400pt}}
\put(170.0,767.0){\rule[-0.200pt]{2.409pt}{0.400pt}}
\put(1429.0,767.0){\rule[-0.200pt]{2.409pt}{0.400pt}}
\put(170.0,768.0){\rule[-0.200pt]{2.409pt}{0.400pt}}
\put(1429.0,768.0){\rule[-0.200pt]{2.409pt}{0.400pt}}
\put(170.0,768.0){\rule[-0.200pt]{2.409pt}{0.400pt}}
\put(1429.0,768.0){\rule[-0.200pt]{2.409pt}{0.400pt}}
\put(170.0,769.0){\rule[-0.200pt]{2.409pt}{0.400pt}}
\put(1429.0,769.0){\rule[-0.200pt]{2.409pt}{0.400pt}}
\put(170.0,769.0){\rule[-0.200pt]{2.409pt}{0.400pt}}
\put(1429.0,769.0){\rule[-0.200pt]{2.409pt}{0.400pt}}
\put(170.0,770.0){\rule[-0.200pt]{2.409pt}{0.400pt}}
\put(1429.0,770.0){\rule[-0.200pt]{2.409pt}{0.400pt}}
\put(170.0,770.0){\rule[-0.200pt]{2.409pt}{0.400pt}}
\put(1429.0,770.0){\rule[-0.200pt]{2.409pt}{0.400pt}}
\put(170.0,770.0){\rule[-0.200pt]{2.409pt}{0.400pt}}
\put(1429.0,770.0){\rule[-0.200pt]{2.409pt}{0.400pt}}
\put(170.0,771.0){\rule[-0.200pt]{2.409pt}{0.400pt}}
\put(1429.0,771.0){\rule[-0.200pt]{2.409pt}{0.400pt}}
\put(170.0,771.0){\rule[-0.200pt]{2.409pt}{0.400pt}}
\put(1429.0,771.0){\rule[-0.200pt]{2.409pt}{0.400pt}}
\put(170.0,772.0){\rule[-0.200pt]{2.409pt}{0.400pt}}
\put(1429.0,772.0){\rule[-0.200pt]{2.409pt}{0.400pt}}
\put(170.0,772.0){\rule[-0.200pt]{2.409pt}{0.400pt}}
\put(1429.0,772.0){\rule[-0.200pt]{2.409pt}{0.400pt}}
\put(170.0,772.0){\rule[-0.200pt]{2.409pt}{0.400pt}}
\put(1429.0,772.0){\rule[-0.200pt]{2.409pt}{0.400pt}}
\put(170.0,773.0){\rule[-0.200pt]{2.409pt}{0.400pt}}
\put(1429.0,773.0){\rule[-0.200pt]{2.409pt}{0.400pt}}
\put(170.0,773.0){\rule[-0.200pt]{2.409pt}{0.400pt}}
\put(1429.0,773.0){\rule[-0.200pt]{2.409pt}{0.400pt}}
\put(170.0,773.0){\rule[-0.200pt]{2.409pt}{0.400pt}}
\put(1429.0,773.0){\rule[-0.200pt]{2.409pt}{0.400pt}}
\put(170.0,774.0){\rule[-0.200pt]{2.409pt}{0.400pt}}
\put(1429.0,774.0){\rule[-0.200pt]{2.409pt}{0.400pt}}
\put(170.0,774.0){\rule[-0.200pt]{2.409pt}{0.400pt}}
\put(1429.0,774.0){\rule[-0.200pt]{2.409pt}{0.400pt}}
\put(170.0,774.0){\rule[-0.200pt]{2.409pt}{0.400pt}}
\put(1429.0,774.0){\rule[-0.200pt]{2.409pt}{0.400pt}}
\put(170.0,775.0){\rule[-0.200pt]{2.409pt}{0.400pt}}
\put(1429.0,775.0){\rule[-0.200pt]{2.409pt}{0.400pt}}
\put(170.0,775.0){\rule[-0.200pt]{2.409pt}{0.400pt}}
\put(1429.0,775.0){\rule[-0.200pt]{2.409pt}{0.400pt}}
\put(170.0,776.0){\rule[-0.200pt]{2.409pt}{0.400pt}}
\put(1429.0,776.0){\rule[-0.200pt]{2.409pt}{0.400pt}}
\put(170.0,776.0){\rule[-0.200pt]{2.409pt}{0.400pt}}
\put(1429.0,776.0){\rule[-0.200pt]{2.409pt}{0.400pt}}
\put(170.0,776.0){\rule[-0.200pt]{2.409pt}{0.400pt}}
\put(1429.0,776.0){\rule[-0.200pt]{2.409pt}{0.400pt}}
\put(170.0,777.0){\rule[-0.200pt]{2.409pt}{0.400pt}}
\put(1429.0,777.0){\rule[-0.200pt]{2.409pt}{0.400pt}}
\put(170.0,777.0){\rule[-0.200pt]{2.409pt}{0.400pt}}
\put(1429.0,777.0){\rule[-0.200pt]{2.409pt}{0.400pt}}
\put(170.0,777.0){\rule[-0.200pt]{2.409pt}{0.400pt}}
\put(1429.0,777.0){\rule[-0.200pt]{2.409pt}{0.400pt}}
\put(170.0,778.0){\rule[-0.200pt]{2.409pt}{0.400pt}}
\put(1429.0,778.0){\rule[-0.200pt]{2.409pt}{0.400pt}}
\put(170.0,778.0){\rule[-0.200pt]{2.409pt}{0.400pt}}
\put(1429.0,778.0){\rule[-0.200pt]{2.409pt}{0.400pt}}
\put(170.0,778.0){\rule[-0.200pt]{2.409pt}{0.400pt}}
\put(1429.0,778.0){\rule[-0.200pt]{2.409pt}{0.400pt}}
\put(170.0,779.0){\rule[-0.200pt]{2.409pt}{0.400pt}}
\put(1429.0,779.0){\rule[-0.200pt]{2.409pt}{0.400pt}}
\put(170.0,779.0){\rule[-0.200pt]{2.409pt}{0.400pt}}
\put(1429.0,779.0){\rule[-0.200pt]{2.409pt}{0.400pt}}
\put(170.0,779.0){\rule[-0.200pt]{2.409pt}{0.400pt}}
\put(1429.0,779.0){\rule[-0.200pt]{2.409pt}{0.400pt}}
\put(170.0,780.0){\rule[-0.200pt]{2.409pt}{0.400pt}}
\put(1429.0,780.0){\rule[-0.200pt]{2.409pt}{0.400pt}}
\put(170.0,780.0){\rule[-0.200pt]{2.409pt}{0.400pt}}
\put(1429.0,780.0){\rule[-0.200pt]{2.409pt}{0.400pt}}
\put(170.0,780.0){\rule[-0.200pt]{2.409pt}{0.400pt}}
\put(1429.0,780.0){\rule[-0.200pt]{2.409pt}{0.400pt}}
\put(170.0,780.0){\rule[-0.200pt]{2.409pt}{0.400pt}}
\put(1429.0,780.0){\rule[-0.200pt]{2.409pt}{0.400pt}}
\put(170.0,781.0){\rule[-0.200pt]{2.409pt}{0.400pt}}
\put(1429.0,781.0){\rule[-0.200pt]{2.409pt}{0.400pt}}
\put(170.0,781.0){\rule[-0.200pt]{2.409pt}{0.400pt}}
\put(1429.0,781.0){\rule[-0.200pt]{2.409pt}{0.400pt}}
\put(170.0,781.0){\rule[-0.200pt]{2.409pt}{0.400pt}}
\put(1429.0,781.0){\rule[-0.200pt]{2.409pt}{0.400pt}}
\put(170.0,782.0){\rule[-0.200pt]{2.409pt}{0.400pt}}
\put(1429.0,782.0){\rule[-0.200pt]{2.409pt}{0.400pt}}
\put(170.0,782.0){\rule[-0.200pt]{2.409pt}{0.400pt}}
\put(1429.0,782.0){\rule[-0.200pt]{2.409pt}{0.400pt}}
\put(170.0,782.0){\rule[-0.200pt]{2.409pt}{0.400pt}}
\put(1429.0,782.0){\rule[-0.200pt]{2.409pt}{0.400pt}}
\put(170.0,783.0){\rule[-0.200pt]{2.409pt}{0.400pt}}
\put(1429.0,783.0){\rule[-0.200pt]{2.409pt}{0.400pt}}
\put(170.0,783.0){\rule[-0.200pt]{2.409pt}{0.400pt}}
\put(1429.0,783.0){\rule[-0.200pt]{2.409pt}{0.400pt}}
\put(170.0,783.0){\rule[-0.200pt]{2.409pt}{0.400pt}}
\put(1429.0,783.0){\rule[-0.200pt]{2.409pt}{0.400pt}}
\put(170.0,783.0){\rule[-0.200pt]{2.409pt}{0.400pt}}
\put(1429.0,783.0){\rule[-0.200pt]{2.409pt}{0.400pt}}
\put(170.0,784.0){\rule[-0.200pt]{2.409pt}{0.400pt}}
\put(1429.0,784.0){\rule[-0.200pt]{2.409pt}{0.400pt}}
\put(170.0,784.0){\rule[-0.200pt]{2.409pt}{0.400pt}}
\put(1429.0,784.0){\rule[-0.200pt]{2.409pt}{0.400pt}}
\put(170.0,784.0){\rule[-0.200pt]{2.409pt}{0.400pt}}
\put(1429.0,784.0){\rule[-0.200pt]{2.409pt}{0.400pt}}
\put(170.0,784.0){\rule[-0.200pt]{2.409pt}{0.400pt}}
\put(1429.0,784.0){\rule[-0.200pt]{2.409pt}{0.400pt}}
\put(170.0,785.0){\rule[-0.200pt]{2.409pt}{0.400pt}}
\put(1429.0,785.0){\rule[-0.200pt]{2.409pt}{0.400pt}}
\put(170.0,785.0){\rule[-0.200pt]{2.409pt}{0.400pt}}
\put(1429.0,785.0){\rule[-0.200pt]{2.409pt}{0.400pt}}
\put(170.0,785.0){\rule[-0.200pt]{2.409pt}{0.400pt}}
\put(1429.0,785.0){\rule[-0.200pt]{2.409pt}{0.400pt}}
\put(170.0,786.0){\rule[-0.200pt]{2.409pt}{0.400pt}}
\put(1429.0,786.0){\rule[-0.200pt]{2.409pt}{0.400pt}}
\put(170.0,786.0){\rule[-0.200pt]{2.409pt}{0.400pt}}
\put(1429.0,786.0){\rule[-0.200pt]{2.409pt}{0.400pt}}
\put(170.0,786.0){\rule[-0.200pt]{2.409pt}{0.400pt}}
\put(1429.0,786.0){\rule[-0.200pt]{2.409pt}{0.400pt}}
\put(170.0,786.0){\rule[-0.200pt]{2.409pt}{0.400pt}}
\put(1429.0,786.0){\rule[-0.200pt]{2.409pt}{0.400pt}}
\put(170.0,787.0){\rule[-0.200pt]{2.409pt}{0.400pt}}
\put(1429.0,787.0){\rule[-0.200pt]{2.409pt}{0.400pt}}
\put(170.0,787.0){\rule[-0.200pt]{2.409pt}{0.400pt}}
\put(1429.0,787.0){\rule[-0.200pt]{2.409pt}{0.400pt}}
\put(170.0,787.0){\rule[-0.200pt]{2.409pt}{0.400pt}}
\put(1429.0,787.0){\rule[-0.200pt]{2.409pt}{0.400pt}}
\put(170.0,787.0){\rule[-0.200pt]{2.409pt}{0.400pt}}
\put(1429.0,787.0){\rule[-0.200pt]{2.409pt}{0.400pt}}
\put(170.0,788.0){\rule[-0.200pt]{2.409pt}{0.400pt}}
\put(1429.0,788.0){\rule[-0.200pt]{2.409pt}{0.400pt}}
\put(170.0,788.0){\rule[-0.200pt]{2.409pt}{0.400pt}}
\put(1429.0,788.0){\rule[-0.200pt]{2.409pt}{0.400pt}}
\put(170.0,788.0){\rule[-0.200pt]{2.409pt}{0.400pt}}
\put(1429.0,788.0){\rule[-0.200pt]{2.409pt}{0.400pt}}
\put(170.0,788.0){\rule[-0.200pt]{2.409pt}{0.400pt}}
\put(1429.0,788.0){\rule[-0.200pt]{2.409pt}{0.400pt}}
\put(170.0,789.0){\rule[-0.200pt]{2.409pt}{0.400pt}}
\put(1429.0,789.0){\rule[-0.200pt]{2.409pt}{0.400pt}}
\put(170.0,789.0){\rule[-0.200pt]{2.409pt}{0.400pt}}
\put(1429.0,789.0){\rule[-0.200pt]{2.409pt}{0.400pt}}
\put(170.0,789.0){\rule[-0.200pt]{2.409pt}{0.400pt}}
\put(1429.0,789.0){\rule[-0.200pt]{2.409pt}{0.400pt}}
\put(170.0,789.0){\rule[-0.200pt]{2.409pt}{0.400pt}}
\put(1429.0,789.0){\rule[-0.200pt]{2.409pt}{0.400pt}}
\put(170.0,790.0){\rule[-0.200pt]{2.409pt}{0.400pt}}
\put(1429.0,790.0){\rule[-0.200pt]{2.409pt}{0.400pt}}
\put(170.0,790.0){\rule[-0.200pt]{2.409pt}{0.400pt}}
\put(1429.0,790.0){\rule[-0.200pt]{2.409pt}{0.400pt}}
\put(170.0,790.0){\rule[-0.200pt]{2.409pt}{0.400pt}}
\put(1429.0,790.0){\rule[-0.200pt]{2.409pt}{0.400pt}}
\put(170.0,790.0){\rule[-0.200pt]{2.409pt}{0.400pt}}
\put(1429.0,790.0){\rule[-0.200pt]{2.409pt}{0.400pt}}
\put(170.0,791.0){\rule[-0.200pt]{2.409pt}{0.400pt}}
\put(1429.0,791.0){\rule[-0.200pt]{2.409pt}{0.400pt}}
\put(170.0,791.0){\rule[-0.200pt]{2.409pt}{0.400pt}}
\put(1429.0,791.0){\rule[-0.200pt]{2.409pt}{0.400pt}}
\put(170.0,791.0){\rule[-0.200pt]{2.409pt}{0.400pt}}
\put(1429.0,791.0){\rule[-0.200pt]{2.409pt}{0.400pt}}
\put(170.0,791.0){\rule[-0.200pt]{2.409pt}{0.400pt}}
\put(1429.0,791.0){\rule[-0.200pt]{2.409pt}{0.400pt}}
\put(170.0,791.0){\rule[-0.200pt]{2.409pt}{0.400pt}}
\put(1429.0,791.0){\rule[-0.200pt]{2.409pt}{0.400pt}}
\put(170.0,792.0){\rule[-0.200pt]{2.409pt}{0.400pt}}
\put(1429.0,792.0){\rule[-0.200pt]{2.409pt}{0.400pt}}
\put(170.0,792.0){\rule[-0.200pt]{2.409pt}{0.400pt}}
\put(1429.0,792.0){\rule[-0.200pt]{2.409pt}{0.400pt}}
\put(170.0,792.0){\rule[-0.200pt]{2.409pt}{0.400pt}}
\put(1429.0,792.0){\rule[-0.200pt]{2.409pt}{0.400pt}}
\put(170.0,792.0){\rule[-0.200pt]{2.409pt}{0.400pt}}
\put(1429.0,792.0){\rule[-0.200pt]{2.409pt}{0.400pt}}
\put(170.0,793.0){\rule[-0.200pt]{2.409pt}{0.400pt}}
\put(1429.0,793.0){\rule[-0.200pt]{2.409pt}{0.400pt}}
\put(170.0,793.0){\rule[-0.200pt]{2.409pt}{0.400pt}}
\put(1429.0,793.0){\rule[-0.200pt]{2.409pt}{0.400pt}}
\put(170.0,793.0){\rule[-0.200pt]{2.409pt}{0.400pt}}
\put(1429.0,793.0){\rule[-0.200pt]{2.409pt}{0.400pt}}
\put(170.0,793.0){\rule[-0.200pt]{2.409pt}{0.400pt}}
\put(1429.0,793.0){\rule[-0.200pt]{2.409pt}{0.400pt}}
\put(170.0,793.0){\rule[-0.200pt]{2.409pt}{0.400pt}}
\put(1429.0,793.0){\rule[-0.200pt]{2.409pt}{0.400pt}}
\put(170.0,794.0){\rule[-0.200pt]{2.409pt}{0.400pt}}
\put(1429.0,794.0){\rule[-0.200pt]{2.409pt}{0.400pt}}
\put(170.0,794.0){\rule[-0.200pt]{2.409pt}{0.400pt}}
\put(1429.0,794.0){\rule[-0.200pt]{2.409pt}{0.400pt}}
\put(170.0,794.0){\rule[-0.200pt]{2.409pt}{0.400pt}}
\put(1429.0,794.0){\rule[-0.200pt]{2.409pt}{0.400pt}}
\put(170.0,794.0){\rule[-0.200pt]{2.409pt}{0.400pt}}
\put(1429.0,794.0){\rule[-0.200pt]{2.409pt}{0.400pt}}
\put(170.0,794.0){\rule[-0.200pt]{2.409pt}{0.400pt}}
\put(1429.0,794.0){\rule[-0.200pt]{2.409pt}{0.400pt}}
\put(170.0,795.0){\rule[-0.200pt]{2.409pt}{0.400pt}}
\put(1429.0,795.0){\rule[-0.200pt]{2.409pt}{0.400pt}}
\put(170.0,795.0){\rule[-0.200pt]{2.409pt}{0.400pt}}
\put(1429.0,795.0){\rule[-0.200pt]{2.409pt}{0.400pt}}
\put(170.0,795.0){\rule[-0.200pt]{2.409pt}{0.400pt}}
\put(1429.0,795.0){\rule[-0.200pt]{2.409pt}{0.400pt}}
\put(170.0,795.0){\rule[-0.200pt]{2.409pt}{0.400pt}}
\put(1429.0,795.0){\rule[-0.200pt]{2.409pt}{0.400pt}}
\put(170.0,796.0){\rule[-0.200pt]{2.409pt}{0.400pt}}
\put(1429.0,796.0){\rule[-0.200pt]{2.409pt}{0.400pt}}
\put(170.0,796.0){\rule[-0.200pt]{2.409pt}{0.400pt}}
\put(1429.0,796.0){\rule[-0.200pt]{2.409pt}{0.400pt}}
\put(170.0,796.0){\rule[-0.200pt]{2.409pt}{0.400pt}}
\put(1429.0,796.0){\rule[-0.200pt]{2.409pt}{0.400pt}}
\put(170.0,796.0){\rule[-0.200pt]{2.409pt}{0.400pt}}
\put(1429.0,796.0){\rule[-0.200pt]{2.409pt}{0.400pt}}
\put(170.0,796.0){\rule[-0.200pt]{2.409pt}{0.400pt}}
\put(1429.0,796.0){\rule[-0.200pt]{2.409pt}{0.400pt}}
\put(170.0,797.0){\rule[-0.200pt]{2.409pt}{0.400pt}}
\put(1429.0,797.0){\rule[-0.200pt]{2.409pt}{0.400pt}}
\put(170.0,797.0){\rule[-0.200pt]{2.409pt}{0.400pt}}
\put(1429.0,797.0){\rule[-0.200pt]{2.409pt}{0.400pt}}
\put(170.0,797.0){\rule[-0.200pt]{2.409pt}{0.400pt}}
\put(1429.0,797.0){\rule[-0.200pt]{2.409pt}{0.400pt}}
\put(170.0,797.0){\rule[-0.200pt]{2.409pt}{0.400pt}}
\put(1429.0,797.0){\rule[-0.200pt]{2.409pt}{0.400pt}}
\put(170.0,797.0){\rule[-0.200pt]{2.409pt}{0.400pt}}
\put(1429.0,797.0){\rule[-0.200pt]{2.409pt}{0.400pt}}
\put(170.0,798.0){\rule[-0.200pt]{2.409pt}{0.400pt}}
\put(1429.0,798.0){\rule[-0.200pt]{2.409pt}{0.400pt}}
\put(170.0,798.0){\rule[-0.200pt]{2.409pt}{0.400pt}}
\put(1429.0,798.0){\rule[-0.200pt]{2.409pt}{0.400pt}}
\put(170.0,798.0){\rule[-0.200pt]{2.409pt}{0.400pt}}
\put(1429.0,798.0){\rule[-0.200pt]{2.409pt}{0.400pt}}
\put(170.0,798.0){\rule[-0.200pt]{2.409pt}{0.400pt}}
\put(1429.0,798.0){\rule[-0.200pt]{2.409pt}{0.400pt}}
\put(170.0,798.0){\rule[-0.200pt]{2.409pt}{0.400pt}}
\put(1429.0,798.0){\rule[-0.200pt]{2.409pt}{0.400pt}}
\put(170.0,798.0){\rule[-0.200pt]{2.409pt}{0.400pt}}
\put(1429.0,798.0){\rule[-0.200pt]{2.409pt}{0.400pt}}
\put(170.0,799.0){\rule[-0.200pt]{2.409pt}{0.400pt}}
\put(1429.0,799.0){\rule[-0.200pt]{2.409pt}{0.400pt}}
\put(170.0,799.0){\rule[-0.200pt]{2.409pt}{0.400pt}}
\put(1429.0,799.0){\rule[-0.200pt]{2.409pt}{0.400pt}}
\put(170.0,799.0){\rule[-0.200pt]{2.409pt}{0.400pt}}
\put(1429.0,799.0){\rule[-0.200pt]{2.409pt}{0.400pt}}
\put(170.0,799.0){\rule[-0.200pt]{2.409pt}{0.400pt}}
\put(1429.0,799.0){\rule[-0.200pt]{2.409pt}{0.400pt}}
\put(170.0,799.0){\rule[-0.200pt]{2.409pt}{0.400pt}}
\put(1429.0,799.0){\rule[-0.200pt]{2.409pt}{0.400pt}}
\put(170.0,800.0){\rule[-0.200pt]{2.409pt}{0.400pt}}
\put(1429.0,800.0){\rule[-0.200pt]{2.409pt}{0.400pt}}
\put(170.0,800.0){\rule[-0.200pt]{2.409pt}{0.400pt}}
\put(1429.0,800.0){\rule[-0.200pt]{2.409pt}{0.400pt}}
\put(170.0,800.0){\rule[-0.200pt]{2.409pt}{0.400pt}}
\put(1429.0,800.0){\rule[-0.200pt]{2.409pt}{0.400pt}}
\put(170.0,800.0){\rule[-0.200pt]{2.409pt}{0.400pt}}
\put(1429.0,800.0){\rule[-0.200pt]{2.409pt}{0.400pt}}
\put(170.0,800.0){\rule[-0.200pt]{2.409pt}{0.400pt}}
\put(1429.0,800.0){\rule[-0.200pt]{2.409pt}{0.400pt}}
\put(170.0,800.0){\rule[-0.200pt]{2.409pt}{0.400pt}}
\put(1429.0,800.0){\rule[-0.200pt]{2.409pt}{0.400pt}}
\put(170.0,801.0){\rule[-0.200pt]{2.409pt}{0.400pt}}
\put(1429.0,801.0){\rule[-0.200pt]{2.409pt}{0.400pt}}
\put(170.0,801.0){\rule[-0.200pt]{2.409pt}{0.400pt}}
\put(1429.0,801.0){\rule[-0.200pt]{2.409pt}{0.400pt}}
\put(170.0,801.0){\rule[-0.200pt]{2.409pt}{0.400pt}}
\put(1429.0,801.0){\rule[-0.200pt]{2.409pt}{0.400pt}}
\put(170.0,801.0){\rule[-0.200pt]{2.409pt}{0.400pt}}
\put(1429.0,801.0){\rule[-0.200pt]{2.409pt}{0.400pt}}
\put(170.0,801.0){\rule[-0.200pt]{2.409pt}{0.400pt}}
\put(1429.0,801.0){\rule[-0.200pt]{2.409pt}{0.400pt}}
\put(170.0,802.0){\rule[-0.200pt]{2.409pt}{0.400pt}}
\put(1429.0,802.0){\rule[-0.200pt]{2.409pt}{0.400pt}}
\put(170.0,802.0){\rule[-0.200pt]{2.409pt}{0.400pt}}
\put(1429.0,802.0){\rule[-0.200pt]{2.409pt}{0.400pt}}
\put(170.0,802.0){\rule[-0.200pt]{2.409pt}{0.400pt}}
\put(1429.0,802.0){\rule[-0.200pt]{2.409pt}{0.400pt}}
\put(170.0,802.0){\rule[-0.200pt]{2.409pt}{0.400pt}}
\put(1429.0,802.0){\rule[-0.200pt]{2.409pt}{0.400pt}}
\put(170.0,802.0){\rule[-0.200pt]{2.409pt}{0.400pt}}
\put(1429.0,802.0){\rule[-0.200pt]{2.409pt}{0.400pt}}
\put(170.0,802.0){\rule[-0.200pt]{2.409pt}{0.400pt}}
\put(1429.0,802.0){\rule[-0.200pt]{2.409pt}{0.400pt}}
\put(170.0,803.0){\rule[-0.200pt]{2.409pt}{0.400pt}}
\put(1429.0,803.0){\rule[-0.200pt]{2.409pt}{0.400pt}}
\put(170.0,803.0){\rule[-0.200pt]{2.409pt}{0.400pt}}
\put(1429.0,803.0){\rule[-0.200pt]{2.409pt}{0.400pt}}
\put(170.0,803.0){\rule[-0.200pt]{2.409pt}{0.400pt}}
\put(1429.0,803.0){\rule[-0.200pt]{2.409pt}{0.400pt}}
\put(170.0,803.0){\rule[-0.200pt]{2.409pt}{0.400pt}}
\put(1429.0,803.0){\rule[-0.200pt]{2.409pt}{0.400pt}}
\put(170.0,803.0){\rule[-0.200pt]{2.409pt}{0.400pt}}
\put(1429.0,803.0){\rule[-0.200pt]{2.409pt}{0.400pt}}
\put(170.0,803.0){\rule[-0.200pt]{2.409pt}{0.400pt}}
\put(1429.0,803.0){\rule[-0.200pt]{2.409pt}{0.400pt}}
\put(170.0,804.0){\rule[-0.200pt]{2.409pt}{0.400pt}}
\put(1429.0,804.0){\rule[-0.200pt]{2.409pt}{0.400pt}}
\put(170.0,804.0){\rule[-0.200pt]{2.409pt}{0.400pt}}
\put(1429.0,804.0){\rule[-0.200pt]{2.409pt}{0.400pt}}
\put(170.0,804.0){\rule[-0.200pt]{2.409pt}{0.400pt}}
\put(1429.0,804.0){\rule[-0.200pt]{2.409pt}{0.400pt}}
\put(170.0,804.0){\rule[-0.200pt]{2.409pt}{0.400pt}}
\put(1429.0,804.0){\rule[-0.200pt]{2.409pt}{0.400pt}}
\put(170.0,804.0){\rule[-0.200pt]{2.409pt}{0.400pt}}
\put(1429.0,804.0){\rule[-0.200pt]{2.409pt}{0.400pt}}
\put(170.0,804.0){\rule[-0.200pt]{2.409pt}{0.400pt}}
\put(1429.0,804.0){\rule[-0.200pt]{2.409pt}{0.400pt}}
\put(170.0,805.0){\rule[-0.200pt]{2.409pt}{0.400pt}}
\put(1429.0,805.0){\rule[-0.200pt]{2.409pt}{0.400pt}}
\put(170.0,805.0){\rule[-0.200pt]{2.409pt}{0.400pt}}
\put(1429.0,805.0){\rule[-0.200pt]{2.409pt}{0.400pt}}
\put(170.0,805.0){\rule[-0.200pt]{2.409pt}{0.400pt}}
\put(1429.0,805.0){\rule[-0.200pt]{2.409pt}{0.400pt}}
\put(170.0,805.0){\rule[-0.200pt]{2.409pt}{0.400pt}}
\put(1429.0,805.0){\rule[-0.200pt]{2.409pt}{0.400pt}}
\put(170.0,805.0){\rule[-0.200pt]{2.409pt}{0.400pt}}
\put(1429.0,805.0){\rule[-0.200pt]{2.409pt}{0.400pt}}
\put(170.0,805.0){\rule[-0.200pt]{2.409pt}{0.400pt}}
\put(1429.0,805.0){\rule[-0.200pt]{2.409pt}{0.400pt}}
\put(170.0,805.0){\rule[-0.200pt]{2.409pt}{0.400pt}}
\put(1429.0,805.0){\rule[-0.200pt]{2.409pt}{0.400pt}}
\put(170.0,806.0){\rule[-0.200pt]{2.409pt}{0.400pt}}
\put(1429.0,806.0){\rule[-0.200pt]{2.409pt}{0.400pt}}
\put(170.0,806.0){\rule[-0.200pt]{2.409pt}{0.400pt}}
\put(1429.0,806.0){\rule[-0.200pt]{2.409pt}{0.400pt}}
\put(170.0,806.0){\rule[-0.200pt]{2.409pt}{0.400pt}}
\put(1429.0,806.0){\rule[-0.200pt]{2.409pt}{0.400pt}}
\put(170.0,806.0){\rule[-0.200pt]{2.409pt}{0.400pt}}
\put(1429.0,806.0){\rule[-0.200pt]{2.409pt}{0.400pt}}
\put(170.0,806.0){\rule[-0.200pt]{2.409pt}{0.400pt}}
\put(1429.0,806.0){\rule[-0.200pt]{2.409pt}{0.400pt}}
\put(170.0,806.0){\rule[-0.200pt]{2.409pt}{0.400pt}}
\put(1429.0,806.0){\rule[-0.200pt]{2.409pt}{0.400pt}}
\put(170.0,807.0){\rule[-0.200pt]{2.409pt}{0.400pt}}
\put(1429.0,807.0){\rule[-0.200pt]{2.409pt}{0.400pt}}
\put(170.0,807.0){\rule[-0.200pt]{2.409pt}{0.400pt}}
\put(1429.0,807.0){\rule[-0.200pt]{2.409pt}{0.400pt}}
\put(170.0,807.0){\rule[-0.200pt]{2.409pt}{0.400pt}}
\put(1429.0,807.0){\rule[-0.200pt]{2.409pt}{0.400pt}}
\put(170.0,807.0){\rule[-0.200pt]{2.409pt}{0.400pt}}
\put(1429.0,807.0){\rule[-0.200pt]{2.409pt}{0.400pt}}
\put(170.0,807.0){\rule[-0.200pt]{2.409pt}{0.400pt}}
\put(1429.0,807.0){\rule[-0.200pt]{2.409pt}{0.400pt}}
\put(170.0,807.0){\rule[-0.200pt]{2.409pt}{0.400pt}}
\put(1429.0,807.0){\rule[-0.200pt]{2.409pt}{0.400pt}}
\put(170.0,807.0){\rule[-0.200pt]{2.409pt}{0.400pt}}
\put(1429.0,807.0){\rule[-0.200pt]{2.409pt}{0.400pt}}
\put(170.0,808.0){\rule[-0.200pt]{2.409pt}{0.400pt}}
\put(1429.0,808.0){\rule[-0.200pt]{2.409pt}{0.400pt}}
\put(170.0,808.0){\rule[-0.200pt]{2.409pt}{0.400pt}}
\put(1429.0,808.0){\rule[-0.200pt]{2.409pt}{0.400pt}}
\put(170.0,808.0){\rule[-0.200pt]{2.409pt}{0.400pt}}
\put(1429.0,808.0){\rule[-0.200pt]{2.409pt}{0.400pt}}
\put(170.0,808.0){\rule[-0.200pt]{2.409pt}{0.400pt}}
\put(1429.0,808.0){\rule[-0.200pt]{2.409pt}{0.400pt}}
\put(170.0,808.0){\rule[-0.200pt]{2.409pt}{0.400pt}}
\put(1429.0,808.0){\rule[-0.200pt]{2.409pt}{0.400pt}}
\put(170.0,808.0){\rule[-0.200pt]{2.409pt}{0.400pt}}
\put(1429.0,808.0){\rule[-0.200pt]{2.409pt}{0.400pt}}
\put(170.0,808.0){\rule[-0.200pt]{2.409pt}{0.400pt}}
\put(1429.0,808.0){\rule[-0.200pt]{2.409pt}{0.400pt}}
\put(170.0,809.0){\rule[-0.200pt]{2.409pt}{0.400pt}}
\put(1429.0,809.0){\rule[-0.200pt]{2.409pt}{0.400pt}}
\put(170.0,809.0){\rule[-0.200pt]{2.409pt}{0.400pt}}
\put(1429.0,809.0){\rule[-0.200pt]{2.409pt}{0.400pt}}
\put(170.0,809.0){\rule[-0.200pt]{2.409pt}{0.400pt}}
\put(1429.0,809.0){\rule[-0.200pt]{2.409pt}{0.400pt}}
\put(170.0,809.0){\rule[-0.200pt]{2.409pt}{0.400pt}}
\put(1429.0,809.0){\rule[-0.200pt]{2.409pt}{0.400pt}}
\put(170.0,809.0){\rule[-0.200pt]{2.409pt}{0.400pt}}
\put(1429.0,809.0){\rule[-0.200pt]{2.409pt}{0.400pt}}
\put(170.0,809.0){\rule[-0.200pt]{2.409pt}{0.400pt}}
\put(1429.0,809.0){\rule[-0.200pt]{2.409pt}{0.400pt}}
\put(170.0,809.0){\rule[-0.200pt]{2.409pt}{0.400pt}}
\put(1429.0,809.0){\rule[-0.200pt]{2.409pt}{0.400pt}}
\put(170.0,810.0){\rule[-0.200pt]{2.409pt}{0.400pt}}
\put(1429.0,810.0){\rule[-0.200pt]{2.409pt}{0.400pt}}
\put(170.0,810.0){\rule[-0.200pt]{2.409pt}{0.400pt}}
\put(1429.0,810.0){\rule[-0.200pt]{2.409pt}{0.400pt}}
\put(170.0,810.0){\rule[-0.200pt]{2.409pt}{0.400pt}}
\put(1429.0,810.0){\rule[-0.200pt]{2.409pt}{0.400pt}}
\put(170.0,810.0){\rule[-0.200pt]{2.409pt}{0.400pt}}
\put(1429.0,810.0){\rule[-0.200pt]{2.409pt}{0.400pt}}
\put(170.0,810.0){\rule[-0.200pt]{2.409pt}{0.400pt}}
\put(1429.0,810.0){\rule[-0.200pt]{2.409pt}{0.400pt}}
\put(170.0,810.0){\rule[-0.200pt]{2.409pt}{0.400pt}}
\put(1429.0,810.0){\rule[-0.200pt]{2.409pt}{0.400pt}}
\put(170.0,810.0){\rule[-0.200pt]{2.409pt}{0.400pt}}
\put(1429.0,810.0){\rule[-0.200pt]{2.409pt}{0.400pt}}
\put(170.0,811.0){\rule[-0.200pt]{2.409pt}{0.400pt}}
\put(1429.0,811.0){\rule[-0.200pt]{2.409pt}{0.400pt}}
\put(170.0,811.0){\rule[-0.200pt]{2.409pt}{0.400pt}}
\put(1429.0,811.0){\rule[-0.200pt]{2.409pt}{0.400pt}}
\put(170.0,811.0){\rule[-0.200pt]{2.409pt}{0.400pt}}
\put(1429.0,811.0){\rule[-0.200pt]{2.409pt}{0.400pt}}
\put(170.0,811.0){\rule[-0.200pt]{2.409pt}{0.400pt}}
\put(1429.0,811.0){\rule[-0.200pt]{2.409pt}{0.400pt}}
\put(170.0,811.0){\rule[-0.200pt]{2.409pt}{0.400pt}}
\put(1429.0,811.0){\rule[-0.200pt]{2.409pt}{0.400pt}}
\put(170.0,811.0){\rule[-0.200pt]{2.409pt}{0.400pt}}
\put(1429.0,811.0){\rule[-0.200pt]{2.409pt}{0.400pt}}
\put(170.0,811.0){\rule[-0.200pt]{2.409pt}{0.400pt}}
\put(1429.0,811.0){\rule[-0.200pt]{2.409pt}{0.400pt}}
\put(170.0,812.0){\rule[-0.200pt]{2.409pt}{0.400pt}}
\put(1429.0,812.0){\rule[-0.200pt]{2.409pt}{0.400pt}}
\put(170.0,812.0){\rule[-0.200pt]{2.409pt}{0.400pt}}
\put(1429.0,812.0){\rule[-0.200pt]{2.409pt}{0.400pt}}
\put(170.0,812.0){\rule[-0.200pt]{2.409pt}{0.400pt}}
\put(1429.0,812.0){\rule[-0.200pt]{2.409pt}{0.400pt}}
\put(170.0,812.0){\rule[-0.200pt]{2.409pt}{0.400pt}}
\put(1429.0,812.0){\rule[-0.200pt]{2.409pt}{0.400pt}}
\put(170.0,812.0){\rule[-0.200pt]{2.409pt}{0.400pt}}
\put(1429.0,812.0){\rule[-0.200pt]{2.409pt}{0.400pt}}
\put(170.0,812.0){\rule[-0.200pt]{2.409pt}{0.400pt}}
\put(1429.0,812.0){\rule[-0.200pt]{2.409pt}{0.400pt}}
\put(170.0,812.0){\rule[-0.200pt]{2.409pt}{0.400pt}}
\put(1429.0,812.0){\rule[-0.200pt]{2.409pt}{0.400pt}}
\put(170.0,812.0){\rule[-0.200pt]{2.409pt}{0.400pt}}
\put(1429.0,812.0){\rule[-0.200pt]{2.409pt}{0.400pt}}
\put(170.0,813.0){\rule[-0.200pt]{2.409pt}{0.400pt}}
\put(1429.0,813.0){\rule[-0.200pt]{2.409pt}{0.400pt}}
\put(170.0,813.0){\rule[-0.200pt]{2.409pt}{0.400pt}}
\put(1429.0,813.0){\rule[-0.200pt]{2.409pt}{0.400pt}}
\put(170.0,813.0){\rule[-0.200pt]{2.409pt}{0.400pt}}
\put(1429.0,813.0){\rule[-0.200pt]{2.409pt}{0.400pt}}
\put(170.0,813.0){\rule[-0.200pt]{2.409pt}{0.400pt}}
\put(1429.0,813.0){\rule[-0.200pt]{2.409pt}{0.400pt}}
\put(170.0,813.0){\rule[-0.200pt]{2.409pt}{0.400pt}}
\put(1429.0,813.0){\rule[-0.200pt]{2.409pt}{0.400pt}}
\put(170.0,813.0){\rule[-0.200pt]{2.409pt}{0.400pt}}
\put(1429.0,813.0){\rule[-0.200pt]{2.409pt}{0.400pt}}
\put(170.0,813.0){\rule[-0.200pt]{2.409pt}{0.400pt}}
\put(1429.0,813.0){\rule[-0.200pt]{2.409pt}{0.400pt}}
\put(170.0,813.0){\rule[-0.200pt]{2.409pt}{0.400pt}}
\put(1429.0,813.0){\rule[-0.200pt]{2.409pt}{0.400pt}}
\put(170.0,814.0){\rule[-0.200pt]{2.409pt}{0.400pt}}
\put(1429.0,814.0){\rule[-0.200pt]{2.409pt}{0.400pt}}
\put(170.0,814.0){\rule[-0.200pt]{2.409pt}{0.400pt}}
\put(1429.0,814.0){\rule[-0.200pt]{2.409pt}{0.400pt}}
\put(170.0,814.0){\rule[-0.200pt]{2.409pt}{0.400pt}}
\put(1429.0,814.0){\rule[-0.200pt]{2.409pt}{0.400pt}}
\put(170.0,814.0){\rule[-0.200pt]{2.409pt}{0.400pt}}
\put(1429.0,814.0){\rule[-0.200pt]{2.409pt}{0.400pt}}
\put(170.0,814.0){\rule[-0.200pt]{2.409pt}{0.400pt}}
\put(1429.0,814.0){\rule[-0.200pt]{2.409pt}{0.400pt}}
\put(170.0,814.0){\rule[-0.200pt]{2.409pt}{0.400pt}}
\put(1429.0,814.0){\rule[-0.200pt]{2.409pt}{0.400pt}}
\put(170.0,814.0){\rule[-0.200pt]{2.409pt}{0.400pt}}
\put(1429.0,814.0){\rule[-0.200pt]{2.409pt}{0.400pt}}
\put(170.0,814.0){\rule[-0.200pt]{2.409pt}{0.400pt}}
\put(1429.0,814.0){\rule[-0.200pt]{2.409pt}{0.400pt}}
\put(170.0,815.0){\rule[-0.200pt]{2.409pt}{0.400pt}}
\put(1429.0,815.0){\rule[-0.200pt]{2.409pt}{0.400pt}}
\put(170.0,815.0){\rule[-0.200pt]{2.409pt}{0.400pt}}
\put(1429.0,815.0){\rule[-0.200pt]{2.409pt}{0.400pt}}
\put(170.0,815.0){\rule[-0.200pt]{2.409pt}{0.400pt}}
\put(1429.0,815.0){\rule[-0.200pt]{2.409pt}{0.400pt}}
\put(170.0,815.0){\rule[-0.200pt]{2.409pt}{0.400pt}}
\put(1429.0,815.0){\rule[-0.200pt]{2.409pt}{0.400pt}}
\put(170.0,815.0){\rule[-0.200pt]{2.409pt}{0.400pt}}
\put(1429.0,815.0){\rule[-0.200pt]{2.409pt}{0.400pt}}
\put(170.0,815.0){\rule[-0.200pt]{2.409pt}{0.400pt}}
\put(1429.0,815.0){\rule[-0.200pt]{2.409pt}{0.400pt}}
\put(170.0,815.0){\rule[-0.200pt]{2.409pt}{0.400pt}}
\put(1429.0,815.0){\rule[-0.200pt]{2.409pt}{0.400pt}}
\put(170.0,815.0){\rule[-0.200pt]{2.409pt}{0.400pt}}
\put(1429.0,815.0){\rule[-0.200pt]{2.409pt}{0.400pt}}
\put(170.0,816.0){\rule[-0.200pt]{2.409pt}{0.400pt}}
\put(1429.0,816.0){\rule[-0.200pt]{2.409pt}{0.400pt}}
\put(170.0,816.0){\rule[-0.200pt]{2.409pt}{0.400pt}}
\put(1429.0,816.0){\rule[-0.200pt]{2.409pt}{0.400pt}}
\put(170.0,816.0){\rule[-0.200pt]{2.409pt}{0.400pt}}
\put(1429.0,816.0){\rule[-0.200pt]{2.409pt}{0.400pt}}
\put(170.0,816.0){\rule[-0.200pt]{2.409pt}{0.400pt}}
\put(1429.0,816.0){\rule[-0.200pt]{2.409pt}{0.400pt}}
\put(170.0,816.0){\rule[-0.200pt]{2.409pt}{0.400pt}}
\put(1429.0,816.0){\rule[-0.200pt]{2.409pt}{0.400pt}}
\put(170.0,816.0){\rule[-0.200pt]{2.409pt}{0.400pt}}
\put(1429.0,816.0){\rule[-0.200pt]{2.409pt}{0.400pt}}
\put(170.0,816.0){\rule[-0.200pt]{2.409pt}{0.400pt}}
\put(1429.0,816.0){\rule[-0.200pt]{2.409pt}{0.400pt}}
\put(170.0,816.0){\rule[-0.200pt]{2.409pt}{0.400pt}}
\put(1429.0,816.0){\rule[-0.200pt]{2.409pt}{0.400pt}}
\put(170.0,817.0){\rule[-0.200pt]{2.409pt}{0.400pt}}
\put(1429.0,817.0){\rule[-0.200pt]{2.409pt}{0.400pt}}
\put(170.0,817.0){\rule[-0.200pt]{2.409pt}{0.400pt}}
\put(1429.0,817.0){\rule[-0.200pt]{2.409pt}{0.400pt}}
\put(170.0,817.0){\rule[-0.200pt]{2.409pt}{0.400pt}}
\put(1429.0,817.0){\rule[-0.200pt]{2.409pt}{0.400pt}}
\put(170.0,817.0){\rule[-0.200pt]{2.409pt}{0.400pt}}
\put(1429.0,817.0){\rule[-0.200pt]{2.409pt}{0.400pt}}
\put(170.0,817.0){\rule[-0.200pt]{2.409pt}{0.400pt}}
\put(1429.0,817.0){\rule[-0.200pt]{2.409pt}{0.400pt}}
\put(170.0,817.0){\rule[-0.200pt]{2.409pt}{0.400pt}}
\put(1429.0,817.0){\rule[-0.200pt]{2.409pt}{0.400pt}}
\put(170.0,817.0){\rule[-0.200pt]{2.409pt}{0.400pt}}
\put(1429.0,817.0){\rule[-0.200pt]{2.409pt}{0.400pt}}
\put(170.0,817.0){\rule[-0.200pt]{2.409pt}{0.400pt}}
\put(1429.0,817.0){\rule[-0.200pt]{2.409pt}{0.400pt}}
\put(170.0,817.0){\rule[-0.200pt]{2.409pt}{0.400pt}}
\put(1429.0,817.0){\rule[-0.200pt]{2.409pt}{0.400pt}}
\put(170.0,818.0){\rule[-0.200pt]{2.409pt}{0.400pt}}
\put(1429.0,818.0){\rule[-0.200pt]{2.409pt}{0.400pt}}
\put(170.0,818.0){\rule[-0.200pt]{2.409pt}{0.400pt}}
\put(1429.0,818.0){\rule[-0.200pt]{2.409pt}{0.400pt}}
\put(170.0,818.0){\rule[-0.200pt]{2.409pt}{0.400pt}}
\put(1429.0,818.0){\rule[-0.200pt]{2.409pt}{0.400pt}}
\put(170.0,818.0){\rule[-0.200pt]{2.409pt}{0.400pt}}
\put(1429.0,818.0){\rule[-0.200pt]{2.409pt}{0.400pt}}
\put(170.0,818.0){\rule[-0.200pt]{2.409pt}{0.400pt}}
\put(1429.0,818.0){\rule[-0.200pt]{2.409pt}{0.400pt}}
\put(170.0,818.0){\rule[-0.200pt]{2.409pt}{0.400pt}}
\put(1429.0,818.0){\rule[-0.200pt]{2.409pt}{0.400pt}}
\put(170.0,818.0){\rule[-0.200pt]{2.409pt}{0.400pt}}
\put(1429.0,818.0){\rule[-0.200pt]{2.409pt}{0.400pt}}
\put(170.0,818.0){\rule[-0.200pt]{2.409pt}{0.400pt}}
\put(1429.0,818.0){\rule[-0.200pt]{2.409pt}{0.400pt}}
\put(170.0,818.0){\rule[-0.200pt]{2.409pt}{0.400pt}}
\put(1429.0,818.0){\rule[-0.200pt]{2.409pt}{0.400pt}}
\put(170.0,819.0){\rule[-0.200pt]{2.409pt}{0.400pt}}
\put(1429.0,819.0){\rule[-0.200pt]{2.409pt}{0.400pt}}
\put(170.0,819.0){\rule[-0.200pt]{2.409pt}{0.400pt}}
\put(1429.0,819.0){\rule[-0.200pt]{2.409pt}{0.400pt}}
\put(170.0,819.0){\rule[-0.200pt]{2.409pt}{0.400pt}}
\put(1429.0,819.0){\rule[-0.200pt]{2.409pt}{0.400pt}}
\put(170.0,819.0){\rule[-0.200pt]{2.409pt}{0.400pt}}
\put(1429.0,819.0){\rule[-0.200pt]{2.409pt}{0.400pt}}
\put(170.0,819.0){\rule[-0.200pt]{2.409pt}{0.400pt}}
\put(1429.0,819.0){\rule[-0.200pt]{2.409pt}{0.400pt}}
\put(170.0,819.0){\rule[-0.200pt]{2.409pt}{0.400pt}}
\put(1429.0,819.0){\rule[-0.200pt]{2.409pt}{0.400pt}}
\put(170.0,819.0){\rule[-0.200pt]{2.409pt}{0.400pt}}
\put(1429.0,819.0){\rule[-0.200pt]{2.409pt}{0.400pt}}
\put(170.0,819.0){\rule[-0.200pt]{2.409pt}{0.400pt}}
\put(1429.0,819.0){\rule[-0.200pt]{2.409pt}{0.400pt}}
\put(170.0,819.0){\rule[-0.200pt]{2.409pt}{0.400pt}}
\put(1429.0,819.0){\rule[-0.200pt]{2.409pt}{0.400pt}}
\put(170.0,820.0){\rule[-0.200pt]{2.409pt}{0.400pt}}
\put(1429.0,820.0){\rule[-0.200pt]{2.409pt}{0.400pt}}
\put(170.0,820.0){\rule[-0.200pt]{2.409pt}{0.400pt}}
\put(1429.0,820.0){\rule[-0.200pt]{2.409pt}{0.400pt}}
\put(170.0,820.0){\rule[-0.200pt]{2.409pt}{0.400pt}}
\put(1429.0,820.0){\rule[-0.200pt]{2.409pt}{0.400pt}}
\put(170.0,820.0){\rule[-0.200pt]{2.409pt}{0.400pt}}
\put(1429.0,820.0){\rule[-0.200pt]{2.409pt}{0.400pt}}
\put(170.0,820.0){\rule[-0.200pt]{2.409pt}{0.400pt}}
\put(1429.0,820.0){\rule[-0.200pt]{2.409pt}{0.400pt}}
\put(170.0,820.0){\rule[-0.200pt]{2.409pt}{0.400pt}}
\put(1429.0,820.0){\rule[-0.200pt]{2.409pt}{0.400pt}}
\put(170.0,820.0){\rule[-0.200pt]{2.409pt}{0.400pt}}
\put(1429.0,820.0){\rule[-0.200pt]{2.409pt}{0.400pt}}
\put(170.0,820.0){\rule[-0.200pt]{2.409pt}{0.400pt}}
\put(1429.0,820.0){\rule[-0.200pt]{2.409pt}{0.400pt}}
\put(170.0,820.0){\rule[-0.200pt]{2.409pt}{0.400pt}}
\put(1429.0,820.0){\rule[-0.200pt]{2.409pt}{0.400pt}}
\put(170.0,820.0){\rule[-0.200pt]{2.409pt}{0.400pt}}
\put(1429.0,820.0){\rule[-0.200pt]{2.409pt}{0.400pt}}
\put(170.0,821.0){\rule[-0.200pt]{2.409pt}{0.400pt}}
\put(1429.0,821.0){\rule[-0.200pt]{2.409pt}{0.400pt}}
\put(170.0,821.0){\rule[-0.200pt]{2.409pt}{0.400pt}}
\put(1429.0,821.0){\rule[-0.200pt]{2.409pt}{0.400pt}}
\put(170.0,821.0){\rule[-0.200pt]{2.409pt}{0.400pt}}
\put(1429.0,821.0){\rule[-0.200pt]{2.409pt}{0.400pt}}
\put(170.0,821.0){\rule[-0.200pt]{2.409pt}{0.400pt}}
\put(1429.0,821.0){\rule[-0.200pt]{2.409pt}{0.400pt}}
\put(170.0,821.0){\rule[-0.200pt]{2.409pt}{0.400pt}}
\put(1429.0,821.0){\rule[-0.200pt]{2.409pt}{0.400pt}}
\put(170.0,821.0){\rule[-0.200pt]{2.409pt}{0.400pt}}
\put(1429.0,821.0){\rule[-0.200pt]{2.409pt}{0.400pt}}
\put(170.0,821.0){\rule[-0.200pt]{2.409pt}{0.400pt}}
\put(1429.0,821.0){\rule[-0.200pt]{2.409pt}{0.400pt}}
\put(170.0,821.0){\rule[-0.200pt]{2.409pt}{0.400pt}}
\put(1429.0,821.0){\rule[-0.200pt]{2.409pt}{0.400pt}}
\put(170.0,821.0){\rule[-0.200pt]{2.409pt}{0.400pt}}
\put(1429.0,821.0){\rule[-0.200pt]{2.409pt}{0.400pt}}
\put(170.0,822.0){\rule[-0.200pt]{2.409pt}{0.400pt}}
\put(1429.0,822.0){\rule[-0.200pt]{2.409pt}{0.400pt}}
\put(170.0,822.0){\rule[-0.200pt]{2.409pt}{0.400pt}}
\put(1429.0,822.0){\rule[-0.200pt]{2.409pt}{0.400pt}}
\put(170.0,822.0){\rule[-0.200pt]{2.409pt}{0.400pt}}
\put(1429.0,822.0){\rule[-0.200pt]{2.409pt}{0.400pt}}
\put(170.0,822.0){\rule[-0.200pt]{2.409pt}{0.400pt}}
\put(1429.0,822.0){\rule[-0.200pt]{2.409pt}{0.400pt}}
\put(170.0,822.0){\rule[-0.200pt]{2.409pt}{0.400pt}}
\put(1429.0,822.0){\rule[-0.200pt]{2.409pt}{0.400pt}}
\put(170.0,822.0){\rule[-0.200pt]{2.409pt}{0.400pt}}
\put(1429.0,822.0){\rule[-0.200pt]{2.409pt}{0.400pt}}
\put(170.0,822.0){\rule[-0.200pt]{2.409pt}{0.400pt}}
\put(1429.0,822.0){\rule[-0.200pt]{2.409pt}{0.400pt}}
\put(170.0,822.0){\rule[-0.200pt]{2.409pt}{0.400pt}}
\put(1429.0,822.0){\rule[-0.200pt]{2.409pt}{0.400pt}}
\put(170.0,822.0){\rule[-0.200pt]{2.409pt}{0.400pt}}
\put(1429.0,822.0){\rule[-0.200pt]{2.409pt}{0.400pt}}
\put(170.0,822.0){\rule[-0.200pt]{2.409pt}{0.400pt}}
\put(1429.0,822.0){\rule[-0.200pt]{2.409pt}{0.400pt}}
\put(170.0,823.0){\rule[-0.200pt]{2.409pt}{0.400pt}}
\put(1429.0,823.0){\rule[-0.200pt]{2.409pt}{0.400pt}}
\put(170.0,823.0){\rule[-0.200pt]{2.409pt}{0.400pt}}
\put(1429.0,823.0){\rule[-0.200pt]{2.409pt}{0.400pt}}
\put(170.0,823.0){\rule[-0.200pt]{2.409pt}{0.400pt}}
\put(1429.0,823.0){\rule[-0.200pt]{2.409pt}{0.400pt}}
\put(170.0,823.0){\rule[-0.200pt]{2.409pt}{0.400pt}}
\put(1429.0,823.0){\rule[-0.200pt]{2.409pt}{0.400pt}}
\put(170.0,823.0){\rule[-0.200pt]{2.409pt}{0.400pt}}
\put(1429.0,823.0){\rule[-0.200pt]{2.409pt}{0.400pt}}
\put(170.0,823.0){\rule[-0.200pt]{2.409pt}{0.400pt}}
\put(1429.0,823.0){\rule[-0.200pt]{2.409pt}{0.400pt}}
\put(170.0,823.0){\rule[-0.200pt]{2.409pt}{0.400pt}}
\put(1429.0,823.0){\rule[-0.200pt]{2.409pt}{0.400pt}}
\put(170.0,823.0){\rule[-0.200pt]{2.409pt}{0.400pt}}
\put(1429.0,823.0){\rule[-0.200pt]{2.409pt}{0.400pt}}
\put(170.0,823.0){\rule[-0.200pt]{2.409pt}{0.400pt}}
\put(1429.0,823.0){\rule[-0.200pt]{2.409pt}{0.400pt}}
\put(170.0,823.0){\rule[-0.200pt]{2.409pt}{0.400pt}}
\put(1429.0,823.0){\rule[-0.200pt]{2.409pt}{0.400pt}}
\put(170.0,824.0){\rule[-0.200pt]{2.409pt}{0.400pt}}
\put(1429.0,824.0){\rule[-0.200pt]{2.409pt}{0.400pt}}
\put(170.0,824.0){\rule[-0.200pt]{2.409pt}{0.400pt}}
\put(1429.0,824.0){\rule[-0.200pt]{2.409pt}{0.400pt}}
\put(170.0,824.0){\rule[-0.200pt]{2.409pt}{0.400pt}}
\put(1429.0,824.0){\rule[-0.200pt]{2.409pt}{0.400pt}}
\put(170.0,824.0){\rule[-0.200pt]{2.409pt}{0.400pt}}
\put(1429.0,824.0){\rule[-0.200pt]{2.409pt}{0.400pt}}
\put(170.0,824.0){\rule[-0.200pt]{2.409pt}{0.400pt}}
\put(1429.0,824.0){\rule[-0.200pt]{2.409pt}{0.400pt}}
\put(170.0,824.0){\rule[-0.200pt]{2.409pt}{0.400pt}}
\put(1429.0,824.0){\rule[-0.200pt]{2.409pt}{0.400pt}}
\put(170.0,824.0){\rule[-0.200pt]{2.409pt}{0.400pt}}
\put(1429.0,824.0){\rule[-0.200pt]{2.409pt}{0.400pt}}
\put(170.0,824.0){\rule[-0.200pt]{2.409pt}{0.400pt}}
\put(1429.0,824.0){\rule[-0.200pt]{2.409pt}{0.400pt}}
\put(170.0,824.0){\rule[-0.200pt]{2.409pt}{0.400pt}}
\put(1429.0,824.0){\rule[-0.200pt]{2.409pt}{0.400pt}}
\put(170.0,824.0){\rule[-0.200pt]{2.409pt}{0.400pt}}
\put(1429.0,824.0){\rule[-0.200pt]{2.409pt}{0.400pt}}
\put(170.0,824.0){\rule[-0.200pt]{2.409pt}{0.400pt}}
\put(1429.0,824.0){\rule[-0.200pt]{2.409pt}{0.400pt}}
\put(170.0,825.0){\rule[-0.200pt]{2.409pt}{0.400pt}}
\put(1429.0,825.0){\rule[-0.200pt]{2.409pt}{0.400pt}}
\put(170.0,825.0){\rule[-0.200pt]{2.409pt}{0.400pt}}
\put(1429.0,825.0){\rule[-0.200pt]{2.409pt}{0.400pt}}
\put(170.0,825.0){\rule[-0.200pt]{2.409pt}{0.400pt}}
\put(1429.0,825.0){\rule[-0.200pt]{2.409pt}{0.400pt}}
\put(170.0,825.0){\rule[-0.200pt]{2.409pt}{0.400pt}}
\put(1429.0,825.0){\rule[-0.200pt]{2.409pt}{0.400pt}}
\put(170.0,825.0){\rule[-0.200pt]{2.409pt}{0.400pt}}
\put(1429.0,825.0){\rule[-0.200pt]{2.409pt}{0.400pt}}
\put(170.0,825.0){\rule[-0.200pt]{2.409pt}{0.400pt}}
\put(1429.0,825.0){\rule[-0.200pt]{2.409pt}{0.400pt}}
\put(170.0,825.0){\rule[-0.200pt]{2.409pt}{0.400pt}}
\put(1429.0,825.0){\rule[-0.200pt]{2.409pt}{0.400pt}}
\put(170.0,825.0){\rule[-0.200pt]{2.409pt}{0.400pt}}
\put(1429.0,825.0){\rule[-0.200pt]{2.409pt}{0.400pt}}
\put(170.0,825.0){\rule[-0.200pt]{2.409pt}{0.400pt}}
\put(1429.0,825.0){\rule[-0.200pt]{2.409pt}{0.400pt}}
\put(170.0,825.0){\rule[-0.200pt]{2.409pt}{0.400pt}}
\put(1429.0,825.0){\rule[-0.200pt]{2.409pt}{0.400pt}}
\put(170.0,825.0){\rule[-0.200pt]{2.409pt}{0.400pt}}
\put(1429.0,825.0){\rule[-0.200pt]{2.409pt}{0.400pt}}
\put(170.0,826.0){\rule[-0.200pt]{2.409pt}{0.400pt}}
\put(1429.0,826.0){\rule[-0.200pt]{2.409pt}{0.400pt}}
\put(170.0,826.0){\rule[-0.200pt]{2.409pt}{0.400pt}}
\put(1429.0,826.0){\rule[-0.200pt]{2.409pt}{0.400pt}}
\put(170.0,826.0){\rule[-0.200pt]{2.409pt}{0.400pt}}
\put(1429.0,826.0){\rule[-0.200pt]{2.409pt}{0.400pt}}
\put(170.0,826.0){\rule[-0.200pt]{2.409pt}{0.400pt}}
\put(1429.0,826.0){\rule[-0.200pt]{2.409pt}{0.400pt}}
\put(170.0,826.0){\rule[-0.200pt]{2.409pt}{0.400pt}}
\put(1429.0,826.0){\rule[-0.200pt]{2.409pt}{0.400pt}}
\put(170.0,826.0){\rule[-0.200pt]{2.409pt}{0.400pt}}
\put(1429.0,826.0){\rule[-0.200pt]{2.409pt}{0.400pt}}
\put(170.0,826.0){\rule[-0.200pt]{2.409pt}{0.400pt}}
\put(1429.0,826.0){\rule[-0.200pt]{2.409pt}{0.400pt}}
\put(170.0,826.0){\rule[-0.200pt]{2.409pt}{0.400pt}}
\put(1429.0,826.0){\rule[-0.200pt]{2.409pt}{0.400pt}}
\put(170.0,826.0){\rule[-0.200pt]{2.409pt}{0.400pt}}
\put(1429.0,826.0){\rule[-0.200pt]{2.409pt}{0.400pt}}
\put(170.0,826.0){\rule[-0.200pt]{2.409pt}{0.400pt}}
\put(1429.0,826.0){\rule[-0.200pt]{2.409pt}{0.400pt}}
\put(170.0,826.0){\rule[-0.200pt]{2.409pt}{0.400pt}}
\put(1429.0,826.0){\rule[-0.200pt]{2.409pt}{0.400pt}}
\put(170.0,827.0){\rule[-0.200pt]{2.409pt}{0.400pt}}
\put(1429.0,827.0){\rule[-0.200pt]{2.409pt}{0.400pt}}
\put(170.0,827.0){\rule[-0.200pt]{2.409pt}{0.400pt}}
\put(1429.0,827.0){\rule[-0.200pt]{2.409pt}{0.400pt}}
\put(170.0,827.0){\rule[-0.200pt]{2.409pt}{0.400pt}}
\put(1429.0,827.0){\rule[-0.200pt]{2.409pt}{0.400pt}}
\put(170.0,827.0){\rule[-0.200pt]{2.409pt}{0.400pt}}
\put(1429.0,827.0){\rule[-0.200pt]{2.409pt}{0.400pt}}
\put(170.0,827.0){\rule[-0.200pt]{2.409pt}{0.400pt}}
\put(1429.0,827.0){\rule[-0.200pt]{2.409pt}{0.400pt}}
\put(170.0,827.0){\rule[-0.200pt]{2.409pt}{0.400pt}}
\put(1429.0,827.0){\rule[-0.200pt]{2.409pt}{0.400pt}}
\put(170.0,827.0){\rule[-0.200pt]{2.409pt}{0.400pt}}
\put(1429.0,827.0){\rule[-0.200pt]{2.409pt}{0.400pt}}
\put(170.0,827.0){\rule[-0.200pt]{2.409pt}{0.400pt}}
\put(1429.0,827.0){\rule[-0.200pt]{2.409pt}{0.400pt}}
\put(170.0,827.0){\rule[-0.200pt]{2.409pt}{0.400pt}}
\put(1429.0,827.0){\rule[-0.200pt]{2.409pt}{0.400pt}}
\put(170.0,827.0){\rule[-0.200pt]{2.409pt}{0.400pt}}
\put(1429.0,827.0){\rule[-0.200pt]{2.409pt}{0.400pt}}
\put(170.0,827.0){\rule[-0.200pt]{2.409pt}{0.400pt}}
\put(1429.0,827.0){\rule[-0.200pt]{2.409pt}{0.400pt}}
\put(170.0,828.0){\rule[-0.200pt]{2.409pt}{0.400pt}}
\put(1429.0,828.0){\rule[-0.200pt]{2.409pt}{0.400pt}}
\put(170.0,828.0){\rule[-0.200pt]{2.409pt}{0.400pt}}
\put(1429.0,828.0){\rule[-0.200pt]{2.409pt}{0.400pt}}
\put(170.0,828.0){\rule[-0.200pt]{2.409pt}{0.400pt}}
\put(1429.0,828.0){\rule[-0.200pt]{2.409pt}{0.400pt}}
\put(170.0,828.0){\rule[-0.200pt]{2.409pt}{0.400pt}}
\put(1429.0,828.0){\rule[-0.200pt]{2.409pt}{0.400pt}}
\put(170.0,828.0){\rule[-0.200pt]{2.409pt}{0.400pt}}
\put(1429.0,828.0){\rule[-0.200pt]{2.409pt}{0.400pt}}
\put(170.0,828.0){\rule[-0.200pt]{2.409pt}{0.400pt}}
\put(1429.0,828.0){\rule[-0.200pt]{2.409pt}{0.400pt}}
\put(170.0,828.0){\rule[-0.200pt]{2.409pt}{0.400pt}}
\put(1429.0,828.0){\rule[-0.200pt]{2.409pt}{0.400pt}}
\put(170.0,828.0){\rule[-0.200pt]{2.409pt}{0.400pt}}
\put(1429.0,828.0){\rule[-0.200pt]{2.409pt}{0.400pt}}
\put(170.0,828.0){\rule[-0.200pt]{2.409pt}{0.400pt}}
\put(1429.0,828.0){\rule[-0.200pt]{2.409pt}{0.400pt}}
\put(170.0,828.0){\rule[-0.200pt]{2.409pt}{0.400pt}}
\put(1429.0,828.0){\rule[-0.200pt]{2.409pt}{0.400pt}}
\put(170.0,828.0){\rule[-0.200pt]{2.409pt}{0.400pt}}
\put(1429.0,828.0){\rule[-0.200pt]{2.409pt}{0.400pt}}
\put(170.0,828.0){\rule[-0.200pt]{2.409pt}{0.400pt}}
\put(1429.0,828.0){\rule[-0.200pt]{2.409pt}{0.400pt}}
\put(170.0,829.0){\rule[-0.200pt]{2.409pt}{0.400pt}}
\put(1429.0,829.0){\rule[-0.200pt]{2.409pt}{0.400pt}}
\put(170.0,829.0){\rule[-0.200pt]{2.409pt}{0.400pt}}
\put(1429.0,829.0){\rule[-0.200pt]{2.409pt}{0.400pt}}
\put(170.0,829.0){\rule[-0.200pt]{2.409pt}{0.400pt}}
\put(1429.0,829.0){\rule[-0.200pt]{2.409pt}{0.400pt}}
\put(170.0,829.0){\rule[-0.200pt]{2.409pt}{0.400pt}}
\put(1429.0,829.0){\rule[-0.200pt]{2.409pt}{0.400pt}}
\put(170.0,829.0){\rule[-0.200pt]{2.409pt}{0.400pt}}
\put(1429.0,829.0){\rule[-0.200pt]{2.409pt}{0.400pt}}
\put(170.0,829.0){\rule[-0.200pt]{2.409pt}{0.400pt}}
\put(1429.0,829.0){\rule[-0.200pt]{2.409pt}{0.400pt}}
\put(170.0,829.0){\rule[-0.200pt]{2.409pt}{0.400pt}}
\put(1429.0,829.0){\rule[-0.200pt]{2.409pt}{0.400pt}}
\put(170.0,829.0){\rule[-0.200pt]{2.409pt}{0.400pt}}
\put(1429.0,829.0){\rule[-0.200pt]{2.409pt}{0.400pt}}
\put(170.0,829.0){\rule[-0.200pt]{2.409pt}{0.400pt}}
\put(1429.0,829.0){\rule[-0.200pt]{2.409pt}{0.400pt}}
\put(170.0,829.0){\rule[-0.200pt]{2.409pt}{0.400pt}}
\put(1429.0,829.0){\rule[-0.200pt]{2.409pt}{0.400pt}}
\put(170.0,829.0){\rule[-0.200pt]{2.409pt}{0.400pt}}
\put(1429.0,829.0){\rule[-0.200pt]{2.409pt}{0.400pt}}
\put(170.0,829.0){\rule[-0.200pt]{2.409pt}{0.400pt}}
\put(1429.0,829.0){\rule[-0.200pt]{2.409pt}{0.400pt}}
\put(170.0,830.0){\rule[-0.200pt]{2.409pt}{0.400pt}}
\put(1429.0,830.0){\rule[-0.200pt]{2.409pt}{0.400pt}}
\put(170.0,830.0){\rule[-0.200pt]{2.409pt}{0.400pt}}
\put(1429.0,830.0){\rule[-0.200pt]{2.409pt}{0.400pt}}
\put(170.0,830.0){\rule[-0.200pt]{2.409pt}{0.400pt}}
\put(1429.0,830.0){\rule[-0.200pt]{2.409pt}{0.400pt}}
\put(170.0,830.0){\rule[-0.200pt]{2.409pt}{0.400pt}}
\put(1429.0,830.0){\rule[-0.200pt]{2.409pt}{0.400pt}}
\put(170.0,830.0){\rule[-0.200pt]{2.409pt}{0.400pt}}
\put(1429.0,830.0){\rule[-0.200pt]{2.409pt}{0.400pt}}
\put(170.0,830.0){\rule[-0.200pt]{2.409pt}{0.400pt}}
\put(1429.0,830.0){\rule[-0.200pt]{2.409pt}{0.400pt}}
\put(170.0,830.0){\rule[-0.200pt]{2.409pt}{0.400pt}}
\put(1429.0,830.0){\rule[-0.200pt]{2.409pt}{0.400pt}}
\put(170.0,830.0){\rule[-0.200pt]{2.409pt}{0.400pt}}
\put(1429.0,830.0){\rule[-0.200pt]{2.409pt}{0.400pt}}
\put(170.0,830.0){\rule[-0.200pt]{2.409pt}{0.400pt}}
\put(1429.0,830.0){\rule[-0.200pt]{2.409pt}{0.400pt}}
\put(170.0,830.0){\rule[-0.200pt]{2.409pt}{0.400pt}}
\put(1429.0,830.0){\rule[-0.200pt]{2.409pt}{0.400pt}}
\put(170.0,830.0){\rule[-0.200pt]{2.409pt}{0.400pt}}
\put(1429.0,830.0){\rule[-0.200pt]{2.409pt}{0.400pt}}
\put(170.0,830.0){\rule[-0.200pt]{2.409pt}{0.400pt}}
\put(1429.0,830.0){\rule[-0.200pt]{2.409pt}{0.400pt}}
\put(170.0,831.0){\rule[-0.200pt]{2.409pt}{0.400pt}}
\put(1429.0,831.0){\rule[-0.200pt]{2.409pt}{0.400pt}}
\put(170.0,831.0){\rule[-0.200pt]{2.409pt}{0.400pt}}
\put(1429.0,831.0){\rule[-0.200pt]{2.409pt}{0.400pt}}
\put(170.0,831.0){\rule[-0.200pt]{2.409pt}{0.400pt}}
\put(1429.0,831.0){\rule[-0.200pt]{2.409pt}{0.400pt}}
\put(170.0,831.0){\rule[-0.200pt]{2.409pt}{0.400pt}}
\put(1429.0,831.0){\rule[-0.200pt]{2.409pt}{0.400pt}}
\put(170.0,831.0){\rule[-0.200pt]{2.409pt}{0.400pt}}
\put(1429.0,831.0){\rule[-0.200pt]{2.409pt}{0.400pt}}
\put(170.0,831.0){\rule[-0.200pt]{2.409pt}{0.400pt}}
\put(1429.0,831.0){\rule[-0.200pt]{2.409pt}{0.400pt}}
\put(170.0,831.0){\rule[-0.200pt]{2.409pt}{0.400pt}}
\put(1429.0,831.0){\rule[-0.200pt]{2.409pt}{0.400pt}}
\put(170.0,831.0){\rule[-0.200pt]{2.409pt}{0.400pt}}
\put(1429.0,831.0){\rule[-0.200pt]{2.409pt}{0.400pt}}
\put(170.0,831.0){\rule[-0.200pt]{2.409pt}{0.400pt}}
\put(1429.0,831.0){\rule[-0.200pt]{2.409pt}{0.400pt}}
\put(170.0,831.0){\rule[-0.200pt]{2.409pt}{0.400pt}}
\put(1429.0,831.0){\rule[-0.200pt]{2.409pt}{0.400pt}}
\put(170.0,831.0){\rule[-0.200pt]{2.409pt}{0.400pt}}
\put(1429.0,831.0){\rule[-0.200pt]{2.409pt}{0.400pt}}
\put(170.0,831.0){\rule[-0.200pt]{2.409pt}{0.400pt}}
\put(1429.0,831.0){\rule[-0.200pt]{2.409pt}{0.400pt}}
\put(170.0,831.0){\rule[-0.200pt]{2.409pt}{0.400pt}}
\put(1429.0,831.0){\rule[-0.200pt]{2.409pt}{0.400pt}}
\put(170.0,832.0){\rule[-0.200pt]{2.409pt}{0.400pt}}
\put(1429.0,832.0){\rule[-0.200pt]{2.409pt}{0.400pt}}
\put(170.0,832.0){\rule[-0.200pt]{2.409pt}{0.400pt}}
\put(1429.0,832.0){\rule[-0.200pt]{2.409pt}{0.400pt}}
\put(170.0,832.0){\rule[-0.200pt]{2.409pt}{0.400pt}}
\put(1429.0,832.0){\rule[-0.200pt]{2.409pt}{0.400pt}}
\put(170.0,832.0){\rule[-0.200pt]{2.409pt}{0.400pt}}
\put(1429.0,832.0){\rule[-0.200pt]{2.409pt}{0.400pt}}
\put(170.0,832.0){\rule[-0.200pt]{2.409pt}{0.400pt}}
\put(1429.0,832.0){\rule[-0.200pt]{2.409pt}{0.400pt}}
\put(170.0,832.0){\rule[-0.200pt]{2.409pt}{0.400pt}}
\put(1429.0,832.0){\rule[-0.200pt]{2.409pt}{0.400pt}}
\put(170.0,832.0){\rule[-0.200pt]{2.409pt}{0.400pt}}
\put(1429.0,832.0){\rule[-0.200pt]{2.409pt}{0.400pt}}
\put(170.0,832.0){\rule[-0.200pt]{2.409pt}{0.400pt}}
\put(1429.0,832.0){\rule[-0.200pt]{2.409pt}{0.400pt}}
\put(170.0,832.0){\rule[-0.200pt]{2.409pt}{0.400pt}}
\put(1429.0,832.0){\rule[-0.200pt]{2.409pt}{0.400pt}}
\put(170.0,832.0){\rule[-0.200pt]{2.409pt}{0.400pt}}
\put(1429.0,832.0){\rule[-0.200pt]{2.409pt}{0.400pt}}
\put(170.0,832.0){\rule[-0.200pt]{2.409pt}{0.400pt}}
\put(1429.0,832.0){\rule[-0.200pt]{2.409pt}{0.400pt}}
\put(170.0,832.0){\rule[-0.200pt]{2.409pt}{0.400pt}}
\put(1429.0,832.0){\rule[-0.200pt]{2.409pt}{0.400pt}}
\put(170.0,832.0){\rule[-0.200pt]{2.409pt}{0.400pt}}
\put(1429.0,832.0){\rule[-0.200pt]{2.409pt}{0.400pt}}
\put(170.0,833.0){\rule[-0.200pt]{2.409pt}{0.400pt}}
\put(1429.0,833.0){\rule[-0.200pt]{2.409pt}{0.400pt}}
\put(170.0,833.0){\rule[-0.200pt]{2.409pt}{0.400pt}}
\put(1429.0,833.0){\rule[-0.200pt]{2.409pt}{0.400pt}}
\put(170.0,833.0){\rule[-0.200pt]{2.409pt}{0.400pt}}
\put(1429.0,833.0){\rule[-0.200pt]{2.409pt}{0.400pt}}
\put(170.0,833.0){\rule[-0.200pt]{2.409pt}{0.400pt}}
\put(1429.0,833.0){\rule[-0.200pt]{2.409pt}{0.400pt}}
\put(170.0,833.0){\rule[-0.200pt]{2.409pt}{0.400pt}}
\put(1429.0,833.0){\rule[-0.200pt]{2.409pt}{0.400pt}}
\put(170.0,833.0){\rule[-0.200pt]{2.409pt}{0.400pt}}
\put(1429.0,833.0){\rule[-0.200pt]{2.409pt}{0.400pt}}
\put(170.0,833.0){\rule[-0.200pt]{2.409pt}{0.400pt}}
\put(1429.0,833.0){\rule[-0.200pt]{2.409pt}{0.400pt}}
\put(170.0,833.0){\rule[-0.200pt]{2.409pt}{0.400pt}}
\put(1429.0,833.0){\rule[-0.200pt]{2.409pt}{0.400pt}}
\put(170.0,833.0){\rule[-0.200pt]{2.409pt}{0.400pt}}
\put(1429.0,833.0){\rule[-0.200pt]{2.409pt}{0.400pt}}
\put(170.0,833.0){\rule[-0.200pt]{2.409pt}{0.400pt}}
\put(1429.0,833.0){\rule[-0.200pt]{2.409pt}{0.400pt}}
\put(170.0,833.0){\rule[-0.200pt]{2.409pt}{0.400pt}}
\put(1429.0,833.0){\rule[-0.200pt]{2.409pt}{0.400pt}}
\put(170.0,833.0){\rule[-0.200pt]{2.409pt}{0.400pt}}
\put(1429.0,833.0){\rule[-0.200pt]{2.409pt}{0.400pt}}
\put(170.0,833.0){\rule[-0.200pt]{2.409pt}{0.400pt}}
\put(1429.0,833.0){\rule[-0.200pt]{2.409pt}{0.400pt}}
\put(170.0,834.0){\rule[-0.200pt]{2.409pt}{0.400pt}}
\put(1429.0,834.0){\rule[-0.200pt]{2.409pt}{0.400pt}}
\put(170.0,834.0){\rule[-0.200pt]{2.409pt}{0.400pt}}
\put(1429.0,834.0){\rule[-0.200pt]{2.409pt}{0.400pt}}
\put(170.0,834.0){\rule[-0.200pt]{2.409pt}{0.400pt}}
\put(1429.0,834.0){\rule[-0.200pt]{2.409pt}{0.400pt}}
\put(170.0,834.0){\rule[-0.200pt]{2.409pt}{0.400pt}}
\put(1429.0,834.0){\rule[-0.200pt]{2.409pt}{0.400pt}}
\put(170.0,834.0){\rule[-0.200pt]{2.409pt}{0.400pt}}
\put(1429.0,834.0){\rule[-0.200pt]{2.409pt}{0.400pt}}
\put(170.0,834.0){\rule[-0.200pt]{2.409pt}{0.400pt}}
\put(1429.0,834.0){\rule[-0.200pt]{2.409pt}{0.400pt}}
\put(170.0,834.0){\rule[-0.200pt]{2.409pt}{0.400pt}}
\put(1429.0,834.0){\rule[-0.200pt]{2.409pt}{0.400pt}}
\put(170.0,834.0){\rule[-0.200pt]{2.409pt}{0.400pt}}
\put(1429.0,834.0){\rule[-0.200pt]{2.409pt}{0.400pt}}
\put(170.0,834.0){\rule[-0.200pt]{2.409pt}{0.400pt}}
\put(1429.0,834.0){\rule[-0.200pt]{2.409pt}{0.400pt}}
\put(170.0,834.0){\rule[-0.200pt]{2.409pt}{0.400pt}}
\put(1429.0,834.0){\rule[-0.200pt]{2.409pt}{0.400pt}}
\put(170.0,834.0){\rule[-0.200pt]{2.409pt}{0.400pt}}
\put(1429.0,834.0){\rule[-0.200pt]{2.409pt}{0.400pt}}
\put(170.0,834.0){\rule[-0.200pt]{2.409pt}{0.400pt}}
\put(1429.0,834.0){\rule[-0.200pt]{2.409pt}{0.400pt}}
\put(170.0,834.0){\rule[-0.200pt]{2.409pt}{0.400pt}}
\put(1429.0,834.0){\rule[-0.200pt]{2.409pt}{0.400pt}}
\put(170.0,834.0){\rule[-0.200pt]{2.409pt}{0.400pt}}
\put(1429.0,834.0){\rule[-0.200pt]{2.409pt}{0.400pt}}
\put(170.0,835.0){\rule[-0.200pt]{2.409pt}{0.400pt}}
\put(1429.0,835.0){\rule[-0.200pt]{2.409pt}{0.400pt}}
\put(170.0,835.0){\rule[-0.200pt]{2.409pt}{0.400pt}}
\put(1429.0,835.0){\rule[-0.200pt]{2.409pt}{0.400pt}}
\put(170.0,835.0){\rule[-0.200pt]{2.409pt}{0.400pt}}
\put(1429.0,835.0){\rule[-0.200pt]{2.409pt}{0.400pt}}
\put(170.0,835.0){\rule[-0.200pt]{2.409pt}{0.400pt}}
\put(1429.0,835.0){\rule[-0.200pt]{2.409pt}{0.400pt}}
\put(170.0,835.0){\rule[-0.200pt]{2.409pt}{0.400pt}}
\put(1429.0,835.0){\rule[-0.200pt]{2.409pt}{0.400pt}}
\put(170.0,835.0){\rule[-0.200pt]{2.409pt}{0.400pt}}
\put(1429.0,835.0){\rule[-0.200pt]{2.409pt}{0.400pt}}
\put(170.0,835.0){\rule[-0.200pt]{2.409pt}{0.400pt}}
\put(1429.0,835.0){\rule[-0.200pt]{2.409pt}{0.400pt}}
\put(170.0,835.0){\rule[-0.200pt]{2.409pt}{0.400pt}}
\put(1429.0,835.0){\rule[-0.200pt]{2.409pt}{0.400pt}}
\put(170.0,835.0){\rule[-0.200pt]{2.409pt}{0.400pt}}
\put(1429.0,835.0){\rule[-0.200pt]{2.409pt}{0.400pt}}
\put(170.0,835.0){\rule[-0.200pt]{2.409pt}{0.400pt}}
\put(1429.0,835.0){\rule[-0.200pt]{2.409pt}{0.400pt}}
\put(170.0,835.0){\rule[-0.200pt]{2.409pt}{0.400pt}}
\put(1429.0,835.0){\rule[-0.200pt]{2.409pt}{0.400pt}}
\put(170.0,835.0){\rule[-0.200pt]{2.409pt}{0.400pt}}
\put(1429.0,835.0){\rule[-0.200pt]{2.409pt}{0.400pt}}
\put(170.0,835.0){\rule[-0.200pt]{2.409pt}{0.400pt}}
\put(1429.0,835.0){\rule[-0.200pt]{2.409pt}{0.400pt}}
\put(170.0,835.0){\rule[-0.200pt]{2.409pt}{0.400pt}}
\put(1429.0,835.0){\rule[-0.200pt]{2.409pt}{0.400pt}}
\put(170.0,836.0){\rule[-0.200pt]{2.409pt}{0.400pt}}
\put(1429.0,836.0){\rule[-0.200pt]{2.409pt}{0.400pt}}
\put(170.0,836.0){\rule[-0.200pt]{2.409pt}{0.400pt}}
\put(1429.0,836.0){\rule[-0.200pt]{2.409pt}{0.400pt}}
\put(170.0,836.0){\rule[-0.200pt]{2.409pt}{0.400pt}}
\put(1429.0,836.0){\rule[-0.200pt]{2.409pt}{0.400pt}}
\put(170.0,836.0){\rule[-0.200pt]{2.409pt}{0.400pt}}
\put(1429.0,836.0){\rule[-0.200pt]{2.409pt}{0.400pt}}
\put(170.0,836.0){\rule[-0.200pt]{2.409pt}{0.400pt}}
\put(1429.0,836.0){\rule[-0.200pt]{2.409pt}{0.400pt}}
\put(170.0,836.0){\rule[-0.200pt]{2.409pt}{0.400pt}}
\put(1429.0,836.0){\rule[-0.200pt]{2.409pt}{0.400pt}}
\put(170.0,836.0){\rule[-0.200pt]{2.409pt}{0.400pt}}
\put(1429.0,836.0){\rule[-0.200pt]{2.409pt}{0.400pt}}
\put(170.0,836.0){\rule[-0.200pt]{2.409pt}{0.400pt}}
\put(1429.0,836.0){\rule[-0.200pt]{2.409pt}{0.400pt}}
\put(170.0,836.0){\rule[-0.200pt]{2.409pt}{0.400pt}}
\put(1429.0,836.0){\rule[-0.200pt]{2.409pt}{0.400pt}}
\put(170.0,836.0){\rule[-0.200pt]{2.409pt}{0.400pt}}
\put(1429.0,836.0){\rule[-0.200pt]{2.409pt}{0.400pt}}
\put(170.0,836.0){\rule[-0.200pt]{2.409pt}{0.400pt}}
\put(1429.0,836.0){\rule[-0.200pt]{2.409pt}{0.400pt}}
\put(170.0,836.0){\rule[-0.200pt]{2.409pt}{0.400pt}}
\put(1429.0,836.0){\rule[-0.200pt]{2.409pt}{0.400pt}}
\put(170.0,836.0){\rule[-0.200pt]{2.409pt}{0.400pt}}
\put(1429.0,836.0){\rule[-0.200pt]{2.409pt}{0.400pt}}
\put(170.0,836.0){\rule[-0.200pt]{2.409pt}{0.400pt}}
\put(1429.0,836.0){\rule[-0.200pt]{2.409pt}{0.400pt}}
\put(170.0,837.0){\rule[-0.200pt]{2.409pt}{0.400pt}}
\put(1429.0,837.0){\rule[-0.200pt]{2.409pt}{0.400pt}}
\put(170.0,837.0){\rule[-0.200pt]{2.409pt}{0.400pt}}
\put(1429.0,837.0){\rule[-0.200pt]{2.409pt}{0.400pt}}
\put(170.0,837.0){\rule[-0.200pt]{2.409pt}{0.400pt}}
\put(1429.0,837.0){\rule[-0.200pt]{2.409pt}{0.400pt}}
\put(170.0,837.0){\rule[-0.200pt]{2.409pt}{0.400pt}}
\put(1429.0,837.0){\rule[-0.200pt]{2.409pt}{0.400pt}}
\put(170.0,837.0){\rule[-0.200pt]{2.409pt}{0.400pt}}
\put(1429.0,837.0){\rule[-0.200pt]{2.409pt}{0.400pt}}
\put(170.0,837.0){\rule[-0.200pt]{2.409pt}{0.400pt}}
\put(1429.0,837.0){\rule[-0.200pt]{2.409pt}{0.400pt}}
\put(170.0,837.0){\rule[-0.200pt]{2.409pt}{0.400pt}}
\put(1429.0,837.0){\rule[-0.200pt]{2.409pt}{0.400pt}}
\put(170.0,837.0){\rule[-0.200pt]{2.409pt}{0.400pt}}
\put(1429.0,837.0){\rule[-0.200pt]{2.409pt}{0.400pt}}
\put(170.0,837.0){\rule[-0.200pt]{2.409pt}{0.400pt}}
\put(1429.0,837.0){\rule[-0.200pt]{2.409pt}{0.400pt}}
\put(170.0,837.0){\rule[-0.200pt]{2.409pt}{0.400pt}}
\put(1429.0,837.0){\rule[-0.200pt]{2.409pt}{0.400pt}}
\put(170.0,837.0){\rule[-0.200pt]{2.409pt}{0.400pt}}
\put(1429.0,837.0){\rule[-0.200pt]{2.409pt}{0.400pt}}
\put(170.0,837.0){\rule[-0.200pt]{2.409pt}{0.400pt}}
\put(1429.0,837.0){\rule[-0.200pt]{2.409pt}{0.400pt}}
\put(170.0,837.0){\rule[-0.200pt]{2.409pt}{0.400pt}}
\put(1429.0,837.0){\rule[-0.200pt]{2.409pt}{0.400pt}}
\put(170.0,837.0){\rule[-0.200pt]{2.409pt}{0.400pt}}
\put(1429.0,837.0){\rule[-0.200pt]{2.409pt}{0.400pt}}
\put(170.0,837.0){\rule[-0.200pt]{2.409pt}{0.400pt}}
\put(1429.0,837.0){\rule[-0.200pt]{2.409pt}{0.400pt}}
\put(170.0,838.0){\rule[-0.200pt]{2.409pt}{0.400pt}}
\put(1429.0,838.0){\rule[-0.200pt]{2.409pt}{0.400pt}}
\put(170.0,838.0){\rule[-0.200pt]{2.409pt}{0.400pt}}
\put(1429.0,838.0){\rule[-0.200pt]{2.409pt}{0.400pt}}
\put(170.0,838.0){\rule[-0.200pt]{2.409pt}{0.400pt}}
\put(1429.0,838.0){\rule[-0.200pt]{2.409pt}{0.400pt}}
\put(170.0,838.0){\rule[-0.200pt]{2.409pt}{0.400pt}}
\put(1429.0,838.0){\rule[-0.200pt]{2.409pt}{0.400pt}}
\put(170.0,838.0){\rule[-0.200pt]{2.409pt}{0.400pt}}
\put(1429.0,838.0){\rule[-0.200pt]{2.409pt}{0.400pt}}
\put(170.0,838.0){\rule[-0.200pt]{2.409pt}{0.400pt}}
\put(1429.0,838.0){\rule[-0.200pt]{2.409pt}{0.400pt}}
\put(170.0,838.0){\rule[-0.200pt]{2.409pt}{0.400pt}}
\put(1429.0,838.0){\rule[-0.200pt]{2.409pt}{0.400pt}}
\put(170.0,838.0){\rule[-0.200pt]{2.409pt}{0.400pt}}
\put(1429.0,838.0){\rule[-0.200pt]{2.409pt}{0.400pt}}
\put(170.0,838.0){\rule[-0.200pt]{2.409pt}{0.400pt}}
\put(1429.0,838.0){\rule[-0.200pt]{2.409pt}{0.400pt}}
\put(170.0,838.0){\rule[-0.200pt]{2.409pt}{0.400pt}}
\put(1429.0,838.0){\rule[-0.200pt]{2.409pt}{0.400pt}}
\put(170.0,838.0){\rule[-0.200pt]{2.409pt}{0.400pt}}
\put(1429.0,838.0){\rule[-0.200pt]{2.409pt}{0.400pt}}
\put(170.0,838.0){\rule[-0.200pt]{2.409pt}{0.400pt}}
\put(1429.0,838.0){\rule[-0.200pt]{2.409pt}{0.400pt}}
\put(170.0,838.0){\rule[-0.200pt]{2.409pt}{0.400pt}}
\put(1429.0,838.0){\rule[-0.200pt]{2.409pt}{0.400pt}}
\put(170.0,838.0){\rule[-0.200pt]{2.409pt}{0.400pt}}
\put(1429.0,838.0){\rule[-0.200pt]{2.409pt}{0.400pt}}
\put(170.0,838.0){\rule[-0.200pt]{2.409pt}{0.400pt}}
\put(1429.0,838.0){\rule[-0.200pt]{2.409pt}{0.400pt}}
\put(170.0,839.0){\rule[-0.200pt]{2.409pt}{0.400pt}}
\put(1429.0,839.0){\rule[-0.200pt]{2.409pt}{0.400pt}}
\put(170.0,839.0){\rule[-0.200pt]{2.409pt}{0.400pt}}
\put(1429.0,839.0){\rule[-0.200pt]{2.409pt}{0.400pt}}
\put(170.0,839.0){\rule[-0.200pt]{2.409pt}{0.400pt}}
\put(1429.0,839.0){\rule[-0.200pt]{2.409pt}{0.400pt}}
\put(170.0,839.0){\rule[-0.200pt]{2.409pt}{0.400pt}}
\put(1429.0,839.0){\rule[-0.200pt]{2.409pt}{0.400pt}}
\put(170.0,839.0){\rule[-0.200pt]{2.409pt}{0.400pt}}
\put(1429.0,839.0){\rule[-0.200pt]{2.409pt}{0.400pt}}
\put(170.0,839.0){\rule[-0.200pt]{2.409pt}{0.400pt}}
\put(1429.0,839.0){\rule[-0.200pt]{2.409pt}{0.400pt}}
\put(170.0,839.0){\rule[-0.200pt]{2.409pt}{0.400pt}}
\put(1429.0,839.0){\rule[-0.200pt]{2.409pt}{0.400pt}}
\put(170.0,839.0){\rule[-0.200pt]{2.409pt}{0.400pt}}
\put(1429.0,839.0){\rule[-0.200pt]{2.409pt}{0.400pt}}
\put(170.0,839.0){\rule[-0.200pt]{2.409pt}{0.400pt}}
\put(1429.0,839.0){\rule[-0.200pt]{2.409pt}{0.400pt}}
\put(170.0,839.0){\rule[-0.200pt]{2.409pt}{0.400pt}}
\put(1429.0,839.0){\rule[-0.200pt]{2.409pt}{0.400pt}}
\put(170.0,839.0){\rule[-0.200pt]{2.409pt}{0.400pt}}
\put(1429.0,839.0){\rule[-0.200pt]{2.409pt}{0.400pt}}
\put(170.0,839.0){\rule[-0.200pt]{2.409pt}{0.400pt}}
\put(1429.0,839.0){\rule[-0.200pt]{2.409pt}{0.400pt}}
\put(170.0,839.0){\rule[-0.200pt]{2.409pt}{0.400pt}}
\put(1429.0,839.0){\rule[-0.200pt]{2.409pt}{0.400pt}}
\put(170.0,839.0){\rule[-0.200pt]{2.409pt}{0.400pt}}
\put(1429.0,839.0){\rule[-0.200pt]{2.409pt}{0.400pt}}
\put(170.0,839.0){\rule[-0.200pt]{2.409pt}{0.400pt}}
\put(1429.0,839.0){\rule[-0.200pt]{2.409pt}{0.400pt}}
\put(170.0,839.0){\rule[-0.200pt]{2.409pt}{0.400pt}}
\put(1429.0,839.0){\rule[-0.200pt]{2.409pt}{0.400pt}}
\put(170.0,840.0){\rule[-0.200pt]{2.409pt}{0.400pt}}
\put(1429.0,840.0){\rule[-0.200pt]{2.409pt}{0.400pt}}
\put(170.0,840.0){\rule[-0.200pt]{2.409pt}{0.400pt}}
\put(1429.0,840.0){\rule[-0.200pt]{2.409pt}{0.400pt}}
\put(170.0,840.0){\rule[-0.200pt]{2.409pt}{0.400pt}}
\put(1429.0,840.0){\rule[-0.200pt]{2.409pt}{0.400pt}}
\put(170.0,840.0){\rule[-0.200pt]{2.409pt}{0.400pt}}
\put(1429.0,840.0){\rule[-0.200pt]{2.409pt}{0.400pt}}
\put(170.0,840.0){\rule[-0.200pt]{2.409pt}{0.400pt}}
\put(1429.0,840.0){\rule[-0.200pt]{2.409pt}{0.400pt}}
\put(170.0,840.0){\rule[-0.200pt]{2.409pt}{0.400pt}}
\put(1429.0,840.0){\rule[-0.200pt]{2.409pt}{0.400pt}}
\put(170.0,840.0){\rule[-0.200pt]{2.409pt}{0.400pt}}
\put(1429.0,840.0){\rule[-0.200pt]{2.409pt}{0.400pt}}
\put(170.0,840.0){\rule[-0.200pt]{2.409pt}{0.400pt}}
\put(1429.0,840.0){\rule[-0.200pt]{2.409pt}{0.400pt}}
\put(170.0,840.0){\rule[-0.200pt]{2.409pt}{0.400pt}}
\put(1429.0,840.0){\rule[-0.200pt]{2.409pt}{0.400pt}}
\put(170.0,840.0){\rule[-0.200pt]{2.409pt}{0.400pt}}
\put(1429.0,840.0){\rule[-0.200pt]{2.409pt}{0.400pt}}
\put(170.0,840.0){\rule[-0.200pt]{2.409pt}{0.400pt}}
\put(1429.0,840.0){\rule[-0.200pt]{2.409pt}{0.400pt}}
\put(170.0,840.0){\rule[-0.200pt]{2.409pt}{0.400pt}}
\put(1429.0,840.0){\rule[-0.200pt]{2.409pt}{0.400pt}}
\put(170.0,840.0){\rule[-0.200pt]{2.409pt}{0.400pt}}
\put(1429.0,840.0){\rule[-0.200pt]{2.409pt}{0.400pt}}
\put(170.0,840.0){\rule[-0.200pt]{2.409pt}{0.400pt}}
\put(1429.0,840.0){\rule[-0.200pt]{2.409pt}{0.400pt}}
\put(170.0,840.0){\rule[-0.200pt]{2.409pt}{0.400pt}}
\put(1429.0,840.0){\rule[-0.200pt]{2.409pt}{0.400pt}}
\put(170.0,840.0){\rule[-0.200pt]{2.409pt}{0.400pt}}
\put(1429.0,840.0){\rule[-0.200pt]{2.409pt}{0.400pt}}
\put(170.0,841.0){\rule[-0.200pt]{2.409pt}{0.400pt}}
\put(1429.0,841.0){\rule[-0.200pt]{2.409pt}{0.400pt}}
\put(170.0,841.0){\rule[-0.200pt]{2.409pt}{0.400pt}}
\put(1429.0,841.0){\rule[-0.200pt]{2.409pt}{0.400pt}}
\put(170.0,841.0){\rule[-0.200pt]{2.409pt}{0.400pt}}
\put(1429.0,841.0){\rule[-0.200pt]{2.409pt}{0.400pt}}
\put(170.0,841.0){\rule[-0.200pt]{2.409pt}{0.400pt}}
\put(1429.0,841.0){\rule[-0.200pt]{2.409pt}{0.400pt}}
\put(170.0,841.0){\rule[-0.200pt]{2.409pt}{0.400pt}}
\put(1429.0,841.0){\rule[-0.200pt]{2.409pt}{0.400pt}}
\put(170.0,841.0){\rule[-0.200pt]{2.409pt}{0.400pt}}
\put(1429.0,841.0){\rule[-0.200pt]{2.409pt}{0.400pt}}
\put(170.0,841.0){\rule[-0.200pt]{2.409pt}{0.400pt}}
\put(1429.0,841.0){\rule[-0.200pt]{2.409pt}{0.400pt}}
\put(170.0,841.0){\rule[-0.200pt]{2.409pt}{0.400pt}}
\put(1429.0,841.0){\rule[-0.200pt]{2.409pt}{0.400pt}}
\put(170.0,841.0){\rule[-0.200pt]{2.409pt}{0.400pt}}
\put(1429.0,841.0){\rule[-0.200pt]{2.409pt}{0.400pt}}
\put(170.0,841.0){\rule[-0.200pt]{2.409pt}{0.400pt}}
\put(1429.0,841.0){\rule[-0.200pt]{2.409pt}{0.400pt}}
\put(170.0,841.0){\rule[-0.200pt]{2.409pt}{0.400pt}}
\put(1429.0,841.0){\rule[-0.200pt]{2.409pt}{0.400pt}}
\put(170.0,841.0){\rule[-0.200pt]{2.409pt}{0.400pt}}
\put(1429.0,841.0){\rule[-0.200pt]{2.409pt}{0.400pt}}
\put(170.0,841.0){\rule[-0.200pt]{2.409pt}{0.400pt}}
\put(1429.0,841.0){\rule[-0.200pt]{2.409pt}{0.400pt}}
\put(170.0,841.0){\rule[-0.200pt]{2.409pt}{0.400pt}}
\put(1429.0,841.0){\rule[-0.200pt]{2.409pt}{0.400pt}}
\put(170.0,841.0){\rule[-0.200pt]{2.409pt}{0.400pt}}
\put(1429.0,841.0){\rule[-0.200pt]{2.409pt}{0.400pt}}
\put(170.0,841.0){\rule[-0.200pt]{2.409pt}{0.400pt}}
\put(1429.0,841.0){\rule[-0.200pt]{2.409pt}{0.400pt}}
\put(170.0,841.0){\rule[-0.200pt]{2.409pt}{0.400pt}}
\put(1429.0,841.0){\rule[-0.200pt]{2.409pt}{0.400pt}}
\put(170.0,842.0){\rule[-0.200pt]{2.409pt}{0.400pt}}
\put(1429.0,842.0){\rule[-0.200pt]{2.409pt}{0.400pt}}
\put(170.0,842.0){\rule[-0.200pt]{2.409pt}{0.400pt}}
\put(1429.0,842.0){\rule[-0.200pt]{2.409pt}{0.400pt}}
\put(170.0,842.0){\rule[-0.200pt]{2.409pt}{0.400pt}}
\put(1429.0,842.0){\rule[-0.200pt]{2.409pt}{0.400pt}}
\put(170.0,842.0){\rule[-0.200pt]{2.409pt}{0.400pt}}
\put(1429.0,842.0){\rule[-0.200pt]{2.409pt}{0.400pt}}
\put(170.0,842.0){\rule[-0.200pt]{2.409pt}{0.400pt}}
\put(1429.0,842.0){\rule[-0.200pt]{2.409pt}{0.400pt}}
\put(170.0,842.0){\rule[-0.200pt]{2.409pt}{0.400pt}}
\put(1429.0,842.0){\rule[-0.200pt]{2.409pt}{0.400pt}}
\put(170.0,842.0){\rule[-0.200pt]{2.409pt}{0.400pt}}
\put(1429.0,842.0){\rule[-0.200pt]{2.409pt}{0.400pt}}
\put(170.0,842.0){\rule[-0.200pt]{2.409pt}{0.400pt}}
\put(1429.0,842.0){\rule[-0.200pt]{2.409pt}{0.400pt}}
\put(170.0,842.0){\rule[-0.200pt]{2.409pt}{0.400pt}}
\put(1429.0,842.0){\rule[-0.200pt]{2.409pt}{0.400pt}}
\put(170.0,842.0){\rule[-0.200pt]{2.409pt}{0.400pt}}
\put(1429.0,842.0){\rule[-0.200pt]{2.409pt}{0.400pt}}
\put(170.0,842.0){\rule[-0.200pt]{2.409pt}{0.400pt}}
\put(1429.0,842.0){\rule[-0.200pt]{2.409pt}{0.400pt}}
\put(170.0,842.0){\rule[-0.200pt]{2.409pt}{0.400pt}}
\put(1429.0,842.0){\rule[-0.200pt]{2.409pt}{0.400pt}}
\put(170.0,842.0){\rule[-0.200pt]{2.409pt}{0.400pt}}
\put(1429.0,842.0){\rule[-0.200pt]{2.409pt}{0.400pt}}
\put(170.0,842.0){\rule[-0.200pt]{2.409pt}{0.400pt}}
\put(1429.0,842.0){\rule[-0.200pt]{2.409pt}{0.400pt}}
\put(170.0,842.0){\rule[-0.200pt]{2.409pt}{0.400pt}}
\put(1429.0,842.0){\rule[-0.200pt]{2.409pt}{0.400pt}}
\put(170.0,842.0){\rule[-0.200pt]{2.409pt}{0.400pt}}
\put(1429.0,842.0){\rule[-0.200pt]{2.409pt}{0.400pt}}
\put(170.0,843.0){\rule[-0.200pt]{2.409pt}{0.400pt}}
\put(1429.0,843.0){\rule[-0.200pt]{2.409pt}{0.400pt}}
\put(170.0,843.0){\rule[-0.200pt]{2.409pt}{0.400pt}}
\put(1429.0,843.0){\rule[-0.200pt]{2.409pt}{0.400pt}}
\put(170.0,843.0){\rule[-0.200pt]{2.409pt}{0.400pt}}
\put(1429.0,843.0){\rule[-0.200pt]{2.409pt}{0.400pt}}
\put(170.0,843.0){\rule[-0.200pt]{2.409pt}{0.400pt}}
\put(1429.0,843.0){\rule[-0.200pt]{2.409pt}{0.400pt}}
\put(170.0,843.0){\rule[-0.200pt]{2.409pt}{0.400pt}}
\put(1429.0,843.0){\rule[-0.200pt]{2.409pt}{0.400pt}}
\put(170.0,843.0){\rule[-0.200pt]{2.409pt}{0.400pt}}
\put(1429.0,843.0){\rule[-0.200pt]{2.409pt}{0.400pt}}
\put(170.0,843.0){\rule[-0.200pt]{2.409pt}{0.400pt}}
\put(1429.0,843.0){\rule[-0.200pt]{2.409pt}{0.400pt}}
\put(170.0,843.0){\rule[-0.200pt]{2.409pt}{0.400pt}}
\put(1429.0,843.0){\rule[-0.200pt]{2.409pt}{0.400pt}}
\put(170.0,843.0){\rule[-0.200pt]{2.409pt}{0.400pt}}
\put(1429.0,843.0){\rule[-0.200pt]{2.409pt}{0.400pt}}
\put(170.0,843.0){\rule[-0.200pt]{2.409pt}{0.400pt}}
\put(1429.0,843.0){\rule[-0.200pt]{2.409pt}{0.400pt}}
\put(170.0,843.0){\rule[-0.200pt]{2.409pt}{0.400pt}}
\put(1429.0,843.0){\rule[-0.200pt]{2.409pt}{0.400pt}}
\put(170.0,843.0){\rule[-0.200pt]{2.409pt}{0.400pt}}
\put(1429.0,843.0){\rule[-0.200pt]{2.409pt}{0.400pt}}
\put(170.0,843.0){\rule[-0.200pt]{2.409pt}{0.400pt}}
\put(1429.0,843.0){\rule[-0.200pt]{2.409pt}{0.400pt}}
\put(170.0,843.0){\rule[-0.200pt]{2.409pt}{0.400pt}}
\put(1429.0,843.0){\rule[-0.200pt]{2.409pt}{0.400pt}}
\put(170.0,843.0){\rule[-0.200pt]{2.409pt}{0.400pt}}
\put(1429.0,843.0){\rule[-0.200pt]{2.409pt}{0.400pt}}
\put(170.0,843.0){\rule[-0.200pt]{2.409pt}{0.400pt}}
\put(1429.0,843.0){\rule[-0.200pt]{2.409pt}{0.400pt}}
\put(170.0,843.0){\rule[-0.200pt]{2.409pt}{0.400pt}}
\put(1429.0,843.0){\rule[-0.200pt]{2.409pt}{0.400pt}}
\put(170.0,843.0){\rule[-0.200pt]{2.409pt}{0.400pt}}
\put(1429.0,843.0){\rule[-0.200pt]{2.409pt}{0.400pt}}
\put(170.0,844.0){\rule[-0.200pt]{2.409pt}{0.400pt}}
\put(1429.0,844.0){\rule[-0.200pt]{2.409pt}{0.400pt}}
\put(170.0,844.0){\rule[-0.200pt]{2.409pt}{0.400pt}}
\put(1429.0,844.0){\rule[-0.200pt]{2.409pt}{0.400pt}}
\put(170.0,844.0){\rule[-0.200pt]{2.409pt}{0.400pt}}
\put(1429.0,844.0){\rule[-0.200pt]{2.409pt}{0.400pt}}
\put(170.0,844.0){\rule[-0.200pt]{2.409pt}{0.400pt}}
\put(1429.0,844.0){\rule[-0.200pt]{2.409pt}{0.400pt}}
\put(170.0,844.0){\rule[-0.200pt]{2.409pt}{0.400pt}}
\put(1429.0,844.0){\rule[-0.200pt]{2.409pt}{0.400pt}}
\put(170.0,844.0){\rule[-0.200pt]{2.409pt}{0.400pt}}
\put(1429.0,844.0){\rule[-0.200pt]{2.409pt}{0.400pt}}
\put(170.0,844.0){\rule[-0.200pt]{2.409pt}{0.400pt}}
\put(1429.0,844.0){\rule[-0.200pt]{2.409pt}{0.400pt}}
\put(170.0,844.0){\rule[-0.200pt]{2.409pt}{0.400pt}}
\put(1429.0,844.0){\rule[-0.200pt]{2.409pt}{0.400pt}}
\put(170.0,844.0){\rule[-0.200pt]{2.409pt}{0.400pt}}
\put(1429.0,844.0){\rule[-0.200pt]{2.409pt}{0.400pt}}
\put(170.0,844.0){\rule[-0.200pt]{2.409pt}{0.400pt}}
\put(1429.0,844.0){\rule[-0.200pt]{2.409pt}{0.400pt}}
\put(170.0,844.0){\rule[-0.200pt]{2.409pt}{0.400pt}}
\put(1429.0,844.0){\rule[-0.200pt]{2.409pt}{0.400pt}}
\put(170.0,844.0){\rule[-0.200pt]{2.409pt}{0.400pt}}
\put(1429.0,844.0){\rule[-0.200pt]{2.409pt}{0.400pt}}
\put(170.0,844.0){\rule[-0.200pt]{2.409pt}{0.400pt}}
\put(1429.0,844.0){\rule[-0.200pt]{2.409pt}{0.400pt}}
\put(170.0,844.0){\rule[-0.200pt]{2.409pt}{0.400pt}}
\put(1429.0,844.0){\rule[-0.200pt]{2.409pt}{0.400pt}}
\put(170.0,844.0){\rule[-0.200pt]{2.409pt}{0.400pt}}
\put(1429.0,844.0){\rule[-0.200pt]{2.409pt}{0.400pt}}
\put(170.0,844.0){\rule[-0.200pt]{2.409pt}{0.400pt}}
\put(1429.0,844.0){\rule[-0.200pt]{2.409pt}{0.400pt}}
\put(170.0,844.0){\rule[-0.200pt]{2.409pt}{0.400pt}}
\put(1429.0,844.0){\rule[-0.200pt]{2.409pt}{0.400pt}}
\put(170.0,844.0){\rule[-0.200pt]{2.409pt}{0.400pt}}
\put(1429.0,844.0){\rule[-0.200pt]{2.409pt}{0.400pt}}
\put(170.0,845.0){\rule[-0.200pt]{2.409pt}{0.400pt}}
\put(1429.0,845.0){\rule[-0.200pt]{2.409pt}{0.400pt}}
\put(170.0,845.0){\rule[-0.200pt]{2.409pt}{0.400pt}}
\put(1429.0,845.0){\rule[-0.200pt]{2.409pt}{0.400pt}}
\put(170.0,845.0){\rule[-0.200pt]{2.409pt}{0.400pt}}
\put(1429.0,845.0){\rule[-0.200pt]{2.409pt}{0.400pt}}
\put(170.0,845.0){\rule[-0.200pt]{2.409pt}{0.400pt}}
\put(1429.0,845.0){\rule[-0.200pt]{2.409pt}{0.400pt}}
\put(170.0,845.0){\rule[-0.200pt]{2.409pt}{0.400pt}}
\put(1429.0,845.0){\rule[-0.200pt]{2.409pt}{0.400pt}}
\put(170.0,845.0){\rule[-0.200pt]{2.409pt}{0.400pt}}
\put(1429.0,845.0){\rule[-0.200pt]{2.409pt}{0.400pt}}
\put(170.0,845.0){\rule[-0.200pt]{2.409pt}{0.400pt}}
\put(1429.0,845.0){\rule[-0.200pt]{2.409pt}{0.400pt}}
\put(170.0,845.0){\rule[-0.200pt]{2.409pt}{0.400pt}}
\put(1429.0,845.0){\rule[-0.200pt]{2.409pt}{0.400pt}}
\put(170.0,845.0){\rule[-0.200pt]{2.409pt}{0.400pt}}
\put(1429.0,845.0){\rule[-0.200pt]{2.409pt}{0.400pt}}
\put(170.0,845.0){\rule[-0.200pt]{2.409pt}{0.400pt}}
\put(1429.0,845.0){\rule[-0.200pt]{2.409pt}{0.400pt}}
\put(170.0,845.0){\rule[-0.200pt]{2.409pt}{0.400pt}}
\put(1429.0,845.0){\rule[-0.200pt]{2.409pt}{0.400pt}}
\put(170.0,845.0){\rule[-0.200pt]{2.409pt}{0.400pt}}
\put(1429.0,845.0){\rule[-0.200pt]{2.409pt}{0.400pt}}
\put(170.0,845.0){\rule[-0.200pt]{2.409pt}{0.400pt}}
\put(1429.0,845.0){\rule[-0.200pt]{2.409pt}{0.400pt}}
\put(170.0,845.0){\rule[-0.200pt]{2.409pt}{0.400pt}}
\put(1429.0,845.0){\rule[-0.200pt]{2.409pt}{0.400pt}}
\put(170.0,845.0){\rule[-0.200pt]{2.409pt}{0.400pt}}
\put(1429.0,845.0){\rule[-0.200pt]{2.409pt}{0.400pt}}
\put(170.0,845.0){\rule[-0.200pt]{2.409pt}{0.400pt}}
\put(1429.0,845.0){\rule[-0.200pt]{2.409pt}{0.400pt}}
\put(170.0,845.0){\rule[-0.200pt]{2.409pt}{0.400pt}}
\put(1429.0,845.0){\rule[-0.200pt]{2.409pt}{0.400pt}}
\put(170.0,845.0){\rule[-0.200pt]{2.409pt}{0.400pt}}
\put(1429.0,845.0){\rule[-0.200pt]{2.409pt}{0.400pt}}
\put(170.0,846.0){\rule[-0.200pt]{2.409pt}{0.400pt}}
\put(1429.0,846.0){\rule[-0.200pt]{2.409pt}{0.400pt}}
\put(170.0,846.0){\rule[-0.200pt]{2.409pt}{0.400pt}}
\put(1429.0,846.0){\rule[-0.200pt]{2.409pt}{0.400pt}}
\put(170.0,846.0){\rule[-0.200pt]{2.409pt}{0.400pt}}
\put(1429.0,846.0){\rule[-0.200pt]{2.409pt}{0.400pt}}
\put(170.0,846.0){\rule[-0.200pt]{2.409pt}{0.400pt}}
\put(1429.0,846.0){\rule[-0.200pt]{2.409pt}{0.400pt}}
\put(170.0,846.0){\rule[-0.200pt]{2.409pt}{0.400pt}}
\put(1429.0,846.0){\rule[-0.200pt]{2.409pt}{0.400pt}}
\put(170.0,846.0){\rule[-0.200pt]{2.409pt}{0.400pt}}
\put(1429.0,846.0){\rule[-0.200pt]{2.409pt}{0.400pt}}
\put(170.0,846.0){\rule[-0.200pt]{2.409pt}{0.400pt}}
\put(1429.0,846.0){\rule[-0.200pt]{2.409pt}{0.400pt}}
\put(170.0,846.0){\rule[-0.200pt]{2.409pt}{0.400pt}}
\put(1429.0,846.0){\rule[-0.200pt]{2.409pt}{0.400pt}}
\put(170.0,846.0){\rule[-0.200pt]{2.409pt}{0.400pt}}
\put(1429.0,846.0){\rule[-0.200pt]{2.409pt}{0.400pt}}
\put(170.0,846.0){\rule[-0.200pt]{2.409pt}{0.400pt}}
\put(1429.0,846.0){\rule[-0.200pt]{2.409pt}{0.400pt}}
\put(170.0,846.0){\rule[-0.200pt]{2.409pt}{0.400pt}}
\put(1429.0,846.0){\rule[-0.200pt]{2.409pt}{0.400pt}}
\put(170.0,846.0){\rule[-0.200pt]{2.409pt}{0.400pt}}
\put(1429.0,846.0){\rule[-0.200pt]{2.409pt}{0.400pt}}
\put(170.0,846.0){\rule[-0.200pt]{2.409pt}{0.400pt}}
\put(1429.0,846.0){\rule[-0.200pt]{2.409pt}{0.400pt}}
\put(170.0,846.0){\rule[-0.200pt]{2.409pt}{0.400pt}}
\put(1429.0,846.0){\rule[-0.200pt]{2.409pt}{0.400pt}}
\put(170.0,846.0){\rule[-0.200pt]{2.409pt}{0.400pt}}
\put(1429.0,846.0){\rule[-0.200pt]{2.409pt}{0.400pt}}
\put(170.0,846.0){\rule[-0.200pt]{2.409pt}{0.400pt}}
\put(1429.0,846.0){\rule[-0.200pt]{2.409pt}{0.400pt}}
\put(170.0,846.0){\rule[-0.200pt]{2.409pt}{0.400pt}}
\put(1429.0,846.0){\rule[-0.200pt]{2.409pt}{0.400pt}}
\put(170.0,846.0){\rule[-0.200pt]{2.409pt}{0.400pt}}
\put(1429.0,846.0){\rule[-0.200pt]{2.409pt}{0.400pt}}
\put(170.0,846.0){\rule[-0.200pt]{2.409pt}{0.400pt}}
\put(1429.0,846.0){\rule[-0.200pt]{2.409pt}{0.400pt}}
\put(170.0,847.0){\rule[-0.200pt]{2.409pt}{0.400pt}}
\put(1429.0,847.0){\rule[-0.200pt]{2.409pt}{0.400pt}}
\put(170.0,847.0){\rule[-0.200pt]{2.409pt}{0.400pt}}
\put(1429.0,847.0){\rule[-0.200pt]{2.409pt}{0.400pt}}
\put(170.0,847.0){\rule[-0.200pt]{2.409pt}{0.400pt}}
\put(1429.0,847.0){\rule[-0.200pt]{2.409pt}{0.400pt}}
\put(170.0,847.0){\rule[-0.200pt]{2.409pt}{0.400pt}}
\put(1429.0,847.0){\rule[-0.200pt]{2.409pt}{0.400pt}}
\put(170.0,847.0){\rule[-0.200pt]{2.409pt}{0.400pt}}
\put(1429.0,847.0){\rule[-0.200pt]{2.409pt}{0.400pt}}
\put(170.0,847.0){\rule[-0.200pt]{2.409pt}{0.400pt}}
\put(1429.0,847.0){\rule[-0.200pt]{2.409pt}{0.400pt}}
\put(170.0,847.0){\rule[-0.200pt]{2.409pt}{0.400pt}}
\put(1429.0,847.0){\rule[-0.200pt]{2.409pt}{0.400pt}}
\put(170.0,847.0){\rule[-0.200pt]{2.409pt}{0.400pt}}
\put(1429.0,847.0){\rule[-0.200pt]{2.409pt}{0.400pt}}
\put(170.0,847.0){\rule[-0.200pt]{2.409pt}{0.400pt}}
\put(1429.0,847.0){\rule[-0.200pt]{2.409pt}{0.400pt}}
\put(170.0,847.0){\rule[-0.200pt]{2.409pt}{0.400pt}}
\put(1429.0,847.0){\rule[-0.200pt]{2.409pt}{0.400pt}}
\put(170.0,847.0){\rule[-0.200pt]{2.409pt}{0.400pt}}
\put(1429.0,847.0){\rule[-0.200pt]{2.409pt}{0.400pt}}
\put(170.0,847.0){\rule[-0.200pt]{2.409pt}{0.400pt}}
\put(1429.0,847.0){\rule[-0.200pt]{2.409pt}{0.400pt}}
\put(170.0,847.0){\rule[-0.200pt]{2.409pt}{0.400pt}}
\put(1429.0,847.0){\rule[-0.200pt]{2.409pt}{0.400pt}}
\put(170.0,847.0){\rule[-0.200pt]{2.409pt}{0.400pt}}
\put(1429.0,847.0){\rule[-0.200pt]{2.409pt}{0.400pt}}
\put(170.0,847.0){\rule[-0.200pt]{2.409pt}{0.400pt}}
\put(1429.0,847.0){\rule[-0.200pt]{2.409pt}{0.400pt}}
\put(170.0,847.0){\rule[-0.200pt]{2.409pt}{0.400pt}}
\put(1429.0,847.0){\rule[-0.200pt]{2.409pt}{0.400pt}}
\put(170.0,847.0){\rule[-0.200pt]{2.409pt}{0.400pt}}
\put(1429.0,847.0){\rule[-0.200pt]{2.409pt}{0.400pt}}
\put(170.0,847.0){\rule[-0.200pt]{2.409pt}{0.400pt}}
\put(1429.0,847.0){\rule[-0.200pt]{2.409pt}{0.400pt}}
\put(170.0,847.0){\rule[-0.200pt]{2.409pt}{0.400pt}}
\put(1429.0,847.0){\rule[-0.200pt]{2.409pt}{0.400pt}}
\put(170.0,848.0){\rule[-0.200pt]{2.409pt}{0.400pt}}
\put(1429.0,848.0){\rule[-0.200pt]{2.409pt}{0.400pt}}
\put(170.0,848.0){\rule[-0.200pt]{2.409pt}{0.400pt}}
\put(1429.0,848.0){\rule[-0.200pt]{2.409pt}{0.400pt}}
\put(170.0,848.0){\rule[-0.200pt]{2.409pt}{0.400pt}}
\put(1429.0,848.0){\rule[-0.200pt]{2.409pt}{0.400pt}}
\put(170.0,848.0){\rule[-0.200pt]{2.409pt}{0.400pt}}
\put(1429.0,848.0){\rule[-0.200pt]{2.409pt}{0.400pt}}
\put(170.0,848.0){\rule[-0.200pt]{2.409pt}{0.400pt}}
\put(1429.0,848.0){\rule[-0.200pt]{2.409pt}{0.400pt}}
\put(170.0,848.0){\rule[-0.200pt]{2.409pt}{0.400pt}}
\put(1429.0,848.0){\rule[-0.200pt]{2.409pt}{0.400pt}}
\put(170.0,848.0){\rule[-0.200pt]{2.409pt}{0.400pt}}
\put(1429.0,848.0){\rule[-0.200pt]{2.409pt}{0.400pt}}
\put(170.0,848.0){\rule[-0.200pt]{2.409pt}{0.400pt}}
\put(1429.0,848.0){\rule[-0.200pt]{2.409pt}{0.400pt}}
\put(170.0,848.0){\rule[-0.200pt]{2.409pt}{0.400pt}}
\put(1429.0,848.0){\rule[-0.200pt]{2.409pt}{0.400pt}}
\put(170.0,848.0){\rule[-0.200pt]{2.409pt}{0.400pt}}
\put(1429.0,848.0){\rule[-0.200pt]{2.409pt}{0.400pt}}
\put(170.0,848.0){\rule[-0.200pt]{2.409pt}{0.400pt}}
\put(1429.0,848.0){\rule[-0.200pt]{2.409pt}{0.400pt}}
\put(170.0,848.0){\rule[-0.200pt]{2.409pt}{0.400pt}}
\put(1429.0,848.0){\rule[-0.200pt]{2.409pt}{0.400pt}}
\put(170.0,848.0){\rule[-0.200pt]{2.409pt}{0.400pt}}
\put(1429.0,848.0){\rule[-0.200pt]{2.409pt}{0.400pt}}
\put(170.0,848.0){\rule[-0.200pt]{2.409pt}{0.400pt}}
\put(1429.0,848.0){\rule[-0.200pt]{2.409pt}{0.400pt}}
\put(170.0,848.0){\rule[-0.200pt]{2.409pt}{0.400pt}}
\put(1429.0,848.0){\rule[-0.200pt]{2.409pt}{0.400pt}}
\put(170.0,848.0){\rule[-0.200pt]{2.409pt}{0.400pt}}
\put(1429.0,848.0){\rule[-0.200pt]{2.409pt}{0.400pt}}
\put(170.0,848.0){\rule[-0.200pt]{2.409pt}{0.400pt}}
\put(1429.0,848.0){\rule[-0.200pt]{2.409pt}{0.400pt}}
\put(170.0,848.0){\rule[-0.200pt]{2.409pt}{0.400pt}}
\put(1429.0,848.0){\rule[-0.200pt]{2.409pt}{0.400pt}}
\put(170.0,848.0){\rule[-0.200pt]{2.409pt}{0.400pt}}
\put(1429.0,848.0){\rule[-0.200pt]{2.409pt}{0.400pt}}
\put(170.0,848.0){\rule[-0.200pt]{2.409pt}{0.400pt}}
\put(1429.0,848.0){\rule[-0.200pt]{2.409pt}{0.400pt}}
\put(170.0,849.0){\rule[-0.200pt]{2.409pt}{0.400pt}}
\put(1429.0,849.0){\rule[-0.200pt]{2.409pt}{0.400pt}}
\put(170.0,849.0){\rule[-0.200pt]{2.409pt}{0.400pt}}
\put(1429.0,849.0){\rule[-0.200pt]{2.409pt}{0.400pt}}
\put(170.0,849.0){\rule[-0.200pt]{2.409pt}{0.400pt}}
\put(1429.0,849.0){\rule[-0.200pt]{2.409pt}{0.400pt}}
\put(170.0,849.0){\rule[-0.200pt]{2.409pt}{0.400pt}}
\put(1429.0,849.0){\rule[-0.200pt]{2.409pt}{0.400pt}}
\put(170.0,849.0){\rule[-0.200pt]{2.409pt}{0.400pt}}
\put(1429.0,849.0){\rule[-0.200pt]{2.409pt}{0.400pt}}
\put(170.0,849.0){\rule[-0.200pt]{2.409pt}{0.400pt}}
\put(1429.0,849.0){\rule[-0.200pt]{2.409pt}{0.400pt}}
\put(170.0,849.0){\rule[-0.200pt]{2.409pt}{0.400pt}}
\put(1429.0,849.0){\rule[-0.200pt]{2.409pt}{0.400pt}}
\put(170.0,849.0){\rule[-0.200pt]{2.409pt}{0.400pt}}
\put(1429.0,849.0){\rule[-0.200pt]{2.409pt}{0.400pt}}
\put(170.0,849.0){\rule[-0.200pt]{2.409pt}{0.400pt}}
\put(1429.0,849.0){\rule[-0.200pt]{2.409pt}{0.400pt}}
\put(170.0,849.0){\rule[-0.200pt]{2.409pt}{0.400pt}}
\put(1429.0,849.0){\rule[-0.200pt]{2.409pt}{0.400pt}}
\put(170.0,849.0){\rule[-0.200pt]{2.409pt}{0.400pt}}
\put(1429.0,849.0){\rule[-0.200pt]{2.409pt}{0.400pt}}
\put(170.0,849.0){\rule[-0.200pt]{2.409pt}{0.400pt}}
\put(1429.0,849.0){\rule[-0.200pt]{2.409pt}{0.400pt}}
\put(170.0,849.0){\rule[-0.200pt]{2.409pt}{0.400pt}}
\put(1429.0,849.0){\rule[-0.200pt]{2.409pt}{0.400pt}}
\put(170.0,849.0){\rule[-0.200pt]{2.409pt}{0.400pt}}
\put(1429.0,849.0){\rule[-0.200pt]{2.409pt}{0.400pt}}
\put(170.0,849.0){\rule[-0.200pt]{2.409pt}{0.400pt}}
\put(1429.0,849.0){\rule[-0.200pt]{2.409pt}{0.400pt}}
\put(170.0,849.0){\rule[-0.200pt]{2.409pt}{0.400pt}}
\put(1429.0,849.0){\rule[-0.200pt]{2.409pt}{0.400pt}}
\put(170.0,849.0){\rule[-0.200pt]{2.409pt}{0.400pt}}
\put(1429.0,849.0){\rule[-0.200pt]{2.409pt}{0.400pt}}
\put(170.0,849.0){\rule[-0.200pt]{2.409pt}{0.400pt}}
\put(1429.0,849.0){\rule[-0.200pt]{2.409pt}{0.400pt}}
\put(170.0,849.0){\rule[-0.200pt]{2.409pt}{0.400pt}}
\put(1429.0,849.0){\rule[-0.200pt]{2.409pt}{0.400pt}}
\put(170.0,849.0){\rule[-0.200pt]{2.409pt}{0.400pt}}
\put(1429.0,849.0){\rule[-0.200pt]{2.409pt}{0.400pt}}
\put(170.0,849.0){\rule[-0.200pt]{2.409pt}{0.400pt}}
\put(1429.0,849.0){\rule[-0.200pt]{2.409pt}{0.400pt}}
\put(170.0,850.0){\rule[-0.200pt]{2.409pt}{0.400pt}}
\put(1429.0,850.0){\rule[-0.200pt]{2.409pt}{0.400pt}}
\put(170.0,850.0){\rule[-0.200pt]{2.409pt}{0.400pt}}
\put(1429.0,850.0){\rule[-0.200pt]{2.409pt}{0.400pt}}
\put(170.0,850.0){\rule[-0.200pt]{2.409pt}{0.400pt}}
\put(1429.0,850.0){\rule[-0.200pt]{2.409pt}{0.400pt}}
\put(170.0,850.0){\rule[-0.200pt]{2.409pt}{0.400pt}}
\put(1429.0,850.0){\rule[-0.200pt]{2.409pt}{0.400pt}}
\put(170.0,850.0){\rule[-0.200pt]{2.409pt}{0.400pt}}
\put(1429.0,850.0){\rule[-0.200pt]{2.409pt}{0.400pt}}
\put(170.0,850.0){\rule[-0.200pt]{2.409pt}{0.400pt}}
\put(1429.0,850.0){\rule[-0.200pt]{2.409pt}{0.400pt}}
\put(170.0,850.0){\rule[-0.200pt]{2.409pt}{0.400pt}}
\put(1429.0,850.0){\rule[-0.200pt]{2.409pt}{0.400pt}}
\put(170.0,850.0){\rule[-0.200pt]{2.409pt}{0.400pt}}
\put(1429.0,850.0){\rule[-0.200pt]{2.409pt}{0.400pt}}
\put(170.0,850.0){\rule[-0.200pt]{2.409pt}{0.400pt}}
\put(1429.0,850.0){\rule[-0.200pt]{2.409pt}{0.400pt}}
\put(170.0,850.0){\rule[-0.200pt]{2.409pt}{0.400pt}}
\put(1429.0,850.0){\rule[-0.200pt]{2.409pt}{0.400pt}}
\put(170.0,850.0){\rule[-0.200pt]{2.409pt}{0.400pt}}
\put(1429.0,850.0){\rule[-0.200pt]{2.409pt}{0.400pt}}
\put(170.0,850.0){\rule[-0.200pt]{2.409pt}{0.400pt}}
\put(1429.0,850.0){\rule[-0.200pt]{2.409pt}{0.400pt}}
\put(170.0,850.0){\rule[-0.200pt]{2.409pt}{0.400pt}}
\put(1429.0,850.0){\rule[-0.200pt]{2.409pt}{0.400pt}}
\put(170.0,850.0){\rule[-0.200pt]{2.409pt}{0.400pt}}
\put(1429.0,850.0){\rule[-0.200pt]{2.409pt}{0.400pt}}
\put(170.0,850.0){\rule[-0.200pt]{2.409pt}{0.400pt}}
\put(1429.0,850.0){\rule[-0.200pt]{2.409pt}{0.400pt}}
\put(170.0,850.0){\rule[-0.200pt]{2.409pt}{0.400pt}}
\put(1429.0,850.0){\rule[-0.200pt]{2.409pt}{0.400pt}}
\put(170.0,850.0){\rule[-0.200pt]{2.409pt}{0.400pt}}
\put(1429.0,850.0){\rule[-0.200pt]{2.409pt}{0.400pt}}
\put(170.0,850.0){\rule[-0.200pt]{2.409pt}{0.400pt}}
\put(1429.0,850.0){\rule[-0.200pt]{2.409pt}{0.400pt}}
\put(170.0,850.0){\rule[-0.200pt]{2.409pt}{0.400pt}}
\put(1429.0,850.0){\rule[-0.200pt]{2.409pt}{0.400pt}}
\put(170.0,850.0){\rule[-0.200pt]{2.409pt}{0.400pt}}
\put(1429.0,850.0){\rule[-0.200pt]{2.409pt}{0.400pt}}
\put(170.0,850.0){\rule[-0.200pt]{2.409pt}{0.400pt}}
\put(1429.0,850.0){\rule[-0.200pt]{2.409pt}{0.400pt}}
\put(170.0,851.0){\rule[-0.200pt]{2.409pt}{0.400pt}}
\put(1429.0,851.0){\rule[-0.200pt]{2.409pt}{0.400pt}}
\put(170.0,851.0){\rule[-0.200pt]{2.409pt}{0.400pt}}
\put(1429.0,851.0){\rule[-0.200pt]{2.409pt}{0.400pt}}
\put(170.0,851.0){\rule[-0.200pt]{2.409pt}{0.400pt}}
\put(1429.0,851.0){\rule[-0.200pt]{2.409pt}{0.400pt}}
\put(170.0,851.0){\rule[-0.200pt]{2.409pt}{0.400pt}}
\put(1429.0,851.0){\rule[-0.200pt]{2.409pt}{0.400pt}}
\put(170.0,851.0){\rule[-0.200pt]{2.409pt}{0.400pt}}
\put(1429.0,851.0){\rule[-0.200pt]{2.409pt}{0.400pt}}
\put(170.0,851.0){\rule[-0.200pt]{2.409pt}{0.400pt}}
\put(1429.0,851.0){\rule[-0.200pt]{2.409pt}{0.400pt}}
\put(170.0,851.0){\rule[-0.200pt]{2.409pt}{0.400pt}}
\put(1429.0,851.0){\rule[-0.200pt]{2.409pt}{0.400pt}}
\put(170.0,851.0){\rule[-0.200pt]{2.409pt}{0.400pt}}
\put(1429.0,851.0){\rule[-0.200pt]{2.409pt}{0.400pt}}
\put(170.0,851.0){\rule[-0.200pt]{2.409pt}{0.400pt}}
\put(1429.0,851.0){\rule[-0.200pt]{2.409pt}{0.400pt}}
\put(170.0,851.0){\rule[-0.200pt]{2.409pt}{0.400pt}}
\put(1429.0,851.0){\rule[-0.200pt]{2.409pt}{0.400pt}}
\put(170.0,851.0){\rule[-0.200pt]{2.409pt}{0.400pt}}
\put(1429.0,851.0){\rule[-0.200pt]{2.409pt}{0.400pt}}
\put(170.0,851.0){\rule[-0.200pt]{2.409pt}{0.400pt}}
\put(1429.0,851.0){\rule[-0.200pt]{2.409pt}{0.400pt}}
\put(170.0,851.0){\rule[-0.200pt]{2.409pt}{0.400pt}}
\put(1429.0,851.0){\rule[-0.200pt]{2.409pt}{0.400pt}}
\put(170.0,851.0){\rule[-0.200pt]{2.409pt}{0.400pt}}
\put(1429.0,851.0){\rule[-0.200pt]{2.409pt}{0.400pt}}
\put(170.0,851.0){\rule[-0.200pt]{2.409pt}{0.400pt}}
\put(1429.0,851.0){\rule[-0.200pt]{2.409pt}{0.400pt}}
\put(170.0,851.0){\rule[-0.200pt]{2.409pt}{0.400pt}}
\put(1429.0,851.0){\rule[-0.200pt]{2.409pt}{0.400pt}}
\put(170.0,851.0){\rule[-0.200pt]{2.409pt}{0.400pt}}
\put(1429.0,851.0){\rule[-0.200pt]{2.409pt}{0.400pt}}
\put(170.0,851.0){\rule[-0.200pt]{2.409pt}{0.400pt}}
\put(1429.0,851.0){\rule[-0.200pt]{2.409pt}{0.400pt}}
\put(170.0,851.0){\rule[-0.200pt]{2.409pt}{0.400pt}}
\put(1429.0,851.0){\rule[-0.200pt]{2.409pt}{0.400pt}}
\put(170.0,851.0){\rule[-0.200pt]{2.409pt}{0.400pt}}
\put(1429.0,851.0){\rule[-0.200pt]{2.409pt}{0.400pt}}
\put(170.0,851.0){\rule[-0.200pt]{2.409pt}{0.400pt}}
\put(1429.0,851.0){\rule[-0.200pt]{2.409pt}{0.400pt}}
\put(170.0,852.0){\rule[-0.200pt]{2.409pt}{0.400pt}}
\put(1429.0,852.0){\rule[-0.200pt]{2.409pt}{0.400pt}}
\put(170.0,852.0){\rule[-0.200pt]{2.409pt}{0.400pt}}
\put(1429.0,852.0){\rule[-0.200pt]{2.409pt}{0.400pt}}
\put(170.0,852.0){\rule[-0.200pt]{2.409pt}{0.400pt}}
\put(1429.0,852.0){\rule[-0.200pt]{2.409pt}{0.400pt}}
\put(170.0,852.0){\rule[-0.200pt]{2.409pt}{0.400pt}}
\put(1429.0,852.0){\rule[-0.200pt]{2.409pt}{0.400pt}}
\put(170.0,852.0){\rule[-0.200pt]{2.409pt}{0.400pt}}
\put(1429.0,852.0){\rule[-0.200pt]{2.409pt}{0.400pt}}
\put(170.0,852.0){\rule[-0.200pt]{2.409pt}{0.400pt}}
\put(1429.0,852.0){\rule[-0.200pt]{2.409pt}{0.400pt}}
\put(170.0,852.0){\rule[-0.200pt]{2.409pt}{0.400pt}}
\put(1429.0,852.0){\rule[-0.200pt]{2.409pt}{0.400pt}}
\put(170.0,852.0){\rule[-0.200pt]{2.409pt}{0.400pt}}
\put(1429.0,852.0){\rule[-0.200pt]{2.409pt}{0.400pt}}
\put(170.0,852.0){\rule[-0.200pt]{2.409pt}{0.400pt}}
\put(1429.0,852.0){\rule[-0.200pt]{2.409pt}{0.400pt}}
\put(170.0,852.0){\rule[-0.200pt]{2.409pt}{0.400pt}}
\put(1429.0,852.0){\rule[-0.200pt]{2.409pt}{0.400pt}}
\put(170.0,852.0){\rule[-0.200pt]{2.409pt}{0.400pt}}
\put(1429.0,852.0){\rule[-0.200pt]{2.409pt}{0.400pt}}
\put(170.0,852.0){\rule[-0.200pt]{2.409pt}{0.400pt}}
\put(1429.0,852.0){\rule[-0.200pt]{2.409pt}{0.400pt}}
\put(170.0,852.0){\rule[-0.200pt]{2.409pt}{0.400pt}}
\put(1429.0,852.0){\rule[-0.200pt]{2.409pt}{0.400pt}}
\put(170.0,852.0){\rule[-0.200pt]{2.409pt}{0.400pt}}
\put(1429.0,852.0){\rule[-0.200pt]{2.409pt}{0.400pt}}
\put(170.0,852.0){\rule[-0.200pt]{2.409pt}{0.400pt}}
\put(1429.0,852.0){\rule[-0.200pt]{2.409pt}{0.400pt}}
\put(170.0,852.0){\rule[-0.200pt]{2.409pt}{0.400pt}}
\put(1429.0,852.0){\rule[-0.200pt]{2.409pt}{0.400pt}}
\put(170.0,852.0){\rule[-0.200pt]{2.409pt}{0.400pt}}
\put(1429.0,852.0){\rule[-0.200pt]{2.409pt}{0.400pt}}
\put(170.0,852.0){\rule[-0.200pt]{2.409pt}{0.400pt}}
\put(1429.0,852.0){\rule[-0.200pt]{2.409pt}{0.400pt}}
\put(170.0,852.0){\rule[-0.200pt]{2.409pt}{0.400pt}}
\put(1429.0,852.0){\rule[-0.200pt]{2.409pt}{0.400pt}}
\put(170.0,852.0){\rule[-0.200pt]{2.409pt}{0.400pt}}
\put(1429.0,852.0){\rule[-0.200pt]{2.409pt}{0.400pt}}
\put(170.0,852.0){\rule[-0.200pt]{2.409pt}{0.400pt}}
\put(1429.0,852.0){\rule[-0.200pt]{2.409pt}{0.400pt}}
\put(170.0,852.0){\rule[-0.200pt]{2.409pt}{0.400pt}}
\put(1429.0,852.0){\rule[-0.200pt]{2.409pt}{0.400pt}}
\put(170.0,853.0){\rule[-0.200pt]{2.409pt}{0.400pt}}
\put(1429.0,853.0){\rule[-0.200pt]{2.409pt}{0.400pt}}
\put(170.0,853.0){\rule[-0.200pt]{2.409pt}{0.400pt}}
\put(1429.0,853.0){\rule[-0.200pt]{2.409pt}{0.400pt}}
\put(170.0,853.0){\rule[-0.200pt]{2.409pt}{0.400pt}}
\put(1429.0,853.0){\rule[-0.200pt]{2.409pt}{0.400pt}}
\put(170.0,853.0){\rule[-0.200pt]{2.409pt}{0.400pt}}
\put(1429.0,853.0){\rule[-0.200pt]{2.409pt}{0.400pt}}
\put(170.0,853.0){\rule[-0.200pt]{2.409pt}{0.400pt}}
\put(1429.0,853.0){\rule[-0.200pt]{2.409pt}{0.400pt}}
\put(170.0,853.0){\rule[-0.200pt]{2.409pt}{0.400pt}}
\put(1429.0,853.0){\rule[-0.200pt]{2.409pt}{0.400pt}}
\put(170.0,853.0){\rule[-0.200pt]{2.409pt}{0.400pt}}
\put(1429.0,853.0){\rule[-0.200pt]{2.409pt}{0.400pt}}
\put(170.0,853.0){\rule[-0.200pt]{2.409pt}{0.400pt}}
\put(1429.0,853.0){\rule[-0.200pt]{2.409pt}{0.400pt}}
\put(170.0,853.0){\rule[-0.200pt]{2.409pt}{0.400pt}}
\put(1429.0,853.0){\rule[-0.200pt]{2.409pt}{0.400pt}}
\put(170.0,853.0){\rule[-0.200pt]{2.409pt}{0.400pt}}
\put(1429.0,853.0){\rule[-0.200pt]{2.409pt}{0.400pt}}
\put(170.0,853.0){\rule[-0.200pt]{2.409pt}{0.400pt}}
\put(1429.0,853.0){\rule[-0.200pt]{2.409pt}{0.400pt}}
\put(170.0,853.0){\rule[-0.200pt]{2.409pt}{0.400pt}}
\put(1429.0,853.0){\rule[-0.200pt]{2.409pt}{0.400pt}}
\put(170.0,853.0){\rule[-0.200pt]{2.409pt}{0.400pt}}
\put(1429.0,853.0){\rule[-0.200pt]{2.409pt}{0.400pt}}
\put(170.0,853.0){\rule[-0.200pt]{2.409pt}{0.400pt}}
\put(1429.0,853.0){\rule[-0.200pt]{2.409pt}{0.400pt}}
\put(170.0,853.0){\rule[-0.200pt]{2.409pt}{0.400pt}}
\put(1429.0,853.0){\rule[-0.200pt]{2.409pt}{0.400pt}}
\put(170.0,853.0){\rule[-0.200pt]{2.409pt}{0.400pt}}
\put(1429.0,853.0){\rule[-0.200pt]{2.409pt}{0.400pt}}
\put(170.0,853.0){\rule[-0.200pt]{2.409pt}{0.400pt}}
\put(1429.0,853.0){\rule[-0.200pt]{2.409pt}{0.400pt}}
\put(170.0,853.0){\rule[-0.200pt]{2.409pt}{0.400pt}}
\put(1429.0,853.0){\rule[-0.200pt]{2.409pt}{0.400pt}}
\put(170.0,853.0){\rule[-0.200pt]{2.409pt}{0.400pt}}
\put(1429.0,853.0){\rule[-0.200pt]{2.409pt}{0.400pt}}
\put(170.0,853.0){\rule[-0.200pt]{2.409pt}{0.400pt}}
\put(1429.0,853.0){\rule[-0.200pt]{2.409pt}{0.400pt}}
\put(170.0,853.0){\rule[-0.200pt]{2.409pt}{0.400pt}}
\put(1429.0,853.0){\rule[-0.200pt]{2.409pt}{0.400pt}}
\put(170.0,853.0){\rule[-0.200pt]{2.409pt}{0.400pt}}
\put(1429.0,853.0){\rule[-0.200pt]{2.409pt}{0.400pt}}
\put(170.0,853.0){\rule[-0.200pt]{2.409pt}{0.400pt}}
\put(1429.0,853.0){\rule[-0.200pt]{2.409pt}{0.400pt}}
\put(170.0,854.0){\rule[-0.200pt]{2.409pt}{0.400pt}}
\put(1429.0,854.0){\rule[-0.200pt]{2.409pt}{0.400pt}}
\put(170.0,854.0){\rule[-0.200pt]{2.409pt}{0.400pt}}
\put(1429.0,854.0){\rule[-0.200pt]{2.409pt}{0.400pt}}
\put(170.0,854.0){\rule[-0.200pt]{2.409pt}{0.400pt}}
\put(1429.0,854.0){\rule[-0.200pt]{2.409pt}{0.400pt}}
\put(170.0,854.0){\rule[-0.200pt]{2.409pt}{0.400pt}}
\put(1429.0,854.0){\rule[-0.200pt]{2.409pt}{0.400pt}}
\put(170.0,854.0){\rule[-0.200pt]{2.409pt}{0.400pt}}
\put(1429.0,854.0){\rule[-0.200pt]{2.409pt}{0.400pt}}
\put(170.0,854.0){\rule[-0.200pt]{2.409pt}{0.400pt}}
\put(1429.0,854.0){\rule[-0.200pt]{2.409pt}{0.400pt}}
\put(170.0,854.0){\rule[-0.200pt]{2.409pt}{0.400pt}}
\put(1429.0,854.0){\rule[-0.200pt]{2.409pt}{0.400pt}}
\put(170.0,854.0){\rule[-0.200pt]{2.409pt}{0.400pt}}
\put(1429.0,854.0){\rule[-0.200pt]{2.409pt}{0.400pt}}
\put(170.0,854.0){\rule[-0.200pt]{2.409pt}{0.400pt}}
\put(1429.0,854.0){\rule[-0.200pt]{2.409pt}{0.400pt}}
\put(170.0,854.0){\rule[-0.200pt]{2.409pt}{0.400pt}}
\put(1429.0,854.0){\rule[-0.200pt]{2.409pt}{0.400pt}}
\put(170.0,854.0){\rule[-0.200pt]{2.409pt}{0.400pt}}
\put(1429.0,854.0){\rule[-0.200pt]{2.409pt}{0.400pt}}
\put(170.0,854.0){\rule[-0.200pt]{2.409pt}{0.400pt}}
\put(1429.0,854.0){\rule[-0.200pt]{2.409pt}{0.400pt}}
\put(170.0,854.0){\rule[-0.200pt]{2.409pt}{0.400pt}}
\put(1429.0,854.0){\rule[-0.200pt]{2.409pt}{0.400pt}}
\put(170.0,854.0){\rule[-0.200pt]{2.409pt}{0.400pt}}
\put(1429.0,854.0){\rule[-0.200pt]{2.409pt}{0.400pt}}
\put(170.0,854.0){\rule[-0.200pt]{2.409pt}{0.400pt}}
\put(1429.0,854.0){\rule[-0.200pt]{2.409pt}{0.400pt}}
\put(170.0,854.0){\rule[-0.200pt]{2.409pt}{0.400pt}}
\put(1429.0,854.0){\rule[-0.200pt]{2.409pt}{0.400pt}}
\put(170.0,854.0){\rule[-0.200pt]{2.409pt}{0.400pt}}
\put(1429.0,854.0){\rule[-0.200pt]{2.409pt}{0.400pt}}
\put(170.0,854.0){\rule[-0.200pt]{2.409pt}{0.400pt}}
\put(1429.0,854.0){\rule[-0.200pt]{2.409pt}{0.400pt}}
\put(170.0,854.0){\rule[-0.200pt]{2.409pt}{0.400pt}}
\put(1429.0,854.0){\rule[-0.200pt]{2.409pt}{0.400pt}}
\put(170.0,854.0){\rule[-0.200pt]{2.409pt}{0.400pt}}
\put(1429.0,854.0){\rule[-0.200pt]{2.409pt}{0.400pt}}
\put(170.0,854.0){\rule[-0.200pt]{2.409pt}{0.400pt}}
\put(1429.0,854.0){\rule[-0.200pt]{2.409pt}{0.400pt}}
\put(170.0,854.0){\rule[-0.200pt]{2.409pt}{0.400pt}}
\put(1429.0,854.0){\rule[-0.200pt]{2.409pt}{0.400pt}}
\put(170.0,854.0){\rule[-0.200pt]{2.409pt}{0.400pt}}
\put(1429.0,854.0){\rule[-0.200pt]{2.409pt}{0.400pt}}
\put(170.0,855.0){\rule[-0.200pt]{2.409pt}{0.400pt}}
\put(1429.0,855.0){\rule[-0.200pt]{2.409pt}{0.400pt}}
\put(170.0,855.0){\rule[-0.200pt]{2.409pt}{0.400pt}}
\put(1429.0,855.0){\rule[-0.200pt]{2.409pt}{0.400pt}}
\put(170.0,855.0){\rule[-0.200pt]{2.409pt}{0.400pt}}
\put(1429.0,855.0){\rule[-0.200pt]{2.409pt}{0.400pt}}
\put(170.0,855.0){\rule[-0.200pt]{2.409pt}{0.400pt}}
\put(1429.0,855.0){\rule[-0.200pt]{2.409pt}{0.400pt}}
\put(170.0,855.0){\rule[-0.200pt]{2.409pt}{0.400pt}}
\put(1429.0,855.0){\rule[-0.200pt]{2.409pt}{0.400pt}}
\put(170.0,855.0){\rule[-0.200pt]{2.409pt}{0.400pt}}
\put(1429.0,855.0){\rule[-0.200pt]{2.409pt}{0.400pt}}
\put(170.0,855.0){\rule[-0.200pt]{2.409pt}{0.400pt}}
\put(1429.0,855.0){\rule[-0.200pt]{2.409pt}{0.400pt}}
\put(170.0,855.0){\rule[-0.200pt]{2.409pt}{0.400pt}}
\put(1429.0,855.0){\rule[-0.200pt]{2.409pt}{0.400pt}}
\put(170.0,855.0){\rule[-0.200pt]{2.409pt}{0.400pt}}
\put(1429.0,855.0){\rule[-0.200pt]{2.409pt}{0.400pt}}
\put(170.0,855.0){\rule[-0.200pt]{2.409pt}{0.400pt}}
\put(1429.0,855.0){\rule[-0.200pt]{2.409pt}{0.400pt}}
\put(170.0,855.0){\rule[-0.200pt]{2.409pt}{0.400pt}}
\put(1429.0,855.0){\rule[-0.200pt]{2.409pt}{0.400pt}}
\put(170.0,855.0){\rule[-0.200pt]{2.409pt}{0.400pt}}
\put(1429.0,855.0){\rule[-0.200pt]{2.409pt}{0.400pt}}
\put(170.0,855.0){\rule[-0.200pt]{2.409pt}{0.400pt}}
\put(1429.0,855.0){\rule[-0.200pt]{2.409pt}{0.400pt}}
\put(170.0,855.0){\rule[-0.200pt]{2.409pt}{0.400pt}}
\put(1429.0,855.0){\rule[-0.200pt]{2.409pt}{0.400pt}}
\put(170.0,855.0){\rule[-0.200pt]{2.409pt}{0.400pt}}
\put(1429.0,855.0){\rule[-0.200pt]{2.409pt}{0.400pt}}
\put(170.0,855.0){\rule[-0.200pt]{2.409pt}{0.400pt}}
\put(1429.0,855.0){\rule[-0.200pt]{2.409pt}{0.400pt}}
\put(170.0,855.0){\rule[-0.200pt]{2.409pt}{0.400pt}}
\put(1429.0,855.0){\rule[-0.200pt]{2.409pt}{0.400pt}}
\put(170.0,855.0){\rule[-0.200pt]{2.409pt}{0.400pt}}
\put(1429.0,855.0){\rule[-0.200pt]{2.409pt}{0.400pt}}
\put(170.0,855.0){\rule[-0.200pt]{2.409pt}{0.400pt}}
\put(1429.0,855.0){\rule[-0.200pt]{2.409pt}{0.400pt}}
\put(170.0,855.0){\rule[-0.200pt]{2.409pt}{0.400pt}}
\put(1429.0,855.0){\rule[-0.200pt]{2.409pt}{0.400pt}}
\put(170.0,855.0){\rule[-0.200pt]{2.409pt}{0.400pt}}
\put(1429.0,855.0){\rule[-0.200pt]{2.409pt}{0.400pt}}
\put(170.0,855.0){\rule[-0.200pt]{2.409pt}{0.400pt}}
\put(1429.0,855.0){\rule[-0.200pt]{2.409pt}{0.400pt}}
\put(170.0,855.0){\rule[-0.200pt]{2.409pt}{0.400pt}}
\put(1429.0,855.0){\rule[-0.200pt]{2.409pt}{0.400pt}}
\put(170.0,855.0){\rule[-0.200pt]{2.409pt}{0.400pt}}
\put(1429.0,855.0){\rule[-0.200pt]{2.409pt}{0.400pt}}
\put(170.0,856.0){\rule[-0.200pt]{2.409pt}{0.400pt}}
\put(1429.0,856.0){\rule[-0.200pt]{2.409pt}{0.400pt}}
\put(170.0,856.0){\rule[-0.200pt]{2.409pt}{0.400pt}}
\put(1429.0,856.0){\rule[-0.200pt]{2.409pt}{0.400pt}}
\put(170.0,856.0){\rule[-0.200pt]{2.409pt}{0.400pt}}
\put(1429.0,856.0){\rule[-0.200pt]{2.409pt}{0.400pt}}
\put(170.0,856.0){\rule[-0.200pt]{2.409pt}{0.400pt}}
\put(1429.0,856.0){\rule[-0.200pt]{2.409pt}{0.400pt}}
\put(170.0,856.0){\rule[-0.200pt]{2.409pt}{0.400pt}}
\put(1429.0,856.0){\rule[-0.200pt]{2.409pt}{0.400pt}}
\put(170.0,856.0){\rule[-0.200pt]{2.409pt}{0.400pt}}
\put(1429.0,856.0){\rule[-0.200pt]{2.409pt}{0.400pt}}
\put(170.0,856.0){\rule[-0.200pt]{2.409pt}{0.400pt}}
\put(1429.0,856.0){\rule[-0.200pt]{2.409pt}{0.400pt}}
\put(170.0,856.0){\rule[-0.200pt]{2.409pt}{0.400pt}}
\put(1429.0,856.0){\rule[-0.200pt]{2.409pt}{0.400pt}}
\put(170.0,856.0){\rule[-0.200pt]{2.409pt}{0.400pt}}
\put(1429.0,856.0){\rule[-0.200pt]{2.409pt}{0.400pt}}
\put(170.0,856.0){\rule[-0.200pt]{2.409pt}{0.400pt}}
\put(1429.0,856.0){\rule[-0.200pt]{2.409pt}{0.400pt}}
\put(170.0,856.0){\rule[-0.200pt]{2.409pt}{0.400pt}}
\put(1429.0,856.0){\rule[-0.200pt]{2.409pt}{0.400pt}}
\put(170.0,856.0){\rule[-0.200pt]{2.409pt}{0.400pt}}
\put(1429.0,856.0){\rule[-0.200pt]{2.409pt}{0.400pt}}
\put(170.0,856.0){\rule[-0.200pt]{2.409pt}{0.400pt}}
\put(1429.0,856.0){\rule[-0.200pt]{2.409pt}{0.400pt}}
\put(170.0,856.0){\rule[-0.200pt]{2.409pt}{0.400pt}}
\put(1429.0,856.0){\rule[-0.200pt]{2.409pt}{0.400pt}}
\put(170.0,856.0){\rule[-0.200pt]{2.409pt}{0.400pt}}
\put(1429.0,856.0){\rule[-0.200pt]{2.409pt}{0.400pt}}
\put(170.0,856.0){\rule[-0.200pt]{2.409pt}{0.400pt}}
\put(1429.0,856.0){\rule[-0.200pt]{2.409pt}{0.400pt}}
\put(170.0,856.0){\rule[-0.200pt]{2.409pt}{0.400pt}}
\put(1429.0,856.0){\rule[-0.200pt]{2.409pt}{0.400pt}}
\put(170.0,856.0){\rule[-0.200pt]{2.409pt}{0.400pt}}
\put(1429.0,856.0){\rule[-0.200pt]{2.409pt}{0.400pt}}
\put(170.0,856.0){\rule[-0.200pt]{2.409pt}{0.400pt}}
\put(1429.0,856.0){\rule[-0.200pt]{2.409pt}{0.400pt}}
\put(170.0,856.0){\rule[-0.200pt]{2.409pt}{0.400pt}}
\put(1429.0,856.0){\rule[-0.200pt]{2.409pt}{0.400pt}}
\put(170.0,856.0){\rule[-0.200pt]{2.409pt}{0.400pt}}
\put(1429.0,856.0){\rule[-0.200pt]{2.409pt}{0.400pt}}
\put(170.0,856.0){\rule[-0.200pt]{2.409pt}{0.400pt}}
\put(1429.0,856.0){\rule[-0.200pt]{2.409pt}{0.400pt}}
\put(170.0,856.0){\rule[-0.200pt]{2.409pt}{0.400pt}}
\put(1429.0,856.0){\rule[-0.200pt]{2.409pt}{0.400pt}}
\put(170.0,856.0){\rule[-0.200pt]{2.409pt}{0.400pt}}
\put(1429.0,856.0){\rule[-0.200pt]{2.409pt}{0.400pt}}
\put(170.0,856.0){\rule[-0.200pt]{2.409pt}{0.400pt}}
\put(1429.0,856.0){\rule[-0.200pt]{2.409pt}{0.400pt}}
\put(170.0,857.0){\rule[-0.200pt]{2.409pt}{0.400pt}}
\put(1429.0,857.0){\rule[-0.200pt]{2.409pt}{0.400pt}}
\put(170.0,857.0){\rule[-0.200pt]{2.409pt}{0.400pt}}
\put(1429.0,857.0){\rule[-0.200pt]{2.409pt}{0.400pt}}
\put(170.0,857.0){\rule[-0.200pt]{2.409pt}{0.400pt}}
\put(1429.0,857.0){\rule[-0.200pt]{2.409pt}{0.400pt}}
\put(170.0,857.0){\rule[-0.200pt]{2.409pt}{0.400pt}}
\put(1429.0,857.0){\rule[-0.200pt]{2.409pt}{0.400pt}}
\put(170.0,857.0){\rule[-0.200pt]{2.409pt}{0.400pt}}
\put(1429.0,857.0){\rule[-0.200pt]{2.409pt}{0.400pt}}
\put(170.0,857.0){\rule[-0.200pt]{2.409pt}{0.400pt}}
\put(1429.0,857.0){\rule[-0.200pt]{2.409pt}{0.400pt}}
\put(170.0,857.0){\rule[-0.200pt]{2.409pt}{0.400pt}}
\put(1429.0,857.0){\rule[-0.200pt]{2.409pt}{0.400pt}}
\put(170.0,857.0){\rule[-0.200pt]{2.409pt}{0.400pt}}
\put(1429.0,857.0){\rule[-0.200pt]{2.409pt}{0.400pt}}
\put(170.0,857.0){\rule[-0.200pt]{2.409pt}{0.400pt}}
\put(1429.0,857.0){\rule[-0.200pt]{2.409pt}{0.400pt}}
\put(170.0,857.0){\rule[-0.200pt]{2.409pt}{0.400pt}}
\put(1429.0,857.0){\rule[-0.200pt]{2.409pt}{0.400pt}}
\put(170.0,857.0){\rule[-0.200pt]{2.409pt}{0.400pt}}
\put(1429.0,857.0){\rule[-0.200pt]{2.409pt}{0.400pt}}
\put(170.0,857.0){\rule[-0.200pt]{2.409pt}{0.400pt}}
\put(1429.0,857.0){\rule[-0.200pt]{2.409pt}{0.400pt}}
\put(170.0,857.0){\rule[-0.200pt]{2.409pt}{0.400pt}}
\put(1429.0,857.0){\rule[-0.200pt]{2.409pt}{0.400pt}}
\put(170.0,857.0){\rule[-0.200pt]{2.409pt}{0.400pt}}
\put(1429.0,857.0){\rule[-0.200pt]{2.409pt}{0.400pt}}
\put(170.0,857.0){\rule[-0.200pt]{2.409pt}{0.400pt}}
\put(1429.0,857.0){\rule[-0.200pt]{2.409pt}{0.400pt}}
\put(170.0,857.0){\rule[-0.200pt]{2.409pt}{0.400pt}}
\put(1429.0,857.0){\rule[-0.200pt]{2.409pt}{0.400pt}}
\put(170.0,857.0){\rule[-0.200pt]{2.409pt}{0.400pt}}
\put(1429.0,857.0){\rule[-0.200pt]{2.409pt}{0.400pt}}
\put(170.0,857.0){\rule[-0.200pt]{2.409pt}{0.400pt}}
\put(1429.0,857.0){\rule[-0.200pt]{2.409pt}{0.400pt}}
\put(170.0,857.0){\rule[-0.200pt]{2.409pt}{0.400pt}}
\put(1429.0,857.0){\rule[-0.200pt]{2.409pt}{0.400pt}}
\put(170.0,857.0){\rule[-0.200pt]{2.409pt}{0.400pt}}
\put(1429.0,857.0){\rule[-0.200pt]{2.409pt}{0.400pt}}
\put(170.0,857.0){\rule[-0.200pt]{2.409pt}{0.400pt}}
\put(1429.0,857.0){\rule[-0.200pt]{2.409pt}{0.400pt}}
\put(170.0,857.0){\rule[-0.200pt]{2.409pt}{0.400pt}}
\put(1429.0,857.0){\rule[-0.200pt]{2.409pt}{0.400pt}}
\put(170.0,857.0){\rule[-0.200pt]{2.409pt}{0.400pt}}
\put(1429.0,857.0){\rule[-0.200pt]{2.409pt}{0.400pt}}
\put(170.0,857.0){\rule[-0.200pt]{2.409pt}{0.400pt}}
\put(1429.0,857.0){\rule[-0.200pt]{2.409pt}{0.400pt}}
\put(170.0,857.0){\rule[-0.200pt]{2.409pt}{0.400pt}}
\put(1429.0,857.0){\rule[-0.200pt]{2.409pt}{0.400pt}}
\put(170.0,858.0){\rule[-0.200pt]{2.409pt}{0.400pt}}
\put(1429.0,858.0){\rule[-0.200pt]{2.409pt}{0.400pt}}
\put(170.0,858.0){\rule[-0.200pt]{2.409pt}{0.400pt}}
\put(1429.0,858.0){\rule[-0.200pt]{2.409pt}{0.400pt}}
\put(170.0,858.0){\rule[-0.200pt]{2.409pt}{0.400pt}}
\put(1429.0,858.0){\rule[-0.200pt]{2.409pt}{0.400pt}}
\put(170.0,858.0){\rule[-0.200pt]{2.409pt}{0.400pt}}
\put(1429.0,858.0){\rule[-0.200pt]{2.409pt}{0.400pt}}
\put(170.0,858.0){\rule[-0.200pt]{2.409pt}{0.400pt}}
\put(1429.0,858.0){\rule[-0.200pt]{2.409pt}{0.400pt}}
\put(170.0,858.0){\rule[-0.200pt]{2.409pt}{0.400pt}}
\put(1429.0,858.0){\rule[-0.200pt]{2.409pt}{0.400pt}}
\put(170.0,858.0){\rule[-0.200pt]{2.409pt}{0.400pt}}
\put(1429.0,858.0){\rule[-0.200pt]{2.409pt}{0.400pt}}
\put(170.0,858.0){\rule[-0.200pt]{2.409pt}{0.400pt}}
\put(1429.0,858.0){\rule[-0.200pt]{2.409pt}{0.400pt}}
\put(170.0,858.0){\rule[-0.200pt]{2.409pt}{0.400pt}}
\put(1429.0,858.0){\rule[-0.200pt]{2.409pt}{0.400pt}}
\put(170.0,858.0){\rule[-0.200pt]{2.409pt}{0.400pt}}
\put(1429.0,858.0){\rule[-0.200pt]{2.409pt}{0.400pt}}
\put(170.0,858.0){\rule[-0.200pt]{2.409pt}{0.400pt}}
\put(1429.0,858.0){\rule[-0.200pt]{2.409pt}{0.400pt}}
\put(170.0,858.0){\rule[-0.200pt]{2.409pt}{0.400pt}}
\put(1429.0,858.0){\rule[-0.200pt]{2.409pt}{0.400pt}}
\put(170.0,858.0){\rule[-0.200pt]{2.409pt}{0.400pt}}
\put(1429.0,858.0){\rule[-0.200pt]{2.409pt}{0.400pt}}
\put(170.0,858.0){\rule[-0.200pt]{2.409pt}{0.400pt}}
\put(1429.0,858.0){\rule[-0.200pt]{2.409pt}{0.400pt}}
\put(170.0,858.0){\rule[-0.200pt]{2.409pt}{0.400pt}}
\put(1429.0,858.0){\rule[-0.200pt]{2.409pt}{0.400pt}}
\put(170.0,858.0){\rule[-0.200pt]{2.409pt}{0.400pt}}
\put(1429.0,858.0){\rule[-0.200pt]{2.409pt}{0.400pt}}
\put(170.0,858.0){\rule[-0.200pt]{2.409pt}{0.400pt}}
\put(1429.0,858.0){\rule[-0.200pt]{2.409pt}{0.400pt}}
\put(170.0,858.0){\rule[-0.200pt]{2.409pt}{0.400pt}}
\put(1429.0,858.0){\rule[-0.200pt]{2.409pt}{0.400pt}}
\put(170.0,858.0){\rule[-0.200pt]{2.409pt}{0.400pt}}
\put(1429.0,858.0){\rule[-0.200pt]{2.409pt}{0.400pt}}
\put(170.0,858.0){\rule[-0.200pt]{2.409pt}{0.400pt}}
\put(1429.0,858.0){\rule[-0.200pt]{2.409pt}{0.400pt}}
\put(170.0,858.0){\rule[-0.200pt]{2.409pt}{0.400pt}}
\put(1429.0,858.0){\rule[-0.200pt]{2.409pt}{0.400pt}}
\put(170.0,858.0){\rule[-0.200pt]{2.409pt}{0.400pt}}
\put(1429.0,858.0){\rule[-0.200pt]{2.409pt}{0.400pt}}
\put(170.0,858.0){\rule[-0.200pt]{2.409pt}{0.400pt}}
\put(1429.0,858.0){\rule[-0.200pt]{2.409pt}{0.400pt}}
\put(170.0,858.0){\rule[-0.200pt]{2.409pt}{0.400pt}}
\put(1429.0,858.0){\rule[-0.200pt]{2.409pt}{0.400pt}}
\put(170.0,858.0){\rule[-0.200pt]{2.409pt}{0.400pt}}
\put(1429.0,858.0){\rule[-0.200pt]{2.409pt}{0.400pt}}
\put(170.0,858.0){\rule[-0.200pt]{2.409pt}{0.400pt}}
\put(1429.0,858.0){\rule[-0.200pt]{2.409pt}{0.400pt}}
\put(170.0,859.0){\rule[-0.200pt]{2.409pt}{0.400pt}}
\put(1429.0,859.0){\rule[-0.200pt]{2.409pt}{0.400pt}}
\put(170.0,859.0){\rule[-0.200pt]{2.409pt}{0.400pt}}
\put(1429.0,859.0){\rule[-0.200pt]{2.409pt}{0.400pt}}
\put(170.0,859.0){\rule[-0.200pt]{2.409pt}{0.400pt}}
\put(1429.0,859.0){\rule[-0.200pt]{2.409pt}{0.400pt}}
\put(170.0,859.0){\rule[-0.200pt]{2.409pt}{0.400pt}}
\put(1429.0,859.0){\rule[-0.200pt]{2.409pt}{0.400pt}}
\put(170.0,859.0){\rule[-0.200pt]{2.409pt}{0.400pt}}
\put(1429.0,859.0){\rule[-0.200pt]{2.409pt}{0.400pt}}
\put(170.0,859.0){\rule[-0.200pt]{2.409pt}{0.400pt}}
\put(1429.0,859.0){\rule[-0.200pt]{2.409pt}{0.400pt}}
\put(170.0,859.0){\rule[-0.200pt]{2.409pt}{0.400pt}}
\put(1429.0,859.0){\rule[-0.200pt]{2.409pt}{0.400pt}}
\put(170.0,859.0){\rule[-0.200pt]{2.409pt}{0.400pt}}
\put(1429.0,859.0){\rule[-0.200pt]{2.409pt}{0.400pt}}
\put(170.0,859.0){\rule[-0.200pt]{2.409pt}{0.400pt}}
\put(1429.0,859.0){\rule[-0.200pt]{2.409pt}{0.400pt}}
\put(170.0,859.0){\rule[-0.200pt]{2.409pt}{0.400pt}}
\put(1429.0,859.0){\rule[-0.200pt]{2.409pt}{0.400pt}}
\put(170.0,859.0){\rule[-0.200pt]{2.409pt}{0.400pt}}
\put(1429.0,859.0){\rule[-0.200pt]{2.409pt}{0.400pt}}
\put(170.0,859.0){\rule[-0.200pt]{2.409pt}{0.400pt}}
\put(1429.0,859.0){\rule[-0.200pt]{2.409pt}{0.400pt}}
\put(170.0,859.0){\rule[-0.200pt]{2.409pt}{0.400pt}}
\put(1429.0,859.0){\rule[-0.200pt]{2.409pt}{0.400pt}}
\put(170.0,859.0){\rule[-0.200pt]{4.818pt}{0.400pt}}
\put(150,859){\makebox(0,0)[r]{ 1e+09}}
\put(1419.0,859.0){\rule[-0.200pt]{4.818pt}{0.400pt}}
\put(170.0,82.0){\rule[-0.200pt]{0.400pt}{4.818pt}}
\put(170,41){\makebox(0,0){ 0}}
\put(170.0,839.0){\rule[-0.200pt]{0.400pt}{4.818pt}}
\put(487.0,82.0){\rule[-0.200pt]{0.400pt}{4.818pt}}
\put(487,41){\makebox(0,0){ 500}}
\put(487.0,839.0){\rule[-0.200pt]{0.400pt}{4.818pt}}
\put(804.0,82.0){\rule[-0.200pt]{0.400pt}{4.818pt}}
\put(804,41){\makebox(0,0){ 1000}}
\put(804.0,839.0){\rule[-0.200pt]{0.400pt}{4.818pt}}
\put(1122.0,82.0){\rule[-0.200pt]{0.400pt}{4.818pt}}
\put(1122,41){\makebox(0,0){ 1500}}
\put(1122.0,839.0){\rule[-0.200pt]{0.400pt}{4.818pt}}
\put(1439.0,82.0){\rule[-0.200pt]{0.400pt}{4.818pt}}
\put(1439,41){\makebox(0,0){ 2000}}
\put(1439.0,839.0){\rule[-0.200pt]{0.400pt}{4.818pt}}
\put(170.0,82.0){\rule[-0.200pt]{0.400pt}{187.179pt}}
\put(170.0,82.0){\rule[-0.200pt]{305.702pt}{0.400pt}}
\put(1439.0,82.0){\rule[-0.200pt]{0.400pt}{187.179pt}}
\put(170.0,859.0){\rule[-0.200pt]{305.702pt}{0.400pt}}
\put(1279,164){\makebox(0,0)[r]{algorytm naturalny}}
\put(1299.0,164.0){\rule[-0.200pt]{24.090pt}{0.400pt}}
\put(171,82){\usebox{\plotpoint}}
\multiput(171.58,82.00)(0.493,4.224){23}{\rule{0.119pt}{3.392pt}}
\multiput(170.17,82.00)(13.000,99.959){2}{\rule{0.400pt}{1.696pt}}
\multiput(184.58,189.00)(0.493,2.994){23}{\rule{0.119pt}{2.438pt}}
\multiput(183.17,189.00)(13.000,70.939){2}{\rule{0.400pt}{1.219pt}}
\multiput(197.58,265.00)(0.494,1.488){25}{\rule{0.119pt}{1.271pt}}
\multiput(196.17,265.00)(14.000,38.361){2}{\rule{0.400pt}{0.636pt}}
\multiput(211.58,306.00)(0.493,0.734){23}{\rule{0.119pt}{0.685pt}}
\multiput(210.17,306.00)(13.000,17.579){2}{\rule{0.400pt}{0.342pt}}
\multiput(224.58,325.00)(0.493,0.616){23}{\rule{0.119pt}{0.592pt}}
\multiput(223.17,325.00)(13.000,14.771){2}{\rule{0.400pt}{0.296pt}}
\multiput(237.58,341.00)(0.493,0.695){23}{\rule{0.119pt}{0.654pt}}
\multiput(236.17,341.00)(13.000,16.643){2}{\rule{0.400pt}{0.327pt}}
\multiput(250.58,359.00)(0.493,0.655){23}{\rule{0.119pt}{0.623pt}}
\multiput(249.17,359.00)(13.000,15.707){2}{\rule{0.400pt}{0.312pt}}
\multiput(263.58,376.00)(0.493,0.576){23}{\rule{0.119pt}{0.562pt}}
\multiput(262.17,376.00)(13.000,13.834){2}{\rule{0.400pt}{0.281pt}}
\multiput(276.00,391.58)(0.497,0.493){23}{\rule{0.500pt}{0.119pt}}
\multiput(276.00,390.17)(11.962,13.000){2}{\rule{0.250pt}{0.400pt}}
\multiput(289.00,404.58)(0.539,0.492){21}{\rule{0.533pt}{0.119pt}}
\multiput(289.00,403.17)(11.893,12.000){2}{\rule{0.267pt}{0.400pt}}
\multiput(302.00,416.58)(0.590,0.492){19}{\rule{0.573pt}{0.118pt}}
\multiput(302.00,415.17)(11.811,11.000){2}{\rule{0.286pt}{0.400pt}}
\multiput(315.00,427.58)(0.652,0.491){17}{\rule{0.620pt}{0.118pt}}
\multiput(315.00,426.17)(11.713,10.000){2}{\rule{0.310pt}{0.400pt}}
\multiput(328.00,437.58)(0.704,0.491){17}{\rule{0.660pt}{0.118pt}}
\multiput(328.00,436.17)(12.630,10.000){2}{\rule{0.330pt}{0.400pt}}
\multiput(342.00,447.59)(0.728,0.489){15}{\rule{0.678pt}{0.118pt}}
\multiput(342.00,446.17)(11.593,9.000){2}{\rule{0.339pt}{0.400pt}}
\multiput(355.00,456.59)(0.824,0.488){13}{\rule{0.750pt}{0.117pt}}
\multiput(355.00,455.17)(11.443,8.000){2}{\rule{0.375pt}{0.400pt}}
\multiput(368.00,464.59)(0.950,0.485){11}{\rule{0.843pt}{0.117pt}}
\multiput(368.00,463.17)(11.251,7.000){2}{\rule{0.421pt}{0.400pt}}
\multiput(381.00,471.59)(0.824,0.488){13}{\rule{0.750pt}{0.117pt}}
\multiput(381.00,470.17)(11.443,8.000){2}{\rule{0.375pt}{0.400pt}}
\multiput(394.00,479.59)(1.123,0.482){9}{\rule{0.967pt}{0.116pt}}
\multiput(394.00,478.17)(10.994,6.000){2}{\rule{0.483pt}{0.400pt}}
\multiput(407.00,485.59)(0.950,0.485){11}{\rule{0.843pt}{0.117pt}}
\multiput(407.00,484.17)(11.251,7.000){2}{\rule{0.421pt}{0.400pt}}
\multiput(420.00,492.59)(1.123,0.482){9}{\rule{0.967pt}{0.116pt}}
\multiput(420.00,491.17)(10.994,6.000){2}{\rule{0.483pt}{0.400pt}}
\multiput(433.00,498.59)(1.123,0.482){9}{\rule{0.967pt}{0.116pt}}
\multiput(433.00,497.17)(10.994,6.000){2}{\rule{0.483pt}{0.400pt}}
\multiput(446.00,504.59)(1.123,0.482){9}{\rule{0.967pt}{0.116pt}}
\multiput(446.00,503.17)(10.994,6.000){2}{\rule{0.483pt}{0.400pt}}
\multiput(459.00,510.59)(1.489,0.477){7}{\rule{1.220pt}{0.115pt}}
\multiput(459.00,509.17)(11.468,5.000){2}{\rule{0.610pt}{0.400pt}}
\multiput(473.00,515.59)(1.123,0.482){9}{\rule{0.967pt}{0.116pt}}
\multiput(473.00,514.17)(10.994,6.000){2}{\rule{0.483pt}{0.400pt}}
\multiput(486.00,521.59)(1.378,0.477){7}{\rule{1.140pt}{0.115pt}}
\multiput(486.00,520.17)(10.634,5.000){2}{\rule{0.570pt}{0.400pt}}
\multiput(499.00,526.59)(1.378,0.477){7}{\rule{1.140pt}{0.115pt}}
\multiput(499.00,525.17)(10.634,5.000){2}{\rule{0.570pt}{0.400pt}}
\multiput(512.00,531.59)(1.378,0.477){7}{\rule{1.140pt}{0.115pt}}
\multiput(512.00,530.17)(10.634,5.000){2}{\rule{0.570pt}{0.400pt}}
\multiput(525.00,536.60)(1.797,0.468){5}{\rule{1.400pt}{0.113pt}}
\multiput(525.00,535.17)(10.094,4.000){2}{\rule{0.700pt}{0.400pt}}
\multiput(538.00,540.59)(1.378,0.477){7}{\rule{1.140pt}{0.115pt}}
\multiput(538.00,539.17)(10.634,5.000){2}{\rule{0.570pt}{0.400pt}}
\multiput(551.00,545.60)(1.797,0.468){5}{\rule{1.400pt}{0.113pt}}
\multiput(551.00,544.17)(10.094,4.000){2}{\rule{0.700pt}{0.400pt}}
\multiput(564.00,549.60)(1.797,0.468){5}{\rule{1.400pt}{0.113pt}}
\multiput(564.00,548.17)(10.094,4.000){2}{\rule{0.700pt}{0.400pt}}
\multiput(577.00,553.59)(1.378,0.477){7}{\rule{1.140pt}{0.115pt}}
\multiput(577.00,552.17)(10.634,5.000){2}{\rule{0.570pt}{0.400pt}}
\multiput(590.00,558.59)(1.214,0.482){9}{\rule{1.033pt}{0.116pt}}
\multiput(590.00,557.17)(11.855,6.000){2}{\rule{0.517pt}{0.400pt}}
\multiput(604.00,564.59)(0.824,0.488){13}{\rule{0.750pt}{0.117pt}}
\multiput(604.00,563.17)(11.443,8.000){2}{\rule{0.375pt}{0.400pt}}
\multiput(617.00,572.58)(0.590,0.492){19}{\rule{0.573pt}{0.118pt}}
\multiput(617.00,571.17)(11.811,11.000){2}{\rule{0.286pt}{0.400pt}}
\multiput(630.00,583.58)(0.539,0.492){21}{\rule{0.533pt}{0.119pt}}
\multiput(630.00,582.17)(11.893,12.000){2}{\rule{0.267pt}{0.400pt}}
\multiput(643.00,595.58)(0.497,0.493){23}{\rule{0.500pt}{0.119pt}}
\multiput(643.00,594.17)(11.962,13.000){2}{\rule{0.250pt}{0.400pt}}
\multiput(656.00,608.58)(0.590,0.492){19}{\rule{0.573pt}{0.118pt}}
\multiput(656.00,607.17)(11.811,11.000){2}{\rule{0.286pt}{0.400pt}}
\multiput(669.00,619.58)(0.652,0.491){17}{\rule{0.620pt}{0.118pt}}
\multiput(669.00,618.17)(11.713,10.000){2}{\rule{0.310pt}{0.400pt}}
\multiput(682.00,629.59)(0.824,0.488){13}{\rule{0.750pt}{0.117pt}}
\multiput(682.00,628.17)(11.443,8.000){2}{\rule{0.375pt}{0.400pt}}
\multiput(695.00,637.59)(1.123,0.482){9}{\rule{0.967pt}{0.116pt}}
\multiput(695.00,636.17)(10.994,6.000){2}{\rule{0.483pt}{0.400pt}}
\multiput(708.00,643.60)(1.797,0.468){5}{\rule{1.400pt}{0.113pt}}
\multiput(708.00,642.17)(10.094,4.000){2}{\rule{0.700pt}{0.400pt}}
\multiput(721.00,647.60)(1.943,0.468){5}{\rule{1.500pt}{0.113pt}}
\multiput(721.00,646.17)(10.887,4.000){2}{\rule{0.750pt}{0.400pt}}
\multiput(735.00,651.60)(1.797,0.468){5}{\rule{1.400pt}{0.113pt}}
\multiput(735.00,650.17)(10.094,4.000){2}{\rule{0.700pt}{0.400pt}}
\multiput(748.00,655.61)(2.695,0.447){3}{\rule{1.833pt}{0.108pt}}
\multiput(748.00,654.17)(9.195,3.000){2}{\rule{0.917pt}{0.400pt}}
\multiput(761.00,658.61)(2.695,0.447){3}{\rule{1.833pt}{0.108pt}}
\multiput(761.00,657.17)(9.195,3.000){2}{\rule{0.917pt}{0.400pt}}
\multiput(774.00,661.61)(2.695,0.447){3}{\rule{1.833pt}{0.108pt}}
\multiput(774.00,660.17)(9.195,3.000){2}{\rule{0.917pt}{0.400pt}}
\multiput(787.00,664.61)(2.695,0.447){3}{\rule{1.833pt}{0.108pt}}
\multiput(787.00,663.17)(9.195,3.000){2}{\rule{0.917pt}{0.400pt}}
\multiput(800.00,667.61)(2.695,0.447){3}{\rule{1.833pt}{0.108pt}}
\multiput(800.00,666.17)(9.195,3.000){2}{\rule{0.917pt}{0.400pt}}
\multiput(813.00,670.61)(2.695,0.447){3}{\rule{1.833pt}{0.108pt}}
\multiput(813.00,669.17)(9.195,3.000){2}{\rule{0.917pt}{0.400pt}}
\put(826,673.17){\rule{2.700pt}{0.400pt}}
\multiput(826.00,672.17)(7.396,2.000){2}{\rule{1.350pt}{0.400pt}}
\multiput(839.00,675.61)(2.695,0.447){3}{\rule{1.833pt}{0.108pt}}
\multiput(839.00,674.17)(9.195,3.000){2}{\rule{0.917pt}{0.400pt}}
\put(852,678.17){\rule{2.900pt}{0.400pt}}
\multiput(852.00,677.17)(7.981,2.000){2}{\rule{1.450pt}{0.400pt}}
\put(866,680.17){\rule{2.700pt}{0.400pt}}
\multiput(866.00,679.17)(7.396,2.000){2}{\rule{1.350pt}{0.400pt}}
\put(879,682.17){\rule{2.700pt}{0.400pt}}
\multiput(879.00,681.17)(7.396,2.000){2}{\rule{1.350pt}{0.400pt}}
\put(892,684.17){\rule{2.700pt}{0.400pt}}
\multiput(892.00,683.17)(7.396,2.000){2}{\rule{1.350pt}{0.400pt}}
\put(905,686.17){\rule{2.700pt}{0.400pt}}
\multiput(905.00,685.17)(7.396,2.000){2}{\rule{1.350pt}{0.400pt}}
\put(918,688.17){\rule{2.700pt}{0.400pt}}
\multiput(918.00,687.17)(7.396,2.000){2}{\rule{1.350pt}{0.400pt}}
\put(931,690.17){\rule{2.700pt}{0.400pt}}
\multiput(931.00,689.17)(7.396,2.000){2}{\rule{1.350pt}{0.400pt}}
\put(944,692.17){\rule{2.700pt}{0.400pt}}
\multiput(944.00,691.17)(7.396,2.000){2}{\rule{1.350pt}{0.400pt}}
\put(957,694.17){\rule{2.700pt}{0.400pt}}
\multiput(957.00,693.17)(7.396,2.000){2}{\rule{1.350pt}{0.400pt}}
\put(970,696.17){\rule{2.700pt}{0.400pt}}
\multiput(970.00,695.17)(7.396,2.000){2}{\rule{1.350pt}{0.400pt}}
\put(983,698.17){\rule{2.900pt}{0.400pt}}
\multiput(983.00,697.17)(7.981,2.000){2}{\rule{1.450pt}{0.400pt}}
\put(997,699.67){\rule{3.132pt}{0.400pt}}
\multiput(997.00,699.17)(6.500,1.000){2}{\rule{1.566pt}{0.400pt}}
\put(1010,701.17){\rule{2.700pt}{0.400pt}}
\multiput(1010.00,700.17)(7.396,2.000){2}{\rule{1.350pt}{0.400pt}}
\put(1023,703.17){\rule{2.700pt}{0.400pt}}
\multiput(1023.00,702.17)(7.396,2.000){2}{\rule{1.350pt}{0.400pt}}
\put(1036,705.17){\rule{2.700pt}{0.400pt}}
\multiput(1036.00,704.17)(7.396,2.000){2}{\rule{1.350pt}{0.400pt}}
\put(1049,706.67){\rule{3.132pt}{0.400pt}}
\multiput(1049.00,706.17)(6.500,1.000){2}{\rule{1.566pt}{0.400pt}}
\put(1062,708.17){\rule{2.700pt}{0.400pt}}
\multiput(1062.00,707.17)(7.396,2.000){2}{\rule{1.350pt}{0.400pt}}
\put(1075,710.17){\rule{2.700pt}{0.400pt}}
\multiput(1075.00,709.17)(7.396,2.000){2}{\rule{1.350pt}{0.400pt}}
\put(1088,711.67){\rule{3.132pt}{0.400pt}}
\multiput(1088.00,711.17)(6.500,1.000){2}{\rule{1.566pt}{0.400pt}}
\put(1101,713.17){\rule{2.700pt}{0.400pt}}
\multiput(1101.00,712.17)(7.396,2.000){2}{\rule{1.350pt}{0.400pt}}
\put(1114,715.17){\rule{2.900pt}{0.400pt}}
\multiput(1114.00,714.17)(7.981,2.000){2}{\rule{1.450pt}{0.400pt}}
\put(1128,716.67){\rule{3.132pt}{0.400pt}}
\multiput(1128.00,716.17)(6.500,1.000){2}{\rule{1.566pt}{0.400pt}}
\put(1141,718.17){\rule{2.700pt}{0.400pt}}
\multiput(1141.00,717.17)(7.396,2.000){2}{\rule{1.350pt}{0.400pt}}
\put(1154,719.67){\rule{3.132pt}{0.400pt}}
\multiput(1154.00,719.17)(6.500,1.000){2}{\rule{1.566pt}{0.400pt}}
\put(1167,721.17){\rule{2.700pt}{0.400pt}}
\multiput(1167.00,720.17)(7.396,2.000){2}{\rule{1.350pt}{0.400pt}}
\put(1180,722.67){\rule{3.132pt}{0.400pt}}
\multiput(1180.00,722.17)(6.500,1.000){2}{\rule{1.566pt}{0.400pt}}
\put(1193,723.67){\rule{3.132pt}{0.400pt}}
\multiput(1193.00,723.17)(6.500,1.000){2}{\rule{1.566pt}{0.400pt}}
\put(1206,725.17){\rule{2.700pt}{0.400pt}}
\multiput(1206.00,724.17)(7.396,2.000){2}{\rule{1.350pt}{0.400pt}}
\put(1219,726.67){\rule{3.132pt}{0.400pt}}
\multiput(1219.00,726.17)(6.500,1.000){2}{\rule{1.566pt}{0.400pt}}
\put(1232,728.17){\rule{2.700pt}{0.400pt}}
\multiput(1232.00,727.17)(7.396,2.000){2}{\rule{1.350pt}{0.400pt}}
\put(1245,729.67){\rule{3.373pt}{0.400pt}}
\multiput(1245.00,729.17)(7.000,1.000){2}{\rule{1.686pt}{0.400pt}}
\put(1259,731.17){\rule{2.700pt}{0.400pt}}
\multiput(1259.00,730.17)(7.396,2.000){2}{\rule{1.350pt}{0.400pt}}
\put(1272,732.67){\rule{3.132pt}{0.400pt}}
\multiput(1272.00,732.17)(6.500,1.000){2}{\rule{1.566pt}{0.400pt}}
\put(1285,733.67){\rule{3.132pt}{0.400pt}}
\multiput(1285.00,733.17)(6.500,1.000){2}{\rule{1.566pt}{0.400pt}}
\put(1298,735.17){\rule{2.700pt}{0.400pt}}
\multiput(1298.00,734.17)(7.396,2.000){2}{\rule{1.350pt}{0.400pt}}
\put(1311,736.67){\rule{3.132pt}{0.400pt}}
\multiput(1311.00,736.17)(6.500,1.000){2}{\rule{1.566pt}{0.400pt}}
\put(1324,737.67){\rule{3.132pt}{0.400pt}}
\multiput(1324.00,737.17)(6.500,1.000){2}{\rule{1.566pt}{0.400pt}}
\put(1337,739.17){\rule{2.700pt}{0.400pt}}
\multiput(1337.00,738.17)(7.396,2.000){2}{\rule{1.350pt}{0.400pt}}
\put(1350,740.67){\rule{3.132pt}{0.400pt}}
\multiput(1350.00,740.17)(6.500,1.000){2}{\rule{1.566pt}{0.400pt}}
\put(1363,741.67){\rule{3.132pt}{0.400pt}}
\multiput(1363.00,741.17)(6.500,1.000){2}{\rule{1.566pt}{0.400pt}}
\put(1376,743.17){\rule{2.900pt}{0.400pt}}
\multiput(1376.00,742.17)(7.981,2.000){2}{\rule{1.450pt}{0.400pt}}
\put(1390,744.67){\rule{3.132pt}{0.400pt}}
\multiput(1390.00,744.17)(6.500,1.000){2}{\rule{1.566pt}{0.400pt}}
\put(1403,745.67){\rule{3.132pt}{0.400pt}}
\multiput(1403.00,745.17)(6.500,1.000){2}{\rule{1.566pt}{0.400pt}}
\put(1416,746.67){\rule{3.132pt}{0.400pt}}
\multiput(1416.00,746.17)(6.500,1.000){2}{\rule{1.566pt}{0.400pt}}
\put(1429,747.67){\rule{2.409pt}{0.400pt}}
\multiput(1429.00,747.17)(5.000,1.000){2}{\rule{1.204pt}{0.400pt}}
\sbox{\plotpoint}{\rule[-0.500pt]{1.000pt}{1.000pt}}%
\sbox{\plotpoint}{\rule[-0.200pt]{0.400pt}{0.400pt}}%
\put(1279,123){\makebox(0,0)[r]{algorytm z progiem}}
\sbox{\plotpoint}{\rule[-0.500pt]{1.000pt}{1.000pt}}%
\multiput(1299,123)(20.756,0.000){5}{\usebox{\plotpoint}}
\put(1399,123){\usebox{\plotpoint}}
\put(171,134){\usebox{\plotpoint}}
\multiput(171,134)(4.466,20.269){3}{\usebox{\plotpoint}}
\multiput(184,193)(3.591,20.442){4}{\usebox{\plotpoint}}
\multiput(197,267)(6.857,19.590){2}{\usebox{\plotpoint}}
\put(216.01,314.32){\usebox{\plotpoint}}
\put(227.25,331.75){\usebox{\plotpoint}}
\multiput(237,349)(7.227,19.457){2}{\usebox{\plotpoint}}
\multiput(250,384)(8.253,19.044){2}{\usebox{\plotpoint}}
\put(271.79,424.81){\usebox{\plotpoint}}
\put(288.78,435.90){\usebox{\plotpoint}}
\put(306.22,446.90){\usebox{\plotpoint}}
\put(319.97,462.27){\usebox{\plotpoint}}
\put(331.63,479.44){\usebox{\plotpoint}}
\put(343.35,496.56){\usebox{\plotpoint}}
\put(357.61,511.41){\usebox{\plotpoint}}
\put(376.73,519.01){\usebox{\plotpoint}}
\put(397.32,521.26){\usebox{\plotpoint}}
\put(418.05,522.00){\usebox{\plotpoint}}
\put(438.70,523.88){\usebox{\plotpoint}}
\put(458.78,528.93){\usebox{\plotpoint}}
\put(475.74,540.74){\usebox{\plotpoint}}
\put(489.94,555.85){\usebox{\plotpoint}}
\put(503.35,571.68){\usebox{\plotpoint}}
\put(518.37,585.90){\usebox{\plotpoint}}
\put(536.57,595.45){\usebox{\plotpoint}}
\put(557.05,598.47){\usebox{\plotpoint}}
\put(577.75,600.00){\usebox{\plotpoint}}
\put(598.50,600.00){\usebox{\plotpoint}}
\put(619.26,600.00){\usebox{\plotpoint}}
\put(639.98,600.77){\usebox{\plotpoint}}
\put(660.73,601.00){\usebox{\plotpoint}}
\put(681.49,601.00){\usebox{\plotpoint}}
\put(702.24,601.00){\usebox{\plotpoint}}
\put(722.96,602.00){\usebox{\plotpoint}}
\put(743.71,602.00){\usebox{\plotpoint}}
\put(764.28,604.50){\usebox{\plotpoint}}
\put(783.91,610.58){\usebox{\plotpoint}}
\put(801.54,621.42){\usebox{\plotpoint}}
\put(816.51,635.78){\usebox{\plotpoint}}
\put(831.00,650.62){\usebox{\plotpoint}}
\put(847.41,663.17){\usebox{\plotpoint}}
\put(866.13,672.02){\usebox{\plotpoint}}
\put(886.71,674.59){\usebox{\plotpoint}}
\put(907.41,676.00){\usebox{\plotpoint}}
\put(928.17,676.00){\usebox{\plotpoint}}
\put(948.92,676.00){\usebox{\plotpoint}}
\put(969.68,676.00){\usebox{\plotpoint}}
\put(990.43,676.00){\usebox{\plotpoint}}
\put(1011.19,676.00){\usebox{\plotpoint}}
\put(1031.94,676.00){\usebox{\plotpoint}}
\put(1052.70,676.00){\usebox{\plotpoint}}
\put(1073.45,676.00){\usebox{\plotpoint}}
\put(1094.19,676.48){\usebox{\plotpoint}}
\put(1114.93,677.00){\usebox{\plotpoint}}
\put(1135.68,677.00){\usebox{\plotpoint}}
\put(1156.44,677.00){\usebox{\plotpoint}}
\put(1177.19,677.00){\usebox{\plotpoint}}
\put(1197.95,677.00){\usebox{\plotpoint}}
\put(1218.70,677.00){\usebox{\plotpoint}}
\put(1239.46,677.00){\usebox{\plotpoint}}
\put(1260.22,677.00){\usebox{\plotpoint}}
\put(1280.97,677.00){\usebox{\plotpoint}}
\put(1301.73,677.00){\usebox{\plotpoint}}
\put(1322.48,677.00){\usebox{\plotpoint}}
\put(1343.20,678.00){\usebox{\plotpoint}}
\put(1363.95,678.00){\usebox{\plotpoint}}
\put(1384.71,678.00){\usebox{\plotpoint}}
\put(1405.47,678.00){\usebox{\plotpoint}}
\put(1426.22,678.00){\usebox{\plotpoint}}
\put(1439,678){\usebox{\plotpoint}}
\sbox{\plotpoint}{\rule[-0.200pt]{0.400pt}{0.400pt}}%
\put(170.0,82.0){\rule[-0.200pt]{0.400pt}{187.179pt}}
\put(170.0,82.0){\rule[-0.200pt]{305.702pt}{0.400pt}}
\put(1439.0,82.0){\rule[-0.200pt]{0.400pt}{187.179pt}}
\put(170.0,859.0){\rule[-0.200pt]{305.702pt}{0.400pt}}
\end{picture}

\caption{Zależność czasu działania od wielkości macierzy dla metody naturalnej i połączonej}
\end{center}
\end{figure}
Wykres przedstawiony na rysunku \textbf{3.1} pozwolił nam na wyciągnięcie wniosku, że dla badanego przez nas
zakresu wielkości macierzy, mimo że algorytm Strassena ma mniejszą złożoność
obliczeniową, jest wolniejszy od algorytmu naturalnego niemalże tysiąckrotnie. W związku z tym
postanowiliśmy zbadać, czy połączenie obydwu algorytmów nie dałoby znacznie
lepszych wyników. Na podstawie przeprowadzonych przez nas doświadczeń wyznaczyliśmy
próg na $n=128$. Oznacza to, że dla mniejszych macierzy (z zakresu
$n \in [4, 128]$)do obliczeń wykorzystujemy naturalny algorytm mnożenia macierzy,
dla reszty zaś algorytm Strassena. Porównaliśmy czas pracy otrzymanej metody z metodą
mnożenia macierzy z definicji dla zakresu danych $n \in [2, 2000]$. Nasze przypuszczenia okazały się być słuszne.
Połączenie odydwu algorytmów wymagakrótszego czasu na mnożnie macierzy, co obrazuje wykres z rysunku \textbf{3.2}.
\subsection{Porównanie dokładności algorytmów}
W celu zbadania dokładności metod, obliczyliśmy wartości odpowiednich
współczynników. Pierwszy z nich zadany jest wzorem:
$$\Delta(XX^{-1}-I),$$
gdzie $I$ jest macierzą jednostkową, natomiast $\Delta(X) = \sum_{i=1}^{n}
\sum_{j=1}^{n} x_{ij}^2.$

Aby obliczyć ten wartość zaprezentowanego powyżej współczynnika potrzebowaliśmy macierzy o znanej odwrotności, która zadana jest wzorem. Wykorzystaliśmy jeden z typów macierzy trójdiagonalnych. Niech więc $B = (b_{ij})$ będzie macierzą $n \times n$ zadaną przez
%$$b_{ij}=b_i,  i=j,$$
%$$b_{ij}=\delta_{i, j-1}, i<j,$$
%$$b_{ij}=\delta_{i-1, j}, \and i>j,$$
$$b_{ij} = \left\{\begin{matrix}b_i & \mbox{jeśli } i=j \\\delta_{i, j-1} & \mbox{jeśli } i<j \\\delta_{i-1,j} & \mbox{jeśli } i>j \end{matrix}\right.$$
gdzie $b_i=b_{n-i+1}$ oraz $\delta_{ij}$ to tzw. delta Korneckera. Definiując
$b_kr_{k-1}+r_{k-2}, k = 2, ... , n-1$ i $r=(b_nr_{n-1}+r_{n-2})$, gdzie
$r_0=1, r_1=-b_1$ i określając macierz $C=(c_{ij})$ rozmiaru $n \times n$, gdzie
%$$c_{ij}=r^{-1}r_{i-1}r_{n-j}, \and i \leq j,$$
%$$c_{ij}=c_{ji}, \and i>j,$$
$$c_{ij} = \left\{\begin{matrix}r^{-1}r_{i-j}r_{n-j} & \mbox{jeśli } i \leq j  \\c_{ji} & \mbox{jeśli } i > j \end{matrix}\right.$$
dostajemy macierz odwrotną do macierzy $B$ ($B^{-1}=C$).

Obliczenia przeprowdziliśmy posługując się arytmetykami single oraz double. Rysunki \textbf{3.3} oraz \textbf{3.4} obrazują wyniki naszych doświadczeń. Macierze trójdiagonalne danych rozmiarów generowaliśmy losowo. 


