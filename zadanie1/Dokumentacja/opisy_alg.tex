\section{Opis algorytmów}
\subsection{Agorytm naturalny mnożenia macierzy}
Pierwszy badany przez nas algorytm jest bezpośrednią realizacją definicji
mnożenia macierzy. Mając dane $A,B \in M_{n}$, gdzie $M_{n}$ jest zbiorem
macierzy kwadratowych o szerokości i długości $n$,ich iloczyn określamy jako
$$ \begin{bmatrix}
a_{11} & a_{12} & \ldots & a_{1n} \\
a_{21} & \hdotsfor{3} \\
\hdotsfor{4}\\
a_{n1} & \hdotsfor{2} & a_{nn} \end{bmatrix}
\begin{bmatrix}
b_{11} & b_{12} & \ldots & b_{1n} \\
b_{21} & \hdotsfor{3} \\
\hdotsfor{4}\\
b_{n1} & \hdotsfor{2} & b_{nn} \end{bmatrix}=
\begin{bmatrix}
c_{11} & c_{12} & \ldots & c_{1n} \\
a_{21} & \hdotsfor{3} \\
\hdotsfor{4}\\
c_{n1} & \hdotsfor{2} & c_{nn} \end{bmatrix},
$$
gdzie$c_{ij}=\sum_{k=1}^{n} a_{ik}b_{kj}$. 

Pozytywną cechą tego algorytmu jest jego prostota, jednakże obliczenie wartości
każdego elementu macierzy wynikowej musimy wykonać $n$ mnożeń oraz $n$ dodawań,
co przy uwzględnieniu ilości elementów w wyniku daje złożoność $O(n^3)$.
Jak się okazuje istnieją algorytmy o lepszej złożoności. Jednym z nich jest
algorytm Strassena\footnote{Tu będzie przypis o hipotezie Strassena, jak
dostaniem ksiazke od Michała}.
\subsection{Algorytm Strassena mnożenia macierzy}
Rozważamy jedynie macierze kwadratowe, których rozmiar wynosi $2^n$, $n \in
\mathbf{N}$ (jeżeli wymiar macierzy nie jest tej postaci, uzupełamy badaną
macierz zerami tak, aby spełniała to założenie). W pojedynczym wywołaniu
algorytmu Strassena, macierze wejściowe $X$ i $Y$ dzielone są na 4 podmacierze
tych samych rozmiarów.
$$X=\begin{bmatrix} A & B \\ C & D \end{bmatrix},	
Y=\begin{bmatrix} E & G \\ F & H \end{bmatrix},		
Z=XY=\begin{bmatrix} R & S \\ T & U \end{bmatrix}$$

Wówczas
$$ R=AE+BF, S=AG+BH, T=CE+DF, U=CG+DH.$$
Aby więc obliczyć iloczyn macierzy musimy wykonć 8 mnożeń macierzy stopnia
$\frac{n}{2}$, stąd złożoność algorytmu nie ulega zmianie. Wystarczy jednak
wprowadzić lekkie modyfikacje, aby  w każdym wykonaniu rekurencyjnym wykonywać
7 zamiast 8 multiplikacji. Mówi o tym poniższy
\begin{lm}{Zachodzą następujące równości:
$$R=P_5+P_4-P_2+P_6, S=P_1+P_2, T=P_3+P_4, U=P_5+P_1-P_3-P_7,$$
gdzie
$$P_1=A(G-H), P_2=(A+B)H, P_3=(C+D)E, P_4=D(F-E),$$
$$P_5=(A+D)(E+H), P_6=(B-D)(F+H), P_7 = (A-C)(E+G).$$}
\end{lm}
\begin{proof}
Sprawdzimy poprawność równości $S=P_1+P_2$ (pozostałe równości sprawdza się
analogicznie).
$$S=P_1+P_2=A(G-H)+(A+B)H=AG-AH+AH-BH=AG+BH$$
\end{proof}
W każdym wywołaniu algorytmu Strassena dla macierzy dwukrotnie mniejszego
rozmiaru wykonujemy 7 mnożeń, stąd złożoność algorytmu to $O(n^{\log{2}{7}})
\approx O(2^{2.807}).$
