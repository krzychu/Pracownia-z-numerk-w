\section{Przebieg doświadczenia numerycznego}
\subsection{Porównanie szybkości algorytmów}
Dla porównania szybkości obydwu algorytmów opisanych powyżej wygenerowaliśmy
losowo macierze z zakresu wielkości $n \in [2, 500]$ i zmierzyliśmy czas ich
mnożenia przez obydwie metody. Otrzymane wyniki wykorzystaliśmy do wygenerowania wykresu zależności
czasu działania od wielkości macierzy dla alogrytmu Strassena i algorytmu
naturalnego.
\begin{figure}[h!tb]
\begin{center}
% GNUPLOT: LaTeX picture
\setlength{\unitlength}{0.240900pt}
\ifx\plotpoint\undefined\newsavebox{\plotpoint}\fi
\sbox{\plotpoint}{\rule[-0.200pt]{0.400pt}{0.400pt}}%
\begin{picture}(1500,900)(0,0)
\sbox{\plotpoint}{\rule[-0.200pt]{0.400pt}{0.400pt}}%
\put(170.0,82.0){\rule[-0.200pt]{4.818pt}{0.400pt}}
\put(150,82){\makebox(0,0)[r]{ 1}}
\put(1419.0,82.0){\rule[-0.200pt]{4.818pt}{0.400pt}}
\put(170.0,108.0){\rule[-0.200pt]{2.409pt}{0.400pt}}
\put(1429.0,108.0){\rule[-0.200pt]{2.409pt}{0.400pt}}
\put(170.0,123.0){\rule[-0.200pt]{2.409pt}{0.400pt}}
\put(1429.0,123.0){\rule[-0.200pt]{2.409pt}{0.400pt}}
\put(170.0,134.0){\rule[-0.200pt]{2.409pt}{0.400pt}}
\put(1429.0,134.0){\rule[-0.200pt]{2.409pt}{0.400pt}}
\put(170.0,142.0){\rule[-0.200pt]{2.409pt}{0.400pt}}
\put(1429.0,142.0){\rule[-0.200pt]{2.409pt}{0.400pt}}
\put(170.0,149.0){\rule[-0.200pt]{2.409pt}{0.400pt}}
\put(1429.0,149.0){\rule[-0.200pt]{2.409pt}{0.400pt}}
\put(170.0,155.0){\rule[-0.200pt]{2.409pt}{0.400pt}}
\put(1429.0,155.0){\rule[-0.200pt]{2.409pt}{0.400pt}}
\put(170.0,160.0){\rule[-0.200pt]{2.409pt}{0.400pt}}
\put(1429.0,160.0){\rule[-0.200pt]{2.409pt}{0.400pt}}
\put(170.0,164.0){\rule[-0.200pt]{2.409pt}{0.400pt}}
\put(1429.0,164.0){\rule[-0.200pt]{2.409pt}{0.400pt}}
\put(170.0,168.0){\rule[-0.200pt]{2.409pt}{0.400pt}}
\put(1429.0,168.0){\rule[-0.200pt]{2.409pt}{0.400pt}}
\put(170.0,172.0){\rule[-0.200pt]{2.409pt}{0.400pt}}
\put(1429.0,172.0){\rule[-0.200pt]{2.409pt}{0.400pt}}
\put(170.0,175.0){\rule[-0.200pt]{2.409pt}{0.400pt}}
\put(1429.0,175.0){\rule[-0.200pt]{2.409pt}{0.400pt}}
\put(170.0,178.0){\rule[-0.200pt]{2.409pt}{0.400pt}}
\put(1429.0,178.0){\rule[-0.200pt]{2.409pt}{0.400pt}}
\put(170.0,181.0){\rule[-0.200pt]{2.409pt}{0.400pt}}
\put(1429.0,181.0){\rule[-0.200pt]{2.409pt}{0.400pt}}
\put(170.0,184.0){\rule[-0.200pt]{2.409pt}{0.400pt}}
\put(1429.0,184.0){\rule[-0.200pt]{2.409pt}{0.400pt}}
\put(170.0,186.0){\rule[-0.200pt]{2.409pt}{0.400pt}}
\put(1429.0,186.0){\rule[-0.200pt]{2.409pt}{0.400pt}}
\put(170.0,188.0){\rule[-0.200pt]{2.409pt}{0.400pt}}
\put(1429.0,188.0){\rule[-0.200pt]{2.409pt}{0.400pt}}
\put(170.0,190.0){\rule[-0.200pt]{2.409pt}{0.400pt}}
\put(1429.0,190.0){\rule[-0.200pt]{2.409pt}{0.400pt}}
\put(170.0,192.0){\rule[-0.200pt]{2.409pt}{0.400pt}}
\put(1429.0,192.0){\rule[-0.200pt]{2.409pt}{0.400pt}}
\put(170.0,194.0){\rule[-0.200pt]{2.409pt}{0.400pt}}
\put(1429.0,194.0){\rule[-0.200pt]{2.409pt}{0.400pt}}
\put(170.0,196.0){\rule[-0.200pt]{2.409pt}{0.400pt}}
\put(1429.0,196.0){\rule[-0.200pt]{2.409pt}{0.400pt}}
\put(170.0,198.0){\rule[-0.200pt]{2.409pt}{0.400pt}}
\put(1429.0,198.0){\rule[-0.200pt]{2.409pt}{0.400pt}}
\put(170.0,200.0){\rule[-0.200pt]{2.409pt}{0.400pt}}
\put(1429.0,200.0){\rule[-0.200pt]{2.409pt}{0.400pt}}
\put(170.0,201.0){\rule[-0.200pt]{2.409pt}{0.400pt}}
\put(1429.0,201.0){\rule[-0.200pt]{2.409pt}{0.400pt}}
\put(170.0,203.0){\rule[-0.200pt]{2.409pt}{0.400pt}}
\put(1429.0,203.0){\rule[-0.200pt]{2.409pt}{0.400pt}}
\put(170.0,204.0){\rule[-0.200pt]{2.409pt}{0.400pt}}
\put(1429.0,204.0){\rule[-0.200pt]{2.409pt}{0.400pt}}
\put(170.0,206.0){\rule[-0.200pt]{2.409pt}{0.400pt}}
\put(1429.0,206.0){\rule[-0.200pt]{2.409pt}{0.400pt}}
\put(170.0,207.0){\rule[-0.200pt]{2.409pt}{0.400pt}}
\put(1429.0,207.0){\rule[-0.200pt]{2.409pt}{0.400pt}}
\put(170.0,208.0){\rule[-0.200pt]{2.409pt}{0.400pt}}
\put(1429.0,208.0){\rule[-0.200pt]{2.409pt}{0.400pt}}
\put(170.0,210.0){\rule[-0.200pt]{2.409pt}{0.400pt}}
\put(1429.0,210.0){\rule[-0.200pt]{2.409pt}{0.400pt}}
\put(170.0,211.0){\rule[-0.200pt]{2.409pt}{0.400pt}}
\put(1429.0,211.0){\rule[-0.200pt]{2.409pt}{0.400pt}}
\put(170.0,212.0){\rule[-0.200pt]{2.409pt}{0.400pt}}
\put(1429.0,212.0){\rule[-0.200pt]{2.409pt}{0.400pt}}
\put(170.0,213.0){\rule[-0.200pt]{2.409pt}{0.400pt}}
\put(1429.0,213.0){\rule[-0.200pt]{2.409pt}{0.400pt}}
\put(170.0,214.0){\rule[-0.200pt]{2.409pt}{0.400pt}}
\put(1429.0,214.0){\rule[-0.200pt]{2.409pt}{0.400pt}}
\put(170.0,215.0){\rule[-0.200pt]{2.409pt}{0.400pt}}
\put(1429.0,215.0){\rule[-0.200pt]{2.409pt}{0.400pt}}
\put(170.0,216.0){\rule[-0.200pt]{2.409pt}{0.400pt}}
\put(1429.0,216.0){\rule[-0.200pt]{2.409pt}{0.400pt}}
\put(170.0,217.0){\rule[-0.200pt]{2.409pt}{0.400pt}}
\put(1429.0,217.0){\rule[-0.200pt]{2.409pt}{0.400pt}}
\put(170.0,218.0){\rule[-0.200pt]{2.409pt}{0.400pt}}
\put(1429.0,218.0){\rule[-0.200pt]{2.409pt}{0.400pt}}
\put(170.0,219.0){\rule[-0.200pt]{2.409pt}{0.400pt}}
\put(1429.0,219.0){\rule[-0.200pt]{2.409pt}{0.400pt}}
\put(170.0,220.0){\rule[-0.200pt]{2.409pt}{0.400pt}}
\put(1429.0,220.0){\rule[-0.200pt]{2.409pt}{0.400pt}}
\put(170.0,221.0){\rule[-0.200pt]{2.409pt}{0.400pt}}
\put(1429.0,221.0){\rule[-0.200pt]{2.409pt}{0.400pt}}
\put(170.0,222.0){\rule[-0.200pt]{2.409pt}{0.400pt}}
\put(1429.0,222.0){\rule[-0.200pt]{2.409pt}{0.400pt}}
\put(170.0,223.0){\rule[-0.200pt]{2.409pt}{0.400pt}}
\put(1429.0,223.0){\rule[-0.200pt]{2.409pt}{0.400pt}}
\put(170.0,224.0){\rule[-0.200pt]{2.409pt}{0.400pt}}
\put(1429.0,224.0){\rule[-0.200pt]{2.409pt}{0.400pt}}
\put(170.0,225.0){\rule[-0.200pt]{2.409pt}{0.400pt}}
\put(1429.0,225.0){\rule[-0.200pt]{2.409pt}{0.400pt}}
\put(170.0,226.0){\rule[-0.200pt]{2.409pt}{0.400pt}}
\put(1429.0,226.0){\rule[-0.200pt]{2.409pt}{0.400pt}}
\put(170.0,226.0){\rule[-0.200pt]{2.409pt}{0.400pt}}
\put(1429.0,226.0){\rule[-0.200pt]{2.409pt}{0.400pt}}
\put(170.0,227.0){\rule[-0.200pt]{2.409pt}{0.400pt}}
\put(1429.0,227.0){\rule[-0.200pt]{2.409pt}{0.400pt}}
\put(170.0,228.0){\rule[-0.200pt]{2.409pt}{0.400pt}}
\put(1429.0,228.0){\rule[-0.200pt]{2.409pt}{0.400pt}}
\put(170.0,229.0){\rule[-0.200pt]{2.409pt}{0.400pt}}
\put(1429.0,229.0){\rule[-0.200pt]{2.409pt}{0.400pt}}
\put(170.0,229.0){\rule[-0.200pt]{2.409pt}{0.400pt}}
\put(1429.0,229.0){\rule[-0.200pt]{2.409pt}{0.400pt}}
\put(170.0,230.0){\rule[-0.200pt]{2.409pt}{0.400pt}}
\put(1429.0,230.0){\rule[-0.200pt]{2.409pt}{0.400pt}}
\put(170.0,231.0){\rule[-0.200pt]{2.409pt}{0.400pt}}
\put(1429.0,231.0){\rule[-0.200pt]{2.409pt}{0.400pt}}
\put(170.0,232.0){\rule[-0.200pt]{2.409pt}{0.400pt}}
\put(1429.0,232.0){\rule[-0.200pt]{2.409pt}{0.400pt}}
\put(170.0,232.0){\rule[-0.200pt]{2.409pt}{0.400pt}}
\put(1429.0,232.0){\rule[-0.200pt]{2.409pt}{0.400pt}}
\put(170.0,233.0){\rule[-0.200pt]{2.409pt}{0.400pt}}
\put(1429.0,233.0){\rule[-0.200pt]{2.409pt}{0.400pt}}
\put(170.0,234.0){\rule[-0.200pt]{2.409pt}{0.400pt}}
\put(1429.0,234.0){\rule[-0.200pt]{2.409pt}{0.400pt}}
\put(170.0,234.0){\rule[-0.200pt]{2.409pt}{0.400pt}}
\put(1429.0,234.0){\rule[-0.200pt]{2.409pt}{0.400pt}}
\put(170.0,235.0){\rule[-0.200pt]{2.409pt}{0.400pt}}
\put(1429.0,235.0){\rule[-0.200pt]{2.409pt}{0.400pt}}
\put(170.0,236.0){\rule[-0.200pt]{2.409pt}{0.400pt}}
\put(1429.0,236.0){\rule[-0.200pt]{2.409pt}{0.400pt}}
\put(170.0,236.0){\rule[-0.200pt]{2.409pt}{0.400pt}}
\put(1429.0,236.0){\rule[-0.200pt]{2.409pt}{0.400pt}}
\put(170.0,237.0){\rule[-0.200pt]{2.409pt}{0.400pt}}
\put(1429.0,237.0){\rule[-0.200pt]{2.409pt}{0.400pt}}
\put(170.0,237.0){\rule[-0.200pt]{2.409pt}{0.400pt}}
\put(1429.0,237.0){\rule[-0.200pt]{2.409pt}{0.400pt}}
\put(170.0,238.0){\rule[-0.200pt]{2.409pt}{0.400pt}}
\put(1429.0,238.0){\rule[-0.200pt]{2.409pt}{0.400pt}}
\put(170.0,239.0){\rule[-0.200pt]{2.409pt}{0.400pt}}
\put(1429.0,239.0){\rule[-0.200pt]{2.409pt}{0.400pt}}
\put(170.0,239.0){\rule[-0.200pt]{2.409pt}{0.400pt}}
\put(1429.0,239.0){\rule[-0.200pt]{2.409pt}{0.400pt}}
\put(170.0,240.0){\rule[-0.200pt]{2.409pt}{0.400pt}}
\put(1429.0,240.0){\rule[-0.200pt]{2.409pt}{0.400pt}}
\put(170.0,240.0){\rule[-0.200pt]{2.409pt}{0.400pt}}
\put(1429.0,240.0){\rule[-0.200pt]{2.409pt}{0.400pt}}
\put(170.0,241.0){\rule[-0.200pt]{2.409pt}{0.400pt}}
\put(1429.0,241.0){\rule[-0.200pt]{2.409pt}{0.400pt}}
\put(170.0,241.0){\rule[-0.200pt]{2.409pt}{0.400pt}}
\put(1429.0,241.0){\rule[-0.200pt]{2.409pt}{0.400pt}}
\put(170.0,242.0){\rule[-0.200pt]{2.409pt}{0.400pt}}
\put(1429.0,242.0){\rule[-0.200pt]{2.409pt}{0.400pt}}
\put(170.0,242.0){\rule[-0.200pt]{2.409pt}{0.400pt}}
\put(1429.0,242.0){\rule[-0.200pt]{2.409pt}{0.400pt}}
\put(170.0,243.0){\rule[-0.200pt]{2.409pt}{0.400pt}}
\put(1429.0,243.0){\rule[-0.200pt]{2.409pt}{0.400pt}}
\put(170.0,243.0){\rule[-0.200pt]{2.409pt}{0.400pt}}
\put(1429.0,243.0){\rule[-0.200pt]{2.409pt}{0.400pt}}
\put(170.0,244.0){\rule[-0.200pt]{2.409pt}{0.400pt}}
\put(1429.0,244.0){\rule[-0.200pt]{2.409pt}{0.400pt}}
\put(170.0,244.0){\rule[-0.200pt]{2.409pt}{0.400pt}}
\put(1429.0,244.0){\rule[-0.200pt]{2.409pt}{0.400pt}}
\put(170.0,245.0){\rule[-0.200pt]{2.409pt}{0.400pt}}
\put(1429.0,245.0){\rule[-0.200pt]{2.409pt}{0.400pt}}
\put(170.0,245.0){\rule[-0.200pt]{2.409pt}{0.400pt}}
\put(1429.0,245.0){\rule[-0.200pt]{2.409pt}{0.400pt}}
\put(170.0,246.0){\rule[-0.200pt]{2.409pt}{0.400pt}}
\put(1429.0,246.0){\rule[-0.200pt]{2.409pt}{0.400pt}}
\put(170.0,246.0){\rule[-0.200pt]{2.409pt}{0.400pt}}
\put(1429.0,246.0){\rule[-0.200pt]{2.409pt}{0.400pt}}
\put(170.0,247.0){\rule[-0.200pt]{2.409pt}{0.400pt}}
\put(1429.0,247.0){\rule[-0.200pt]{2.409pt}{0.400pt}}
\put(170.0,247.0){\rule[-0.200pt]{2.409pt}{0.400pt}}
\put(1429.0,247.0){\rule[-0.200pt]{2.409pt}{0.400pt}}
\put(170.0,248.0){\rule[-0.200pt]{2.409pt}{0.400pt}}
\put(1429.0,248.0){\rule[-0.200pt]{2.409pt}{0.400pt}}
\put(170.0,248.0){\rule[-0.200pt]{2.409pt}{0.400pt}}
\put(1429.0,248.0){\rule[-0.200pt]{2.409pt}{0.400pt}}
\put(170.0,249.0){\rule[-0.200pt]{2.409pt}{0.400pt}}
\put(1429.0,249.0){\rule[-0.200pt]{2.409pt}{0.400pt}}
\put(170.0,249.0){\rule[-0.200pt]{2.409pt}{0.400pt}}
\put(1429.0,249.0){\rule[-0.200pt]{2.409pt}{0.400pt}}
\put(170.0,249.0){\rule[-0.200pt]{2.409pt}{0.400pt}}
\put(1429.0,249.0){\rule[-0.200pt]{2.409pt}{0.400pt}}
\put(170.0,250.0){\rule[-0.200pt]{2.409pt}{0.400pt}}
\put(1429.0,250.0){\rule[-0.200pt]{2.409pt}{0.400pt}}
\put(170.0,250.0){\rule[-0.200pt]{2.409pt}{0.400pt}}
\put(1429.0,250.0){\rule[-0.200pt]{2.409pt}{0.400pt}}
\put(170.0,251.0){\rule[-0.200pt]{2.409pt}{0.400pt}}
\put(1429.0,251.0){\rule[-0.200pt]{2.409pt}{0.400pt}}
\put(170.0,251.0){\rule[-0.200pt]{2.409pt}{0.400pt}}
\put(1429.0,251.0){\rule[-0.200pt]{2.409pt}{0.400pt}}
\put(170.0,252.0){\rule[-0.200pt]{2.409pt}{0.400pt}}
\put(1429.0,252.0){\rule[-0.200pt]{2.409pt}{0.400pt}}
\put(170.0,252.0){\rule[-0.200pt]{2.409pt}{0.400pt}}
\put(1429.0,252.0){\rule[-0.200pt]{2.409pt}{0.400pt}}
\put(170.0,252.0){\rule[-0.200pt]{2.409pt}{0.400pt}}
\put(1429.0,252.0){\rule[-0.200pt]{2.409pt}{0.400pt}}
\put(170.0,253.0){\rule[-0.200pt]{2.409pt}{0.400pt}}
\put(1429.0,253.0){\rule[-0.200pt]{2.409pt}{0.400pt}}
\put(170.0,253.0){\rule[-0.200pt]{2.409pt}{0.400pt}}
\put(1429.0,253.0){\rule[-0.200pt]{2.409pt}{0.400pt}}
\put(170.0,254.0){\rule[-0.200pt]{2.409pt}{0.400pt}}
\put(1429.0,254.0){\rule[-0.200pt]{2.409pt}{0.400pt}}
\put(170.0,254.0){\rule[-0.200pt]{2.409pt}{0.400pt}}
\put(1429.0,254.0){\rule[-0.200pt]{2.409pt}{0.400pt}}
\put(170.0,254.0){\rule[-0.200pt]{2.409pt}{0.400pt}}
\put(1429.0,254.0){\rule[-0.200pt]{2.409pt}{0.400pt}}
\put(170.0,255.0){\rule[-0.200pt]{2.409pt}{0.400pt}}
\put(1429.0,255.0){\rule[-0.200pt]{2.409pt}{0.400pt}}
\put(170.0,255.0){\rule[-0.200pt]{2.409pt}{0.400pt}}
\put(1429.0,255.0){\rule[-0.200pt]{2.409pt}{0.400pt}}
\put(170.0,255.0){\rule[-0.200pt]{2.409pt}{0.400pt}}
\put(1429.0,255.0){\rule[-0.200pt]{2.409pt}{0.400pt}}
\put(170.0,256.0){\rule[-0.200pt]{2.409pt}{0.400pt}}
\put(1429.0,256.0){\rule[-0.200pt]{2.409pt}{0.400pt}}
\put(170.0,256.0){\rule[-0.200pt]{2.409pt}{0.400pt}}
\put(1429.0,256.0){\rule[-0.200pt]{2.409pt}{0.400pt}}
\put(170.0,256.0){\rule[-0.200pt]{2.409pt}{0.400pt}}
\put(1429.0,256.0){\rule[-0.200pt]{2.409pt}{0.400pt}}
\put(170.0,257.0){\rule[-0.200pt]{2.409pt}{0.400pt}}
\put(1429.0,257.0){\rule[-0.200pt]{2.409pt}{0.400pt}}
\put(170.0,257.0){\rule[-0.200pt]{2.409pt}{0.400pt}}
\put(1429.0,257.0){\rule[-0.200pt]{2.409pt}{0.400pt}}
\put(170.0,258.0){\rule[-0.200pt]{2.409pt}{0.400pt}}
\put(1429.0,258.0){\rule[-0.200pt]{2.409pt}{0.400pt}}
\put(170.0,258.0){\rule[-0.200pt]{2.409pt}{0.400pt}}
\put(1429.0,258.0){\rule[-0.200pt]{2.409pt}{0.400pt}}
\put(170.0,258.0){\rule[-0.200pt]{2.409pt}{0.400pt}}
\put(1429.0,258.0){\rule[-0.200pt]{2.409pt}{0.400pt}}
\put(170.0,259.0){\rule[-0.200pt]{2.409pt}{0.400pt}}
\put(1429.0,259.0){\rule[-0.200pt]{2.409pt}{0.400pt}}
\put(170.0,259.0){\rule[-0.200pt]{2.409pt}{0.400pt}}
\put(1429.0,259.0){\rule[-0.200pt]{2.409pt}{0.400pt}}
\put(170.0,259.0){\rule[-0.200pt]{2.409pt}{0.400pt}}
\put(1429.0,259.0){\rule[-0.200pt]{2.409pt}{0.400pt}}
\put(170.0,260.0){\rule[-0.200pt]{2.409pt}{0.400pt}}
\put(1429.0,260.0){\rule[-0.200pt]{2.409pt}{0.400pt}}
\put(170.0,260.0){\rule[-0.200pt]{2.409pt}{0.400pt}}
\put(1429.0,260.0){\rule[-0.200pt]{2.409pt}{0.400pt}}
\put(170.0,260.0){\rule[-0.200pt]{2.409pt}{0.400pt}}
\put(1429.0,260.0){\rule[-0.200pt]{2.409pt}{0.400pt}}
\put(170.0,261.0){\rule[-0.200pt]{2.409pt}{0.400pt}}
\put(1429.0,261.0){\rule[-0.200pt]{2.409pt}{0.400pt}}
\put(170.0,261.0){\rule[-0.200pt]{2.409pt}{0.400pt}}
\put(1429.0,261.0){\rule[-0.200pt]{2.409pt}{0.400pt}}
\put(170.0,261.0){\rule[-0.200pt]{2.409pt}{0.400pt}}
\put(1429.0,261.0){\rule[-0.200pt]{2.409pt}{0.400pt}}
\put(170.0,262.0){\rule[-0.200pt]{2.409pt}{0.400pt}}
\put(1429.0,262.0){\rule[-0.200pt]{2.409pt}{0.400pt}}
\put(170.0,262.0){\rule[-0.200pt]{2.409pt}{0.400pt}}
\put(1429.0,262.0){\rule[-0.200pt]{2.409pt}{0.400pt}}
\put(170.0,262.0){\rule[-0.200pt]{2.409pt}{0.400pt}}
\put(1429.0,262.0){\rule[-0.200pt]{2.409pt}{0.400pt}}
\put(170.0,262.0){\rule[-0.200pt]{2.409pt}{0.400pt}}
\put(1429.0,262.0){\rule[-0.200pt]{2.409pt}{0.400pt}}
\put(170.0,263.0){\rule[-0.200pt]{2.409pt}{0.400pt}}
\put(1429.0,263.0){\rule[-0.200pt]{2.409pt}{0.400pt}}
\put(170.0,263.0){\rule[-0.200pt]{2.409pt}{0.400pt}}
\put(1429.0,263.0){\rule[-0.200pt]{2.409pt}{0.400pt}}
\put(170.0,263.0){\rule[-0.200pt]{2.409pt}{0.400pt}}
\put(1429.0,263.0){\rule[-0.200pt]{2.409pt}{0.400pt}}
\put(170.0,264.0){\rule[-0.200pt]{2.409pt}{0.400pt}}
\put(1429.0,264.0){\rule[-0.200pt]{2.409pt}{0.400pt}}
\put(170.0,264.0){\rule[-0.200pt]{2.409pt}{0.400pt}}
\put(1429.0,264.0){\rule[-0.200pt]{2.409pt}{0.400pt}}
\put(170.0,264.0){\rule[-0.200pt]{2.409pt}{0.400pt}}
\put(1429.0,264.0){\rule[-0.200pt]{2.409pt}{0.400pt}}
\put(170.0,265.0){\rule[-0.200pt]{2.409pt}{0.400pt}}
\put(1429.0,265.0){\rule[-0.200pt]{2.409pt}{0.400pt}}
\put(170.0,265.0){\rule[-0.200pt]{2.409pt}{0.400pt}}
\put(1429.0,265.0){\rule[-0.200pt]{2.409pt}{0.400pt}}
\put(170.0,265.0){\rule[-0.200pt]{2.409pt}{0.400pt}}
\put(1429.0,265.0){\rule[-0.200pt]{2.409pt}{0.400pt}}
\put(170.0,265.0){\rule[-0.200pt]{2.409pt}{0.400pt}}
\put(1429.0,265.0){\rule[-0.200pt]{2.409pt}{0.400pt}}
\put(170.0,266.0){\rule[-0.200pt]{2.409pt}{0.400pt}}
\put(1429.0,266.0){\rule[-0.200pt]{2.409pt}{0.400pt}}
\put(170.0,266.0){\rule[-0.200pt]{2.409pt}{0.400pt}}
\put(1429.0,266.0){\rule[-0.200pt]{2.409pt}{0.400pt}}
\put(170.0,266.0){\rule[-0.200pt]{2.409pt}{0.400pt}}
\put(1429.0,266.0){\rule[-0.200pt]{2.409pt}{0.400pt}}
\put(170.0,266.0){\rule[-0.200pt]{2.409pt}{0.400pt}}
\put(1429.0,266.0){\rule[-0.200pt]{2.409pt}{0.400pt}}
\put(170.0,267.0){\rule[-0.200pt]{2.409pt}{0.400pt}}
\put(1429.0,267.0){\rule[-0.200pt]{2.409pt}{0.400pt}}
\put(170.0,267.0){\rule[-0.200pt]{2.409pt}{0.400pt}}
\put(1429.0,267.0){\rule[-0.200pt]{2.409pt}{0.400pt}}
\put(170.0,267.0){\rule[-0.200pt]{2.409pt}{0.400pt}}
\put(1429.0,267.0){\rule[-0.200pt]{2.409pt}{0.400pt}}
\put(170.0,268.0){\rule[-0.200pt]{2.409pt}{0.400pt}}
\put(1429.0,268.0){\rule[-0.200pt]{2.409pt}{0.400pt}}
\put(170.0,268.0){\rule[-0.200pt]{2.409pt}{0.400pt}}
\put(1429.0,268.0){\rule[-0.200pt]{2.409pt}{0.400pt}}
\put(170.0,268.0){\rule[-0.200pt]{2.409pt}{0.400pt}}
\put(1429.0,268.0){\rule[-0.200pt]{2.409pt}{0.400pt}}
\put(170.0,268.0){\rule[-0.200pt]{2.409pt}{0.400pt}}
\put(1429.0,268.0){\rule[-0.200pt]{2.409pt}{0.400pt}}
\put(170.0,269.0){\rule[-0.200pt]{2.409pt}{0.400pt}}
\put(1429.0,269.0){\rule[-0.200pt]{2.409pt}{0.400pt}}
\put(170.0,269.0){\rule[-0.200pt]{2.409pt}{0.400pt}}
\put(1429.0,269.0){\rule[-0.200pt]{2.409pt}{0.400pt}}
\put(170.0,269.0){\rule[-0.200pt]{2.409pt}{0.400pt}}
\put(1429.0,269.0){\rule[-0.200pt]{2.409pt}{0.400pt}}
\put(170.0,269.0){\rule[-0.200pt]{2.409pt}{0.400pt}}
\put(1429.0,269.0){\rule[-0.200pt]{2.409pt}{0.400pt}}
\put(170.0,270.0){\rule[-0.200pt]{2.409pt}{0.400pt}}
\put(1429.0,270.0){\rule[-0.200pt]{2.409pt}{0.400pt}}
\put(170.0,270.0){\rule[-0.200pt]{2.409pt}{0.400pt}}
\put(1429.0,270.0){\rule[-0.200pt]{2.409pt}{0.400pt}}
\put(170.0,270.0){\rule[-0.200pt]{2.409pt}{0.400pt}}
\put(1429.0,270.0){\rule[-0.200pt]{2.409pt}{0.400pt}}
\put(170.0,270.0){\rule[-0.200pt]{2.409pt}{0.400pt}}
\put(1429.0,270.0){\rule[-0.200pt]{2.409pt}{0.400pt}}
\put(170.0,271.0){\rule[-0.200pt]{2.409pt}{0.400pt}}
\put(1429.0,271.0){\rule[-0.200pt]{2.409pt}{0.400pt}}
\put(170.0,271.0){\rule[-0.200pt]{2.409pt}{0.400pt}}
\put(1429.0,271.0){\rule[-0.200pt]{2.409pt}{0.400pt}}
\put(170.0,271.0){\rule[-0.200pt]{2.409pt}{0.400pt}}
\put(1429.0,271.0){\rule[-0.200pt]{2.409pt}{0.400pt}}
\put(170.0,271.0){\rule[-0.200pt]{2.409pt}{0.400pt}}
\put(1429.0,271.0){\rule[-0.200pt]{2.409pt}{0.400pt}}
\put(170.0,272.0){\rule[-0.200pt]{2.409pt}{0.400pt}}
\put(1429.0,272.0){\rule[-0.200pt]{2.409pt}{0.400pt}}
\put(170.0,272.0){\rule[-0.200pt]{2.409pt}{0.400pt}}
\put(1429.0,272.0){\rule[-0.200pt]{2.409pt}{0.400pt}}
\put(170.0,272.0){\rule[-0.200pt]{2.409pt}{0.400pt}}
\put(1429.0,272.0){\rule[-0.200pt]{2.409pt}{0.400pt}}
\put(170.0,272.0){\rule[-0.200pt]{2.409pt}{0.400pt}}
\put(1429.0,272.0){\rule[-0.200pt]{2.409pt}{0.400pt}}
\put(170.0,273.0){\rule[-0.200pt]{2.409pt}{0.400pt}}
\put(1429.0,273.0){\rule[-0.200pt]{2.409pt}{0.400pt}}
\put(170.0,273.0){\rule[-0.200pt]{2.409pt}{0.400pt}}
\put(1429.0,273.0){\rule[-0.200pt]{2.409pt}{0.400pt}}
\put(170.0,273.0){\rule[-0.200pt]{2.409pt}{0.400pt}}
\put(1429.0,273.0){\rule[-0.200pt]{2.409pt}{0.400pt}}
\put(170.0,273.0){\rule[-0.200pt]{2.409pt}{0.400pt}}
\put(1429.0,273.0){\rule[-0.200pt]{2.409pt}{0.400pt}}
\put(170.0,273.0){\rule[-0.200pt]{2.409pt}{0.400pt}}
\put(1429.0,273.0){\rule[-0.200pt]{2.409pt}{0.400pt}}
\put(170.0,274.0){\rule[-0.200pt]{2.409pt}{0.400pt}}
\put(1429.0,274.0){\rule[-0.200pt]{2.409pt}{0.400pt}}
\put(170.0,274.0){\rule[-0.200pt]{2.409pt}{0.400pt}}
\put(1429.0,274.0){\rule[-0.200pt]{2.409pt}{0.400pt}}
\put(170.0,274.0){\rule[-0.200pt]{2.409pt}{0.400pt}}
\put(1429.0,274.0){\rule[-0.200pt]{2.409pt}{0.400pt}}
\put(170.0,274.0){\rule[-0.200pt]{2.409pt}{0.400pt}}
\put(1429.0,274.0){\rule[-0.200pt]{2.409pt}{0.400pt}}
\put(170.0,275.0){\rule[-0.200pt]{2.409pt}{0.400pt}}
\put(1429.0,275.0){\rule[-0.200pt]{2.409pt}{0.400pt}}
\put(170.0,275.0){\rule[-0.200pt]{2.409pt}{0.400pt}}
\put(1429.0,275.0){\rule[-0.200pt]{2.409pt}{0.400pt}}
\put(170.0,275.0){\rule[-0.200pt]{2.409pt}{0.400pt}}
\put(1429.0,275.0){\rule[-0.200pt]{2.409pt}{0.400pt}}
\put(170.0,275.0){\rule[-0.200pt]{2.409pt}{0.400pt}}
\put(1429.0,275.0){\rule[-0.200pt]{2.409pt}{0.400pt}}
\put(170.0,275.0){\rule[-0.200pt]{2.409pt}{0.400pt}}
\put(1429.0,275.0){\rule[-0.200pt]{2.409pt}{0.400pt}}
\put(170.0,276.0){\rule[-0.200pt]{2.409pt}{0.400pt}}
\put(1429.0,276.0){\rule[-0.200pt]{2.409pt}{0.400pt}}
\put(170.0,276.0){\rule[-0.200pt]{2.409pt}{0.400pt}}
\put(1429.0,276.0){\rule[-0.200pt]{2.409pt}{0.400pt}}
\put(170.0,276.0){\rule[-0.200pt]{2.409pt}{0.400pt}}
\put(1429.0,276.0){\rule[-0.200pt]{2.409pt}{0.400pt}}
\put(170.0,276.0){\rule[-0.200pt]{2.409pt}{0.400pt}}
\put(1429.0,276.0){\rule[-0.200pt]{2.409pt}{0.400pt}}
\put(170.0,276.0){\rule[-0.200pt]{2.409pt}{0.400pt}}
\put(1429.0,276.0){\rule[-0.200pt]{2.409pt}{0.400pt}}
\put(170.0,277.0){\rule[-0.200pt]{2.409pt}{0.400pt}}
\put(1429.0,277.0){\rule[-0.200pt]{2.409pt}{0.400pt}}
\put(170.0,277.0){\rule[-0.200pt]{2.409pt}{0.400pt}}
\put(1429.0,277.0){\rule[-0.200pt]{2.409pt}{0.400pt}}
\put(170.0,277.0){\rule[-0.200pt]{2.409pt}{0.400pt}}
\put(1429.0,277.0){\rule[-0.200pt]{2.409pt}{0.400pt}}
\put(170.0,277.0){\rule[-0.200pt]{2.409pt}{0.400pt}}
\put(1429.0,277.0){\rule[-0.200pt]{2.409pt}{0.400pt}}
\put(170.0,278.0){\rule[-0.200pt]{2.409pt}{0.400pt}}
\put(1429.0,278.0){\rule[-0.200pt]{2.409pt}{0.400pt}}
\put(170.0,278.0){\rule[-0.200pt]{2.409pt}{0.400pt}}
\put(1429.0,278.0){\rule[-0.200pt]{2.409pt}{0.400pt}}
\put(170.0,278.0){\rule[-0.200pt]{2.409pt}{0.400pt}}
\put(1429.0,278.0){\rule[-0.200pt]{2.409pt}{0.400pt}}
\put(170.0,278.0){\rule[-0.200pt]{2.409pt}{0.400pt}}
\put(1429.0,278.0){\rule[-0.200pt]{2.409pt}{0.400pt}}
\put(170.0,278.0){\rule[-0.200pt]{2.409pt}{0.400pt}}
\put(1429.0,278.0){\rule[-0.200pt]{2.409pt}{0.400pt}}
\put(170.0,279.0){\rule[-0.200pt]{2.409pt}{0.400pt}}
\put(1429.0,279.0){\rule[-0.200pt]{2.409pt}{0.400pt}}
\put(170.0,279.0){\rule[-0.200pt]{2.409pt}{0.400pt}}
\put(1429.0,279.0){\rule[-0.200pt]{2.409pt}{0.400pt}}
\put(170.0,279.0){\rule[-0.200pt]{2.409pt}{0.400pt}}
\put(1429.0,279.0){\rule[-0.200pt]{2.409pt}{0.400pt}}
\put(170.0,279.0){\rule[-0.200pt]{2.409pt}{0.400pt}}
\put(1429.0,279.0){\rule[-0.200pt]{2.409pt}{0.400pt}}
\put(170.0,279.0){\rule[-0.200pt]{2.409pt}{0.400pt}}
\put(1429.0,279.0){\rule[-0.200pt]{2.409pt}{0.400pt}}
\put(170.0,280.0){\rule[-0.200pt]{2.409pt}{0.400pt}}
\put(1429.0,280.0){\rule[-0.200pt]{2.409pt}{0.400pt}}
\put(170.0,280.0){\rule[-0.200pt]{2.409pt}{0.400pt}}
\put(1429.0,280.0){\rule[-0.200pt]{2.409pt}{0.400pt}}
\put(170.0,280.0){\rule[-0.200pt]{2.409pt}{0.400pt}}
\put(1429.0,280.0){\rule[-0.200pt]{2.409pt}{0.400pt}}
\put(170.0,280.0){\rule[-0.200pt]{2.409pt}{0.400pt}}
\put(1429.0,280.0){\rule[-0.200pt]{2.409pt}{0.400pt}}
\put(170.0,280.0){\rule[-0.200pt]{2.409pt}{0.400pt}}
\put(1429.0,280.0){\rule[-0.200pt]{2.409pt}{0.400pt}}
\put(170.0,280.0){\rule[-0.200pt]{2.409pt}{0.400pt}}
\put(1429.0,280.0){\rule[-0.200pt]{2.409pt}{0.400pt}}
\put(170.0,281.0){\rule[-0.200pt]{2.409pt}{0.400pt}}
\put(1429.0,281.0){\rule[-0.200pt]{2.409pt}{0.400pt}}
\put(170.0,281.0){\rule[-0.200pt]{2.409pt}{0.400pt}}
\put(1429.0,281.0){\rule[-0.200pt]{2.409pt}{0.400pt}}
\put(170.0,281.0){\rule[-0.200pt]{2.409pt}{0.400pt}}
\put(1429.0,281.0){\rule[-0.200pt]{2.409pt}{0.400pt}}
\put(170.0,281.0){\rule[-0.200pt]{2.409pt}{0.400pt}}
\put(1429.0,281.0){\rule[-0.200pt]{2.409pt}{0.400pt}}
\put(170.0,281.0){\rule[-0.200pt]{2.409pt}{0.400pt}}
\put(1429.0,281.0){\rule[-0.200pt]{2.409pt}{0.400pt}}
\put(170.0,282.0){\rule[-0.200pt]{2.409pt}{0.400pt}}
\put(1429.0,282.0){\rule[-0.200pt]{2.409pt}{0.400pt}}
\put(170.0,282.0){\rule[-0.200pt]{2.409pt}{0.400pt}}
\put(1429.0,282.0){\rule[-0.200pt]{2.409pt}{0.400pt}}
\put(170.0,282.0){\rule[-0.200pt]{2.409pt}{0.400pt}}
\put(1429.0,282.0){\rule[-0.200pt]{2.409pt}{0.400pt}}
\put(170.0,282.0){\rule[-0.200pt]{2.409pt}{0.400pt}}
\put(1429.0,282.0){\rule[-0.200pt]{2.409pt}{0.400pt}}
\put(170.0,282.0){\rule[-0.200pt]{2.409pt}{0.400pt}}
\put(1429.0,282.0){\rule[-0.200pt]{2.409pt}{0.400pt}}
\put(170.0,282.0){\rule[-0.200pt]{2.409pt}{0.400pt}}
\put(1429.0,282.0){\rule[-0.200pt]{2.409pt}{0.400pt}}
\put(170.0,283.0){\rule[-0.200pt]{2.409pt}{0.400pt}}
\put(1429.0,283.0){\rule[-0.200pt]{2.409pt}{0.400pt}}
\put(170.0,283.0){\rule[-0.200pt]{2.409pt}{0.400pt}}
\put(1429.0,283.0){\rule[-0.200pt]{2.409pt}{0.400pt}}
\put(170.0,283.0){\rule[-0.200pt]{2.409pt}{0.400pt}}
\put(1429.0,283.0){\rule[-0.200pt]{2.409pt}{0.400pt}}
\put(170.0,283.0){\rule[-0.200pt]{2.409pt}{0.400pt}}
\put(1429.0,283.0){\rule[-0.200pt]{2.409pt}{0.400pt}}
\put(170.0,283.0){\rule[-0.200pt]{2.409pt}{0.400pt}}
\put(1429.0,283.0){\rule[-0.200pt]{2.409pt}{0.400pt}}
\put(170.0,284.0){\rule[-0.200pt]{2.409pt}{0.400pt}}
\put(1429.0,284.0){\rule[-0.200pt]{2.409pt}{0.400pt}}
\put(170.0,284.0){\rule[-0.200pt]{2.409pt}{0.400pt}}
\put(1429.0,284.0){\rule[-0.200pt]{2.409pt}{0.400pt}}
\put(170.0,284.0){\rule[-0.200pt]{2.409pt}{0.400pt}}
\put(1429.0,284.0){\rule[-0.200pt]{2.409pt}{0.400pt}}
\put(170.0,284.0){\rule[-0.200pt]{2.409pt}{0.400pt}}
\put(1429.0,284.0){\rule[-0.200pt]{2.409pt}{0.400pt}}
\put(170.0,284.0){\rule[-0.200pt]{2.409pt}{0.400pt}}
\put(1429.0,284.0){\rule[-0.200pt]{2.409pt}{0.400pt}}
\put(170.0,284.0){\rule[-0.200pt]{2.409pt}{0.400pt}}
\put(1429.0,284.0){\rule[-0.200pt]{2.409pt}{0.400pt}}
\put(170.0,285.0){\rule[-0.200pt]{2.409pt}{0.400pt}}
\put(1429.0,285.0){\rule[-0.200pt]{2.409pt}{0.400pt}}
\put(170.0,285.0){\rule[-0.200pt]{2.409pt}{0.400pt}}
\put(1429.0,285.0){\rule[-0.200pt]{2.409pt}{0.400pt}}
\put(170.0,285.0){\rule[-0.200pt]{2.409pt}{0.400pt}}
\put(1429.0,285.0){\rule[-0.200pt]{2.409pt}{0.400pt}}
\put(170.0,285.0){\rule[-0.200pt]{2.409pt}{0.400pt}}
\put(1429.0,285.0){\rule[-0.200pt]{2.409pt}{0.400pt}}
\put(170.0,285.0){\rule[-0.200pt]{2.409pt}{0.400pt}}
\put(1429.0,285.0){\rule[-0.200pt]{2.409pt}{0.400pt}}
\put(170.0,285.0){\rule[-0.200pt]{2.409pt}{0.400pt}}
\put(1429.0,285.0){\rule[-0.200pt]{2.409pt}{0.400pt}}
\put(170.0,286.0){\rule[-0.200pt]{2.409pt}{0.400pt}}
\put(1429.0,286.0){\rule[-0.200pt]{2.409pt}{0.400pt}}
\put(170.0,286.0){\rule[-0.200pt]{2.409pt}{0.400pt}}
\put(1429.0,286.0){\rule[-0.200pt]{2.409pt}{0.400pt}}
\put(170.0,286.0){\rule[-0.200pt]{2.409pt}{0.400pt}}
\put(1429.0,286.0){\rule[-0.200pt]{2.409pt}{0.400pt}}
\put(170.0,286.0){\rule[-0.200pt]{2.409pt}{0.400pt}}
\put(1429.0,286.0){\rule[-0.200pt]{2.409pt}{0.400pt}}
\put(170.0,286.0){\rule[-0.200pt]{2.409pt}{0.400pt}}
\put(1429.0,286.0){\rule[-0.200pt]{2.409pt}{0.400pt}}
\put(170.0,286.0){\rule[-0.200pt]{2.409pt}{0.400pt}}
\put(1429.0,286.0){\rule[-0.200pt]{2.409pt}{0.400pt}}
\put(170.0,287.0){\rule[-0.200pt]{2.409pt}{0.400pt}}
\put(1429.0,287.0){\rule[-0.200pt]{2.409pt}{0.400pt}}
\put(170.0,287.0){\rule[-0.200pt]{2.409pt}{0.400pt}}
\put(1429.0,287.0){\rule[-0.200pt]{2.409pt}{0.400pt}}
\put(170.0,287.0){\rule[-0.200pt]{2.409pt}{0.400pt}}
\put(1429.0,287.0){\rule[-0.200pt]{2.409pt}{0.400pt}}
\put(170.0,287.0){\rule[-0.200pt]{2.409pt}{0.400pt}}
\put(1429.0,287.0){\rule[-0.200pt]{2.409pt}{0.400pt}}
\put(170.0,287.0){\rule[-0.200pt]{2.409pt}{0.400pt}}
\put(1429.0,287.0){\rule[-0.200pt]{2.409pt}{0.400pt}}
\put(170.0,287.0){\rule[-0.200pt]{2.409pt}{0.400pt}}
\put(1429.0,287.0){\rule[-0.200pt]{2.409pt}{0.400pt}}
\put(170.0,287.0){\rule[-0.200pt]{2.409pt}{0.400pt}}
\put(1429.0,287.0){\rule[-0.200pt]{2.409pt}{0.400pt}}
\put(170.0,288.0){\rule[-0.200pt]{2.409pt}{0.400pt}}
\put(1429.0,288.0){\rule[-0.200pt]{2.409pt}{0.400pt}}
\put(170.0,288.0){\rule[-0.200pt]{2.409pt}{0.400pt}}
\put(1429.0,288.0){\rule[-0.200pt]{2.409pt}{0.400pt}}
\put(170.0,288.0){\rule[-0.200pt]{2.409pt}{0.400pt}}
\put(1429.0,288.0){\rule[-0.200pt]{2.409pt}{0.400pt}}
\put(170.0,288.0){\rule[-0.200pt]{2.409pt}{0.400pt}}
\put(1429.0,288.0){\rule[-0.200pt]{2.409pt}{0.400pt}}
\put(170.0,288.0){\rule[-0.200pt]{2.409pt}{0.400pt}}
\put(1429.0,288.0){\rule[-0.200pt]{2.409pt}{0.400pt}}
\put(170.0,288.0){\rule[-0.200pt]{2.409pt}{0.400pt}}
\put(1429.0,288.0){\rule[-0.200pt]{2.409pt}{0.400pt}}
\put(170.0,289.0){\rule[-0.200pt]{2.409pt}{0.400pt}}
\put(1429.0,289.0){\rule[-0.200pt]{2.409pt}{0.400pt}}
\put(170.0,289.0){\rule[-0.200pt]{2.409pt}{0.400pt}}
\put(1429.0,289.0){\rule[-0.200pt]{2.409pt}{0.400pt}}
\put(170.0,289.0){\rule[-0.200pt]{2.409pt}{0.400pt}}
\put(1429.0,289.0){\rule[-0.200pt]{2.409pt}{0.400pt}}
\put(170.0,289.0){\rule[-0.200pt]{2.409pt}{0.400pt}}
\put(1429.0,289.0){\rule[-0.200pt]{2.409pt}{0.400pt}}
\put(170.0,289.0){\rule[-0.200pt]{2.409pt}{0.400pt}}
\put(1429.0,289.0){\rule[-0.200pt]{2.409pt}{0.400pt}}
\put(170.0,289.0){\rule[-0.200pt]{2.409pt}{0.400pt}}
\put(1429.0,289.0){\rule[-0.200pt]{2.409pt}{0.400pt}}
\put(170.0,289.0){\rule[-0.200pt]{2.409pt}{0.400pt}}
\put(1429.0,289.0){\rule[-0.200pt]{2.409pt}{0.400pt}}
\put(170.0,290.0){\rule[-0.200pt]{2.409pt}{0.400pt}}
\put(1429.0,290.0){\rule[-0.200pt]{2.409pt}{0.400pt}}
\put(170.0,290.0){\rule[-0.200pt]{2.409pt}{0.400pt}}
\put(1429.0,290.0){\rule[-0.200pt]{2.409pt}{0.400pt}}
\put(170.0,290.0){\rule[-0.200pt]{2.409pt}{0.400pt}}
\put(1429.0,290.0){\rule[-0.200pt]{2.409pt}{0.400pt}}
\put(170.0,290.0){\rule[-0.200pt]{2.409pt}{0.400pt}}
\put(1429.0,290.0){\rule[-0.200pt]{2.409pt}{0.400pt}}
\put(170.0,290.0){\rule[-0.200pt]{2.409pt}{0.400pt}}
\put(1429.0,290.0){\rule[-0.200pt]{2.409pt}{0.400pt}}
\put(170.0,290.0){\rule[-0.200pt]{2.409pt}{0.400pt}}
\put(1429.0,290.0){\rule[-0.200pt]{2.409pt}{0.400pt}}
\put(170.0,290.0){\rule[-0.200pt]{2.409pt}{0.400pt}}
\put(1429.0,290.0){\rule[-0.200pt]{2.409pt}{0.400pt}}
\put(170.0,291.0){\rule[-0.200pt]{2.409pt}{0.400pt}}
\put(1429.0,291.0){\rule[-0.200pt]{2.409pt}{0.400pt}}
\put(170.0,291.0){\rule[-0.200pt]{2.409pt}{0.400pt}}
\put(1429.0,291.0){\rule[-0.200pt]{2.409pt}{0.400pt}}
\put(170.0,291.0){\rule[-0.200pt]{2.409pt}{0.400pt}}
\put(1429.0,291.0){\rule[-0.200pt]{2.409pt}{0.400pt}}
\put(170.0,291.0){\rule[-0.200pt]{2.409pt}{0.400pt}}
\put(1429.0,291.0){\rule[-0.200pt]{2.409pt}{0.400pt}}
\put(170.0,291.0){\rule[-0.200pt]{2.409pt}{0.400pt}}
\put(1429.0,291.0){\rule[-0.200pt]{2.409pt}{0.400pt}}
\put(170.0,291.0){\rule[-0.200pt]{2.409pt}{0.400pt}}
\put(1429.0,291.0){\rule[-0.200pt]{2.409pt}{0.400pt}}
\put(170.0,291.0){\rule[-0.200pt]{2.409pt}{0.400pt}}
\put(1429.0,291.0){\rule[-0.200pt]{2.409pt}{0.400pt}}
\put(170.0,292.0){\rule[-0.200pt]{2.409pt}{0.400pt}}
\put(1429.0,292.0){\rule[-0.200pt]{2.409pt}{0.400pt}}
\put(170.0,292.0){\rule[-0.200pt]{2.409pt}{0.400pt}}
\put(1429.0,292.0){\rule[-0.200pt]{2.409pt}{0.400pt}}
\put(170.0,292.0){\rule[-0.200pt]{2.409pt}{0.400pt}}
\put(1429.0,292.0){\rule[-0.200pt]{2.409pt}{0.400pt}}
\put(170.0,292.0){\rule[-0.200pt]{2.409pt}{0.400pt}}
\put(1429.0,292.0){\rule[-0.200pt]{2.409pt}{0.400pt}}
\put(170.0,292.0){\rule[-0.200pt]{2.409pt}{0.400pt}}
\put(1429.0,292.0){\rule[-0.200pt]{2.409pt}{0.400pt}}
\put(170.0,292.0){\rule[-0.200pt]{2.409pt}{0.400pt}}
\put(1429.0,292.0){\rule[-0.200pt]{2.409pt}{0.400pt}}
\put(170.0,292.0){\rule[-0.200pt]{2.409pt}{0.400pt}}
\put(1429.0,292.0){\rule[-0.200pt]{2.409pt}{0.400pt}}
\put(170.0,293.0){\rule[-0.200pt]{2.409pt}{0.400pt}}
\put(1429.0,293.0){\rule[-0.200pt]{2.409pt}{0.400pt}}
\put(170.0,293.0){\rule[-0.200pt]{2.409pt}{0.400pt}}
\put(1429.0,293.0){\rule[-0.200pt]{2.409pt}{0.400pt}}
\put(170.0,293.0){\rule[-0.200pt]{2.409pt}{0.400pt}}
\put(1429.0,293.0){\rule[-0.200pt]{2.409pt}{0.400pt}}
\put(170.0,293.0){\rule[-0.200pt]{2.409pt}{0.400pt}}
\put(1429.0,293.0){\rule[-0.200pt]{2.409pt}{0.400pt}}
\put(170.0,293.0){\rule[-0.200pt]{2.409pt}{0.400pt}}
\put(1429.0,293.0){\rule[-0.200pt]{2.409pt}{0.400pt}}
\put(170.0,293.0){\rule[-0.200pt]{2.409pt}{0.400pt}}
\put(1429.0,293.0){\rule[-0.200pt]{2.409pt}{0.400pt}}
\put(170.0,293.0){\rule[-0.200pt]{2.409pt}{0.400pt}}
\put(1429.0,293.0){\rule[-0.200pt]{2.409pt}{0.400pt}}
\put(170.0,294.0){\rule[-0.200pt]{2.409pt}{0.400pt}}
\put(1429.0,294.0){\rule[-0.200pt]{2.409pt}{0.400pt}}
\put(170.0,294.0){\rule[-0.200pt]{2.409pt}{0.400pt}}
\put(1429.0,294.0){\rule[-0.200pt]{2.409pt}{0.400pt}}
\put(170.0,294.0){\rule[-0.200pt]{2.409pt}{0.400pt}}
\put(1429.0,294.0){\rule[-0.200pt]{2.409pt}{0.400pt}}
\put(170.0,294.0){\rule[-0.200pt]{2.409pt}{0.400pt}}
\put(1429.0,294.0){\rule[-0.200pt]{2.409pt}{0.400pt}}
\put(170.0,294.0){\rule[-0.200pt]{2.409pt}{0.400pt}}
\put(1429.0,294.0){\rule[-0.200pt]{2.409pt}{0.400pt}}
\put(170.0,294.0){\rule[-0.200pt]{2.409pt}{0.400pt}}
\put(1429.0,294.0){\rule[-0.200pt]{2.409pt}{0.400pt}}
\put(170.0,294.0){\rule[-0.200pt]{2.409pt}{0.400pt}}
\put(1429.0,294.0){\rule[-0.200pt]{2.409pt}{0.400pt}}
\put(170.0,294.0){\rule[-0.200pt]{2.409pt}{0.400pt}}
\put(1429.0,294.0){\rule[-0.200pt]{2.409pt}{0.400pt}}
\put(170.0,295.0){\rule[-0.200pt]{2.409pt}{0.400pt}}
\put(1429.0,295.0){\rule[-0.200pt]{2.409pt}{0.400pt}}
\put(170.0,295.0){\rule[-0.200pt]{2.409pt}{0.400pt}}
\put(1429.0,295.0){\rule[-0.200pt]{2.409pt}{0.400pt}}
\put(170.0,295.0){\rule[-0.200pt]{2.409pt}{0.400pt}}
\put(1429.0,295.0){\rule[-0.200pt]{2.409pt}{0.400pt}}
\put(170.0,295.0){\rule[-0.200pt]{2.409pt}{0.400pt}}
\put(1429.0,295.0){\rule[-0.200pt]{2.409pt}{0.400pt}}
\put(170.0,295.0){\rule[-0.200pt]{2.409pt}{0.400pt}}
\put(1429.0,295.0){\rule[-0.200pt]{2.409pt}{0.400pt}}
\put(170.0,295.0){\rule[-0.200pt]{2.409pt}{0.400pt}}
\put(1429.0,295.0){\rule[-0.200pt]{2.409pt}{0.400pt}}
\put(170.0,295.0){\rule[-0.200pt]{2.409pt}{0.400pt}}
\put(1429.0,295.0){\rule[-0.200pt]{2.409pt}{0.400pt}}
\put(170.0,295.0){\rule[-0.200pt]{2.409pt}{0.400pt}}
\put(1429.0,295.0){\rule[-0.200pt]{2.409pt}{0.400pt}}
\put(170.0,296.0){\rule[-0.200pt]{2.409pt}{0.400pt}}
\put(1429.0,296.0){\rule[-0.200pt]{2.409pt}{0.400pt}}
\put(170.0,296.0){\rule[-0.200pt]{2.409pt}{0.400pt}}
\put(1429.0,296.0){\rule[-0.200pt]{2.409pt}{0.400pt}}
\put(170.0,296.0){\rule[-0.200pt]{2.409pt}{0.400pt}}
\put(1429.0,296.0){\rule[-0.200pt]{2.409pt}{0.400pt}}
\put(170.0,296.0){\rule[-0.200pt]{2.409pt}{0.400pt}}
\put(1429.0,296.0){\rule[-0.200pt]{2.409pt}{0.400pt}}
\put(170.0,296.0){\rule[-0.200pt]{2.409pt}{0.400pt}}
\put(1429.0,296.0){\rule[-0.200pt]{2.409pt}{0.400pt}}
\put(170.0,296.0){\rule[-0.200pt]{2.409pt}{0.400pt}}
\put(1429.0,296.0){\rule[-0.200pt]{2.409pt}{0.400pt}}
\put(170.0,296.0){\rule[-0.200pt]{2.409pt}{0.400pt}}
\put(1429.0,296.0){\rule[-0.200pt]{2.409pt}{0.400pt}}
\put(170.0,296.0){\rule[-0.200pt]{2.409pt}{0.400pt}}
\put(1429.0,296.0){\rule[-0.200pt]{2.409pt}{0.400pt}}
\put(170.0,297.0){\rule[-0.200pt]{2.409pt}{0.400pt}}
\put(1429.0,297.0){\rule[-0.200pt]{2.409pt}{0.400pt}}
\put(170.0,297.0){\rule[-0.200pt]{2.409pt}{0.400pt}}
\put(1429.0,297.0){\rule[-0.200pt]{2.409pt}{0.400pt}}
\put(170.0,297.0){\rule[-0.200pt]{2.409pt}{0.400pt}}
\put(1429.0,297.0){\rule[-0.200pt]{2.409pt}{0.400pt}}
\put(170.0,297.0){\rule[-0.200pt]{2.409pt}{0.400pt}}
\put(1429.0,297.0){\rule[-0.200pt]{2.409pt}{0.400pt}}
\put(170.0,297.0){\rule[-0.200pt]{2.409pt}{0.400pt}}
\put(1429.0,297.0){\rule[-0.200pt]{2.409pt}{0.400pt}}
\put(170.0,297.0){\rule[-0.200pt]{2.409pt}{0.400pt}}
\put(1429.0,297.0){\rule[-0.200pt]{2.409pt}{0.400pt}}
\put(170.0,297.0){\rule[-0.200pt]{2.409pt}{0.400pt}}
\put(1429.0,297.0){\rule[-0.200pt]{2.409pt}{0.400pt}}
\put(170.0,297.0){\rule[-0.200pt]{2.409pt}{0.400pt}}
\put(1429.0,297.0){\rule[-0.200pt]{2.409pt}{0.400pt}}
\put(170.0,298.0){\rule[-0.200pt]{2.409pt}{0.400pt}}
\put(1429.0,298.0){\rule[-0.200pt]{2.409pt}{0.400pt}}
\put(170.0,298.0){\rule[-0.200pt]{2.409pt}{0.400pt}}
\put(1429.0,298.0){\rule[-0.200pt]{2.409pt}{0.400pt}}
\put(170.0,298.0){\rule[-0.200pt]{2.409pt}{0.400pt}}
\put(1429.0,298.0){\rule[-0.200pt]{2.409pt}{0.400pt}}
\put(170.0,298.0){\rule[-0.200pt]{2.409pt}{0.400pt}}
\put(1429.0,298.0){\rule[-0.200pt]{2.409pt}{0.400pt}}
\put(170.0,298.0){\rule[-0.200pt]{2.409pt}{0.400pt}}
\put(1429.0,298.0){\rule[-0.200pt]{2.409pt}{0.400pt}}
\put(170.0,298.0){\rule[-0.200pt]{2.409pt}{0.400pt}}
\put(1429.0,298.0){\rule[-0.200pt]{2.409pt}{0.400pt}}
\put(170.0,298.0){\rule[-0.200pt]{2.409pt}{0.400pt}}
\put(1429.0,298.0){\rule[-0.200pt]{2.409pt}{0.400pt}}
\put(170.0,298.0){\rule[-0.200pt]{2.409pt}{0.400pt}}
\put(1429.0,298.0){\rule[-0.200pt]{2.409pt}{0.400pt}}
\put(170.0,299.0){\rule[-0.200pt]{2.409pt}{0.400pt}}
\put(1429.0,299.0){\rule[-0.200pt]{2.409pt}{0.400pt}}
\put(170.0,299.0){\rule[-0.200pt]{2.409pt}{0.400pt}}
\put(1429.0,299.0){\rule[-0.200pt]{2.409pt}{0.400pt}}
\put(170.0,299.0){\rule[-0.200pt]{2.409pt}{0.400pt}}
\put(1429.0,299.0){\rule[-0.200pt]{2.409pt}{0.400pt}}
\put(170.0,299.0){\rule[-0.200pt]{2.409pt}{0.400pt}}
\put(1429.0,299.0){\rule[-0.200pt]{2.409pt}{0.400pt}}
\put(170.0,299.0){\rule[-0.200pt]{2.409pt}{0.400pt}}
\put(1429.0,299.0){\rule[-0.200pt]{2.409pt}{0.400pt}}
\put(170.0,299.0){\rule[-0.200pt]{2.409pt}{0.400pt}}
\put(1429.0,299.0){\rule[-0.200pt]{2.409pt}{0.400pt}}
\put(170.0,299.0){\rule[-0.200pt]{2.409pt}{0.400pt}}
\put(1429.0,299.0){\rule[-0.200pt]{2.409pt}{0.400pt}}
\put(170.0,299.0){\rule[-0.200pt]{2.409pt}{0.400pt}}
\put(1429.0,299.0){\rule[-0.200pt]{2.409pt}{0.400pt}}
\put(170.0,299.0){\rule[-0.200pt]{2.409pt}{0.400pt}}
\put(1429.0,299.0){\rule[-0.200pt]{2.409pt}{0.400pt}}
\put(170.0,300.0){\rule[-0.200pt]{2.409pt}{0.400pt}}
\put(1429.0,300.0){\rule[-0.200pt]{2.409pt}{0.400pt}}
\put(170.0,300.0){\rule[-0.200pt]{2.409pt}{0.400pt}}
\put(1429.0,300.0){\rule[-0.200pt]{2.409pt}{0.400pt}}
\put(170.0,300.0){\rule[-0.200pt]{2.409pt}{0.400pt}}
\put(1429.0,300.0){\rule[-0.200pt]{2.409pt}{0.400pt}}
\put(170.0,300.0){\rule[-0.200pt]{2.409pt}{0.400pt}}
\put(1429.0,300.0){\rule[-0.200pt]{2.409pt}{0.400pt}}
\put(170.0,300.0){\rule[-0.200pt]{2.409pt}{0.400pt}}
\put(1429.0,300.0){\rule[-0.200pt]{2.409pt}{0.400pt}}
\put(170.0,300.0){\rule[-0.200pt]{2.409pt}{0.400pt}}
\put(1429.0,300.0){\rule[-0.200pt]{2.409pt}{0.400pt}}
\put(170.0,300.0){\rule[-0.200pt]{2.409pt}{0.400pt}}
\put(1429.0,300.0){\rule[-0.200pt]{2.409pt}{0.400pt}}
\put(170.0,300.0){\rule[-0.200pt]{2.409pt}{0.400pt}}
\put(1429.0,300.0){\rule[-0.200pt]{2.409pt}{0.400pt}}
\put(170.0,300.0){\rule[-0.200pt]{2.409pt}{0.400pt}}
\put(1429.0,300.0){\rule[-0.200pt]{2.409pt}{0.400pt}}
\put(170.0,301.0){\rule[-0.200pt]{2.409pt}{0.400pt}}
\put(1429.0,301.0){\rule[-0.200pt]{2.409pt}{0.400pt}}
\put(170.0,301.0){\rule[-0.200pt]{2.409pt}{0.400pt}}
\put(1429.0,301.0){\rule[-0.200pt]{2.409pt}{0.400pt}}
\put(170.0,301.0){\rule[-0.200pt]{2.409pt}{0.400pt}}
\put(1429.0,301.0){\rule[-0.200pt]{2.409pt}{0.400pt}}
\put(170.0,301.0){\rule[-0.200pt]{2.409pt}{0.400pt}}
\put(1429.0,301.0){\rule[-0.200pt]{2.409pt}{0.400pt}}
\put(170.0,301.0){\rule[-0.200pt]{2.409pt}{0.400pt}}
\put(1429.0,301.0){\rule[-0.200pt]{2.409pt}{0.400pt}}
\put(170.0,301.0){\rule[-0.200pt]{2.409pt}{0.400pt}}
\put(1429.0,301.0){\rule[-0.200pt]{2.409pt}{0.400pt}}
\put(170.0,301.0){\rule[-0.200pt]{2.409pt}{0.400pt}}
\put(1429.0,301.0){\rule[-0.200pt]{2.409pt}{0.400pt}}
\put(170.0,301.0){\rule[-0.200pt]{2.409pt}{0.400pt}}
\put(1429.0,301.0){\rule[-0.200pt]{2.409pt}{0.400pt}}
\put(170.0,301.0){\rule[-0.200pt]{2.409pt}{0.400pt}}
\put(1429.0,301.0){\rule[-0.200pt]{2.409pt}{0.400pt}}
\put(170.0,302.0){\rule[-0.200pt]{2.409pt}{0.400pt}}
\put(1429.0,302.0){\rule[-0.200pt]{2.409pt}{0.400pt}}
\put(170.0,302.0){\rule[-0.200pt]{2.409pt}{0.400pt}}
\put(1429.0,302.0){\rule[-0.200pt]{2.409pt}{0.400pt}}
\put(170.0,302.0){\rule[-0.200pt]{2.409pt}{0.400pt}}
\put(1429.0,302.0){\rule[-0.200pt]{2.409pt}{0.400pt}}
\put(170.0,302.0){\rule[-0.200pt]{2.409pt}{0.400pt}}
\put(1429.0,302.0){\rule[-0.200pt]{2.409pt}{0.400pt}}
\put(170.0,302.0){\rule[-0.200pt]{2.409pt}{0.400pt}}
\put(1429.0,302.0){\rule[-0.200pt]{2.409pt}{0.400pt}}
\put(170.0,302.0){\rule[-0.200pt]{2.409pt}{0.400pt}}
\put(1429.0,302.0){\rule[-0.200pt]{2.409pt}{0.400pt}}
\put(170.0,302.0){\rule[-0.200pt]{2.409pt}{0.400pt}}
\put(1429.0,302.0){\rule[-0.200pt]{2.409pt}{0.400pt}}
\put(170.0,302.0){\rule[-0.200pt]{2.409pt}{0.400pt}}
\put(1429.0,302.0){\rule[-0.200pt]{2.409pt}{0.400pt}}
\put(170.0,302.0){\rule[-0.200pt]{2.409pt}{0.400pt}}
\put(1429.0,302.0){\rule[-0.200pt]{2.409pt}{0.400pt}}
\put(170.0,302.0){\rule[-0.200pt]{2.409pt}{0.400pt}}
\put(1429.0,302.0){\rule[-0.200pt]{2.409pt}{0.400pt}}
\put(170.0,303.0){\rule[-0.200pt]{2.409pt}{0.400pt}}
\put(1429.0,303.0){\rule[-0.200pt]{2.409pt}{0.400pt}}
\put(170.0,303.0){\rule[-0.200pt]{2.409pt}{0.400pt}}
\put(1429.0,303.0){\rule[-0.200pt]{2.409pt}{0.400pt}}
\put(170.0,303.0){\rule[-0.200pt]{2.409pt}{0.400pt}}
\put(1429.0,303.0){\rule[-0.200pt]{2.409pt}{0.400pt}}
\put(170.0,303.0){\rule[-0.200pt]{2.409pt}{0.400pt}}
\put(1429.0,303.0){\rule[-0.200pt]{2.409pt}{0.400pt}}
\put(170.0,303.0){\rule[-0.200pt]{2.409pt}{0.400pt}}
\put(1429.0,303.0){\rule[-0.200pt]{2.409pt}{0.400pt}}
\put(170.0,303.0){\rule[-0.200pt]{2.409pt}{0.400pt}}
\put(1429.0,303.0){\rule[-0.200pt]{2.409pt}{0.400pt}}
\put(170.0,303.0){\rule[-0.200pt]{2.409pt}{0.400pt}}
\put(1429.0,303.0){\rule[-0.200pt]{2.409pt}{0.400pt}}
\put(170.0,303.0){\rule[-0.200pt]{2.409pt}{0.400pt}}
\put(1429.0,303.0){\rule[-0.200pt]{2.409pt}{0.400pt}}
\put(170.0,303.0){\rule[-0.200pt]{2.409pt}{0.400pt}}
\put(1429.0,303.0){\rule[-0.200pt]{2.409pt}{0.400pt}}
\put(170.0,304.0){\rule[-0.200pt]{2.409pt}{0.400pt}}
\put(1429.0,304.0){\rule[-0.200pt]{2.409pt}{0.400pt}}
\put(170.0,304.0){\rule[-0.200pt]{2.409pt}{0.400pt}}
\put(1429.0,304.0){\rule[-0.200pt]{2.409pt}{0.400pt}}
\put(170.0,304.0){\rule[-0.200pt]{2.409pt}{0.400pt}}
\put(1429.0,304.0){\rule[-0.200pt]{2.409pt}{0.400pt}}
\put(170.0,304.0){\rule[-0.200pt]{2.409pt}{0.400pt}}
\put(1429.0,304.0){\rule[-0.200pt]{2.409pt}{0.400pt}}
\put(170.0,304.0){\rule[-0.200pt]{2.409pt}{0.400pt}}
\put(1429.0,304.0){\rule[-0.200pt]{2.409pt}{0.400pt}}
\put(170.0,304.0){\rule[-0.200pt]{2.409pt}{0.400pt}}
\put(1429.0,304.0){\rule[-0.200pt]{2.409pt}{0.400pt}}
\put(170.0,304.0){\rule[-0.200pt]{2.409pt}{0.400pt}}
\put(1429.0,304.0){\rule[-0.200pt]{2.409pt}{0.400pt}}
\put(170.0,304.0){\rule[-0.200pt]{2.409pt}{0.400pt}}
\put(1429.0,304.0){\rule[-0.200pt]{2.409pt}{0.400pt}}
\put(170.0,304.0){\rule[-0.200pt]{2.409pt}{0.400pt}}
\put(1429.0,304.0){\rule[-0.200pt]{2.409pt}{0.400pt}}
\put(170.0,304.0){\rule[-0.200pt]{2.409pt}{0.400pt}}
\put(1429.0,304.0){\rule[-0.200pt]{2.409pt}{0.400pt}}
\put(170.0,305.0){\rule[-0.200pt]{2.409pt}{0.400pt}}
\put(1429.0,305.0){\rule[-0.200pt]{2.409pt}{0.400pt}}
\put(170.0,305.0){\rule[-0.200pt]{2.409pt}{0.400pt}}
\put(1429.0,305.0){\rule[-0.200pt]{2.409pt}{0.400pt}}
\put(170.0,305.0){\rule[-0.200pt]{2.409pt}{0.400pt}}
\put(1429.0,305.0){\rule[-0.200pt]{2.409pt}{0.400pt}}
\put(170.0,305.0){\rule[-0.200pt]{2.409pt}{0.400pt}}
\put(1429.0,305.0){\rule[-0.200pt]{2.409pt}{0.400pt}}
\put(170.0,305.0){\rule[-0.200pt]{2.409pt}{0.400pt}}
\put(1429.0,305.0){\rule[-0.200pt]{2.409pt}{0.400pt}}
\put(170.0,305.0){\rule[-0.200pt]{2.409pt}{0.400pt}}
\put(1429.0,305.0){\rule[-0.200pt]{2.409pt}{0.400pt}}
\put(170.0,305.0){\rule[-0.200pt]{2.409pt}{0.400pt}}
\put(1429.0,305.0){\rule[-0.200pt]{2.409pt}{0.400pt}}
\put(170.0,305.0){\rule[-0.200pt]{2.409pt}{0.400pt}}
\put(1429.0,305.0){\rule[-0.200pt]{2.409pt}{0.400pt}}
\put(170.0,305.0){\rule[-0.200pt]{2.409pt}{0.400pt}}
\put(1429.0,305.0){\rule[-0.200pt]{2.409pt}{0.400pt}}
\put(170.0,305.0){\rule[-0.200pt]{2.409pt}{0.400pt}}
\put(1429.0,305.0){\rule[-0.200pt]{2.409pt}{0.400pt}}
\put(170.0,306.0){\rule[-0.200pt]{2.409pt}{0.400pt}}
\put(1429.0,306.0){\rule[-0.200pt]{2.409pt}{0.400pt}}
\put(170.0,306.0){\rule[-0.200pt]{2.409pt}{0.400pt}}
\put(1429.0,306.0){\rule[-0.200pt]{2.409pt}{0.400pt}}
\put(170.0,306.0){\rule[-0.200pt]{2.409pt}{0.400pt}}
\put(1429.0,306.0){\rule[-0.200pt]{2.409pt}{0.400pt}}
\put(170.0,306.0){\rule[-0.200pt]{2.409pt}{0.400pt}}
\put(1429.0,306.0){\rule[-0.200pt]{2.409pt}{0.400pt}}
\put(170.0,306.0){\rule[-0.200pt]{2.409pt}{0.400pt}}
\put(1429.0,306.0){\rule[-0.200pt]{2.409pt}{0.400pt}}
\put(170.0,306.0){\rule[-0.200pt]{2.409pt}{0.400pt}}
\put(1429.0,306.0){\rule[-0.200pt]{2.409pt}{0.400pt}}
\put(170.0,306.0){\rule[-0.200pt]{2.409pt}{0.400pt}}
\put(1429.0,306.0){\rule[-0.200pt]{2.409pt}{0.400pt}}
\put(170.0,306.0){\rule[-0.200pt]{2.409pt}{0.400pt}}
\put(1429.0,306.0){\rule[-0.200pt]{2.409pt}{0.400pt}}
\put(170.0,306.0){\rule[-0.200pt]{2.409pt}{0.400pt}}
\put(1429.0,306.0){\rule[-0.200pt]{2.409pt}{0.400pt}}
\put(170.0,306.0){\rule[-0.200pt]{2.409pt}{0.400pt}}
\put(1429.0,306.0){\rule[-0.200pt]{2.409pt}{0.400pt}}
\put(170.0,306.0){\rule[-0.200pt]{2.409pt}{0.400pt}}
\put(1429.0,306.0){\rule[-0.200pt]{2.409pt}{0.400pt}}
\put(170.0,307.0){\rule[-0.200pt]{2.409pt}{0.400pt}}
\put(1429.0,307.0){\rule[-0.200pt]{2.409pt}{0.400pt}}
\put(170.0,307.0){\rule[-0.200pt]{2.409pt}{0.400pt}}
\put(1429.0,307.0){\rule[-0.200pt]{2.409pt}{0.400pt}}
\put(170.0,307.0){\rule[-0.200pt]{2.409pt}{0.400pt}}
\put(1429.0,307.0){\rule[-0.200pt]{2.409pt}{0.400pt}}
\put(170.0,307.0){\rule[-0.200pt]{2.409pt}{0.400pt}}
\put(1429.0,307.0){\rule[-0.200pt]{2.409pt}{0.400pt}}
\put(170.0,307.0){\rule[-0.200pt]{2.409pt}{0.400pt}}
\put(1429.0,307.0){\rule[-0.200pt]{2.409pt}{0.400pt}}
\put(170.0,307.0){\rule[-0.200pt]{2.409pt}{0.400pt}}
\put(1429.0,307.0){\rule[-0.200pt]{2.409pt}{0.400pt}}
\put(170.0,307.0){\rule[-0.200pt]{2.409pt}{0.400pt}}
\put(1429.0,307.0){\rule[-0.200pt]{2.409pt}{0.400pt}}
\put(170.0,307.0){\rule[-0.200pt]{2.409pt}{0.400pt}}
\put(1429.0,307.0){\rule[-0.200pt]{2.409pt}{0.400pt}}
\put(170.0,307.0){\rule[-0.200pt]{2.409pt}{0.400pt}}
\put(1429.0,307.0){\rule[-0.200pt]{2.409pt}{0.400pt}}
\put(170.0,307.0){\rule[-0.200pt]{2.409pt}{0.400pt}}
\put(1429.0,307.0){\rule[-0.200pt]{2.409pt}{0.400pt}}
\put(170.0,307.0){\rule[-0.200pt]{2.409pt}{0.400pt}}
\put(1429.0,307.0){\rule[-0.200pt]{2.409pt}{0.400pt}}
\put(170.0,308.0){\rule[-0.200pt]{2.409pt}{0.400pt}}
\put(1429.0,308.0){\rule[-0.200pt]{2.409pt}{0.400pt}}
\put(170.0,308.0){\rule[-0.200pt]{2.409pt}{0.400pt}}
\put(1429.0,308.0){\rule[-0.200pt]{2.409pt}{0.400pt}}
\put(170.0,308.0){\rule[-0.200pt]{2.409pt}{0.400pt}}
\put(1429.0,308.0){\rule[-0.200pt]{2.409pt}{0.400pt}}
\put(170.0,308.0){\rule[-0.200pt]{2.409pt}{0.400pt}}
\put(1429.0,308.0){\rule[-0.200pt]{2.409pt}{0.400pt}}
\put(170.0,308.0){\rule[-0.200pt]{2.409pt}{0.400pt}}
\put(1429.0,308.0){\rule[-0.200pt]{2.409pt}{0.400pt}}
\put(170.0,308.0){\rule[-0.200pt]{2.409pt}{0.400pt}}
\put(1429.0,308.0){\rule[-0.200pt]{2.409pt}{0.400pt}}
\put(170.0,308.0){\rule[-0.200pt]{2.409pt}{0.400pt}}
\put(1429.0,308.0){\rule[-0.200pt]{2.409pt}{0.400pt}}
\put(170.0,308.0){\rule[-0.200pt]{2.409pt}{0.400pt}}
\put(1429.0,308.0){\rule[-0.200pt]{2.409pt}{0.400pt}}
\put(170.0,308.0){\rule[-0.200pt]{2.409pt}{0.400pt}}
\put(1429.0,308.0){\rule[-0.200pt]{2.409pt}{0.400pt}}
\put(170.0,308.0){\rule[-0.200pt]{2.409pt}{0.400pt}}
\put(1429.0,308.0){\rule[-0.200pt]{2.409pt}{0.400pt}}
\put(170.0,308.0){\rule[-0.200pt]{2.409pt}{0.400pt}}
\put(1429.0,308.0){\rule[-0.200pt]{2.409pt}{0.400pt}}
\put(170.0,309.0){\rule[-0.200pt]{2.409pt}{0.400pt}}
\put(1429.0,309.0){\rule[-0.200pt]{2.409pt}{0.400pt}}
\put(170.0,309.0){\rule[-0.200pt]{2.409pt}{0.400pt}}
\put(1429.0,309.0){\rule[-0.200pt]{2.409pt}{0.400pt}}
\put(170.0,309.0){\rule[-0.200pt]{2.409pt}{0.400pt}}
\put(1429.0,309.0){\rule[-0.200pt]{2.409pt}{0.400pt}}
\put(170.0,309.0){\rule[-0.200pt]{2.409pt}{0.400pt}}
\put(1429.0,309.0){\rule[-0.200pt]{2.409pt}{0.400pt}}
\put(170.0,309.0){\rule[-0.200pt]{2.409pt}{0.400pt}}
\put(1429.0,309.0){\rule[-0.200pt]{2.409pt}{0.400pt}}
\put(170.0,309.0){\rule[-0.200pt]{2.409pt}{0.400pt}}
\put(1429.0,309.0){\rule[-0.200pt]{2.409pt}{0.400pt}}
\put(170.0,309.0){\rule[-0.200pt]{2.409pt}{0.400pt}}
\put(1429.0,309.0){\rule[-0.200pt]{2.409pt}{0.400pt}}
\put(170.0,309.0){\rule[-0.200pt]{2.409pt}{0.400pt}}
\put(1429.0,309.0){\rule[-0.200pt]{2.409pt}{0.400pt}}
\put(170.0,309.0){\rule[-0.200pt]{2.409pt}{0.400pt}}
\put(1429.0,309.0){\rule[-0.200pt]{2.409pt}{0.400pt}}
\put(170.0,309.0){\rule[-0.200pt]{2.409pt}{0.400pt}}
\put(1429.0,309.0){\rule[-0.200pt]{2.409pt}{0.400pt}}
\put(170.0,309.0){\rule[-0.200pt]{2.409pt}{0.400pt}}
\put(1429.0,309.0){\rule[-0.200pt]{2.409pt}{0.400pt}}
\put(170.0,310.0){\rule[-0.200pt]{2.409pt}{0.400pt}}
\put(1429.0,310.0){\rule[-0.200pt]{2.409pt}{0.400pt}}
\put(170.0,310.0){\rule[-0.200pt]{2.409pt}{0.400pt}}
\put(1429.0,310.0){\rule[-0.200pt]{2.409pt}{0.400pt}}
\put(170.0,310.0){\rule[-0.200pt]{2.409pt}{0.400pt}}
\put(1429.0,310.0){\rule[-0.200pt]{2.409pt}{0.400pt}}
\put(170.0,310.0){\rule[-0.200pt]{2.409pt}{0.400pt}}
\put(1429.0,310.0){\rule[-0.200pt]{2.409pt}{0.400pt}}
\put(170.0,310.0){\rule[-0.200pt]{2.409pt}{0.400pt}}
\put(1429.0,310.0){\rule[-0.200pt]{2.409pt}{0.400pt}}
\put(170.0,310.0){\rule[-0.200pt]{2.409pt}{0.400pt}}
\put(1429.0,310.0){\rule[-0.200pt]{2.409pt}{0.400pt}}
\put(170.0,310.0){\rule[-0.200pt]{2.409pt}{0.400pt}}
\put(1429.0,310.0){\rule[-0.200pt]{2.409pt}{0.400pt}}
\put(170.0,310.0){\rule[-0.200pt]{2.409pt}{0.400pt}}
\put(1429.0,310.0){\rule[-0.200pt]{2.409pt}{0.400pt}}
\put(170.0,310.0){\rule[-0.200pt]{2.409pt}{0.400pt}}
\put(1429.0,310.0){\rule[-0.200pt]{2.409pt}{0.400pt}}
\put(170.0,310.0){\rule[-0.200pt]{2.409pt}{0.400pt}}
\put(1429.0,310.0){\rule[-0.200pt]{2.409pt}{0.400pt}}
\put(170.0,310.0){\rule[-0.200pt]{2.409pt}{0.400pt}}
\put(1429.0,310.0){\rule[-0.200pt]{2.409pt}{0.400pt}}
\put(170.0,310.0){\rule[-0.200pt]{2.409pt}{0.400pt}}
\put(1429.0,310.0){\rule[-0.200pt]{2.409pt}{0.400pt}}
\put(170.0,311.0){\rule[-0.200pt]{2.409pt}{0.400pt}}
\put(1429.0,311.0){\rule[-0.200pt]{2.409pt}{0.400pt}}
\put(170.0,311.0){\rule[-0.200pt]{2.409pt}{0.400pt}}
\put(1429.0,311.0){\rule[-0.200pt]{2.409pt}{0.400pt}}
\put(170.0,311.0){\rule[-0.200pt]{2.409pt}{0.400pt}}
\put(1429.0,311.0){\rule[-0.200pt]{2.409pt}{0.400pt}}
\put(170.0,311.0){\rule[-0.200pt]{2.409pt}{0.400pt}}
\put(1429.0,311.0){\rule[-0.200pt]{2.409pt}{0.400pt}}
\put(170.0,311.0){\rule[-0.200pt]{2.409pt}{0.400pt}}
\put(1429.0,311.0){\rule[-0.200pt]{2.409pt}{0.400pt}}
\put(170.0,311.0){\rule[-0.200pt]{2.409pt}{0.400pt}}
\put(1429.0,311.0){\rule[-0.200pt]{2.409pt}{0.400pt}}
\put(170.0,311.0){\rule[-0.200pt]{2.409pt}{0.400pt}}
\put(1429.0,311.0){\rule[-0.200pt]{2.409pt}{0.400pt}}
\put(170.0,311.0){\rule[-0.200pt]{2.409pt}{0.400pt}}
\put(1429.0,311.0){\rule[-0.200pt]{2.409pt}{0.400pt}}
\put(170.0,311.0){\rule[-0.200pt]{2.409pt}{0.400pt}}
\put(1429.0,311.0){\rule[-0.200pt]{2.409pt}{0.400pt}}
\put(170.0,311.0){\rule[-0.200pt]{2.409pt}{0.400pt}}
\put(1429.0,311.0){\rule[-0.200pt]{2.409pt}{0.400pt}}
\put(170.0,311.0){\rule[-0.200pt]{2.409pt}{0.400pt}}
\put(1429.0,311.0){\rule[-0.200pt]{2.409pt}{0.400pt}}
\put(170.0,311.0){\rule[-0.200pt]{2.409pt}{0.400pt}}
\put(1429.0,311.0){\rule[-0.200pt]{2.409pt}{0.400pt}}
\put(170.0,312.0){\rule[-0.200pt]{2.409pt}{0.400pt}}
\put(1429.0,312.0){\rule[-0.200pt]{2.409pt}{0.400pt}}
\put(170.0,312.0){\rule[-0.200pt]{2.409pt}{0.400pt}}
\put(1429.0,312.0){\rule[-0.200pt]{2.409pt}{0.400pt}}
\put(170.0,312.0){\rule[-0.200pt]{2.409pt}{0.400pt}}
\put(1429.0,312.0){\rule[-0.200pt]{2.409pt}{0.400pt}}
\put(170.0,312.0){\rule[-0.200pt]{2.409pt}{0.400pt}}
\put(1429.0,312.0){\rule[-0.200pt]{2.409pt}{0.400pt}}
\put(170.0,312.0){\rule[-0.200pt]{2.409pt}{0.400pt}}
\put(1429.0,312.0){\rule[-0.200pt]{2.409pt}{0.400pt}}
\put(170.0,312.0){\rule[-0.200pt]{2.409pt}{0.400pt}}
\put(1429.0,312.0){\rule[-0.200pt]{2.409pt}{0.400pt}}
\put(170.0,312.0){\rule[-0.200pt]{2.409pt}{0.400pt}}
\put(1429.0,312.0){\rule[-0.200pt]{2.409pt}{0.400pt}}
\put(170.0,312.0){\rule[-0.200pt]{2.409pt}{0.400pt}}
\put(1429.0,312.0){\rule[-0.200pt]{2.409pt}{0.400pt}}
\put(170.0,312.0){\rule[-0.200pt]{2.409pt}{0.400pt}}
\put(1429.0,312.0){\rule[-0.200pt]{2.409pt}{0.400pt}}
\put(170.0,312.0){\rule[-0.200pt]{2.409pt}{0.400pt}}
\put(1429.0,312.0){\rule[-0.200pt]{2.409pt}{0.400pt}}
\put(170.0,312.0){\rule[-0.200pt]{2.409pt}{0.400pt}}
\put(1429.0,312.0){\rule[-0.200pt]{2.409pt}{0.400pt}}
\put(170.0,312.0){\rule[-0.200pt]{2.409pt}{0.400pt}}
\put(1429.0,312.0){\rule[-0.200pt]{2.409pt}{0.400pt}}
\put(170.0,313.0){\rule[-0.200pt]{2.409pt}{0.400pt}}
\put(1429.0,313.0){\rule[-0.200pt]{2.409pt}{0.400pt}}
\put(170.0,313.0){\rule[-0.200pt]{2.409pt}{0.400pt}}
\put(1429.0,313.0){\rule[-0.200pt]{2.409pt}{0.400pt}}
\put(170.0,313.0){\rule[-0.200pt]{2.409pt}{0.400pt}}
\put(1429.0,313.0){\rule[-0.200pt]{2.409pt}{0.400pt}}
\put(170.0,313.0){\rule[-0.200pt]{2.409pt}{0.400pt}}
\put(1429.0,313.0){\rule[-0.200pt]{2.409pt}{0.400pt}}
\put(170.0,313.0){\rule[-0.200pt]{2.409pt}{0.400pt}}
\put(1429.0,313.0){\rule[-0.200pt]{2.409pt}{0.400pt}}
\put(170.0,313.0){\rule[-0.200pt]{2.409pt}{0.400pt}}
\put(1429.0,313.0){\rule[-0.200pt]{2.409pt}{0.400pt}}
\put(170.0,313.0){\rule[-0.200pt]{2.409pt}{0.400pt}}
\put(1429.0,313.0){\rule[-0.200pt]{2.409pt}{0.400pt}}
\put(170.0,313.0){\rule[-0.200pt]{2.409pt}{0.400pt}}
\put(1429.0,313.0){\rule[-0.200pt]{2.409pt}{0.400pt}}
\put(170.0,313.0){\rule[-0.200pt]{2.409pt}{0.400pt}}
\put(1429.0,313.0){\rule[-0.200pt]{2.409pt}{0.400pt}}
\put(170.0,313.0){\rule[-0.200pt]{2.409pt}{0.400pt}}
\put(1429.0,313.0){\rule[-0.200pt]{2.409pt}{0.400pt}}
\put(170.0,313.0){\rule[-0.200pt]{2.409pt}{0.400pt}}
\put(1429.0,313.0){\rule[-0.200pt]{2.409pt}{0.400pt}}
\put(170.0,313.0){\rule[-0.200pt]{2.409pt}{0.400pt}}
\put(1429.0,313.0){\rule[-0.200pt]{2.409pt}{0.400pt}}
\put(170.0,313.0){\rule[-0.200pt]{2.409pt}{0.400pt}}
\put(1429.0,313.0){\rule[-0.200pt]{2.409pt}{0.400pt}}
\put(170.0,314.0){\rule[-0.200pt]{2.409pt}{0.400pt}}
\put(1429.0,314.0){\rule[-0.200pt]{2.409pt}{0.400pt}}
\put(170.0,314.0){\rule[-0.200pt]{2.409pt}{0.400pt}}
\put(1429.0,314.0){\rule[-0.200pt]{2.409pt}{0.400pt}}
\put(170.0,314.0){\rule[-0.200pt]{2.409pt}{0.400pt}}
\put(1429.0,314.0){\rule[-0.200pt]{2.409pt}{0.400pt}}
\put(170.0,314.0){\rule[-0.200pt]{2.409pt}{0.400pt}}
\put(1429.0,314.0){\rule[-0.200pt]{2.409pt}{0.400pt}}
\put(170.0,314.0){\rule[-0.200pt]{2.409pt}{0.400pt}}
\put(1429.0,314.0){\rule[-0.200pt]{2.409pt}{0.400pt}}
\put(170.0,314.0){\rule[-0.200pt]{2.409pt}{0.400pt}}
\put(1429.0,314.0){\rule[-0.200pt]{2.409pt}{0.400pt}}
\put(170.0,314.0){\rule[-0.200pt]{2.409pt}{0.400pt}}
\put(1429.0,314.0){\rule[-0.200pt]{2.409pt}{0.400pt}}
\put(170.0,314.0){\rule[-0.200pt]{2.409pt}{0.400pt}}
\put(1429.0,314.0){\rule[-0.200pt]{2.409pt}{0.400pt}}
\put(170.0,314.0){\rule[-0.200pt]{2.409pt}{0.400pt}}
\put(1429.0,314.0){\rule[-0.200pt]{2.409pt}{0.400pt}}
\put(170.0,314.0){\rule[-0.200pt]{2.409pt}{0.400pt}}
\put(1429.0,314.0){\rule[-0.200pt]{2.409pt}{0.400pt}}
\put(170.0,314.0){\rule[-0.200pt]{2.409pt}{0.400pt}}
\put(1429.0,314.0){\rule[-0.200pt]{2.409pt}{0.400pt}}
\put(170.0,314.0){\rule[-0.200pt]{2.409pt}{0.400pt}}
\put(1429.0,314.0){\rule[-0.200pt]{2.409pt}{0.400pt}}
\put(170.0,314.0){\rule[-0.200pt]{2.409pt}{0.400pt}}
\put(1429.0,314.0){\rule[-0.200pt]{2.409pt}{0.400pt}}
\put(170.0,315.0){\rule[-0.200pt]{2.409pt}{0.400pt}}
\put(1429.0,315.0){\rule[-0.200pt]{2.409pt}{0.400pt}}
\put(170.0,315.0){\rule[-0.200pt]{2.409pt}{0.400pt}}
\put(1429.0,315.0){\rule[-0.200pt]{2.409pt}{0.400pt}}
\put(170.0,315.0){\rule[-0.200pt]{2.409pt}{0.400pt}}
\put(1429.0,315.0){\rule[-0.200pt]{2.409pt}{0.400pt}}
\put(170.0,315.0){\rule[-0.200pt]{2.409pt}{0.400pt}}
\put(1429.0,315.0){\rule[-0.200pt]{2.409pt}{0.400pt}}
\put(170.0,315.0){\rule[-0.200pt]{2.409pt}{0.400pt}}
\put(1429.0,315.0){\rule[-0.200pt]{2.409pt}{0.400pt}}
\put(170.0,315.0){\rule[-0.200pt]{2.409pt}{0.400pt}}
\put(1429.0,315.0){\rule[-0.200pt]{2.409pt}{0.400pt}}
\put(170.0,315.0){\rule[-0.200pt]{2.409pt}{0.400pt}}
\put(1429.0,315.0){\rule[-0.200pt]{2.409pt}{0.400pt}}
\put(170.0,315.0){\rule[-0.200pt]{2.409pt}{0.400pt}}
\put(1429.0,315.0){\rule[-0.200pt]{2.409pt}{0.400pt}}
\put(170.0,315.0){\rule[-0.200pt]{2.409pt}{0.400pt}}
\put(1429.0,315.0){\rule[-0.200pt]{2.409pt}{0.400pt}}
\put(170.0,315.0){\rule[-0.200pt]{2.409pt}{0.400pt}}
\put(1429.0,315.0){\rule[-0.200pt]{2.409pt}{0.400pt}}
\put(170.0,315.0){\rule[-0.200pt]{2.409pt}{0.400pt}}
\put(1429.0,315.0){\rule[-0.200pt]{2.409pt}{0.400pt}}
\put(170.0,315.0){\rule[-0.200pt]{2.409pt}{0.400pt}}
\put(1429.0,315.0){\rule[-0.200pt]{2.409pt}{0.400pt}}
\put(170.0,315.0){\rule[-0.200pt]{2.409pt}{0.400pt}}
\put(1429.0,315.0){\rule[-0.200pt]{2.409pt}{0.400pt}}
\put(170.0,316.0){\rule[-0.200pt]{2.409pt}{0.400pt}}
\put(1429.0,316.0){\rule[-0.200pt]{2.409pt}{0.400pt}}
\put(170.0,316.0){\rule[-0.200pt]{2.409pt}{0.400pt}}
\put(1429.0,316.0){\rule[-0.200pt]{2.409pt}{0.400pt}}
\put(170.0,316.0){\rule[-0.200pt]{2.409pt}{0.400pt}}
\put(1429.0,316.0){\rule[-0.200pt]{2.409pt}{0.400pt}}
\put(170.0,316.0){\rule[-0.200pt]{2.409pt}{0.400pt}}
\put(1429.0,316.0){\rule[-0.200pt]{2.409pt}{0.400pt}}
\put(170.0,316.0){\rule[-0.200pt]{2.409pt}{0.400pt}}
\put(1429.0,316.0){\rule[-0.200pt]{2.409pt}{0.400pt}}
\put(170.0,316.0){\rule[-0.200pt]{2.409pt}{0.400pt}}
\put(1429.0,316.0){\rule[-0.200pt]{2.409pt}{0.400pt}}
\put(170.0,316.0){\rule[-0.200pt]{2.409pt}{0.400pt}}
\put(1429.0,316.0){\rule[-0.200pt]{2.409pt}{0.400pt}}
\put(170.0,316.0){\rule[-0.200pt]{2.409pt}{0.400pt}}
\put(1429.0,316.0){\rule[-0.200pt]{2.409pt}{0.400pt}}
\put(170.0,316.0){\rule[-0.200pt]{2.409pt}{0.400pt}}
\put(1429.0,316.0){\rule[-0.200pt]{2.409pt}{0.400pt}}
\put(170.0,316.0){\rule[-0.200pt]{2.409pt}{0.400pt}}
\put(1429.0,316.0){\rule[-0.200pt]{2.409pt}{0.400pt}}
\put(170.0,316.0){\rule[-0.200pt]{2.409pt}{0.400pt}}
\put(1429.0,316.0){\rule[-0.200pt]{2.409pt}{0.400pt}}
\put(170.0,316.0){\rule[-0.200pt]{2.409pt}{0.400pt}}
\put(1429.0,316.0){\rule[-0.200pt]{2.409pt}{0.400pt}}
\put(170.0,316.0){\rule[-0.200pt]{2.409pt}{0.400pt}}
\put(1429.0,316.0){\rule[-0.200pt]{2.409pt}{0.400pt}}
\put(170.0,316.0){\rule[-0.200pt]{2.409pt}{0.400pt}}
\put(1429.0,316.0){\rule[-0.200pt]{2.409pt}{0.400pt}}
\put(170.0,317.0){\rule[-0.200pt]{2.409pt}{0.400pt}}
\put(1429.0,317.0){\rule[-0.200pt]{2.409pt}{0.400pt}}
\put(170.0,317.0){\rule[-0.200pt]{2.409pt}{0.400pt}}
\put(1429.0,317.0){\rule[-0.200pt]{2.409pt}{0.400pt}}
\put(170.0,317.0){\rule[-0.200pt]{2.409pt}{0.400pt}}
\put(1429.0,317.0){\rule[-0.200pt]{2.409pt}{0.400pt}}
\put(170.0,317.0){\rule[-0.200pt]{2.409pt}{0.400pt}}
\put(1429.0,317.0){\rule[-0.200pt]{2.409pt}{0.400pt}}
\put(170.0,317.0){\rule[-0.200pt]{2.409pt}{0.400pt}}
\put(1429.0,317.0){\rule[-0.200pt]{2.409pt}{0.400pt}}
\put(170.0,317.0){\rule[-0.200pt]{2.409pt}{0.400pt}}
\put(1429.0,317.0){\rule[-0.200pt]{2.409pt}{0.400pt}}
\put(170.0,317.0){\rule[-0.200pt]{2.409pt}{0.400pt}}
\put(1429.0,317.0){\rule[-0.200pt]{2.409pt}{0.400pt}}
\put(170.0,317.0){\rule[-0.200pt]{2.409pt}{0.400pt}}
\put(1429.0,317.0){\rule[-0.200pt]{2.409pt}{0.400pt}}
\put(170.0,317.0){\rule[-0.200pt]{2.409pt}{0.400pt}}
\put(1429.0,317.0){\rule[-0.200pt]{2.409pt}{0.400pt}}
\put(170.0,317.0){\rule[-0.200pt]{2.409pt}{0.400pt}}
\put(1429.0,317.0){\rule[-0.200pt]{2.409pt}{0.400pt}}
\put(170.0,317.0){\rule[-0.200pt]{2.409pt}{0.400pt}}
\put(1429.0,317.0){\rule[-0.200pt]{2.409pt}{0.400pt}}
\put(170.0,317.0){\rule[-0.200pt]{2.409pt}{0.400pt}}
\put(1429.0,317.0){\rule[-0.200pt]{2.409pt}{0.400pt}}
\put(170.0,317.0){\rule[-0.200pt]{2.409pt}{0.400pt}}
\put(1429.0,317.0){\rule[-0.200pt]{2.409pt}{0.400pt}}
\put(170.0,317.0){\rule[-0.200pt]{2.409pt}{0.400pt}}
\put(1429.0,317.0){\rule[-0.200pt]{2.409pt}{0.400pt}}
\put(170.0,318.0){\rule[-0.200pt]{2.409pt}{0.400pt}}
\put(1429.0,318.0){\rule[-0.200pt]{2.409pt}{0.400pt}}
\put(170.0,318.0){\rule[-0.200pt]{2.409pt}{0.400pt}}
\put(1429.0,318.0){\rule[-0.200pt]{2.409pt}{0.400pt}}
\put(170.0,318.0){\rule[-0.200pt]{2.409pt}{0.400pt}}
\put(1429.0,318.0){\rule[-0.200pt]{2.409pt}{0.400pt}}
\put(170.0,318.0){\rule[-0.200pt]{2.409pt}{0.400pt}}
\put(1429.0,318.0){\rule[-0.200pt]{2.409pt}{0.400pt}}
\put(170.0,318.0){\rule[-0.200pt]{2.409pt}{0.400pt}}
\put(1429.0,318.0){\rule[-0.200pt]{2.409pt}{0.400pt}}
\put(170.0,318.0){\rule[-0.200pt]{2.409pt}{0.400pt}}
\put(1429.0,318.0){\rule[-0.200pt]{2.409pt}{0.400pt}}
\put(170.0,318.0){\rule[-0.200pt]{2.409pt}{0.400pt}}
\put(1429.0,318.0){\rule[-0.200pt]{2.409pt}{0.400pt}}
\put(170.0,318.0){\rule[-0.200pt]{2.409pt}{0.400pt}}
\put(1429.0,318.0){\rule[-0.200pt]{2.409pt}{0.400pt}}
\put(170.0,318.0){\rule[-0.200pt]{2.409pt}{0.400pt}}
\put(1429.0,318.0){\rule[-0.200pt]{2.409pt}{0.400pt}}
\put(170.0,318.0){\rule[-0.200pt]{2.409pt}{0.400pt}}
\put(1429.0,318.0){\rule[-0.200pt]{2.409pt}{0.400pt}}
\put(170.0,318.0){\rule[-0.200pt]{2.409pt}{0.400pt}}
\put(1429.0,318.0){\rule[-0.200pt]{2.409pt}{0.400pt}}
\put(170.0,318.0){\rule[-0.200pt]{2.409pt}{0.400pt}}
\put(1429.0,318.0){\rule[-0.200pt]{2.409pt}{0.400pt}}
\put(170.0,318.0){\rule[-0.200pt]{2.409pt}{0.400pt}}
\put(1429.0,318.0){\rule[-0.200pt]{2.409pt}{0.400pt}}
\put(170.0,318.0){\rule[-0.200pt]{2.409pt}{0.400pt}}
\put(1429.0,318.0){\rule[-0.200pt]{2.409pt}{0.400pt}}
\put(170.0,319.0){\rule[-0.200pt]{2.409pt}{0.400pt}}
\put(1429.0,319.0){\rule[-0.200pt]{2.409pt}{0.400pt}}
\put(170.0,319.0){\rule[-0.200pt]{2.409pt}{0.400pt}}
\put(1429.0,319.0){\rule[-0.200pt]{2.409pt}{0.400pt}}
\put(170.0,319.0){\rule[-0.200pt]{2.409pt}{0.400pt}}
\put(1429.0,319.0){\rule[-0.200pt]{2.409pt}{0.400pt}}
\put(170.0,319.0){\rule[-0.200pt]{2.409pt}{0.400pt}}
\put(1429.0,319.0){\rule[-0.200pt]{2.409pt}{0.400pt}}
\put(170.0,319.0){\rule[-0.200pt]{2.409pt}{0.400pt}}
\put(1429.0,319.0){\rule[-0.200pt]{2.409pt}{0.400pt}}
\put(170.0,319.0){\rule[-0.200pt]{2.409pt}{0.400pt}}
\put(1429.0,319.0){\rule[-0.200pt]{2.409pt}{0.400pt}}
\put(170.0,319.0){\rule[-0.200pt]{2.409pt}{0.400pt}}
\put(1429.0,319.0){\rule[-0.200pt]{2.409pt}{0.400pt}}
\put(170.0,319.0){\rule[-0.200pt]{2.409pt}{0.400pt}}
\put(1429.0,319.0){\rule[-0.200pt]{2.409pt}{0.400pt}}
\put(170.0,319.0){\rule[-0.200pt]{2.409pt}{0.400pt}}
\put(1429.0,319.0){\rule[-0.200pt]{2.409pt}{0.400pt}}
\put(170.0,319.0){\rule[-0.200pt]{2.409pt}{0.400pt}}
\put(1429.0,319.0){\rule[-0.200pt]{2.409pt}{0.400pt}}
\put(170.0,319.0){\rule[-0.200pt]{2.409pt}{0.400pt}}
\put(1429.0,319.0){\rule[-0.200pt]{2.409pt}{0.400pt}}
\put(170.0,319.0){\rule[-0.200pt]{2.409pt}{0.400pt}}
\put(1429.0,319.0){\rule[-0.200pt]{2.409pt}{0.400pt}}
\put(170.0,319.0){\rule[-0.200pt]{2.409pt}{0.400pt}}
\put(1429.0,319.0){\rule[-0.200pt]{2.409pt}{0.400pt}}
\put(170.0,319.0){\rule[-0.200pt]{2.409pt}{0.400pt}}
\put(1429.0,319.0){\rule[-0.200pt]{2.409pt}{0.400pt}}
\put(170.0,319.0){\rule[-0.200pt]{2.409pt}{0.400pt}}
\put(1429.0,319.0){\rule[-0.200pt]{2.409pt}{0.400pt}}
\put(170.0,320.0){\rule[-0.200pt]{2.409pt}{0.400pt}}
\put(1429.0,320.0){\rule[-0.200pt]{2.409pt}{0.400pt}}
\put(170.0,320.0){\rule[-0.200pt]{2.409pt}{0.400pt}}
\put(1429.0,320.0){\rule[-0.200pt]{2.409pt}{0.400pt}}
\put(170.0,320.0){\rule[-0.200pt]{2.409pt}{0.400pt}}
\put(1429.0,320.0){\rule[-0.200pt]{2.409pt}{0.400pt}}
\put(170.0,320.0){\rule[-0.200pt]{2.409pt}{0.400pt}}
\put(1429.0,320.0){\rule[-0.200pt]{2.409pt}{0.400pt}}
\put(170.0,320.0){\rule[-0.200pt]{2.409pt}{0.400pt}}
\put(1429.0,320.0){\rule[-0.200pt]{2.409pt}{0.400pt}}
\put(170.0,320.0){\rule[-0.200pt]{2.409pt}{0.400pt}}
\put(1429.0,320.0){\rule[-0.200pt]{2.409pt}{0.400pt}}
\put(170.0,320.0){\rule[-0.200pt]{2.409pt}{0.400pt}}
\put(1429.0,320.0){\rule[-0.200pt]{2.409pt}{0.400pt}}
\put(170.0,320.0){\rule[-0.200pt]{2.409pt}{0.400pt}}
\put(1429.0,320.0){\rule[-0.200pt]{2.409pt}{0.400pt}}
\put(170.0,320.0){\rule[-0.200pt]{2.409pt}{0.400pt}}
\put(1429.0,320.0){\rule[-0.200pt]{2.409pt}{0.400pt}}
\put(170.0,320.0){\rule[-0.200pt]{2.409pt}{0.400pt}}
\put(1429.0,320.0){\rule[-0.200pt]{2.409pt}{0.400pt}}
\put(170.0,320.0){\rule[-0.200pt]{2.409pt}{0.400pt}}
\put(1429.0,320.0){\rule[-0.200pt]{2.409pt}{0.400pt}}
\put(170.0,320.0){\rule[-0.200pt]{2.409pt}{0.400pt}}
\put(1429.0,320.0){\rule[-0.200pt]{2.409pt}{0.400pt}}
\put(170.0,320.0){\rule[-0.200pt]{2.409pt}{0.400pt}}
\put(1429.0,320.0){\rule[-0.200pt]{2.409pt}{0.400pt}}
\put(170.0,320.0){\rule[-0.200pt]{2.409pt}{0.400pt}}
\put(1429.0,320.0){\rule[-0.200pt]{2.409pt}{0.400pt}}
\put(170.0,320.0){\rule[-0.200pt]{2.409pt}{0.400pt}}
\put(1429.0,320.0){\rule[-0.200pt]{2.409pt}{0.400pt}}
\put(170.0,321.0){\rule[-0.200pt]{2.409pt}{0.400pt}}
\put(1429.0,321.0){\rule[-0.200pt]{2.409pt}{0.400pt}}
\put(170.0,321.0){\rule[-0.200pt]{2.409pt}{0.400pt}}
\put(1429.0,321.0){\rule[-0.200pt]{2.409pt}{0.400pt}}
\put(170.0,321.0){\rule[-0.200pt]{2.409pt}{0.400pt}}
\put(1429.0,321.0){\rule[-0.200pt]{2.409pt}{0.400pt}}
\put(170.0,321.0){\rule[-0.200pt]{2.409pt}{0.400pt}}
\put(1429.0,321.0){\rule[-0.200pt]{2.409pt}{0.400pt}}
\put(170.0,321.0){\rule[-0.200pt]{2.409pt}{0.400pt}}
\put(1429.0,321.0){\rule[-0.200pt]{2.409pt}{0.400pt}}
\put(170.0,321.0){\rule[-0.200pt]{2.409pt}{0.400pt}}
\put(1429.0,321.0){\rule[-0.200pt]{2.409pt}{0.400pt}}
\put(170.0,321.0){\rule[-0.200pt]{2.409pt}{0.400pt}}
\put(1429.0,321.0){\rule[-0.200pt]{2.409pt}{0.400pt}}
\put(170.0,321.0){\rule[-0.200pt]{2.409pt}{0.400pt}}
\put(1429.0,321.0){\rule[-0.200pt]{2.409pt}{0.400pt}}
\put(170.0,321.0){\rule[-0.200pt]{2.409pt}{0.400pt}}
\put(1429.0,321.0){\rule[-0.200pt]{2.409pt}{0.400pt}}
\put(170.0,321.0){\rule[-0.200pt]{2.409pt}{0.400pt}}
\put(1429.0,321.0){\rule[-0.200pt]{2.409pt}{0.400pt}}
\put(170.0,321.0){\rule[-0.200pt]{2.409pt}{0.400pt}}
\put(1429.0,321.0){\rule[-0.200pt]{2.409pt}{0.400pt}}
\put(170.0,321.0){\rule[-0.200pt]{2.409pt}{0.400pt}}
\put(1429.0,321.0){\rule[-0.200pt]{2.409pt}{0.400pt}}
\put(170.0,321.0){\rule[-0.200pt]{2.409pt}{0.400pt}}
\put(1429.0,321.0){\rule[-0.200pt]{2.409pt}{0.400pt}}
\put(170.0,321.0){\rule[-0.200pt]{2.409pt}{0.400pt}}
\put(1429.0,321.0){\rule[-0.200pt]{2.409pt}{0.400pt}}
\put(170.0,321.0){\rule[-0.200pt]{2.409pt}{0.400pt}}
\put(1429.0,321.0){\rule[-0.200pt]{2.409pt}{0.400pt}}
\put(170.0,321.0){\rule[-0.200pt]{2.409pt}{0.400pt}}
\put(1429.0,321.0){\rule[-0.200pt]{2.409pt}{0.400pt}}
\put(170.0,322.0){\rule[-0.200pt]{2.409pt}{0.400pt}}
\put(1429.0,322.0){\rule[-0.200pt]{2.409pt}{0.400pt}}
\put(170.0,322.0){\rule[-0.200pt]{2.409pt}{0.400pt}}
\put(1429.0,322.0){\rule[-0.200pt]{2.409pt}{0.400pt}}
\put(170.0,322.0){\rule[-0.200pt]{2.409pt}{0.400pt}}
\put(1429.0,322.0){\rule[-0.200pt]{2.409pt}{0.400pt}}
\put(170.0,322.0){\rule[-0.200pt]{2.409pt}{0.400pt}}
\put(1429.0,322.0){\rule[-0.200pt]{2.409pt}{0.400pt}}
\put(170.0,322.0){\rule[-0.200pt]{2.409pt}{0.400pt}}
\put(1429.0,322.0){\rule[-0.200pt]{2.409pt}{0.400pt}}
\put(170.0,322.0){\rule[-0.200pt]{2.409pt}{0.400pt}}
\put(1429.0,322.0){\rule[-0.200pt]{2.409pt}{0.400pt}}
\put(170.0,322.0){\rule[-0.200pt]{2.409pt}{0.400pt}}
\put(1429.0,322.0){\rule[-0.200pt]{2.409pt}{0.400pt}}
\put(170.0,322.0){\rule[-0.200pt]{2.409pt}{0.400pt}}
\put(1429.0,322.0){\rule[-0.200pt]{2.409pt}{0.400pt}}
\put(170.0,322.0){\rule[-0.200pt]{2.409pt}{0.400pt}}
\put(1429.0,322.0){\rule[-0.200pt]{2.409pt}{0.400pt}}
\put(170.0,322.0){\rule[-0.200pt]{2.409pt}{0.400pt}}
\put(1429.0,322.0){\rule[-0.200pt]{2.409pt}{0.400pt}}
\put(170.0,322.0){\rule[-0.200pt]{2.409pt}{0.400pt}}
\put(1429.0,322.0){\rule[-0.200pt]{2.409pt}{0.400pt}}
\put(170.0,322.0){\rule[-0.200pt]{2.409pt}{0.400pt}}
\put(1429.0,322.0){\rule[-0.200pt]{2.409pt}{0.400pt}}
\put(170.0,322.0){\rule[-0.200pt]{2.409pt}{0.400pt}}
\put(1429.0,322.0){\rule[-0.200pt]{2.409pt}{0.400pt}}
\put(170.0,322.0){\rule[-0.200pt]{2.409pt}{0.400pt}}
\put(1429.0,322.0){\rule[-0.200pt]{2.409pt}{0.400pt}}
\put(170.0,322.0){\rule[-0.200pt]{2.409pt}{0.400pt}}
\put(1429.0,322.0){\rule[-0.200pt]{2.409pt}{0.400pt}}
\put(170.0,322.0){\rule[-0.200pt]{2.409pt}{0.400pt}}
\put(1429.0,322.0){\rule[-0.200pt]{2.409pt}{0.400pt}}
\put(170.0,323.0){\rule[-0.200pt]{2.409pt}{0.400pt}}
\put(1429.0,323.0){\rule[-0.200pt]{2.409pt}{0.400pt}}
\put(170.0,323.0){\rule[-0.200pt]{2.409pt}{0.400pt}}
\put(1429.0,323.0){\rule[-0.200pt]{2.409pt}{0.400pt}}
\put(170.0,323.0){\rule[-0.200pt]{2.409pt}{0.400pt}}
\put(1429.0,323.0){\rule[-0.200pt]{2.409pt}{0.400pt}}
\put(170.0,323.0){\rule[-0.200pt]{2.409pt}{0.400pt}}
\put(1429.0,323.0){\rule[-0.200pt]{2.409pt}{0.400pt}}
\put(170.0,323.0){\rule[-0.200pt]{2.409pt}{0.400pt}}
\put(1429.0,323.0){\rule[-0.200pt]{2.409pt}{0.400pt}}
\put(170.0,323.0){\rule[-0.200pt]{2.409pt}{0.400pt}}
\put(1429.0,323.0){\rule[-0.200pt]{2.409pt}{0.400pt}}
\put(170.0,323.0){\rule[-0.200pt]{2.409pt}{0.400pt}}
\put(1429.0,323.0){\rule[-0.200pt]{2.409pt}{0.400pt}}
\put(170.0,323.0){\rule[-0.200pt]{2.409pt}{0.400pt}}
\put(1429.0,323.0){\rule[-0.200pt]{2.409pt}{0.400pt}}
\put(170.0,323.0){\rule[-0.200pt]{2.409pt}{0.400pt}}
\put(1429.0,323.0){\rule[-0.200pt]{2.409pt}{0.400pt}}
\put(170.0,323.0){\rule[-0.200pt]{2.409pt}{0.400pt}}
\put(1429.0,323.0){\rule[-0.200pt]{2.409pt}{0.400pt}}
\put(170.0,323.0){\rule[-0.200pt]{2.409pt}{0.400pt}}
\put(1429.0,323.0){\rule[-0.200pt]{2.409pt}{0.400pt}}
\put(170.0,323.0){\rule[-0.200pt]{2.409pt}{0.400pt}}
\put(1429.0,323.0){\rule[-0.200pt]{2.409pt}{0.400pt}}
\put(170.0,323.0){\rule[-0.200pt]{2.409pt}{0.400pt}}
\put(1429.0,323.0){\rule[-0.200pt]{2.409pt}{0.400pt}}
\put(170.0,323.0){\rule[-0.200pt]{2.409pt}{0.400pt}}
\put(1429.0,323.0){\rule[-0.200pt]{2.409pt}{0.400pt}}
\put(170.0,323.0){\rule[-0.200pt]{2.409pt}{0.400pt}}
\put(1429.0,323.0){\rule[-0.200pt]{2.409pt}{0.400pt}}
\put(170.0,323.0){\rule[-0.200pt]{2.409pt}{0.400pt}}
\put(1429.0,323.0){\rule[-0.200pt]{2.409pt}{0.400pt}}
\put(170.0,323.0){\rule[-0.200pt]{2.409pt}{0.400pt}}
\put(1429.0,323.0){\rule[-0.200pt]{2.409pt}{0.400pt}}
\put(170.0,324.0){\rule[-0.200pt]{2.409pt}{0.400pt}}
\put(1429.0,324.0){\rule[-0.200pt]{2.409pt}{0.400pt}}
\put(170.0,324.0){\rule[-0.200pt]{2.409pt}{0.400pt}}
\put(1429.0,324.0){\rule[-0.200pt]{2.409pt}{0.400pt}}
\put(170.0,324.0){\rule[-0.200pt]{2.409pt}{0.400pt}}
\put(1429.0,324.0){\rule[-0.200pt]{2.409pt}{0.400pt}}
\put(170.0,324.0){\rule[-0.200pt]{2.409pt}{0.400pt}}
\put(1429.0,324.0){\rule[-0.200pt]{2.409pt}{0.400pt}}
\put(170.0,324.0){\rule[-0.200pt]{2.409pt}{0.400pt}}
\put(1429.0,324.0){\rule[-0.200pt]{2.409pt}{0.400pt}}
\put(170.0,324.0){\rule[-0.200pt]{2.409pt}{0.400pt}}
\put(1429.0,324.0){\rule[-0.200pt]{2.409pt}{0.400pt}}
\put(170.0,324.0){\rule[-0.200pt]{2.409pt}{0.400pt}}
\put(1429.0,324.0){\rule[-0.200pt]{2.409pt}{0.400pt}}
\put(170.0,324.0){\rule[-0.200pt]{2.409pt}{0.400pt}}
\put(1429.0,324.0){\rule[-0.200pt]{2.409pt}{0.400pt}}
\put(170.0,324.0){\rule[-0.200pt]{2.409pt}{0.400pt}}
\put(1429.0,324.0){\rule[-0.200pt]{2.409pt}{0.400pt}}
\put(170.0,324.0){\rule[-0.200pt]{2.409pt}{0.400pt}}
\put(1429.0,324.0){\rule[-0.200pt]{2.409pt}{0.400pt}}
\put(170.0,324.0){\rule[-0.200pt]{2.409pt}{0.400pt}}
\put(1429.0,324.0){\rule[-0.200pt]{2.409pt}{0.400pt}}
\put(170.0,324.0){\rule[-0.200pt]{2.409pt}{0.400pt}}
\put(1429.0,324.0){\rule[-0.200pt]{2.409pt}{0.400pt}}
\put(170.0,324.0){\rule[-0.200pt]{2.409pt}{0.400pt}}
\put(1429.0,324.0){\rule[-0.200pt]{2.409pt}{0.400pt}}
\put(170.0,324.0){\rule[-0.200pt]{2.409pt}{0.400pt}}
\put(1429.0,324.0){\rule[-0.200pt]{2.409pt}{0.400pt}}
\put(170.0,324.0){\rule[-0.200pt]{2.409pt}{0.400pt}}
\put(1429.0,324.0){\rule[-0.200pt]{2.409pt}{0.400pt}}
\put(170.0,324.0){\rule[-0.200pt]{2.409pt}{0.400pt}}
\put(1429.0,324.0){\rule[-0.200pt]{2.409pt}{0.400pt}}
\put(170.0,325.0){\rule[-0.200pt]{2.409pt}{0.400pt}}
\put(1429.0,325.0){\rule[-0.200pt]{2.409pt}{0.400pt}}
\put(170.0,325.0){\rule[-0.200pt]{2.409pt}{0.400pt}}
\put(1429.0,325.0){\rule[-0.200pt]{2.409pt}{0.400pt}}
\put(170.0,325.0){\rule[-0.200pt]{2.409pt}{0.400pt}}
\put(1429.0,325.0){\rule[-0.200pt]{2.409pt}{0.400pt}}
\put(170.0,325.0){\rule[-0.200pt]{2.409pt}{0.400pt}}
\put(1429.0,325.0){\rule[-0.200pt]{2.409pt}{0.400pt}}
\put(170.0,325.0){\rule[-0.200pt]{2.409pt}{0.400pt}}
\put(1429.0,325.0){\rule[-0.200pt]{2.409pt}{0.400pt}}
\put(170.0,325.0){\rule[-0.200pt]{2.409pt}{0.400pt}}
\put(1429.0,325.0){\rule[-0.200pt]{2.409pt}{0.400pt}}
\put(170.0,325.0){\rule[-0.200pt]{2.409pt}{0.400pt}}
\put(1429.0,325.0){\rule[-0.200pt]{2.409pt}{0.400pt}}
\put(170.0,325.0){\rule[-0.200pt]{2.409pt}{0.400pt}}
\put(1429.0,325.0){\rule[-0.200pt]{2.409pt}{0.400pt}}
\put(170.0,325.0){\rule[-0.200pt]{2.409pt}{0.400pt}}
\put(1429.0,325.0){\rule[-0.200pt]{2.409pt}{0.400pt}}
\put(170.0,325.0){\rule[-0.200pt]{2.409pt}{0.400pt}}
\put(1429.0,325.0){\rule[-0.200pt]{2.409pt}{0.400pt}}
\put(170.0,325.0){\rule[-0.200pt]{2.409pt}{0.400pt}}
\put(1429.0,325.0){\rule[-0.200pt]{2.409pt}{0.400pt}}
\put(170.0,325.0){\rule[-0.200pt]{2.409pt}{0.400pt}}
\put(1429.0,325.0){\rule[-0.200pt]{2.409pt}{0.400pt}}
\put(170.0,325.0){\rule[-0.200pt]{2.409pt}{0.400pt}}
\put(1429.0,325.0){\rule[-0.200pt]{2.409pt}{0.400pt}}
\put(170.0,325.0){\rule[-0.200pt]{2.409pt}{0.400pt}}
\put(1429.0,325.0){\rule[-0.200pt]{2.409pt}{0.400pt}}
\put(170.0,325.0){\rule[-0.200pt]{2.409pt}{0.400pt}}
\put(1429.0,325.0){\rule[-0.200pt]{2.409pt}{0.400pt}}
\put(170.0,325.0){\rule[-0.200pt]{2.409pt}{0.400pt}}
\put(1429.0,325.0){\rule[-0.200pt]{2.409pt}{0.400pt}}
\put(170.0,325.0){\rule[-0.200pt]{2.409pt}{0.400pt}}
\put(1429.0,325.0){\rule[-0.200pt]{2.409pt}{0.400pt}}
\put(170.0,325.0){\rule[-0.200pt]{2.409pt}{0.400pt}}
\put(1429.0,325.0){\rule[-0.200pt]{2.409pt}{0.400pt}}
\put(170.0,326.0){\rule[-0.200pt]{2.409pt}{0.400pt}}
\put(1429.0,326.0){\rule[-0.200pt]{2.409pt}{0.400pt}}
\put(170.0,326.0){\rule[-0.200pt]{2.409pt}{0.400pt}}
\put(1429.0,326.0){\rule[-0.200pt]{2.409pt}{0.400pt}}
\put(170.0,326.0){\rule[-0.200pt]{2.409pt}{0.400pt}}
\put(1429.0,326.0){\rule[-0.200pt]{2.409pt}{0.400pt}}
\put(170.0,326.0){\rule[-0.200pt]{2.409pt}{0.400pt}}
\put(1429.0,326.0){\rule[-0.200pt]{2.409pt}{0.400pt}}
\put(170.0,326.0){\rule[-0.200pt]{2.409pt}{0.400pt}}
\put(1429.0,326.0){\rule[-0.200pt]{2.409pt}{0.400pt}}
\put(170.0,326.0){\rule[-0.200pt]{2.409pt}{0.400pt}}
\put(1429.0,326.0){\rule[-0.200pt]{2.409pt}{0.400pt}}
\put(170.0,326.0){\rule[-0.200pt]{2.409pt}{0.400pt}}
\put(1429.0,326.0){\rule[-0.200pt]{2.409pt}{0.400pt}}
\put(170.0,326.0){\rule[-0.200pt]{2.409pt}{0.400pt}}
\put(1429.0,326.0){\rule[-0.200pt]{2.409pt}{0.400pt}}
\put(170.0,326.0){\rule[-0.200pt]{2.409pt}{0.400pt}}
\put(1429.0,326.0){\rule[-0.200pt]{2.409pt}{0.400pt}}
\put(170.0,326.0){\rule[-0.200pt]{2.409pt}{0.400pt}}
\put(1429.0,326.0){\rule[-0.200pt]{2.409pt}{0.400pt}}
\put(170.0,326.0){\rule[-0.200pt]{2.409pt}{0.400pt}}
\put(1429.0,326.0){\rule[-0.200pt]{2.409pt}{0.400pt}}
\put(170.0,326.0){\rule[-0.200pt]{2.409pt}{0.400pt}}
\put(1429.0,326.0){\rule[-0.200pt]{2.409pt}{0.400pt}}
\put(170.0,326.0){\rule[-0.200pt]{2.409pt}{0.400pt}}
\put(1429.0,326.0){\rule[-0.200pt]{2.409pt}{0.400pt}}
\put(170.0,326.0){\rule[-0.200pt]{2.409pt}{0.400pt}}
\put(1429.0,326.0){\rule[-0.200pt]{2.409pt}{0.400pt}}
\put(170.0,326.0){\rule[-0.200pt]{2.409pt}{0.400pt}}
\put(1429.0,326.0){\rule[-0.200pt]{2.409pt}{0.400pt}}
\put(170.0,326.0){\rule[-0.200pt]{2.409pt}{0.400pt}}
\put(1429.0,326.0){\rule[-0.200pt]{2.409pt}{0.400pt}}
\put(170.0,326.0){\rule[-0.200pt]{2.409pt}{0.400pt}}
\put(1429.0,326.0){\rule[-0.200pt]{2.409pt}{0.400pt}}
\put(170.0,326.0){\rule[-0.200pt]{2.409pt}{0.400pt}}
\put(1429.0,326.0){\rule[-0.200pt]{2.409pt}{0.400pt}}
\put(170.0,327.0){\rule[-0.200pt]{2.409pt}{0.400pt}}
\put(1429.0,327.0){\rule[-0.200pt]{2.409pt}{0.400pt}}
\put(170.0,327.0){\rule[-0.200pt]{2.409pt}{0.400pt}}
\put(1429.0,327.0){\rule[-0.200pt]{2.409pt}{0.400pt}}
\put(170.0,327.0){\rule[-0.200pt]{2.409pt}{0.400pt}}
\put(1429.0,327.0){\rule[-0.200pt]{2.409pt}{0.400pt}}
\put(170.0,327.0){\rule[-0.200pt]{2.409pt}{0.400pt}}
\put(1429.0,327.0){\rule[-0.200pt]{2.409pt}{0.400pt}}
\put(170.0,327.0){\rule[-0.200pt]{2.409pt}{0.400pt}}
\put(1429.0,327.0){\rule[-0.200pt]{2.409pt}{0.400pt}}
\put(170.0,327.0){\rule[-0.200pt]{2.409pt}{0.400pt}}
\put(1429.0,327.0){\rule[-0.200pt]{2.409pt}{0.400pt}}
\put(170.0,327.0){\rule[-0.200pt]{2.409pt}{0.400pt}}
\put(1429.0,327.0){\rule[-0.200pt]{2.409pt}{0.400pt}}
\put(170.0,327.0){\rule[-0.200pt]{2.409pt}{0.400pt}}
\put(1429.0,327.0){\rule[-0.200pt]{2.409pt}{0.400pt}}
\put(170.0,327.0){\rule[-0.200pt]{2.409pt}{0.400pt}}
\put(1429.0,327.0){\rule[-0.200pt]{2.409pt}{0.400pt}}
\put(170.0,327.0){\rule[-0.200pt]{2.409pt}{0.400pt}}
\put(1429.0,327.0){\rule[-0.200pt]{2.409pt}{0.400pt}}
\put(170.0,327.0){\rule[-0.200pt]{2.409pt}{0.400pt}}
\put(1429.0,327.0){\rule[-0.200pt]{2.409pt}{0.400pt}}
\put(170.0,327.0){\rule[-0.200pt]{2.409pt}{0.400pt}}
\put(1429.0,327.0){\rule[-0.200pt]{2.409pt}{0.400pt}}
\put(170.0,327.0){\rule[-0.200pt]{2.409pt}{0.400pt}}
\put(1429.0,327.0){\rule[-0.200pt]{2.409pt}{0.400pt}}
\put(170.0,327.0){\rule[-0.200pt]{2.409pt}{0.400pt}}
\put(1429.0,327.0){\rule[-0.200pt]{2.409pt}{0.400pt}}
\put(170.0,327.0){\rule[-0.200pt]{2.409pt}{0.400pt}}
\put(1429.0,327.0){\rule[-0.200pt]{2.409pt}{0.400pt}}
\put(170.0,327.0){\rule[-0.200pt]{2.409pt}{0.400pt}}
\put(1429.0,327.0){\rule[-0.200pt]{2.409pt}{0.400pt}}
\put(170.0,327.0){\rule[-0.200pt]{2.409pt}{0.400pt}}
\put(1429.0,327.0){\rule[-0.200pt]{2.409pt}{0.400pt}}
\put(170.0,327.0){\rule[-0.200pt]{2.409pt}{0.400pt}}
\put(1429.0,327.0){\rule[-0.200pt]{2.409pt}{0.400pt}}
\put(170.0,328.0){\rule[-0.200pt]{2.409pt}{0.400pt}}
\put(1429.0,328.0){\rule[-0.200pt]{2.409pt}{0.400pt}}
\put(170.0,328.0){\rule[-0.200pt]{2.409pt}{0.400pt}}
\put(1429.0,328.0){\rule[-0.200pt]{2.409pt}{0.400pt}}
\put(170.0,328.0){\rule[-0.200pt]{2.409pt}{0.400pt}}
\put(1429.0,328.0){\rule[-0.200pt]{2.409pt}{0.400pt}}
\put(170.0,328.0){\rule[-0.200pt]{2.409pt}{0.400pt}}
\put(1429.0,328.0){\rule[-0.200pt]{2.409pt}{0.400pt}}
\put(170.0,328.0){\rule[-0.200pt]{2.409pt}{0.400pt}}
\put(1429.0,328.0){\rule[-0.200pt]{2.409pt}{0.400pt}}
\put(170.0,328.0){\rule[-0.200pt]{2.409pt}{0.400pt}}
\put(1429.0,328.0){\rule[-0.200pt]{2.409pt}{0.400pt}}
\put(170.0,328.0){\rule[-0.200pt]{2.409pt}{0.400pt}}
\put(1429.0,328.0){\rule[-0.200pt]{2.409pt}{0.400pt}}
\put(170.0,328.0){\rule[-0.200pt]{2.409pt}{0.400pt}}
\put(1429.0,328.0){\rule[-0.200pt]{2.409pt}{0.400pt}}
\put(170.0,328.0){\rule[-0.200pt]{2.409pt}{0.400pt}}
\put(1429.0,328.0){\rule[-0.200pt]{2.409pt}{0.400pt}}
\put(170.0,328.0){\rule[-0.200pt]{2.409pt}{0.400pt}}
\put(1429.0,328.0){\rule[-0.200pt]{2.409pt}{0.400pt}}
\put(170.0,328.0){\rule[-0.200pt]{2.409pt}{0.400pt}}
\put(1429.0,328.0){\rule[-0.200pt]{2.409pt}{0.400pt}}
\put(170.0,328.0){\rule[-0.200pt]{2.409pt}{0.400pt}}
\put(1429.0,328.0){\rule[-0.200pt]{2.409pt}{0.400pt}}
\put(170.0,328.0){\rule[-0.200pt]{2.409pt}{0.400pt}}
\put(1429.0,328.0){\rule[-0.200pt]{2.409pt}{0.400pt}}
\put(170.0,328.0){\rule[-0.200pt]{2.409pt}{0.400pt}}
\put(1429.0,328.0){\rule[-0.200pt]{2.409pt}{0.400pt}}
\put(170.0,328.0){\rule[-0.200pt]{2.409pt}{0.400pt}}
\put(1429.0,328.0){\rule[-0.200pt]{2.409pt}{0.400pt}}
\put(170.0,328.0){\rule[-0.200pt]{2.409pt}{0.400pt}}
\put(1429.0,328.0){\rule[-0.200pt]{2.409pt}{0.400pt}}
\put(170.0,328.0){\rule[-0.200pt]{2.409pt}{0.400pt}}
\put(1429.0,328.0){\rule[-0.200pt]{2.409pt}{0.400pt}}
\put(170.0,328.0){\rule[-0.200pt]{2.409pt}{0.400pt}}
\put(1429.0,328.0){\rule[-0.200pt]{2.409pt}{0.400pt}}
\put(170.0,328.0){\rule[-0.200pt]{2.409pt}{0.400pt}}
\put(1429.0,328.0){\rule[-0.200pt]{2.409pt}{0.400pt}}
\put(170.0,329.0){\rule[-0.200pt]{2.409pt}{0.400pt}}
\put(1429.0,329.0){\rule[-0.200pt]{2.409pt}{0.400pt}}
\put(170.0,329.0){\rule[-0.200pt]{2.409pt}{0.400pt}}
\put(1429.0,329.0){\rule[-0.200pt]{2.409pt}{0.400pt}}
\put(170.0,329.0){\rule[-0.200pt]{2.409pt}{0.400pt}}
\put(1429.0,329.0){\rule[-0.200pt]{2.409pt}{0.400pt}}
\put(170.0,329.0){\rule[-0.200pt]{2.409pt}{0.400pt}}
\put(1429.0,329.0){\rule[-0.200pt]{2.409pt}{0.400pt}}
\put(170.0,329.0){\rule[-0.200pt]{2.409pt}{0.400pt}}
\put(1429.0,329.0){\rule[-0.200pt]{2.409pt}{0.400pt}}
\put(170.0,329.0){\rule[-0.200pt]{2.409pt}{0.400pt}}
\put(1429.0,329.0){\rule[-0.200pt]{2.409pt}{0.400pt}}
\put(170.0,329.0){\rule[-0.200pt]{2.409pt}{0.400pt}}
\put(1429.0,329.0){\rule[-0.200pt]{2.409pt}{0.400pt}}
\put(170.0,329.0){\rule[-0.200pt]{2.409pt}{0.400pt}}
\put(1429.0,329.0){\rule[-0.200pt]{2.409pt}{0.400pt}}
\put(170.0,329.0){\rule[-0.200pt]{2.409pt}{0.400pt}}
\put(1429.0,329.0){\rule[-0.200pt]{2.409pt}{0.400pt}}
\put(170.0,329.0){\rule[-0.200pt]{2.409pt}{0.400pt}}
\put(1429.0,329.0){\rule[-0.200pt]{2.409pt}{0.400pt}}
\put(170.0,329.0){\rule[-0.200pt]{2.409pt}{0.400pt}}
\put(1429.0,329.0){\rule[-0.200pt]{2.409pt}{0.400pt}}
\put(170.0,329.0){\rule[-0.200pt]{2.409pt}{0.400pt}}
\put(1429.0,329.0){\rule[-0.200pt]{2.409pt}{0.400pt}}
\put(170.0,329.0){\rule[-0.200pt]{2.409pt}{0.400pt}}
\put(1429.0,329.0){\rule[-0.200pt]{2.409pt}{0.400pt}}
\put(170.0,329.0){\rule[-0.200pt]{2.409pt}{0.400pt}}
\put(1429.0,329.0){\rule[-0.200pt]{2.409pt}{0.400pt}}
\put(170.0,329.0){\rule[-0.200pt]{2.409pt}{0.400pt}}
\put(1429.0,329.0){\rule[-0.200pt]{2.409pt}{0.400pt}}
\put(170.0,329.0){\rule[-0.200pt]{2.409pt}{0.400pt}}
\put(1429.0,329.0){\rule[-0.200pt]{2.409pt}{0.400pt}}
\put(170.0,329.0){\rule[-0.200pt]{2.409pt}{0.400pt}}
\put(1429.0,329.0){\rule[-0.200pt]{2.409pt}{0.400pt}}
\put(170.0,329.0){\rule[-0.200pt]{2.409pt}{0.400pt}}
\put(1429.0,329.0){\rule[-0.200pt]{2.409pt}{0.400pt}}
\put(170.0,329.0){\rule[-0.200pt]{2.409pt}{0.400pt}}
\put(1429.0,329.0){\rule[-0.200pt]{2.409pt}{0.400pt}}
\put(170.0,330.0){\rule[-0.200pt]{2.409pt}{0.400pt}}
\put(1429.0,330.0){\rule[-0.200pt]{2.409pt}{0.400pt}}
\put(170.0,330.0){\rule[-0.200pt]{2.409pt}{0.400pt}}
\put(1429.0,330.0){\rule[-0.200pt]{2.409pt}{0.400pt}}
\put(170.0,330.0){\rule[-0.200pt]{2.409pt}{0.400pt}}
\put(1429.0,330.0){\rule[-0.200pt]{2.409pt}{0.400pt}}
\put(170.0,330.0){\rule[-0.200pt]{2.409pt}{0.400pt}}
\put(1429.0,330.0){\rule[-0.200pt]{2.409pt}{0.400pt}}
\put(170.0,330.0){\rule[-0.200pt]{2.409pt}{0.400pt}}
\put(1429.0,330.0){\rule[-0.200pt]{2.409pt}{0.400pt}}
\put(170.0,330.0){\rule[-0.200pt]{2.409pt}{0.400pt}}
\put(1429.0,330.0){\rule[-0.200pt]{2.409pt}{0.400pt}}
\put(170.0,330.0){\rule[-0.200pt]{2.409pt}{0.400pt}}
\put(1429.0,330.0){\rule[-0.200pt]{2.409pt}{0.400pt}}
\put(170.0,330.0){\rule[-0.200pt]{2.409pt}{0.400pt}}
\put(1429.0,330.0){\rule[-0.200pt]{2.409pt}{0.400pt}}
\put(170.0,330.0){\rule[-0.200pt]{2.409pt}{0.400pt}}
\put(1429.0,330.0){\rule[-0.200pt]{2.409pt}{0.400pt}}
\put(170.0,330.0){\rule[-0.200pt]{2.409pt}{0.400pt}}
\put(1429.0,330.0){\rule[-0.200pt]{2.409pt}{0.400pt}}
\put(170.0,330.0){\rule[-0.200pt]{2.409pt}{0.400pt}}
\put(1429.0,330.0){\rule[-0.200pt]{2.409pt}{0.400pt}}
\put(170.0,330.0){\rule[-0.200pt]{2.409pt}{0.400pt}}
\put(1429.0,330.0){\rule[-0.200pt]{2.409pt}{0.400pt}}
\put(170.0,330.0){\rule[-0.200pt]{2.409pt}{0.400pt}}
\put(1429.0,330.0){\rule[-0.200pt]{2.409pt}{0.400pt}}
\put(170.0,330.0){\rule[-0.200pt]{2.409pt}{0.400pt}}
\put(1429.0,330.0){\rule[-0.200pt]{2.409pt}{0.400pt}}
\put(170.0,330.0){\rule[-0.200pt]{2.409pt}{0.400pt}}
\put(1429.0,330.0){\rule[-0.200pt]{2.409pt}{0.400pt}}
\put(170.0,330.0){\rule[-0.200pt]{2.409pt}{0.400pt}}
\put(1429.0,330.0){\rule[-0.200pt]{2.409pt}{0.400pt}}
\put(170.0,330.0){\rule[-0.200pt]{2.409pt}{0.400pt}}
\put(1429.0,330.0){\rule[-0.200pt]{2.409pt}{0.400pt}}
\put(170.0,330.0){\rule[-0.200pt]{2.409pt}{0.400pt}}
\put(1429.0,330.0){\rule[-0.200pt]{2.409pt}{0.400pt}}
\put(170.0,330.0){\rule[-0.200pt]{2.409pt}{0.400pt}}
\put(1429.0,330.0){\rule[-0.200pt]{2.409pt}{0.400pt}}
\put(170.0,330.0){\rule[-0.200pt]{2.409pt}{0.400pt}}
\put(1429.0,330.0){\rule[-0.200pt]{2.409pt}{0.400pt}}
\put(170.0,331.0){\rule[-0.200pt]{2.409pt}{0.400pt}}
\put(1429.0,331.0){\rule[-0.200pt]{2.409pt}{0.400pt}}
\put(170.0,331.0){\rule[-0.200pt]{2.409pt}{0.400pt}}
\put(1429.0,331.0){\rule[-0.200pt]{2.409pt}{0.400pt}}
\put(170.0,331.0){\rule[-0.200pt]{2.409pt}{0.400pt}}
\put(1429.0,331.0){\rule[-0.200pt]{2.409pt}{0.400pt}}
\put(170.0,331.0){\rule[-0.200pt]{2.409pt}{0.400pt}}
\put(1429.0,331.0){\rule[-0.200pt]{2.409pt}{0.400pt}}
\put(170.0,331.0){\rule[-0.200pt]{2.409pt}{0.400pt}}
\put(1429.0,331.0){\rule[-0.200pt]{2.409pt}{0.400pt}}
\put(170.0,331.0){\rule[-0.200pt]{2.409pt}{0.400pt}}
\put(1429.0,331.0){\rule[-0.200pt]{2.409pt}{0.400pt}}
\put(170.0,331.0){\rule[-0.200pt]{2.409pt}{0.400pt}}
\put(1429.0,331.0){\rule[-0.200pt]{2.409pt}{0.400pt}}
\put(170.0,331.0){\rule[-0.200pt]{2.409pt}{0.400pt}}
\put(1429.0,331.0){\rule[-0.200pt]{2.409pt}{0.400pt}}
\put(170.0,331.0){\rule[-0.200pt]{2.409pt}{0.400pt}}
\put(1429.0,331.0){\rule[-0.200pt]{2.409pt}{0.400pt}}
\put(170.0,331.0){\rule[-0.200pt]{2.409pt}{0.400pt}}
\put(1429.0,331.0){\rule[-0.200pt]{2.409pt}{0.400pt}}
\put(170.0,331.0){\rule[-0.200pt]{2.409pt}{0.400pt}}
\put(1429.0,331.0){\rule[-0.200pt]{2.409pt}{0.400pt}}
\put(170.0,331.0){\rule[-0.200pt]{2.409pt}{0.400pt}}
\put(1429.0,331.0){\rule[-0.200pt]{2.409pt}{0.400pt}}
\put(170.0,331.0){\rule[-0.200pt]{2.409pt}{0.400pt}}
\put(1429.0,331.0){\rule[-0.200pt]{2.409pt}{0.400pt}}
\put(170.0,331.0){\rule[-0.200pt]{2.409pt}{0.400pt}}
\put(1429.0,331.0){\rule[-0.200pt]{2.409pt}{0.400pt}}
\put(170.0,331.0){\rule[-0.200pt]{2.409pt}{0.400pt}}
\put(1429.0,331.0){\rule[-0.200pt]{2.409pt}{0.400pt}}
\put(170.0,331.0){\rule[-0.200pt]{2.409pt}{0.400pt}}
\put(1429.0,331.0){\rule[-0.200pt]{2.409pt}{0.400pt}}
\put(170.0,331.0){\rule[-0.200pt]{2.409pt}{0.400pt}}
\put(1429.0,331.0){\rule[-0.200pt]{2.409pt}{0.400pt}}
\put(170.0,331.0){\rule[-0.200pt]{2.409pt}{0.400pt}}
\put(1429.0,331.0){\rule[-0.200pt]{2.409pt}{0.400pt}}
\put(170.0,331.0){\rule[-0.200pt]{2.409pt}{0.400pt}}
\put(1429.0,331.0){\rule[-0.200pt]{2.409pt}{0.400pt}}
\put(170.0,331.0){\rule[-0.200pt]{2.409pt}{0.400pt}}
\put(1429.0,331.0){\rule[-0.200pt]{2.409pt}{0.400pt}}
\put(170.0,331.0){\rule[-0.200pt]{2.409pt}{0.400pt}}
\put(1429.0,331.0){\rule[-0.200pt]{2.409pt}{0.400pt}}
\put(170.0,332.0){\rule[-0.200pt]{2.409pt}{0.400pt}}
\put(1429.0,332.0){\rule[-0.200pt]{2.409pt}{0.400pt}}
\put(170.0,332.0){\rule[-0.200pt]{2.409pt}{0.400pt}}
\put(1429.0,332.0){\rule[-0.200pt]{2.409pt}{0.400pt}}
\put(170.0,332.0){\rule[-0.200pt]{2.409pt}{0.400pt}}
\put(1429.0,332.0){\rule[-0.200pt]{2.409pt}{0.400pt}}
\put(170.0,332.0){\rule[-0.200pt]{2.409pt}{0.400pt}}
\put(1429.0,332.0){\rule[-0.200pt]{2.409pt}{0.400pt}}
\put(170.0,332.0){\rule[-0.200pt]{2.409pt}{0.400pt}}
\put(1429.0,332.0){\rule[-0.200pt]{2.409pt}{0.400pt}}
\put(170.0,332.0){\rule[-0.200pt]{2.409pt}{0.400pt}}
\put(1429.0,332.0){\rule[-0.200pt]{2.409pt}{0.400pt}}
\put(170.0,332.0){\rule[-0.200pt]{2.409pt}{0.400pt}}
\put(1429.0,332.0){\rule[-0.200pt]{2.409pt}{0.400pt}}
\put(170.0,332.0){\rule[-0.200pt]{2.409pt}{0.400pt}}
\put(1429.0,332.0){\rule[-0.200pt]{2.409pt}{0.400pt}}
\put(170.0,332.0){\rule[-0.200pt]{2.409pt}{0.400pt}}
\put(1429.0,332.0){\rule[-0.200pt]{2.409pt}{0.400pt}}
\put(170.0,332.0){\rule[-0.200pt]{2.409pt}{0.400pt}}
\put(1429.0,332.0){\rule[-0.200pt]{2.409pt}{0.400pt}}
\put(170.0,332.0){\rule[-0.200pt]{2.409pt}{0.400pt}}
\put(1429.0,332.0){\rule[-0.200pt]{2.409pt}{0.400pt}}
\put(170.0,332.0){\rule[-0.200pt]{2.409pt}{0.400pt}}
\put(1429.0,332.0){\rule[-0.200pt]{2.409pt}{0.400pt}}
\put(170.0,332.0){\rule[-0.200pt]{2.409pt}{0.400pt}}
\put(1429.0,332.0){\rule[-0.200pt]{2.409pt}{0.400pt}}
\put(170.0,332.0){\rule[-0.200pt]{2.409pt}{0.400pt}}
\put(1429.0,332.0){\rule[-0.200pt]{2.409pt}{0.400pt}}
\put(170.0,332.0){\rule[-0.200pt]{2.409pt}{0.400pt}}
\put(1429.0,332.0){\rule[-0.200pt]{2.409pt}{0.400pt}}
\put(170.0,332.0){\rule[-0.200pt]{2.409pt}{0.400pt}}
\put(1429.0,332.0){\rule[-0.200pt]{2.409pt}{0.400pt}}
\put(170.0,332.0){\rule[-0.200pt]{2.409pt}{0.400pt}}
\put(1429.0,332.0){\rule[-0.200pt]{2.409pt}{0.400pt}}
\put(170.0,332.0){\rule[-0.200pt]{2.409pt}{0.400pt}}
\put(1429.0,332.0){\rule[-0.200pt]{2.409pt}{0.400pt}}
\put(170.0,332.0){\rule[-0.200pt]{2.409pt}{0.400pt}}
\put(1429.0,332.0){\rule[-0.200pt]{2.409pt}{0.400pt}}
\put(170.0,332.0){\rule[-0.200pt]{2.409pt}{0.400pt}}
\put(1429.0,332.0){\rule[-0.200pt]{2.409pt}{0.400pt}}
\put(170.0,332.0){\rule[-0.200pt]{2.409pt}{0.400pt}}
\put(1429.0,332.0){\rule[-0.200pt]{2.409pt}{0.400pt}}
\put(170.0,333.0){\rule[-0.200pt]{2.409pt}{0.400pt}}
\put(1429.0,333.0){\rule[-0.200pt]{2.409pt}{0.400pt}}
\put(170.0,333.0){\rule[-0.200pt]{2.409pt}{0.400pt}}
\put(1429.0,333.0){\rule[-0.200pt]{2.409pt}{0.400pt}}
\put(170.0,333.0){\rule[-0.200pt]{2.409pt}{0.400pt}}
\put(1429.0,333.0){\rule[-0.200pt]{2.409pt}{0.400pt}}
\put(170.0,333.0){\rule[-0.200pt]{2.409pt}{0.400pt}}
\put(1429.0,333.0){\rule[-0.200pt]{2.409pt}{0.400pt}}
\put(170.0,333.0){\rule[-0.200pt]{2.409pt}{0.400pt}}
\put(1429.0,333.0){\rule[-0.200pt]{2.409pt}{0.400pt}}
\put(170.0,333.0){\rule[-0.200pt]{2.409pt}{0.400pt}}
\put(1429.0,333.0){\rule[-0.200pt]{2.409pt}{0.400pt}}
\put(170.0,333.0){\rule[-0.200pt]{2.409pt}{0.400pt}}
\put(1429.0,333.0){\rule[-0.200pt]{2.409pt}{0.400pt}}
\put(170.0,333.0){\rule[-0.200pt]{2.409pt}{0.400pt}}
\put(1429.0,333.0){\rule[-0.200pt]{2.409pt}{0.400pt}}
\put(170.0,333.0){\rule[-0.200pt]{2.409pt}{0.400pt}}
\put(1429.0,333.0){\rule[-0.200pt]{2.409pt}{0.400pt}}
\put(170.0,333.0){\rule[-0.200pt]{2.409pt}{0.400pt}}
\put(1429.0,333.0){\rule[-0.200pt]{2.409pt}{0.400pt}}
\put(170.0,333.0){\rule[-0.200pt]{2.409pt}{0.400pt}}
\put(1429.0,333.0){\rule[-0.200pt]{2.409pt}{0.400pt}}
\put(170.0,333.0){\rule[-0.200pt]{2.409pt}{0.400pt}}
\put(1429.0,333.0){\rule[-0.200pt]{2.409pt}{0.400pt}}
\put(170.0,333.0){\rule[-0.200pt]{2.409pt}{0.400pt}}
\put(1429.0,333.0){\rule[-0.200pt]{2.409pt}{0.400pt}}
\put(170.0,333.0){\rule[-0.200pt]{2.409pt}{0.400pt}}
\put(1429.0,333.0){\rule[-0.200pt]{2.409pt}{0.400pt}}
\put(170.0,333.0){\rule[-0.200pt]{2.409pt}{0.400pt}}
\put(1429.0,333.0){\rule[-0.200pt]{2.409pt}{0.400pt}}
\put(170.0,333.0){\rule[-0.200pt]{2.409pt}{0.400pt}}
\put(1429.0,333.0){\rule[-0.200pt]{2.409pt}{0.400pt}}
\put(170.0,333.0){\rule[-0.200pt]{2.409pt}{0.400pt}}
\put(1429.0,333.0){\rule[-0.200pt]{2.409pt}{0.400pt}}
\put(170.0,333.0){\rule[-0.200pt]{2.409pt}{0.400pt}}
\put(1429.0,333.0){\rule[-0.200pt]{2.409pt}{0.400pt}}
\put(170.0,333.0){\rule[-0.200pt]{2.409pt}{0.400pt}}
\put(1429.0,333.0){\rule[-0.200pt]{2.409pt}{0.400pt}}
\put(170.0,333.0){\rule[-0.200pt]{2.409pt}{0.400pt}}
\put(1429.0,333.0){\rule[-0.200pt]{2.409pt}{0.400pt}}
\put(170.0,333.0){\rule[-0.200pt]{2.409pt}{0.400pt}}
\put(1429.0,333.0){\rule[-0.200pt]{2.409pt}{0.400pt}}
\put(170.0,334.0){\rule[-0.200pt]{2.409pt}{0.400pt}}
\put(1429.0,334.0){\rule[-0.200pt]{2.409pt}{0.400pt}}
\put(170.0,334.0){\rule[-0.200pt]{2.409pt}{0.400pt}}
\put(1429.0,334.0){\rule[-0.200pt]{2.409pt}{0.400pt}}
\put(170.0,334.0){\rule[-0.200pt]{2.409pt}{0.400pt}}
\put(1429.0,334.0){\rule[-0.200pt]{2.409pt}{0.400pt}}
\put(170.0,334.0){\rule[-0.200pt]{2.409pt}{0.400pt}}
\put(1429.0,334.0){\rule[-0.200pt]{2.409pt}{0.400pt}}
\put(170.0,334.0){\rule[-0.200pt]{2.409pt}{0.400pt}}
\put(1429.0,334.0){\rule[-0.200pt]{2.409pt}{0.400pt}}
\put(170.0,334.0){\rule[-0.200pt]{2.409pt}{0.400pt}}
\put(1429.0,334.0){\rule[-0.200pt]{2.409pt}{0.400pt}}
\put(170.0,334.0){\rule[-0.200pt]{2.409pt}{0.400pt}}
\put(1429.0,334.0){\rule[-0.200pt]{2.409pt}{0.400pt}}
\put(170.0,334.0){\rule[-0.200pt]{2.409pt}{0.400pt}}
\put(1429.0,334.0){\rule[-0.200pt]{2.409pt}{0.400pt}}
\put(170.0,334.0){\rule[-0.200pt]{2.409pt}{0.400pt}}
\put(1429.0,334.0){\rule[-0.200pt]{2.409pt}{0.400pt}}
\put(170.0,334.0){\rule[-0.200pt]{2.409pt}{0.400pt}}
\put(1429.0,334.0){\rule[-0.200pt]{2.409pt}{0.400pt}}
\put(170.0,334.0){\rule[-0.200pt]{2.409pt}{0.400pt}}
\put(1429.0,334.0){\rule[-0.200pt]{2.409pt}{0.400pt}}
\put(170.0,334.0){\rule[-0.200pt]{2.409pt}{0.400pt}}
\put(1429.0,334.0){\rule[-0.200pt]{2.409pt}{0.400pt}}
\put(170.0,334.0){\rule[-0.200pt]{2.409pt}{0.400pt}}
\put(1429.0,334.0){\rule[-0.200pt]{2.409pt}{0.400pt}}
\put(170.0,334.0){\rule[-0.200pt]{2.409pt}{0.400pt}}
\put(1429.0,334.0){\rule[-0.200pt]{2.409pt}{0.400pt}}
\put(170.0,334.0){\rule[-0.200pt]{2.409pt}{0.400pt}}
\put(1429.0,334.0){\rule[-0.200pt]{2.409pt}{0.400pt}}
\put(170.0,334.0){\rule[-0.200pt]{2.409pt}{0.400pt}}
\put(1429.0,334.0){\rule[-0.200pt]{2.409pt}{0.400pt}}
\put(170.0,334.0){\rule[-0.200pt]{2.409pt}{0.400pt}}
\put(1429.0,334.0){\rule[-0.200pt]{2.409pt}{0.400pt}}
\put(170.0,334.0){\rule[-0.200pt]{2.409pt}{0.400pt}}
\put(1429.0,334.0){\rule[-0.200pt]{2.409pt}{0.400pt}}
\put(170.0,334.0){\rule[-0.200pt]{2.409pt}{0.400pt}}
\put(1429.0,334.0){\rule[-0.200pt]{2.409pt}{0.400pt}}
\put(170.0,334.0){\rule[-0.200pt]{2.409pt}{0.400pt}}
\put(1429.0,334.0){\rule[-0.200pt]{2.409pt}{0.400pt}}
\put(170.0,334.0){\rule[-0.200pt]{2.409pt}{0.400pt}}
\put(1429.0,334.0){\rule[-0.200pt]{2.409pt}{0.400pt}}
\put(170.0,334.0){\rule[-0.200pt]{2.409pt}{0.400pt}}
\put(1429.0,334.0){\rule[-0.200pt]{2.409pt}{0.400pt}}
\put(170.0,335.0){\rule[-0.200pt]{2.409pt}{0.400pt}}
\put(1429.0,335.0){\rule[-0.200pt]{2.409pt}{0.400pt}}
\put(170.0,335.0){\rule[-0.200pt]{2.409pt}{0.400pt}}
\put(1429.0,335.0){\rule[-0.200pt]{2.409pt}{0.400pt}}
\put(170.0,335.0){\rule[-0.200pt]{2.409pt}{0.400pt}}
\put(1429.0,335.0){\rule[-0.200pt]{2.409pt}{0.400pt}}
\put(170.0,335.0){\rule[-0.200pt]{2.409pt}{0.400pt}}
\put(1429.0,335.0){\rule[-0.200pt]{2.409pt}{0.400pt}}
\put(170.0,335.0){\rule[-0.200pt]{2.409pt}{0.400pt}}
\put(1429.0,335.0){\rule[-0.200pt]{2.409pt}{0.400pt}}
\put(170.0,335.0){\rule[-0.200pt]{2.409pt}{0.400pt}}
\put(1429.0,335.0){\rule[-0.200pt]{2.409pt}{0.400pt}}
\put(170.0,335.0){\rule[-0.200pt]{2.409pt}{0.400pt}}
\put(1429.0,335.0){\rule[-0.200pt]{2.409pt}{0.400pt}}
\put(170.0,335.0){\rule[-0.200pt]{2.409pt}{0.400pt}}
\put(1429.0,335.0){\rule[-0.200pt]{2.409pt}{0.400pt}}
\put(170.0,335.0){\rule[-0.200pt]{2.409pt}{0.400pt}}
\put(1429.0,335.0){\rule[-0.200pt]{2.409pt}{0.400pt}}
\put(170.0,335.0){\rule[-0.200pt]{2.409pt}{0.400pt}}
\put(1429.0,335.0){\rule[-0.200pt]{2.409pt}{0.400pt}}
\put(170.0,335.0){\rule[-0.200pt]{2.409pt}{0.400pt}}
\put(1429.0,335.0){\rule[-0.200pt]{2.409pt}{0.400pt}}
\put(170.0,335.0){\rule[-0.200pt]{2.409pt}{0.400pt}}
\put(1429.0,335.0){\rule[-0.200pt]{2.409pt}{0.400pt}}
\put(170.0,335.0){\rule[-0.200pt]{2.409pt}{0.400pt}}
\put(1429.0,335.0){\rule[-0.200pt]{2.409pt}{0.400pt}}
\put(170.0,335.0){\rule[-0.200pt]{2.409pt}{0.400pt}}
\put(1429.0,335.0){\rule[-0.200pt]{2.409pt}{0.400pt}}
\put(170.0,335.0){\rule[-0.200pt]{2.409pt}{0.400pt}}
\put(1429.0,335.0){\rule[-0.200pt]{2.409pt}{0.400pt}}
\put(170.0,335.0){\rule[-0.200pt]{2.409pt}{0.400pt}}
\put(1429.0,335.0){\rule[-0.200pt]{2.409pt}{0.400pt}}
\put(170.0,335.0){\rule[-0.200pt]{2.409pt}{0.400pt}}
\put(1429.0,335.0){\rule[-0.200pt]{2.409pt}{0.400pt}}
\put(170.0,335.0){\rule[-0.200pt]{2.409pt}{0.400pt}}
\put(1429.0,335.0){\rule[-0.200pt]{2.409pt}{0.400pt}}
\put(170.0,335.0){\rule[-0.200pt]{2.409pt}{0.400pt}}
\put(1429.0,335.0){\rule[-0.200pt]{2.409pt}{0.400pt}}
\put(170.0,335.0){\rule[-0.200pt]{2.409pt}{0.400pt}}
\put(1429.0,335.0){\rule[-0.200pt]{2.409pt}{0.400pt}}
\put(170.0,335.0){\rule[-0.200pt]{2.409pt}{0.400pt}}
\put(1429.0,335.0){\rule[-0.200pt]{2.409pt}{0.400pt}}
\put(170.0,335.0){\rule[-0.200pt]{2.409pt}{0.400pt}}
\put(1429.0,335.0){\rule[-0.200pt]{2.409pt}{0.400pt}}
\put(170.0,335.0){\rule[-0.200pt]{2.409pt}{0.400pt}}
\put(1429.0,335.0){\rule[-0.200pt]{2.409pt}{0.400pt}}
\put(170.0,336.0){\rule[-0.200pt]{2.409pt}{0.400pt}}
\put(1429.0,336.0){\rule[-0.200pt]{2.409pt}{0.400pt}}
\put(170.0,336.0){\rule[-0.200pt]{2.409pt}{0.400pt}}
\put(1429.0,336.0){\rule[-0.200pt]{2.409pt}{0.400pt}}
\put(170.0,336.0){\rule[-0.200pt]{2.409pt}{0.400pt}}
\put(1429.0,336.0){\rule[-0.200pt]{2.409pt}{0.400pt}}
\put(170.0,336.0){\rule[-0.200pt]{2.409pt}{0.400pt}}
\put(1429.0,336.0){\rule[-0.200pt]{2.409pt}{0.400pt}}
\put(170.0,336.0){\rule[-0.200pt]{2.409pt}{0.400pt}}
\put(1429.0,336.0){\rule[-0.200pt]{2.409pt}{0.400pt}}
\put(170.0,336.0){\rule[-0.200pt]{2.409pt}{0.400pt}}
\put(1429.0,336.0){\rule[-0.200pt]{2.409pt}{0.400pt}}
\put(170.0,336.0){\rule[-0.200pt]{2.409pt}{0.400pt}}
\put(1429.0,336.0){\rule[-0.200pt]{2.409pt}{0.400pt}}
\put(170.0,336.0){\rule[-0.200pt]{2.409pt}{0.400pt}}
\put(1429.0,336.0){\rule[-0.200pt]{2.409pt}{0.400pt}}
\put(170.0,336.0){\rule[-0.200pt]{2.409pt}{0.400pt}}
\put(1429.0,336.0){\rule[-0.200pt]{2.409pt}{0.400pt}}
\put(170.0,336.0){\rule[-0.200pt]{2.409pt}{0.400pt}}
\put(1429.0,336.0){\rule[-0.200pt]{2.409pt}{0.400pt}}
\put(170.0,336.0){\rule[-0.200pt]{2.409pt}{0.400pt}}
\put(1429.0,336.0){\rule[-0.200pt]{2.409pt}{0.400pt}}
\put(170.0,336.0){\rule[-0.200pt]{2.409pt}{0.400pt}}
\put(1429.0,336.0){\rule[-0.200pt]{2.409pt}{0.400pt}}
\put(170.0,336.0){\rule[-0.200pt]{2.409pt}{0.400pt}}
\put(1429.0,336.0){\rule[-0.200pt]{2.409pt}{0.400pt}}
\put(170.0,336.0){\rule[-0.200pt]{2.409pt}{0.400pt}}
\put(1429.0,336.0){\rule[-0.200pt]{2.409pt}{0.400pt}}
\put(170.0,336.0){\rule[-0.200pt]{2.409pt}{0.400pt}}
\put(1429.0,336.0){\rule[-0.200pt]{2.409pt}{0.400pt}}
\put(170.0,336.0){\rule[-0.200pt]{2.409pt}{0.400pt}}
\put(1429.0,336.0){\rule[-0.200pt]{2.409pt}{0.400pt}}
\put(170.0,336.0){\rule[-0.200pt]{2.409pt}{0.400pt}}
\put(1429.0,336.0){\rule[-0.200pt]{2.409pt}{0.400pt}}
\put(170.0,336.0){\rule[-0.200pt]{2.409pt}{0.400pt}}
\put(1429.0,336.0){\rule[-0.200pt]{2.409pt}{0.400pt}}
\put(170.0,336.0){\rule[-0.200pt]{2.409pt}{0.400pt}}
\put(1429.0,336.0){\rule[-0.200pt]{2.409pt}{0.400pt}}
\put(170.0,336.0){\rule[-0.200pt]{2.409pt}{0.400pt}}
\put(1429.0,336.0){\rule[-0.200pt]{2.409pt}{0.400pt}}
\put(170.0,336.0){\rule[-0.200pt]{2.409pt}{0.400pt}}
\put(1429.0,336.0){\rule[-0.200pt]{2.409pt}{0.400pt}}
\put(170.0,336.0){\rule[-0.200pt]{2.409pt}{0.400pt}}
\put(1429.0,336.0){\rule[-0.200pt]{2.409pt}{0.400pt}}
\put(170.0,336.0){\rule[-0.200pt]{2.409pt}{0.400pt}}
\put(1429.0,336.0){\rule[-0.200pt]{2.409pt}{0.400pt}}
\put(170.0,337.0){\rule[-0.200pt]{2.409pt}{0.400pt}}
\put(1429.0,337.0){\rule[-0.200pt]{2.409pt}{0.400pt}}
\put(170.0,337.0){\rule[-0.200pt]{2.409pt}{0.400pt}}
\put(1429.0,337.0){\rule[-0.200pt]{2.409pt}{0.400pt}}
\put(170.0,337.0){\rule[-0.200pt]{2.409pt}{0.400pt}}
\put(1429.0,337.0){\rule[-0.200pt]{2.409pt}{0.400pt}}
\put(170.0,337.0){\rule[-0.200pt]{2.409pt}{0.400pt}}
\put(1429.0,337.0){\rule[-0.200pt]{2.409pt}{0.400pt}}
\put(170.0,337.0){\rule[-0.200pt]{2.409pt}{0.400pt}}
\put(1429.0,337.0){\rule[-0.200pt]{2.409pt}{0.400pt}}
\put(170.0,337.0){\rule[-0.200pt]{2.409pt}{0.400pt}}
\put(1429.0,337.0){\rule[-0.200pt]{2.409pt}{0.400pt}}
\put(170.0,337.0){\rule[-0.200pt]{2.409pt}{0.400pt}}
\put(1429.0,337.0){\rule[-0.200pt]{2.409pt}{0.400pt}}
\put(170.0,337.0){\rule[-0.200pt]{2.409pt}{0.400pt}}
\put(1429.0,337.0){\rule[-0.200pt]{2.409pt}{0.400pt}}
\put(170.0,337.0){\rule[-0.200pt]{2.409pt}{0.400pt}}
\put(1429.0,337.0){\rule[-0.200pt]{2.409pt}{0.400pt}}
\put(170.0,337.0){\rule[-0.200pt]{2.409pt}{0.400pt}}
\put(1429.0,337.0){\rule[-0.200pt]{2.409pt}{0.400pt}}
\put(170.0,337.0){\rule[-0.200pt]{2.409pt}{0.400pt}}
\put(1429.0,337.0){\rule[-0.200pt]{2.409pt}{0.400pt}}
\put(170.0,337.0){\rule[-0.200pt]{2.409pt}{0.400pt}}
\put(1429.0,337.0){\rule[-0.200pt]{2.409pt}{0.400pt}}
\put(170.0,337.0){\rule[-0.200pt]{2.409pt}{0.400pt}}
\put(1429.0,337.0){\rule[-0.200pt]{2.409pt}{0.400pt}}
\put(170.0,337.0){\rule[-0.200pt]{2.409pt}{0.400pt}}
\put(1429.0,337.0){\rule[-0.200pt]{2.409pt}{0.400pt}}
\put(170.0,337.0){\rule[-0.200pt]{2.409pt}{0.400pt}}
\put(1429.0,337.0){\rule[-0.200pt]{2.409pt}{0.400pt}}
\put(170.0,337.0){\rule[-0.200pt]{2.409pt}{0.400pt}}
\put(1429.0,337.0){\rule[-0.200pt]{2.409pt}{0.400pt}}
\put(170.0,337.0){\rule[-0.200pt]{2.409pt}{0.400pt}}
\put(1429.0,337.0){\rule[-0.200pt]{2.409pt}{0.400pt}}
\put(170.0,337.0){\rule[-0.200pt]{2.409pt}{0.400pt}}
\put(1429.0,337.0){\rule[-0.200pt]{2.409pt}{0.400pt}}
\put(170.0,337.0){\rule[-0.200pt]{2.409pt}{0.400pt}}
\put(1429.0,337.0){\rule[-0.200pt]{2.409pt}{0.400pt}}
\put(170.0,337.0){\rule[-0.200pt]{2.409pt}{0.400pt}}
\put(1429.0,337.0){\rule[-0.200pt]{2.409pt}{0.400pt}}
\put(170.0,337.0){\rule[-0.200pt]{2.409pt}{0.400pt}}
\put(1429.0,337.0){\rule[-0.200pt]{2.409pt}{0.400pt}}
\put(170.0,337.0){\rule[-0.200pt]{2.409pt}{0.400pt}}
\put(1429.0,337.0){\rule[-0.200pt]{2.409pt}{0.400pt}}
\put(170.0,337.0){\rule[-0.200pt]{2.409pt}{0.400pt}}
\put(1429.0,337.0){\rule[-0.200pt]{2.409pt}{0.400pt}}
\put(170.0,337.0){\rule[-0.200pt]{2.409pt}{0.400pt}}
\put(1429.0,337.0){\rule[-0.200pt]{2.409pt}{0.400pt}}
\put(170.0,338.0){\rule[-0.200pt]{2.409pt}{0.400pt}}
\put(1429.0,338.0){\rule[-0.200pt]{2.409pt}{0.400pt}}
\put(170.0,338.0){\rule[-0.200pt]{2.409pt}{0.400pt}}
\put(1429.0,338.0){\rule[-0.200pt]{2.409pt}{0.400pt}}
\put(170.0,338.0){\rule[-0.200pt]{2.409pt}{0.400pt}}
\put(1429.0,338.0){\rule[-0.200pt]{2.409pt}{0.400pt}}
\put(170.0,338.0){\rule[-0.200pt]{2.409pt}{0.400pt}}
\put(1429.0,338.0){\rule[-0.200pt]{2.409pt}{0.400pt}}
\put(170.0,338.0){\rule[-0.200pt]{2.409pt}{0.400pt}}
\put(1429.0,338.0){\rule[-0.200pt]{2.409pt}{0.400pt}}
\put(170.0,338.0){\rule[-0.200pt]{2.409pt}{0.400pt}}
\put(1429.0,338.0){\rule[-0.200pt]{2.409pt}{0.400pt}}
\put(170.0,338.0){\rule[-0.200pt]{2.409pt}{0.400pt}}
\put(1429.0,338.0){\rule[-0.200pt]{2.409pt}{0.400pt}}
\put(170.0,338.0){\rule[-0.200pt]{2.409pt}{0.400pt}}
\put(1429.0,338.0){\rule[-0.200pt]{2.409pt}{0.400pt}}
\put(170.0,338.0){\rule[-0.200pt]{2.409pt}{0.400pt}}
\put(1429.0,338.0){\rule[-0.200pt]{2.409pt}{0.400pt}}
\put(170.0,338.0){\rule[-0.200pt]{2.409pt}{0.400pt}}
\put(1429.0,338.0){\rule[-0.200pt]{2.409pt}{0.400pt}}
\put(170.0,338.0){\rule[-0.200pt]{2.409pt}{0.400pt}}
\put(1429.0,338.0){\rule[-0.200pt]{2.409pt}{0.400pt}}
\put(170.0,338.0){\rule[-0.200pt]{2.409pt}{0.400pt}}
\put(1429.0,338.0){\rule[-0.200pt]{2.409pt}{0.400pt}}
\put(170.0,338.0){\rule[-0.200pt]{2.409pt}{0.400pt}}
\put(1429.0,338.0){\rule[-0.200pt]{2.409pt}{0.400pt}}
\put(170.0,338.0){\rule[-0.200pt]{2.409pt}{0.400pt}}
\put(1429.0,338.0){\rule[-0.200pt]{2.409pt}{0.400pt}}
\put(170.0,338.0){\rule[-0.200pt]{2.409pt}{0.400pt}}
\put(1429.0,338.0){\rule[-0.200pt]{2.409pt}{0.400pt}}
\put(170.0,338.0){\rule[-0.200pt]{2.409pt}{0.400pt}}
\put(1429.0,338.0){\rule[-0.200pt]{2.409pt}{0.400pt}}
\put(170.0,338.0){\rule[-0.200pt]{2.409pt}{0.400pt}}
\put(1429.0,338.0){\rule[-0.200pt]{2.409pt}{0.400pt}}
\put(170.0,338.0){\rule[-0.200pt]{2.409pt}{0.400pt}}
\put(1429.0,338.0){\rule[-0.200pt]{2.409pt}{0.400pt}}
\put(170.0,338.0){\rule[-0.200pt]{2.409pt}{0.400pt}}
\put(1429.0,338.0){\rule[-0.200pt]{2.409pt}{0.400pt}}
\put(170.0,338.0){\rule[-0.200pt]{2.409pt}{0.400pt}}
\put(1429.0,338.0){\rule[-0.200pt]{2.409pt}{0.400pt}}
\put(170.0,338.0){\rule[-0.200pt]{2.409pt}{0.400pt}}
\put(1429.0,338.0){\rule[-0.200pt]{2.409pt}{0.400pt}}
\put(170.0,338.0){\rule[-0.200pt]{2.409pt}{0.400pt}}
\put(1429.0,338.0){\rule[-0.200pt]{2.409pt}{0.400pt}}
\put(170.0,338.0){\rule[-0.200pt]{2.409pt}{0.400pt}}
\put(1429.0,338.0){\rule[-0.200pt]{2.409pt}{0.400pt}}
\put(170.0,338.0){\rule[-0.200pt]{2.409pt}{0.400pt}}
\put(1429.0,338.0){\rule[-0.200pt]{2.409pt}{0.400pt}}
\put(170.0,338.0){\rule[-0.200pt]{2.409pt}{0.400pt}}
\put(1429.0,338.0){\rule[-0.200pt]{2.409pt}{0.400pt}}
\put(170.0,339.0){\rule[-0.200pt]{2.409pt}{0.400pt}}
\put(1429.0,339.0){\rule[-0.200pt]{2.409pt}{0.400pt}}
\put(170.0,339.0){\rule[-0.200pt]{2.409pt}{0.400pt}}
\put(1429.0,339.0){\rule[-0.200pt]{2.409pt}{0.400pt}}
\put(170.0,339.0){\rule[-0.200pt]{2.409pt}{0.400pt}}
\put(1429.0,339.0){\rule[-0.200pt]{2.409pt}{0.400pt}}
\put(170.0,339.0){\rule[-0.200pt]{2.409pt}{0.400pt}}
\put(1429.0,339.0){\rule[-0.200pt]{2.409pt}{0.400pt}}
\put(170.0,339.0){\rule[-0.200pt]{2.409pt}{0.400pt}}
\put(1429.0,339.0){\rule[-0.200pt]{2.409pt}{0.400pt}}
\put(170.0,339.0){\rule[-0.200pt]{2.409pt}{0.400pt}}
\put(1429.0,339.0){\rule[-0.200pt]{2.409pt}{0.400pt}}
\put(170.0,339.0){\rule[-0.200pt]{2.409pt}{0.400pt}}
\put(1429.0,339.0){\rule[-0.200pt]{2.409pt}{0.400pt}}
\put(170.0,339.0){\rule[-0.200pt]{2.409pt}{0.400pt}}
\put(1429.0,339.0){\rule[-0.200pt]{2.409pt}{0.400pt}}
\put(170.0,339.0){\rule[-0.200pt]{2.409pt}{0.400pt}}
\put(1429.0,339.0){\rule[-0.200pt]{2.409pt}{0.400pt}}
\put(170.0,339.0){\rule[-0.200pt]{2.409pt}{0.400pt}}
\put(1429.0,339.0){\rule[-0.200pt]{2.409pt}{0.400pt}}
\put(170.0,339.0){\rule[-0.200pt]{2.409pt}{0.400pt}}
\put(1429.0,339.0){\rule[-0.200pt]{2.409pt}{0.400pt}}
\put(170.0,339.0){\rule[-0.200pt]{2.409pt}{0.400pt}}
\put(1429.0,339.0){\rule[-0.200pt]{2.409pt}{0.400pt}}
\put(170.0,339.0){\rule[-0.200pt]{2.409pt}{0.400pt}}
\put(1429.0,339.0){\rule[-0.200pt]{2.409pt}{0.400pt}}
\put(170.0,339.0){\rule[-0.200pt]{2.409pt}{0.400pt}}
\put(1429.0,339.0){\rule[-0.200pt]{2.409pt}{0.400pt}}
\put(170.0,339.0){\rule[-0.200pt]{2.409pt}{0.400pt}}
\put(1429.0,339.0){\rule[-0.200pt]{2.409pt}{0.400pt}}
\put(170.0,339.0){\rule[-0.200pt]{2.409pt}{0.400pt}}
\put(1429.0,339.0){\rule[-0.200pt]{2.409pt}{0.400pt}}
\put(170.0,339.0){\rule[-0.200pt]{2.409pt}{0.400pt}}
\put(1429.0,339.0){\rule[-0.200pt]{2.409pt}{0.400pt}}
\put(170.0,339.0){\rule[-0.200pt]{2.409pt}{0.400pt}}
\put(1429.0,339.0){\rule[-0.200pt]{2.409pt}{0.400pt}}
\put(170.0,339.0){\rule[-0.200pt]{2.409pt}{0.400pt}}
\put(1429.0,339.0){\rule[-0.200pt]{2.409pt}{0.400pt}}
\put(170.0,339.0){\rule[-0.200pt]{2.409pt}{0.400pt}}
\put(1429.0,339.0){\rule[-0.200pt]{2.409pt}{0.400pt}}
\put(170.0,339.0){\rule[-0.200pt]{2.409pt}{0.400pt}}
\put(1429.0,339.0){\rule[-0.200pt]{2.409pt}{0.400pt}}
\put(170.0,339.0){\rule[-0.200pt]{2.409pt}{0.400pt}}
\put(1429.0,339.0){\rule[-0.200pt]{2.409pt}{0.400pt}}
\put(170.0,339.0){\rule[-0.200pt]{2.409pt}{0.400pt}}
\put(1429.0,339.0){\rule[-0.200pt]{2.409pt}{0.400pt}}
\put(170.0,339.0){\rule[-0.200pt]{2.409pt}{0.400pt}}
\put(1429.0,339.0){\rule[-0.200pt]{2.409pt}{0.400pt}}
\put(170.0,339.0){\rule[-0.200pt]{2.409pt}{0.400pt}}
\put(1429.0,339.0){\rule[-0.200pt]{2.409pt}{0.400pt}}
\put(170.0,340.0){\rule[-0.200pt]{2.409pt}{0.400pt}}
\put(1429.0,340.0){\rule[-0.200pt]{2.409pt}{0.400pt}}
\put(170.0,340.0){\rule[-0.200pt]{2.409pt}{0.400pt}}
\put(1429.0,340.0){\rule[-0.200pt]{2.409pt}{0.400pt}}
\put(170.0,340.0){\rule[-0.200pt]{2.409pt}{0.400pt}}
\put(1429.0,340.0){\rule[-0.200pt]{2.409pt}{0.400pt}}
\put(170.0,340.0){\rule[-0.200pt]{2.409pt}{0.400pt}}
\put(1429.0,340.0){\rule[-0.200pt]{2.409pt}{0.400pt}}
\put(170.0,340.0){\rule[-0.200pt]{2.409pt}{0.400pt}}
\put(1429.0,340.0){\rule[-0.200pt]{2.409pt}{0.400pt}}
\put(170.0,340.0){\rule[-0.200pt]{2.409pt}{0.400pt}}
\put(1429.0,340.0){\rule[-0.200pt]{2.409pt}{0.400pt}}
\put(170.0,340.0){\rule[-0.200pt]{2.409pt}{0.400pt}}
\put(1429.0,340.0){\rule[-0.200pt]{2.409pt}{0.400pt}}
\put(170.0,340.0){\rule[-0.200pt]{2.409pt}{0.400pt}}
\put(1429.0,340.0){\rule[-0.200pt]{2.409pt}{0.400pt}}
\put(170.0,340.0){\rule[-0.200pt]{2.409pt}{0.400pt}}
\put(1429.0,340.0){\rule[-0.200pt]{2.409pt}{0.400pt}}
\put(170.0,340.0){\rule[-0.200pt]{2.409pt}{0.400pt}}
\put(1429.0,340.0){\rule[-0.200pt]{2.409pt}{0.400pt}}
\put(170.0,340.0){\rule[-0.200pt]{2.409pt}{0.400pt}}
\put(1429.0,340.0){\rule[-0.200pt]{2.409pt}{0.400pt}}
\put(170.0,340.0){\rule[-0.200pt]{2.409pt}{0.400pt}}
\put(1429.0,340.0){\rule[-0.200pt]{2.409pt}{0.400pt}}
\put(170.0,340.0){\rule[-0.200pt]{2.409pt}{0.400pt}}
\put(1429.0,340.0){\rule[-0.200pt]{2.409pt}{0.400pt}}
\put(170.0,340.0){\rule[-0.200pt]{2.409pt}{0.400pt}}
\put(1429.0,340.0){\rule[-0.200pt]{2.409pt}{0.400pt}}
\put(170.0,340.0){\rule[-0.200pt]{2.409pt}{0.400pt}}
\put(1429.0,340.0){\rule[-0.200pt]{2.409pt}{0.400pt}}
\put(170.0,340.0){\rule[-0.200pt]{2.409pt}{0.400pt}}
\put(1429.0,340.0){\rule[-0.200pt]{2.409pt}{0.400pt}}
\put(170.0,340.0){\rule[-0.200pt]{2.409pt}{0.400pt}}
\put(1429.0,340.0){\rule[-0.200pt]{2.409pt}{0.400pt}}
\put(170.0,340.0){\rule[-0.200pt]{2.409pt}{0.400pt}}
\put(1429.0,340.0){\rule[-0.200pt]{2.409pt}{0.400pt}}
\put(170.0,340.0){\rule[-0.200pt]{2.409pt}{0.400pt}}
\put(1429.0,340.0){\rule[-0.200pt]{2.409pt}{0.400pt}}
\put(170.0,340.0){\rule[-0.200pt]{2.409pt}{0.400pt}}
\put(1429.0,340.0){\rule[-0.200pt]{2.409pt}{0.400pt}}
\put(170.0,340.0){\rule[-0.200pt]{2.409pt}{0.400pt}}
\put(1429.0,340.0){\rule[-0.200pt]{2.409pt}{0.400pt}}
\put(170.0,340.0){\rule[-0.200pt]{2.409pt}{0.400pt}}
\put(1429.0,340.0){\rule[-0.200pt]{2.409pt}{0.400pt}}
\put(170.0,340.0){\rule[-0.200pt]{2.409pt}{0.400pt}}
\put(1429.0,340.0){\rule[-0.200pt]{2.409pt}{0.400pt}}
\put(170.0,340.0){\rule[-0.200pt]{2.409pt}{0.400pt}}
\put(1429.0,340.0){\rule[-0.200pt]{2.409pt}{0.400pt}}
\put(170.0,340.0){\rule[-0.200pt]{2.409pt}{0.400pt}}
\put(1429.0,340.0){\rule[-0.200pt]{2.409pt}{0.400pt}}
\put(170.0,340.0){\rule[-0.200pt]{2.409pt}{0.400pt}}
\put(1429.0,340.0){\rule[-0.200pt]{2.409pt}{0.400pt}}
\put(170.0,341.0){\rule[-0.200pt]{2.409pt}{0.400pt}}
\put(1429.0,341.0){\rule[-0.200pt]{2.409pt}{0.400pt}}
\put(170.0,341.0){\rule[-0.200pt]{2.409pt}{0.400pt}}
\put(1429.0,341.0){\rule[-0.200pt]{2.409pt}{0.400pt}}
\put(170.0,341.0){\rule[-0.200pt]{2.409pt}{0.400pt}}
\put(1429.0,341.0){\rule[-0.200pt]{2.409pt}{0.400pt}}
\put(170.0,341.0){\rule[-0.200pt]{2.409pt}{0.400pt}}
\put(1429.0,341.0){\rule[-0.200pt]{2.409pt}{0.400pt}}
\put(170.0,341.0){\rule[-0.200pt]{2.409pt}{0.400pt}}
\put(1429.0,341.0){\rule[-0.200pt]{2.409pt}{0.400pt}}
\put(170.0,341.0){\rule[-0.200pt]{2.409pt}{0.400pt}}
\put(1429.0,341.0){\rule[-0.200pt]{2.409pt}{0.400pt}}
\put(170.0,341.0){\rule[-0.200pt]{2.409pt}{0.400pt}}
\put(1429.0,341.0){\rule[-0.200pt]{2.409pt}{0.400pt}}
\put(170.0,341.0){\rule[-0.200pt]{2.409pt}{0.400pt}}
\put(1429.0,341.0){\rule[-0.200pt]{2.409pt}{0.400pt}}
\put(170.0,341.0){\rule[-0.200pt]{2.409pt}{0.400pt}}
\put(1429.0,341.0){\rule[-0.200pt]{2.409pt}{0.400pt}}
\put(170.0,341.0){\rule[-0.200pt]{2.409pt}{0.400pt}}
\put(1429.0,341.0){\rule[-0.200pt]{2.409pt}{0.400pt}}
\put(170.0,341.0){\rule[-0.200pt]{2.409pt}{0.400pt}}
\put(1429.0,341.0){\rule[-0.200pt]{2.409pt}{0.400pt}}
\put(170.0,341.0){\rule[-0.200pt]{2.409pt}{0.400pt}}
\put(1429.0,341.0){\rule[-0.200pt]{2.409pt}{0.400pt}}
\put(170.0,341.0){\rule[-0.200pt]{2.409pt}{0.400pt}}
\put(1429.0,341.0){\rule[-0.200pt]{2.409pt}{0.400pt}}
\put(170.0,341.0){\rule[-0.200pt]{4.818pt}{0.400pt}}
\put(150,341){\makebox(0,0)[r]{ 1000}}
\put(1419.0,341.0){\rule[-0.200pt]{4.818pt}{0.400pt}}
\put(170.0,367.0){\rule[-0.200pt]{2.409pt}{0.400pt}}
\put(1429.0,367.0){\rule[-0.200pt]{2.409pt}{0.400pt}}
\put(170.0,382.0){\rule[-0.200pt]{2.409pt}{0.400pt}}
\put(1429.0,382.0){\rule[-0.200pt]{2.409pt}{0.400pt}}
\put(170.0,393.0){\rule[-0.200pt]{2.409pt}{0.400pt}}
\put(1429.0,393.0){\rule[-0.200pt]{2.409pt}{0.400pt}}
\put(170.0,401.0){\rule[-0.200pt]{2.409pt}{0.400pt}}
\put(1429.0,401.0){\rule[-0.200pt]{2.409pt}{0.400pt}}
\put(170.0,408.0){\rule[-0.200pt]{2.409pt}{0.400pt}}
\put(1429.0,408.0){\rule[-0.200pt]{2.409pt}{0.400pt}}
\put(170.0,414.0){\rule[-0.200pt]{2.409pt}{0.400pt}}
\put(1429.0,414.0){\rule[-0.200pt]{2.409pt}{0.400pt}}
\put(170.0,419.0){\rule[-0.200pt]{2.409pt}{0.400pt}}
\put(1429.0,419.0){\rule[-0.200pt]{2.409pt}{0.400pt}}
\put(170.0,423.0){\rule[-0.200pt]{2.409pt}{0.400pt}}
\put(1429.0,423.0){\rule[-0.200pt]{2.409pt}{0.400pt}}
\put(170.0,427.0){\rule[-0.200pt]{2.409pt}{0.400pt}}
\put(1429.0,427.0){\rule[-0.200pt]{2.409pt}{0.400pt}}
\put(170.0,431.0){\rule[-0.200pt]{2.409pt}{0.400pt}}
\put(1429.0,431.0){\rule[-0.200pt]{2.409pt}{0.400pt}}
\put(170.0,434.0){\rule[-0.200pt]{2.409pt}{0.400pt}}
\put(1429.0,434.0){\rule[-0.200pt]{2.409pt}{0.400pt}}
\put(170.0,437.0){\rule[-0.200pt]{2.409pt}{0.400pt}}
\put(1429.0,437.0){\rule[-0.200pt]{2.409pt}{0.400pt}}
\put(170.0,440.0){\rule[-0.200pt]{2.409pt}{0.400pt}}
\put(1429.0,440.0){\rule[-0.200pt]{2.409pt}{0.400pt}}
\put(170.0,443.0){\rule[-0.200pt]{2.409pt}{0.400pt}}
\put(1429.0,443.0){\rule[-0.200pt]{2.409pt}{0.400pt}}
\put(170.0,445.0){\rule[-0.200pt]{2.409pt}{0.400pt}}
\put(1429.0,445.0){\rule[-0.200pt]{2.409pt}{0.400pt}}
\put(170.0,447.0){\rule[-0.200pt]{2.409pt}{0.400pt}}
\put(1429.0,447.0){\rule[-0.200pt]{2.409pt}{0.400pt}}
\put(170.0,449.0){\rule[-0.200pt]{2.409pt}{0.400pt}}
\put(1429.0,449.0){\rule[-0.200pt]{2.409pt}{0.400pt}}
\put(170.0,451.0){\rule[-0.200pt]{2.409pt}{0.400pt}}
\put(1429.0,451.0){\rule[-0.200pt]{2.409pt}{0.400pt}}
\put(170.0,453.0){\rule[-0.200pt]{2.409pt}{0.400pt}}
\put(1429.0,453.0){\rule[-0.200pt]{2.409pt}{0.400pt}}
\put(170.0,455.0){\rule[-0.200pt]{2.409pt}{0.400pt}}
\put(1429.0,455.0){\rule[-0.200pt]{2.409pt}{0.400pt}}
\put(170.0,457.0){\rule[-0.200pt]{2.409pt}{0.400pt}}
\put(1429.0,457.0){\rule[-0.200pt]{2.409pt}{0.400pt}}
\put(170.0,459.0){\rule[-0.200pt]{2.409pt}{0.400pt}}
\put(1429.0,459.0){\rule[-0.200pt]{2.409pt}{0.400pt}}
\put(170.0,460.0){\rule[-0.200pt]{2.409pt}{0.400pt}}
\put(1429.0,460.0){\rule[-0.200pt]{2.409pt}{0.400pt}}
\put(170.0,462.0){\rule[-0.200pt]{2.409pt}{0.400pt}}
\put(1429.0,462.0){\rule[-0.200pt]{2.409pt}{0.400pt}}
\put(170.0,463.0){\rule[-0.200pt]{2.409pt}{0.400pt}}
\put(1429.0,463.0){\rule[-0.200pt]{2.409pt}{0.400pt}}
\put(170.0,465.0){\rule[-0.200pt]{2.409pt}{0.400pt}}
\put(1429.0,465.0){\rule[-0.200pt]{2.409pt}{0.400pt}}
\put(170.0,466.0){\rule[-0.200pt]{2.409pt}{0.400pt}}
\put(1429.0,466.0){\rule[-0.200pt]{2.409pt}{0.400pt}}
\put(170.0,467.0){\rule[-0.200pt]{2.409pt}{0.400pt}}
\put(1429.0,467.0){\rule[-0.200pt]{2.409pt}{0.400pt}}
\put(170.0,469.0){\rule[-0.200pt]{2.409pt}{0.400pt}}
\put(1429.0,469.0){\rule[-0.200pt]{2.409pt}{0.400pt}}
\put(170.0,470.0){\rule[-0.200pt]{2.409pt}{0.400pt}}
\put(1429.0,470.0){\rule[-0.200pt]{2.409pt}{0.400pt}}
\put(170.0,471.0){\rule[-0.200pt]{2.409pt}{0.400pt}}
\put(1429.0,471.0){\rule[-0.200pt]{2.409pt}{0.400pt}}
\put(170.0,472.0){\rule[-0.200pt]{2.409pt}{0.400pt}}
\put(1429.0,472.0){\rule[-0.200pt]{2.409pt}{0.400pt}}
\put(170.0,473.0){\rule[-0.200pt]{2.409pt}{0.400pt}}
\put(1429.0,473.0){\rule[-0.200pt]{2.409pt}{0.400pt}}
\put(170.0,474.0){\rule[-0.200pt]{2.409pt}{0.400pt}}
\put(1429.0,474.0){\rule[-0.200pt]{2.409pt}{0.400pt}}
\put(170.0,475.0){\rule[-0.200pt]{2.409pt}{0.400pt}}
\put(1429.0,475.0){\rule[-0.200pt]{2.409pt}{0.400pt}}
\put(170.0,476.0){\rule[-0.200pt]{2.409pt}{0.400pt}}
\put(1429.0,476.0){\rule[-0.200pt]{2.409pt}{0.400pt}}
\put(170.0,477.0){\rule[-0.200pt]{2.409pt}{0.400pt}}
\put(1429.0,477.0){\rule[-0.200pt]{2.409pt}{0.400pt}}
\put(170.0,478.0){\rule[-0.200pt]{2.409pt}{0.400pt}}
\put(1429.0,478.0){\rule[-0.200pt]{2.409pt}{0.400pt}}
\put(170.0,479.0){\rule[-0.200pt]{2.409pt}{0.400pt}}
\put(1429.0,479.0){\rule[-0.200pt]{2.409pt}{0.400pt}}
\put(170.0,480.0){\rule[-0.200pt]{2.409pt}{0.400pt}}
\put(1429.0,480.0){\rule[-0.200pt]{2.409pt}{0.400pt}}
\put(170.0,481.0){\rule[-0.200pt]{2.409pt}{0.400pt}}
\put(1429.0,481.0){\rule[-0.200pt]{2.409pt}{0.400pt}}
\put(170.0,482.0){\rule[-0.200pt]{2.409pt}{0.400pt}}
\put(1429.0,482.0){\rule[-0.200pt]{2.409pt}{0.400pt}}
\put(170.0,483.0){\rule[-0.200pt]{2.409pt}{0.400pt}}
\put(1429.0,483.0){\rule[-0.200pt]{2.409pt}{0.400pt}}
\put(170.0,484.0){\rule[-0.200pt]{2.409pt}{0.400pt}}
\put(1429.0,484.0){\rule[-0.200pt]{2.409pt}{0.400pt}}
\put(170.0,485.0){\rule[-0.200pt]{2.409pt}{0.400pt}}
\put(1429.0,485.0){\rule[-0.200pt]{2.409pt}{0.400pt}}
\put(170.0,485.0){\rule[-0.200pt]{2.409pt}{0.400pt}}
\put(1429.0,485.0){\rule[-0.200pt]{2.409pt}{0.400pt}}
\put(170.0,486.0){\rule[-0.200pt]{2.409pt}{0.400pt}}
\put(1429.0,486.0){\rule[-0.200pt]{2.409pt}{0.400pt}}
\put(170.0,487.0){\rule[-0.200pt]{2.409pt}{0.400pt}}
\put(1429.0,487.0){\rule[-0.200pt]{2.409pt}{0.400pt}}
\put(170.0,488.0){\rule[-0.200pt]{2.409pt}{0.400pt}}
\put(1429.0,488.0){\rule[-0.200pt]{2.409pt}{0.400pt}}
\put(170.0,488.0){\rule[-0.200pt]{2.409pt}{0.400pt}}
\put(1429.0,488.0){\rule[-0.200pt]{2.409pt}{0.400pt}}
\put(170.0,489.0){\rule[-0.200pt]{2.409pt}{0.400pt}}
\put(1429.0,489.0){\rule[-0.200pt]{2.409pt}{0.400pt}}
\put(170.0,490.0){\rule[-0.200pt]{2.409pt}{0.400pt}}
\put(1429.0,490.0){\rule[-0.200pt]{2.409pt}{0.400pt}}
\put(170.0,491.0){\rule[-0.200pt]{2.409pt}{0.400pt}}
\put(1429.0,491.0){\rule[-0.200pt]{2.409pt}{0.400pt}}
\put(170.0,491.0){\rule[-0.200pt]{2.409pt}{0.400pt}}
\put(1429.0,491.0){\rule[-0.200pt]{2.409pt}{0.400pt}}
\put(170.0,492.0){\rule[-0.200pt]{2.409pt}{0.400pt}}
\put(1429.0,492.0){\rule[-0.200pt]{2.409pt}{0.400pt}}
\put(170.0,493.0){\rule[-0.200pt]{2.409pt}{0.400pt}}
\put(1429.0,493.0){\rule[-0.200pt]{2.409pt}{0.400pt}}
\put(170.0,493.0){\rule[-0.200pt]{2.409pt}{0.400pt}}
\put(1429.0,493.0){\rule[-0.200pt]{2.409pt}{0.400pt}}
\put(170.0,494.0){\rule[-0.200pt]{2.409pt}{0.400pt}}
\put(1429.0,494.0){\rule[-0.200pt]{2.409pt}{0.400pt}}
\put(170.0,495.0){\rule[-0.200pt]{2.409pt}{0.400pt}}
\put(1429.0,495.0){\rule[-0.200pt]{2.409pt}{0.400pt}}
\put(170.0,495.0){\rule[-0.200pt]{2.409pt}{0.400pt}}
\put(1429.0,495.0){\rule[-0.200pt]{2.409pt}{0.400pt}}
\put(170.0,496.0){\rule[-0.200pt]{2.409pt}{0.400pt}}
\put(1429.0,496.0){\rule[-0.200pt]{2.409pt}{0.400pt}}
\put(170.0,496.0){\rule[-0.200pt]{2.409pt}{0.400pt}}
\put(1429.0,496.0){\rule[-0.200pt]{2.409pt}{0.400pt}}
\put(170.0,497.0){\rule[-0.200pt]{2.409pt}{0.400pt}}
\put(1429.0,497.0){\rule[-0.200pt]{2.409pt}{0.400pt}}
\put(170.0,498.0){\rule[-0.200pt]{2.409pt}{0.400pt}}
\put(1429.0,498.0){\rule[-0.200pt]{2.409pt}{0.400pt}}
\put(170.0,498.0){\rule[-0.200pt]{2.409pt}{0.400pt}}
\put(1429.0,498.0){\rule[-0.200pt]{2.409pt}{0.400pt}}
\put(170.0,499.0){\rule[-0.200pt]{2.409pt}{0.400pt}}
\put(1429.0,499.0){\rule[-0.200pt]{2.409pt}{0.400pt}}
\put(170.0,499.0){\rule[-0.200pt]{2.409pt}{0.400pt}}
\put(1429.0,499.0){\rule[-0.200pt]{2.409pt}{0.400pt}}
\put(170.0,500.0){\rule[-0.200pt]{2.409pt}{0.400pt}}
\put(1429.0,500.0){\rule[-0.200pt]{2.409pt}{0.400pt}}
\put(170.0,500.0){\rule[-0.200pt]{2.409pt}{0.400pt}}
\put(1429.0,500.0){\rule[-0.200pt]{2.409pt}{0.400pt}}
\put(170.0,501.0){\rule[-0.200pt]{2.409pt}{0.400pt}}
\put(1429.0,501.0){\rule[-0.200pt]{2.409pt}{0.400pt}}
\put(170.0,501.0){\rule[-0.200pt]{2.409pt}{0.400pt}}
\put(1429.0,501.0){\rule[-0.200pt]{2.409pt}{0.400pt}}
\put(170.0,502.0){\rule[-0.200pt]{2.409pt}{0.400pt}}
\put(1429.0,502.0){\rule[-0.200pt]{2.409pt}{0.400pt}}
\put(170.0,502.0){\rule[-0.200pt]{2.409pt}{0.400pt}}
\put(1429.0,502.0){\rule[-0.200pt]{2.409pt}{0.400pt}}
\put(170.0,503.0){\rule[-0.200pt]{2.409pt}{0.400pt}}
\put(1429.0,503.0){\rule[-0.200pt]{2.409pt}{0.400pt}}
\put(170.0,503.0){\rule[-0.200pt]{2.409pt}{0.400pt}}
\put(1429.0,503.0){\rule[-0.200pt]{2.409pt}{0.400pt}}
\put(170.0,504.0){\rule[-0.200pt]{2.409pt}{0.400pt}}
\put(1429.0,504.0){\rule[-0.200pt]{2.409pt}{0.400pt}}
\put(170.0,504.0){\rule[-0.200pt]{2.409pt}{0.400pt}}
\put(1429.0,504.0){\rule[-0.200pt]{2.409pt}{0.400pt}}
\put(170.0,505.0){\rule[-0.200pt]{2.409pt}{0.400pt}}
\put(1429.0,505.0){\rule[-0.200pt]{2.409pt}{0.400pt}}
\put(170.0,505.0){\rule[-0.200pt]{2.409pt}{0.400pt}}
\put(1429.0,505.0){\rule[-0.200pt]{2.409pt}{0.400pt}}
\put(170.0,506.0){\rule[-0.200pt]{2.409pt}{0.400pt}}
\put(1429.0,506.0){\rule[-0.200pt]{2.409pt}{0.400pt}}
\put(170.0,506.0){\rule[-0.200pt]{2.409pt}{0.400pt}}
\put(1429.0,506.0){\rule[-0.200pt]{2.409pt}{0.400pt}}
\put(170.0,507.0){\rule[-0.200pt]{2.409pt}{0.400pt}}
\put(1429.0,507.0){\rule[-0.200pt]{2.409pt}{0.400pt}}
\put(170.0,507.0){\rule[-0.200pt]{2.409pt}{0.400pt}}
\put(1429.0,507.0){\rule[-0.200pt]{2.409pt}{0.400pt}}
\put(170.0,508.0){\rule[-0.200pt]{2.409pt}{0.400pt}}
\put(1429.0,508.0){\rule[-0.200pt]{2.409pt}{0.400pt}}
\put(170.0,508.0){\rule[-0.200pt]{2.409pt}{0.400pt}}
\put(1429.0,508.0){\rule[-0.200pt]{2.409pt}{0.400pt}}
\put(170.0,508.0){\rule[-0.200pt]{2.409pt}{0.400pt}}
\put(1429.0,508.0){\rule[-0.200pt]{2.409pt}{0.400pt}}
\put(170.0,509.0){\rule[-0.200pt]{2.409pt}{0.400pt}}
\put(1429.0,509.0){\rule[-0.200pt]{2.409pt}{0.400pt}}
\put(170.0,509.0){\rule[-0.200pt]{2.409pt}{0.400pt}}
\put(1429.0,509.0){\rule[-0.200pt]{2.409pt}{0.400pt}}
\put(170.0,510.0){\rule[-0.200pt]{2.409pt}{0.400pt}}
\put(1429.0,510.0){\rule[-0.200pt]{2.409pt}{0.400pt}}
\put(170.0,510.0){\rule[-0.200pt]{2.409pt}{0.400pt}}
\put(1429.0,510.0){\rule[-0.200pt]{2.409pt}{0.400pt}}
\put(170.0,511.0){\rule[-0.200pt]{2.409pt}{0.400pt}}
\put(1429.0,511.0){\rule[-0.200pt]{2.409pt}{0.400pt}}
\put(170.0,511.0){\rule[-0.200pt]{2.409pt}{0.400pt}}
\put(1429.0,511.0){\rule[-0.200pt]{2.409pt}{0.400pt}}
\put(170.0,511.0){\rule[-0.200pt]{2.409pt}{0.400pt}}
\put(1429.0,511.0){\rule[-0.200pt]{2.409pt}{0.400pt}}
\put(170.0,512.0){\rule[-0.200pt]{2.409pt}{0.400pt}}
\put(1429.0,512.0){\rule[-0.200pt]{2.409pt}{0.400pt}}
\put(170.0,512.0){\rule[-0.200pt]{2.409pt}{0.400pt}}
\put(1429.0,512.0){\rule[-0.200pt]{2.409pt}{0.400pt}}
\put(170.0,513.0){\rule[-0.200pt]{2.409pt}{0.400pt}}
\put(1429.0,513.0){\rule[-0.200pt]{2.409pt}{0.400pt}}
\put(170.0,513.0){\rule[-0.200pt]{2.409pt}{0.400pt}}
\put(1429.0,513.0){\rule[-0.200pt]{2.409pt}{0.400pt}}
\put(170.0,513.0){\rule[-0.200pt]{2.409pt}{0.400pt}}
\put(1429.0,513.0){\rule[-0.200pt]{2.409pt}{0.400pt}}
\put(170.0,514.0){\rule[-0.200pt]{2.409pt}{0.400pt}}
\put(1429.0,514.0){\rule[-0.200pt]{2.409pt}{0.400pt}}
\put(170.0,514.0){\rule[-0.200pt]{2.409pt}{0.400pt}}
\put(1429.0,514.0){\rule[-0.200pt]{2.409pt}{0.400pt}}
\put(170.0,514.0){\rule[-0.200pt]{2.409pt}{0.400pt}}
\put(1429.0,514.0){\rule[-0.200pt]{2.409pt}{0.400pt}}
\put(170.0,515.0){\rule[-0.200pt]{2.409pt}{0.400pt}}
\put(1429.0,515.0){\rule[-0.200pt]{2.409pt}{0.400pt}}
\put(170.0,515.0){\rule[-0.200pt]{2.409pt}{0.400pt}}
\put(1429.0,515.0){\rule[-0.200pt]{2.409pt}{0.400pt}}
\put(170.0,515.0){\rule[-0.200pt]{2.409pt}{0.400pt}}
\put(1429.0,515.0){\rule[-0.200pt]{2.409pt}{0.400pt}}
\put(170.0,516.0){\rule[-0.200pt]{2.409pt}{0.400pt}}
\put(1429.0,516.0){\rule[-0.200pt]{2.409pt}{0.400pt}}
\put(170.0,516.0){\rule[-0.200pt]{2.409pt}{0.400pt}}
\put(1429.0,516.0){\rule[-0.200pt]{2.409pt}{0.400pt}}
\put(170.0,517.0){\rule[-0.200pt]{2.409pt}{0.400pt}}
\put(1429.0,517.0){\rule[-0.200pt]{2.409pt}{0.400pt}}
\put(170.0,517.0){\rule[-0.200pt]{2.409pt}{0.400pt}}
\put(1429.0,517.0){\rule[-0.200pt]{2.409pt}{0.400pt}}
\put(170.0,517.0){\rule[-0.200pt]{2.409pt}{0.400pt}}
\put(1429.0,517.0){\rule[-0.200pt]{2.409pt}{0.400pt}}
\put(170.0,518.0){\rule[-0.200pt]{2.409pt}{0.400pt}}
\put(1429.0,518.0){\rule[-0.200pt]{2.409pt}{0.400pt}}
\put(170.0,518.0){\rule[-0.200pt]{2.409pt}{0.400pt}}
\put(1429.0,518.0){\rule[-0.200pt]{2.409pt}{0.400pt}}
\put(170.0,518.0){\rule[-0.200pt]{2.409pt}{0.400pt}}
\put(1429.0,518.0){\rule[-0.200pt]{2.409pt}{0.400pt}}
\put(170.0,519.0){\rule[-0.200pt]{2.409pt}{0.400pt}}
\put(1429.0,519.0){\rule[-0.200pt]{2.409pt}{0.400pt}}
\put(170.0,519.0){\rule[-0.200pt]{2.409pt}{0.400pt}}
\put(1429.0,519.0){\rule[-0.200pt]{2.409pt}{0.400pt}}
\put(170.0,519.0){\rule[-0.200pt]{2.409pt}{0.400pt}}
\put(1429.0,519.0){\rule[-0.200pt]{2.409pt}{0.400pt}}
\put(170.0,520.0){\rule[-0.200pt]{2.409pt}{0.400pt}}
\put(1429.0,520.0){\rule[-0.200pt]{2.409pt}{0.400pt}}
\put(170.0,520.0){\rule[-0.200pt]{2.409pt}{0.400pt}}
\put(1429.0,520.0){\rule[-0.200pt]{2.409pt}{0.400pt}}
\put(170.0,520.0){\rule[-0.200pt]{2.409pt}{0.400pt}}
\put(1429.0,520.0){\rule[-0.200pt]{2.409pt}{0.400pt}}
\put(170.0,521.0){\rule[-0.200pt]{2.409pt}{0.400pt}}
\put(1429.0,521.0){\rule[-0.200pt]{2.409pt}{0.400pt}}
\put(170.0,521.0){\rule[-0.200pt]{2.409pt}{0.400pt}}
\put(1429.0,521.0){\rule[-0.200pt]{2.409pt}{0.400pt}}
\put(170.0,521.0){\rule[-0.200pt]{2.409pt}{0.400pt}}
\put(1429.0,521.0){\rule[-0.200pt]{2.409pt}{0.400pt}}
\put(170.0,521.0){\rule[-0.200pt]{2.409pt}{0.400pt}}
\put(1429.0,521.0){\rule[-0.200pt]{2.409pt}{0.400pt}}
\put(170.0,522.0){\rule[-0.200pt]{2.409pt}{0.400pt}}
\put(1429.0,522.0){\rule[-0.200pt]{2.409pt}{0.400pt}}
\put(170.0,522.0){\rule[-0.200pt]{2.409pt}{0.400pt}}
\put(1429.0,522.0){\rule[-0.200pt]{2.409pt}{0.400pt}}
\put(170.0,522.0){\rule[-0.200pt]{2.409pt}{0.400pt}}
\put(1429.0,522.0){\rule[-0.200pt]{2.409pt}{0.400pt}}
\put(170.0,523.0){\rule[-0.200pt]{2.409pt}{0.400pt}}
\put(1429.0,523.0){\rule[-0.200pt]{2.409pt}{0.400pt}}
\put(170.0,523.0){\rule[-0.200pt]{2.409pt}{0.400pt}}
\put(1429.0,523.0){\rule[-0.200pt]{2.409pt}{0.400pt}}
\put(170.0,523.0){\rule[-0.200pt]{2.409pt}{0.400pt}}
\put(1429.0,523.0){\rule[-0.200pt]{2.409pt}{0.400pt}}
\put(170.0,524.0){\rule[-0.200pt]{2.409pt}{0.400pt}}
\put(1429.0,524.0){\rule[-0.200pt]{2.409pt}{0.400pt}}
\put(170.0,524.0){\rule[-0.200pt]{2.409pt}{0.400pt}}
\put(1429.0,524.0){\rule[-0.200pt]{2.409pt}{0.400pt}}
\put(170.0,524.0){\rule[-0.200pt]{2.409pt}{0.400pt}}
\put(1429.0,524.0){\rule[-0.200pt]{2.409pt}{0.400pt}}
\put(170.0,524.0){\rule[-0.200pt]{2.409pt}{0.400pt}}
\put(1429.0,524.0){\rule[-0.200pt]{2.409pt}{0.400pt}}
\put(170.0,525.0){\rule[-0.200pt]{2.409pt}{0.400pt}}
\put(1429.0,525.0){\rule[-0.200pt]{2.409pt}{0.400pt}}
\put(170.0,525.0){\rule[-0.200pt]{2.409pt}{0.400pt}}
\put(1429.0,525.0){\rule[-0.200pt]{2.409pt}{0.400pt}}
\put(170.0,525.0){\rule[-0.200pt]{2.409pt}{0.400pt}}
\put(1429.0,525.0){\rule[-0.200pt]{2.409pt}{0.400pt}}
\put(170.0,525.0){\rule[-0.200pt]{2.409pt}{0.400pt}}
\put(1429.0,525.0){\rule[-0.200pt]{2.409pt}{0.400pt}}
\put(170.0,526.0){\rule[-0.200pt]{2.409pt}{0.400pt}}
\put(1429.0,526.0){\rule[-0.200pt]{2.409pt}{0.400pt}}
\put(170.0,526.0){\rule[-0.200pt]{2.409pt}{0.400pt}}
\put(1429.0,526.0){\rule[-0.200pt]{2.409pt}{0.400pt}}
\put(170.0,526.0){\rule[-0.200pt]{2.409pt}{0.400pt}}
\put(1429.0,526.0){\rule[-0.200pt]{2.409pt}{0.400pt}}
\put(170.0,527.0){\rule[-0.200pt]{2.409pt}{0.400pt}}
\put(1429.0,527.0){\rule[-0.200pt]{2.409pt}{0.400pt}}
\put(170.0,527.0){\rule[-0.200pt]{2.409pt}{0.400pt}}
\put(1429.0,527.0){\rule[-0.200pt]{2.409pt}{0.400pt}}
\put(170.0,527.0){\rule[-0.200pt]{2.409pt}{0.400pt}}
\put(1429.0,527.0){\rule[-0.200pt]{2.409pt}{0.400pt}}
\put(170.0,527.0){\rule[-0.200pt]{2.409pt}{0.400pt}}
\put(1429.0,527.0){\rule[-0.200pt]{2.409pt}{0.400pt}}
\put(170.0,528.0){\rule[-0.200pt]{2.409pt}{0.400pt}}
\put(1429.0,528.0){\rule[-0.200pt]{2.409pt}{0.400pt}}
\put(170.0,528.0){\rule[-0.200pt]{2.409pt}{0.400pt}}
\put(1429.0,528.0){\rule[-0.200pt]{2.409pt}{0.400pt}}
\put(170.0,528.0){\rule[-0.200pt]{2.409pt}{0.400pt}}
\put(1429.0,528.0){\rule[-0.200pt]{2.409pt}{0.400pt}}
\put(170.0,528.0){\rule[-0.200pt]{2.409pt}{0.400pt}}
\put(1429.0,528.0){\rule[-0.200pt]{2.409pt}{0.400pt}}
\put(170.0,529.0){\rule[-0.200pt]{2.409pt}{0.400pt}}
\put(1429.0,529.0){\rule[-0.200pt]{2.409pt}{0.400pt}}
\put(170.0,529.0){\rule[-0.200pt]{2.409pt}{0.400pt}}
\put(1429.0,529.0){\rule[-0.200pt]{2.409pt}{0.400pt}}
\put(170.0,529.0){\rule[-0.200pt]{2.409pt}{0.400pt}}
\put(1429.0,529.0){\rule[-0.200pt]{2.409pt}{0.400pt}}
\put(170.0,529.0){\rule[-0.200pt]{2.409pt}{0.400pt}}
\put(1429.0,529.0){\rule[-0.200pt]{2.409pt}{0.400pt}}
\put(170.0,530.0){\rule[-0.200pt]{2.409pt}{0.400pt}}
\put(1429.0,530.0){\rule[-0.200pt]{2.409pt}{0.400pt}}
\put(170.0,530.0){\rule[-0.200pt]{2.409pt}{0.400pt}}
\put(1429.0,530.0){\rule[-0.200pt]{2.409pt}{0.400pt}}
\put(170.0,530.0){\rule[-0.200pt]{2.409pt}{0.400pt}}
\put(1429.0,530.0){\rule[-0.200pt]{2.409pt}{0.400pt}}
\put(170.0,530.0){\rule[-0.200pt]{2.409pt}{0.400pt}}
\put(1429.0,530.0){\rule[-0.200pt]{2.409pt}{0.400pt}}
\put(170.0,531.0){\rule[-0.200pt]{2.409pt}{0.400pt}}
\put(1429.0,531.0){\rule[-0.200pt]{2.409pt}{0.400pt}}
\put(170.0,531.0){\rule[-0.200pt]{2.409pt}{0.400pt}}
\put(1429.0,531.0){\rule[-0.200pt]{2.409pt}{0.400pt}}
\put(170.0,531.0){\rule[-0.200pt]{2.409pt}{0.400pt}}
\put(1429.0,531.0){\rule[-0.200pt]{2.409pt}{0.400pt}}
\put(170.0,531.0){\rule[-0.200pt]{2.409pt}{0.400pt}}
\put(1429.0,531.0){\rule[-0.200pt]{2.409pt}{0.400pt}}
\put(170.0,532.0){\rule[-0.200pt]{2.409pt}{0.400pt}}
\put(1429.0,532.0){\rule[-0.200pt]{2.409pt}{0.400pt}}
\put(170.0,532.0){\rule[-0.200pt]{2.409pt}{0.400pt}}
\put(1429.0,532.0){\rule[-0.200pt]{2.409pt}{0.400pt}}
\put(170.0,532.0){\rule[-0.200pt]{2.409pt}{0.400pt}}
\put(1429.0,532.0){\rule[-0.200pt]{2.409pt}{0.400pt}}
\put(170.0,532.0){\rule[-0.200pt]{2.409pt}{0.400pt}}
\put(1429.0,532.0){\rule[-0.200pt]{2.409pt}{0.400pt}}
\put(170.0,532.0){\rule[-0.200pt]{2.409pt}{0.400pt}}
\put(1429.0,532.0){\rule[-0.200pt]{2.409pt}{0.400pt}}
\put(170.0,533.0){\rule[-0.200pt]{2.409pt}{0.400pt}}
\put(1429.0,533.0){\rule[-0.200pt]{2.409pt}{0.400pt}}
\put(170.0,533.0){\rule[-0.200pt]{2.409pt}{0.400pt}}
\put(1429.0,533.0){\rule[-0.200pt]{2.409pt}{0.400pt}}
\put(170.0,533.0){\rule[-0.200pt]{2.409pt}{0.400pt}}
\put(1429.0,533.0){\rule[-0.200pt]{2.409pt}{0.400pt}}
\put(170.0,533.0){\rule[-0.200pt]{2.409pt}{0.400pt}}
\put(1429.0,533.0){\rule[-0.200pt]{2.409pt}{0.400pt}}
\put(170.0,534.0){\rule[-0.200pt]{2.409pt}{0.400pt}}
\put(1429.0,534.0){\rule[-0.200pt]{2.409pt}{0.400pt}}
\put(170.0,534.0){\rule[-0.200pt]{2.409pt}{0.400pt}}
\put(1429.0,534.0){\rule[-0.200pt]{2.409pt}{0.400pt}}
\put(170.0,534.0){\rule[-0.200pt]{2.409pt}{0.400pt}}
\put(1429.0,534.0){\rule[-0.200pt]{2.409pt}{0.400pt}}
\put(170.0,534.0){\rule[-0.200pt]{2.409pt}{0.400pt}}
\put(1429.0,534.0){\rule[-0.200pt]{2.409pt}{0.400pt}}
\put(170.0,534.0){\rule[-0.200pt]{2.409pt}{0.400pt}}
\put(1429.0,534.0){\rule[-0.200pt]{2.409pt}{0.400pt}}
\put(170.0,535.0){\rule[-0.200pt]{2.409pt}{0.400pt}}
\put(1429.0,535.0){\rule[-0.200pt]{2.409pt}{0.400pt}}
\put(170.0,535.0){\rule[-0.200pt]{2.409pt}{0.400pt}}
\put(1429.0,535.0){\rule[-0.200pt]{2.409pt}{0.400pt}}
\put(170.0,535.0){\rule[-0.200pt]{2.409pt}{0.400pt}}
\put(1429.0,535.0){\rule[-0.200pt]{2.409pt}{0.400pt}}
\put(170.0,535.0){\rule[-0.200pt]{2.409pt}{0.400pt}}
\put(1429.0,535.0){\rule[-0.200pt]{2.409pt}{0.400pt}}
\put(170.0,535.0){\rule[-0.200pt]{2.409pt}{0.400pt}}
\put(1429.0,535.0){\rule[-0.200pt]{2.409pt}{0.400pt}}
\put(170.0,536.0){\rule[-0.200pt]{2.409pt}{0.400pt}}
\put(1429.0,536.0){\rule[-0.200pt]{2.409pt}{0.400pt}}
\put(170.0,536.0){\rule[-0.200pt]{2.409pt}{0.400pt}}
\put(1429.0,536.0){\rule[-0.200pt]{2.409pt}{0.400pt}}
\put(170.0,536.0){\rule[-0.200pt]{2.409pt}{0.400pt}}
\put(1429.0,536.0){\rule[-0.200pt]{2.409pt}{0.400pt}}
\put(170.0,536.0){\rule[-0.200pt]{2.409pt}{0.400pt}}
\put(1429.0,536.0){\rule[-0.200pt]{2.409pt}{0.400pt}}
\put(170.0,537.0){\rule[-0.200pt]{2.409pt}{0.400pt}}
\put(1429.0,537.0){\rule[-0.200pt]{2.409pt}{0.400pt}}
\put(170.0,537.0){\rule[-0.200pt]{2.409pt}{0.400pt}}
\put(1429.0,537.0){\rule[-0.200pt]{2.409pt}{0.400pt}}
\put(170.0,537.0){\rule[-0.200pt]{2.409pt}{0.400pt}}
\put(1429.0,537.0){\rule[-0.200pt]{2.409pt}{0.400pt}}
\put(170.0,537.0){\rule[-0.200pt]{2.409pt}{0.400pt}}
\put(1429.0,537.0){\rule[-0.200pt]{2.409pt}{0.400pt}}
\put(170.0,537.0){\rule[-0.200pt]{2.409pt}{0.400pt}}
\put(1429.0,537.0){\rule[-0.200pt]{2.409pt}{0.400pt}}
\put(170.0,538.0){\rule[-0.200pt]{2.409pt}{0.400pt}}
\put(1429.0,538.0){\rule[-0.200pt]{2.409pt}{0.400pt}}
\put(170.0,538.0){\rule[-0.200pt]{2.409pt}{0.400pt}}
\put(1429.0,538.0){\rule[-0.200pt]{2.409pt}{0.400pt}}
\put(170.0,538.0){\rule[-0.200pt]{2.409pt}{0.400pt}}
\put(1429.0,538.0){\rule[-0.200pt]{2.409pt}{0.400pt}}
\put(170.0,538.0){\rule[-0.200pt]{2.409pt}{0.400pt}}
\put(1429.0,538.0){\rule[-0.200pt]{2.409pt}{0.400pt}}
\put(170.0,538.0){\rule[-0.200pt]{2.409pt}{0.400pt}}
\put(1429.0,538.0){\rule[-0.200pt]{2.409pt}{0.400pt}}
\put(170.0,539.0){\rule[-0.200pt]{2.409pt}{0.400pt}}
\put(1429.0,539.0){\rule[-0.200pt]{2.409pt}{0.400pt}}
\put(170.0,539.0){\rule[-0.200pt]{2.409pt}{0.400pt}}
\put(1429.0,539.0){\rule[-0.200pt]{2.409pt}{0.400pt}}
\put(170.0,539.0){\rule[-0.200pt]{2.409pt}{0.400pt}}
\put(1429.0,539.0){\rule[-0.200pt]{2.409pt}{0.400pt}}
\put(170.0,539.0){\rule[-0.200pt]{2.409pt}{0.400pt}}
\put(1429.0,539.0){\rule[-0.200pt]{2.409pt}{0.400pt}}
\put(170.0,539.0){\rule[-0.200pt]{2.409pt}{0.400pt}}
\put(1429.0,539.0){\rule[-0.200pt]{2.409pt}{0.400pt}}
\put(170.0,539.0){\rule[-0.200pt]{2.409pt}{0.400pt}}
\put(1429.0,539.0){\rule[-0.200pt]{2.409pt}{0.400pt}}
\put(170.0,540.0){\rule[-0.200pt]{2.409pt}{0.400pt}}
\put(1429.0,540.0){\rule[-0.200pt]{2.409pt}{0.400pt}}
\put(170.0,540.0){\rule[-0.200pt]{2.409pt}{0.400pt}}
\put(1429.0,540.0){\rule[-0.200pt]{2.409pt}{0.400pt}}
\put(170.0,540.0){\rule[-0.200pt]{2.409pt}{0.400pt}}
\put(1429.0,540.0){\rule[-0.200pt]{2.409pt}{0.400pt}}
\put(170.0,540.0){\rule[-0.200pt]{2.409pt}{0.400pt}}
\put(1429.0,540.0){\rule[-0.200pt]{2.409pt}{0.400pt}}
\put(170.0,540.0){\rule[-0.200pt]{2.409pt}{0.400pt}}
\put(1429.0,540.0){\rule[-0.200pt]{2.409pt}{0.400pt}}
\put(170.0,541.0){\rule[-0.200pt]{2.409pt}{0.400pt}}
\put(1429.0,541.0){\rule[-0.200pt]{2.409pt}{0.400pt}}
\put(170.0,541.0){\rule[-0.200pt]{2.409pt}{0.400pt}}
\put(1429.0,541.0){\rule[-0.200pt]{2.409pt}{0.400pt}}
\put(170.0,541.0){\rule[-0.200pt]{2.409pt}{0.400pt}}
\put(1429.0,541.0){\rule[-0.200pt]{2.409pt}{0.400pt}}
\put(170.0,541.0){\rule[-0.200pt]{2.409pt}{0.400pt}}
\put(1429.0,541.0){\rule[-0.200pt]{2.409pt}{0.400pt}}
\put(170.0,541.0){\rule[-0.200pt]{2.409pt}{0.400pt}}
\put(1429.0,541.0){\rule[-0.200pt]{2.409pt}{0.400pt}}
\put(170.0,541.0){\rule[-0.200pt]{2.409pt}{0.400pt}}
\put(1429.0,541.0){\rule[-0.200pt]{2.409pt}{0.400pt}}
\put(170.0,542.0){\rule[-0.200pt]{2.409pt}{0.400pt}}
\put(1429.0,542.0){\rule[-0.200pt]{2.409pt}{0.400pt}}
\put(170.0,542.0){\rule[-0.200pt]{2.409pt}{0.400pt}}
\put(1429.0,542.0){\rule[-0.200pt]{2.409pt}{0.400pt}}
\put(170.0,542.0){\rule[-0.200pt]{2.409pt}{0.400pt}}
\put(1429.0,542.0){\rule[-0.200pt]{2.409pt}{0.400pt}}
\put(170.0,542.0){\rule[-0.200pt]{2.409pt}{0.400pt}}
\put(1429.0,542.0){\rule[-0.200pt]{2.409pt}{0.400pt}}
\put(170.0,542.0){\rule[-0.200pt]{2.409pt}{0.400pt}}
\put(1429.0,542.0){\rule[-0.200pt]{2.409pt}{0.400pt}}
\put(170.0,543.0){\rule[-0.200pt]{2.409pt}{0.400pt}}
\put(1429.0,543.0){\rule[-0.200pt]{2.409pt}{0.400pt}}
\put(170.0,543.0){\rule[-0.200pt]{2.409pt}{0.400pt}}
\put(1429.0,543.0){\rule[-0.200pt]{2.409pt}{0.400pt}}
\put(170.0,543.0){\rule[-0.200pt]{2.409pt}{0.400pt}}
\put(1429.0,543.0){\rule[-0.200pt]{2.409pt}{0.400pt}}
\put(170.0,543.0){\rule[-0.200pt]{2.409pt}{0.400pt}}
\put(1429.0,543.0){\rule[-0.200pt]{2.409pt}{0.400pt}}
\put(170.0,543.0){\rule[-0.200pt]{2.409pt}{0.400pt}}
\put(1429.0,543.0){\rule[-0.200pt]{2.409pt}{0.400pt}}
\put(170.0,543.0){\rule[-0.200pt]{2.409pt}{0.400pt}}
\put(1429.0,543.0){\rule[-0.200pt]{2.409pt}{0.400pt}}
\put(170.0,544.0){\rule[-0.200pt]{2.409pt}{0.400pt}}
\put(1429.0,544.0){\rule[-0.200pt]{2.409pt}{0.400pt}}
\put(170.0,544.0){\rule[-0.200pt]{2.409pt}{0.400pt}}
\put(1429.0,544.0){\rule[-0.200pt]{2.409pt}{0.400pt}}
\put(170.0,544.0){\rule[-0.200pt]{2.409pt}{0.400pt}}
\put(1429.0,544.0){\rule[-0.200pt]{2.409pt}{0.400pt}}
\put(170.0,544.0){\rule[-0.200pt]{2.409pt}{0.400pt}}
\put(1429.0,544.0){\rule[-0.200pt]{2.409pt}{0.400pt}}
\put(170.0,544.0){\rule[-0.200pt]{2.409pt}{0.400pt}}
\put(1429.0,544.0){\rule[-0.200pt]{2.409pt}{0.400pt}}
\put(170.0,544.0){\rule[-0.200pt]{2.409pt}{0.400pt}}
\put(1429.0,544.0){\rule[-0.200pt]{2.409pt}{0.400pt}}
\put(170.0,545.0){\rule[-0.200pt]{2.409pt}{0.400pt}}
\put(1429.0,545.0){\rule[-0.200pt]{2.409pt}{0.400pt}}
\put(170.0,545.0){\rule[-0.200pt]{2.409pt}{0.400pt}}
\put(1429.0,545.0){\rule[-0.200pt]{2.409pt}{0.400pt}}
\put(170.0,545.0){\rule[-0.200pt]{2.409pt}{0.400pt}}
\put(1429.0,545.0){\rule[-0.200pt]{2.409pt}{0.400pt}}
\put(170.0,545.0){\rule[-0.200pt]{2.409pt}{0.400pt}}
\put(1429.0,545.0){\rule[-0.200pt]{2.409pt}{0.400pt}}
\put(170.0,545.0){\rule[-0.200pt]{2.409pt}{0.400pt}}
\put(1429.0,545.0){\rule[-0.200pt]{2.409pt}{0.400pt}}
\put(170.0,545.0){\rule[-0.200pt]{2.409pt}{0.400pt}}
\put(1429.0,545.0){\rule[-0.200pt]{2.409pt}{0.400pt}}
\put(170.0,546.0){\rule[-0.200pt]{2.409pt}{0.400pt}}
\put(1429.0,546.0){\rule[-0.200pt]{2.409pt}{0.400pt}}
\put(170.0,546.0){\rule[-0.200pt]{2.409pt}{0.400pt}}
\put(1429.0,546.0){\rule[-0.200pt]{2.409pt}{0.400pt}}
\put(170.0,546.0){\rule[-0.200pt]{2.409pt}{0.400pt}}
\put(1429.0,546.0){\rule[-0.200pt]{2.409pt}{0.400pt}}
\put(170.0,546.0){\rule[-0.200pt]{2.409pt}{0.400pt}}
\put(1429.0,546.0){\rule[-0.200pt]{2.409pt}{0.400pt}}
\put(170.0,546.0){\rule[-0.200pt]{2.409pt}{0.400pt}}
\put(1429.0,546.0){\rule[-0.200pt]{2.409pt}{0.400pt}}
\put(170.0,546.0){\rule[-0.200pt]{2.409pt}{0.400pt}}
\put(1429.0,546.0){\rule[-0.200pt]{2.409pt}{0.400pt}}
\put(170.0,546.0){\rule[-0.200pt]{2.409pt}{0.400pt}}
\put(1429.0,546.0){\rule[-0.200pt]{2.409pt}{0.400pt}}
\put(170.0,547.0){\rule[-0.200pt]{2.409pt}{0.400pt}}
\put(1429.0,547.0){\rule[-0.200pt]{2.409pt}{0.400pt}}
\put(170.0,547.0){\rule[-0.200pt]{2.409pt}{0.400pt}}
\put(1429.0,547.0){\rule[-0.200pt]{2.409pt}{0.400pt}}
\put(170.0,547.0){\rule[-0.200pt]{2.409pt}{0.400pt}}
\put(1429.0,547.0){\rule[-0.200pt]{2.409pt}{0.400pt}}
\put(170.0,547.0){\rule[-0.200pt]{2.409pt}{0.400pt}}
\put(1429.0,547.0){\rule[-0.200pt]{2.409pt}{0.400pt}}
\put(170.0,547.0){\rule[-0.200pt]{2.409pt}{0.400pt}}
\put(1429.0,547.0){\rule[-0.200pt]{2.409pt}{0.400pt}}
\put(170.0,547.0){\rule[-0.200pt]{2.409pt}{0.400pt}}
\put(1429.0,547.0){\rule[-0.200pt]{2.409pt}{0.400pt}}
\put(170.0,548.0){\rule[-0.200pt]{2.409pt}{0.400pt}}
\put(1429.0,548.0){\rule[-0.200pt]{2.409pt}{0.400pt}}
\put(170.0,548.0){\rule[-0.200pt]{2.409pt}{0.400pt}}
\put(1429.0,548.0){\rule[-0.200pt]{2.409pt}{0.400pt}}
\put(170.0,548.0){\rule[-0.200pt]{2.409pt}{0.400pt}}
\put(1429.0,548.0){\rule[-0.200pt]{2.409pt}{0.400pt}}
\put(170.0,548.0){\rule[-0.200pt]{2.409pt}{0.400pt}}
\put(1429.0,548.0){\rule[-0.200pt]{2.409pt}{0.400pt}}
\put(170.0,548.0){\rule[-0.200pt]{2.409pt}{0.400pt}}
\put(1429.0,548.0){\rule[-0.200pt]{2.409pt}{0.400pt}}
\put(170.0,548.0){\rule[-0.200pt]{2.409pt}{0.400pt}}
\put(1429.0,548.0){\rule[-0.200pt]{2.409pt}{0.400pt}}
\put(170.0,548.0){\rule[-0.200pt]{2.409pt}{0.400pt}}
\put(1429.0,548.0){\rule[-0.200pt]{2.409pt}{0.400pt}}
\put(170.0,549.0){\rule[-0.200pt]{2.409pt}{0.400pt}}
\put(1429.0,549.0){\rule[-0.200pt]{2.409pt}{0.400pt}}
\put(170.0,549.0){\rule[-0.200pt]{2.409pt}{0.400pt}}
\put(1429.0,549.0){\rule[-0.200pt]{2.409pt}{0.400pt}}
\put(170.0,549.0){\rule[-0.200pt]{2.409pt}{0.400pt}}
\put(1429.0,549.0){\rule[-0.200pt]{2.409pt}{0.400pt}}
\put(170.0,549.0){\rule[-0.200pt]{2.409pt}{0.400pt}}
\put(1429.0,549.0){\rule[-0.200pt]{2.409pt}{0.400pt}}
\put(170.0,549.0){\rule[-0.200pt]{2.409pt}{0.400pt}}
\put(1429.0,549.0){\rule[-0.200pt]{2.409pt}{0.400pt}}
\put(170.0,549.0){\rule[-0.200pt]{2.409pt}{0.400pt}}
\put(1429.0,549.0){\rule[-0.200pt]{2.409pt}{0.400pt}}
\put(170.0,549.0){\rule[-0.200pt]{2.409pt}{0.400pt}}
\put(1429.0,549.0){\rule[-0.200pt]{2.409pt}{0.400pt}}
\put(170.0,550.0){\rule[-0.200pt]{2.409pt}{0.400pt}}
\put(1429.0,550.0){\rule[-0.200pt]{2.409pt}{0.400pt}}
\put(170.0,550.0){\rule[-0.200pt]{2.409pt}{0.400pt}}
\put(1429.0,550.0){\rule[-0.200pt]{2.409pt}{0.400pt}}
\put(170.0,550.0){\rule[-0.200pt]{2.409pt}{0.400pt}}
\put(1429.0,550.0){\rule[-0.200pt]{2.409pt}{0.400pt}}
\put(170.0,550.0){\rule[-0.200pt]{2.409pt}{0.400pt}}
\put(1429.0,550.0){\rule[-0.200pt]{2.409pt}{0.400pt}}
\put(170.0,550.0){\rule[-0.200pt]{2.409pt}{0.400pt}}
\put(1429.0,550.0){\rule[-0.200pt]{2.409pt}{0.400pt}}
\put(170.0,550.0){\rule[-0.200pt]{2.409pt}{0.400pt}}
\put(1429.0,550.0){\rule[-0.200pt]{2.409pt}{0.400pt}}
\put(170.0,550.0){\rule[-0.200pt]{2.409pt}{0.400pt}}
\put(1429.0,550.0){\rule[-0.200pt]{2.409pt}{0.400pt}}
\put(170.0,551.0){\rule[-0.200pt]{2.409pt}{0.400pt}}
\put(1429.0,551.0){\rule[-0.200pt]{2.409pt}{0.400pt}}
\put(170.0,551.0){\rule[-0.200pt]{2.409pt}{0.400pt}}
\put(1429.0,551.0){\rule[-0.200pt]{2.409pt}{0.400pt}}
\put(170.0,551.0){\rule[-0.200pt]{2.409pt}{0.400pt}}
\put(1429.0,551.0){\rule[-0.200pt]{2.409pt}{0.400pt}}
\put(170.0,551.0){\rule[-0.200pt]{2.409pt}{0.400pt}}
\put(1429.0,551.0){\rule[-0.200pt]{2.409pt}{0.400pt}}
\put(170.0,551.0){\rule[-0.200pt]{2.409pt}{0.400pt}}
\put(1429.0,551.0){\rule[-0.200pt]{2.409pt}{0.400pt}}
\put(170.0,551.0){\rule[-0.200pt]{2.409pt}{0.400pt}}
\put(1429.0,551.0){\rule[-0.200pt]{2.409pt}{0.400pt}}
\put(170.0,551.0){\rule[-0.200pt]{2.409pt}{0.400pt}}
\put(1429.0,551.0){\rule[-0.200pt]{2.409pt}{0.400pt}}
\put(170.0,552.0){\rule[-0.200pt]{2.409pt}{0.400pt}}
\put(1429.0,552.0){\rule[-0.200pt]{2.409pt}{0.400pt}}
\put(170.0,552.0){\rule[-0.200pt]{2.409pt}{0.400pt}}
\put(1429.0,552.0){\rule[-0.200pt]{2.409pt}{0.400pt}}
\put(170.0,552.0){\rule[-0.200pt]{2.409pt}{0.400pt}}
\put(1429.0,552.0){\rule[-0.200pt]{2.409pt}{0.400pt}}
\put(170.0,552.0){\rule[-0.200pt]{2.409pt}{0.400pt}}
\put(1429.0,552.0){\rule[-0.200pt]{2.409pt}{0.400pt}}
\put(170.0,552.0){\rule[-0.200pt]{2.409pt}{0.400pt}}
\put(1429.0,552.0){\rule[-0.200pt]{2.409pt}{0.400pt}}
\put(170.0,552.0){\rule[-0.200pt]{2.409pt}{0.400pt}}
\put(1429.0,552.0){\rule[-0.200pt]{2.409pt}{0.400pt}}
\put(170.0,552.0){\rule[-0.200pt]{2.409pt}{0.400pt}}
\put(1429.0,552.0){\rule[-0.200pt]{2.409pt}{0.400pt}}
\put(170.0,553.0){\rule[-0.200pt]{2.409pt}{0.400pt}}
\put(1429.0,553.0){\rule[-0.200pt]{2.409pt}{0.400pt}}
\put(170.0,553.0){\rule[-0.200pt]{2.409pt}{0.400pt}}
\put(1429.0,553.0){\rule[-0.200pt]{2.409pt}{0.400pt}}
\put(170.0,553.0){\rule[-0.200pt]{2.409pt}{0.400pt}}
\put(1429.0,553.0){\rule[-0.200pt]{2.409pt}{0.400pt}}
\put(170.0,553.0){\rule[-0.200pt]{2.409pt}{0.400pt}}
\put(1429.0,553.0){\rule[-0.200pt]{2.409pt}{0.400pt}}
\put(170.0,553.0){\rule[-0.200pt]{2.409pt}{0.400pt}}
\put(1429.0,553.0){\rule[-0.200pt]{2.409pt}{0.400pt}}
\put(170.0,553.0){\rule[-0.200pt]{2.409pt}{0.400pt}}
\put(1429.0,553.0){\rule[-0.200pt]{2.409pt}{0.400pt}}
\put(170.0,553.0){\rule[-0.200pt]{2.409pt}{0.400pt}}
\put(1429.0,553.0){\rule[-0.200pt]{2.409pt}{0.400pt}}
\put(170.0,553.0){\rule[-0.200pt]{2.409pt}{0.400pt}}
\put(1429.0,553.0){\rule[-0.200pt]{2.409pt}{0.400pt}}
\put(170.0,554.0){\rule[-0.200pt]{2.409pt}{0.400pt}}
\put(1429.0,554.0){\rule[-0.200pt]{2.409pt}{0.400pt}}
\put(170.0,554.0){\rule[-0.200pt]{2.409pt}{0.400pt}}
\put(1429.0,554.0){\rule[-0.200pt]{2.409pt}{0.400pt}}
\put(170.0,554.0){\rule[-0.200pt]{2.409pt}{0.400pt}}
\put(1429.0,554.0){\rule[-0.200pt]{2.409pt}{0.400pt}}
\put(170.0,554.0){\rule[-0.200pt]{2.409pt}{0.400pt}}
\put(1429.0,554.0){\rule[-0.200pt]{2.409pt}{0.400pt}}
\put(170.0,554.0){\rule[-0.200pt]{2.409pt}{0.400pt}}
\put(1429.0,554.0){\rule[-0.200pt]{2.409pt}{0.400pt}}
\put(170.0,554.0){\rule[-0.200pt]{2.409pt}{0.400pt}}
\put(1429.0,554.0){\rule[-0.200pt]{2.409pt}{0.400pt}}
\put(170.0,554.0){\rule[-0.200pt]{2.409pt}{0.400pt}}
\put(1429.0,554.0){\rule[-0.200pt]{2.409pt}{0.400pt}}
\put(170.0,554.0){\rule[-0.200pt]{2.409pt}{0.400pt}}
\put(1429.0,554.0){\rule[-0.200pt]{2.409pt}{0.400pt}}
\put(170.0,555.0){\rule[-0.200pt]{2.409pt}{0.400pt}}
\put(1429.0,555.0){\rule[-0.200pt]{2.409pt}{0.400pt}}
\put(170.0,555.0){\rule[-0.200pt]{2.409pt}{0.400pt}}
\put(1429.0,555.0){\rule[-0.200pt]{2.409pt}{0.400pt}}
\put(170.0,555.0){\rule[-0.200pt]{2.409pt}{0.400pt}}
\put(1429.0,555.0){\rule[-0.200pt]{2.409pt}{0.400pt}}
\put(170.0,555.0){\rule[-0.200pt]{2.409pt}{0.400pt}}
\put(1429.0,555.0){\rule[-0.200pt]{2.409pt}{0.400pt}}
\put(170.0,555.0){\rule[-0.200pt]{2.409pt}{0.400pt}}
\put(1429.0,555.0){\rule[-0.200pt]{2.409pt}{0.400pt}}
\put(170.0,555.0){\rule[-0.200pt]{2.409pt}{0.400pt}}
\put(1429.0,555.0){\rule[-0.200pt]{2.409pt}{0.400pt}}
\put(170.0,555.0){\rule[-0.200pt]{2.409pt}{0.400pt}}
\put(1429.0,555.0){\rule[-0.200pt]{2.409pt}{0.400pt}}
\put(170.0,555.0){\rule[-0.200pt]{2.409pt}{0.400pt}}
\put(1429.0,555.0){\rule[-0.200pt]{2.409pt}{0.400pt}}
\put(170.0,556.0){\rule[-0.200pt]{2.409pt}{0.400pt}}
\put(1429.0,556.0){\rule[-0.200pt]{2.409pt}{0.400pt}}
\put(170.0,556.0){\rule[-0.200pt]{2.409pt}{0.400pt}}
\put(1429.0,556.0){\rule[-0.200pt]{2.409pt}{0.400pt}}
\put(170.0,556.0){\rule[-0.200pt]{2.409pt}{0.400pt}}
\put(1429.0,556.0){\rule[-0.200pt]{2.409pt}{0.400pt}}
\put(170.0,556.0){\rule[-0.200pt]{2.409pt}{0.400pt}}
\put(1429.0,556.0){\rule[-0.200pt]{2.409pt}{0.400pt}}
\put(170.0,556.0){\rule[-0.200pt]{2.409pt}{0.400pt}}
\put(1429.0,556.0){\rule[-0.200pt]{2.409pt}{0.400pt}}
\put(170.0,556.0){\rule[-0.200pt]{2.409pt}{0.400pt}}
\put(1429.0,556.0){\rule[-0.200pt]{2.409pt}{0.400pt}}
\put(170.0,556.0){\rule[-0.200pt]{2.409pt}{0.400pt}}
\put(1429.0,556.0){\rule[-0.200pt]{2.409pt}{0.400pt}}
\put(170.0,556.0){\rule[-0.200pt]{2.409pt}{0.400pt}}
\put(1429.0,556.0){\rule[-0.200pt]{2.409pt}{0.400pt}}
\put(170.0,557.0){\rule[-0.200pt]{2.409pt}{0.400pt}}
\put(1429.0,557.0){\rule[-0.200pt]{2.409pt}{0.400pt}}
\put(170.0,557.0){\rule[-0.200pt]{2.409pt}{0.400pt}}
\put(1429.0,557.0){\rule[-0.200pt]{2.409pt}{0.400pt}}
\put(170.0,557.0){\rule[-0.200pt]{2.409pt}{0.400pt}}
\put(1429.0,557.0){\rule[-0.200pt]{2.409pt}{0.400pt}}
\put(170.0,557.0){\rule[-0.200pt]{2.409pt}{0.400pt}}
\put(1429.0,557.0){\rule[-0.200pt]{2.409pt}{0.400pt}}
\put(170.0,557.0){\rule[-0.200pt]{2.409pt}{0.400pt}}
\put(1429.0,557.0){\rule[-0.200pt]{2.409pt}{0.400pt}}
\put(170.0,557.0){\rule[-0.200pt]{2.409pt}{0.400pt}}
\put(1429.0,557.0){\rule[-0.200pt]{2.409pt}{0.400pt}}
\put(170.0,557.0){\rule[-0.200pt]{2.409pt}{0.400pt}}
\put(1429.0,557.0){\rule[-0.200pt]{2.409pt}{0.400pt}}
\put(170.0,557.0){\rule[-0.200pt]{2.409pt}{0.400pt}}
\put(1429.0,557.0){\rule[-0.200pt]{2.409pt}{0.400pt}}
\put(170.0,558.0){\rule[-0.200pt]{2.409pt}{0.400pt}}
\put(1429.0,558.0){\rule[-0.200pt]{2.409pt}{0.400pt}}
\put(170.0,558.0){\rule[-0.200pt]{2.409pt}{0.400pt}}
\put(1429.0,558.0){\rule[-0.200pt]{2.409pt}{0.400pt}}
\put(170.0,558.0){\rule[-0.200pt]{2.409pt}{0.400pt}}
\put(1429.0,558.0){\rule[-0.200pt]{2.409pt}{0.400pt}}
\put(170.0,558.0){\rule[-0.200pt]{2.409pt}{0.400pt}}
\put(1429.0,558.0){\rule[-0.200pt]{2.409pt}{0.400pt}}
\put(170.0,558.0){\rule[-0.200pt]{2.409pt}{0.400pt}}
\put(1429.0,558.0){\rule[-0.200pt]{2.409pt}{0.400pt}}
\put(170.0,558.0){\rule[-0.200pt]{2.409pt}{0.400pt}}
\put(1429.0,558.0){\rule[-0.200pt]{2.409pt}{0.400pt}}
\put(170.0,558.0){\rule[-0.200pt]{2.409pt}{0.400pt}}
\put(1429.0,558.0){\rule[-0.200pt]{2.409pt}{0.400pt}}
\put(170.0,558.0){\rule[-0.200pt]{2.409pt}{0.400pt}}
\put(1429.0,558.0){\rule[-0.200pt]{2.409pt}{0.400pt}}
\put(170.0,558.0){\rule[-0.200pt]{2.409pt}{0.400pt}}
\put(1429.0,558.0){\rule[-0.200pt]{2.409pt}{0.400pt}}
\put(170.0,559.0){\rule[-0.200pt]{2.409pt}{0.400pt}}
\put(1429.0,559.0){\rule[-0.200pt]{2.409pt}{0.400pt}}
\put(170.0,559.0){\rule[-0.200pt]{2.409pt}{0.400pt}}
\put(1429.0,559.0){\rule[-0.200pt]{2.409pt}{0.400pt}}
\put(170.0,559.0){\rule[-0.200pt]{2.409pt}{0.400pt}}
\put(1429.0,559.0){\rule[-0.200pt]{2.409pt}{0.400pt}}
\put(170.0,559.0){\rule[-0.200pt]{2.409pt}{0.400pt}}
\put(1429.0,559.0){\rule[-0.200pt]{2.409pt}{0.400pt}}
\put(170.0,559.0){\rule[-0.200pt]{2.409pt}{0.400pt}}
\put(1429.0,559.0){\rule[-0.200pt]{2.409pt}{0.400pt}}
\put(170.0,559.0){\rule[-0.200pt]{2.409pt}{0.400pt}}
\put(1429.0,559.0){\rule[-0.200pt]{2.409pt}{0.400pt}}
\put(170.0,559.0){\rule[-0.200pt]{2.409pt}{0.400pt}}
\put(1429.0,559.0){\rule[-0.200pt]{2.409pt}{0.400pt}}
\put(170.0,559.0){\rule[-0.200pt]{2.409pt}{0.400pt}}
\put(1429.0,559.0){\rule[-0.200pt]{2.409pt}{0.400pt}}
\put(170.0,559.0){\rule[-0.200pt]{2.409pt}{0.400pt}}
\put(1429.0,559.0){\rule[-0.200pt]{2.409pt}{0.400pt}}
\put(170.0,560.0){\rule[-0.200pt]{2.409pt}{0.400pt}}
\put(1429.0,560.0){\rule[-0.200pt]{2.409pt}{0.400pt}}
\put(170.0,560.0){\rule[-0.200pt]{2.409pt}{0.400pt}}
\put(1429.0,560.0){\rule[-0.200pt]{2.409pt}{0.400pt}}
\put(170.0,560.0){\rule[-0.200pt]{2.409pt}{0.400pt}}
\put(1429.0,560.0){\rule[-0.200pt]{2.409pt}{0.400pt}}
\put(170.0,560.0){\rule[-0.200pt]{2.409pt}{0.400pt}}
\put(1429.0,560.0){\rule[-0.200pt]{2.409pt}{0.400pt}}
\put(170.0,560.0){\rule[-0.200pt]{2.409pt}{0.400pt}}
\put(1429.0,560.0){\rule[-0.200pt]{2.409pt}{0.400pt}}
\put(170.0,560.0){\rule[-0.200pt]{2.409pt}{0.400pt}}
\put(1429.0,560.0){\rule[-0.200pt]{2.409pt}{0.400pt}}
\put(170.0,560.0){\rule[-0.200pt]{2.409pt}{0.400pt}}
\put(1429.0,560.0){\rule[-0.200pt]{2.409pt}{0.400pt}}
\put(170.0,560.0){\rule[-0.200pt]{2.409pt}{0.400pt}}
\put(1429.0,560.0){\rule[-0.200pt]{2.409pt}{0.400pt}}
\put(170.0,560.0){\rule[-0.200pt]{2.409pt}{0.400pt}}
\put(1429.0,560.0){\rule[-0.200pt]{2.409pt}{0.400pt}}
\put(170.0,561.0){\rule[-0.200pt]{2.409pt}{0.400pt}}
\put(1429.0,561.0){\rule[-0.200pt]{2.409pt}{0.400pt}}
\put(170.0,561.0){\rule[-0.200pt]{2.409pt}{0.400pt}}
\put(1429.0,561.0){\rule[-0.200pt]{2.409pt}{0.400pt}}
\put(170.0,561.0){\rule[-0.200pt]{2.409pt}{0.400pt}}
\put(1429.0,561.0){\rule[-0.200pt]{2.409pt}{0.400pt}}
\put(170.0,561.0){\rule[-0.200pt]{2.409pt}{0.400pt}}
\put(1429.0,561.0){\rule[-0.200pt]{2.409pt}{0.400pt}}
\put(170.0,561.0){\rule[-0.200pt]{2.409pt}{0.400pt}}
\put(1429.0,561.0){\rule[-0.200pt]{2.409pt}{0.400pt}}
\put(170.0,561.0){\rule[-0.200pt]{2.409pt}{0.400pt}}
\put(1429.0,561.0){\rule[-0.200pt]{2.409pt}{0.400pt}}
\put(170.0,561.0){\rule[-0.200pt]{2.409pt}{0.400pt}}
\put(1429.0,561.0){\rule[-0.200pt]{2.409pt}{0.400pt}}
\put(170.0,561.0){\rule[-0.200pt]{2.409pt}{0.400pt}}
\put(1429.0,561.0){\rule[-0.200pt]{2.409pt}{0.400pt}}
\put(170.0,561.0){\rule[-0.200pt]{2.409pt}{0.400pt}}
\put(1429.0,561.0){\rule[-0.200pt]{2.409pt}{0.400pt}}
\put(170.0,561.0){\rule[-0.200pt]{2.409pt}{0.400pt}}
\put(1429.0,561.0){\rule[-0.200pt]{2.409pt}{0.400pt}}
\put(170.0,562.0){\rule[-0.200pt]{2.409pt}{0.400pt}}
\put(1429.0,562.0){\rule[-0.200pt]{2.409pt}{0.400pt}}
\put(170.0,562.0){\rule[-0.200pt]{2.409pt}{0.400pt}}
\put(1429.0,562.0){\rule[-0.200pt]{2.409pt}{0.400pt}}
\put(170.0,562.0){\rule[-0.200pt]{2.409pt}{0.400pt}}
\put(1429.0,562.0){\rule[-0.200pt]{2.409pt}{0.400pt}}
\put(170.0,562.0){\rule[-0.200pt]{2.409pt}{0.400pt}}
\put(1429.0,562.0){\rule[-0.200pt]{2.409pt}{0.400pt}}
\put(170.0,562.0){\rule[-0.200pt]{2.409pt}{0.400pt}}
\put(1429.0,562.0){\rule[-0.200pt]{2.409pt}{0.400pt}}
\put(170.0,562.0){\rule[-0.200pt]{2.409pt}{0.400pt}}
\put(1429.0,562.0){\rule[-0.200pt]{2.409pt}{0.400pt}}
\put(170.0,562.0){\rule[-0.200pt]{2.409pt}{0.400pt}}
\put(1429.0,562.0){\rule[-0.200pt]{2.409pt}{0.400pt}}
\put(170.0,562.0){\rule[-0.200pt]{2.409pt}{0.400pt}}
\put(1429.0,562.0){\rule[-0.200pt]{2.409pt}{0.400pt}}
\put(170.0,562.0){\rule[-0.200pt]{2.409pt}{0.400pt}}
\put(1429.0,562.0){\rule[-0.200pt]{2.409pt}{0.400pt}}
\put(170.0,563.0){\rule[-0.200pt]{2.409pt}{0.400pt}}
\put(1429.0,563.0){\rule[-0.200pt]{2.409pt}{0.400pt}}
\put(170.0,563.0){\rule[-0.200pt]{2.409pt}{0.400pt}}
\put(1429.0,563.0){\rule[-0.200pt]{2.409pt}{0.400pt}}
\put(170.0,563.0){\rule[-0.200pt]{2.409pt}{0.400pt}}
\put(1429.0,563.0){\rule[-0.200pt]{2.409pt}{0.400pt}}
\put(170.0,563.0){\rule[-0.200pt]{2.409pt}{0.400pt}}
\put(1429.0,563.0){\rule[-0.200pt]{2.409pt}{0.400pt}}
\put(170.0,563.0){\rule[-0.200pt]{2.409pt}{0.400pt}}
\put(1429.0,563.0){\rule[-0.200pt]{2.409pt}{0.400pt}}
\put(170.0,563.0){\rule[-0.200pt]{2.409pt}{0.400pt}}
\put(1429.0,563.0){\rule[-0.200pt]{2.409pt}{0.400pt}}
\put(170.0,563.0){\rule[-0.200pt]{2.409pt}{0.400pt}}
\put(1429.0,563.0){\rule[-0.200pt]{2.409pt}{0.400pt}}
\put(170.0,563.0){\rule[-0.200pt]{2.409pt}{0.400pt}}
\put(1429.0,563.0){\rule[-0.200pt]{2.409pt}{0.400pt}}
\put(170.0,563.0){\rule[-0.200pt]{2.409pt}{0.400pt}}
\put(1429.0,563.0){\rule[-0.200pt]{2.409pt}{0.400pt}}
\put(170.0,563.0){\rule[-0.200pt]{2.409pt}{0.400pt}}
\put(1429.0,563.0){\rule[-0.200pt]{2.409pt}{0.400pt}}
\put(170.0,564.0){\rule[-0.200pt]{2.409pt}{0.400pt}}
\put(1429.0,564.0){\rule[-0.200pt]{2.409pt}{0.400pt}}
\put(170.0,564.0){\rule[-0.200pt]{2.409pt}{0.400pt}}
\put(1429.0,564.0){\rule[-0.200pt]{2.409pt}{0.400pt}}
\put(170.0,564.0){\rule[-0.200pt]{2.409pt}{0.400pt}}
\put(1429.0,564.0){\rule[-0.200pt]{2.409pt}{0.400pt}}
\put(170.0,564.0){\rule[-0.200pt]{2.409pt}{0.400pt}}
\put(1429.0,564.0){\rule[-0.200pt]{2.409pt}{0.400pt}}
\put(170.0,564.0){\rule[-0.200pt]{2.409pt}{0.400pt}}
\put(1429.0,564.0){\rule[-0.200pt]{2.409pt}{0.400pt}}
\put(170.0,564.0){\rule[-0.200pt]{2.409pt}{0.400pt}}
\put(1429.0,564.0){\rule[-0.200pt]{2.409pt}{0.400pt}}
\put(170.0,564.0){\rule[-0.200pt]{2.409pt}{0.400pt}}
\put(1429.0,564.0){\rule[-0.200pt]{2.409pt}{0.400pt}}
\put(170.0,564.0){\rule[-0.200pt]{2.409pt}{0.400pt}}
\put(1429.0,564.0){\rule[-0.200pt]{2.409pt}{0.400pt}}
\put(170.0,564.0){\rule[-0.200pt]{2.409pt}{0.400pt}}
\put(1429.0,564.0){\rule[-0.200pt]{2.409pt}{0.400pt}}
\put(170.0,564.0){\rule[-0.200pt]{2.409pt}{0.400pt}}
\put(1429.0,564.0){\rule[-0.200pt]{2.409pt}{0.400pt}}
\put(170.0,565.0){\rule[-0.200pt]{2.409pt}{0.400pt}}
\put(1429.0,565.0){\rule[-0.200pt]{2.409pt}{0.400pt}}
\put(170.0,565.0){\rule[-0.200pt]{2.409pt}{0.400pt}}
\put(1429.0,565.0){\rule[-0.200pt]{2.409pt}{0.400pt}}
\put(170.0,565.0){\rule[-0.200pt]{2.409pt}{0.400pt}}
\put(1429.0,565.0){\rule[-0.200pt]{2.409pt}{0.400pt}}
\put(170.0,565.0){\rule[-0.200pt]{2.409pt}{0.400pt}}
\put(1429.0,565.0){\rule[-0.200pt]{2.409pt}{0.400pt}}
\put(170.0,565.0){\rule[-0.200pt]{2.409pt}{0.400pt}}
\put(1429.0,565.0){\rule[-0.200pt]{2.409pt}{0.400pt}}
\put(170.0,565.0){\rule[-0.200pt]{2.409pt}{0.400pt}}
\put(1429.0,565.0){\rule[-0.200pt]{2.409pt}{0.400pt}}
\put(170.0,565.0){\rule[-0.200pt]{2.409pt}{0.400pt}}
\put(1429.0,565.0){\rule[-0.200pt]{2.409pt}{0.400pt}}
\put(170.0,565.0){\rule[-0.200pt]{2.409pt}{0.400pt}}
\put(1429.0,565.0){\rule[-0.200pt]{2.409pt}{0.400pt}}
\put(170.0,565.0){\rule[-0.200pt]{2.409pt}{0.400pt}}
\put(1429.0,565.0){\rule[-0.200pt]{2.409pt}{0.400pt}}
\put(170.0,565.0){\rule[-0.200pt]{2.409pt}{0.400pt}}
\put(1429.0,565.0){\rule[-0.200pt]{2.409pt}{0.400pt}}
\put(170.0,565.0){\rule[-0.200pt]{2.409pt}{0.400pt}}
\put(1429.0,565.0){\rule[-0.200pt]{2.409pt}{0.400pt}}
\put(170.0,566.0){\rule[-0.200pt]{2.409pt}{0.400pt}}
\put(1429.0,566.0){\rule[-0.200pt]{2.409pt}{0.400pt}}
\put(170.0,566.0){\rule[-0.200pt]{2.409pt}{0.400pt}}
\put(1429.0,566.0){\rule[-0.200pt]{2.409pt}{0.400pt}}
\put(170.0,566.0){\rule[-0.200pt]{2.409pt}{0.400pt}}
\put(1429.0,566.0){\rule[-0.200pt]{2.409pt}{0.400pt}}
\put(170.0,566.0){\rule[-0.200pt]{2.409pt}{0.400pt}}
\put(1429.0,566.0){\rule[-0.200pt]{2.409pt}{0.400pt}}
\put(170.0,566.0){\rule[-0.200pt]{2.409pt}{0.400pt}}
\put(1429.0,566.0){\rule[-0.200pt]{2.409pt}{0.400pt}}
\put(170.0,566.0){\rule[-0.200pt]{2.409pt}{0.400pt}}
\put(1429.0,566.0){\rule[-0.200pt]{2.409pt}{0.400pt}}
\put(170.0,566.0){\rule[-0.200pt]{2.409pt}{0.400pt}}
\put(1429.0,566.0){\rule[-0.200pt]{2.409pt}{0.400pt}}
\put(170.0,566.0){\rule[-0.200pt]{2.409pt}{0.400pt}}
\put(1429.0,566.0){\rule[-0.200pt]{2.409pt}{0.400pt}}
\put(170.0,566.0){\rule[-0.200pt]{2.409pt}{0.400pt}}
\put(1429.0,566.0){\rule[-0.200pt]{2.409pt}{0.400pt}}
\put(170.0,566.0){\rule[-0.200pt]{2.409pt}{0.400pt}}
\put(1429.0,566.0){\rule[-0.200pt]{2.409pt}{0.400pt}}
\put(170.0,566.0){\rule[-0.200pt]{2.409pt}{0.400pt}}
\put(1429.0,566.0){\rule[-0.200pt]{2.409pt}{0.400pt}}
\put(170.0,567.0){\rule[-0.200pt]{2.409pt}{0.400pt}}
\put(1429.0,567.0){\rule[-0.200pt]{2.409pt}{0.400pt}}
\put(170.0,567.0){\rule[-0.200pt]{2.409pt}{0.400pt}}
\put(1429.0,567.0){\rule[-0.200pt]{2.409pt}{0.400pt}}
\put(170.0,567.0){\rule[-0.200pt]{2.409pt}{0.400pt}}
\put(1429.0,567.0){\rule[-0.200pt]{2.409pt}{0.400pt}}
\put(170.0,567.0){\rule[-0.200pt]{2.409pt}{0.400pt}}
\put(1429.0,567.0){\rule[-0.200pt]{2.409pt}{0.400pt}}
\put(170.0,567.0){\rule[-0.200pt]{2.409pt}{0.400pt}}
\put(1429.0,567.0){\rule[-0.200pt]{2.409pt}{0.400pt}}
\put(170.0,567.0){\rule[-0.200pt]{2.409pt}{0.400pt}}
\put(1429.0,567.0){\rule[-0.200pt]{2.409pt}{0.400pt}}
\put(170.0,567.0){\rule[-0.200pt]{2.409pt}{0.400pt}}
\put(1429.0,567.0){\rule[-0.200pt]{2.409pt}{0.400pt}}
\put(170.0,567.0){\rule[-0.200pt]{2.409pt}{0.400pt}}
\put(1429.0,567.0){\rule[-0.200pt]{2.409pt}{0.400pt}}
\put(170.0,567.0){\rule[-0.200pt]{2.409pt}{0.400pt}}
\put(1429.0,567.0){\rule[-0.200pt]{2.409pt}{0.400pt}}
\put(170.0,567.0){\rule[-0.200pt]{2.409pt}{0.400pt}}
\put(1429.0,567.0){\rule[-0.200pt]{2.409pt}{0.400pt}}
\put(170.0,567.0){\rule[-0.200pt]{2.409pt}{0.400pt}}
\put(1429.0,567.0){\rule[-0.200pt]{2.409pt}{0.400pt}}
\put(170.0,568.0){\rule[-0.200pt]{2.409pt}{0.400pt}}
\put(1429.0,568.0){\rule[-0.200pt]{2.409pt}{0.400pt}}
\put(170.0,568.0){\rule[-0.200pt]{2.409pt}{0.400pt}}
\put(1429.0,568.0){\rule[-0.200pt]{2.409pt}{0.400pt}}
\put(170.0,568.0){\rule[-0.200pt]{2.409pt}{0.400pt}}
\put(1429.0,568.0){\rule[-0.200pt]{2.409pt}{0.400pt}}
\put(170.0,568.0){\rule[-0.200pt]{2.409pt}{0.400pt}}
\put(1429.0,568.0){\rule[-0.200pt]{2.409pt}{0.400pt}}
\put(170.0,568.0){\rule[-0.200pt]{2.409pt}{0.400pt}}
\put(1429.0,568.0){\rule[-0.200pt]{2.409pt}{0.400pt}}
\put(170.0,568.0){\rule[-0.200pt]{2.409pt}{0.400pt}}
\put(1429.0,568.0){\rule[-0.200pt]{2.409pt}{0.400pt}}
\put(170.0,568.0){\rule[-0.200pt]{2.409pt}{0.400pt}}
\put(1429.0,568.0){\rule[-0.200pt]{2.409pt}{0.400pt}}
\put(170.0,568.0){\rule[-0.200pt]{2.409pt}{0.400pt}}
\put(1429.0,568.0){\rule[-0.200pt]{2.409pt}{0.400pt}}
\put(170.0,568.0){\rule[-0.200pt]{2.409pt}{0.400pt}}
\put(1429.0,568.0){\rule[-0.200pt]{2.409pt}{0.400pt}}
\put(170.0,568.0){\rule[-0.200pt]{2.409pt}{0.400pt}}
\put(1429.0,568.0){\rule[-0.200pt]{2.409pt}{0.400pt}}
\put(170.0,568.0){\rule[-0.200pt]{2.409pt}{0.400pt}}
\put(1429.0,568.0){\rule[-0.200pt]{2.409pt}{0.400pt}}
\put(170.0,569.0){\rule[-0.200pt]{2.409pt}{0.400pt}}
\put(1429.0,569.0){\rule[-0.200pt]{2.409pt}{0.400pt}}
\put(170.0,569.0){\rule[-0.200pt]{2.409pt}{0.400pt}}
\put(1429.0,569.0){\rule[-0.200pt]{2.409pt}{0.400pt}}
\put(170.0,569.0){\rule[-0.200pt]{2.409pt}{0.400pt}}
\put(1429.0,569.0){\rule[-0.200pt]{2.409pt}{0.400pt}}
\put(170.0,569.0){\rule[-0.200pt]{2.409pt}{0.400pt}}
\put(1429.0,569.0){\rule[-0.200pt]{2.409pt}{0.400pt}}
\put(170.0,569.0){\rule[-0.200pt]{2.409pt}{0.400pt}}
\put(1429.0,569.0){\rule[-0.200pt]{2.409pt}{0.400pt}}
\put(170.0,569.0){\rule[-0.200pt]{2.409pt}{0.400pt}}
\put(1429.0,569.0){\rule[-0.200pt]{2.409pt}{0.400pt}}
\put(170.0,569.0){\rule[-0.200pt]{2.409pt}{0.400pt}}
\put(1429.0,569.0){\rule[-0.200pt]{2.409pt}{0.400pt}}
\put(170.0,569.0){\rule[-0.200pt]{2.409pt}{0.400pt}}
\put(1429.0,569.0){\rule[-0.200pt]{2.409pt}{0.400pt}}
\put(170.0,569.0){\rule[-0.200pt]{2.409pt}{0.400pt}}
\put(1429.0,569.0){\rule[-0.200pt]{2.409pt}{0.400pt}}
\put(170.0,569.0){\rule[-0.200pt]{2.409pt}{0.400pt}}
\put(1429.0,569.0){\rule[-0.200pt]{2.409pt}{0.400pt}}
\put(170.0,569.0){\rule[-0.200pt]{2.409pt}{0.400pt}}
\put(1429.0,569.0){\rule[-0.200pt]{2.409pt}{0.400pt}}
\put(170.0,569.0){\rule[-0.200pt]{2.409pt}{0.400pt}}
\put(1429.0,569.0){\rule[-0.200pt]{2.409pt}{0.400pt}}
\put(170.0,570.0){\rule[-0.200pt]{2.409pt}{0.400pt}}
\put(1429.0,570.0){\rule[-0.200pt]{2.409pt}{0.400pt}}
\put(170.0,570.0){\rule[-0.200pt]{2.409pt}{0.400pt}}
\put(1429.0,570.0){\rule[-0.200pt]{2.409pt}{0.400pt}}
\put(170.0,570.0){\rule[-0.200pt]{2.409pt}{0.400pt}}
\put(1429.0,570.0){\rule[-0.200pt]{2.409pt}{0.400pt}}
\put(170.0,570.0){\rule[-0.200pt]{2.409pt}{0.400pt}}
\put(1429.0,570.0){\rule[-0.200pt]{2.409pt}{0.400pt}}
\put(170.0,570.0){\rule[-0.200pt]{2.409pt}{0.400pt}}
\put(1429.0,570.0){\rule[-0.200pt]{2.409pt}{0.400pt}}
\put(170.0,570.0){\rule[-0.200pt]{2.409pt}{0.400pt}}
\put(1429.0,570.0){\rule[-0.200pt]{2.409pt}{0.400pt}}
\put(170.0,570.0){\rule[-0.200pt]{2.409pt}{0.400pt}}
\put(1429.0,570.0){\rule[-0.200pt]{2.409pt}{0.400pt}}
\put(170.0,570.0){\rule[-0.200pt]{2.409pt}{0.400pt}}
\put(1429.0,570.0){\rule[-0.200pt]{2.409pt}{0.400pt}}
\put(170.0,570.0){\rule[-0.200pt]{2.409pt}{0.400pt}}
\put(1429.0,570.0){\rule[-0.200pt]{2.409pt}{0.400pt}}
\put(170.0,570.0){\rule[-0.200pt]{2.409pt}{0.400pt}}
\put(1429.0,570.0){\rule[-0.200pt]{2.409pt}{0.400pt}}
\put(170.0,570.0){\rule[-0.200pt]{2.409pt}{0.400pt}}
\put(1429.0,570.0){\rule[-0.200pt]{2.409pt}{0.400pt}}
\put(170.0,570.0){\rule[-0.200pt]{2.409pt}{0.400pt}}
\put(1429.0,570.0){\rule[-0.200pt]{2.409pt}{0.400pt}}
\put(170.0,571.0){\rule[-0.200pt]{2.409pt}{0.400pt}}
\put(1429.0,571.0){\rule[-0.200pt]{2.409pt}{0.400pt}}
\put(170.0,571.0){\rule[-0.200pt]{2.409pt}{0.400pt}}
\put(1429.0,571.0){\rule[-0.200pt]{2.409pt}{0.400pt}}
\put(170.0,571.0){\rule[-0.200pt]{2.409pt}{0.400pt}}
\put(1429.0,571.0){\rule[-0.200pt]{2.409pt}{0.400pt}}
\put(170.0,571.0){\rule[-0.200pt]{2.409pt}{0.400pt}}
\put(1429.0,571.0){\rule[-0.200pt]{2.409pt}{0.400pt}}
\put(170.0,571.0){\rule[-0.200pt]{2.409pt}{0.400pt}}
\put(1429.0,571.0){\rule[-0.200pt]{2.409pt}{0.400pt}}
\put(170.0,571.0){\rule[-0.200pt]{2.409pt}{0.400pt}}
\put(1429.0,571.0){\rule[-0.200pt]{2.409pt}{0.400pt}}
\put(170.0,571.0){\rule[-0.200pt]{2.409pt}{0.400pt}}
\put(1429.0,571.0){\rule[-0.200pt]{2.409pt}{0.400pt}}
\put(170.0,571.0){\rule[-0.200pt]{2.409pt}{0.400pt}}
\put(1429.0,571.0){\rule[-0.200pt]{2.409pt}{0.400pt}}
\put(170.0,571.0){\rule[-0.200pt]{2.409pt}{0.400pt}}
\put(1429.0,571.0){\rule[-0.200pt]{2.409pt}{0.400pt}}
\put(170.0,571.0){\rule[-0.200pt]{2.409pt}{0.400pt}}
\put(1429.0,571.0){\rule[-0.200pt]{2.409pt}{0.400pt}}
\put(170.0,571.0){\rule[-0.200pt]{2.409pt}{0.400pt}}
\put(1429.0,571.0){\rule[-0.200pt]{2.409pt}{0.400pt}}
\put(170.0,571.0){\rule[-0.200pt]{2.409pt}{0.400pt}}
\put(1429.0,571.0){\rule[-0.200pt]{2.409pt}{0.400pt}}
\put(170.0,572.0){\rule[-0.200pt]{2.409pt}{0.400pt}}
\put(1429.0,572.0){\rule[-0.200pt]{2.409pt}{0.400pt}}
\put(170.0,572.0){\rule[-0.200pt]{2.409pt}{0.400pt}}
\put(1429.0,572.0){\rule[-0.200pt]{2.409pt}{0.400pt}}
\put(170.0,572.0){\rule[-0.200pt]{2.409pt}{0.400pt}}
\put(1429.0,572.0){\rule[-0.200pt]{2.409pt}{0.400pt}}
\put(170.0,572.0){\rule[-0.200pt]{2.409pt}{0.400pt}}
\put(1429.0,572.0){\rule[-0.200pt]{2.409pt}{0.400pt}}
\put(170.0,572.0){\rule[-0.200pt]{2.409pt}{0.400pt}}
\put(1429.0,572.0){\rule[-0.200pt]{2.409pt}{0.400pt}}
\put(170.0,572.0){\rule[-0.200pt]{2.409pt}{0.400pt}}
\put(1429.0,572.0){\rule[-0.200pt]{2.409pt}{0.400pt}}
\put(170.0,572.0){\rule[-0.200pt]{2.409pt}{0.400pt}}
\put(1429.0,572.0){\rule[-0.200pt]{2.409pt}{0.400pt}}
\put(170.0,572.0){\rule[-0.200pt]{2.409pt}{0.400pt}}
\put(1429.0,572.0){\rule[-0.200pt]{2.409pt}{0.400pt}}
\put(170.0,572.0){\rule[-0.200pt]{2.409pt}{0.400pt}}
\put(1429.0,572.0){\rule[-0.200pt]{2.409pt}{0.400pt}}
\put(170.0,572.0){\rule[-0.200pt]{2.409pt}{0.400pt}}
\put(1429.0,572.0){\rule[-0.200pt]{2.409pt}{0.400pt}}
\put(170.0,572.0){\rule[-0.200pt]{2.409pt}{0.400pt}}
\put(1429.0,572.0){\rule[-0.200pt]{2.409pt}{0.400pt}}
\put(170.0,572.0){\rule[-0.200pt]{2.409pt}{0.400pt}}
\put(1429.0,572.0){\rule[-0.200pt]{2.409pt}{0.400pt}}
\put(170.0,572.0){\rule[-0.200pt]{2.409pt}{0.400pt}}
\put(1429.0,572.0){\rule[-0.200pt]{2.409pt}{0.400pt}}
\put(170.0,573.0){\rule[-0.200pt]{2.409pt}{0.400pt}}
\put(1429.0,573.0){\rule[-0.200pt]{2.409pt}{0.400pt}}
\put(170.0,573.0){\rule[-0.200pt]{2.409pt}{0.400pt}}
\put(1429.0,573.0){\rule[-0.200pt]{2.409pt}{0.400pt}}
\put(170.0,573.0){\rule[-0.200pt]{2.409pt}{0.400pt}}
\put(1429.0,573.0){\rule[-0.200pt]{2.409pt}{0.400pt}}
\put(170.0,573.0){\rule[-0.200pt]{2.409pt}{0.400pt}}
\put(1429.0,573.0){\rule[-0.200pt]{2.409pt}{0.400pt}}
\put(170.0,573.0){\rule[-0.200pt]{2.409pt}{0.400pt}}
\put(1429.0,573.0){\rule[-0.200pt]{2.409pt}{0.400pt}}
\put(170.0,573.0){\rule[-0.200pt]{2.409pt}{0.400pt}}
\put(1429.0,573.0){\rule[-0.200pt]{2.409pt}{0.400pt}}
\put(170.0,573.0){\rule[-0.200pt]{2.409pt}{0.400pt}}
\put(1429.0,573.0){\rule[-0.200pt]{2.409pt}{0.400pt}}
\put(170.0,573.0){\rule[-0.200pt]{2.409pt}{0.400pt}}
\put(1429.0,573.0){\rule[-0.200pt]{2.409pt}{0.400pt}}
\put(170.0,573.0){\rule[-0.200pt]{2.409pt}{0.400pt}}
\put(1429.0,573.0){\rule[-0.200pt]{2.409pt}{0.400pt}}
\put(170.0,573.0){\rule[-0.200pt]{2.409pt}{0.400pt}}
\put(1429.0,573.0){\rule[-0.200pt]{2.409pt}{0.400pt}}
\put(170.0,573.0){\rule[-0.200pt]{2.409pt}{0.400pt}}
\put(1429.0,573.0){\rule[-0.200pt]{2.409pt}{0.400pt}}
\put(170.0,573.0){\rule[-0.200pt]{2.409pt}{0.400pt}}
\put(1429.0,573.0){\rule[-0.200pt]{2.409pt}{0.400pt}}
\put(170.0,573.0){\rule[-0.200pt]{2.409pt}{0.400pt}}
\put(1429.0,573.0){\rule[-0.200pt]{2.409pt}{0.400pt}}
\put(170.0,574.0){\rule[-0.200pt]{2.409pt}{0.400pt}}
\put(1429.0,574.0){\rule[-0.200pt]{2.409pt}{0.400pt}}
\put(170.0,574.0){\rule[-0.200pt]{2.409pt}{0.400pt}}
\put(1429.0,574.0){\rule[-0.200pt]{2.409pt}{0.400pt}}
\put(170.0,574.0){\rule[-0.200pt]{2.409pt}{0.400pt}}
\put(1429.0,574.0){\rule[-0.200pt]{2.409pt}{0.400pt}}
\put(170.0,574.0){\rule[-0.200pt]{2.409pt}{0.400pt}}
\put(1429.0,574.0){\rule[-0.200pt]{2.409pt}{0.400pt}}
\put(170.0,574.0){\rule[-0.200pt]{2.409pt}{0.400pt}}
\put(1429.0,574.0){\rule[-0.200pt]{2.409pt}{0.400pt}}
\put(170.0,574.0){\rule[-0.200pt]{2.409pt}{0.400pt}}
\put(1429.0,574.0){\rule[-0.200pt]{2.409pt}{0.400pt}}
\put(170.0,574.0){\rule[-0.200pt]{2.409pt}{0.400pt}}
\put(1429.0,574.0){\rule[-0.200pt]{2.409pt}{0.400pt}}
\put(170.0,574.0){\rule[-0.200pt]{2.409pt}{0.400pt}}
\put(1429.0,574.0){\rule[-0.200pt]{2.409pt}{0.400pt}}
\put(170.0,574.0){\rule[-0.200pt]{2.409pt}{0.400pt}}
\put(1429.0,574.0){\rule[-0.200pt]{2.409pt}{0.400pt}}
\put(170.0,574.0){\rule[-0.200pt]{2.409pt}{0.400pt}}
\put(1429.0,574.0){\rule[-0.200pt]{2.409pt}{0.400pt}}
\put(170.0,574.0){\rule[-0.200pt]{2.409pt}{0.400pt}}
\put(1429.0,574.0){\rule[-0.200pt]{2.409pt}{0.400pt}}
\put(170.0,574.0){\rule[-0.200pt]{2.409pt}{0.400pt}}
\put(1429.0,574.0){\rule[-0.200pt]{2.409pt}{0.400pt}}
\put(170.0,574.0){\rule[-0.200pt]{2.409pt}{0.400pt}}
\put(1429.0,574.0){\rule[-0.200pt]{2.409pt}{0.400pt}}
\put(170.0,575.0){\rule[-0.200pt]{2.409pt}{0.400pt}}
\put(1429.0,575.0){\rule[-0.200pt]{2.409pt}{0.400pt}}
\put(170.0,575.0){\rule[-0.200pt]{2.409pt}{0.400pt}}
\put(1429.0,575.0){\rule[-0.200pt]{2.409pt}{0.400pt}}
\put(170.0,575.0){\rule[-0.200pt]{2.409pt}{0.400pt}}
\put(1429.0,575.0){\rule[-0.200pt]{2.409pt}{0.400pt}}
\put(170.0,575.0){\rule[-0.200pt]{2.409pt}{0.400pt}}
\put(1429.0,575.0){\rule[-0.200pt]{2.409pt}{0.400pt}}
\put(170.0,575.0){\rule[-0.200pt]{2.409pt}{0.400pt}}
\put(1429.0,575.0){\rule[-0.200pt]{2.409pt}{0.400pt}}
\put(170.0,575.0){\rule[-0.200pt]{2.409pt}{0.400pt}}
\put(1429.0,575.0){\rule[-0.200pt]{2.409pt}{0.400pt}}
\put(170.0,575.0){\rule[-0.200pt]{2.409pt}{0.400pt}}
\put(1429.0,575.0){\rule[-0.200pt]{2.409pt}{0.400pt}}
\put(170.0,575.0){\rule[-0.200pt]{2.409pt}{0.400pt}}
\put(1429.0,575.0){\rule[-0.200pt]{2.409pt}{0.400pt}}
\put(170.0,575.0){\rule[-0.200pt]{2.409pt}{0.400pt}}
\put(1429.0,575.0){\rule[-0.200pt]{2.409pt}{0.400pt}}
\put(170.0,575.0){\rule[-0.200pt]{2.409pt}{0.400pt}}
\put(1429.0,575.0){\rule[-0.200pt]{2.409pt}{0.400pt}}
\put(170.0,575.0){\rule[-0.200pt]{2.409pt}{0.400pt}}
\put(1429.0,575.0){\rule[-0.200pt]{2.409pt}{0.400pt}}
\put(170.0,575.0){\rule[-0.200pt]{2.409pt}{0.400pt}}
\put(1429.0,575.0){\rule[-0.200pt]{2.409pt}{0.400pt}}
\put(170.0,575.0){\rule[-0.200pt]{2.409pt}{0.400pt}}
\put(1429.0,575.0){\rule[-0.200pt]{2.409pt}{0.400pt}}
\put(170.0,575.0){\rule[-0.200pt]{2.409pt}{0.400pt}}
\put(1429.0,575.0){\rule[-0.200pt]{2.409pt}{0.400pt}}
\put(170.0,576.0){\rule[-0.200pt]{2.409pt}{0.400pt}}
\put(1429.0,576.0){\rule[-0.200pt]{2.409pt}{0.400pt}}
\put(170.0,576.0){\rule[-0.200pt]{2.409pt}{0.400pt}}
\put(1429.0,576.0){\rule[-0.200pt]{2.409pt}{0.400pt}}
\put(170.0,576.0){\rule[-0.200pt]{2.409pt}{0.400pt}}
\put(1429.0,576.0){\rule[-0.200pt]{2.409pt}{0.400pt}}
\put(170.0,576.0){\rule[-0.200pt]{2.409pt}{0.400pt}}
\put(1429.0,576.0){\rule[-0.200pt]{2.409pt}{0.400pt}}
\put(170.0,576.0){\rule[-0.200pt]{2.409pt}{0.400pt}}
\put(1429.0,576.0){\rule[-0.200pt]{2.409pt}{0.400pt}}
\put(170.0,576.0){\rule[-0.200pt]{2.409pt}{0.400pt}}
\put(1429.0,576.0){\rule[-0.200pt]{2.409pt}{0.400pt}}
\put(170.0,576.0){\rule[-0.200pt]{2.409pt}{0.400pt}}
\put(1429.0,576.0){\rule[-0.200pt]{2.409pt}{0.400pt}}
\put(170.0,576.0){\rule[-0.200pt]{2.409pt}{0.400pt}}
\put(1429.0,576.0){\rule[-0.200pt]{2.409pt}{0.400pt}}
\put(170.0,576.0){\rule[-0.200pt]{2.409pt}{0.400pt}}
\put(1429.0,576.0){\rule[-0.200pt]{2.409pt}{0.400pt}}
\put(170.0,576.0){\rule[-0.200pt]{2.409pt}{0.400pt}}
\put(1429.0,576.0){\rule[-0.200pt]{2.409pt}{0.400pt}}
\put(170.0,576.0){\rule[-0.200pt]{2.409pt}{0.400pt}}
\put(1429.0,576.0){\rule[-0.200pt]{2.409pt}{0.400pt}}
\put(170.0,576.0){\rule[-0.200pt]{2.409pt}{0.400pt}}
\put(1429.0,576.0){\rule[-0.200pt]{2.409pt}{0.400pt}}
\put(170.0,576.0){\rule[-0.200pt]{2.409pt}{0.400pt}}
\put(1429.0,576.0){\rule[-0.200pt]{2.409pt}{0.400pt}}
\put(170.0,576.0){\rule[-0.200pt]{2.409pt}{0.400pt}}
\put(1429.0,576.0){\rule[-0.200pt]{2.409pt}{0.400pt}}
\put(170.0,577.0){\rule[-0.200pt]{2.409pt}{0.400pt}}
\put(1429.0,577.0){\rule[-0.200pt]{2.409pt}{0.400pt}}
\put(170.0,577.0){\rule[-0.200pt]{2.409pt}{0.400pt}}
\put(1429.0,577.0){\rule[-0.200pt]{2.409pt}{0.400pt}}
\put(170.0,577.0){\rule[-0.200pt]{2.409pt}{0.400pt}}
\put(1429.0,577.0){\rule[-0.200pt]{2.409pt}{0.400pt}}
\put(170.0,577.0){\rule[-0.200pt]{2.409pt}{0.400pt}}
\put(1429.0,577.0){\rule[-0.200pt]{2.409pt}{0.400pt}}
\put(170.0,577.0){\rule[-0.200pt]{2.409pt}{0.400pt}}
\put(1429.0,577.0){\rule[-0.200pt]{2.409pt}{0.400pt}}
\put(170.0,577.0){\rule[-0.200pt]{2.409pt}{0.400pt}}
\put(1429.0,577.0){\rule[-0.200pt]{2.409pt}{0.400pt}}
\put(170.0,577.0){\rule[-0.200pt]{2.409pt}{0.400pt}}
\put(1429.0,577.0){\rule[-0.200pt]{2.409pt}{0.400pt}}
\put(170.0,577.0){\rule[-0.200pt]{2.409pt}{0.400pt}}
\put(1429.0,577.0){\rule[-0.200pt]{2.409pt}{0.400pt}}
\put(170.0,577.0){\rule[-0.200pt]{2.409pt}{0.400pt}}
\put(1429.0,577.0){\rule[-0.200pt]{2.409pt}{0.400pt}}
\put(170.0,577.0){\rule[-0.200pt]{2.409pt}{0.400pt}}
\put(1429.0,577.0){\rule[-0.200pt]{2.409pt}{0.400pt}}
\put(170.0,577.0){\rule[-0.200pt]{2.409pt}{0.400pt}}
\put(1429.0,577.0){\rule[-0.200pt]{2.409pt}{0.400pt}}
\put(170.0,577.0){\rule[-0.200pt]{2.409pt}{0.400pt}}
\put(1429.0,577.0){\rule[-0.200pt]{2.409pt}{0.400pt}}
\put(170.0,577.0){\rule[-0.200pt]{2.409pt}{0.400pt}}
\put(1429.0,577.0){\rule[-0.200pt]{2.409pt}{0.400pt}}
\put(170.0,577.0){\rule[-0.200pt]{2.409pt}{0.400pt}}
\put(1429.0,577.0){\rule[-0.200pt]{2.409pt}{0.400pt}}
\put(170.0,578.0){\rule[-0.200pt]{2.409pt}{0.400pt}}
\put(1429.0,578.0){\rule[-0.200pt]{2.409pt}{0.400pt}}
\put(170.0,578.0){\rule[-0.200pt]{2.409pt}{0.400pt}}
\put(1429.0,578.0){\rule[-0.200pt]{2.409pt}{0.400pt}}
\put(170.0,578.0){\rule[-0.200pt]{2.409pt}{0.400pt}}
\put(1429.0,578.0){\rule[-0.200pt]{2.409pt}{0.400pt}}
\put(170.0,578.0){\rule[-0.200pt]{2.409pt}{0.400pt}}
\put(1429.0,578.0){\rule[-0.200pt]{2.409pt}{0.400pt}}
\put(170.0,578.0){\rule[-0.200pt]{2.409pt}{0.400pt}}
\put(1429.0,578.0){\rule[-0.200pt]{2.409pt}{0.400pt}}
\put(170.0,578.0){\rule[-0.200pt]{2.409pt}{0.400pt}}
\put(1429.0,578.0){\rule[-0.200pt]{2.409pt}{0.400pt}}
\put(170.0,578.0){\rule[-0.200pt]{2.409pt}{0.400pt}}
\put(1429.0,578.0){\rule[-0.200pt]{2.409pt}{0.400pt}}
\put(170.0,578.0){\rule[-0.200pt]{2.409pt}{0.400pt}}
\put(1429.0,578.0){\rule[-0.200pt]{2.409pt}{0.400pt}}
\put(170.0,578.0){\rule[-0.200pt]{2.409pt}{0.400pt}}
\put(1429.0,578.0){\rule[-0.200pt]{2.409pt}{0.400pt}}
\put(170.0,578.0){\rule[-0.200pt]{2.409pt}{0.400pt}}
\put(1429.0,578.0){\rule[-0.200pt]{2.409pt}{0.400pt}}
\put(170.0,578.0){\rule[-0.200pt]{2.409pt}{0.400pt}}
\put(1429.0,578.0){\rule[-0.200pt]{2.409pt}{0.400pt}}
\put(170.0,578.0){\rule[-0.200pt]{2.409pt}{0.400pt}}
\put(1429.0,578.0){\rule[-0.200pt]{2.409pt}{0.400pt}}
\put(170.0,578.0){\rule[-0.200pt]{2.409pt}{0.400pt}}
\put(1429.0,578.0){\rule[-0.200pt]{2.409pt}{0.400pt}}
\put(170.0,578.0){\rule[-0.200pt]{2.409pt}{0.400pt}}
\put(1429.0,578.0){\rule[-0.200pt]{2.409pt}{0.400pt}}
\put(170.0,578.0){\rule[-0.200pt]{2.409pt}{0.400pt}}
\put(1429.0,578.0){\rule[-0.200pt]{2.409pt}{0.400pt}}
\put(170.0,579.0){\rule[-0.200pt]{2.409pt}{0.400pt}}
\put(1429.0,579.0){\rule[-0.200pt]{2.409pt}{0.400pt}}
\put(170.0,579.0){\rule[-0.200pt]{2.409pt}{0.400pt}}
\put(1429.0,579.0){\rule[-0.200pt]{2.409pt}{0.400pt}}
\put(170.0,579.0){\rule[-0.200pt]{2.409pt}{0.400pt}}
\put(1429.0,579.0){\rule[-0.200pt]{2.409pt}{0.400pt}}
\put(170.0,579.0){\rule[-0.200pt]{2.409pt}{0.400pt}}
\put(1429.0,579.0){\rule[-0.200pt]{2.409pt}{0.400pt}}
\put(170.0,579.0){\rule[-0.200pt]{2.409pt}{0.400pt}}
\put(1429.0,579.0){\rule[-0.200pt]{2.409pt}{0.400pt}}
\put(170.0,579.0){\rule[-0.200pt]{2.409pt}{0.400pt}}
\put(1429.0,579.0){\rule[-0.200pt]{2.409pt}{0.400pt}}
\put(170.0,579.0){\rule[-0.200pt]{2.409pt}{0.400pt}}
\put(1429.0,579.0){\rule[-0.200pt]{2.409pt}{0.400pt}}
\put(170.0,579.0){\rule[-0.200pt]{2.409pt}{0.400pt}}
\put(1429.0,579.0){\rule[-0.200pt]{2.409pt}{0.400pt}}
\put(170.0,579.0){\rule[-0.200pt]{2.409pt}{0.400pt}}
\put(1429.0,579.0){\rule[-0.200pt]{2.409pt}{0.400pt}}
\put(170.0,579.0){\rule[-0.200pt]{2.409pt}{0.400pt}}
\put(1429.0,579.0){\rule[-0.200pt]{2.409pt}{0.400pt}}
\put(170.0,579.0){\rule[-0.200pt]{2.409pt}{0.400pt}}
\put(1429.0,579.0){\rule[-0.200pt]{2.409pt}{0.400pt}}
\put(170.0,579.0){\rule[-0.200pt]{2.409pt}{0.400pt}}
\put(1429.0,579.0){\rule[-0.200pt]{2.409pt}{0.400pt}}
\put(170.0,579.0){\rule[-0.200pt]{2.409pt}{0.400pt}}
\put(1429.0,579.0){\rule[-0.200pt]{2.409pt}{0.400pt}}
\put(170.0,579.0){\rule[-0.200pt]{2.409pt}{0.400pt}}
\put(1429.0,579.0){\rule[-0.200pt]{2.409pt}{0.400pt}}
\put(170.0,579.0){\rule[-0.200pt]{2.409pt}{0.400pt}}
\put(1429.0,579.0){\rule[-0.200pt]{2.409pt}{0.400pt}}
\put(170.0,580.0){\rule[-0.200pt]{2.409pt}{0.400pt}}
\put(1429.0,580.0){\rule[-0.200pt]{2.409pt}{0.400pt}}
\put(170.0,580.0){\rule[-0.200pt]{2.409pt}{0.400pt}}
\put(1429.0,580.0){\rule[-0.200pt]{2.409pt}{0.400pt}}
\put(170.0,580.0){\rule[-0.200pt]{2.409pt}{0.400pt}}
\put(1429.0,580.0){\rule[-0.200pt]{2.409pt}{0.400pt}}
\put(170.0,580.0){\rule[-0.200pt]{2.409pt}{0.400pt}}
\put(1429.0,580.0){\rule[-0.200pt]{2.409pt}{0.400pt}}
\put(170.0,580.0){\rule[-0.200pt]{2.409pt}{0.400pt}}
\put(1429.0,580.0){\rule[-0.200pt]{2.409pt}{0.400pt}}
\put(170.0,580.0){\rule[-0.200pt]{2.409pt}{0.400pt}}
\put(1429.0,580.0){\rule[-0.200pt]{2.409pt}{0.400pt}}
\put(170.0,580.0){\rule[-0.200pt]{2.409pt}{0.400pt}}
\put(1429.0,580.0){\rule[-0.200pt]{2.409pt}{0.400pt}}
\put(170.0,580.0){\rule[-0.200pt]{2.409pt}{0.400pt}}
\put(1429.0,580.0){\rule[-0.200pt]{2.409pt}{0.400pt}}
\put(170.0,580.0){\rule[-0.200pt]{2.409pt}{0.400pt}}
\put(1429.0,580.0){\rule[-0.200pt]{2.409pt}{0.400pt}}
\put(170.0,580.0){\rule[-0.200pt]{2.409pt}{0.400pt}}
\put(1429.0,580.0){\rule[-0.200pt]{2.409pt}{0.400pt}}
\put(170.0,580.0){\rule[-0.200pt]{2.409pt}{0.400pt}}
\put(1429.0,580.0){\rule[-0.200pt]{2.409pt}{0.400pt}}
\put(170.0,580.0){\rule[-0.200pt]{2.409pt}{0.400pt}}
\put(1429.0,580.0){\rule[-0.200pt]{2.409pt}{0.400pt}}
\put(170.0,580.0){\rule[-0.200pt]{2.409pt}{0.400pt}}
\put(1429.0,580.0){\rule[-0.200pt]{2.409pt}{0.400pt}}
\put(170.0,580.0){\rule[-0.200pt]{2.409pt}{0.400pt}}
\put(1429.0,580.0){\rule[-0.200pt]{2.409pt}{0.400pt}}
\put(170.0,580.0){\rule[-0.200pt]{2.409pt}{0.400pt}}
\put(1429.0,580.0){\rule[-0.200pt]{2.409pt}{0.400pt}}
\put(170.0,580.0){\rule[-0.200pt]{2.409pt}{0.400pt}}
\put(1429.0,580.0){\rule[-0.200pt]{2.409pt}{0.400pt}}
\put(170.0,581.0){\rule[-0.200pt]{2.409pt}{0.400pt}}
\put(1429.0,581.0){\rule[-0.200pt]{2.409pt}{0.400pt}}
\put(170.0,581.0){\rule[-0.200pt]{2.409pt}{0.400pt}}
\put(1429.0,581.0){\rule[-0.200pt]{2.409pt}{0.400pt}}
\put(170.0,581.0){\rule[-0.200pt]{2.409pt}{0.400pt}}
\put(1429.0,581.0){\rule[-0.200pt]{2.409pt}{0.400pt}}
\put(170.0,581.0){\rule[-0.200pt]{2.409pt}{0.400pt}}
\put(1429.0,581.0){\rule[-0.200pt]{2.409pt}{0.400pt}}
\put(170.0,581.0){\rule[-0.200pt]{2.409pt}{0.400pt}}
\put(1429.0,581.0){\rule[-0.200pt]{2.409pt}{0.400pt}}
\put(170.0,581.0){\rule[-0.200pt]{2.409pt}{0.400pt}}
\put(1429.0,581.0){\rule[-0.200pt]{2.409pt}{0.400pt}}
\put(170.0,581.0){\rule[-0.200pt]{2.409pt}{0.400pt}}
\put(1429.0,581.0){\rule[-0.200pt]{2.409pt}{0.400pt}}
\put(170.0,581.0){\rule[-0.200pt]{2.409pt}{0.400pt}}
\put(1429.0,581.0){\rule[-0.200pt]{2.409pt}{0.400pt}}
\put(170.0,581.0){\rule[-0.200pt]{2.409pt}{0.400pt}}
\put(1429.0,581.0){\rule[-0.200pt]{2.409pt}{0.400pt}}
\put(170.0,581.0){\rule[-0.200pt]{2.409pt}{0.400pt}}
\put(1429.0,581.0){\rule[-0.200pt]{2.409pt}{0.400pt}}
\put(170.0,581.0){\rule[-0.200pt]{2.409pt}{0.400pt}}
\put(1429.0,581.0){\rule[-0.200pt]{2.409pt}{0.400pt}}
\put(170.0,581.0){\rule[-0.200pt]{2.409pt}{0.400pt}}
\put(1429.0,581.0){\rule[-0.200pt]{2.409pt}{0.400pt}}
\put(170.0,581.0){\rule[-0.200pt]{2.409pt}{0.400pt}}
\put(1429.0,581.0){\rule[-0.200pt]{2.409pt}{0.400pt}}
\put(170.0,581.0){\rule[-0.200pt]{2.409pt}{0.400pt}}
\put(1429.0,581.0){\rule[-0.200pt]{2.409pt}{0.400pt}}
\put(170.0,581.0){\rule[-0.200pt]{2.409pt}{0.400pt}}
\put(1429.0,581.0){\rule[-0.200pt]{2.409pt}{0.400pt}}
\put(170.0,581.0){\rule[-0.200pt]{2.409pt}{0.400pt}}
\put(1429.0,581.0){\rule[-0.200pt]{2.409pt}{0.400pt}}
\put(170.0,582.0){\rule[-0.200pt]{2.409pt}{0.400pt}}
\put(1429.0,582.0){\rule[-0.200pt]{2.409pt}{0.400pt}}
\put(170.0,582.0){\rule[-0.200pt]{2.409pt}{0.400pt}}
\put(1429.0,582.0){\rule[-0.200pt]{2.409pt}{0.400pt}}
\put(170.0,582.0){\rule[-0.200pt]{2.409pt}{0.400pt}}
\put(1429.0,582.0){\rule[-0.200pt]{2.409pt}{0.400pt}}
\put(170.0,582.0){\rule[-0.200pt]{2.409pt}{0.400pt}}
\put(1429.0,582.0){\rule[-0.200pt]{2.409pt}{0.400pt}}
\put(170.0,582.0){\rule[-0.200pt]{2.409pt}{0.400pt}}
\put(1429.0,582.0){\rule[-0.200pt]{2.409pt}{0.400pt}}
\put(170.0,582.0){\rule[-0.200pt]{2.409pt}{0.400pt}}
\put(1429.0,582.0){\rule[-0.200pt]{2.409pt}{0.400pt}}
\put(170.0,582.0){\rule[-0.200pt]{2.409pt}{0.400pt}}
\put(1429.0,582.0){\rule[-0.200pt]{2.409pt}{0.400pt}}
\put(170.0,582.0){\rule[-0.200pt]{2.409pt}{0.400pt}}
\put(1429.0,582.0){\rule[-0.200pt]{2.409pt}{0.400pt}}
\put(170.0,582.0){\rule[-0.200pt]{2.409pt}{0.400pt}}
\put(1429.0,582.0){\rule[-0.200pt]{2.409pt}{0.400pt}}
\put(170.0,582.0){\rule[-0.200pt]{2.409pt}{0.400pt}}
\put(1429.0,582.0){\rule[-0.200pt]{2.409pt}{0.400pt}}
\put(170.0,582.0){\rule[-0.200pt]{2.409pt}{0.400pt}}
\put(1429.0,582.0){\rule[-0.200pt]{2.409pt}{0.400pt}}
\put(170.0,582.0){\rule[-0.200pt]{2.409pt}{0.400pt}}
\put(1429.0,582.0){\rule[-0.200pt]{2.409pt}{0.400pt}}
\put(170.0,582.0){\rule[-0.200pt]{2.409pt}{0.400pt}}
\put(1429.0,582.0){\rule[-0.200pt]{2.409pt}{0.400pt}}
\put(170.0,582.0){\rule[-0.200pt]{2.409pt}{0.400pt}}
\put(1429.0,582.0){\rule[-0.200pt]{2.409pt}{0.400pt}}
\put(170.0,582.0){\rule[-0.200pt]{2.409pt}{0.400pt}}
\put(1429.0,582.0){\rule[-0.200pt]{2.409pt}{0.400pt}}
\put(170.0,582.0){\rule[-0.200pt]{2.409pt}{0.400pt}}
\put(1429.0,582.0){\rule[-0.200pt]{2.409pt}{0.400pt}}
\put(170.0,582.0){\rule[-0.200pt]{2.409pt}{0.400pt}}
\put(1429.0,582.0){\rule[-0.200pt]{2.409pt}{0.400pt}}
\put(170.0,583.0){\rule[-0.200pt]{2.409pt}{0.400pt}}
\put(1429.0,583.0){\rule[-0.200pt]{2.409pt}{0.400pt}}
\put(170.0,583.0){\rule[-0.200pt]{2.409pt}{0.400pt}}
\put(1429.0,583.0){\rule[-0.200pt]{2.409pt}{0.400pt}}
\put(170.0,583.0){\rule[-0.200pt]{2.409pt}{0.400pt}}
\put(1429.0,583.0){\rule[-0.200pt]{2.409pt}{0.400pt}}
\put(170.0,583.0){\rule[-0.200pt]{2.409pt}{0.400pt}}
\put(1429.0,583.0){\rule[-0.200pt]{2.409pt}{0.400pt}}
\put(170.0,583.0){\rule[-0.200pt]{2.409pt}{0.400pt}}
\put(1429.0,583.0){\rule[-0.200pt]{2.409pt}{0.400pt}}
\put(170.0,583.0){\rule[-0.200pt]{2.409pt}{0.400pt}}
\put(1429.0,583.0){\rule[-0.200pt]{2.409pt}{0.400pt}}
\put(170.0,583.0){\rule[-0.200pt]{2.409pt}{0.400pt}}
\put(1429.0,583.0){\rule[-0.200pt]{2.409pt}{0.400pt}}
\put(170.0,583.0){\rule[-0.200pt]{2.409pt}{0.400pt}}
\put(1429.0,583.0){\rule[-0.200pt]{2.409pt}{0.400pt}}
\put(170.0,583.0){\rule[-0.200pt]{2.409pt}{0.400pt}}
\put(1429.0,583.0){\rule[-0.200pt]{2.409pt}{0.400pt}}
\put(170.0,583.0){\rule[-0.200pt]{2.409pt}{0.400pt}}
\put(1429.0,583.0){\rule[-0.200pt]{2.409pt}{0.400pt}}
\put(170.0,583.0){\rule[-0.200pt]{2.409pt}{0.400pt}}
\put(1429.0,583.0){\rule[-0.200pt]{2.409pt}{0.400pt}}
\put(170.0,583.0){\rule[-0.200pt]{2.409pt}{0.400pt}}
\put(1429.0,583.0){\rule[-0.200pt]{2.409pt}{0.400pt}}
\put(170.0,583.0){\rule[-0.200pt]{2.409pt}{0.400pt}}
\put(1429.0,583.0){\rule[-0.200pt]{2.409pt}{0.400pt}}
\put(170.0,583.0){\rule[-0.200pt]{2.409pt}{0.400pt}}
\put(1429.0,583.0){\rule[-0.200pt]{2.409pt}{0.400pt}}
\put(170.0,583.0){\rule[-0.200pt]{2.409pt}{0.400pt}}
\put(1429.0,583.0){\rule[-0.200pt]{2.409pt}{0.400pt}}
\put(170.0,583.0){\rule[-0.200pt]{2.409pt}{0.400pt}}
\put(1429.0,583.0){\rule[-0.200pt]{2.409pt}{0.400pt}}
\put(170.0,584.0){\rule[-0.200pt]{2.409pt}{0.400pt}}
\put(1429.0,584.0){\rule[-0.200pt]{2.409pt}{0.400pt}}
\put(170.0,584.0){\rule[-0.200pt]{2.409pt}{0.400pt}}
\put(1429.0,584.0){\rule[-0.200pt]{2.409pt}{0.400pt}}
\put(170.0,584.0){\rule[-0.200pt]{2.409pt}{0.400pt}}
\put(1429.0,584.0){\rule[-0.200pt]{2.409pt}{0.400pt}}
\put(170.0,584.0){\rule[-0.200pt]{2.409pt}{0.400pt}}
\put(1429.0,584.0){\rule[-0.200pt]{2.409pt}{0.400pt}}
\put(170.0,584.0){\rule[-0.200pt]{2.409pt}{0.400pt}}
\put(1429.0,584.0){\rule[-0.200pt]{2.409pt}{0.400pt}}
\put(170.0,584.0){\rule[-0.200pt]{2.409pt}{0.400pt}}
\put(1429.0,584.0){\rule[-0.200pt]{2.409pt}{0.400pt}}
\put(170.0,584.0){\rule[-0.200pt]{2.409pt}{0.400pt}}
\put(1429.0,584.0){\rule[-0.200pt]{2.409pt}{0.400pt}}
\put(170.0,584.0){\rule[-0.200pt]{2.409pt}{0.400pt}}
\put(1429.0,584.0){\rule[-0.200pt]{2.409pt}{0.400pt}}
\put(170.0,584.0){\rule[-0.200pt]{2.409pt}{0.400pt}}
\put(1429.0,584.0){\rule[-0.200pt]{2.409pt}{0.400pt}}
\put(170.0,584.0){\rule[-0.200pt]{2.409pt}{0.400pt}}
\put(1429.0,584.0){\rule[-0.200pt]{2.409pt}{0.400pt}}
\put(170.0,584.0){\rule[-0.200pt]{2.409pt}{0.400pt}}
\put(1429.0,584.0){\rule[-0.200pt]{2.409pt}{0.400pt}}
\put(170.0,584.0){\rule[-0.200pt]{2.409pt}{0.400pt}}
\put(1429.0,584.0){\rule[-0.200pt]{2.409pt}{0.400pt}}
\put(170.0,584.0){\rule[-0.200pt]{2.409pt}{0.400pt}}
\put(1429.0,584.0){\rule[-0.200pt]{2.409pt}{0.400pt}}
\put(170.0,584.0){\rule[-0.200pt]{2.409pt}{0.400pt}}
\put(1429.0,584.0){\rule[-0.200pt]{2.409pt}{0.400pt}}
\put(170.0,584.0){\rule[-0.200pt]{2.409pt}{0.400pt}}
\put(1429.0,584.0){\rule[-0.200pt]{2.409pt}{0.400pt}}
\put(170.0,584.0){\rule[-0.200pt]{2.409pt}{0.400pt}}
\put(1429.0,584.0){\rule[-0.200pt]{2.409pt}{0.400pt}}
\put(170.0,584.0){\rule[-0.200pt]{2.409pt}{0.400pt}}
\put(1429.0,584.0){\rule[-0.200pt]{2.409pt}{0.400pt}}
\put(170.0,584.0){\rule[-0.200pt]{2.409pt}{0.400pt}}
\put(1429.0,584.0){\rule[-0.200pt]{2.409pt}{0.400pt}}
\put(170.0,585.0){\rule[-0.200pt]{2.409pt}{0.400pt}}
\put(1429.0,585.0){\rule[-0.200pt]{2.409pt}{0.400pt}}
\put(170.0,585.0){\rule[-0.200pt]{2.409pt}{0.400pt}}
\put(1429.0,585.0){\rule[-0.200pt]{2.409pt}{0.400pt}}
\put(170.0,585.0){\rule[-0.200pt]{2.409pt}{0.400pt}}
\put(1429.0,585.0){\rule[-0.200pt]{2.409pt}{0.400pt}}
\put(170.0,585.0){\rule[-0.200pt]{2.409pt}{0.400pt}}
\put(1429.0,585.0){\rule[-0.200pt]{2.409pt}{0.400pt}}
\put(170.0,585.0){\rule[-0.200pt]{2.409pt}{0.400pt}}
\put(1429.0,585.0){\rule[-0.200pt]{2.409pt}{0.400pt}}
\put(170.0,585.0){\rule[-0.200pt]{2.409pt}{0.400pt}}
\put(1429.0,585.0){\rule[-0.200pt]{2.409pt}{0.400pt}}
\put(170.0,585.0){\rule[-0.200pt]{2.409pt}{0.400pt}}
\put(1429.0,585.0){\rule[-0.200pt]{2.409pt}{0.400pt}}
\put(170.0,585.0){\rule[-0.200pt]{2.409pt}{0.400pt}}
\put(1429.0,585.0){\rule[-0.200pt]{2.409pt}{0.400pt}}
\put(170.0,585.0){\rule[-0.200pt]{2.409pt}{0.400pt}}
\put(1429.0,585.0){\rule[-0.200pt]{2.409pt}{0.400pt}}
\put(170.0,585.0){\rule[-0.200pt]{2.409pt}{0.400pt}}
\put(1429.0,585.0){\rule[-0.200pt]{2.409pt}{0.400pt}}
\put(170.0,585.0){\rule[-0.200pt]{2.409pt}{0.400pt}}
\put(1429.0,585.0){\rule[-0.200pt]{2.409pt}{0.400pt}}
\put(170.0,585.0){\rule[-0.200pt]{2.409pt}{0.400pt}}
\put(1429.0,585.0){\rule[-0.200pt]{2.409pt}{0.400pt}}
\put(170.0,585.0){\rule[-0.200pt]{2.409pt}{0.400pt}}
\put(1429.0,585.0){\rule[-0.200pt]{2.409pt}{0.400pt}}
\put(170.0,585.0){\rule[-0.200pt]{2.409pt}{0.400pt}}
\put(1429.0,585.0){\rule[-0.200pt]{2.409pt}{0.400pt}}
\put(170.0,585.0){\rule[-0.200pt]{2.409pt}{0.400pt}}
\put(1429.0,585.0){\rule[-0.200pt]{2.409pt}{0.400pt}}
\put(170.0,585.0){\rule[-0.200pt]{2.409pt}{0.400pt}}
\put(1429.0,585.0){\rule[-0.200pt]{2.409pt}{0.400pt}}
\put(170.0,585.0){\rule[-0.200pt]{2.409pt}{0.400pt}}
\put(1429.0,585.0){\rule[-0.200pt]{2.409pt}{0.400pt}}
\put(170.0,585.0){\rule[-0.200pt]{2.409pt}{0.400pt}}
\put(1429.0,585.0){\rule[-0.200pt]{2.409pt}{0.400pt}}
\put(170.0,586.0){\rule[-0.200pt]{2.409pt}{0.400pt}}
\put(1429.0,586.0){\rule[-0.200pt]{2.409pt}{0.400pt}}
\put(170.0,586.0){\rule[-0.200pt]{2.409pt}{0.400pt}}
\put(1429.0,586.0){\rule[-0.200pt]{2.409pt}{0.400pt}}
\put(170.0,586.0){\rule[-0.200pt]{2.409pt}{0.400pt}}
\put(1429.0,586.0){\rule[-0.200pt]{2.409pt}{0.400pt}}
\put(170.0,586.0){\rule[-0.200pt]{2.409pt}{0.400pt}}
\put(1429.0,586.0){\rule[-0.200pt]{2.409pt}{0.400pt}}
\put(170.0,586.0){\rule[-0.200pt]{2.409pt}{0.400pt}}
\put(1429.0,586.0){\rule[-0.200pt]{2.409pt}{0.400pt}}
\put(170.0,586.0){\rule[-0.200pt]{2.409pt}{0.400pt}}
\put(1429.0,586.0){\rule[-0.200pt]{2.409pt}{0.400pt}}
\put(170.0,586.0){\rule[-0.200pt]{2.409pt}{0.400pt}}
\put(1429.0,586.0){\rule[-0.200pt]{2.409pt}{0.400pt}}
\put(170.0,586.0){\rule[-0.200pt]{2.409pt}{0.400pt}}
\put(1429.0,586.0){\rule[-0.200pt]{2.409pt}{0.400pt}}
\put(170.0,586.0){\rule[-0.200pt]{2.409pt}{0.400pt}}
\put(1429.0,586.0){\rule[-0.200pt]{2.409pt}{0.400pt}}
\put(170.0,586.0){\rule[-0.200pt]{2.409pt}{0.400pt}}
\put(1429.0,586.0){\rule[-0.200pt]{2.409pt}{0.400pt}}
\put(170.0,586.0){\rule[-0.200pt]{2.409pt}{0.400pt}}
\put(1429.0,586.0){\rule[-0.200pt]{2.409pt}{0.400pt}}
\put(170.0,586.0){\rule[-0.200pt]{2.409pt}{0.400pt}}
\put(1429.0,586.0){\rule[-0.200pt]{2.409pt}{0.400pt}}
\put(170.0,586.0){\rule[-0.200pt]{2.409pt}{0.400pt}}
\put(1429.0,586.0){\rule[-0.200pt]{2.409pt}{0.400pt}}
\put(170.0,586.0){\rule[-0.200pt]{2.409pt}{0.400pt}}
\put(1429.0,586.0){\rule[-0.200pt]{2.409pt}{0.400pt}}
\put(170.0,586.0){\rule[-0.200pt]{2.409pt}{0.400pt}}
\put(1429.0,586.0){\rule[-0.200pt]{2.409pt}{0.400pt}}
\put(170.0,586.0){\rule[-0.200pt]{2.409pt}{0.400pt}}
\put(1429.0,586.0){\rule[-0.200pt]{2.409pt}{0.400pt}}
\put(170.0,586.0){\rule[-0.200pt]{2.409pt}{0.400pt}}
\put(1429.0,586.0){\rule[-0.200pt]{2.409pt}{0.400pt}}
\put(170.0,586.0){\rule[-0.200pt]{2.409pt}{0.400pt}}
\put(1429.0,586.0){\rule[-0.200pt]{2.409pt}{0.400pt}}
\put(170.0,587.0){\rule[-0.200pt]{2.409pt}{0.400pt}}
\put(1429.0,587.0){\rule[-0.200pt]{2.409pt}{0.400pt}}
\put(170.0,587.0){\rule[-0.200pt]{2.409pt}{0.400pt}}
\put(1429.0,587.0){\rule[-0.200pt]{2.409pt}{0.400pt}}
\put(170.0,587.0){\rule[-0.200pt]{2.409pt}{0.400pt}}
\put(1429.0,587.0){\rule[-0.200pt]{2.409pt}{0.400pt}}
\put(170.0,587.0){\rule[-0.200pt]{2.409pt}{0.400pt}}
\put(1429.0,587.0){\rule[-0.200pt]{2.409pt}{0.400pt}}
\put(170.0,587.0){\rule[-0.200pt]{2.409pt}{0.400pt}}
\put(1429.0,587.0){\rule[-0.200pt]{2.409pt}{0.400pt}}
\put(170.0,587.0){\rule[-0.200pt]{2.409pt}{0.400pt}}
\put(1429.0,587.0){\rule[-0.200pt]{2.409pt}{0.400pt}}
\put(170.0,587.0){\rule[-0.200pt]{2.409pt}{0.400pt}}
\put(1429.0,587.0){\rule[-0.200pt]{2.409pt}{0.400pt}}
\put(170.0,587.0){\rule[-0.200pt]{2.409pt}{0.400pt}}
\put(1429.0,587.0){\rule[-0.200pt]{2.409pt}{0.400pt}}
\put(170.0,587.0){\rule[-0.200pt]{2.409pt}{0.400pt}}
\put(1429.0,587.0){\rule[-0.200pt]{2.409pt}{0.400pt}}
\put(170.0,587.0){\rule[-0.200pt]{2.409pt}{0.400pt}}
\put(1429.0,587.0){\rule[-0.200pt]{2.409pt}{0.400pt}}
\put(170.0,587.0){\rule[-0.200pt]{2.409pt}{0.400pt}}
\put(1429.0,587.0){\rule[-0.200pt]{2.409pt}{0.400pt}}
\put(170.0,587.0){\rule[-0.200pt]{2.409pt}{0.400pt}}
\put(1429.0,587.0){\rule[-0.200pt]{2.409pt}{0.400pt}}
\put(170.0,587.0){\rule[-0.200pt]{2.409pt}{0.400pt}}
\put(1429.0,587.0){\rule[-0.200pt]{2.409pt}{0.400pt}}
\put(170.0,587.0){\rule[-0.200pt]{2.409pt}{0.400pt}}
\put(1429.0,587.0){\rule[-0.200pt]{2.409pt}{0.400pt}}
\put(170.0,587.0){\rule[-0.200pt]{2.409pt}{0.400pt}}
\put(1429.0,587.0){\rule[-0.200pt]{2.409pt}{0.400pt}}
\put(170.0,587.0){\rule[-0.200pt]{2.409pt}{0.400pt}}
\put(1429.0,587.0){\rule[-0.200pt]{2.409pt}{0.400pt}}
\put(170.0,587.0){\rule[-0.200pt]{2.409pt}{0.400pt}}
\put(1429.0,587.0){\rule[-0.200pt]{2.409pt}{0.400pt}}
\put(170.0,587.0){\rule[-0.200pt]{2.409pt}{0.400pt}}
\put(1429.0,587.0){\rule[-0.200pt]{2.409pt}{0.400pt}}
\put(170.0,587.0){\rule[-0.200pt]{2.409pt}{0.400pt}}
\put(1429.0,587.0){\rule[-0.200pt]{2.409pt}{0.400pt}}
\put(170.0,588.0){\rule[-0.200pt]{2.409pt}{0.400pt}}
\put(1429.0,588.0){\rule[-0.200pt]{2.409pt}{0.400pt}}
\put(170.0,588.0){\rule[-0.200pt]{2.409pt}{0.400pt}}
\put(1429.0,588.0){\rule[-0.200pt]{2.409pt}{0.400pt}}
\put(170.0,588.0){\rule[-0.200pt]{2.409pt}{0.400pt}}
\put(1429.0,588.0){\rule[-0.200pt]{2.409pt}{0.400pt}}
\put(170.0,588.0){\rule[-0.200pt]{2.409pt}{0.400pt}}
\put(1429.0,588.0){\rule[-0.200pt]{2.409pt}{0.400pt}}
\put(170.0,588.0){\rule[-0.200pt]{2.409pt}{0.400pt}}
\put(1429.0,588.0){\rule[-0.200pt]{2.409pt}{0.400pt}}
\put(170.0,588.0){\rule[-0.200pt]{2.409pt}{0.400pt}}
\put(1429.0,588.0){\rule[-0.200pt]{2.409pt}{0.400pt}}
\put(170.0,588.0){\rule[-0.200pt]{2.409pt}{0.400pt}}
\put(1429.0,588.0){\rule[-0.200pt]{2.409pt}{0.400pt}}
\put(170.0,588.0){\rule[-0.200pt]{2.409pt}{0.400pt}}
\put(1429.0,588.0){\rule[-0.200pt]{2.409pt}{0.400pt}}
\put(170.0,588.0){\rule[-0.200pt]{2.409pt}{0.400pt}}
\put(1429.0,588.0){\rule[-0.200pt]{2.409pt}{0.400pt}}
\put(170.0,588.0){\rule[-0.200pt]{2.409pt}{0.400pt}}
\put(1429.0,588.0){\rule[-0.200pt]{2.409pt}{0.400pt}}
\put(170.0,588.0){\rule[-0.200pt]{2.409pt}{0.400pt}}
\put(1429.0,588.0){\rule[-0.200pt]{2.409pt}{0.400pt}}
\put(170.0,588.0){\rule[-0.200pt]{2.409pt}{0.400pt}}
\put(1429.0,588.0){\rule[-0.200pt]{2.409pt}{0.400pt}}
\put(170.0,588.0){\rule[-0.200pt]{2.409pt}{0.400pt}}
\put(1429.0,588.0){\rule[-0.200pt]{2.409pt}{0.400pt}}
\put(170.0,588.0){\rule[-0.200pt]{2.409pt}{0.400pt}}
\put(1429.0,588.0){\rule[-0.200pt]{2.409pt}{0.400pt}}
\put(170.0,588.0){\rule[-0.200pt]{2.409pt}{0.400pt}}
\put(1429.0,588.0){\rule[-0.200pt]{2.409pt}{0.400pt}}
\put(170.0,588.0){\rule[-0.200pt]{2.409pt}{0.400pt}}
\put(1429.0,588.0){\rule[-0.200pt]{2.409pt}{0.400pt}}
\put(170.0,588.0){\rule[-0.200pt]{2.409pt}{0.400pt}}
\put(1429.0,588.0){\rule[-0.200pt]{2.409pt}{0.400pt}}
\put(170.0,588.0){\rule[-0.200pt]{2.409pt}{0.400pt}}
\put(1429.0,588.0){\rule[-0.200pt]{2.409pt}{0.400pt}}
\put(170.0,588.0){\rule[-0.200pt]{2.409pt}{0.400pt}}
\put(1429.0,588.0){\rule[-0.200pt]{2.409pt}{0.400pt}}
\put(170.0,589.0){\rule[-0.200pt]{2.409pt}{0.400pt}}
\put(1429.0,589.0){\rule[-0.200pt]{2.409pt}{0.400pt}}
\put(170.0,589.0){\rule[-0.200pt]{2.409pt}{0.400pt}}
\put(1429.0,589.0){\rule[-0.200pt]{2.409pt}{0.400pt}}
\put(170.0,589.0){\rule[-0.200pt]{2.409pt}{0.400pt}}
\put(1429.0,589.0){\rule[-0.200pt]{2.409pt}{0.400pt}}
\put(170.0,589.0){\rule[-0.200pt]{2.409pt}{0.400pt}}
\put(1429.0,589.0){\rule[-0.200pt]{2.409pt}{0.400pt}}
\put(170.0,589.0){\rule[-0.200pt]{2.409pt}{0.400pt}}
\put(1429.0,589.0){\rule[-0.200pt]{2.409pt}{0.400pt}}
\put(170.0,589.0){\rule[-0.200pt]{2.409pt}{0.400pt}}
\put(1429.0,589.0){\rule[-0.200pt]{2.409pt}{0.400pt}}
\put(170.0,589.0){\rule[-0.200pt]{2.409pt}{0.400pt}}
\put(1429.0,589.0){\rule[-0.200pt]{2.409pt}{0.400pt}}
\put(170.0,589.0){\rule[-0.200pt]{2.409pt}{0.400pt}}
\put(1429.0,589.0){\rule[-0.200pt]{2.409pt}{0.400pt}}
\put(170.0,589.0){\rule[-0.200pt]{2.409pt}{0.400pt}}
\put(1429.0,589.0){\rule[-0.200pt]{2.409pt}{0.400pt}}
\put(170.0,589.0){\rule[-0.200pt]{2.409pt}{0.400pt}}
\put(1429.0,589.0){\rule[-0.200pt]{2.409pt}{0.400pt}}
\put(170.0,589.0){\rule[-0.200pt]{2.409pt}{0.400pt}}
\put(1429.0,589.0){\rule[-0.200pt]{2.409pt}{0.400pt}}
\put(170.0,589.0){\rule[-0.200pt]{2.409pt}{0.400pt}}
\put(1429.0,589.0){\rule[-0.200pt]{2.409pt}{0.400pt}}
\put(170.0,589.0){\rule[-0.200pt]{2.409pt}{0.400pt}}
\put(1429.0,589.0){\rule[-0.200pt]{2.409pt}{0.400pt}}
\put(170.0,589.0){\rule[-0.200pt]{2.409pt}{0.400pt}}
\put(1429.0,589.0){\rule[-0.200pt]{2.409pt}{0.400pt}}
\put(170.0,589.0){\rule[-0.200pt]{2.409pt}{0.400pt}}
\put(1429.0,589.0){\rule[-0.200pt]{2.409pt}{0.400pt}}
\put(170.0,589.0){\rule[-0.200pt]{2.409pt}{0.400pt}}
\put(1429.0,589.0){\rule[-0.200pt]{2.409pt}{0.400pt}}
\put(170.0,589.0){\rule[-0.200pt]{2.409pt}{0.400pt}}
\put(1429.0,589.0){\rule[-0.200pt]{2.409pt}{0.400pt}}
\put(170.0,589.0){\rule[-0.200pt]{2.409pt}{0.400pt}}
\put(1429.0,589.0){\rule[-0.200pt]{2.409pt}{0.400pt}}
\put(170.0,589.0){\rule[-0.200pt]{2.409pt}{0.400pt}}
\put(1429.0,589.0){\rule[-0.200pt]{2.409pt}{0.400pt}}
\put(170.0,589.0){\rule[-0.200pt]{2.409pt}{0.400pt}}
\put(1429.0,589.0){\rule[-0.200pt]{2.409pt}{0.400pt}}
\put(170.0,590.0){\rule[-0.200pt]{2.409pt}{0.400pt}}
\put(1429.0,590.0){\rule[-0.200pt]{2.409pt}{0.400pt}}
\put(170.0,590.0){\rule[-0.200pt]{2.409pt}{0.400pt}}
\put(1429.0,590.0){\rule[-0.200pt]{2.409pt}{0.400pt}}
\put(170.0,590.0){\rule[-0.200pt]{2.409pt}{0.400pt}}
\put(1429.0,590.0){\rule[-0.200pt]{2.409pt}{0.400pt}}
\put(170.0,590.0){\rule[-0.200pt]{2.409pt}{0.400pt}}
\put(1429.0,590.0){\rule[-0.200pt]{2.409pt}{0.400pt}}
\put(170.0,590.0){\rule[-0.200pt]{2.409pt}{0.400pt}}
\put(1429.0,590.0){\rule[-0.200pt]{2.409pt}{0.400pt}}
\put(170.0,590.0){\rule[-0.200pt]{2.409pt}{0.400pt}}
\put(1429.0,590.0){\rule[-0.200pt]{2.409pt}{0.400pt}}
\put(170.0,590.0){\rule[-0.200pt]{2.409pt}{0.400pt}}
\put(1429.0,590.0){\rule[-0.200pt]{2.409pt}{0.400pt}}
\put(170.0,590.0){\rule[-0.200pt]{2.409pt}{0.400pt}}
\put(1429.0,590.0){\rule[-0.200pt]{2.409pt}{0.400pt}}
\put(170.0,590.0){\rule[-0.200pt]{2.409pt}{0.400pt}}
\put(1429.0,590.0){\rule[-0.200pt]{2.409pt}{0.400pt}}
\put(170.0,590.0){\rule[-0.200pt]{2.409pt}{0.400pt}}
\put(1429.0,590.0){\rule[-0.200pt]{2.409pt}{0.400pt}}
\put(170.0,590.0){\rule[-0.200pt]{2.409pt}{0.400pt}}
\put(1429.0,590.0){\rule[-0.200pt]{2.409pt}{0.400pt}}
\put(170.0,590.0){\rule[-0.200pt]{2.409pt}{0.400pt}}
\put(1429.0,590.0){\rule[-0.200pt]{2.409pt}{0.400pt}}
\put(170.0,590.0){\rule[-0.200pt]{2.409pt}{0.400pt}}
\put(1429.0,590.0){\rule[-0.200pt]{2.409pt}{0.400pt}}
\put(170.0,590.0){\rule[-0.200pt]{2.409pt}{0.400pt}}
\put(1429.0,590.0){\rule[-0.200pt]{2.409pt}{0.400pt}}
\put(170.0,590.0){\rule[-0.200pt]{2.409pt}{0.400pt}}
\put(1429.0,590.0){\rule[-0.200pt]{2.409pt}{0.400pt}}
\put(170.0,590.0){\rule[-0.200pt]{2.409pt}{0.400pt}}
\put(1429.0,590.0){\rule[-0.200pt]{2.409pt}{0.400pt}}
\put(170.0,590.0){\rule[-0.200pt]{2.409pt}{0.400pt}}
\put(1429.0,590.0){\rule[-0.200pt]{2.409pt}{0.400pt}}
\put(170.0,590.0){\rule[-0.200pt]{2.409pt}{0.400pt}}
\put(1429.0,590.0){\rule[-0.200pt]{2.409pt}{0.400pt}}
\put(170.0,590.0){\rule[-0.200pt]{2.409pt}{0.400pt}}
\put(1429.0,590.0){\rule[-0.200pt]{2.409pt}{0.400pt}}
\put(170.0,590.0){\rule[-0.200pt]{2.409pt}{0.400pt}}
\put(1429.0,590.0){\rule[-0.200pt]{2.409pt}{0.400pt}}
\put(170.0,590.0){\rule[-0.200pt]{2.409pt}{0.400pt}}
\put(1429.0,590.0){\rule[-0.200pt]{2.409pt}{0.400pt}}
\put(170.0,591.0){\rule[-0.200pt]{2.409pt}{0.400pt}}
\put(1429.0,591.0){\rule[-0.200pt]{2.409pt}{0.400pt}}
\put(170.0,591.0){\rule[-0.200pt]{2.409pt}{0.400pt}}
\put(1429.0,591.0){\rule[-0.200pt]{2.409pt}{0.400pt}}
\put(170.0,591.0){\rule[-0.200pt]{2.409pt}{0.400pt}}
\put(1429.0,591.0){\rule[-0.200pt]{2.409pt}{0.400pt}}
\put(170.0,591.0){\rule[-0.200pt]{2.409pt}{0.400pt}}
\put(1429.0,591.0){\rule[-0.200pt]{2.409pt}{0.400pt}}
\put(170.0,591.0){\rule[-0.200pt]{2.409pt}{0.400pt}}
\put(1429.0,591.0){\rule[-0.200pt]{2.409pt}{0.400pt}}
\put(170.0,591.0){\rule[-0.200pt]{2.409pt}{0.400pt}}
\put(1429.0,591.0){\rule[-0.200pt]{2.409pt}{0.400pt}}
\put(170.0,591.0){\rule[-0.200pt]{2.409pt}{0.400pt}}
\put(1429.0,591.0){\rule[-0.200pt]{2.409pt}{0.400pt}}
\put(170.0,591.0){\rule[-0.200pt]{2.409pt}{0.400pt}}
\put(1429.0,591.0){\rule[-0.200pt]{2.409pt}{0.400pt}}
\put(170.0,591.0){\rule[-0.200pt]{2.409pt}{0.400pt}}
\put(1429.0,591.0){\rule[-0.200pt]{2.409pt}{0.400pt}}
\put(170.0,591.0){\rule[-0.200pt]{2.409pt}{0.400pt}}
\put(1429.0,591.0){\rule[-0.200pt]{2.409pt}{0.400pt}}
\put(170.0,591.0){\rule[-0.200pt]{2.409pt}{0.400pt}}
\put(1429.0,591.0){\rule[-0.200pt]{2.409pt}{0.400pt}}
\put(170.0,591.0){\rule[-0.200pt]{2.409pt}{0.400pt}}
\put(1429.0,591.0){\rule[-0.200pt]{2.409pt}{0.400pt}}
\put(170.0,591.0){\rule[-0.200pt]{2.409pt}{0.400pt}}
\put(1429.0,591.0){\rule[-0.200pt]{2.409pt}{0.400pt}}
\put(170.0,591.0){\rule[-0.200pt]{2.409pt}{0.400pt}}
\put(1429.0,591.0){\rule[-0.200pt]{2.409pt}{0.400pt}}
\put(170.0,591.0){\rule[-0.200pt]{2.409pt}{0.400pt}}
\put(1429.0,591.0){\rule[-0.200pt]{2.409pt}{0.400pt}}
\put(170.0,591.0){\rule[-0.200pt]{2.409pt}{0.400pt}}
\put(1429.0,591.0){\rule[-0.200pt]{2.409pt}{0.400pt}}
\put(170.0,591.0){\rule[-0.200pt]{2.409pt}{0.400pt}}
\put(1429.0,591.0){\rule[-0.200pt]{2.409pt}{0.400pt}}
\put(170.0,591.0){\rule[-0.200pt]{2.409pt}{0.400pt}}
\put(1429.0,591.0){\rule[-0.200pt]{2.409pt}{0.400pt}}
\put(170.0,591.0){\rule[-0.200pt]{2.409pt}{0.400pt}}
\put(1429.0,591.0){\rule[-0.200pt]{2.409pt}{0.400pt}}
\put(170.0,591.0){\rule[-0.200pt]{2.409pt}{0.400pt}}
\put(1429.0,591.0){\rule[-0.200pt]{2.409pt}{0.400pt}}
\put(170.0,591.0){\rule[-0.200pt]{2.409pt}{0.400pt}}
\put(1429.0,591.0){\rule[-0.200pt]{2.409pt}{0.400pt}}
\put(170.0,592.0){\rule[-0.200pt]{2.409pt}{0.400pt}}
\put(1429.0,592.0){\rule[-0.200pt]{2.409pt}{0.400pt}}
\put(170.0,592.0){\rule[-0.200pt]{2.409pt}{0.400pt}}
\put(1429.0,592.0){\rule[-0.200pt]{2.409pt}{0.400pt}}
\put(170.0,592.0){\rule[-0.200pt]{2.409pt}{0.400pt}}
\put(1429.0,592.0){\rule[-0.200pt]{2.409pt}{0.400pt}}
\put(170.0,592.0){\rule[-0.200pt]{2.409pt}{0.400pt}}
\put(1429.0,592.0){\rule[-0.200pt]{2.409pt}{0.400pt}}
\put(170.0,592.0){\rule[-0.200pt]{2.409pt}{0.400pt}}
\put(1429.0,592.0){\rule[-0.200pt]{2.409pt}{0.400pt}}
\put(170.0,592.0){\rule[-0.200pt]{2.409pt}{0.400pt}}
\put(1429.0,592.0){\rule[-0.200pt]{2.409pt}{0.400pt}}
\put(170.0,592.0){\rule[-0.200pt]{2.409pt}{0.400pt}}
\put(1429.0,592.0){\rule[-0.200pt]{2.409pt}{0.400pt}}
\put(170.0,592.0){\rule[-0.200pt]{2.409pt}{0.400pt}}
\put(1429.0,592.0){\rule[-0.200pt]{2.409pt}{0.400pt}}
\put(170.0,592.0){\rule[-0.200pt]{2.409pt}{0.400pt}}
\put(1429.0,592.0){\rule[-0.200pt]{2.409pt}{0.400pt}}
\put(170.0,592.0){\rule[-0.200pt]{2.409pt}{0.400pt}}
\put(1429.0,592.0){\rule[-0.200pt]{2.409pt}{0.400pt}}
\put(170.0,592.0){\rule[-0.200pt]{2.409pt}{0.400pt}}
\put(1429.0,592.0){\rule[-0.200pt]{2.409pt}{0.400pt}}
\put(170.0,592.0){\rule[-0.200pt]{2.409pt}{0.400pt}}
\put(1429.0,592.0){\rule[-0.200pt]{2.409pt}{0.400pt}}
\put(170.0,592.0){\rule[-0.200pt]{2.409pt}{0.400pt}}
\put(1429.0,592.0){\rule[-0.200pt]{2.409pt}{0.400pt}}
\put(170.0,592.0){\rule[-0.200pt]{2.409pt}{0.400pt}}
\put(1429.0,592.0){\rule[-0.200pt]{2.409pt}{0.400pt}}
\put(170.0,592.0){\rule[-0.200pt]{2.409pt}{0.400pt}}
\put(1429.0,592.0){\rule[-0.200pt]{2.409pt}{0.400pt}}
\put(170.0,592.0){\rule[-0.200pt]{2.409pt}{0.400pt}}
\put(1429.0,592.0){\rule[-0.200pt]{2.409pt}{0.400pt}}
\put(170.0,592.0){\rule[-0.200pt]{2.409pt}{0.400pt}}
\put(1429.0,592.0){\rule[-0.200pt]{2.409pt}{0.400pt}}
\put(170.0,592.0){\rule[-0.200pt]{2.409pt}{0.400pt}}
\put(1429.0,592.0){\rule[-0.200pt]{2.409pt}{0.400pt}}
\put(170.0,592.0){\rule[-0.200pt]{2.409pt}{0.400pt}}
\put(1429.0,592.0){\rule[-0.200pt]{2.409pt}{0.400pt}}
\put(170.0,592.0){\rule[-0.200pt]{2.409pt}{0.400pt}}
\put(1429.0,592.0){\rule[-0.200pt]{2.409pt}{0.400pt}}
\put(170.0,592.0){\rule[-0.200pt]{2.409pt}{0.400pt}}
\put(1429.0,592.0){\rule[-0.200pt]{2.409pt}{0.400pt}}
\put(170.0,593.0){\rule[-0.200pt]{2.409pt}{0.400pt}}
\put(1429.0,593.0){\rule[-0.200pt]{2.409pt}{0.400pt}}
\put(170.0,593.0){\rule[-0.200pt]{2.409pt}{0.400pt}}
\put(1429.0,593.0){\rule[-0.200pt]{2.409pt}{0.400pt}}
\put(170.0,593.0){\rule[-0.200pt]{2.409pt}{0.400pt}}
\put(1429.0,593.0){\rule[-0.200pt]{2.409pt}{0.400pt}}
\put(170.0,593.0){\rule[-0.200pt]{2.409pt}{0.400pt}}
\put(1429.0,593.0){\rule[-0.200pt]{2.409pt}{0.400pt}}
\put(170.0,593.0){\rule[-0.200pt]{2.409pt}{0.400pt}}
\put(1429.0,593.0){\rule[-0.200pt]{2.409pt}{0.400pt}}
\put(170.0,593.0){\rule[-0.200pt]{2.409pt}{0.400pt}}
\put(1429.0,593.0){\rule[-0.200pt]{2.409pt}{0.400pt}}
\put(170.0,593.0){\rule[-0.200pt]{2.409pt}{0.400pt}}
\put(1429.0,593.0){\rule[-0.200pt]{2.409pt}{0.400pt}}
\put(170.0,593.0){\rule[-0.200pt]{2.409pt}{0.400pt}}
\put(1429.0,593.0){\rule[-0.200pt]{2.409pt}{0.400pt}}
\put(170.0,593.0){\rule[-0.200pt]{2.409pt}{0.400pt}}
\put(1429.0,593.0){\rule[-0.200pt]{2.409pt}{0.400pt}}
\put(170.0,593.0){\rule[-0.200pt]{2.409pt}{0.400pt}}
\put(1429.0,593.0){\rule[-0.200pt]{2.409pt}{0.400pt}}
\put(170.0,593.0){\rule[-0.200pt]{2.409pt}{0.400pt}}
\put(1429.0,593.0){\rule[-0.200pt]{2.409pt}{0.400pt}}
\put(170.0,593.0){\rule[-0.200pt]{2.409pt}{0.400pt}}
\put(1429.0,593.0){\rule[-0.200pt]{2.409pt}{0.400pt}}
\put(170.0,593.0){\rule[-0.200pt]{2.409pt}{0.400pt}}
\put(1429.0,593.0){\rule[-0.200pt]{2.409pt}{0.400pt}}
\put(170.0,593.0){\rule[-0.200pt]{2.409pt}{0.400pt}}
\put(1429.0,593.0){\rule[-0.200pt]{2.409pt}{0.400pt}}
\put(170.0,593.0){\rule[-0.200pt]{2.409pt}{0.400pt}}
\put(1429.0,593.0){\rule[-0.200pt]{2.409pt}{0.400pt}}
\put(170.0,593.0){\rule[-0.200pt]{2.409pt}{0.400pt}}
\put(1429.0,593.0){\rule[-0.200pt]{2.409pt}{0.400pt}}
\put(170.0,593.0){\rule[-0.200pt]{2.409pt}{0.400pt}}
\put(1429.0,593.0){\rule[-0.200pt]{2.409pt}{0.400pt}}
\put(170.0,593.0){\rule[-0.200pt]{2.409pt}{0.400pt}}
\put(1429.0,593.0){\rule[-0.200pt]{2.409pt}{0.400pt}}
\put(170.0,593.0){\rule[-0.200pt]{2.409pt}{0.400pt}}
\put(1429.0,593.0){\rule[-0.200pt]{2.409pt}{0.400pt}}
\put(170.0,593.0){\rule[-0.200pt]{2.409pt}{0.400pt}}
\put(1429.0,593.0){\rule[-0.200pt]{2.409pt}{0.400pt}}
\put(170.0,593.0){\rule[-0.200pt]{2.409pt}{0.400pt}}
\put(1429.0,593.0){\rule[-0.200pt]{2.409pt}{0.400pt}}
\put(170.0,593.0){\rule[-0.200pt]{2.409pt}{0.400pt}}
\put(1429.0,593.0){\rule[-0.200pt]{2.409pt}{0.400pt}}
\put(170.0,594.0){\rule[-0.200pt]{2.409pt}{0.400pt}}
\put(1429.0,594.0){\rule[-0.200pt]{2.409pt}{0.400pt}}
\put(170.0,594.0){\rule[-0.200pt]{2.409pt}{0.400pt}}
\put(1429.0,594.0){\rule[-0.200pt]{2.409pt}{0.400pt}}
\put(170.0,594.0){\rule[-0.200pt]{2.409pt}{0.400pt}}
\put(1429.0,594.0){\rule[-0.200pt]{2.409pt}{0.400pt}}
\put(170.0,594.0){\rule[-0.200pt]{2.409pt}{0.400pt}}
\put(1429.0,594.0){\rule[-0.200pt]{2.409pt}{0.400pt}}
\put(170.0,594.0){\rule[-0.200pt]{2.409pt}{0.400pt}}
\put(1429.0,594.0){\rule[-0.200pt]{2.409pt}{0.400pt}}
\put(170.0,594.0){\rule[-0.200pt]{2.409pt}{0.400pt}}
\put(1429.0,594.0){\rule[-0.200pt]{2.409pt}{0.400pt}}
\put(170.0,594.0){\rule[-0.200pt]{2.409pt}{0.400pt}}
\put(1429.0,594.0){\rule[-0.200pt]{2.409pt}{0.400pt}}
\put(170.0,594.0){\rule[-0.200pt]{2.409pt}{0.400pt}}
\put(1429.0,594.0){\rule[-0.200pt]{2.409pt}{0.400pt}}
\put(170.0,594.0){\rule[-0.200pt]{2.409pt}{0.400pt}}
\put(1429.0,594.0){\rule[-0.200pt]{2.409pt}{0.400pt}}
\put(170.0,594.0){\rule[-0.200pt]{2.409pt}{0.400pt}}
\put(1429.0,594.0){\rule[-0.200pt]{2.409pt}{0.400pt}}
\put(170.0,594.0){\rule[-0.200pt]{2.409pt}{0.400pt}}
\put(1429.0,594.0){\rule[-0.200pt]{2.409pt}{0.400pt}}
\put(170.0,594.0){\rule[-0.200pt]{2.409pt}{0.400pt}}
\put(1429.0,594.0){\rule[-0.200pt]{2.409pt}{0.400pt}}
\put(170.0,594.0){\rule[-0.200pt]{2.409pt}{0.400pt}}
\put(1429.0,594.0){\rule[-0.200pt]{2.409pt}{0.400pt}}
\put(170.0,594.0){\rule[-0.200pt]{2.409pt}{0.400pt}}
\put(1429.0,594.0){\rule[-0.200pt]{2.409pt}{0.400pt}}
\put(170.0,594.0){\rule[-0.200pt]{2.409pt}{0.400pt}}
\put(1429.0,594.0){\rule[-0.200pt]{2.409pt}{0.400pt}}
\put(170.0,594.0){\rule[-0.200pt]{2.409pt}{0.400pt}}
\put(1429.0,594.0){\rule[-0.200pt]{2.409pt}{0.400pt}}
\put(170.0,594.0){\rule[-0.200pt]{2.409pt}{0.400pt}}
\put(1429.0,594.0){\rule[-0.200pt]{2.409pt}{0.400pt}}
\put(170.0,594.0){\rule[-0.200pt]{2.409pt}{0.400pt}}
\put(1429.0,594.0){\rule[-0.200pt]{2.409pt}{0.400pt}}
\put(170.0,594.0){\rule[-0.200pt]{2.409pt}{0.400pt}}
\put(1429.0,594.0){\rule[-0.200pt]{2.409pt}{0.400pt}}
\put(170.0,594.0){\rule[-0.200pt]{2.409pt}{0.400pt}}
\put(1429.0,594.0){\rule[-0.200pt]{2.409pt}{0.400pt}}
\put(170.0,594.0){\rule[-0.200pt]{2.409pt}{0.400pt}}
\put(1429.0,594.0){\rule[-0.200pt]{2.409pt}{0.400pt}}
\put(170.0,594.0){\rule[-0.200pt]{2.409pt}{0.400pt}}
\put(1429.0,594.0){\rule[-0.200pt]{2.409pt}{0.400pt}}
\put(170.0,594.0){\rule[-0.200pt]{2.409pt}{0.400pt}}
\put(1429.0,594.0){\rule[-0.200pt]{2.409pt}{0.400pt}}
\put(170.0,595.0){\rule[-0.200pt]{2.409pt}{0.400pt}}
\put(1429.0,595.0){\rule[-0.200pt]{2.409pt}{0.400pt}}
\put(170.0,595.0){\rule[-0.200pt]{2.409pt}{0.400pt}}
\put(1429.0,595.0){\rule[-0.200pt]{2.409pt}{0.400pt}}
\put(170.0,595.0){\rule[-0.200pt]{2.409pt}{0.400pt}}
\put(1429.0,595.0){\rule[-0.200pt]{2.409pt}{0.400pt}}
\put(170.0,595.0){\rule[-0.200pt]{2.409pt}{0.400pt}}
\put(1429.0,595.0){\rule[-0.200pt]{2.409pt}{0.400pt}}
\put(170.0,595.0){\rule[-0.200pt]{2.409pt}{0.400pt}}
\put(1429.0,595.0){\rule[-0.200pt]{2.409pt}{0.400pt}}
\put(170.0,595.0){\rule[-0.200pt]{2.409pt}{0.400pt}}
\put(1429.0,595.0){\rule[-0.200pt]{2.409pt}{0.400pt}}
\put(170.0,595.0){\rule[-0.200pt]{2.409pt}{0.400pt}}
\put(1429.0,595.0){\rule[-0.200pt]{2.409pt}{0.400pt}}
\put(170.0,595.0){\rule[-0.200pt]{2.409pt}{0.400pt}}
\put(1429.0,595.0){\rule[-0.200pt]{2.409pt}{0.400pt}}
\put(170.0,595.0){\rule[-0.200pt]{2.409pt}{0.400pt}}
\put(1429.0,595.0){\rule[-0.200pt]{2.409pt}{0.400pt}}
\put(170.0,595.0){\rule[-0.200pt]{2.409pt}{0.400pt}}
\put(1429.0,595.0){\rule[-0.200pt]{2.409pt}{0.400pt}}
\put(170.0,595.0){\rule[-0.200pt]{2.409pt}{0.400pt}}
\put(1429.0,595.0){\rule[-0.200pt]{2.409pt}{0.400pt}}
\put(170.0,595.0){\rule[-0.200pt]{2.409pt}{0.400pt}}
\put(1429.0,595.0){\rule[-0.200pt]{2.409pt}{0.400pt}}
\put(170.0,595.0){\rule[-0.200pt]{2.409pt}{0.400pt}}
\put(1429.0,595.0){\rule[-0.200pt]{2.409pt}{0.400pt}}
\put(170.0,595.0){\rule[-0.200pt]{2.409pt}{0.400pt}}
\put(1429.0,595.0){\rule[-0.200pt]{2.409pt}{0.400pt}}
\put(170.0,595.0){\rule[-0.200pt]{2.409pt}{0.400pt}}
\put(1429.0,595.0){\rule[-0.200pt]{2.409pt}{0.400pt}}
\put(170.0,595.0){\rule[-0.200pt]{2.409pt}{0.400pt}}
\put(1429.0,595.0){\rule[-0.200pt]{2.409pt}{0.400pt}}
\put(170.0,595.0){\rule[-0.200pt]{2.409pt}{0.400pt}}
\put(1429.0,595.0){\rule[-0.200pt]{2.409pt}{0.400pt}}
\put(170.0,595.0){\rule[-0.200pt]{2.409pt}{0.400pt}}
\put(1429.0,595.0){\rule[-0.200pt]{2.409pt}{0.400pt}}
\put(170.0,595.0){\rule[-0.200pt]{2.409pt}{0.400pt}}
\put(1429.0,595.0){\rule[-0.200pt]{2.409pt}{0.400pt}}
\put(170.0,595.0){\rule[-0.200pt]{2.409pt}{0.400pt}}
\put(1429.0,595.0){\rule[-0.200pt]{2.409pt}{0.400pt}}
\put(170.0,595.0){\rule[-0.200pt]{2.409pt}{0.400pt}}
\put(1429.0,595.0){\rule[-0.200pt]{2.409pt}{0.400pt}}
\put(170.0,595.0){\rule[-0.200pt]{2.409pt}{0.400pt}}
\put(1429.0,595.0){\rule[-0.200pt]{2.409pt}{0.400pt}}
\put(170.0,595.0){\rule[-0.200pt]{2.409pt}{0.400pt}}
\put(1429.0,595.0){\rule[-0.200pt]{2.409pt}{0.400pt}}
\put(170.0,596.0){\rule[-0.200pt]{2.409pt}{0.400pt}}
\put(1429.0,596.0){\rule[-0.200pt]{2.409pt}{0.400pt}}
\put(170.0,596.0){\rule[-0.200pt]{2.409pt}{0.400pt}}
\put(1429.0,596.0){\rule[-0.200pt]{2.409pt}{0.400pt}}
\put(170.0,596.0){\rule[-0.200pt]{2.409pt}{0.400pt}}
\put(1429.0,596.0){\rule[-0.200pt]{2.409pt}{0.400pt}}
\put(170.0,596.0){\rule[-0.200pt]{2.409pt}{0.400pt}}
\put(1429.0,596.0){\rule[-0.200pt]{2.409pt}{0.400pt}}
\put(170.0,596.0){\rule[-0.200pt]{2.409pt}{0.400pt}}
\put(1429.0,596.0){\rule[-0.200pt]{2.409pt}{0.400pt}}
\put(170.0,596.0){\rule[-0.200pt]{2.409pt}{0.400pt}}
\put(1429.0,596.0){\rule[-0.200pt]{2.409pt}{0.400pt}}
\put(170.0,596.0){\rule[-0.200pt]{2.409pt}{0.400pt}}
\put(1429.0,596.0){\rule[-0.200pt]{2.409pt}{0.400pt}}
\put(170.0,596.0){\rule[-0.200pt]{2.409pt}{0.400pt}}
\put(1429.0,596.0){\rule[-0.200pt]{2.409pt}{0.400pt}}
\put(170.0,596.0){\rule[-0.200pt]{2.409pt}{0.400pt}}
\put(1429.0,596.0){\rule[-0.200pt]{2.409pt}{0.400pt}}
\put(170.0,596.0){\rule[-0.200pt]{2.409pt}{0.400pt}}
\put(1429.0,596.0){\rule[-0.200pt]{2.409pt}{0.400pt}}
\put(170.0,596.0){\rule[-0.200pt]{2.409pt}{0.400pt}}
\put(1429.0,596.0){\rule[-0.200pt]{2.409pt}{0.400pt}}
\put(170.0,596.0){\rule[-0.200pt]{2.409pt}{0.400pt}}
\put(1429.0,596.0){\rule[-0.200pt]{2.409pt}{0.400pt}}
\put(170.0,596.0){\rule[-0.200pt]{2.409pt}{0.400pt}}
\put(1429.0,596.0){\rule[-0.200pt]{2.409pt}{0.400pt}}
\put(170.0,596.0){\rule[-0.200pt]{2.409pt}{0.400pt}}
\put(1429.0,596.0){\rule[-0.200pt]{2.409pt}{0.400pt}}
\put(170.0,596.0){\rule[-0.200pt]{2.409pt}{0.400pt}}
\put(1429.0,596.0){\rule[-0.200pt]{2.409pt}{0.400pt}}
\put(170.0,596.0){\rule[-0.200pt]{2.409pt}{0.400pt}}
\put(1429.0,596.0){\rule[-0.200pt]{2.409pt}{0.400pt}}
\put(170.0,596.0){\rule[-0.200pt]{2.409pt}{0.400pt}}
\put(1429.0,596.0){\rule[-0.200pt]{2.409pt}{0.400pt}}
\put(170.0,596.0){\rule[-0.200pt]{2.409pt}{0.400pt}}
\put(1429.0,596.0){\rule[-0.200pt]{2.409pt}{0.400pt}}
\put(170.0,596.0){\rule[-0.200pt]{2.409pt}{0.400pt}}
\put(1429.0,596.0){\rule[-0.200pt]{2.409pt}{0.400pt}}
\put(170.0,596.0){\rule[-0.200pt]{2.409pt}{0.400pt}}
\put(1429.0,596.0){\rule[-0.200pt]{2.409pt}{0.400pt}}
\put(170.0,596.0){\rule[-0.200pt]{2.409pt}{0.400pt}}
\put(1429.0,596.0){\rule[-0.200pt]{2.409pt}{0.400pt}}
\put(170.0,596.0){\rule[-0.200pt]{2.409pt}{0.400pt}}
\put(1429.0,596.0){\rule[-0.200pt]{2.409pt}{0.400pt}}
\put(170.0,596.0){\rule[-0.200pt]{2.409pt}{0.400pt}}
\put(1429.0,596.0){\rule[-0.200pt]{2.409pt}{0.400pt}}
\put(170.0,596.0){\rule[-0.200pt]{2.409pt}{0.400pt}}
\put(1429.0,596.0){\rule[-0.200pt]{2.409pt}{0.400pt}}
\put(170.0,597.0){\rule[-0.200pt]{2.409pt}{0.400pt}}
\put(1429.0,597.0){\rule[-0.200pt]{2.409pt}{0.400pt}}
\put(170.0,597.0){\rule[-0.200pt]{2.409pt}{0.400pt}}
\put(1429.0,597.0){\rule[-0.200pt]{2.409pt}{0.400pt}}
\put(170.0,597.0){\rule[-0.200pt]{2.409pt}{0.400pt}}
\put(1429.0,597.0){\rule[-0.200pt]{2.409pt}{0.400pt}}
\put(170.0,597.0){\rule[-0.200pt]{2.409pt}{0.400pt}}
\put(1429.0,597.0){\rule[-0.200pt]{2.409pt}{0.400pt}}
\put(170.0,597.0){\rule[-0.200pt]{2.409pt}{0.400pt}}
\put(1429.0,597.0){\rule[-0.200pt]{2.409pt}{0.400pt}}
\put(170.0,597.0){\rule[-0.200pt]{2.409pt}{0.400pt}}
\put(1429.0,597.0){\rule[-0.200pt]{2.409pt}{0.400pt}}
\put(170.0,597.0){\rule[-0.200pt]{2.409pt}{0.400pt}}
\put(1429.0,597.0){\rule[-0.200pt]{2.409pt}{0.400pt}}
\put(170.0,597.0){\rule[-0.200pt]{2.409pt}{0.400pt}}
\put(1429.0,597.0){\rule[-0.200pt]{2.409pt}{0.400pt}}
\put(170.0,597.0){\rule[-0.200pt]{2.409pt}{0.400pt}}
\put(1429.0,597.0){\rule[-0.200pt]{2.409pt}{0.400pt}}
\put(170.0,597.0){\rule[-0.200pt]{2.409pt}{0.400pt}}
\put(1429.0,597.0){\rule[-0.200pt]{2.409pt}{0.400pt}}
\put(170.0,597.0){\rule[-0.200pt]{2.409pt}{0.400pt}}
\put(1429.0,597.0){\rule[-0.200pt]{2.409pt}{0.400pt}}
\put(170.0,597.0){\rule[-0.200pt]{2.409pt}{0.400pt}}
\put(1429.0,597.0){\rule[-0.200pt]{2.409pt}{0.400pt}}
\put(170.0,597.0){\rule[-0.200pt]{2.409pt}{0.400pt}}
\put(1429.0,597.0){\rule[-0.200pt]{2.409pt}{0.400pt}}
\put(170.0,597.0){\rule[-0.200pt]{2.409pt}{0.400pt}}
\put(1429.0,597.0){\rule[-0.200pt]{2.409pt}{0.400pt}}
\put(170.0,597.0){\rule[-0.200pt]{2.409pt}{0.400pt}}
\put(1429.0,597.0){\rule[-0.200pt]{2.409pt}{0.400pt}}
\put(170.0,597.0){\rule[-0.200pt]{2.409pt}{0.400pt}}
\put(1429.0,597.0){\rule[-0.200pt]{2.409pt}{0.400pt}}
\put(170.0,597.0){\rule[-0.200pt]{2.409pt}{0.400pt}}
\put(1429.0,597.0){\rule[-0.200pt]{2.409pt}{0.400pt}}
\put(170.0,597.0){\rule[-0.200pt]{2.409pt}{0.400pt}}
\put(1429.0,597.0){\rule[-0.200pt]{2.409pt}{0.400pt}}
\put(170.0,597.0){\rule[-0.200pt]{2.409pt}{0.400pt}}
\put(1429.0,597.0){\rule[-0.200pt]{2.409pt}{0.400pt}}
\put(170.0,597.0){\rule[-0.200pt]{2.409pt}{0.400pt}}
\put(1429.0,597.0){\rule[-0.200pt]{2.409pt}{0.400pt}}
\put(170.0,597.0){\rule[-0.200pt]{2.409pt}{0.400pt}}
\put(1429.0,597.0){\rule[-0.200pt]{2.409pt}{0.400pt}}
\put(170.0,597.0){\rule[-0.200pt]{2.409pt}{0.400pt}}
\put(1429.0,597.0){\rule[-0.200pt]{2.409pt}{0.400pt}}
\put(170.0,597.0){\rule[-0.200pt]{2.409pt}{0.400pt}}
\put(1429.0,597.0){\rule[-0.200pt]{2.409pt}{0.400pt}}
\put(170.0,597.0){\rule[-0.200pt]{2.409pt}{0.400pt}}
\put(1429.0,597.0){\rule[-0.200pt]{2.409pt}{0.400pt}}
\put(170.0,597.0){\rule[-0.200pt]{2.409pt}{0.400pt}}
\put(1429.0,597.0){\rule[-0.200pt]{2.409pt}{0.400pt}}
\put(170.0,598.0){\rule[-0.200pt]{2.409pt}{0.400pt}}
\put(1429.0,598.0){\rule[-0.200pt]{2.409pt}{0.400pt}}
\put(170.0,598.0){\rule[-0.200pt]{2.409pt}{0.400pt}}
\put(1429.0,598.0){\rule[-0.200pt]{2.409pt}{0.400pt}}
\put(170.0,598.0){\rule[-0.200pt]{2.409pt}{0.400pt}}
\put(1429.0,598.0){\rule[-0.200pt]{2.409pt}{0.400pt}}
\put(170.0,598.0){\rule[-0.200pt]{2.409pt}{0.400pt}}
\put(1429.0,598.0){\rule[-0.200pt]{2.409pt}{0.400pt}}
\put(170.0,598.0){\rule[-0.200pt]{2.409pt}{0.400pt}}
\put(1429.0,598.0){\rule[-0.200pt]{2.409pt}{0.400pt}}
\put(170.0,598.0){\rule[-0.200pt]{2.409pt}{0.400pt}}
\put(1429.0,598.0){\rule[-0.200pt]{2.409pt}{0.400pt}}
\put(170.0,598.0){\rule[-0.200pt]{2.409pt}{0.400pt}}
\put(1429.0,598.0){\rule[-0.200pt]{2.409pt}{0.400pt}}
\put(170.0,598.0){\rule[-0.200pt]{2.409pt}{0.400pt}}
\put(1429.0,598.0){\rule[-0.200pt]{2.409pt}{0.400pt}}
\put(170.0,598.0){\rule[-0.200pt]{2.409pt}{0.400pt}}
\put(1429.0,598.0){\rule[-0.200pt]{2.409pt}{0.400pt}}
\put(170.0,598.0){\rule[-0.200pt]{2.409pt}{0.400pt}}
\put(1429.0,598.0){\rule[-0.200pt]{2.409pt}{0.400pt}}
\put(170.0,598.0){\rule[-0.200pt]{2.409pt}{0.400pt}}
\put(1429.0,598.0){\rule[-0.200pt]{2.409pt}{0.400pt}}
\put(170.0,598.0){\rule[-0.200pt]{2.409pt}{0.400pt}}
\put(1429.0,598.0){\rule[-0.200pt]{2.409pt}{0.400pt}}
\put(170.0,598.0){\rule[-0.200pt]{2.409pt}{0.400pt}}
\put(1429.0,598.0){\rule[-0.200pt]{2.409pt}{0.400pt}}
\put(170.0,598.0){\rule[-0.200pt]{2.409pt}{0.400pt}}
\put(1429.0,598.0){\rule[-0.200pt]{2.409pt}{0.400pt}}
\put(170.0,598.0){\rule[-0.200pt]{2.409pt}{0.400pt}}
\put(1429.0,598.0){\rule[-0.200pt]{2.409pt}{0.400pt}}
\put(170.0,598.0){\rule[-0.200pt]{2.409pt}{0.400pt}}
\put(1429.0,598.0){\rule[-0.200pt]{2.409pt}{0.400pt}}
\put(170.0,598.0){\rule[-0.200pt]{2.409pt}{0.400pt}}
\put(1429.0,598.0){\rule[-0.200pt]{2.409pt}{0.400pt}}
\put(170.0,598.0){\rule[-0.200pt]{2.409pt}{0.400pt}}
\put(1429.0,598.0){\rule[-0.200pt]{2.409pt}{0.400pt}}
\put(170.0,598.0){\rule[-0.200pt]{2.409pt}{0.400pt}}
\put(1429.0,598.0){\rule[-0.200pt]{2.409pt}{0.400pt}}
\put(170.0,598.0){\rule[-0.200pt]{2.409pt}{0.400pt}}
\put(1429.0,598.0){\rule[-0.200pt]{2.409pt}{0.400pt}}
\put(170.0,598.0){\rule[-0.200pt]{2.409pt}{0.400pt}}
\put(1429.0,598.0){\rule[-0.200pt]{2.409pt}{0.400pt}}
\put(170.0,598.0){\rule[-0.200pt]{2.409pt}{0.400pt}}
\put(1429.0,598.0){\rule[-0.200pt]{2.409pt}{0.400pt}}
\put(170.0,598.0){\rule[-0.200pt]{2.409pt}{0.400pt}}
\put(1429.0,598.0){\rule[-0.200pt]{2.409pt}{0.400pt}}
\put(170.0,598.0){\rule[-0.200pt]{2.409pt}{0.400pt}}
\put(1429.0,598.0){\rule[-0.200pt]{2.409pt}{0.400pt}}
\put(170.0,598.0){\rule[-0.200pt]{2.409pt}{0.400pt}}
\put(1429.0,598.0){\rule[-0.200pt]{2.409pt}{0.400pt}}
\put(170.0,599.0){\rule[-0.200pt]{2.409pt}{0.400pt}}
\put(1429.0,599.0){\rule[-0.200pt]{2.409pt}{0.400pt}}
\put(170.0,599.0){\rule[-0.200pt]{2.409pt}{0.400pt}}
\put(1429.0,599.0){\rule[-0.200pt]{2.409pt}{0.400pt}}
\put(170.0,599.0){\rule[-0.200pt]{2.409pt}{0.400pt}}
\put(1429.0,599.0){\rule[-0.200pt]{2.409pt}{0.400pt}}
\put(170.0,599.0){\rule[-0.200pt]{2.409pt}{0.400pt}}
\put(1429.0,599.0){\rule[-0.200pt]{2.409pt}{0.400pt}}
\put(170.0,599.0){\rule[-0.200pt]{2.409pt}{0.400pt}}
\put(1429.0,599.0){\rule[-0.200pt]{2.409pt}{0.400pt}}
\put(170.0,599.0){\rule[-0.200pt]{2.409pt}{0.400pt}}
\put(1429.0,599.0){\rule[-0.200pt]{2.409pt}{0.400pt}}
\put(170.0,599.0){\rule[-0.200pt]{2.409pt}{0.400pt}}
\put(1429.0,599.0){\rule[-0.200pt]{2.409pt}{0.400pt}}
\put(170.0,599.0){\rule[-0.200pt]{2.409pt}{0.400pt}}
\put(1429.0,599.0){\rule[-0.200pt]{2.409pt}{0.400pt}}
\put(170.0,599.0){\rule[-0.200pt]{2.409pt}{0.400pt}}
\put(1429.0,599.0){\rule[-0.200pt]{2.409pt}{0.400pt}}
\put(170.0,599.0){\rule[-0.200pt]{2.409pt}{0.400pt}}
\put(1429.0,599.0){\rule[-0.200pt]{2.409pt}{0.400pt}}
\put(170.0,599.0){\rule[-0.200pt]{2.409pt}{0.400pt}}
\put(1429.0,599.0){\rule[-0.200pt]{2.409pt}{0.400pt}}
\put(170.0,599.0){\rule[-0.200pt]{2.409pt}{0.400pt}}
\put(1429.0,599.0){\rule[-0.200pt]{2.409pt}{0.400pt}}
\put(170.0,599.0){\rule[-0.200pt]{2.409pt}{0.400pt}}
\put(1429.0,599.0){\rule[-0.200pt]{2.409pt}{0.400pt}}
\put(170.0,599.0){\rule[-0.200pt]{2.409pt}{0.400pt}}
\put(1429.0,599.0){\rule[-0.200pt]{2.409pt}{0.400pt}}
\put(170.0,599.0){\rule[-0.200pt]{2.409pt}{0.400pt}}
\put(1429.0,599.0){\rule[-0.200pt]{2.409pt}{0.400pt}}
\put(170.0,599.0){\rule[-0.200pt]{2.409pt}{0.400pt}}
\put(1429.0,599.0){\rule[-0.200pt]{2.409pt}{0.400pt}}
\put(170.0,599.0){\rule[-0.200pt]{2.409pt}{0.400pt}}
\put(1429.0,599.0){\rule[-0.200pt]{2.409pt}{0.400pt}}
\put(170.0,599.0){\rule[-0.200pt]{2.409pt}{0.400pt}}
\put(1429.0,599.0){\rule[-0.200pt]{2.409pt}{0.400pt}}
\put(170.0,599.0){\rule[-0.200pt]{2.409pt}{0.400pt}}
\put(1429.0,599.0){\rule[-0.200pt]{2.409pt}{0.400pt}}
\put(170.0,599.0){\rule[-0.200pt]{2.409pt}{0.400pt}}
\put(1429.0,599.0){\rule[-0.200pt]{2.409pt}{0.400pt}}
\put(170.0,599.0){\rule[-0.200pt]{2.409pt}{0.400pt}}
\put(1429.0,599.0){\rule[-0.200pt]{2.409pt}{0.400pt}}
\put(170.0,599.0){\rule[-0.200pt]{2.409pt}{0.400pt}}
\put(1429.0,599.0){\rule[-0.200pt]{2.409pt}{0.400pt}}
\put(170.0,599.0){\rule[-0.200pt]{2.409pt}{0.400pt}}
\put(1429.0,599.0){\rule[-0.200pt]{2.409pt}{0.400pt}}
\put(170.0,599.0){\rule[-0.200pt]{2.409pt}{0.400pt}}
\put(1429.0,599.0){\rule[-0.200pt]{2.409pt}{0.400pt}}
\put(170.0,599.0){\rule[-0.200pt]{2.409pt}{0.400pt}}
\put(1429.0,599.0){\rule[-0.200pt]{2.409pt}{0.400pt}}
\put(170.0,599.0){\rule[-0.200pt]{2.409pt}{0.400pt}}
\put(1429.0,599.0){\rule[-0.200pt]{2.409pt}{0.400pt}}
\put(170.0,600.0){\rule[-0.200pt]{2.409pt}{0.400pt}}
\put(1429.0,600.0){\rule[-0.200pt]{2.409pt}{0.400pt}}
\put(170.0,600.0){\rule[-0.200pt]{2.409pt}{0.400pt}}
\put(1429.0,600.0){\rule[-0.200pt]{2.409pt}{0.400pt}}
\put(170.0,600.0){\rule[-0.200pt]{2.409pt}{0.400pt}}
\put(1429.0,600.0){\rule[-0.200pt]{2.409pt}{0.400pt}}
\put(170.0,600.0){\rule[-0.200pt]{2.409pt}{0.400pt}}
\put(1429.0,600.0){\rule[-0.200pt]{2.409pt}{0.400pt}}
\put(170.0,600.0){\rule[-0.200pt]{2.409pt}{0.400pt}}
\put(1429.0,600.0){\rule[-0.200pt]{2.409pt}{0.400pt}}
\put(170.0,600.0){\rule[-0.200pt]{2.409pt}{0.400pt}}
\put(1429.0,600.0){\rule[-0.200pt]{2.409pt}{0.400pt}}
\put(170.0,600.0){\rule[-0.200pt]{2.409pt}{0.400pt}}
\put(1429.0,600.0){\rule[-0.200pt]{2.409pt}{0.400pt}}
\put(170.0,600.0){\rule[-0.200pt]{2.409pt}{0.400pt}}
\put(1429.0,600.0){\rule[-0.200pt]{2.409pt}{0.400pt}}
\put(170.0,600.0){\rule[-0.200pt]{2.409pt}{0.400pt}}
\put(1429.0,600.0){\rule[-0.200pt]{2.409pt}{0.400pt}}
\put(170.0,600.0){\rule[-0.200pt]{2.409pt}{0.400pt}}
\put(1429.0,600.0){\rule[-0.200pt]{2.409pt}{0.400pt}}
\put(170.0,600.0){\rule[-0.200pt]{2.409pt}{0.400pt}}
\put(1429.0,600.0){\rule[-0.200pt]{2.409pt}{0.400pt}}
\put(170.0,600.0){\rule[-0.200pt]{2.409pt}{0.400pt}}
\put(1429.0,600.0){\rule[-0.200pt]{2.409pt}{0.400pt}}
\put(170.0,600.0){\rule[-0.200pt]{2.409pt}{0.400pt}}
\put(1429.0,600.0){\rule[-0.200pt]{2.409pt}{0.400pt}}
\put(170.0,600.0){\rule[-0.200pt]{4.818pt}{0.400pt}}
\put(150,600){\makebox(0,0)[r]{ 1e+06}}
\put(1419.0,600.0){\rule[-0.200pt]{4.818pt}{0.400pt}}
\put(170.0,626.0){\rule[-0.200pt]{2.409pt}{0.400pt}}
\put(1429.0,626.0){\rule[-0.200pt]{2.409pt}{0.400pt}}
\put(170.0,641.0){\rule[-0.200pt]{2.409pt}{0.400pt}}
\put(1429.0,641.0){\rule[-0.200pt]{2.409pt}{0.400pt}}
\put(170.0,652.0){\rule[-0.200pt]{2.409pt}{0.400pt}}
\put(1429.0,652.0){\rule[-0.200pt]{2.409pt}{0.400pt}}
\put(170.0,660.0){\rule[-0.200pt]{2.409pt}{0.400pt}}
\put(1429.0,660.0){\rule[-0.200pt]{2.409pt}{0.400pt}}
\put(170.0,667.0){\rule[-0.200pt]{2.409pt}{0.400pt}}
\put(1429.0,667.0){\rule[-0.200pt]{2.409pt}{0.400pt}}
\put(170.0,673.0){\rule[-0.200pt]{2.409pt}{0.400pt}}
\put(1429.0,673.0){\rule[-0.200pt]{2.409pt}{0.400pt}}
\put(170.0,678.0){\rule[-0.200pt]{2.409pt}{0.400pt}}
\put(1429.0,678.0){\rule[-0.200pt]{2.409pt}{0.400pt}}
\put(170.0,682.0){\rule[-0.200pt]{2.409pt}{0.400pt}}
\put(1429.0,682.0){\rule[-0.200pt]{2.409pt}{0.400pt}}
\put(170.0,686.0){\rule[-0.200pt]{2.409pt}{0.400pt}}
\put(1429.0,686.0){\rule[-0.200pt]{2.409pt}{0.400pt}}
\put(170.0,690.0){\rule[-0.200pt]{2.409pt}{0.400pt}}
\put(1429.0,690.0){\rule[-0.200pt]{2.409pt}{0.400pt}}
\put(170.0,693.0){\rule[-0.200pt]{2.409pt}{0.400pt}}
\put(1429.0,693.0){\rule[-0.200pt]{2.409pt}{0.400pt}}
\put(170.0,696.0){\rule[-0.200pt]{2.409pt}{0.400pt}}
\put(1429.0,696.0){\rule[-0.200pt]{2.409pt}{0.400pt}}
\put(170.0,699.0){\rule[-0.200pt]{2.409pt}{0.400pt}}
\put(1429.0,699.0){\rule[-0.200pt]{2.409pt}{0.400pt}}
\put(170.0,702.0){\rule[-0.200pt]{2.409pt}{0.400pt}}
\put(1429.0,702.0){\rule[-0.200pt]{2.409pt}{0.400pt}}
\put(170.0,704.0){\rule[-0.200pt]{2.409pt}{0.400pt}}
\put(1429.0,704.0){\rule[-0.200pt]{2.409pt}{0.400pt}}
\put(170.0,706.0){\rule[-0.200pt]{2.409pt}{0.400pt}}
\put(1429.0,706.0){\rule[-0.200pt]{2.409pt}{0.400pt}}
\put(170.0,708.0){\rule[-0.200pt]{2.409pt}{0.400pt}}
\put(1429.0,708.0){\rule[-0.200pt]{2.409pt}{0.400pt}}
\put(170.0,710.0){\rule[-0.200pt]{2.409pt}{0.400pt}}
\put(1429.0,710.0){\rule[-0.200pt]{2.409pt}{0.400pt}}
\put(170.0,712.0){\rule[-0.200pt]{2.409pt}{0.400pt}}
\put(1429.0,712.0){\rule[-0.200pt]{2.409pt}{0.400pt}}
\put(170.0,714.0){\rule[-0.200pt]{2.409pt}{0.400pt}}
\put(1429.0,714.0){\rule[-0.200pt]{2.409pt}{0.400pt}}
\put(170.0,716.0){\rule[-0.200pt]{2.409pt}{0.400pt}}
\put(1429.0,716.0){\rule[-0.200pt]{2.409pt}{0.400pt}}
\put(170.0,718.0){\rule[-0.200pt]{2.409pt}{0.400pt}}
\put(1429.0,718.0){\rule[-0.200pt]{2.409pt}{0.400pt}}
\put(170.0,719.0){\rule[-0.200pt]{2.409pt}{0.400pt}}
\put(1429.0,719.0){\rule[-0.200pt]{2.409pt}{0.400pt}}
\put(170.0,721.0){\rule[-0.200pt]{2.409pt}{0.400pt}}
\put(1429.0,721.0){\rule[-0.200pt]{2.409pt}{0.400pt}}
\put(170.0,722.0){\rule[-0.200pt]{2.409pt}{0.400pt}}
\put(1429.0,722.0){\rule[-0.200pt]{2.409pt}{0.400pt}}
\put(170.0,724.0){\rule[-0.200pt]{2.409pt}{0.400pt}}
\put(1429.0,724.0){\rule[-0.200pt]{2.409pt}{0.400pt}}
\put(170.0,725.0){\rule[-0.200pt]{2.409pt}{0.400pt}}
\put(1429.0,725.0){\rule[-0.200pt]{2.409pt}{0.400pt}}
\put(170.0,726.0){\rule[-0.200pt]{2.409pt}{0.400pt}}
\put(1429.0,726.0){\rule[-0.200pt]{2.409pt}{0.400pt}}
\put(170.0,728.0){\rule[-0.200pt]{2.409pt}{0.400pt}}
\put(1429.0,728.0){\rule[-0.200pt]{2.409pt}{0.400pt}}
\put(170.0,729.0){\rule[-0.200pt]{2.409pt}{0.400pt}}
\put(1429.0,729.0){\rule[-0.200pt]{2.409pt}{0.400pt}}
\put(170.0,730.0){\rule[-0.200pt]{2.409pt}{0.400pt}}
\put(1429.0,730.0){\rule[-0.200pt]{2.409pt}{0.400pt}}
\put(170.0,731.0){\rule[-0.200pt]{2.409pt}{0.400pt}}
\put(1429.0,731.0){\rule[-0.200pt]{2.409pt}{0.400pt}}
\put(170.0,732.0){\rule[-0.200pt]{2.409pt}{0.400pt}}
\put(1429.0,732.0){\rule[-0.200pt]{2.409pt}{0.400pt}}
\put(170.0,733.0){\rule[-0.200pt]{2.409pt}{0.400pt}}
\put(1429.0,733.0){\rule[-0.200pt]{2.409pt}{0.400pt}}
\put(170.0,734.0){\rule[-0.200pt]{2.409pt}{0.400pt}}
\put(1429.0,734.0){\rule[-0.200pt]{2.409pt}{0.400pt}}
\put(170.0,735.0){\rule[-0.200pt]{2.409pt}{0.400pt}}
\put(1429.0,735.0){\rule[-0.200pt]{2.409pt}{0.400pt}}
\put(170.0,736.0){\rule[-0.200pt]{2.409pt}{0.400pt}}
\put(1429.0,736.0){\rule[-0.200pt]{2.409pt}{0.400pt}}
\put(170.0,737.0){\rule[-0.200pt]{2.409pt}{0.400pt}}
\put(1429.0,737.0){\rule[-0.200pt]{2.409pt}{0.400pt}}
\put(170.0,738.0){\rule[-0.200pt]{2.409pt}{0.400pt}}
\put(1429.0,738.0){\rule[-0.200pt]{2.409pt}{0.400pt}}
\put(170.0,739.0){\rule[-0.200pt]{2.409pt}{0.400pt}}
\put(1429.0,739.0){\rule[-0.200pt]{2.409pt}{0.400pt}}
\put(170.0,740.0){\rule[-0.200pt]{2.409pt}{0.400pt}}
\put(1429.0,740.0){\rule[-0.200pt]{2.409pt}{0.400pt}}
\put(170.0,741.0){\rule[-0.200pt]{2.409pt}{0.400pt}}
\put(1429.0,741.0){\rule[-0.200pt]{2.409pt}{0.400pt}}
\put(170.0,742.0){\rule[-0.200pt]{2.409pt}{0.400pt}}
\put(1429.0,742.0){\rule[-0.200pt]{2.409pt}{0.400pt}}
\put(170.0,743.0){\rule[-0.200pt]{2.409pt}{0.400pt}}
\put(1429.0,743.0){\rule[-0.200pt]{2.409pt}{0.400pt}}
\put(170.0,744.0){\rule[-0.200pt]{2.409pt}{0.400pt}}
\put(1429.0,744.0){\rule[-0.200pt]{2.409pt}{0.400pt}}
\put(170.0,744.0){\rule[-0.200pt]{2.409pt}{0.400pt}}
\put(1429.0,744.0){\rule[-0.200pt]{2.409pt}{0.400pt}}
\put(170.0,745.0){\rule[-0.200pt]{2.409pt}{0.400pt}}
\put(1429.0,745.0){\rule[-0.200pt]{2.409pt}{0.400pt}}
\put(170.0,746.0){\rule[-0.200pt]{2.409pt}{0.400pt}}
\put(1429.0,746.0){\rule[-0.200pt]{2.409pt}{0.400pt}}
\put(170.0,747.0){\rule[-0.200pt]{2.409pt}{0.400pt}}
\put(1429.0,747.0){\rule[-0.200pt]{2.409pt}{0.400pt}}
\put(170.0,747.0){\rule[-0.200pt]{2.409pt}{0.400pt}}
\put(1429.0,747.0){\rule[-0.200pt]{2.409pt}{0.400pt}}
\put(170.0,748.0){\rule[-0.200pt]{2.409pt}{0.400pt}}
\put(1429.0,748.0){\rule[-0.200pt]{2.409pt}{0.400pt}}
\put(170.0,749.0){\rule[-0.200pt]{2.409pt}{0.400pt}}
\put(1429.0,749.0){\rule[-0.200pt]{2.409pt}{0.400pt}}
\put(170.0,750.0){\rule[-0.200pt]{2.409pt}{0.400pt}}
\put(1429.0,750.0){\rule[-0.200pt]{2.409pt}{0.400pt}}
\put(170.0,750.0){\rule[-0.200pt]{2.409pt}{0.400pt}}
\put(1429.0,750.0){\rule[-0.200pt]{2.409pt}{0.400pt}}
\put(170.0,751.0){\rule[-0.200pt]{2.409pt}{0.400pt}}
\put(1429.0,751.0){\rule[-0.200pt]{2.409pt}{0.400pt}}
\put(170.0,752.0){\rule[-0.200pt]{2.409pt}{0.400pt}}
\put(1429.0,752.0){\rule[-0.200pt]{2.409pt}{0.400pt}}
\put(170.0,752.0){\rule[-0.200pt]{2.409pt}{0.400pt}}
\put(1429.0,752.0){\rule[-0.200pt]{2.409pt}{0.400pt}}
\put(170.0,753.0){\rule[-0.200pt]{2.409pt}{0.400pt}}
\put(1429.0,753.0){\rule[-0.200pt]{2.409pt}{0.400pt}}
\put(170.0,754.0){\rule[-0.200pt]{2.409pt}{0.400pt}}
\put(1429.0,754.0){\rule[-0.200pt]{2.409pt}{0.400pt}}
\put(170.0,754.0){\rule[-0.200pt]{2.409pt}{0.400pt}}
\put(1429.0,754.0){\rule[-0.200pt]{2.409pt}{0.400pt}}
\put(170.0,755.0){\rule[-0.200pt]{2.409pt}{0.400pt}}
\put(1429.0,755.0){\rule[-0.200pt]{2.409pt}{0.400pt}}
\put(170.0,755.0){\rule[-0.200pt]{2.409pt}{0.400pt}}
\put(1429.0,755.0){\rule[-0.200pt]{2.409pt}{0.400pt}}
\put(170.0,756.0){\rule[-0.200pt]{2.409pt}{0.400pt}}
\put(1429.0,756.0){\rule[-0.200pt]{2.409pt}{0.400pt}}
\put(170.0,757.0){\rule[-0.200pt]{2.409pt}{0.400pt}}
\put(1429.0,757.0){\rule[-0.200pt]{2.409pt}{0.400pt}}
\put(170.0,757.0){\rule[-0.200pt]{2.409pt}{0.400pt}}
\put(1429.0,757.0){\rule[-0.200pt]{2.409pt}{0.400pt}}
\put(170.0,758.0){\rule[-0.200pt]{2.409pt}{0.400pt}}
\put(1429.0,758.0){\rule[-0.200pt]{2.409pt}{0.400pt}}
\put(170.0,758.0){\rule[-0.200pt]{2.409pt}{0.400pt}}
\put(1429.0,758.0){\rule[-0.200pt]{2.409pt}{0.400pt}}
\put(170.0,759.0){\rule[-0.200pt]{2.409pt}{0.400pt}}
\put(1429.0,759.0){\rule[-0.200pt]{2.409pt}{0.400pt}}
\put(170.0,759.0){\rule[-0.200pt]{2.409pt}{0.400pt}}
\put(1429.0,759.0){\rule[-0.200pt]{2.409pt}{0.400pt}}
\put(170.0,760.0){\rule[-0.200pt]{2.409pt}{0.400pt}}
\put(1429.0,760.0){\rule[-0.200pt]{2.409pt}{0.400pt}}
\put(170.0,760.0){\rule[-0.200pt]{2.409pt}{0.400pt}}
\put(1429.0,760.0){\rule[-0.200pt]{2.409pt}{0.400pt}}
\put(170.0,761.0){\rule[-0.200pt]{2.409pt}{0.400pt}}
\put(1429.0,761.0){\rule[-0.200pt]{2.409pt}{0.400pt}}
\put(170.0,761.0){\rule[-0.200pt]{2.409pt}{0.400pt}}
\put(1429.0,761.0){\rule[-0.200pt]{2.409pt}{0.400pt}}
\put(170.0,762.0){\rule[-0.200pt]{2.409pt}{0.400pt}}
\put(1429.0,762.0){\rule[-0.200pt]{2.409pt}{0.400pt}}
\put(170.0,762.0){\rule[-0.200pt]{2.409pt}{0.400pt}}
\put(1429.0,762.0){\rule[-0.200pt]{2.409pt}{0.400pt}}
\put(170.0,763.0){\rule[-0.200pt]{2.409pt}{0.400pt}}
\put(1429.0,763.0){\rule[-0.200pt]{2.409pt}{0.400pt}}
\put(170.0,763.0){\rule[-0.200pt]{2.409pt}{0.400pt}}
\put(1429.0,763.0){\rule[-0.200pt]{2.409pt}{0.400pt}}
\put(170.0,764.0){\rule[-0.200pt]{2.409pt}{0.400pt}}
\put(1429.0,764.0){\rule[-0.200pt]{2.409pt}{0.400pt}}
\put(170.0,764.0){\rule[-0.200pt]{2.409pt}{0.400pt}}
\put(1429.0,764.0){\rule[-0.200pt]{2.409pt}{0.400pt}}
\put(170.0,765.0){\rule[-0.200pt]{2.409pt}{0.400pt}}
\put(1429.0,765.0){\rule[-0.200pt]{2.409pt}{0.400pt}}
\put(170.0,765.0){\rule[-0.200pt]{2.409pt}{0.400pt}}
\put(1429.0,765.0){\rule[-0.200pt]{2.409pt}{0.400pt}}
\put(170.0,766.0){\rule[-0.200pt]{2.409pt}{0.400pt}}
\put(1429.0,766.0){\rule[-0.200pt]{2.409pt}{0.400pt}}
\put(170.0,766.0){\rule[-0.200pt]{2.409pt}{0.400pt}}
\put(1429.0,766.0){\rule[-0.200pt]{2.409pt}{0.400pt}}
\put(170.0,767.0){\rule[-0.200pt]{2.409pt}{0.400pt}}
\put(1429.0,767.0){\rule[-0.200pt]{2.409pt}{0.400pt}}
\put(170.0,767.0){\rule[-0.200pt]{2.409pt}{0.400pt}}
\put(1429.0,767.0){\rule[-0.200pt]{2.409pt}{0.400pt}}
\put(170.0,767.0){\rule[-0.200pt]{2.409pt}{0.400pt}}
\put(1429.0,767.0){\rule[-0.200pt]{2.409pt}{0.400pt}}
\put(170.0,768.0){\rule[-0.200pt]{2.409pt}{0.400pt}}
\put(1429.0,768.0){\rule[-0.200pt]{2.409pt}{0.400pt}}
\put(170.0,768.0){\rule[-0.200pt]{2.409pt}{0.400pt}}
\put(1429.0,768.0){\rule[-0.200pt]{2.409pt}{0.400pt}}
\put(170.0,769.0){\rule[-0.200pt]{2.409pt}{0.400pt}}
\put(1429.0,769.0){\rule[-0.200pt]{2.409pt}{0.400pt}}
\put(170.0,769.0){\rule[-0.200pt]{2.409pt}{0.400pt}}
\put(1429.0,769.0){\rule[-0.200pt]{2.409pt}{0.400pt}}
\put(170.0,770.0){\rule[-0.200pt]{2.409pt}{0.400pt}}
\put(1429.0,770.0){\rule[-0.200pt]{2.409pt}{0.400pt}}
\put(170.0,770.0){\rule[-0.200pt]{2.409pt}{0.400pt}}
\put(1429.0,770.0){\rule[-0.200pt]{2.409pt}{0.400pt}}
\put(170.0,770.0){\rule[-0.200pt]{2.409pt}{0.400pt}}
\put(1429.0,770.0){\rule[-0.200pt]{2.409pt}{0.400pt}}
\put(170.0,771.0){\rule[-0.200pt]{2.409pt}{0.400pt}}
\put(1429.0,771.0){\rule[-0.200pt]{2.409pt}{0.400pt}}
\put(170.0,771.0){\rule[-0.200pt]{2.409pt}{0.400pt}}
\put(1429.0,771.0){\rule[-0.200pt]{2.409pt}{0.400pt}}
\put(170.0,772.0){\rule[-0.200pt]{2.409pt}{0.400pt}}
\put(1429.0,772.0){\rule[-0.200pt]{2.409pt}{0.400pt}}
\put(170.0,772.0){\rule[-0.200pt]{2.409pt}{0.400pt}}
\put(1429.0,772.0){\rule[-0.200pt]{2.409pt}{0.400pt}}
\put(170.0,772.0){\rule[-0.200pt]{2.409pt}{0.400pt}}
\put(1429.0,772.0){\rule[-0.200pt]{2.409pt}{0.400pt}}
\put(170.0,773.0){\rule[-0.200pt]{2.409pt}{0.400pt}}
\put(1429.0,773.0){\rule[-0.200pt]{2.409pt}{0.400pt}}
\put(170.0,773.0){\rule[-0.200pt]{2.409pt}{0.400pt}}
\put(1429.0,773.0){\rule[-0.200pt]{2.409pt}{0.400pt}}
\put(170.0,773.0){\rule[-0.200pt]{2.409pt}{0.400pt}}
\put(1429.0,773.0){\rule[-0.200pt]{2.409pt}{0.400pt}}
\put(170.0,774.0){\rule[-0.200pt]{2.409pt}{0.400pt}}
\put(1429.0,774.0){\rule[-0.200pt]{2.409pt}{0.400pt}}
\put(170.0,774.0){\rule[-0.200pt]{2.409pt}{0.400pt}}
\put(1429.0,774.0){\rule[-0.200pt]{2.409pt}{0.400pt}}
\put(170.0,774.0){\rule[-0.200pt]{2.409pt}{0.400pt}}
\put(1429.0,774.0){\rule[-0.200pt]{2.409pt}{0.400pt}}
\put(170.0,775.0){\rule[-0.200pt]{2.409pt}{0.400pt}}
\put(1429.0,775.0){\rule[-0.200pt]{2.409pt}{0.400pt}}
\put(170.0,775.0){\rule[-0.200pt]{2.409pt}{0.400pt}}
\put(1429.0,775.0){\rule[-0.200pt]{2.409pt}{0.400pt}}
\put(170.0,776.0){\rule[-0.200pt]{2.409pt}{0.400pt}}
\put(1429.0,776.0){\rule[-0.200pt]{2.409pt}{0.400pt}}
\put(170.0,776.0){\rule[-0.200pt]{2.409pt}{0.400pt}}
\put(1429.0,776.0){\rule[-0.200pt]{2.409pt}{0.400pt}}
\put(170.0,776.0){\rule[-0.200pt]{2.409pt}{0.400pt}}
\put(1429.0,776.0){\rule[-0.200pt]{2.409pt}{0.400pt}}
\put(170.0,777.0){\rule[-0.200pt]{2.409pt}{0.400pt}}
\put(1429.0,777.0){\rule[-0.200pt]{2.409pt}{0.400pt}}
\put(170.0,777.0){\rule[-0.200pt]{2.409pt}{0.400pt}}
\put(1429.0,777.0){\rule[-0.200pt]{2.409pt}{0.400pt}}
\put(170.0,777.0){\rule[-0.200pt]{2.409pt}{0.400pt}}
\put(1429.0,777.0){\rule[-0.200pt]{2.409pt}{0.400pt}}
\put(170.0,778.0){\rule[-0.200pt]{2.409pt}{0.400pt}}
\put(1429.0,778.0){\rule[-0.200pt]{2.409pt}{0.400pt}}
\put(170.0,778.0){\rule[-0.200pt]{2.409pt}{0.400pt}}
\put(1429.0,778.0){\rule[-0.200pt]{2.409pt}{0.400pt}}
\put(170.0,778.0){\rule[-0.200pt]{2.409pt}{0.400pt}}
\put(1429.0,778.0){\rule[-0.200pt]{2.409pt}{0.400pt}}
\put(170.0,779.0){\rule[-0.200pt]{2.409pt}{0.400pt}}
\put(1429.0,779.0){\rule[-0.200pt]{2.409pt}{0.400pt}}
\put(170.0,779.0){\rule[-0.200pt]{2.409pt}{0.400pt}}
\put(1429.0,779.0){\rule[-0.200pt]{2.409pt}{0.400pt}}
\put(170.0,779.0){\rule[-0.200pt]{2.409pt}{0.400pt}}
\put(1429.0,779.0){\rule[-0.200pt]{2.409pt}{0.400pt}}
\put(170.0,780.0){\rule[-0.200pt]{2.409pt}{0.400pt}}
\put(1429.0,780.0){\rule[-0.200pt]{2.409pt}{0.400pt}}
\put(170.0,780.0){\rule[-0.200pt]{2.409pt}{0.400pt}}
\put(1429.0,780.0){\rule[-0.200pt]{2.409pt}{0.400pt}}
\put(170.0,780.0){\rule[-0.200pt]{2.409pt}{0.400pt}}
\put(1429.0,780.0){\rule[-0.200pt]{2.409pt}{0.400pt}}
\put(170.0,780.0){\rule[-0.200pt]{2.409pt}{0.400pt}}
\put(1429.0,780.0){\rule[-0.200pt]{2.409pt}{0.400pt}}
\put(170.0,781.0){\rule[-0.200pt]{2.409pt}{0.400pt}}
\put(1429.0,781.0){\rule[-0.200pt]{2.409pt}{0.400pt}}
\put(170.0,781.0){\rule[-0.200pt]{2.409pt}{0.400pt}}
\put(1429.0,781.0){\rule[-0.200pt]{2.409pt}{0.400pt}}
\put(170.0,781.0){\rule[-0.200pt]{2.409pt}{0.400pt}}
\put(1429.0,781.0){\rule[-0.200pt]{2.409pt}{0.400pt}}
\put(170.0,782.0){\rule[-0.200pt]{2.409pt}{0.400pt}}
\put(1429.0,782.0){\rule[-0.200pt]{2.409pt}{0.400pt}}
\put(170.0,782.0){\rule[-0.200pt]{2.409pt}{0.400pt}}
\put(1429.0,782.0){\rule[-0.200pt]{2.409pt}{0.400pt}}
\put(170.0,782.0){\rule[-0.200pt]{2.409pt}{0.400pt}}
\put(1429.0,782.0){\rule[-0.200pt]{2.409pt}{0.400pt}}
\put(170.0,783.0){\rule[-0.200pt]{2.409pt}{0.400pt}}
\put(1429.0,783.0){\rule[-0.200pt]{2.409pt}{0.400pt}}
\put(170.0,783.0){\rule[-0.200pt]{2.409pt}{0.400pt}}
\put(1429.0,783.0){\rule[-0.200pt]{2.409pt}{0.400pt}}
\put(170.0,783.0){\rule[-0.200pt]{2.409pt}{0.400pt}}
\put(1429.0,783.0){\rule[-0.200pt]{2.409pt}{0.400pt}}
\put(170.0,783.0){\rule[-0.200pt]{2.409pt}{0.400pt}}
\put(1429.0,783.0){\rule[-0.200pt]{2.409pt}{0.400pt}}
\put(170.0,784.0){\rule[-0.200pt]{2.409pt}{0.400pt}}
\put(1429.0,784.0){\rule[-0.200pt]{2.409pt}{0.400pt}}
\put(170.0,784.0){\rule[-0.200pt]{2.409pt}{0.400pt}}
\put(1429.0,784.0){\rule[-0.200pt]{2.409pt}{0.400pt}}
\put(170.0,784.0){\rule[-0.200pt]{2.409pt}{0.400pt}}
\put(1429.0,784.0){\rule[-0.200pt]{2.409pt}{0.400pt}}
\put(170.0,784.0){\rule[-0.200pt]{2.409pt}{0.400pt}}
\put(1429.0,784.0){\rule[-0.200pt]{2.409pt}{0.400pt}}
\put(170.0,785.0){\rule[-0.200pt]{2.409pt}{0.400pt}}
\put(1429.0,785.0){\rule[-0.200pt]{2.409pt}{0.400pt}}
\put(170.0,785.0){\rule[-0.200pt]{2.409pt}{0.400pt}}
\put(1429.0,785.0){\rule[-0.200pt]{2.409pt}{0.400pt}}
\put(170.0,785.0){\rule[-0.200pt]{2.409pt}{0.400pt}}
\put(1429.0,785.0){\rule[-0.200pt]{2.409pt}{0.400pt}}
\put(170.0,786.0){\rule[-0.200pt]{2.409pt}{0.400pt}}
\put(1429.0,786.0){\rule[-0.200pt]{2.409pt}{0.400pt}}
\put(170.0,786.0){\rule[-0.200pt]{2.409pt}{0.400pt}}
\put(1429.0,786.0){\rule[-0.200pt]{2.409pt}{0.400pt}}
\put(170.0,786.0){\rule[-0.200pt]{2.409pt}{0.400pt}}
\put(1429.0,786.0){\rule[-0.200pt]{2.409pt}{0.400pt}}
\put(170.0,786.0){\rule[-0.200pt]{2.409pt}{0.400pt}}
\put(1429.0,786.0){\rule[-0.200pt]{2.409pt}{0.400pt}}
\put(170.0,787.0){\rule[-0.200pt]{2.409pt}{0.400pt}}
\put(1429.0,787.0){\rule[-0.200pt]{2.409pt}{0.400pt}}
\put(170.0,787.0){\rule[-0.200pt]{2.409pt}{0.400pt}}
\put(1429.0,787.0){\rule[-0.200pt]{2.409pt}{0.400pt}}
\put(170.0,787.0){\rule[-0.200pt]{2.409pt}{0.400pt}}
\put(1429.0,787.0){\rule[-0.200pt]{2.409pt}{0.400pt}}
\put(170.0,787.0){\rule[-0.200pt]{2.409pt}{0.400pt}}
\put(1429.0,787.0){\rule[-0.200pt]{2.409pt}{0.400pt}}
\put(170.0,788.0){\rule[-0.200pt]{2.409pt}{0.400pt}}
\put(1429.0,788.0){\rule[-0.200pt]{2.409pt}{0.400pt}}
\put(170.0,788.0){\rule[-0.200pt]{2.409pt}{0.400pt}}
\put(1429.0,788.0){\rule[-0.200pt]{2.409pt}{0.400pt}}
\put(170.0,788.0){\rule[-0.200pt]{2.409pt}{0.400pt}}
\put(1429.0,788.0){\rule[-0.200pt]{2.409pt}{0.400pt}}
\put(170.0,788.0){\rule[-0.200pt]{2.409pt}{0.400pt}}
\put(1429.0,788.0){\rule[-0.200pt]{2.409pt}{0.400pt}}
\put(170.0,789.0){\rule[-0.200pt]{2.409pt}{0.400pt}}
\put(1429.0,789.0){\rule[-0.200pt]{2.409pt}{0.400pt}}
\put(170.0,789.0){\rule[-0.200pt]{2.409pt}{0.400pt}}
\put(1429.0,789.0){\rule[-0.200pt]{2.409pt}{0.400pt}}
\put(170.0,789.0){\rule[-0.200pt]{2.409pt}{0.400pt}}
\put(1429.0,789.0){\rule[-0.200pt]{2.409pt}{0.400pt}}
\put(170.0,789.0){\rule[-0.200pt]{2.409pt}{0.400pt}}
\put(1429.0,789.0){\rule[-0.200pt]{2.409pt}{0.400pt}}
\put(170.0,790.0){\rule[-0.200pt]{2.409pt}{0.400pt}}
\put(1429.0,790.0){\rule[-0.200pt]{2.409pt}{0.400pt}}
\put(170.0,790.0){\rule[-0.200pt]{2.409pt}{0.400pt}}
\put(1429.0,790.0){\rule[-0.200pt]{2.409pt}{0.400pt}}
\put(170.0,790.0){\rule[-0.200pt]{2.409pt}{0.400pt}}
\put(1429.0,790.0){\rule[-0.200pt]{2.409pt}{0.400pt}}
\put(170.0,790.0){\rule[-0.200pt]{2.409pt}{0.400pt}}
\put(1429.0,790.0){\rule[-0.200pt]{2.409pt}{0.400pt}}
\put(170.0,791.0){\rule[-0.200pt]{2.409pt}{0.400pt}}
\put(1429.0,791.0){\rule[-0.200pt]{2.409pt}{0.400pt}}
\put(170.0,791.0){\rule[-0.200pt]{2.409pt}{0.400pt}}
\put(1429.0,791.0){\rule[-0.200pt]{2.409pt}{0.400pt}}
\put(170.0,791.0){\rule[-0.200pt]{2.409pt}{0.400pt}}
\put(1429.0,791.0){\rule[-0.200pt]{2.409pt}{0.400pt}}
\put(170.0,791.0){\rule[-0.200pt]{2.409pt}{0.400pt}}
\put(1429.0,791.0){\rule[-0.200pt]{2.409pt}{0.400pt}}
\put(170.0,791.0){\rule[-0.200pt]{2.409pt}{0.400pt}}
\put(1429.0,791.0){\rule[-0.200pt]{2.409pt}{0.400pt}}
\put(170.0,792.0){\rule[-0.200pt]{2.409pt}{0.400pt}}
\put(1429.0,792.0){\rule[-0.200pt]{2.409pt}{0.400pt}}
\put(170.0,792.0){\rule[-0.200pt]{2.409pt}{0.400pt}}
\put(1429.0,792.0){\rule[-0.200pt]{2.409pt}{0.400pt}}
\put(170.0,792.0){\rule[-0.200pt]{2.409pt}{0.400pt}}
\put(1429.0,792.0){\rule[-0.200pt]{2.409pt}{0.400pt}}
\put(170.0,792.0){\rule[-0.200pt]{2.409pt}{0.400pt}}
\put(1429.0,792.0){\rule[-0.200pt]{2.409pt}{0.400pt}}
\put(170.0,793.0){\rule[-0.200pt]{2.409pt}{0.400pt}}
\put(1429.0,793.0){\rule[-0.200pt]{2.409pt}{0.400pt}}
\put(170.0,793.0){\rule[-0.200pt]{2.409pt}{0.400pt}}
\put(1429.0,793.0){\rule[-0.200pt]{2.409pt}{0.400pt}}
\put(170.0,793.0){\rule[-0.200pt]{2.409pt}{0.400pt}}
\put(1429.0,793.0){\rule[-0.200pt]{2.409pt}{0.400pt}}
\put(170.0,793.0){\rule[-0.200pt]{2.409pt}{0.400pt}}
\put(1429.0,793.0){\rule[-0.200pt]{2.409pt}{0.400pt}}
\put(170.0,793.0){\rule[-0.200pt]{2.409pt}{0.400pt}}
\put(1429.0,793.0){\rule[-0.200pt]{2.409pt}{0.400pt}}
\put(170.0,794.0){\rule[-0.200pt]{2.409pt}{0.400pt}}
\put(1429.0,794.0){\rule[-0.200pt]{2.409pt}{0.400pt}}
\put(170.0,794.0){\rule[-0.200pt]{2.409pt}{0.400pt}}
\put(1429.0,794.0){\rule[-0.200pt]{2.409pt}{0.400pt}}
\put(170.0,794.0){\rule[-0.200pt]{2.409pt}{0.400pt}}
\put(1429.0,794.0){\rule[-0.200pt]{2.409pt}{0.400pt}}
\put(170.0,794.0){\rule[-0.200pt]{2.409pt}{0.400pt}}
\put(1429.0,794.0){\rule[-0.200pt]{2.409pt}{0.400pt}}
\put(170.0,794.0){\rule[-0.200pt]{2.409pt}{0.400pt}}
\put(1429.0,794.0){\rule[-0.200pt]{2.409pt}{0.400pt}}
\put(170.0,795.0){\rule[-0.200pt]{2.409pt}{0.400pt}}
\put(1429.0,795.0){\rule[-0.200pt]{2.409pt}{0.400pt}}
\put(170.0,795.0){\rule[-0.200pt]{2.409pt}{0.400pt}}
\put(1429.0,795.0){\rule[-0.200pt]{2.409pt}{0.400pt}}
\put(170.0,795.0){\rule[-0.200pt]{2.409pt}{0.400pt}}
\put(1429.0,795.0){\rule[-0.200pt]{2.409pt}{0.400pt}}
\put(170.0,795.0){\rule[-0.200pt]{2.409pt}{0.400pt}}
\put(1429.0,795.0){\rule[-0.200pt]{2.409pt}{0.400pt}}
\put(170.0,796.0){\rule[-0.200pt]{2.409pt}{0.400pt}}
\put(1429.0,796.0){\rule[-0.200pt]{2.409pt}{0.400pt}}
\put(170.0,796.0){\rule[-0.200pt]{2.409pt}{0.400pt}}
\put(1429.0,796.0){\rule[-0.200pt]{2.409pt}{0.400pt}}
\put(170.0,796.0){\rule[-0.200pt]{2.409pt}{0.400pt}}
\put(1429.0,796.0){\rule[-0.200pt]{2.409pt}{0.400pt}}
\put(170.0,796.0){\rule[-0.200pt]{2.409pt}{0.400pt}}
\put(1429.0,796.0){\rule[-0.200pt]{2.409pt}{0.400pt}}
\put(170.0,796.0){\rule[-0.200pt]{2.409pt}{0.400pt}}
\put(1429.0,796.0){\rule[-0.200pt]{2.409pt}{0.400pt}}
\put(170.0,797.0){\rule[-0.200pt]{2.409pt}{0.400pt}}
\put(1429.0,797.0){\rule[-0.200pt]{2.409pt}{0.400pt}}
\put(170.0,797.0){\rule[-0.200pt]{2.409pt}{0.400pt}}
\put(1429.0,797.0){\rule[-0.200pt]{2.409pt}{0.400pt}}
\put(170.0,797.0){\rule[-0.200pt]{2.409pt}{0.400pt}}
\put(1429.0,797.0){\rule[-0.200pt]{2.409pt}{0.400pt}}
\put(170.0,797.0){\rule[-0.200pt]{2.409pt}{0.400pt}}
\put(1429.0,797.0){\rule[-0.200pt]{2.409pt}{0.400pt}}
\put(170.0,797.0){\rule[-0.200pt]{2.409pt}{0.400pt}}
\put(1429.0,797.0){\rule[-0.200pt]{2.409pt}{0.400pt}}
\put(170.0,798.0){\rule[-0.200pt]{2.409pt}{0.400pt}}
\put(1429.0,798.0){\rule[-0.200pt]{2.409pt}{0.400pt}}
\put(170.0,798.0){\rule[-0.200pt]{2.409pt}{0.400pt}}
\put(1429.0,798.0){\rule[-0.200pt]{2.409pt}{0.400pt}}
\put(170.0,798.0){\rule[-0.200pt]{2.409pt}{0.400pt}}
\put(1429.0,798.0){\rule[-0.200pt]{2.409pt}{0.400pt}}
\put(170.0,798.0){\rule[-0.200pt]{2.409pt}{0.400pt}}
\put(1429.0,798.0){\rule[-0.200pt]{2.409pt}{0.400pt}}
\put(170.0,798.0){\rule[-0.200pt]{2.409pt}{0.400pt}}
\put(1429.0,798.0){\rule[-0.200pt]{2.409pt}{0.400pt}}
\put(170.0,798.0){\rule[-0.200pt]{2.409pt}{0.400pt}}
\put(1429.0,798.0){\rule[-0.200pt]{2.409pt}{0.400pt}}
\put(170.0,799.0){\rule[-0.200pt]{2.409pt}{0.400pt}}
\put(1429.0,799.0){\rule[-0.200pt]{2.409pt}{0.400pt}}
\put(170.0,799.0){\rule[-0.200pt]{2.409pt}{0.400pt}}
\put(1429.0,799.0){\rule[-0.200pt]{2.409pt}{0.400pt}}
\put(170.0,799.0){\rule[-0.200pt]{2.409pt}{0.400pt}}
\put(1429.0,799.0){\rule[-0.200pt]{2.409pt}{0.400pt}}
\put(170.0,799.0){\rule[-0.200pt]{2.409pt}{0.400pt}}
\put(1429.0,799.0){\rule[-0.200pt]{2.409pt}{0.400pt}}
\put(170.0,799.0){\rule[-0.200pt]{2.409pt}{0.400pt}}
\put(1429.0,799.0){\rule[-0.200pt]{2.409pt}{0.400pt}}
\put(170.0,800.0){\rule[-0.200pt]{2.409pt}{0.400pt}}
\put(1429.0,800.0){\rule[-0.200pt]{2.409pt}{0.400pt}}
\put(170.0,800.0){\rule[-0.200pt]{2.409pt}{0.400pt}}
\put(1429.0,800.0){\rule[-0.200pt]{2.409pt}{0.400pt}}
\put(170.0,800.0){\rule[-0.200pt]{2.409pt}{0.400pt}}
\put(1429.0,800.0){\rule[-0.200pt]{2.409pt}{0.400pt}}
\put(170.0,800.0){\rule[-0.200pt]{2.409pt}{0.400pt}}
\put(1429.0,800.0){\rule[-0.200pt]{2.409pt}{0.400pt}}
\put(170.0,800.0){\rule[-0.200pt]{2.409pt}{0.400pt}}
\put(1429.0,800.0){\rule[-0.200pt]{2.409pt}{0.400pt}}
\put(170.0,800.0){\rule[-0.200pt]{2.409pt}{0.400pt}}
\put(1429.0,800.0){\rule[-0.200pt]{2.409pt}{0.400pt}}
\put(170.0,801.0){\rule[-0.200pt]{2.409pt}{0.400pt}}
\put(1429.0,801.0){\rule[-0.200pt]{2.409pt}{0.400pt}}
\put(170.0,801.0){\rule[-0.200pt]{2.409pt}{0.400pt}}
\put(1429.0,801.0){\rule[-0.200pt]{2.409pt}{0.400pt}}
\put(170.0,801.0){\rule[-0.200pt]{2.409pt}{0.400pt}}
\put(1429.0,801.0){\rule[-0.200pt]{2.409pt}{0.400pt}}
\put(170.0,801.0){\rule[-0.200pt]{2.409pt}{0.400pt}}
\put(1429.0,801.0){\rule[-0.200pt]{2.409pt}{0.400pt}}
\put(170.0,801.0){\rule[-0.200pt]{2.409pt}{0.400pt}}
\put(1429.0,801.0){\rule[-0.200pt]{2.409pt}{0.400pt}}
\put(170.0,802.0){\rule[-0.200pt]{2.409pt}{0.400pt}}
\put(1429.0,802.0){\rule[-0.200pt]{2.409pt}{0.400pt}}
\put(170.0,802.0){\rule[-0.200pt]{2.409pt}{0.400pt}}
\put(1429.0,802.0){\rule[-0.200pt]{2.409pt}{0.400pt}}
\put(170.0,802.0){\rule[-0.200pt]{2.409pt}{0.400pt}}
\put(1429.0,802.0){\rule[-0.200pt]{2.409pt}{0.400pt}}
\put(170.0,802.0){\rule[-0.200pt]{2.409pt}{0.400pt}}
\put(1429.0,802.0){\rule[-0.200pt]{2.409pt}{0.400pt}}
\put(170.0,802.0){\rule[-0.200pt]{2.409pt}{0.400pt}}
\put(1429.0,802.0){\rule[-0.200pt]{2.409pt}{0.400pt}}
\put(170.0,802.0){\rule[-0.200pt]{2.409pt}{0.400pt}}
\put(1429.0,802.0){\rule[-0.200pt]{2.409pt}{0.400pt}}
\put(170.0,803.0){\rule[-0.200pt]{2.409pt}{0.400pt}}
\put(1429.0,803.0){\rule[-0.200pt]{2.409pt}{0.400pt}}
\put(170.0,803.0){\rule[-0.200pt]{2.409pt}{0.400pt}}
\put(1429.0,803.0){\rule[-0.200pt]{2.409pt}{0.400pt}}
\put(170.0,803.0){\rule[-0.200pt]{2.409pt}{0.400pt}}
\put(1429.0,803.0){\rule[-0.200pt]{2.409pt}{0.400pt}}
\put(170.0,803.0){\rule[-0.200pt]{2.409pt}{0.400pt}}
\put(1429.0,803.0){\rule[-0.200pt]{2.409pt}{0.400pt}}
\put(170.0,803.0){\rule[-0.200pt]{2.409pt}{0.400pt}}
\put(1429.0,803.0){\rule[-0.200pt]{2.409pt}{0.400pt}}
\put(170.0,803.0){\rule[-0.200pt]{2.409pt}{0.400pt}}
\put(1429.0,803.0){\rule[-0.200pt]{2.409pt}{0.400pt}}
\put(170.0,804.0){\rule[-0.200pt]{2.409pt}{0.400pt}}
\put(1429.0,804.0){\rule[-0.200pt]{2.409pt}{0.400pt}}
\put(170.0,804.0){\rule[-0.200pt]{2.409pt}{0.400pt}}
\put(1429.0,804.0){\rule[-0.200pt]{2.409pt}{0.400pt}}
\put(170.0,804.0){\rule[-0.200pt]{2.409pt}{0.400pt}}
\put(1429.0,804.0){\rule[-0.200pt]{2.409pt}{0.400pt}}
\put(170.0,804.0){\rule[-0.200pt]{2.409pt}{0.400pt}}
\put(1429.0,804.0){\rule[-0.200pt]{2.409pt}{0.400pt}}
\put(170.0,804.0){\rule[-0.200pt]{2.409pt}{0.400pt}}
\put(1429.0,804.0){\rule[-0.200pt]{2.409pt}{0.400pt}}
\put(170.0,804.0){\rule[-0.200pt]{2.409pt}{0.400pt}}
\put(1429.0,804.0){\rule[-0.200pt]{2.409pt}{0.400pt}}
\put(170.0,805.0){\rule[-0.200pt]{2.409pt}{0.400pt}}
\put(1429.0,805.0){\rule[-0.200pt]{2.409pt}{0.400pt}}
\put(170.0,805.0){\rule[-0.200pt]{2.409pt}{0.400pt}}
\put(1429.0,805.0){\rule[-0.200pt]{2.409pt}{0.400pt}}
\put(170.0,805.0){\rule[-0.200pt]{2.409pt}{0.400pt}}
\put(1429.0,805.0){\rule[-0.200pt]{2.409pt}{0.400pt}}
\put(170.0,805.0){\rule[-0.200pt]{2.409pt}{0.400pt}}
\put(1429.0,805.0){\rule[-0.200pt]{2.409pt}{0.400pt}}
\put(170.0,805.0){\rule[-0.200pt]{2.409pt}{0.400pt}}
\put(1429.0,805.0){\rule[-0.200pt]{2.409pt}{0.400pt}}
\put(170.0,805.0){\rule[-0.200pt]{2.409pt}{0.400pt}}
\put(1429.0,805.0){\rule[-0.200pt]{2.409pt}{0.400pt}}
\put(170.0,805.0){\rule[-0.200pt]{2.409pt}{0.400pt}}
\put(1429.0,805.0){\rule[-0.200pt]{2.409pt}{0.400pt}}
\put(170.0,806.0){\rule[-0.200pt]{2.409pt}{0.400pt}}
\put(1429.0,806.0){\rule[-0.200pt]{2.409pt}{0.400pt}}
\put(170.0,806.0){\rule[-0.200pt]{2.409pt}{0.400pt}}
\put(1429.0,806.0){\rule[-0.200pt]{2.409pt}{0.400pt}}
\put(170.0,806.0){\rule[-0.200pt]{2.409pt}{0.400pt}}
\put(1429.0,806.0){\rule[-0.200pt]{2.409pt}{0.400pt}}
\put(170.0,806.0){\rule[-0.200pt]{2.409pt}{0.400pt}}
\put(1429.0,806.0){\rule[-0.200pt]{2.409pt}{0.400pt}}
\put(170.0,806.0){\rule[-0.200pt]{2.409pt}{0.400pt}}
\put(1429.0,806.0){\rule[-0.200pt]{2.409pt}{0.400pt}}
\put(170.0,806.0){\rule[-0.200pt]{2.409pt}{0.400pt}}
\put(1429.0,806.0){\rule[-0.200pt]{2.409pt}{0.400pt}}
\put(170.0,807.0){\rule[-0.200pt]{2.409pt}{0.400pt}}
\put(1429.0,807.0){\rule[-0.200pt]{2.409pt}{0.400pt}}
\put(170.0,807.0){\rule[-0.200pt]{2.409pt}{0.400pt}}
\put(1429.0,807.0){\rule[-0.200pt]{2.409pt}{0.400pt}}
\put(170.0,807.0){\rule[-0.200pt]{2.409pt}{0.400pt}}
\put(1429.0,807.0){\rule[-0.200pt]{2.409pt}{0.400pt}}
\put(170.0,807.0){\rule[-0.200pt]{2.409pt}{0.400pt}}
\put(1429.0,807.0){\rule[-0.200pt]{2.409pt}{0.400pt}}
\put(170.0,807.0){\rule[-0.200pt]{2.409pt}{0.400pt}}
\put(1429.0,807.0){\rule[-0.200pt]{2.409pt}{0.400pt}}
\put(170.0,807.0){\rule[-0.200pt]{2.409pt}{0.400pt}}
\put(1429.0,807.0){\rule[-0.200pt]{2.409pt}{0.400pt}}
\put(170.0,807.0){\rule[-0.200pt]{2.409pt}{0.400pt}}
\put(1429.0,807.0){\rule[-0.200pt]{2.409pt}{0.400pt}}
\put(170.0,808.0){\rule[-0.200pt]{2.409pt}{0.400pt}}
\put(1429.0,808.0){\rule[-0.200pt]{2.409pt}{0.400pt}}
\put(170.0,808.0){\rule[-0.200pt]{2.409pt}{0.400pt}}
\put(1429.0,808.0){\rule[-0.200pt]{2.409pt}{0.400pt}}
\put(170.0,808.0){\rule[-0.200pt]{2.409pt}{0.400pt}}
\put(1429.0,808.0){\rule[-0.200pt]{2.409pt}{0.400pt}}
\put(170.0,808.0){\rule[-0.200pt]{2.409pt}{0.400pt}}
\put(1429.0,808.0){\rule[-0.200pt]{2.409pt}{0.400pt}}
\put(170.0,808.0){\rule[-0.200pt]{2.409pt}{0.400pt}}
\put(1429.0,808.0){\rule[-0.200pt]{2.409pt}{0.400pt}}
\put(170.0,808.0){\rule[-0.200pt]{2.409pt}{0.400pt}}
\put(1429.0,808.0){\rule[-0.200pt]{2.409pt}{0.400pt}}
\put(170.0,808.0){\rule[-0.200pt]{2.409pt}{0.400pt}}
\put(1429.0,808.0){\rule[-0.200pt]{2.409pt}{0.400pt}}
\put(170.0,809.0){\rule[-0.200pt]{2.409pt}{0.400pt}}
\put(1429.0,809.0){\rule[-0.200pt]{2.409pt}{0.400pt}}
\put(170.0,809.0){\rule[-0.200pt]{2.409pt}{0.400pt}}
\put(1429.0,809.0){\rule[-0.200pt]{2.409pt}{0.400pt}}
\put(170.0,809.0){\rule[-0.200pt]{2.409pt}{0.400pt}}
\put(1429.0,809.0){\rule[-0.200pt]{2.409pt}{0.400pt}}
\put(170.0,809.0){\rule[-0.200pt]{2.409pt}{0.400pt}}
\put(1429.0,809.0){\rule[-0.200pt]{2.409pt}{0.400pt}}
\put(170.0,809.0){\rule[-0.200pt]{2.409pt}{0.400pt}}
\put(1429.0,809.0){\rule[-0.200pt]{2.409pt}{0.400pt}}
\put(170.0,809.0){\rule[-0.200pt]{2.409pt}{0.400pt}}
\put(1429.0,809.0){\rule[-0.200pt]{2.409pt}{0.400pt}}
\put(170.0,809.0){\rule[-0.200pt]{2.409pt}{0.400pt}}
\put(1429.0,809.0){\rule[-0.200pt]{2.409pt}{0.400pt}}
\put(170.0,810.0){\rule[-0.200pt]{2.409pt}{0.400pt}}
\put(1429.0,810.0){\rule[-0.200pt]{2.409pt}{0.400pt}}
\put(170.0,810.0){\rule[-0.200pt]{2.409pt}{0.400pt}}
\put(1429.0,810.0){\rule[-0.200pt]{2.409pt}{0.400pt}}
\put(170.0,810.0){\rule[-0.200pt]{2.409pt}{0.400pt}}
\put(1429.0,810.0){\rule[-0.200pt]{2.409pt}{0.400pt}}
\put(170.0,810.0){\rule[-0.200pt]{2.409pt}{0.400pt}}
\put(1429.0,810.0){\rule[-0.200pt]{2.409pt}{0.400pt}}
\put(170.0,810.0){\rule[-0.200pt]{2.409pt}{0.400pt}}
\put(1429.0,810.0){\rule[-0.200pt]{2.409pt}{0.400pt}}
\put(170.0,810.0){\rule[-0.200pt]{2.409pt}{0.400pt}}
\put(1429.0,810.0){\rule[-0.200pt]{2.409pt}{0.400pt}}
\put(170.0,810.0){\rule[-0.200pt]{2.409pt}{0.400pt}}
\put(1429.0,810.0){\rule[-0.200pt]{2.409pt}{0.400pt}}
\put(170.0,811.0){\rule[-0.200pt]{2.409pt}{0.400pt}}
\put(1429.0,811.0){\rule[-0.200pt]{2.409pt}{0.400pt}}
\put(170.0,811.0){\rule[-0.200pt]{2.409pt}{0.400pt}}
\put(1429.0,811.0){\rule[-0.200pt]{2.409pt}{0.400pt}}
\put(170.0,811.0){\rule[-0.200pt]{2.409pt}{0.400pt}}
\put(1429.0,811.0){\rule[-0.200pt]{2.409pt}{0.400pt}}
\put(170.0,811.0){\rule[-0.200pt]{2.409pt}{0.400pt}}
\put(1429.0,811.0){\rule[-0.200pt]{2.409pt}{0.400pt}}
\put(170.0,811.0){\rule[-0.200pt]{2.409pt}{0.400pt}}
\put(1429.0,811.0){\rule[-0.200pt]{2.409pt}{0.400pt}}
\put(170.0,811.0){\rule[-0.200pt]{2.409pt}{0.400pt}}
\put(1429.0,811.0){\rule[-0.200pt]{2.409pt}{0.400pt}}
\put(170.0,811.0){\rule[-0.200pt]{2.409pt}{0.400pt}}
\put(1429.0,811.0){\rule[-0.200pt]{2.409pt}{0.400pt}}
\put(170.0,812.0){\rule[-0.200pt]{2.409pt}{0.400pt}}
\put(1429.0,812.0){\rule[-0.200pt]{2.409pt}{0.400pt}}
\put(170.0,812.0){\rule[-0.200pt]{2.409pt}{0.400pt}}
\put(1429.0,812.0){\rule[-0.200pt]{2.409pt}{0.400pt}}
\put(170.0,812.0){\rule[-0.200pt]{2.409pt}{0.400pt}}
\put(1429.0,812.0){\rule[-0.200pt]{2.409pt}{0.400pt}}
\put(170.0,812.0){\rule[-0.200pt]{2.409pt}{0.400pt}}
\put(1429.0,812.0){\rule[-0.200pt]{2.409pt}{0.400pt}}
\put(170.0,812.0){\rule[-0.200pt]{2.409pt}{0.400pt}}
\put(1429.0,812.0){\rule[-0.200pt]{2.409pt}{0.400pt}}
\put(170.0,812.0){\rule[-0.200pt]{2.409pt}{0.400pt}}
\put(1429.0,812.0){\rule[-0.200pt]{2.409pt}{0.400pt}}
\put(170.0,812.0){\rule[-0.200pt]{2.409pt}{0.400pt}}
\put(1429.0,812.0){\rule[-0.200pt]{2.409pt}{0.400pt}}
\put(170.0,812.0){\rule[-0.200pt]{2.409pt}{0.400pt}}
\put(1429.0,812.0){\rule[-0.200pt]{2.409pt}{0.400pt}}
\put(170.0,813.0){\rule[-0.200pt]{2.409pt}{0.400pt}}
\put(1429.0,813.0){\rule[-0.200pt]{2.409pt}{0.400pt}}
\put(170.0,813.0){\rule[-0.200pt]{2.409pt}{0.400pt}}
\put(1429.0,813.0){\rule[-0.200pt]{2.409pt}{0.400pt}}
\put(170.0,813.0){\rule[-0.200pt]{2.409pt}{0.400pt}}
\put(1429.0,813.0){\rule[-0.200pt]{2.409pt}{0.400pt}}
\put(170.0,813.0){\rule[-0.200pt]{2.409pt}{0.400pt}}
\put(1429.0,813.0){\rule[-0.200pt]{2.409pt}{0.400pt}}
\put(170.0,813.0){\rule[-0.200pt]{2.409pt}{0.400pt}}
\put(1429.0,813.0){\rule[-0.200pt]{2.409pt}{0.400pt}}
\put(170.0,813.0){\rule[-0.200pt]{2.409pt}{0.400pt}}
\put(1429.0,813.0){\rule[-0.200pt]{2.409pt}{0.400pt}}
\put(170.0,813.0){\rule[-0.200pt]{2.409pt}{0.400pt}}
\put(1429.0,813.0){\rule[-0.200pt]{2.409pt}{0.400pt}}
\put(170.0,813.0){\rule[-0.200pt]{2.409pt}{0.400pt}}
\put(1429.0,813.0){\rule[-0.200pt]{2.409pt}{0.400pt}}
\put(170.0,814.0){\rule[-0.200pt]{2.409pt}{0.400pt}}
\put(1429.0,814.0){\rule[-0.200pt]{2.409pt}{0.400pt}}
\put(170.0,814.0){\rule[-0.200pt]{2.409pt}{0.400pt}}
\put(1429.0,814.0){\rule[-0.200pt]{2.409pt}{0.400pt}}
\put(170.0,814.0){\rule[-0.200pt]{2.409pt}{0.400pt}}
\put(1429.0,814.0){\rule[-0.200pt]{2.409pt}{0.400pt}}
\put(170.0,814.0){\rule[-0.200pt]{2.409pt}{0.400pt}}
\put(1429.0,814.0){\rule[-0.200pt]{2.409pt}{0.400pt}}
\put(170.0,814.0){\rule[-0.200pt]{2.409pt}{0.400pt}}
\put(1429.0,814.0){\rule[-0.200pt]{2.409pt}{0.400pt}}
\put(170.0,814.0){\rule[-0.200pt]{2.409pt}{0.400pt}}
\put(1429.0,814.0){\rule[-0.200pt]{2.409pt}{0.400pt}}
\put(170.0,814.0){\rule[-0.200pt]{2.409pt}{0.400pt}}
\put(1429.0,814.0){\rule[-0.200pt]{2.409pt}{0.400pt}}
\put(170.0,814.0){\rule[-0.200pt]{2.409pt}{0.400pt}}
\put(1429.0,814.0){\rule[-0.200pt]{2.409pt}{0.400pt}}
\put(170.0,815.0){\rule[-0.200pt]{2.409pt}{0.400pt}}
\put(1429.0,815.0){\rule[-0.200pt]{2.409pt}{0.400pt}}
\put(170.0,815.0){\rule[-0.200pt]{2.409pt}{0.400pt}}
\put(1429.0,815.0){\rule[-0.200pt]{2.409pt}{0.400pt}}
\put(170.0,815.0){\rule[-0.200pt]{2.409pt}{0.400pt}}
\put(1429.0,815.0){\rule[-0.200pt]{2.409pt}{0.400pt}}
\put(170.0,815.0){\rule[-0.200pt]{2.409pt}{0.400pt}}
\put(1429.0,815.0){\rule[-0.200pt]{2.409pt}{0.400pt}}
\put(170.0,815.0){\rule[-0.200pt]{2.409pt}{0.400pt}}
\put(1429.0,815.0){\rule[-0.200pt]{2.409pt}{0.400pt}}
\put(170.0,815.0){\rule[-0.200pt]{2.409pt}{0.400pt}}
\put(1429.0,815.0){\rule[-0.200pt]{2.409pt}{0.400pt}}
\put(170.0,815.0){\rule[-0.200pt]{2.409pt}{0.400pt}}
\put(1429.0,815.0){\rule[-0.200pt]{2.409pt}{0.400pt}}
\put(170.0,815.0){\rule[-0.200pt]{2.409pt}{0.400pt}}
\put(1429.0,815.0){\rule[-0.200pt]{2.409pt}{0.400pt}}
\put(170.0,816.0){\rule[-0.200pt]{2.409pt}{0.400pt}}
\put(1429.0,816.0){\rule[-0.200pt]{2.409pt}{0.400pt}}
\put(170.0,816.0){\rule[-0.200pt]{2.409pt}{0.400pt}}
\put(1429.0,816.0){\rule[-0.200pt]{2.409pt}{0.400pt}}
\put(170.0,816.0){\rule[-0.200pt]{2.409pt}{0.400pt}}
\put(1429.0,816.0){\rule[-0.200pt]{2.409pt}{0.400pt}}
\put(170.0,816.0){\rule[-0.200pt]{2.409pt}{0.400pt}}
\put(1429.0,816.0){\rule[-0.200pt]{2.409pt}{0.400pt}}
\put(170.0,816.0){\rule[-0.200pt]{2.409pt}{0.400pt}}
\put(1429.0,816.0){\rule[-0.200pt]{2.409pt}{0.400pt}}
\put(170.0,816.0){\rule[-0.200pt]{2.409pt}{0.400pt}}
\put(1429.0,816.0){\rule[-0.200pt]{2.409pt}{0.400pt}}
\put(170.0,816.0){\rule[-0.200pt]{2.409pt}{0.400pt}}
\put(1429.0,816.0){\rule[-0.200pt]{2.409pt}{0.400pt}}
\put(170.0,816.0){\rule[-0.200pt]{2.409pt}{0.400pt}}
\put(1429.0,816.0){\rule[-0.200pt]{2.409pt}{0.400pt}}
\put(170.0,817.0){\rule[-0.200pt]{2.409pt}{0.400pt}}
\put(1429.0,817.0){\rule[-0.200pt]{2.409pt}{0.400pt}}
\put(170.0,817.0){\rule[-0.200pt]{2.409pt}{0.400pt}}
\put(1429.0,817.0){\rule[-0.200pt]{2.409pt}{0.400pt}}
\put(170.0,817.0){\rule[-0.200pt]{2.409pt}{0.400pt}}
\put(1429.0,817.0){\rule[-0.200pt]{2.409pt}{0.400pt}}
\put(170.0,817.0){\rule[-0.200pt]{2.409pt}{0.400pt}}
\put(1429.0,817.0){\rule[-0.200pt]{2.409pt}{0.400pt}}
\put(170.0,817.0){\rule[-0.200pt]{2.409pt}{0.400pt}}
\put(1429.0,817.0){\rule[-0.200pt]{2.409pt}{0.400pt}}
\put(170.0,817.0){\rule[-0.200pt]{2.409pt}{0.400pt}}
\put(1429.0,817.0){\rule[-0.200pt]{2.409pt}{0.400pt}}
\put(170.0,817.0){\rule[-0.200pt]{2.409pt}{0.400pt}}
\put(1429.0,817.0){\rule[-0.200pt]{2.409pt}{0.400pt}}
\put(170.0,817.0){\rule[-0.200pt]{2.409pt}{0.400pt}}
\put(1429.0,817.0){\rule[-0.200pt]{2.409pt}{0.400pt}}
\put(170.0,817.0){\rule[-0.200pt]{2.409pt}{0.400pt}}
\put(1429.0,817.0){\rule[-0.200pt]{2.409pt}{0.400pt}}
\put(170.0,818.0){\rule[-0.200pt]{2.409pt}{0.400pt}}
\put(1429.0,818.0){\rule[-0.200pt]{2.409pt}{0.400pt}}
\put(170.0,818.0){\rule[-0.200pt]{2.409pt}{0.400pt}}
\put(1429.0,818.0){\rule[-0.200pt]{2.409pt}{0.400pt}}
\put(170.0,818.0){\rule[-0.200pt]{2.409pt}{0.400pt}}
\put(1429.0,818.0){\rule[-0.200pt]{2.409pt}{0.400pt}}
\put(170.0,818.0){\rule[-0.200pt]{2.409pt}{0.400pt}}
\put(1429.0,818.0){\rule[-0.200pt]{2.409pt}{0.400pt}}
\put(170.0,818.0){\rule[-0.200pt]{2.409pt}{0.400pt}}
\put(1429.0,818.0){\rule[-0.200pt]{2.409pt}{0.400pt}}
\put(170.0,818.0){\rule[-0.200pt]{2.409pt}{0.400pt}}
\put(1429.0,818.0){\rule[-0.200pt]{2.409pt}{0.400pt}}
\put(170.0,818.0){\rule[-0.200pt]{2.409pt}{0.400pt}}
\put(1429.0,818.0){\rule[-0.200pt]{2.409pt}{0.400pt}}
\put(170.0,818.0){\rule[-0.200pt]{2.409pt}{0.400pt}}
\put(1429.0,818.0){\rule[-0.200pt]{2.409pt}{0.400pt}}
\put(170.0,818.0){\rule[-0.200pt]{2.409pt}{0.400pt}}
\put(1429.0,818.0){\rule[-0.200pt]{2.409pt}{0.400pt}}
\put(170.0,819.0){\rule[-0.200pt]{2.409pt}{0.400pt}}
\put(1429.0,819.0){\rule[-0.200pt]{2.409pt}{0.400pt}}
\put(170.0,819.0){\rule[-0.200pt]{2.409pt}{0.400pt}}
\put(1429.0,819.0){\rule[-0.200pt]{2.409pt}{0.400pt}}
\put(170.0,819.0){\rule[-0.200pt]{2.409pt}{0.400pt}}
\put(1429.0,819.0){\rule[-0.200pt]{2.409pt}{0.400pt}}
\put(170.0,819.0){\rule[-0.200pt]{2.409pt}{0.400pt}}
\put(1429.0,819.0){\rule[-0.200pt]{2.409pt}{0.400pt}}
\put(170.0,819.0){\rule[-0.200pt]{2.409pt}{0.400pt}}
\put(1429.0,819.0){\rule[-0.200pt]{2.409pt}{0.400pt}}
\put(170.0,819.0){\rule[-0.200pt]{2.409pt}{0.400pt}}
\put(1429.0,819.0){\rule[-0.200pt]{2.409pt}{0.400pt}}
\put(170.0,819.0){\rule[-0.200pt]{2.409pt}{0.400pt}}
\put(1429.0,819.0){\rule[-0.200pt]{2.409pt}{0.400pt}}
\put(170.0,819.0){\rule[-0.200pt]{2.409pt}{0.400pt}}
\put(1429.0,819.0){\rule[-0.200pt]{2.409pt}{0.400pt}}
\put(170.0,819.0){\rule[-0.200pt]{2.409pt}{0.400pt}}
\put(1429.0,819.0){\rule[-0.200pt]{2.409pt}{0.400pt}}
\put(170.0,820.0){\rule[-0.200pt]{2.409pt}{0.400pt}}
\put(1429.0,820.0){\rule[-0.200pt]{2.409pt}{0.400pt}}
\put(170.0,820.0){\rule[-0.200pt]{2.409pt}{0.400pt}}
\put(1429.0,820.0){\rule[-0.200pt]{2.409pt}{0.400pt}}
\put(170.0,820.0){\rule[-0.200pt]{2.409pt}{0.400pt}}
\put(1429.0,820.0){\rule[-0.200pt]{2.409pt}{0.400pt}}
\put(170.0,820.0){\rule[-0.200pt]{2.409pt}{0.400pt}}
\put(1429.0,820.0){\rule[-0.200pt]{2.409pt}{0.400pt}}
\put(170.0,820.0){\rule[-0.200pt]{2.409pt}{0.400pt}}
\put(1429.0,820.0){\rule[-0.200pt]{2.409pt}{0.400pt}}
\put(170.0,820.0){\rule[-0.200pt]{2.409pt}{0.400pt}}
\put(1429.0,820.0){\rule[-0.200pt]{2.409pt}{0.400pt}}
\put(170.0,820.0){\rule[-0.200pt]{2.409pt}{0.400pt}}
\put(1429.0,820.0){\rule[-0.200pt]{2.409pt}{0.400pt}}
\put(170.0,820.0){\rule[-0.200pt]{2.409pt}{0.400pt}}
\put(1429.0,820.0){\rule[-0.200pt]{2.409pt}{0.400pt}}
\put(170.0,820.0){\rule[-0.200pt]{2.409pt}{0.400pt}}
\put(1429.0,820.0){\rule[-0.200pt]{2.409pt}{0.400pt}}
\put(170.0,820.0){\rule[-0.200pt]{2.409pt}{0.400pt}}
\put(1429.0,820.0){\rule[-0.200pt]{2.409pt}{0.400pt}}
\put(170.0,821.0){\rule[-0.200pt]{2.409pt}{0.400pt}}
\put(1429.0,821.0){\rule[-0.200pt]{2.409pt}{0.400pt}}
\put(170.0,821.0){\rule[-0.200pt]{2.409pt}{0.400pt}}
\put(1429.0,821.0){\rule[-0.200pt]{2.409pt}{0.400pt}}
\put(170.0,821.0){\rule[-0.200pt]{2.409pt}{0.400pt}}
\put(1429.0,821.0){\rule[-0.200pt]{2.409pt}{0.400pt}}
\put(170.0,821.0){\rule[-0.200pt]{2.409pt}{0.400pt}}
\put(1429.0,821.0){\rule[-0.200pt]{2.409pt}{0.400pt}}
\put(170.0,821.0){\rule[-0.200pt]{2.409pt}{0.400pt}}
\put(1429.0,821.0){\rule[-0.200pt]{2.409pt}{0.400pt}}
\put(170.0,821.0){\rule[-0.200pt]{2.409pt}{0.400pt}}
\put(1429.0,821.0){\rule[-0.200pt]{2.409pt}{0.400pt}}
\put(170.0,821.0){\rule[-0.200pt]{2.409pt}{0.400pt}}
\put(1429.0,821.0){\rule[-0.200pt]{2.409pt}{0.400pt}}
\put(170.0,821.0){\rule[-0.200pt]{2.409pt}{0.400pt}}
\put(1429.0,821.0){\rule[-0.200pt]{2.409pt}{0.400pt}}
\put(170.0,821.0){\rule[-0.200pt]{2.409pt}{0.400pt}}
\put(1429.0,821.0){\rule[-0.200pt]{2.409pt}{0.400pt}}
\put(170.0,822.0){\rule[-0.200pt]{2.409pt}{0.400pt}}
\put(1429.0,822.0){\rule[-0.200pt]{2.409pt}{0.400pt}}
\put(170.0,822.0){\rule[-0.200pt]{2.409pt}{0.400pt}}
\put(1429.0,822.0){\rule[-0.200pt]{2.409pt}{0.400pt}}
\put(170.0,822.0){\rule[-0.200pt]{2.409pt}{0.400pt}}
\put(1429.0,822.0){\rule[-0.200pt]{2.409pt}{0.400pt}}
\put(170.0,822.0){\rule[-0.200pt]{2.409pt}{0.400pt}}
\put(1429.0,822.0){\rule[-0.200pt]{2.409pt}{0.400pt}}
\put(170.0,822.0){\rule[-0.200pt]{2.409pt}{0.400pt}}
\put(1429.0,822.0){\rule[-0.200pt]{2.409pt}{0.400pt}}
\put(170.0,822.0){\rule[-0.200pt]{2.409pt}{0.400pt}}
\put(1429.0,822.0){\rule[-0.200pt]{2.409pt}{0.400pt}}
\put(170.0,822.0){\rule[-0.200pt]{2.409pt}{0.400pt}}
\put(1429.0,822.0){\rule[-0.200pt]{2.409pt}{0.400pt}}
\put(170.0,822.0){\rule[-0.200pt]{2.409pt}{0.400pt}}
\put(1429.0,822.0){\rule[-0.200pt]{2.409pt}{0.400pt}}
\put(170.0,822.0){\rule[-0.200pt]{2.409pt}{0.400pt}}
\put(1429.0,822.0){\rule[-0.200pt]{2.409pt}{0.400pt}}
\put(170.0,822.0){\rule[-0.200pt]{2.409pt}{0.400pt}}
\put(1429.0,822.0){\rule[-0.200pt]{2.409pt}{0.400pt}}
\put(170.0,823.0){\rule[-0.200pt]{2.409pt}{0.400pt}}
\put(1429.0,823.0){\rule[-0.200pt]{2.409pt}{0.400pt}}
\put(170.0,823.0){\rule[-0.200pt]{2.409pt}{0.400pt}}
\put(1429.0,823.0){\rule[-0.200pt]{2.409pt}{0.400pt}}
\put(170.0,823.0){\rule[-0.200pt]{2.409pt}{0.400pt}}
\put(1429.0,823.0){\rule[-0.200pt]{2.409pt}{0.400pt}}
\put(170.0,823.0){\rule[-0.200pt]{2.409pt}{0.400pt}}
\put(1429.0,823.0){\rule[-0.200pt]{2.409pt}{0.400pt}}
\put(170.0,823.0){\rule[-0.200pt]{2.409pt}{0.400pt}}
\put(1429.0,823.0){\rule[-0.200pt]{2.409pt}{0.400pt}}
\put(170.0,823.0){\rule[-0.200pt]{2.409pt}{0.400pt}}
\put(1429.0,823.0){\rule[-0.200pt]{2.409pt}{0.400pt}}
\put(170.0,823.0){\rule[-0.200pt]{2.409pt}{0.400pt}}
\put(1429.0,823.0){\rule[-0.200pt]{2.409pt}{0.400pt}}
\put(170.0,823.0){\rule[-0.200pt]{2.409pt}{0.400pt}}
\put(1429.0,823.0){\rule[-0.200pt]{2.409pt}{0.400pt}}
\put(170.0,823.0){\rule[-0.200pt]{2.409pt}{0.400pt}}
\put(1429.0,823.0){\rule[-0.200pt]{2.409pt}{0.400pt}}
\put(170.0,823.0){\rule[-0.200pt]{2.409pt}{0.400pt}}
\put(1429.0,823.0){\rule[-0.200pt]{2.409pt}{0.400pt}}
\put(170.0,824.0){\rule[-0.200pt]{2.409pt}{0.400pt}}
\put(1429.0,824.0){\rule[-0.200pt]{2.409pt}{0.400pt}}
\put(170.0,824.0){\rule[-0.200pt]{2.409pt}{0.400pt}}
\put(1429.0,824.0){\rule[-0.200pt]{2.409pt}{0.400pt}}
\put(170.0,824.0){\rule[-0.200pt]{2.409pt}{0.400pt}}
\put(1429.0,824.0){\rule[-0.200pt]{2.409pt}{0.400pt}}
\put(170.0,824.0){\rule[-0.200pt]{2.409pt}{0.400pt}}
\put(1429.0,824.0){\rule[-0.200pt]{2.409pt}{0.400pt}}
\put(170.0,824.0){\rule[-0.200pt]{2.409pt}{0.400pt}}
\put(1429.0,824.0){\rule[-0.200pt]{2.409pt}{0.400pt}}
\put(170.0,824.0){\rule[-0.200pt]{2.409pt}{0.400pt}}
\put(1429.0,824.0){\rule[-0.200pt]{2.409pt}{0.400pt}}
\put(170.0,824.0){\rule[-0.200pt]{2.409pt}{0.400pt}}
\put(1429.0,824.0){\rule[-0.200pt]{2.409pt}{0.400pt}}
\put(170.0,824.0){\rule[-0.200pt]{2.409pt}{0.400pt}}
\put(1429.0,824.0){\rule[-0.200pt]{2.409pt}{0.400pt}}
\put(170.0,824.0){\rule[-0.200pt]{2.409pt}{0.400pt}}
\put(1429.0,824.0){\rule[-0.200pt]{2.409pt}{0.400pt}}
\put(170.0,824.0){\rule[-0.200pt]{2.409pt}{0.400pt}}
\put(1429.0,824.0){\rule[-0.200pt]{2.409pt}{0.400pt}}
\put(170.0,824.0){\rule[-0.200pt]{2.409pt}{0.400pt}}
\put(1429.0,824.0){\rule[-0.200pt]{2.409pt}{0.400pt}}
\put(170.0,825.0){\rule[-0.200pt]{2.409pt}{0.400pt}}
\put(1429.0,825.0){\rule[-0.200pt]{2.409pt}{0.400pt}}
\put(170.0,825.0){\rule[-0.200pt]{2.409pt}{0.400pt}}
\put(1429.0,825.0){\rule[-0.200pt]{2.409pt}{0.400pt}}
\put(170.0,825.0){\rule[-0.200pt]{2.409pt}{0.400pt}}
\put(1429.0,825.0){\rule[-0.200pt]{2.409pt}{0.400pt}}
\put(170.0,825.0){\rule[-0.200pt]{2.409pt}{0.400pt}}
\put(1429.0,825.0){\rule[-0.200pt]{2.409pt}{0.400pt}}
\put(170.0,825.0){\rule[-0.200pt]{2.409pt}{0.400pt}}
\put(1429.0,825.0){\rule[-0.200pt]{2.409pt}{0.400pt}}
\put(170.0,825.0){\rule[-0.200pt]{2.409pt}{0.400pt}}
\put(1429.0,825.0){\rule[-0.200pt]{2.409pt}{0.400pt}}
\put(170.0,825.0){\rule[-0.200pt]{2.409pt}{0.400pt}}
\put(1429.0,825.0){\rule[-0.200pt]{2.409pt}{0.400pt}}
\put(170.0,825.0){\rule[-0.200pt]{2.409pt}{0.400pt}}
\put(1429.0,825.0){\rule[-0.200pt]{2.409pt}{0.400pt}}
\put(170.0,825.0){\rule[-0.200pt]{2.409pt}{0.400pt}}
\put(1429.0,825.0){\rule[-0.200pt]{2.409pt}{0.400pt}}
\put(170.0,825.0){\rule[-0.200pt]{2.409pt}{0.400pt}}
\put(1429.0,825.0){\rule[-0.200pt]{2.409pt}{0.400pt}}
\put(170.0,825.0){\rule[-0.200pt]{2.409pt}{0.400pt}}
\put(1429.0,825.0){\rule[-0.200pt]{2.409pt}{0.400pt}}
\put(170.0,826.0){\rule[-0.200pt]{2.409pt}{0.400pt}}
\put(1429.0,826.0){\rule[-0.200pt]{2.409pt}{0.400pt}}
\put(170.0,826.0){\rule[-0.200pt]{2.409pt}{0.400pt}}
\put(1429.0,826.0){\rule[-0.200pt]{2.409pt}{0.400pt}}
\put(170.0,826.0){\rule[-0.200pt]{2.409pt}{0.400pt}}
\put(1429.0,826.0){\rule[-0.200pt]{2.409pt}{0.400pt}}
\put(170.0,826.0){\rule[-0.200pt]{2.409pt}{0.400pt}}
\put(1429.0,826.0){\rule[-0.200pt]{2.409pt}{0.400pt}}
\put(170.0,826.0){\rule[-0.200pt]{2.409pt}{0.400pt}}
\put(1429.0,826.0){\rule[-0.200pt]{2.409pt}{0.400pt}}
\put(170.0,826.0){\rule[-0.200pt]{2.409pt}{0.400pt}}
\put(1429.0,826.0){\rule[-0.200pt]{2.409pt}{0.400pt}}
\put(170.0,826.0){\rule[-0.200pt]{2.409pt}{0.400pt}}
\put(1429.0,826.0){\rule[-0.200pt]{2.409pt}{0.400pt}}
\put(170.0,826.0){\rule[-0.200pt]{2.409pt}{0.400pt}}
\put(1429.0,826.0){\rule[-0.200pt]{2.409pt}{0.400pt}}
\put(170.0,826.0){\rule[-0.200pt]{2.409pt}{0.400pt}}
\put(1429.0,826.0){\rule[-0.200pt]{2.409pt}{0.400pt}}
\put(170.0,826.0){\rule[-0.200pt]{2.409pt}{0.400pt}}
\put(1429.0,826.0){\rule[-0.200pt]{2.409pt}{0.400pt}}
\put(170.0,826.0){\rule[-0.200pt]{2.409pt}{0.400pt}}
\put(1429.0,826.0){\rule[-0.200pt]{2.409pt}{0.400pt}}
\put(170.0,827.0){\rule[-0.200pt]{2.409pt}{0.400pt}}
\put(1429.0,827.0){\rule[-0.200pt]{2.409pt}{0.400pt}}
\put(170.0,827.0){\rule[-0.200pt]{2.409pt}{0.400pt}}
\put(1429.0,827.0){\rule[-0.200pt]{2.409pt}{0.400pt}}
\put(170.0,827.0){\rule[-0.200pt]{2.409pt}{0.400pt}}
\put(1429.0,827.0){\rule[-0.200pt]{2.409pt}{0.400pt}}
\put(170.0,827.0){\rule[-0.200pt]{2.409pt}{0.400pt}}
\put(1429.0,827.0){\rule[-0.200pt]{2.409pt}{0.400pt}}
\put(170.0,827.0){\rule[-0.200pt]{2.409pt}{0.400pt}}
\put(1429.0,827.0){\rule[-0.200pt]{2.409pt}{0.400pt}}
\put(170.0,827.0){\rule[-0.200pt]{2.409pt}{0.400pt}}
\put(1429.0,827.0){\rule[-0.200pt]{2.409pt}{0.400pt}}
\put(170.0,827.0){\rule[-0.200pt]{2.409pt}{0.400pt}}
\put(1429.0,827.0){\rule[-0.200pt]{2.409pt}{0.400pt}}
\put(170.0,827.0){\rule[-0.200pt]{2.409pt}{0.400pt}}
\put(1429.0,827.0){\rule[-0.200pt]{2.409pt}{0.400pt}}
\put(170.0,827.0){\rule[-0.200pt]{2.409pt}{0.400pt}}
\put(1429.0,827.0){\rule[-0.200pt]{2.409pt}{0.400pt}}
\put(170.0,827.0){\rule[-0.200pt]{2.409pt}{0.400pt}}
\put(1429.0,827.0){\rule[-0.200pt]{2.409pt}{0.400pt}}
\put(170.0,827.0){\rule[-0.200pt]{2.409pt}{0.400pt}}
\put(1429.0,827.0){\rule[-0.200pt]{2.409pt}{0.400pt}}
\put(170.0,828.0){\rule[-0.200pt]{2.409pt}{0.400pt}}
\put(1429.0,828.0){\rule[-0.200pt]{2.409pt}{0.400pt}}
\put(170.0,828.0){\rule[-0.200pt]{2.409pt}{0.400pt}}
\put(1429.0,828.0){\rule[-0.200pt]{2.409pt}{0.400pt}}
\put(170.0,828.0){\rule[-0.200pt]{2.409pt}{0.400pt}}
\put(1429.0,828.0){\rule[-0.200pt]{2.409pt}{0.400pt}}
\put(170.0,828.0){\rule[-0.200pt]{2.409pt}{0.400pt}}
\put(1429.0,828.0){\rule[-0.200pt]{2.409pt}{0.400pt}}
\put(170.0,828.0){\rule[-0.200pt]{2.409pt}{0.400pt}}
\put(1429.0,828.0){\rule[-0.200pt]{2.409pt}{0.400pt}}
\put(170.0,828.0){\rule[-0.200pt]{2.409pt}{0.400pt}}
\put(1429.0,828.0){\rule[-0.200pt]{2.409pt}{0.400pt}}
\put(170.0,828.0){\rule[-0.200pt]{2.409pt}{0.400pt}}
\put(1429.0,828.0){\rule[-0.200pt]{2.409pt}{0.400pt}}
\put(170.0,828.0){\rule[-0.200pt]{2.409pt}{0.400pt}}
\put(1429.0,828.0){\rule[-0.200pt]{2.409pt}{0.400pt}}
\put(170.0,828.0){\rule[-0.200pt]{2.409pt}{0.400pt}}
\put(1429.0,828.0){\rule[-0.200pt]{2.409pt}{0.400pt}}
\put(170.0,828.0){\rule[-0.200pt]{2.409pt}{0.400pt}}
\put(1429.0,828.0){\rule[-0.200pt]{2.409pt}{0.400pt}}
\put(170.0,828.0){\rule[-0.200pt]{2.409pt}{0.400pt}}
\put(1429.0,828.0){\rule[-0.200pt]{2.409pt}{0.400pt}}
\put(170.0,828.0){\rule[-0.200pt]{2.409pt}{0.400pt}}
\put(1429.0,828.0){\rule[-0.200pt]{2.409pt}{0.400pt}}
\put(170.0,829.0){\rule[-0.200pt]{2.409pt}{0.400pt}}
\put(1429.0,829.0){\rule[-0.200pt]{2.409pt}{0.400pt}}
\put(170.0,829.0){\rule[-0.200pt]{2.409pt}{0.400pt}}
\put(1429.0,829.0){\rule[-0.200pt]{2.409pt}{0.400pt}}
\put(170.0,829.0){\rule[-0.200pt]{2.409pt}{0.400pt}}
\put(1429.0,829.0){\rule[-0.200pt]{2.409pt}{0.400pt}}
\put(170.0,829.0){\rule[-0.200pt]{2.409pt}{0.400pt}}
\put(1429.0,829.0){\rule[-0.200pt]{2.409pt}{0.400pt}}
\put(170.0,829.0){\rule[-0.200pt]{2.409pt}{0.400pt}}
\put(1429.0,829.0){\rule[-0.200pt]{2.409pt}{0.400pt}}
\put(170.0,829.0){\rule[-0.200pt]{2.409pt}{0.400pt}}
\put(1429.0,829.0){\rule[-0.200pt]{2.409pt}{0.400pt}}
\put(170.0,829.0){\rule[-0.200pt]{2.409pt}{0.400pt}}
\put(1429.0,829.0){\rule[-0.200pt]{2.409pt}{0.400pt}}
\put(170.0,829.0){\rule[-0.200pt]{2.409pt}{0.400pt}}
\put(1429.0,829.0){\rule[-0.200pt]{2.409pt}{0.400pt}}
\put(170.0,829.0){\rule[-0.200pt]{2.409pt}{0.400pt}}
\put(1429.0,829.0){\rule[-0.200pt]{2.409pt}{0.400pt}}
\put(170.0,829.0){\rule[-0.200pt]{2.409pt}{0.400pt}}
\put(1429.0,829.0){\rule[-0.200pt]{2.409pt}{0.400pt}}
\put(170.0,829.0){\rule[-0.200pt]{2.409pt}{0.400pt}}
\put(1429.0,829.0){\rule[-0.200pt]{2.409pt}{0.400pt}}
\put(170.0,829.0){\rule[-0.200pt]{2.409pt}{0.400pt}}
\put(1429.0,829.0){\rule[-0.200pt]{2.409pt}{0.400pt}}
\put(170.0,830.0){\rule[-0.200pt]{2.409pt}{0.400pt}}
\put(1429.0,830.0){\rule[-0.200pt]{2.409pt}{0.400pt}}
\put(170.0,830.0){\rule[-0.200pt]{2.409pt}{0.400pt}}
\put(1429.0,830.0){\rule[-0.200pt]{2.409pt}{0.400pt}}
\put(170.0,830.0){\rule[-0.200pt]{2.409pt}{0.400pt}}
\put(1429.0,830.0){\rule[-0.200pt]{2.409pt}{0.400pt}}
\put(170.0,830.0){\rule[-0.200pt]{2.409pt}{0.400pt}}
\put(1429.0,830.0){\rule[-0.200pt]{2.409pt}{0.400pt}}
\put(170.0,830.0){\rule[-0.200pt]{2.409pt}{0.400pt}}
\put(1429.0,830.0){\rule[-0.200pt]{2.409pt}{0.400pt}}
\put(170.0,830.0){\rule[-0.200pt]{2.409pt}{0.400pt}}
\put(1429.0,830.0){\rule[-0.200pt]{2.409pt}{0.400pt}}
\put(170.0,830.0){\rule[-0.200pt]{2.409pt}{0.400pt}}
\put(1429.0,830.0){\rule[-0.200pt]{2.409pt}{0.400pt}}
\put(170.0,830.0){\rule[-0.200pt]{2.409pt}{0.400pt}}
\put(1429.0,830.0){\rule[-0.200pt]{2.409pt}{0.400pt}}
\put(170.0,830.0){\rule[-0.200pt]{2.409pt}{0.400pt}}
\put(1429.0,830.0){\rule[-0.200pt]{2.409pt}{0.400pt}}
\put(170.0,830.0){\rule[-0.200pt]{2.409pt}{0.400pt}}
\put(1429.0,830.0){\rule[-0.200pt]{2.409pt}{0.400pt}}
\put(170.0,830.0){\rule[-0.200pt]{2.409pt}{0.400pt}}
\put(1429.0,830.0){\rule[-0.200pt]{2.409pt}{0.400pt}}
\put(170.0,830.0){\rule[-0.200pt]{2.409pt}{0.400pt}}
\put(1429.0,830.0){\rule[-0.200pt]{2.409pt}{0.400pt}}
\put(170.0,831.0){\rule[-0.200pt]{2.409pt}{0.400pt}}
\put(1429.0,831.0){\rule[-0.200pt]{2.409pt}{0.400pt}}
\put(170.0,831.0){\rule[-0.200pt]{2.409pt}{0.400pt}}
\put(1429.0,831.0){\rule[-0.200pt]{2.409pt}{0.400pt}}
\put(170.0,831.0){\rule[-0.200pt]{2.409pt}{0.400pt}}
\put(1429.0,831.0){\rule[-0.200pt]{2.409pt}{0.400pt}}
\put(170.0,831.0){\rule[-0.200pt]{2.409pt}{0.400pt}}
\put(1429.0,831.0){\rule[-0.200pt]{2.409pt}{0.400pt}}
\put(170.0,831.0){\rule[-0.200pt]{2.409pt}{0.400pt}}
\put(1429.0,831.0){\rule[-0.200pt]{2.409pt}{0.400pt}}
\put(170.0,831.0){\rule[-0.200pt]{2.409pt}{0.400pt}}
\put(1429.0,831.0){\rule[-0.200pt]{2.409pt}{0.400pt}}
\put(170.0,831.0){\rule[-0.200pt]{2.409pt}{0.400pt}}
\put(1429.0,831.0){\rule[-0.200pt]{2.409pt}{0.400pt}}
\put(170.0,831.0){\rule[-0.200pt]{2.409pt}{0.400pt}}
\put(1429.0,831.0){\rule[-0.200pt]{2.409pt}{0.400pt}}
\put(170.0,831.0){\rule[-0.200pt]{2.409pt}{0.400pt}}
\put(1429.0,831.0){\rule[-0.200pt]{2.409pt}{0.400pt}}
\put(170.0,831.0){\rule[-0.200pt]{2.409pt}{0.400pt}}
\put(1429.0,831.0){\rule[-0.200pt]{2.409pt}{0.400pt}}
\put(170.0,831.0){\rule[-0.200pt]{2.409pt}{0.400pt}}
\put(1429.0,831.0){\rule[-0.200pt]{2.409pt}{0.400pt}}
\put(170.0,831.0){\rule[-0.200pt]{2.409pt}{0.400pt}}
\put(1429.0,831.0){\rule[-0.200pt]{2.409pt}{0.400pt}}
\put(170.0,831.0){\rule[-0.200pt]{2.409pt}{0.400pt}}
\put(1429.0,831.0){\rule[-0.200pt]{2.409pt}{0.400pt}}
\put(170.0,832.0){\rule[-0.200pt]{2.409pt}{0.400pt}}
\put(1429.0,832.0){\rule[-0.200pt]{2.409pt}{0.400pt}}
\put(170.0,832.0){\rule[-0.200pt]{2.409pt}{0.400pt}}
\put(1429.0,832.0){\rule[-0.200pt]{2.409pt}{0.400pt}}
\put(170.0,832.0){\rule[-0.200pt]{2.409pt}{0.400pt}}
\put(1429.0,832.0){\rule[-0.200pt]{2.409pt}{0.400pt}}
\put(170.0,832.0){\rule[-0.200pt]{2.409pt}{0.400pt}}
\put(1429.0,832.0){\rule[-0.200pt]{2.409pt}{0.400pt}}
\put(170.0,832.0){\rule[-0.200pt]{2.409pt}{0.400pt}}
\put(1429.0,832.0){\rule[-0.200pt]{2.409pt}{0.400pt}}
\put(170.0,832.0){\rule[-0.200pt]{2.409pt}{0.400pt}}
\put(1429.0,832.0){\rule[-0.200pt]{2.409pt}{0.400pt}}
\put(170.0,832.0){\rule[-0.200pt]{2.409pt}{0.400pt}}
\put(1429.0,832.0){\rule[-0.200pt]{2.409pt}{0.400pt}}
\put(170.0,832.0){\rule[-0.200pt]{2.409pt}{0.400pt}}
\put(1429.0,832.0){\rule[-0.200pt]{2.409pt}{0.400pt}}
\put(170.0,832.0){\rule[-0.200pt]{2.409pt}{0.400pt}}
\put(1429.0,832.0){\rule[-0.200pt]{2.409pt}{0.400pt}}
\put(170.0,832.0){\rule[-0.200pt]{2.409pt}{0.400pt}}
\put(1429.0,832.0){\rule[-0.200pt]{2.409pt}{0.400pt}}
\put(170.0,832.0){\rule[-0.200pt]{2.409pt}{0.400pt}}
\put(1429.0,832.0){\rule[-0.200pt]{2.409pt}{0.400pt}}
\put(170.0,832.0){\rule[-0.200pt]{2.409pt}{0.400pt}}
\put(1429.0,832.0){\rule[-0.200pt]{2.409pt}{0.400pt}}
\put(170.0,832.0){\rule[-0.200pt]{2.409pt}{0.400pt}}
\put(1429.0,832.0){\rule[-0.200pt]{2.409pt}{0.400pt}}
\put(170.0,833.0){\rule[-0.200pt]{2.409pt}{0.400pt}}
\put(1429.0,833.0){\rule[-0.200pt]{2.409pt}{0.400pt}}
\put(170.0,833.0){\rule[-0.200pt]{2.409pt}{0.400pt}}
\put(1429.0,833.0){\rule[-0.200pt]{2.409pt}{0.400pt}}
\put(170.0,833.0){\rule[-0.200pt]{2.409pt}{0.400pt}}
\put(1429.0,833.0){\rule[-0.200pt]{2.409pt}{0.400pt}}
\put(170.0,833.0){\rule[-0.200pt]{2.409pt}{0.400pt}}
\put(1429.0,833.0){\rule[-0.200pt]{2.409pt}{0.400pt}}
\put(170.0,833.0){\rule[-0.200pt]{2.409pt}{0.400pt}}
\put(1429.0,833.0){\rule[-0.200pt]{2.409pt}{0.400pt}}
\put(170.0,833.0){\rule[-0.200pt]{2.409pt}{0.400pt}}
\put(1429.0,833.0){\rule[-0.200pt]{2.409pt}{0.400pt}}
\put(170.0,833.0){\rule[-0.200pt]{2.409pt}{0.400pt}}
\put(1429.0,833.0){\rule[-0.200pt]{2.409pt}{0.400pt}}
\put(170.0,833.0){\rule[-0.200pt]{2.409pt}{0.400pt}}
\put(1429.0,833.0){\rule[-0.200pt]{2.409pt}{0.400pt}}
\put(170.0,833.0){\rule[-0.200pt]{2.409pt}{0.400pt}}
\put(1429.0,833.0){\rule[-0.200pt]{2.409pt}{0.400pt}}
\put(170.0,833.0){\rule[-0.200pt]{2.409pt}{0.400pt}}
\put(1429.0,833.0){\rule[-0.200pt]{2.409pt}{0.400pt}}
\put(170.0,833.0){\rule[-0.200pt]{2.409pt}{0.400pt}}
\put(1429.0,833.0){\rule[-0.200pt]{2.409pt}{0.400pt}}
\put(170.0,833.0){\rule[-0.200pt]{2.409pt}{0.400pt}}
\put(1429.0,833.0){\rule[-0.200pt]{2.409pt}{0.400pt}}
\put(170.0,833.0){\rule[-0.200pt]{2.409pt}{0.400pt}}
\put(1429.0,833.0){\rule[-0.200pt]{2.409pt}{0.400pt}}
\put(170.0,834.0){\rule[-0.200pt]{2.409pt}{0.400pt}}
\put(1429.0,834.0){\rule[-0.200pt]{2.409pt}{0.400pt}}
\put(170.0,834.0){\rule[-0.200pt]{2.409pt}{0.400pt}}
\put(1429.0,834.0){\rule[-0.200pt]{2.409pt}{0.400pt}}
\put(170.0,834.0){\rule[-0.200pt]{2.409pt}{0.400pt}}
\put(1429.0,834.0){\rule[-0.200pt]{2.409pt}{0.400pt}}
\put(170.0,834.0){\rule[-0.200pt]{2.409pt}{0.400pt}}
\put(1429.0,834.0){\rule[-0.200pt]{2.409pt}{0.400pt}}
\put(170.0,834.0){\rule[-0.200pt]{2.409pt}{0.400pt}}
\put(1429.0,834.0){\rule[-0.200pt]{2.409pt}{0.400pt}}
\put(170.0,834.0){\rule[-0.200pt]{2.409pt}{0.400pt}}
\put(1429.0,834.0){\rule[-0.200pt]{2.409pt}{0.400pt}}
\put(170.0,834.0){\rule[-0.200pt]{2.409pt}{0.400pt}}
\put(1429.0,834.0){\rule[-0.200pt]{2.409pt}{0.400pt}}
\put(170.0,834.0){\rule[-0.200pt]{2.409pt}{0.400pt}}
\put(1429.0,834.0){\rule[-0.200pt]{2.409pt}{0.400pt}}
\put(170.0,834.0){\rule[-0.200pt]{2.409pt}{0.400pt}}
\put(1429.0,834.0){\rule[-0.200pt]{2.409pt}{0.400pt}}
\put(170.0,834.0){\rule[-0.200pt]{2.409pt}{0.400pt}}
\put(1429.0,834.0){\rule[-0.200pt]{2.409pt}{0.400pt}}
\put(170.0,834.0){\rule[-0.200pt]{2.409pt}{0.400pt}}
\put(1429.0,834.0){\rule[-0.200pt]{2.409pt}{0.400pt}}
\put(170.0,834.0){\rule[-0.200pt]{2.409pt}{0.400pt}}
\put(1429.0,834.0){\rule[-0.200pt]{2.409pt}{0.400pt}}
\put(170.0,834.0){\rule[-0.200pt]{2.409pt}{0.400pt}}
\put(1429.0,834.0){\rule[-0.200pt]{2.409pt}{0.400pt}}
\put(170.0,834.0){\rule[-0.200pt]{2.409pt}{0.400pt}}
\put(1429.0,834.0){\rule[-0.200pt]{2.409pt}{0.400pt}}
\put(170.0,835.0){\rule[-0.200pt]{2.409pt}{0.400pt}}
\put(1429.0,835.0){\rule[-0.200pt]{2.409pt}{0.400pt}}
\put(170.0,835.0){\rule[-0.200pt]{2.409pt}{0.400pt}}
\put(1429.0,835.0){\rule[-0.200pt]{2.409pt}{0.400pt}}
\put(170.0,835.0){\rule[-0.200pt]{2.409pt}{0.400pt}}
\put(1429.0,835.0){\rule[-0.200pt]{2.409pt}{0.400pt}}
\put(170.0,835.0){\rule[-0.200pt]{2.409pt}{0.400pt}}
\put(1429.0,835.0){\rule[-0.200pt]{2.409pt}{0.400pt}}
\put(170.0,835.0){\rule[-0.200pt]{2.409pt}{0.400pt}}
\put(1429.0,835.0){\rule[-0.200pt]{2.409pt}{0.400pt}}
\put(170.0,835.0){\rule[-0.200pt]{2.409pt}{0.400pt}}
\put(1429.0,835.0){\rule[-0.200pt]{2.409pt}{0.400pt}}
\put(170.0,835.0){\rule[-0.200pt]{2.409pt}{0.400pt}}
\put(1429.0,835.0){\rule[-0.200pt]{2.409pt}{0.400pt}}
\put(170.0,835.0){\rule[-0.200pt]{2.409pt}{0.400pt}}
\put(1429.0,835.0){\rule[-0.200pt]{2.409pt}{0.400pt}}
\put(170.0,835.0){\rule[-0.200pt]{2.409pt}{0.400pt}}
\put(1429.0,835.0){\rule[-0.200pt]{2.409pt}{0.400pt}}
\put(170.0,835.0){\rule[-0.200pt]{2.409pt}{0.400pt}}
\put(1429.0,835.0){\rule[-0.200pt]{2.409pt}{0.400pt}}
\put(170.0,835.0){\rule[-0.200pt]{2.409pt}{0.400pt}}
\put(1429.0,835.0){\rule[-0.200pt]{2.409pt}{0.400pt}}
\put(170.0,835.0){\rule[-0.200pt]{2.409pt}{0.400pt}}
\put(1429.0,835.0){\rule[-0.200pt]{2.409pt}{0.400pt}}
\put(170.0,835.0){\rule[-0.200pt]{2.409pt}{0.400pt}}
\put(1429.0,835.0){\rule[-0.200pt]{2.409pt}{0.400pt}}
\put(170.0,835.0){\rule[-0.200pt]{2.409pt}{0.400pt}}
\put(1429.0,835.0){\rule[-0.200pt]{2.409pt}{0.400pt}}
\put(170.0,836.0){\rule[-0.200pt]{2.409pt}{0.400pt}}
\put(1429.0,836.0){\rule[-0.200pt]{2.409pt}{0.400pt}}
\put(170.0,836.0){\rule[-0.200pt]{2.409pt}{0.400pt}}
\put(1429.0,836.0){\rule[-0.200pt]{2.409pt}{0.400pt}}
\put(170.0,836.0){\rule[-0.200pt]{2.409pt}{0.400pt}}
\put(1429.0,836.0){\rule[-0.200pt]{2.409pt}{0.400pt}}
\put(170.0,836.0){\rule[-0.200pt]{2.409pt}{0.400pt}}
\put(1429.0,836.0){\rule[-0.200pt]{2.409pt}{0.400pt}}
\put(170.0,836.0){\rule[-0.200pt]{2.409pt}{0.400pt}}
\put(1429.0,836.0){\rule[-0.200pt]{2.409pt}{0.400pt}}
\put(170.0,836.0){\rule[-0.200pt]{2.409pt}{0.400pt}}
\put(1429.0,836.0){\rule[-0.200pt]{2.409pt}{0.400pt}}
\put(170.0,836.0){\rule[-0.200pt]{2.409pt}{0.400pt}}
\put(1429.0,836.0){\rule[-0.200pt]{2.409pt}{0.400pt}}
\put(170.0,836.0){\rule[-0.200pt]{2.409pt}{0.400pt}}
\put(1429.0,836.0){\rule[-0.200pt]{2.409pt}{0.400pt}}
\put(170.0,836.0){\rule[-0.200pt]{2.409pt}{0.400pt}}
\put(1429.0,836.0){\rule[-0.200pt]{2.409pt}{0.400pt}}
\put(170.0,836.0){\rule[-0.200pt]{2.409pt}{0.400pt}}
\put(1429.0,836.0){\rule[-0.200pt]{2.409pt}{0.400pt}}
\put(170.0,836.0){\rule[-0.200pt]{2.409pt}{0.400pt}}
\put(1429.0,836.0){\rule[-0.200pt]{2.409pt}{0.400pt}}
\put(170.0,836.0){\rule[-0.200pt]{2.409pt}{0.400pt}}
\put(1429.0,836.0){\rule[-0.200pt]{2.409pt}{0.400pt}}
\put(170.0,836.0){\rule[-0.200pt]{2.409pt}{0.400pt}}
\put(1429.0,836.0){\rule[-0.200pt]{2.409pt}{0.400pt}}
\put(170.0,836.0){\rule[-0.200pt]{2.409pt}{0.400pt}}
\put(1429.0,836.0){\rule[-0.200pt]{2.409pt}{0.400pt}}
\put(170.0,837.0){\rule[-0.200pt]{2.409pt}{0.400pt}}
\put(1429.0,837.0){\rule[-0.200pt]{2.409pt}{0.400pt}}
\put(170.0,837.0){\rule[-0.200pt]{2.409pt}{0.400pt}}
\put(1429.0,837.0){\rule[-0.200pt]{2.409pt}{0.400pt}}
\put(170.0,837.0){\rule[-0.200pt]{2.409pt}{0.400pt}}
\put(1429.0,837.0){\rule[-0.200pt]{2.409pt}{0.400pt}}
\put(170.0,837.0){\rule[-0.200pt]{2.409pt}{0.400pt}}
\put(1429.0,837.0){\rule[-0.200pt]{2.409pt}{0.400pt}}
\put(170.0,837.0){\rule[-0.200pt]{2.409pt}{0.400pt}}
\put(1429.0,837.0){\rule[-0.200pt]{2.409pt}{0.400pt}}
\put(170.0,837.0){\rule[-0.200pt]{2.409pt}{0.400pt}}
\put(1429.0,837.0){\rule[-0.200pt]{2.409pt}{0.400pt}}
\put(170.0,837.0){\rule[-0.200pt]{2.409pt}{0.400pt}}
\put(1429.0,837.0){\rule[-0.200pt]{2.409pt}{0.400pt}}
\put(170.0,837.0){\rule[-0.200pt]{2.409pt}{0.400pt}}
\put(1429.0,837.0){\rule[-0.200pt]{2.409pt}{0.400pt}}
\put(170.0,837.0){\rule[-0.200pt]{2.409pt}{0.400pt}}
\put(1429.0,837.0){\rule[-0.200pt]{2.409pt}{0.400pt}}
\put(170.0,837.0){\rule[-0.200pt]{2.409pt}{0.400pt}}
\put(1429.0,837.0){\rule[-0.200pt]{2.409pt}{0.400pt}}
\put(170.0,837.0){\rule[-0.200pt]{2.409pt}{0.400pt}}
\put(1429.0,837.0){\rule[-0.200pt]{2.409pt}{0.400pt}}
\put(170.0,837.0){\rule[-0.200pt]{2.409pt}{0.400pt}}
\put(1429.0,837.0){\rule[-0.200pt]{2.409pt}{0.400pt}}
\put(170.0,837.0){\rule[-0.200pt]{2.409pt}{0.400pt}}
\put(1429.0,837.0){\rule[-0.200pt]{2.409pt}{0.400pt}}
\put(170.0,837.0){\rule[-0.200pt]{2.409pt}{0.400pt}}
\put(1429.0,837.0){\rule[-0.200pt]{2.409pt}{0.400pt}}
\put(170.0,837.0){\rule[-0.200pt]{2.409pt}{0.400pt}}
\put(1429.0,837.0){\rule[-0.200pt]{2.409pt}{0.400pt}}
\put(170.0,838.0){\rule[-0.200pt]{2.409pt}{0.400pt}}
\put(1429.0,838.0){\rule[-0.200pt]{2.409pt}{0.400pt}}
\put(170.0,838.0){\rule[-0.200pt]{2.409pt}{0.400pt}}
\put(1429.0,838.0){\rule[-0.200pt]{2.409pt}{0.400pt}}
\put(170.0,838.0){\rule[-0.200pt]{2.409pt}{0.400pt}}
\put(1429.0,838.0){\rule[-0.200pt]{2.409pt}{0.400pt}}
\put(170.0,838.0){\rule[-0.200pt]{2.409pt}{0.400pt}}
\put(1429.0,838.0){\rule[-0.200pt]{2.409pt}{0.400pt}}
\put(170.0,838.0){\rule[-0.200pt]{2.409pt}{0.400pt}}
\put(1429.0,838.0){\rule[-0.200pt]{2.409pt}{0.400pt}}
\put(170.0,838.0){\rule[-0.200pt]{2.409pt}{0.400pt}}
\put(1429.0,838.0){\rule[-0.200pt]{2.409pt}{0.400pt}}
\put(170.0,838.0){\rule[-0.200pt]{2.409pt}{0.400pt}}
\put(1429.0,838.0){\rule[-0.200pt]{2.409pt}{0.400pt}}
\put(170.0,838.0){\rule[-0.200pt]{2.409pt}{0.400pt}}
\put(1429.0,838.0){\rule[-0.200pt]{2.409pt}{0.400pt}}
\put(170.0,838.0){\rule[-0.200pt]{2.409pt}{0.400pt}}
\put(1429.0,838.0){\rule[-0.200pt]{2.409pt}{0.400pt}}
\put(170.0,838.0){\rule[-0.200pt]{2.409pt}{0.400pt}}
\put(1429.0,838.0){\rule[-0.200pt]{2.409pt}{0.400pt}}
\put(170.0,838.0){\rule[-0.200pt]{2.409pt}{0.400pt}}
\put(1429.0,838.0){\rule[-0.200pt]{2.409pt}{0.400pt}}
\put(170.0,838.0){\rule[-0.200pt]{2.409pt}{0.400pt}}
\put(1429.0,838.0){\rule[-0.200pt]{2.409pt}{0.400pt}}
\put(170.0,838.0){\rule[-0.200pt]{2.409pt}{0.400pt}}
\put(1429.0,838.0){\rule[-0.200pt]{2.409pt}{0.400pt}}
\put(170.0,838.0){\rule[-0.200pt]{2.409pt}{0.400pt}}
\put(1429.0,838.0){\rule[-0.200pt]{2.409pt}{0.400pt}}
\put(170.0,838.0){\rule[-0.200pt]{2.409pt}{0.400pt}}
\put(1429.0,838.0){\rule[-0.200pt]{2.409pt}{0.400pt}}
\put(170.0,839.0){\rule[-0.200pt]{2.409pt}{0.400pt}}
\put(1429.0,839.0){\rule[-0.200pt]{2.409pt}{0.400pt}}
\put(170.0,839.0){\rule[-0.200pt]{2.409pt}{0.400pt}}
\put(1429.0,839.0){\rule[-0.200pt]{2.409pt}{0.400pt}}
\put(170.0,839.0){\rule[-0.200pt]{2.409pt}{0.400pt}}
\put(1429.0,839.0){\rule[-0.200pt]{2.409pt}{0.400pt}}
\put(170.0,839.0){\rule[-0.200pt]{2.409pt}{0.400pt}}
\put(1429.0,839.0){\rule[-0.200pt]{2.409pt}{0.400pt}}
\put(170.0,839.0){\rule[-0.200pt]{2.409pt}{0.400pt}}
\put(1429.0,839.0){\rule[-0.200pt]{2.409pt}{0.400pt}}
\put(170.0,839.0){\rule[-0.200pt]{2.409pt}{0.400pt}}
\put(1429.0,839.0){\rule[-0.200pt]{2.409pt}{0.400pt}}
\put(170.0,839.0){\rule[-0.200pt]{2.409pt}{0.400pt}}
\put(1429.0,839.0){\rule[-0.200pt]{2.409pt}{0.400pt}}
\put(170.0,839.0){\rule[-0.200pt]{2.409pt}{0.400pt}}
\put(1429.0,839.0){\rule[-0.200pt]{2.409pt}{0.400pt}}
\put(170.0,839.0){\rule[-0.200pt]{2.409pt}{0.400pt}}
\put(1429.0,839.0){\rule[-0.200pt]{2.409pt}{0.400pt}}
\put(170.0,839.0){\rule[-0.200pt]{2.409pt}{0.400pt}}
\put(1429.0,839.0){\rule[-0.200pt]{2.409pt}{0.400pt}}
\put(170.0,839.0){\rule[-0.200pt]{2.409pt}{0.400pt}}
\put(1429.0,839.0){\rule[-0.200pt]{2.409pt}{0.400pt}}
\put(170.0,839.0){\rule[-0.200pt]{2.409pt}{0.400pt}}
\put(1429.0,839.0){\rule[-0.200pt]{2.409pt}{0.400pt}}
\put(170.0,839.0){\rule[-0.200pt]{2.409pt}{0.400pt}}
\put(1429.0,839.0){\rule[-0.200pt]{2.409pt}{0.400pt}}
\put(170.0,839.0){\rule[-0.200pt]{2.409pt}{0.400pt}}
\put(1429.0,839.0){\rule[-0.200pt]{2.409pt}{0.400pt}}
\put(170.0,839.0){\rule[-0.200pt]{2.409pt}{0.400pt}}
\put(1429.0,839.0){\rule[-0.200pt]{2.409pt}{0.400pt}}
\put(170.0,839.0){\rule[-0.200pt]{2.409pt}{0.400pt}}
\put(1429.0,839.0){\rule[-0.200pt]{2.409pt}{0.400pt}}
\put(170.0,840.0){\rule[-0.200pt]{2.409pt}{0.400pt}}
\put(1429.0,840.0){\rule[-0.200pt]{2.409pt}{0.400pt}}
\put(170.0,840.0){\rule[-0.200pt]{2.409pt}{0.400pt}}
\put(1429.0,840.0){\rule[-0.200pt]{2.409pt}{0.400pt}}
\put(170.0,840.0){\rule[-0.200pt]{2.409pt}{0.400pt}}
\put(1429.0,840.0){\rule[-0.200pt]{2.409pt}{0.400pt}}
\put(170.0,840.0){\rule[-0.200pt]{2.409pt}{0.400pt}}
\put(1429.0,840.0){\rule[-0.200pt]{2.409pt}{0.400pt}}
\put(170.0,840.0){\rule[-0.200pt]{2.409pt}{0.400pt}}
\put(1429.0,840.0){\rule[-0.200pt]{2.409pt}{0.400pt}}
\put(170.0,840.0){\rule[-0.200pt]{2.409pt}{0.400pt}}
\put(1429.0,840.0){\rule[-0.200pt]{2.409pt}{0.400pt}}
\put(170.0,840.0){\rule[-0.200pt]{2.409pt}{0.400pt}}
\put(1429.0,840.0){\rule[-0.200pt]{2.409pt}{0.400pt}}
\put(170.0,840.0){\rule[-0.200pt]{2.409pt}{0.400pt}}
\put(1429.0,840.0){\rule[-0.200pt]{2.409pt}{0.400pt}}
\put(170.0,840.0){\rule[-0.200pt]{2.409pt}{0.400pt}}
\put(1429.0,840.0){\rule[-0.200pt]{2.409pt}{0.400pt}}
\put(170.0,840.0){\rule[-0.200pt]{2.409pt}{0.400pt}}
\put(1429.0,840.0){\rule[-0.200pt]{2.409pt}{0.400pt}}
\put(170.0,840.0){\rule[-0.200pt]{2.409pt}{0.400pt}}
\put(1429.0,840.0){\rule[-0.200pt]{2.409pt}{0.400pt}}
\put(170.0,840.0){\rule[-0.200pt]{2.409pt}{0.400pt}}
\put(1429.0,840.0){\rule[-0.200pt]{2.409pt}{0.400pt}}
\put(170.0,840.0){\rule[-0.200pt]{2.409pt}{0.400pt}}
\put(1429.0,840.0){\rule[-0.200pt]{2.409pt}{0.400pt}}
\put(170.0,840.0){\rule[-0.200pt]{2.409pt}{0.400pt}}
\put(1429.0,840.0){\rule[-0.200pt]{2.409pt}{0.400pt}}
\put(170.0,840.0){\rule[-0.200pt]{2.409pt}{0.400pt}}
\put(1429.0,840.0){\rule[-0.200pt]{2.409pt}{0.400pt}}
\put(170.0,840.0){\rule[-0.200pt]{2.409pt}{0.400pt}}
\put(1429.0,840.0){\rule[-0.200pt]{2.409pt}{0.400pt}}
\put(170.0,841.0){\rule[-0.200pt]{2.409pt}{0.400pt}}
\put(1429.0,841.0){\rule[-0.200pt]{2.409pt}{0.400pt}}
\put(170.0,841.0){\rule[-0.200pt]{2.409pt}{0.400pt}}
\put(1429.0,841.0){\rule[-0.200pt]{2.409pt}{0.400pt}}
\put(170.0,841.0){\rule[-0.200pt]{2.409pt}{0.400pt}}
\put(1429.0,841.0){\rule[-0.200pt]{2.409pt}{0.400pt}}
\put(170.0,841.0){\rule[-0.200pt]{2.409pt}{0.400pt}}
\put(1429.0,841.0){\rule[-0.200pt]{2.409pt}{0.400pt}}
\put(170.0,841.0){\rule[-0.200pt]{2.409pt}{0.400pt}}
\put(1429.0,841.0){\rule[-0.200pt]{2.409pt}{0.400pt}}
\put(170.0,841.0){\rule[-0.200pt]{2.409pt}{0.400pt}}
\put(1429.0,841.0){\rule[-0.200pt]{2.409pt}{0.400pt}}
\put(170.0,841.0){\rule[-0.200pt]{2.409pt}{0.400pt}}
\put(1429.0,841.0){\rule[-0.200pt]{2.409pt}{0.400pt}}
\put(170.0,841.0){\rule[-0.200pt]{2.409pt}{0.400pt}}
\put(1429.0,841.0){\rule[-0.200pt]{2.409pt}{0.400pt}}
\put(170.0,841.0){\rule[-0.200pt]{2.409pt}{0.400pt}}
\put(1429.0,841.0){\rule[-0.200pt]{2.409pt}{0.400pt}}
\put(170.0,841.0){\rule[-0.200pt]{2.409pt}{0.400pt}}
\put(1429.0,841.0){\rule[-0.200pt]{2.409pt}{0.400pt}}
\put(170.0,841.0){\rule[-0.200pt]{2.409pt}{0.400pt}}
\put(1429.0,841.0){\rule[-0.200pt]{2.409pt}{0.400pt}}
\put(170.0,841.0){\rule[-0.200pt]{2.409pt}{0.400pt}}
\put(1429.0,841.0){\rule[-0.200pt]{2.409pt}{0.400pt}}
\put(170.0,841.0){\rule[-0.200pt]{2.409pt}{0.400pt}}
\put(1429.0,841.0){\rule[-0.200pt]{2.409pt}{0.400pt}}
\put(170.0,841.0){\rule[-0.200pt]{2.409pt}{0.400pt}}
\put(1429.0,841.0){\rule[-0.200pt]{2.409pt}{0.400pt}}
\put(170.0,841.0){\rule[-0.200pt]{2.409pt}{0.400pt}}
\put(1429.0,841.0){\rule[-0.200pt]{2.409pt}{0.400pt}}
\put(170.0,841.0){\rule[-0.200pt]{2.409pt}{0.400pt}}
\put(1429.0,841.0){\rule[-0.200pt]{2.409pt}{0.400pt}}
\put(170.0,841.0){\rule[-0.200pt]{2.409pt}{0.400pt}}
\put(1429.0,841.0){\rule[-0.200pt]{2.409pt}{0.400pt}}
\put(170.0,842.0){\rule[-0.200pt]{2.409pt}{0.400pt}}
\put(1429.0,842.0){\rule[-0.200pt]{2.409pt}{0.400pt}}
\put(170.0,842.0){\rule[-0.200pt]{2.409pt}{0.400pt}}
\put(1429.0,842.0){\rule[-0.200pt]{2.409pt}{0.400pt}}
\put(170.0,842.0){\rule[-0.200pt]{2.409pt}{0.400pt}}
\put(1429.0,842.0){\rule[-0.200pt]{2.409pt}{0.400pt}}
\put(170.0,842.0){\rule[-0.200pt]{2.409pt}{0.400pt}}
\put(1429.0,842.0){\rule[-0.200pt]{2.409pt}{0.400pt}}
\put(170.0,842.0){\rule[-0.200pt]{2.409pt}{0.400pt}}
\put(1429.0,842.0){\rule[-0.200pt]{2.409pt}{0.400pt}}
\put(170.0,842.0){\rule[-0.200pt]{2.409pt}{0.400pt}}
\put(1429.0,842.0){\rule[-0.200pt]{2.409pt}{0.400pt}}
\put(170.0,842.0){\rule[-0.200pt]{2.409pt}{0.400pt}}
\put(1429.0,842.0){\rule[-0.200pt]{2.409pt}{0.400pt}}
\put(170.0,842.0){\rule[-0.200pt]{2.409pt}{0.400pt}}
\put(1429.0,842.0){\rule[-0.200pt]{2.409pt}{0.400pt}}
\put(170.0,842.0){\rule[-0.200pt]{2.409pt}{0.400pt}}
\put(1429.0,842.0){\rule[-0.200pt]{2.409pt}{0.400pt}}
\put(170.0,842.0){\rule[-0.200pt]{2.409pt}{0.400pt}}
\put(1429.0,842.0){\rule[-0.200pt]{2.409pt}{0.400pt}}
\put(170.0,842.0){\rule[-0.200pt]{2.409pt}{0.400pt}}
\put(1429.0,842.0){\rule[-0.200pt]{2.409pt}{0.400pt}}
\put(170.0,842.0){\rule[-0.200pt]{2.409pt}{0.400pt}}
\put(1429.0,842.0){\rule[-0.200pt]{2.409pt}{0.400pt}}
\put(170.0,842.0){\rule[-0.200pt]{2.409pt}{0.400pt}}
\put(1429.0,842.0){\rule[-0.200pt]{2.409pt}{0.400pt}}
\put(170.0,842.0){\rule[-0.200pt]{2.409pt}{0.400pt}}
\put(1429.0,842.0){\rule[-0.200pt]{2.409pt}{0.400pt}}
\put(170.0,842.0){\rule[-0.200pt]{2.409pt}{0.400pt}}
\put(1429.0,842.0){\rule[-0.200pt]{2.409pt}{0.400pt}}
\put(170.0,842.0){\rule[-0.200pt]{2.409pt}{0.400pt}}
\put(1429.0,842.0){\rule[-0.200pt]{2.409pt}{0.400pt}}
\put(170.0,843.0){\rule[-0.200pt]{2.409pt}{0.400pt}}
\put(1429.0,843.0){\rule[-0.200pt]{2.409pt}{0.400pt}}
\put(170.0,843.0){\rule[-0.200pt]{2.409pt}{0.400pt}}
\put(1429.0,843.0){\rule[-0.200pt]{2.409pt}{0.400pt}}
\put(170.0,843.0){\rule[-0.200pt]{2.409pt}{0.400pt}}
\put(1429.0,843.0){\rule[-0.200pt]{2.409pt}{0.400pt}}
\put(170.0,843.0){\rule[-0.200pt]{2.409pt}{0.400pt}}
\put(1429.0,843.0){\rule[-0.200pt]{2.409pt}{0.400pt}}
\put(170.0,843.0){\rule[-0.200pt]{2.409pt}{0.400pt}}
\put(1429.0,843.0){\rule[-0.200pt]{2.409pt}{0.400pt}}
\put(170.0,843.0){\rule[-0.200pt]{2.409pt}{0.400pt}}
\put(1429.0,843.0){\rule[-0.200pt]{2.409pt}{0.400pt}}
\put(170.0,843.0){\rule[-0.200pt]{2.409pt}{0.400pt}}
\put(1429.0,843.0){\rule[-0.200pt]{2.409pt}{0.400pt}}
\put(170.0,843.0){\rule[-0.200pt]{2.409pt}{0.400pt}}
\put(1429.0,843.0){\rule[-0.200pt]{2.409pt}{0.400pt}}
\put(170.0,843.0){\rule[-0.200pt]{2.409pt}{0.400pt}}
\put(1429.0,843.0){\rule[-0.200pt]{2.409pt}{0.400pt}}
\put(170.0,843.0){\rule[-0.200pt]{2.409pt}{0.400pt}}
\put(1429.0,843.0){\rule[-0.200pt]{2.409pt}{0.400pt}}
\put(170.0,843.0){\rule[-0.200pt]{2.409pt}{0.400pt}}
\put(1429.0,843.0){\rule[-0.200pt]{2.409pt}{0.400pt}}
\put(170.0,843.0){\rule[-0.200pt]{2.409pt}{0.400pt}}
\put(1429.0,843.0){\rule[-0.200pt]{2.409pt}{0.400pt}}
\put(170.0,843.0){\rule[-0.200pt]{2.409pt}{0.400pt}}
\put(1429.0,843.0){\rule[-0.200pt]{2.409pt}{0.400pt}}
\put(170.0,843.0){\rule[-0.200pt]{2.409pt}{0.400pt}}
\put(1429.0,843.0){\rule[-0.200pt]{2.409pt}{0.400pt}}
\put(170.0,843.0){\rule[-0.200pt]{2.409pt}{0.400pt}}
\put(1429.0,843.0){\rule[-0.200pt]{2.409pt}{0.400pt}}
\put(170.0,843.0){\rule[-0.200pt]{2.409pt}{0.400pt}}
\put(1429.0,843.0){\rule[-0.200pt]{2.409pt}{0.400pt}}
\put(170.0,843.0){\rule[-0.200pt]{2.409pt}{0.400pt}}
\put(1429.0,843.0){\rule[-0.200pt]{2.409pt}{0.400pt}}
\put(170.0,843.0){\rule[-0.200pt]{2.409pt}{0.400pt}}
\put(1429.0,843.0){\rule[-0.200pt]{2.409pt}{0.400pt}}
\put(170.0,844.0){\rule[-0.200pt]{2.409pt}{0.400pt}}
\put(1429.0,844.0){\rule[-0.200pt]{2.409pt}{0.400pt}}
\put(170.0,844.0){\rule[-0.200pt]{2.409pt}{0.400pt}}
\put(1429.0,844.0){\rule[-0.200pt]{2.409pt}{0.400pt}}
\put(170.0,844.0){\rule[-0.200pt]{2.409pt}{0.400pt}}
\put(1429.0,844.0){\rule[-0.200pt]{2.409pt}{0.400pt}}
\put(170.0,844.0){\rule[-0.200pt]{2.409pt}{0.400pt}}
\put(1429.0,844.0){\rule[-0.200pt]{2.409pt}{0.400pt}}
\put(170.0,844.0){\rule[-0.200pt]{2.409pt}{0.400pt}}
\put(1429.0,844.0){\rule[-0.200pt]{2.409pt}{0.400pt}}
\put(170.0,844.0){\rule[-0.200pt]{2.409pt}{0.400pt}}
\put(1429.0,844.0){\rule[-0.200pt]{2.409pt}{0.400pt}}
\put(170.0,844.0){\rule[-0.200pt]{2.409pt}{0.400pt}}
\put(1429.0,844.0){\rule[-0.200pt]{2.409pt}{0.400pt}}
\put(170.0,844.0){\rule[-0.200pt]{2.409pt}{0.400pt}}
\put(1429.0,844.0){\rule[-0.200pt]{2.409pt}{0.400pt}}
\put(170.0,844.0){\rule[-0.200pt]{2.409pt}{0.400pt}}
\put(1429.0,844.0){\rule[-0.200pt]{2.409pt}{0.400pt}}
\put(170.0,844.0){\rule[-0.200pt]{2.409pt}{0.400pt}}
\put(1429.0,844.0){\rule[-0.200pt]{2.409pt}{0.400pt}}
\put(170.0,844.0){\rule[-0.200pt]{2.409pt}{0.400pt}}
\put(1429.0,844.0){\rule[-0.200pt]{2.409pt}{0.400pt}}
\put(170.0,844.0){\rule[-0.200pt]{2.409pt}{0.400pt}}
\put(1429.0,844.0){\rule[-0.200pt]{2.409pt}{0.400pt}}
\put(170.0,844.0){\rule[-0.200pt]{2.409pt}{0.400pt}}
\put(1429.0,844.0){\rule[-0.200pt]{2.409pt}{0.400pt}}
\put(170.0,844.0){\rule[-0.200pt]{2.409pt}{0.400pt}}
\put(1429.0,844.0){\rule[-0.200pt]{2.409pt}{0.400pt}}
\put(170.0,844.0){\rule[-0.200pt]{2.409pt}{0.400pt}}
\put(1429.0,844.0){\rule[-0.200pt]{2.409pt}{0.400pt}}
\put(170.0,844.0){\rule[-0.200pt]{2.409pt}{0.400pt}}
\put(1429.0,844.0){\rule[-0.200pt]{2.409pt}{0.400pt}}
\put(170.0,844.0){\rule[-0.200pt]{2.409pt}{0.400pt}}
\put(1429.0,844.0){\rule[-0.200pt]{2.409pt}{0.400pt}}
\put(170.0,844.0){\rule[-0.200pt]{2.409pt}{0.400pt}}
\put(1429.0,844.0){\rule[-0.200pt]{2.409pt}{0.400pt}}
\put(170.0,845.0){\rule[-0.200pt]{2.409pt}{0.400pt}}
\put(1429.0,845.0){\rule[-0.200pt]{2.409pt}{0.400pt}}
\put(170.0,845.0){\rule[-0.200pt]{2.409pt}{0.400pt}}
\put(1429.0,845.0){\rule[-0.200pt]{2.409pt}{0.400pt}}
\put(170.0,845.0){\rule[-0.200pt]{2.409pt}{0.400pt}}
\put(1429.0,845.0){\rule[-0.200pt]{2.409pt}{0.400pt}}
\put(170.0,845.0){\rule[-0.200pt]{2.409pt}{0.400pt}}
\put(1429.0,845.0){\rule[-0.200pt]{2.409pt}{0.400pt}}
\put(170.0,845.0){\rule[-0.200pt]{2.409pt}{0.400pt}}
\put(1429.0,845.0){\rule[-0.200pt]{2.409pt}{0.400pt}}
\put(170.0,845.0){\rule[-0.200pt]{2.409pt}{0.400pt}}
\put(1429.0,845.0){\rule[-0.200pt]{2.409pt}{0.400pt}}
\put(170.0,845.0){\rule[-0.200pt]{2.409pt}{0.400pt}}
\put(1429.0,845.0){\rule[-0.200pt]{2.409pt}{0.400pt}}
\put(170.0,845.0){\rule[-0.200pt]{2.409pt}{0.400pt}}
\put(1429.0,845.0){\rule[-0.200pt]{2.409pt}{0.400pt}}
\put(170.0,845.0){\rule[-0.200pt]{2.409pt}{0.400pt}}
\put(1429.0,845.0){\rule[-0.200pt]{2.409pt}{0.400pt}}
\put(170.0,845.0){\rule[-0.200pt]{2.409pt}{0.400pt}}
\put(1429.0,845.0){\rule[-0.200pt]{2.409pt}{0.400pt}}
\put(170.0,845.0){\rule[-0.200pt]{2.409pt}{0.400pt}}
\put(1429.0,845.0){\rule[-0.200pt]{2.409pt}{0.400pt}}
\put(170.0,845.0){\rule[-0.200pt]{2.409pt}{0.400pt}}
\put(1429.0,845.0){\rule[-0.200pt]{2.409pt}{0.400pt}}
\put(170.0,845.0){\rule[-0.200pt]{2.409pt}{0.400pt}}
\put(1429.0,845.0){\rule[-0.200pt]{2.409pt}{0.400pt}}
\put(170.0,845.0){\rule[-0.200pt]{2.409pt}{0.400pt}}
\put(1429.0,845.0){\rule[-0.200pt]{2.409pt}{0.400pt}}
\put(170.0,845.0){\rule[-0.200pt]{2.409pt}{0.400pt}}
\put(1429.0,845.0){\rule[-0.200pt]{2.409pt}{0.400pt}}
\put(170.0,845.0){\rule[-0.200pt]{2.409pt}{0.400pt}}
\put(1429.0,845.0){\rule[-0.200pt]{2.409pt}{0.400pt}}
\put(170.0,845.0){\rule[-0.200pt]{2.409pt}{0.400pt}}
\put(1429.0,845.0){\rule[-0.200pt]{2.409pt}{0.400pt}}
\put(170.0,845.0){\rule[-0.200pt]{2.409pt}{0.400pt}}
\put(1429.0,845.0){\rule[-0.200pt]{2.409pt}{0.400pt}}
\put(170.0,846.0){\rule[-0.200pt]{2.409pt}{0.400pt}}
\put(1429.0,846.0){\rule[-0.200pt]{2.409pt}{0.400pt}}
\put(170.0,846.0){\rule[-0.200pt]{2.409pt}{0.400pt}}
\put(1429.0,846.0){\rule[-0.200pt]{2.409pt}{0.400pt}}
\put(170.0,846.0){\rule[-0.200pt]{2.409pt}{0.400pt}}
\put(1429.0,846.0){\rule[-0.200pt]{2.409pt}{0.400pt}}
\put(170.0,846.0){\rule[-0.200pt]{2.409pt}{0.400pt}}
\put(1429.0,846.0){\rule[-0.200pt]{2.409pt}{0.400pt}}
\put(170.0,846.0){\rule[-0.200pt]{2.409pt}{0.400pt}}
\put(1429.0,846.0){\rule[-0.200pt]{2.409pt}{0.400pt}}
\put(170.0,846.0){\rule[-0.200pt]{2.409pt}{0.400pt}}
\put(1429.0,846.0){\rule[-0.200pt]{2.409pt}{0.400pt}}
\put(170.0,846.0){\rule[-0.200pt]{2.409pt}{0.400pt}}
\put(1429.0,846.0){\rule[-0.200pt]{2.409pt}{0.400pt}}
\put(170.0,846.0){\rule[-0.200pt]{2.409pt}{0.400pt}}
\put(1429.0,846.0){\rule[-0.200pt]{2.409pt}{0.400pt}}
\put(170.0,846.0){\rule[-0.200pt]{2.409pt}{0.400pt}}
\put(1429.0,846.0){\rule[-0.200pt]{2.409pt}{0.400pt}}
\put(170.0,846.0){\rule[-0.200pt]{2.409pt}{0.400pt}}
\put(1429.0,846.0){\rule[-0.200pt]{2.409pt}{0.400pt}}
\put(170.0,846.0){\rule[-0.200pt]{2.409pt}{0.400pt}}
\put(1429.0,846.0){\rule[-0.200pt]{2.409pt}{0.400pt}}
\put(170.0,846.0){\rule[-0.200pt]{2.409pt}{0.400pt}}
\put(1429.0,846.0){\rule[-0.200pt]{2.409pt}{0.400pt}}
\put(170.0,846.0){\rule[-0.200pt]{2.409pt}{0.400pt}}
\put(1429.0,846.0){\rule[-0.200pt]{2.409pt}{0.400pt}}
\put(170.0,846.0){\rule[-0.200pt]{2.409pt}{0.400pt}}
\put(1429.0,846.0){\rule[-0.200pt]{2.409pt}{0.400pt}}
\put(170.0,846.0){\rule[-0.200pt]{2.409pt}{0.400pt}}
\put(1429.0,846.0){\rule[-0.200pt]{2.409pt}{0.400pt}}
\put(170.0,846.0){\rule[-0.200pt]{2.409pt}{0.400pt}}
\put(1429.0,846.0){\rule[-0.200pt]{2.409pt}{0.400pt}}
\put(170.0,846.0){\rule[-0.200pt]{2.409pt}{0.400pt}}
\put(1429.0,846.0){\rule[-0.200pt]{2.409pt}{0.400pt}}
\put(170.0,846.0){\rule[-0.200pt]{2.409pt}{0.400pt}}
\put(1429.0,846.0){\rule[-0.200pt]{2.409pt}{0.400pt}}
\put(170.0,846.0){\rule[-0.200pt]{2.409pt}{0.400pt}}
\put(1429.0,846.0){\rule[-0.200pt]{2.409pt}{0.400pt}}
\put(170.0,847.0){\rule[-0.200pt]{2.409pt}{0.400pt}}
\put(1429.0,847.0){\rule[-0.200pt]{2.409pt}{0.400pt}}
\put(170.0,847.0){\rule[-0.200pt]{2.409pt}{0.400pt}}
\put(1429.0,847.0){\rule[-0.200pt]{2.409pt}{0.400pt}}
\put(170.0,847.0){\rule[-0.200pt]{2.409pt}{0.400pt}}
\put(1429.0,847.0){\rule[-0.200pt]{2.409pt}{0.400pt}}
\put(170.0,847.0){\rule[-0.200pt]{2.409pt}{0.400pt}}
\put(1429.0,847.0){\rule[-0.200pt]{2.409pt}{0.400pt}}
\put(170.0,847.0){\rule[-0.200pt]{2.409pt}{0.400pt}}
\put(1429.0,847.0){\rule[-0.200pt]{2.409pt}{0.400pt}}
\put(170.0,847.0){\rule[-0.200pt]{2.409pt}{0.400pt}}
\put(1429.0,847.0){\rule[-0.200pt]{2.409pt}{0.400pt}}
\put(170.0,847.0){\rule[-0.200pt]{2.409pt}{0.400pt}}
\put(1429.0,847.0){\rule[-0.200pt]{2.409pt}{0.400pt}}
\put(170.0,847.0){\rule[-0.200pt]{2.409pt}{0.400pt}}
\put(1429.0,847.0){\rule[-0.200pt]{2.409pt}{0.400pt}}
\put(170.0,847.0){\rule[-0.200pt]{2.409pt}{0.400pt}}
\put(1429.0,847.0){\rule[-0.200pt]{2.409pt}{0.400pt}}
\put(170.0,847.0){\rule[-0.200pt]{2.409pt}{0.400pt}}
\put(1429.0,847.0){\rule[-0.200pt]{2.409pt}{0.400pt}}
\put(170.0,847.0){\rule[-0.200pt]{2.409pt}{0.400pt}}
\put(1429.0,847.0){\rule[-0.200pt]{2.409pt}{0.400pt}}
\put(170.0,847.0){\rule[-0.200pt]{2.409pt}{0.400pt}}
\put(1429.0,847.0){\rule[-0.200pt]{2.409pt}{0.400pt}}
\put(170.0,847.0){\rule[-0.200pt]{2.409pt}{0.400pt}}
\put(1429.0,847.0){\rule[-0.200pt]{2.409pt}{0.400pt}}
\put(170.0,847.0){\rule[-0.200pt]{2.409pt}{0.400pt}}
\put(1429.0,847.0){\rule[-0.200pt]{2.409pt}{0.400pt}}
\put(170.0,847.0){\rule[-0.200pt]{2.409pt}{0.400pt}}
\put(1429.0,847.0){\rule[-0.200pt]{2.409pt}{0.400pt}}
\put(170.0,847.0){\rule[-0.200pt]{2.409pt}{0.400pt}}
\put(1429.0,847.0){\rule[-0.200pt]{2.409pt}{0.400pt}}
\put(170.0,847.0){\rule[-0.200pt]{2.409pt}{0.400pt}}
\put(1429.0,847.0){\rule[-0.200pt]{2.409pt}{0.400pt}}
\put(170.0,847.0){\rule[-0.200pt]{2.409pt}{0.400pt}}
\put(1429.0,847.0){\rule[-0.200pt]{2.409pt}{0.400pt}}
\put(170.0,847.0){\rule[-0.200pt]{2.409pt}{0.400pt}}
\put(1429.0,847.0){\rule[-0.200pt]{2.409pt}{0.400pt}}
\put(170.0,848.0){\rule[-0.200pt]{2.409pt}{0.400pt}}
\put(1429.0,848.0){\rule[-0.200pt]{2.409pt}{0.400pt}}
\put(170.0,848.0){\rule[-0.200pt]{2.409pt}{0.400pt}}
\put(1429.0,848.0){\rule[-0.200pt]{2.409pt}{0.400pt}}
\put(170.0,848.0){\rule[-0.200pt]{2.409pt}{0.400pt}}
\put(1429.0,848.0){\rule[-0.200pt]{2.409pt}{0.400pt}}
\put(170.0,848.0){\rule[-0.200pt]{2.409pt}{0.400pt}}
\put(1429.0,848.0){\rule[-0.200pt]{2.409pt}{0.400pt}}
\put(170.0,848.0){\rule[-0.200pt]{2.409pt}{0.400pt}}
\put(1429.0,848.0){\rule[-0.200pt]{2.409pt}{0.400pt}}
\put(170.0,848.0){\rule[-0.200pt]{2.409pt}{0.400pt}}
\put(1429.0,848.0){\rule[-0.200pt]{2.409pt}{0.400pt}}
\put(170.0,848.0){\rule[-0.200pt]{2.409pt}{0.400pt}}
\put(1429.0,848.0){\rule[-0.200pt]{2.409pt}{0.400pt}}
\put(170.0,848.0){\rule[-0.200pt]{2.409pt}{0.400pt}}
\put(1429.0,848.0){\rule[-0.200pt]{2.409pt}{0.400pt}}
\put(170.0,848.0){\rule[-0.200pt]{2.409pt}{0.400pt}}
\put(1429.0,848.0){\rule[-0.200pt]{2.409pt}{0.400pt}}
\put(170.0,848.0){\rule[-0.200pt]{2.409pt}{0.400pt}}
\put(1429.0,848.0){\rule[-0.200pt]{2.409pt}{0.400pt}}
\put(170.0,848.0){\rule[-0.200pt]{2.409pt}{0.400pt}}
\put(1429.0,848.0){\rule[-0.200pt]{2.409pt}{0.400pt}}
\put(170.0,848.0){\rule[-0.200pt]{2.409pt}{0.400pt}}
\put(1429.0,848.0){\rule[-0.200pt]{2.409pt}{0.400pt}}
\put(170.0,848.0){\rule[-0.200pt]{2.409pt}{0.400pt}}
\put(1429.0,848.0){\rule[-0.200pt]{2.409pt}{0.400pt}}
\put(170.0,848.0){\rule[-0.200pt]{2.409pt}{0.400pt}}
\put(1429.0,848.0){\rule[-0.200pt]{2.409pt}{0.400pt}}
\put(170.0,848.0){\rule[-0.200pt]{2.409pt}{0.400pt}}
\put(1429.0,848.0){\rule[-0.200pt]{2.409pt}{0.400pt}}
\put(170.0,848.0){\rule[-0.200pt]{2.409pt}{0.400pt}}
\put(1429.0,848.0){\rule[-0.200pt]{2.409pt}{0.400pt}}
\put(170.0,848.0){\rule[-0.200pt]{2.409pt}{0.400pt}}
\put(1429.0,848.0){\rule[-0.200pt]{2.409pt}{0.400pt}}
\put(170.0,848.0){\rule[-0.200pt]{2.409pt}{0.400pt}}
\put(1429.0,848.0){\rule[-0.200pt]{2.409pt}{0.400pt}}
\put(170.0,848.0){\rule[-0.200pt]{2.409pt}{0.400pt}}
\put(1429.0,848.0){\rule[-0.200pt]{2.409pt}{0.400pt}}
\put(170.0,848.0){\rule[-0.200pt]{2.409pt}{0.400pt}}
\put(1429.0,848.0){\rule[-0.200pt]{2.409pt}{0.400pt}}
\put(170.0,849.0){\rule[-0.200pt]{2.409pt}{0.400pt}}
\put(1429.0,849.0){\rule[-0.200pt]{2.409pt}{0.400pt}}
\put(170.0,849.0){\rule[-0.200pt]{2.409pt}{0.400pt}}
\put(1429.0,849.0){\rule[-0.200pt]{2.409pt}{0.400pt}}
\put(170.0,849.0){\rule[-0.200pt]{2.409pt}{0.400pt}}
\put(1429.0,849.0){\rule[-0.200pt]{2.409pt}{0.400pt}}
\put(170.0,849.0){\rule[-0.200pt]{2.409pt}{0.400pt}}
\put(1429.0,849.0){\rule[-0.200pt]{2.409pt}{0.400pt}}
\put(170.0,849.0){\rule[-0.200pt]{2.409pt}{0.400pt}}
\put(1429.0,849.0){\rule[-0.200pt]{2.409pt}{0.400pt}}
\put(170.0,849.0){\rule[-0.200pt]{2.409pt}{0.400pt}}
\put(1429.0,849.0){\rule[-0.200pt]{2.409pt}{0.400pt}}
\put(170.0,849.0){\rule[-0.200pt]{2.409pt}{0.400pt}}
\put(1429.0,849.0){\rule[-0.200pt]{2.409pt}{0.400pt}}
\put(170.0,849.0){\rule[-0.200pt]{2.409pt}{0.400pt}}
\put(1429.0,849.0){\rule[-0.200pt]{2.409pt}{0.400pt}}
\put(170.0,849.0){\rule[-0.200pt]{2.409pt}{0.400pt}}
\put(1429.0,849.0){\rule[-0.200pt]{2.409pt}{0.400pt}}
\put(170.0,849.0){\rule[-0.200pt]{2.409pt}{0.400pt}}
\put(1429.0,849.0){\rule[-0.200pt]{2.409pt}{0.400pt}}
\put(170.0,849.0){\rule[-0.200pt]{2.409pt}{0.400pt}}
\put(1429.0,849.0){\rule[-0.200pt]{2.409pt}{0.400pt}}
\put(170.0,849.0){\rule[-0.200pt]{2.409pt}{0.400pt}}
\put(1429.0,849.0){\rule[-0.200pt]{2.409pt}{0.400pt}}
\put(170.0,849.0){\rule[-0.200pt]{2.409pt}{0.400pt}}
\put(1429.0,849.0){\rule[-0.200pt]{2.409pt}{0.400pt}}
\put(170.0,849.0){\rule[-0.200pt]{2.409pt}{0.400pt}}
\put(1429.0,849.0){\rule[-0.200pt]{2.409pt}{0.400pt}}
\put(170.0,849.0){\rule[-0.200pt]{2.409pt}{0.400pt}}
\put(1429.0,849.0){\rule[-0.200pt]{2.409pt}{0.400pt}}
\put(170.0,849.0){\rule[-0.200pt]{2.409pt}{0.400pt}}
\put(1429.0,849.0){\rule[-0.200pt]{2.409pt}{0.400pt}}
\put(170.0,849.0){\rule[-0.200pt]{2.409pt}{0.400pt}}
\put(1429.0,849.0){\rule[-0.200pt]{2.409pt}{0.400pt}}
\put(170.0,849.0){\rule[-0.200pt]{2.409pt}{0.400pt}}
\put(1429.0,849.0){\rule[-0.200pt]{2.409pt}{0.400pt}}
\put(170.0,849.0){\rule[-0.200pt]{2.409pt}{0.400pt}}
\put(1429.0,849.0){\rule[-0.200pt]{2.409pt}{0.400pt}}
\put(170.0,849.0){\rule[-0.200pt]{2.409pt}{0.400pt}}
\put(1429.0,849.0){\rule[-0.200pt]{2.409pt}{0.400pt}}
\put(170.0,849.0){\rule[-0.200pt]{2.409pt}{0.400pt}}
\put(1429.0,849.0){\rule[-0.200pt]{2.409pt}{0.400pt}}
\put(170.0,850.0){\rule[-0.200pt]{2.409pt}{0.400pt}}
\put(1429.0,850.0){\rule[-0.200pt]{2.409pt}{0.400pt}}
\put(170.0,850.0){\rule[-0.200pt]{2.409pt}{0.400pt}}
\put(1429.0,850.0){\rule[-0.200pt]{2.409pt}{0.400pt}}
\put(170.0,850.0){\rule[-0.200pt]{2.409pt}{0.400pt}}
\put(1429.0,850.0){\rule[-0.200pt]{2.409pt}{0.400pt}}
\put(170.0,850.0){\rule[-0.200pt]{2.409pt}{0.400pt}}
\put(1429.0,850.0){\rule[-0.200pt]{2.409pt}{0.400pt}}
\put(170.0,850.0){\rule[-0.200pt]{2.409pt}{0.400pt}}
\put(1429.0,850.0){\rule[-0.200pt]{2.409pt}{0.400pt}}
\put(170.0,850.0){\rule[-0.200pt]{2.409pt}{0.400pt}}
\put(1429.0,850.0){\rule[-0.200pt]{2.409pt}{0.400pt}}
\put(170.0,850.0){\rule[-0.200pt]{2.409pt}{0.400pt}}
\put(1429.0,850.0){\rule[-0.200pt]{2.409pt}{0.400pt}}
\put(170.0,850.0){\rule[-0.200pt]{2.409pt}{0.400pt}}
\put(1429.0,850.0){\rule[-0.200pt]{2.409pt}{0.400pt}}
\put(170.0,850.0){\rule[-0.200pt]{2.409pt}{0.400pt}}
\put(1429.0,850.0){\rule[-0.200pt]{2.409pt}{0.400pt}}
\put(170.0,850.0){\rule[-0.200pt]{2.409pt}{0.400pt}}
\put(1429.0,850.0){\rule[-0.200pt]{2.409pt}{0.400pt}}
\put(170.0,850.0){\rule[-0.200pt]{2.409pt}{0.400pt}}
\put(1429.0,850.0){\rule[-0.200pt]{2.409pt}{0.400pt}}
\put(170.0,850.0){\rule[-0.200pt]{2.409pt}{0.400pt}}
\put(1429.0,850.0){\rule[-0.200pt]{2.409pt}{0.400pt}}
\put(170.0,850.0){\rule[-0.200pt]{2.409pt}{0.400pt}}
\put(1429.0,850.0){\rule[-0.200pt]{2.409pt}{0.400pt}}
\put(170.0,850.0){\rule[-0.200pt]{2.409pt}{0.400pt}}
\put(1429.0,850.0){\rule[-0.200pt]{2.409pt}{0.400pt}}
\put(170.0,850.0){\rule[-0.200pt]{2.409pt}{0.400pt}}
\put(1429.0,850.0){\rule[-0.200pt]{2.409pt}{0.400pt}}
\put(170.0,850.0){\rule[-0.200pt]{2.409pt}{0.400pt}}
\put(1429.0,850.0){\rule[-0.200pt]{2.409pt}{0.400pt}}
\put(170.0,850.0){\rule[-0.200pt]{2.409pt}{0.400pt}}
\put(1429.0,850.0){\rule[-0.200pt]{2.409pt}{0.400pt}}
\put(170.0,850.0){\rule[-0.200pt]{2.409pt}{0.400pt}}
\put(1429.0,850.0){\rule[-0.200pt]{2.409pt}{0.400pt}}
\put(170.0,850.0){\rule[-0.200pt]{2.409pt}{0.400pt}}
\put(1429.0,850.0){\rule[-0.200pt]{2.409pt}{0.400pt}}
\put(170.0,850.0){\rule[-0.200pt]{2.409pt}{0.400pt}}
\put(1429.0,850.0){\rule[-0.200pt]{2.409pt}{0.400pt}}
\put(170.0,850.0){\rule[-0.200pt]{2.409pt}{0.400pt}}
\put(1429.0,850.0){\rule[-0.200pt]{2.409pt}{0.400pt}}
\put(170.0,851.0){\rule[-0.200pt]{2.409pt}{0.400pt}}
\put(1429.0,851.0){\rule[-0.200pt]{2.409pt}{0.400pt}}
\put(170.0,851.0){\rule[-0.200pt]{2.409pt}{0.400pt}}
\put(1429.0,851.0){\rule[-0.200pt]{2.409pt}{0.400pt}}
\put(170.0,851.0){\rule[-0.200pt]{2.409pt}{0.400pt}}
\put(1429.0,851.0){\rule[-0.200pt]{2.409pt}{0.400pt}}
\put(170.0,851.0){\rule[-0.200pt]{2.409pt}{0.400pt}}
\put(1429.0,851.0){\rule[-0.200pt]{2.409pt}{0.400pt}}
\put(170.0,851.0){\rule[-0.200pt]{2.409pt}{0.400pt}}
\put(1429.0,851.0){\rule[-0.200pt]{2.409pt}{0.400pt}}
\put(170.0,851.0){\rule[-0.200pt]{2.409pt}{0.400pt}}
\put(1429.0,851.0){\rule[-0.200pt]{2.409pt}{0.400pt}}
\put(170.0,851.0){\rule[-0.200pt]{2.409pt}{0.400pt}}
\put(1429.0,851.0){\rule[-0.200pt]{2.409pt}{0.400pt}}
\put(170.0,851.0){\rule[-0.200pt]{2.409pt}{0.400pt}}
\put(1429.0,851.0){\rule[-0.200pt]{2.409pt}{0.400pt}}
\put(170.0,851.0){\rule[-0.200pt]{2.409pt}{0.400pt}}
\put(1429.0,851.0){\rule[-0.200pt]{2.409pt}{0.400pt}}
\put(170.0,851.0){\rule[-0.200pt]{2.409pt}{0.400pt}}
\put(1429.0,851.0){\rule[-0.200pt]{2.409pt}{0.400pt}}
\put(170.0,851.0){\rule[-0.200pt]{2.409pt}{0.400pt}}
\put(1429.0,851.0){\rule[-0.200pt]{2.409pt}{0.400pt}}
\put(170.0,851.0){\rule[-0.200pt]{2.409pt}{0.400pt}}
\put(1429.0,851.0){\rule[-0.200pt]{2.409pt}{0.400pt}}
\put(170.0,851.0){\rule[-0.200pt]{2.409pt}{0.400pt}}
\put(1429.0,851.0){\rule[-0.200pt]{2.409pt}{0.400pt}}
\put(170.0,851.0){\rule[-0.200pt]{2.409pt}{0.400pt}}
\put(1429.0,851.0){\rule[-0.200pt]{2.409pt}{0.400pt}}
\put(170.0,851.0){\rule[-0.200pt]{2.409pt}{0.400pt}}
\put(1429.0,851.0){\rule[-0.200pt]{2.409pt}{0.400pt}}
\put(170.0,851.0){\rule[-0.200pt]{2.409pt}{0.400pt}}
\put(1429.0,851.0){\rule[-0.200pt]{2.409pt}{0.400pt}}
\put(170.0,851.0){\rule[-0.200pt]{2.409pt}{0.400pt}}
\put(1429.0,851.0){\rule[-0.200pt]{2.409pt}{0.400pt}}
\put(170.0,851.0){\rule[-0.200pt]{2.409pt}{0.400pt}}
\put(1429.0,851.0){\rule[-0.200pt]{2.409pt}{0.400pt}}
\put(170.0,851.0){\rule[-0.200pt]{2.409pt}{0.400pt}}
\put(1429.0,851.0){\rule[-0.200pt]{2.409pt}{0.400pt}}
\put(170.0,851.0){\rule[-0.200pt]{2.409pt}{0.400pt}}
\put(1429.0,851.0){\rule[-0.200pt]{2.409pt}{0.400pt}}
\put(170.0,851.0){\rule[-0.200pt]{2.409pt}{0.400pt}}
\put(1429.0,851.0){\rule[-0.200pt]{2.409pt}{0.400pt}}
\put(170.0,852.0){\rule[-0.200pt]{2.409pt}{0.400pt}}
\put(1429.0,852.0){\rule[-0.200pt]{2.409pt}{0.400pt}}
\put(170.0,852.0){\rule[-0.200pt]{2.409pt}{0.400pt}}
\put(1429.0,852.0){\rule[-0.200pt]{2.409pt}{0.400pt}}
\put(170.0,852.0){\rule[-0.200pt]{2.409pt}{0.400pt}}
\put(1429.0,852.0){\rule[-0.200pt]{2.409pt}{0.400pt}}
\put(170.0,852.0){\rule[-0.200pt]{2.409pt}{0.400pt}}
\put(1429.0,852.0){\rule[-0.200pt]{2.409pt}{0.400pt}}
\put(170.0,852.0){\rule[-0.200pt]{2.409pt}{0.400pt}}
\put(1429.0,852.0){\rule[-0.200pt]{2.409pt}{0.400pt}}
\put(170.0,852.0){\rule[-0.200pt]{2.409pt}{0.400pt}}
\put(1429.0,852.0){\rule[-0.200pt]{2.409pt}{0.400pt}}
\put(170.0,852.0){\rule[-0.200pt]{2.409pt}{0.400pt}}
\put(1429.0,852.0){\rule[-0.200pt]{2.409pt}{0.400pt}}
\put(170.0,852.0){\rule[-0.200pt]{2.409pt}{0.400pt}}
\put(1429.0,852.0){\rule[-0.200pt]{2.409pt}{0.400pt}}
\put(170.0,852.0){\rule[-0.200pt]{2.409pt}{0.400pt}}
\put(1429.0,852.0){\rule[-0.200pt]{2.409pt}{0.400pt}}
\put(170.0,852.0){\rule[-0.200pt]{2.409pt}{0.400pt}}
\put(1429.0,852.0){\rule[-0.200pt]{2.409pt}{0.400pt}}
\put(170.0,852.0){\rule[-0.200pt]{2.409pt}{0.400pt}}
\put(1429.0,852.0){\rule[-0.200pt]{2.409pt}{0.400pt}}
\put(170.0,852.0){\rule[-0.200pt]{2.409pt}{0.400pt}}
\put(1429.0,852.0){\rule[-0.200pt]{2.409pt}{0.400pt}}
\put(170.0,852.0){\rule[-0.200pt]{2.409pt}{0.400pt}}
\put(1429.0,852.0){\rule[-0.200pt]{2.409pt}{0.400pt}}
\put(170.0,852.0){\rule[-0.200pt]{2.409pt}{0.400pt}}
\put(1429.0,852.0){\rule[-0.200pt]{2.409pt}{0.400pt}}
\put(170.0,852.0){\rule[-0.200pt]{2.409pt}{0.400pt}}
\put(1429.0,852.0){\rule[-0.200pt]{2.409pt}{0.400pt}}
\put(170.0,852.0){\rule[-0.200pt]{2.409pt}{0.400pt}}
\put(1429.0,852.0){\rule[-0.200pt]{2.409pt}{0.400pt}}
\put(170.0,852.0){\rule[-0.200pt]{2.409pt}{0.400pt}}
\put(1429.0,852.0){\rule[-0.200pt]{2.409pt}{0.400pt}}
\put(170.0,852.0){\rule[-0.200pt]{2.409pt}{0.400pt}}
\put(1429.0,852.0){\rule[-0.200pt]{2.409pt}{0.400pt}}
\put(170.0,852.0){\rule[-0.200pt]{2.409pt}{0.400pt}}
\put(1429.0,852.0){\rule[-0.200pt]{2.409pt}{0.400pt}}
\put(170.0,852.0){\rule[-0.200pt]{2.409pt}{0.400pt}}
\put(1429.0,852.0){\rule[-0.200pt]{2.409pt}{0.400pt}}
\put(170.0,852.0){\rule[-0.200pt]{2.409pt}{0.400pt}}
\put(1429.0,852.0){\rule[-0.200pt]{2.409pt}{0.400pt}}
\put(170.0,852.0){\rule[-0.200pt]{2.409pt}{0.400pt}}
\put(1429.0,852.0){\rule[-0.200pt]{2.409pt}{0.400pt}}
\put(170.0,853.0){\rule[-0.200pt]{2.409pt}{0.400pt}}
\put(1429.0,853.0){\rule[-0.200pt]{2.409pt}{0.400pt}}
\put(170.0,853.0){\rule[-0.200pt]{2.409pt}{0.400pt}}
\put(1429.0,853.0){\rule[-0.200pt]{2.409pt}{0.400pt}}
\put(170.0,853.0){\rule[-0.200pt]{2.409pt}{0.400pt}}
\put(1429.0,853.0){\rule[-0.200pt]{2.409pt}{0.400pt}}
\put(170.0,853.0){\rule[-0.200pt]{2.409pt}{0.400pt}}
\put(1429.0,853.0){\rule[-0.200pt]{2.409pt}{0.400pt}}
\put(170.0,853.0){\rule[-0.200pt]{2.409pt}{0.400pt}}
\put(1429.0,853.0){\rule[-0.200pt]{2.409pt}{0.400pt}}
\put(170.0,853.0){\rule[-0.200pt]{2.409pt}{0.400pt}}
\put(1429.0,853.0){\rule[-0.200pt]{2.409pt}{0.400pt}}
\put(170.0,853.0){\rule[-0.200pt]{2.409pt}{0.400pt}}
\put(1429.0,853.0){\rule[-0.200pt]{2.409pt}{0.400pt}}
\put(170.0,853.0){\rule[-0.200pt]{2.409pt}{0.400pt}}
\put(1429.0,853.0){\rule[-0.200pt]{2.409pt}{0.400pt}}
\put(170.0,853.0){\rule[-0.200pt]{2.409pt}{0.400pt}}
\put(1429.0,853.0){\rule[-0.200pt]{2.409pt}{0.400pt}}
\put(170.0,853.0){\rule[-0.200pt]{2.409pt}{0.400pt}}
\put(1429.0,853.0){\rule[-0.200pt]{2.409pt}{0.400pt}}
\put(170.0,853.0){\rule[-0.200pt]{2.409pt}{0.400pt}}
\put(1429.0,853.0){\rule[-0.200pt]{2.409pt}{0.400pt}}
\put(170.0,853.0){\rule[-0.200pt]{2.409pt}{0.400pt}}
\put(1429.0,853.0){\rule[-0.200pt]{2.409pt}{0.400pt}}
\put(170.0,853.0){\rule[-0.200pt]{2.409pt}{0.400pt}}
\put(1429.0,853.0){\rule[-0.200pt]{2.409pt}{0.400pt}}
\put(170.0,853.0){\rule[-0.200pt]{2.409pt}{0.400pt}}
\put(1429.0,853.0){\rule[-0.200pt]{2.409pt}{0.400pt}}
\put(170.0,853.0){\rule[-0.200pt]{2.409pt}{0.400pt}}
\put(1429.0,853.0){\rule[-0.200pt]{2.409pt}{0.400pt}}
\put(170.0,853.0){\rule[-0.200pt]{2.409pt}{0.400pt}}
\put(1429.0,853.0){\rule[-0.200pt]{2.409pt}{0.400pt}}
\put(170.0,853.0){\rule[-0.200pt]{2.409pt}{0.400pt}}
\put(1429.0,853.0){\rule[-0.200pt]{2.409pt}{0.400pt}}
\put(170.0,853.0){\rule[-0.200pt]{2.409pt}{0.400pt}}
\put(1429.0,853.0){\rule[-0.200pt]{2.409pt}{0.400pt}}
\put(170.0,853.0){\rule[-0.200pt]{2.409pt}{0.400pt}}
\put(1429.0,853.0){\rule[-0.200pt]{2.409pt}{0.400pt}}
\put(170.0,853.0){\rule[-0.200pt]{2.409pt}{0.400pt}}
\put(1429.0,853.0){\rule[-0.200pt]{2.409pt}{0.400pt}}
\put(170.0,853.0){\rule[-0.200pt]{2.409pt}{0.400pt}}
\put(1429.0,853.0){\rule[-0.200pt]{2.409pt}{0.400pt}}
\put(170.0,853.0){\rule[-0.200pt]{2.409pt}{0.400pt}}
\put(1429.0,853.0){\rule[-0.200pt]{2.409pt}{0.400pt}}
\put(170.0,853.0){\rule[-0.200pt]{2.409pt}{0.400pt}}
\put(1429.0,853.0){\rule[-0.200pt]{2.409pt}{0.400pt}}
\put(170.0,854.0){\rule[-0.200pt]{2.409pt}{0.400pt}}
\put(1429.0,854.0){\rule[-0.200pt]{2.409pt}{0.400pt}}
\put(170.0,854.0){\rule[-0.200pt]{2.409pt}{0.400pt}}
\put(1429.0,854.0){\rule[-0.200pt]{2.409pt}{0.400pt}}
\put(170.0,854.0){\rule[-0.200pt]{2.409pt}{0.400pt}}
\put(1429.0,854.0){\rule[-0.200pt]{2.409pt}{0.400pt}}
\put(170.0,854.0){\rule[-0.200pt]{2.409pt}{0.400pt}}
\put(1429.0,854.0){\rule[-0.200pt]{2.409pt}{0.400pt}}
\put(170.0,854.0){\rule[-0.200pt]{2.409pt}{0.400pt}}
\put(1429.0,854.0){\rule[-0.200pt]{2.409pt}{0.400pt}}
\put(170.0,854.0){\rule[-0.200pt]{2.409pt}{0.400pt}}
\put(1429.0,854.0){\rule[-0.200pt]{2.409pt}{0.400pt}}
\put(170.0,854.0){\rule[-0.200pt]{2.409pt}{0.400pt}}
\put(1429.0,854.0){\rule[-0.200pt]{2.409pt}{0.400pt}}
\put(170.0,854.0){\rule[-0.200pt]{2.409pt}{0.400pt}}
\put(1429.0,854.0){\rule[-0.200pt]{2.409pt}{0.400pt}}
\put(170.0,854.0){\rule[-0.200pt]{2.409pt}{0.400pt}}
\put(1429.0,854.0){\rule[-0.200pt]{2.409pt}{0.400pt}}
\put(170.0,854.0){\rule[-0.200pt]{2.409pt}{0.400pt}}
\put(1429.0,854.0){\rule[-0.200pt]{2.409pt}{0.400pt}}
\put(170.0,854.0){\rule[-0.200pt]{2.409pt}{0.400pt}}
\put(1429.0,854.0){\rule[-0.200pt]{2.409pt}{0.400pt}}
\put(170.0,854.0){\rule[-0.200pt]{2.409pt}{0.400pt}}
\put(1429.0,854.0){\rule[-0.200pt]{2.409pt}{0.400pt}}
\put(170.0,854.0){\rule[-0.200pt]{2.409pt}{0.400pt}}
\put(1429.0,854.0){\rule[-0.200pt]{2.409pt}{0.400pt}}
\put(170.0,854.0){\rule[-0.200pt]{2.409pt}{0.400pt}}
\put(1429.0,854.0){\rule[-0.200pt]{2.409pt}{0.400pt}}
\put(170.0,854.0){\rule[-0.200pt]{2.409pt}{0.400pt}}
\put(1429.0,854.0){\rule[-0.200pt]{2.409pt}{0.400pt}}
\put(170.0,854.0){\rule[-0.200pt]{2.409pt}{0.400pt}}
\put(1429.0,854.0){\rule[-0.200pt]{2.409pt}{0.400pt}}
\put(170.0,854.0){\rule[-0.200pt]{2.409pt}{0.400pt}}
\put(1429.0,854.0){\rule[-0.200pt]{2.409pt}{0.400pt}}
\put(170.0,854.0){\rule[-0.200pt]{2.409pt}{0.400pt}}
\put(1429.0,854.0){\rule[-0.200pt]{2.409pt}{0.400pt}}
\put(170.0,854.0){\rule[-0.200pt]{2.409pt}{0.400pt}}
\put(1429.0,854.0){\rule[-0.200pt]{2.409pt}{0.400pt}}
\put(170.0,854.0){\rule[-0.200pt]{2.409pt}{0.400pt}}
\put(1429.0,854.0){\rule[-0.200pt]{2.409pt}{0.400pt}}
\put(170.0,854.0){\rule[-0.200pt]{2.409pt}{0.400pt}}
\put(1429.0,854.0){\rule[-0.200pt]{2.409pt}{0.400pt}}
\put(170.0,854.0){\rule[-0.200pt]{2.409pt}{0.400pt}}
\put(1429.0,854.0){\rule[-0.200pt]{2.409pt}{0.400pt}}
\put(170.0,854.0){\rule[-0.200pt]{2.409pt}{0.400pt}}
\put(1429.0,854.0){\rule[-0.200pt]{2.409pt}{0.400pt}}
\put(170.0,855.0){\rule[-0.200pt]{2.409pt}{0.400pt}}
\put(1429.0,855.0){\rule[-0.200pt]{2.409pt}{0.400pt}}
\put(170.0,855.0){\rule[-0.200pt]{2.409pt}{0.400pt}}
\put(1429.0,855.0){\rule[-0.200pt]{2.409pt}{0.400pt}}
\put(170.0,855.0){\rule[-0.200pt]{2.409pt}{0.400pt}}
\put(1429.0,855.0){\rule[-0.200pt]{2.409pt}{0.400pt}}
\put(170.0,855.0){\rule[-0.200pt]{2.409pt}{0.400pt}}
\put(1429.0,855.0){\rule[-0.200pt]{2.409pt}{0.400pt}}
\put(170.0,855.0){\rule[-0.200pt]{2.409pt}{0.400pt}}
\put(1429.0,855.0){\rule[-0.200pt]{2.409pt}{0.400pt}}
\put(170.0,855.0){\rule[-0.200pt]{2.409pt}{0.400pt}}
\put(1429.0,855.0){\rule[-0.200pt]{2.409pt}{0.400pt}}
\put(170.0,855.0){\rule[-0.200pt]{2.409pt}{0.400pt}}
\put(1429.0,855.0){\rule[-0.200pt]{2.409pt}{0.400pt}}
\put(170.0,855.0){\rule[-0.200pt]{2.409pt}{0.400pt}}
\put(1429.0,855.0){\rule[-0.200pt]{2.409pt}{0.400pt}}
\put(170.0,855.0){\rule[-0.200pt]{2.409pt}{0.400pt}}
\put(1429.0,855.0){\rule[-0.200pt]{2.409pt}{0.400pt}}
\put(170.0,855.0){\rule[-0.200pt]{2.409pt}{0.400pt}}
\put(1429.0,855.0){\rule[-0.200pt]{2.409pt}{0.400pt}}
\put(170.0,855.0){\rule[-0.200pt]{2.409pt}{0.400pt}}
\put(1429.0,855.0){\rule[-0.200pt]{2.409pt}{0.400pt}}
\put(170.0,855.0){\rule[-0.200pt]{2.409pt}{0.400pt}}
\put(1429.0,855.0){\rule[-0.200pt]{2.409pt}{0.400pt}}
\put(170.0,855.0){\rule[-0.200pt]{2.409pt}{0.400pt}}
\put(1429.0,855.0){\rule[-0.200pt]{2.409pt}{0.400pt}}
\put(170.0,855.0){\rule[-0.200pt]{2.409pt}{0.400pt}}
\put(1429.0,855.0){\rule[-0.200pt]{2.409pt}{0.400pt}}
\put(170.0,855.0){\rule[-0.200pt]{2.409pt}{0.400pt}}
\put(1429.0,855.0){\rule[-0.200pt]{2.409pt}{0.400pt}}
\put(170.0,855.0){\rule[-0.200pt]{2.409pt}{0.400pt}}
\put(1429.0,855.0){\rule[-0.200pt]{2.409pt}{0.400pt}}
\put(170.0,855.0){\rule[-0.200pt]{2.409pt}{0.400pt}}
\put(1429.0,855.0){\rule[-0.200pt]{2.409pt}{0.400pt}}
\put(170.0,855.0){\rule[-0.200pt]{2.409pt}{0.400pt}}
\put(1429.0,855.0){\rule[-0.200pt]{2.409pt}{0.400pt}}
\put(170.0,855.0){\rule[-0.200pt]{2.409pt}{0.400pt}}
\put(1429.0,855.0){\rule[-0.200pt]{2.409pt}{0.400pt}}
\put(170.0,855.0){\rule[-0.200pt]{2.409pt}{0.400pt}}
\put(1429.0,855.0){\rule[-0.200pt]{2.409pt}{0.400pt}}
\put(170.0,855.0){\rule[-0.200pt]{2.409pt}{0.400pt}}
\put(1429.0,855.0){\rule[-0.200pt]{2.409pt}{0.400pt}}
\put(170.0,855.0){\rule[-0.200pt]{2.409pt}{0.400pt}}
\put(1429.0,855.0){\rule[-0.200pt]{2.409pt}{0.400pt}}
\put(170.0,855.0){\rule[-0.200pt]{2.409pt}{0.400pt}}
\put(1429.0,855.0){\rule[-0.200pt]{2.409pt}{0.400pt}}
\put(170.0,855.0){\rule[-0.200pt]{2.409pt}{0.400pt}}
\put(1429.0,855.0){\rule[-0.200pt]{2.409pt}{0.400pt}}
\put(170.0,856.0){\rule[-0.200pt]{2.409pt}{0.400pt}}
\put(1429.0,856.0){\rule[-0.200pt]{2.409pt}{0.400pt}}
\put(170.0,856.0){\rule[-0.200pt]{2.409pt}{0.400pt}}
\put(1429.0,856.0){\rule[-0.200pt]{2.409pt}{0.400pt}}
\put(170.0,856.0){\rule[-0.200pt]{2.409pt}{0.400pt}}
\put(1429.0,856.0){\rule[-0.200pt]{2.409pt}{0.400pt}}
\put(170.0,856.0){\rule[-0.200pt]{2.409pt}{0.400pt}}
\put(1429.0,856.0){\rule[-0.200pt]{2.409pt}{0.400pt}}
\put(170.0,856.0){\rule[-0.200pt]{2.409pt}{0.400pt}}
\put(1429.0,856.0){\rule[-0.200pt]{2.409pt}{0.400pt}}
\put(170.0,856.0){\rule[-0.200pt]{2.409pt}{0.400pt}}
\put(1429.0,856.0){\rule[-0.200pt]{2.409pt}{0.400pt}}
\put(170.0,856.0){\rule[-0.200pt]{2.409pt}{0.400pt}}
\put(1429.0,856.0){\rule[-0.200pt]{2.409pt}{0.400pt}}
\put(170.0,856.0){\rule[-0.200pt]{2.409pt}{0.400pt}}
\put(1429.0,856.0){\rule[-0.200pt]{2.409pt}{0.400pt}}
\put(170.0,856.0){\rule[-0.200pt]{2.409pt}{0.400pt}}
\put(1429.0,856.0){\rule[-0.200pt]{2.409pt}{0.400pt}}
\put(170.0,856.0){\rule[-0.200pt]{2.409pt}{0.400pt}}
\put(1429.0,856.0){\rule[-0.200pt]{2.409pt}{0.400pt}}
\put(170.0,856.0){\rule[-0.200pt]{2.409pt}{0.400pt}}
\put(1429.0,856.0){\rule[-0.200pt]{2.409pt}{0.400pt}}
\put(170.0,856.0){\rule[-0.200pt]{2.409pt}{0.400pt}}
\put(1429.0,856.0){\rule[-0.200pt]{2.409pt}{0.400pt}}
\put(170.0,856.0){\rule[-0.200pt]{2.409pt}{0.400pt}}
\put(1429.0,856.0){\rule[-0.200pt]{2.409pt}{0.400pt}}
\put(170.0,856.0){\rule[-0.200pt]{2.409pt}{0.400pt}}
\put(1429.0,856.0){\rule[-0.200pt]{2.409pt}{0.400pt}}
\put(170.0,856.0){\rule[-0.200pt]{2.409pt}{0.400pt}}
\put(1429.0,856.0){\rule[-0.200pt]{2.409pt}{0.400pt}}
\put(170.0,856.0){\rule[-0.200pt]{2.409pt}{0.400pt}}
\put(1429.0,856.0){\rule[-0.200pt]{2.409pt}{0.400pt}}
\put(170.0,856.0){\rule[-0.200pt]{2.409pt}{0.400pt}}
\put(1429.0,856.0){\rule[-0.200pt]{2.409pt}{0.400pt}}
\put(170.0,856.0){\rule[-0.200pt]{2.409pt}{0.400pt}}
\put(1429.0,856.0){\rule[-0.200pt]{2.409pt}{0.400pt}}
\put(170.0,856.0){\rule[-0.200pt]{2.409pt}{0.400pt}}
\put(1429.0,856.0){\rule[-0.200pt]{2.409pt}{0.400pt}}
\put(170.0,856.0){\rule[-0.200pt]{2.409pt}{0.400pt}}
\put(1429.0,856.0){\rule[-0.200pt]{2.409pt}{0.400pt}}
\put(170.0,856.0){\rule[-0.200pt]{2.409pt}{0.400pt}}
\put(1429.0,856.0){\rule[-0.200pt]{2.409pt}{0.400pt}}
\put(170.0,856.0){\rule[-0.200pt]{2.409pt}{0.400pt}}
\put(1429.0,856.0){\rule[-0.200pt]{2.409pt}{0.400pt}}
\put(170.0,856.0){\rule[-0.200pt]{2.409pt}{0.400pt}}
\put(1429.0,856.0){\rule[-0.200pt]{2.409pt}{0.400pt}}
\put(170.0,856.0){\rule[-0.200pt]{2.409pt}{0.400pt}}
\put(1429.0,856.0){\rule[-0.200pt]{2.409pt}{0.400pt}}
\put(170.0,856.0){\rule[-0.200pt]{2.409pt}{0.400pt}}
\put(1429.0,856.0){\rule[-0.200pt]{2.409pt}{0.400pt}}
\put(170.0,857.0){\rule[-0.200pt]{2.409pt}{0.400pt}}
\put(1429.0,857.0){\rule[-0.200pt]{2.409pt}{0.400pt}}
\put(170.0,857.0){\rule[-0.200pt]{2.409pt}{0.400pt}}
\put(1429.0,857.0){\rule[-0.200pt]{2.409pt}{0.400pt}}
\put(170.0,857.0){\rule[-0.200pt]{2.409pt}{0.400pt}}
\put(1429.0,857.0){\rule[-0.200pt]{2.409pt}{0.400pt}}
\put(170.0,857.0){\rule[-0.200pt]{2.409pt}{0.400pt}}
\put(1429.0,857.0){\rule[-0.200pt]{2.409pt}{0.400pt}}
\put(170.0,857.0){\rule[-0.200pt]{2.409pt}{0.400pt}}
\put(1429.0,857.0){\rule[-0.200pt]{2.409pt}{0.400pt}}
\put(170.0,857.0){\rule[-0.200pt]{2.409pt}{0.400pt}}
\put(1429.0,857.0){\rule[-0.200pt]{2.409pt}{0.400pt}}
\put(170.0,857.0){\rule[-0.200pt]{2.409pt}{0.400pt}}
\put(1429.0,857.0){\rule[-0.200pt]{2.409pt}{0.400pt}}
\put(170.0,857.0){\rule[-0.200pt]{2.409pt}{0.400pt}}
\put(1429.0,857.0){\rule[-0.200pt]{2.409pt}{0.400pt}}
\put(170.0,857.0){\rule[-0.200pt]{2.409pt}{0.400pt}}
\put(1429.0,857.0){\rule[-0.200pt]{2.409pt}{0.400pt}}
\put(170.0,857.0){\rule[-0.200pt]{2.409pt}{0.400pt}}
\put(1429.0,857.0){\rule[-0.200pt]{2.409pt}{0.400pt}}
\put(170.0,857.0){\rule[-0.200pt]{2.409pt}{0.400pt}}
\put(1429.0,857.0){\rule[-0.200pt]{2.409pt}{0.400pt}}
\put(170.0,857.0){\rule[-0.200pt]{2.409pt}{0.400pt}}
\put(1429.0,857.0){\rule[-0.200pt]{2.409pt}{0.400pt}}
\put(170.0,857.0){\rule[-0.200pt]{2.409pt}{0.400pt}}
\put(1429.0,857.0){\rule[-0.200pt]{2.409pt}{0.400pt}}
\put(170.0,857.0){\rule[-0.200pt]{2.409pt}{0.400pt}}
\put(1429.0,857.0){\rule[-0.200pt]{2.409pt}{0.400pt}}
\put(170.0,857.0){\rule[-0.200pt]{2.409pt}{0.400pt}}
\put(1429.0,857.0){\rule[-0.200pt]{2.409pt}{0.400pt}}
\put(170.0,857.0){\rule[-0.200pt]{2.409pt}{0.400pt}}
\put(1429.0,857.0){\rule[-0.200pt]{2.409pt}{0.400pt}}
\put(170.0,857.0){\rule[-0.200pt]{2.409pt}{0.400pt}}
\put(1429.0,857.0){\rule[-0.200pt]{2.409pt}{0.400pt}}
\put(170.0,857.0){\rule[-0.200pt]{2.409pt}{0.400pt}}
\put(1429.0,857.0){\rule[-0.200pt]{2.409pt}{0.400pt}}
\put(170.0,857.0){\rule[-0.200pt]{2.409pt}{0.400pt}}
\put(1429.0,857.0){\rule[-0.200pt]{2.409pt}{0.400pt}}
\put(170.0,857.0){\rule[-0.200pt]{2.409pt}{0.400pt}}
\put(1429.0,857.0){\rule[-0.200pt]{2.409pt}{0.400pt}}
\put(170.0,857.0){\rule[-0.200pt]{2.409pt}{0.400pt}}
\put(1429.0,857.0){\rule[-0.200pt]{2.409pt}{0.400pt}}
\put(170.0,857.0){\rule[-0.200pt]{2.409pt}{0.400pt}}
\put(1429.0,857.0){\rule[-0.200pt]{2.409pt}{0.400pt}}
\put(170.0,857.0){\rule[-0.200pt]{2.409pt}{0.400pt}}
\put(1429.0,857.0){\rule[-0.200pt]{2.409pt}{0.400pt}}
\put(170.0,857.0){\rule[-0.200pt]{2.409pt}{0.400pt}}
\put(1429.0,857.0){\rule[-0.200pt]{2.409pt}{0.400pt}}
\put(170.0,857.0){\rule[-0.200pt]{2.409pt}{0.400pt}}
\put(1429.0,857.0){\rule[-0.200pt]{2.409pt}{0.400pt}}
\put(170.0,858.0){\rule[-0.200pt]{2.409pt}{0.400pt}}
\put(1429.0,858.0){\rule[-0.200pt]{2.409pt}{0.400pt}}
\put(170.0,858.0){\rule[-0.200pt]{2.409pt}{0.400pt}}
\put(1429.0,858.0){\rule[-0.200pt]{2.409pt}{0.400pt}}
\put(170.0,858.0){\rule[-0.200pt]{2.409pt}{0.400pt}}
\put(1429.0,858.0){\rule[-0.200pt]{2.409pt}{0.400pt}}
\put(170.0,858.0){\rule[-0.200pt]{2.409pt}{0.400pt}}
\put(1429.0,858.0){\rule[-0.200pt]{2.409pt}{0.400pt}}
\put(170.0,858.0){\rule[-0.200pt]{2.409pt}{0.400pt}}
\put(1429.0,858.0){\rule[-0.200pt]{2.409pt}{0.400pt}}
\put(170.0,858.0){\rule[-0.200pt]{2.409pt}{0.400pt}}
\put(1429.0,858.0){\rule[-0.200pt]{2.409pt}{0.400pt}}
\put(170.0,858.0){\rule[-0.200pt]{2.409pt}{0.400pt}}
\put(1429.0,858.0){\rule[-0.200pt]{2.409pt}{0.400pt}}
\put(170.0,858.0){\rule[-0.200pt]{2.409pt}{0.400pt}}
\put(1429.0,858.0){\rule[-0.200pt]{2.409pt}{0.400pt}}
\put(170.0,858.0){\rule[-0.200pt]{2.409pt}{0.400pt}}
\put(1429.0,858.0){\rule[-0.200pt]{2.409pt}{0.400pt}}
\put(170.0,858.0){\rule[-0.200pt]{2.409pt}{0.400pt}}
\put(1429.0,858.0){\rule[-0.200pt]{2.409pt}{0.400pt}}
\put(170.0,858.0){\rule[-0.200pt]{2.409pt}{0.400pt}}
\put(1429.0,858.0){\rule[-0.200pt]{2.409pt}{0.400pt}}
\put(170.0,858.0){\rule[-0.200pt]{2.409pt}{0.400pt}}
\put(1429.0,858.0){\rule[-0.200pt]{2.409pt}{0.400pt}}
\put(170.0,858.0){\rule[-0.200pt]{2.409pt}{0.400pt}}
\put(1429.0,858.0){\rule[-0.200pt]{2.409pt}{0.400pt}}
\put(170.0,858.0){\rule[-0.200pt]{2.409pt}{0.400pt}}
\put(1429.0,858.0){\rule[-0.200pt]{2.409pt}{0.400pt}}
\put(170.0,858.0){\rule[-0.200pt]{2.409pt}{0.400pt}}
\put(1429.0,858.0){\rule[-0.200pt]{2.409pt}{0.400pt}}
\put(170.0,858.0){\rule[-0.200pt]{2.409pt}{0.400pt}}
\put(1429.0,858.0){\rule[-0.200pt]{2.409pt}{0.400pt}}
\put(170.0,858.0){\rule[-0.200pt]{2.409pt}{0.400pt}}
\put(1429.0,858.0){\rule[-0.200pt]{2.409pt}{0.400pt}}
\put(170.0,858.0){\rule[-0.200pt]{2.409pt}{0.400pt}}
\put(1429.0,858.0){\rule[-0.200pt]{2.409pt}{0.400pt}}
\put(170.0,858.0){\rule[-0.200pt]{2.409pt}{0.400pt}}
\put(1429.0,858.0){\rule[-0.200pt]{2.409pt}{0.400pt}}
\put(170.0,858.0){\rule[-0.200pt]{2.409pt}{0.400pt}}
\put(1429.0,858.0){\rule[-0.200pt]{2.409pt}{0.400pt}}
\put(170.0,858.0){\rule[-0.200pt]{2.409pt}{0.400pt}}
\put(1429.0,858.0){\rule[-0.200pt]{2.409pt}{0.400pt}}
\put(170.0,858.0){\rule[-0.200pt]{2.409pt}{0.400pt}}
\put(1429.0,858.0){\rule[-0.200pt]{2.409pt}{0.400pt}}
\put(170.0,858.0){\rule[-0.200pt]{2.409pt}{0.400pt}}
\put(1429.0,858.0){\rule[-0.200pt]{2.409pt}{0.400pt}}
\put(170.0,858.0){\rule[-0.200pt]{2.409pt}{0.400pt}}
\put(1429.0,858.0){\rule[-0.200pt]{2.409pt}{0.400pt}}
\put(170.0,858.0){\rule[-0.200pt]{2.409pt}{0.400pt}}
\put(1429.0,858.0){\rule[-0.200pt]{2.409pt}{0.400pt}}
\put(170.0,858.0){\rule[-0.200pt]{2.409pt}{0.400pt}}
\put(1429.0,858.0){\rule[-0.200pt]{2.409pt}{0.400pt}}
\put(170.0,859.0){\rule[-0.200pt]{2.409pt}{0.400pt}}
\put(1429.0,859.0){\rule[-0.200pt]{2.409pt}{0.400pt}}
\put(170.0,859.0){\rule[-0.200pt]{2.409pt}{0.400pt}}
\put(1429.0,859.0){\rule[-0.200pt]{2.409pt}{0.400pt}}
\put(170.0,859.0){\rule[-0.200pt]{2.409pt}{0.400pt}}
\put(1429.0,859.0){\rule[-0.200pt]{2.409pt}{0.400pt}}
\put(170.0,859.0){\rule[-0.200pt]{2.409pt}{0.400pt}}
\put(1429.0,859.0){\rule[-0.200pt]{2.409pt}{0.400pt}}
\put(170.0,859.0){\rule[-0.200pt]{2.409pt}{0.400pt}}
\put(1429.0,859.0){\rule[-0.200pt]{2.409pt}{0.400pt}}
\put(170.0,859.0){\rule[-0.200pt]{2.409pt}{0.400pt}}
\put(1429.0,859.0){\rule[-0.200pt]{2.409pt}{0.400pt}}
\put(170.0,859.0){\rule[-0.200pt]{2.409pt}{0.400pt}}
\put(1429.0,859.0){\rule[-0.200pt]{2.409pt}{0.400pt}}
\put(170.0,859.0){\rule[-0.200pt]{2.409pt}{0.400pt}}
\put(1429.0,859.0){\rule[-0.200pt]{2.409pt}{0.400pt}}
\put(170.0,859.0){\rule[-0.200pt]{2.409pt}{0.400pt}}
\put(1429.0,859.0){\rule[-0.200pt]{2.409pt}{0.400pt}}
\put(170.0,859.0){\rule[-0.200pt]{2.409pt}{0.400pt}}
\put(1429.0,859.0){\rule[-0.200pt]{2.409pt}{0.400pt}}
\put(170.0,859.0){\rule[-0.200pt]{2.409pt}{0.400pt}}
\put(1429.0,859.0){\rule[-0.200pt]{2.409pt}{0.400pt}}
\put(170.0,859.0){\rule[-0.200pt]{2.409pt}{0.400pt}}
\put(1429.0,859.0){\rule[-0.200pt]{2.409pt}{0.400pt}}
\put(170.0,859.0){\rule[-0.200pt]{2.409pt}{0.400pt}}
\put(1429.0,859.0){\rule[-0.200pt]{2.409pt}{0.400pt}}
\put(170.0,859.0){\rule[-0.200pt]{4.818pt}{0.400pt}}
\put(150,859){\makebox(0,0)[r]{ 1e+09}}
\put(1419.0,859.0){\rule[-0.200pt]{4.818pt}{0.400pt}}
\put(170.0,82.0){\rule[-0.200pt]{0.400pt}{4.818pt}}
\put(170,41){\makebox(0,0){ 0}}
\put(170.0,839.0){\rule[-0.200pt]{0.400pt}{4.818pt}}
\put(424.0,82.0){\rule[-0.200pt]{0.400pt}{4.818pt}}
\put(424,41){\makebox(0,0){ 100}}
\put(424.0,839.0){\rule[-0.200pt]{0.400pt}{4.818pt}}
\put(678.0,82.0){\rule[-0.200pt]{0.400pt}{4.818pt}}
\put(678,41){\makebox(0,0){ 200}}
\put(678.0,839.0){\rule[-0.200pt]{0.400pt}{4.818pt}}
\put(931.0,82.0){\rule[-0.200pt]{0.400pt}{4.818pt}}
\put(931,41){\makebox(0,0){ 300}}
\put(931.0,839.0){\rule[-0.200pt]{0.400pt}{4.818pt}}
\put(1185.0,82.0){\rule[-0.200pt]{0.400pt}{4.818pt}}
\put(1185,41){\makebox(0,0){ 400}}
\put(1185.0,839.0){\rule[-0.200pt]{0.400pt}{4.818pt}}
\put(1439.0,82.0){\rule[-0.200pt]{0.400pt}{4.818pt}}
\put(1439,41){\makebox(0,0){ 500}}
\put(1439.0,839.0){\rule[-0.200pt]{0.400pt}{4.818pt}}
\put(170.0,82.0){\rule[-0.200pt]{0.400pt}{187.179pt}}
\put(170.0,82.0){\rule[-0.200pt]{305.702pt}{0.400pt}}
\put(1439.0,82.0){\rule[-0.200pt]{0.400pt}{187.179pt}}
\put(170.0,859.0){\rule[-0.200pt]{305.702pt}{0.400pt}}
\put(1279,164){\makebox(0,0)[r]{algorytm naturalny}}
\put(1299.0,164.0){\rule[-0.200pt]{24.090pt}{0.400pt}}
\put(175,82){\usebox{\plotpoint}}
\multiput(175.58,82.00)(0.493,1.091){23}{\rule{0.119pt}{0.962pt}}
\multiput(174.17,82.00)(13.000,26.004){2}{\rule{0.400pt}{0.481pt}}
\multiput(188.58,110.00)(0.493,0.853){23}{\rule{0.119pt}{0.777pt}}
\multiput(187.17,110.00)(13.000,20.387){2}{\rule{0.400pt}{0.388pt}}
\multiput(201.58,132.00)(0.493,0.734){23}{\rule{0.119pt}{0.685pt}}
\multiput(200.17,132.00)(13.000,17.579){2}{\rule{0.400pt}{0.342pt}}
\multiput(214.58,151.00)(0.493,0.734){23}{\rule{0.119pt}{0.685pt}}
\multiput(213.17,151.00)(13.000,17.579){2}{\rule{0.400pt}{0.342pt}}
\multiput(227.58,170.00)(0.493,0.734){23}{\rule{0.119pt}{0.685pt}}
\multiput(226.17,170.00)(13.000,17.579){2}{\rule{0.400pt}{0.342pt}}
\multiput(240.58,189.00)(0.493,0.695){23}{\rule{0.119pt}{0.654pt}}
\multiput(239.17,189.00)(13.000,16.643){2}{\rule{0.400pt}{0.327pt}}
\multiput(253.58,207.00)(0.493,0.655){23}{\rule{0.119pt}{0.623pt}}
\multiput(252.17,207.00)(13.000,15.707){2}{\rule{0.400pt}{0.312pt}}
\multiput(266.58,224.00)(0.493,0.536){23}{\rule{0.119pt}{0.531pt}}
\multiput(265.17,224.00)(13.000,12.898){2}{\rule{0.400pt}{0.265pt}}
\multiput(279.00,238.58)(0.539,0.492){21}{\rule{0.533pt}{0.119pt}}
\multiput(279.00,237.17)(11.893,12.000){2}{\rule{0.267pt}{0.400pt}}
\multiput(292.00,250.58)(0.539,0.492){21}{\rule{0.533pt}{0.119pt}}
\multiput(292.00,249.17)(11.893,12.000){2}{\rule{0.267pt}{0.400pt}}
\multiput(305.00,262.58)(0.652,0.491){17}{\rule{0.620pt}{0.118pt}}
\multiput(305.00,261.17)(11.713,10.000){2}{\rule{0.310pt}{0.400pt}}
\multiput(318.00,272.59)(0.728,0.489){15}{\rule{0.678pt}{0.118pt}}
\multiput(318.00,271.17)(11.593,9.000){2}{\rule{0.339pt}{0.400pt}}
\multiput(331.00,281.59)(0.728,0.489){15}{\rule{0.678pt}{0.118pt}}
\multiput(331.00,280.17)(11.593,9.000){2}{\rule{0.339pt}{0.400pt}}
\multiput(344.00,290.59)(0.824,0.488){13}{\rule{0.750pt}{0.117pt}}
\multiput(344.00,289.17)(11.443,8.000){2}{\rule{0.375pt}{0.400pt}}
\multiput(357.00,298.59)(0.824,0.488){13}{\rule{0.750pt}{0.117pt}}
\multiput(357.00,297.17)(11.443,8.000){2}{\rule{0.375pt}{0.400pt}}
\multiput(370.00,306.59)(0.950,0.485){11}{\rule{0.843pt}{0.117pt}}
\multiput(370.00,305.17)(11.251,7.000){2}{\rule{0.421pt}{0.400pt}}
\multiput(383.00,313.59)(1.123,0.482){9}{\rule{0.967pt}{0.116pt}}
\multiput(383.00,312.17)(10.994,6.000){2}{\rule{0.483pt}{0.400pt}}
\multiput(396.00,319.59)(0.950,0.485){11}{\rule{0.843pt}{0.117pt}}
\multiput(396.00,318.17)(11.251,7.000){2}{\rule{0.421pt}{0.400pt}}
\multiput(409.00,326.59)(1.214,0.482){9}{\rule{1.033pt}{0.116pt}}
\multiput(409.00,325.17)(11.855,6.000){2}{\rule{0.517pt}{0.400pt}}
\multiput(423.00,332.59)(1.378,0.477){7}{\rule{1.140pt}{0.115pt}}
\multiput(423.00,331.17)(10.634,5.000){2}{\rule{0.570pt}{0.400pt}}
\multiput(436.00,337.59)(1.123,0.482){9}{\rule{0.967pt}{0.116pt}}
\multiput(436.00,336.17)(10.994,6.000){2}{\rule{0.483pt}{0.400pt}}
\multiput(449.00,343.59)(1.378,0.477){7}{\rule{1.140pt}{0.115pt}}
\multiput(449.00,342.17)(10.634,5.000){2}{\rule{0.570pt}{0.400pt}}
\multiput(462.00,348.59)(1.378,0.477){7}{\rule{1.140pt}{0.115pt}}
\multiput(462.00,347.17)(10.634,5.000){2}{\rule{0.570pt}{0.400pt}}
\multiput(475.00,353.59)(1.378,0.477){7}{\rule{1.140pt}{0.115pt}}
\multiput(475.00,352.17)(10.634,5.000){2}{\rule{0.570pt}{0.400pt}}
\multiput(488.00,358.60)(1.797,0.468){5}{\rule{1.400pt}{0.113pt}}
\multiput(488.00,357.17)(10.094,4.000){2}{\rule{0.700pt}{0.400pt}}
\multiput(501.00,362.59)(1.378,0.477){7}{\rule{1.140pt}{0.115pt}}
\multiput(501.00,361.17)(10.634,5.000){2}{\rule{0.570pt}{0.400pt}}
\multiput(514.00,367.60)(1.797,0.468){5}{\rule{1.400pt}{0.113pt}}
\multiput(514.00,366.17)(10.094,4.000){2}{\rule{0.700pt}{0.400pt}}
\multiput(527.00,371.60)(1.797,0.468){5}{\rule{1.400pt}{0.113pt}}
\multiput(527.00,370.17)(10.094,4.000){2}{\rule{0.700pt}{0.400pt}}
\multiput(540.00,375.60)(1.797,0.468){5}{\rule{1.400pt}{0.113pt}}
\multiput(540.00,374.17)(10.094,4.000){2}{\rule{0.700pt}{0.400pt}}
\multiput(553.00,379.60)(1.797,0.468){5}{\rule{1.400pt}{0.113pt}}
\multiput(553.00,378.17)(10.094,4.000){2}{\rule{0.700pt}{0.400pt}}
\multiput(566.00,383.61)(2.695,0.447){3}{\rule{1.833pt}{0.108pt}}
\multiput(566.00,382.17)(9.195,3.000){2}{\rule{0.917pt}{0.400pt}}
\multiput(579.00,386.60)(1.797,0.468){5}{\rule{1.400pt}{0.113pt}}
\multiput(579.00,385.17)(10.094,4.000){2}{\rule{0.700pt}{0.400pt}}
\multiput(592.00,390.61)(2.695,0.447){3}{\rule{1.833pt}{0.108pt}}
\multiput(592.00,389.17)(9.195,3.000){2}{\rule{0.917pt}{0.400pt}}
\multiput(605.00,393.60)(1.797,0.468){5}{\rule{1.400pt}{0.113pt}}
\multiput(605.00,392.17)(10.094,4.000){2}{\rule{0.700pt}{0.400pt}}
\multiput(618.00,397.61)(2.695,0.447){3}{\rule{1.833pt}{0.108pt}}
\multiput(618.00,396.17)(9.195,3.000){2}{\rule{0.917pt}{0.400pt}}
\multiput(631.00,400.61)(2.695,0.447){3}{\rule{1.833pt}{0.108pt}}
\multiput(631.00,399.17)(9.195,3.000){2}{\rule{0.917pt}{0.400pt}}
\multiput(644.00,403.61)(2.695,0.447){3}{\rule{1.833pt}{0.108pt}}
\multiput(644.00,402.17)(9.195,3.000){2}{\rule{0.917pt}{0.400pt}}
\multiput(657.00,406.61)(2.695,0.447){3}{\rule{1.833pt}{0.108pt}}
\multiput(657.00,405.17)(9.195,3.000){2}{\rule{0.917pt}{0.400pt}}
\multiput(670.00,409.61)(2.695,0.447){3}{\rule{1.833pt}{0.108pt}}
\multiput(670.00,408.17)(9.195,3.000){2}{\rule{0.917pt}{0.400pt}}
\multiput(683.00,412.61)(2.695,0.447){3}{\rule{1.833pt}{0.108pt}}
\multiput(683.00,411.17)(9.195,3.000){2}{\rule{0.917pt}{0.400pt}}
\multiput(696.00,415.61)(2.695,0.447){3}{\rule{1.833pt}{0.108pt}}
\multiput(696.00,414.17)(9.195,3.000){2}{\rule{0.917pt}{0.400pt}}
\multiput(709.00,418.61)(2.695,0.447){3}{\rule{1.833pt}{0.108pt}}
\multiput(709.00,417.17)(9.195,3.000){2}{\rule{0.917pt}{0.400pt}}
\put(722,421.17){\rule{2.700pt}{0.400pt}}
\multiput(722.00,420.17)(7.396,2.000){2}{\rule{1.350pt}{0.400pt}}
\multiput(735.00,423.61)(2.695,0.447){3}{\rule{1.833pt}{0.108pt}}
\multiput(735.00,422.17)(9.195,3.000){2}{\rule{0.917pt}{0.400pt}}
\multiput(748.00,426.61)(2.695,0.447){3}{\rule{1.833pt}{0.108pt}}
\multiput(748.00,425.17)(9.195,3.000){2}{\rule{0.917pt}{0.400pt}}
\put(761,429.17){\rule{2.700pt}{0.400pt}}
\multiput(761.00,428.17)(7.396,2.000){2}{\rule{1.350pt}{0.400pt}}
\multiput(774.00,431.61)(2.695,0.447){3}{\rule{1.833pt}{0.108pt}}
\multiput(774.00,430.17)(9.195,3.000){2}{\rule{0.917pt}{0.400pt}}
\put(787,434.17){\rule{2.700pt}{0.400pt}}
\multiput(787.00,433.17)(7.396,2.000){2}{\rule{1.350pt}{0.400pt}}
\multiput(800.00,436.61)(2.695,0.447){3}{\rule{1.833pt}{0.108pt}}
\multiput(800.00,435.17)(9.195,3.000){2}{\rule{0.917pt}{0.400pt}}
\put(813,439.17){\rule{2.700pt}{0.400pt}}
\multiput(813.00,438.17)(7.396,2.000){2}{\rule{1.350pt}{0.400pt}}
\multiput(826.00,441.61)(2.695,0.447){3}{\rule{1.833pt}{0.108pt}}
\multiput(826.00,440.17)(9.195,3.000){2}{\rule{0.917pt}{0.400pt}}
\put(839,444.17){\rule{2.700pt}{0.400pt}}
\multiput(839.00,443.17)(7.396,2.000){2}{\rule{1.350pt}{0.400pt}}
\put(852,446.17){\rule{2.700pt}{0.400pt}}
\multiput(852.00,445.17)(7.396,2.000){2}{\rule{1.350pt}{0.400pt}}
\multiput(865.00,448.61)(2.695,0.447){3}{\rule{1.833pt}{0.108pt}}
\multiput(865.00,447.17)(9.195,3.000){2}{\rule{0.917pt}{0.400pt}}
\put(878,451.17){\rule{2.700pt}{0.400pt}}
\multiput(878.00,450.17)(7.396,2.000){2}{\rule{1.350pt}{0.400pt}}
\put(891,453.17){\rule{2.700pt}{0.400pt}}
\multiput(891.00,452.17)(7.396,2.000){2}{\rule{1.350pt}{0.400pt}}
\put(904,455.17){\rule{2.700pt}{0.400pt}}
\multiput(904.00,454.17)(7.396,2.000){2}{\rule{1.350pt}{0.400pt}}
\put(917,457.17){\rule{2.700pt}{0.400pt}}
\multiput(917.00,456.17)(7.396,2.000){2}{\rule{1.350pt}{0.400pt}}
\put(930,459.17){\rule{2.700pt}{0.400pt}}
\multiput(930.00,458.17)(7.396,2.000){2}{\rule{1.350pt}{0.400pt}}
\put(943,461.17){\rule{2.700pt}{0.400pt}}
\multiput(943.00,460.17)(7.396,2.000){2}{\rule{1.350pt}{0.400pt}}
\put(956,463.17){\rule{2.700pt}{0.400pt}}
\multiput(956.00,462.17)(7.396,2.000){2}{\rule{1.350pt}{0.400pt}}
\put(969,465.17){\rule{2.900pt}{0.400pt}}
\multiput(969.00,464.17)(7.981,2.000){2}{\rule{1.450pt}{0.400pt}}
\put(983,467.17){\rule{2.700pt}{0.400pt}}
\multiput(983.00,466.17)(7.396,2.000){2}{\rule{1.350pt}{0.400pt}}
\put(996,469.17){\rule{2.700pt}{0.400pt}}
\multiput(996.00,468.17)(7.396,2.000){2}{\rule{1.350pt}{0.400pt}}
\put(1009,471.17){\rule{2.700pt}{0.400pt}}
\multiput(1009.00,470.17)(7.396,2.000){2}{\rule{1.350pt}{0.400pt}}
\put(1022,473.17){\rule{2.700pt}{0.400pt}}
\multiput(1022.00,472.17)(7.396,2.000){2}{\rule{1.350pt}{0.400pt}}
\put(1035,475.17){\rule{2.700pt}{0.400pt}}
\multiput(1035.00,474.17)(7.396,2.000){2}{\rule{1.350pt}{0.400pt}}
\put(1048,476.67){\rule{3.132pt}{0.400pt}}
\multiput(1048.00,476.17)(6.500,1.000){2}{\rule{1.566pt}{0.400pt}}
\put(1061,478.17){\rule{2.700pt}{0.400pt}}
\multiput(1061.00,477.17)(7.396,2.000){2}{\rule{1.350pt}{0.400pt}}
\put(1074,480.17){\rule{2.700pt}{0.400pt}}
\multiput(1074.00,479.17)(7.396,2.000){2}{\rule{1.350pt}{0.400pt}}
\put(1087,481.67){\rule{3.132pt}{0.400pt}}
\multiput(1087.00,481.17)(6.500,1.000){2}{\rule{1.566pt}{0.400pt}}
\put(1100,483.17){\rule{2.700pt}{0.400pt}}
\multiput(1100.00,482.17)(7.396,2.000){2}{\rule{1.350pt}{0.400pt}}
\put(1113,485.17){\rule{2.700pt}{0.400pt}}
\multiput(1113.00,484.17)(7.396,2.000){2}{\rule{1.350pt}{0.400pt}}
\put(1126,486.67){\rule{3.132pt}{0.400pt}}
\multiput(1126.00,486.17)(6.500,1.000){2}{\rule{1.566pt}{0.400pt}}
\put(1139,488.17){\rule{2.700pt}{0.400pt}}
\multiput(1139.00,487.17)(7.396,2.000){2}{\rule{1.350pt}{0.400pt}}
\put(1152,490.17){\rule{2.700pt}{0.400pt}}
\multiput(1152.00,489.17)(7.396,2.000){2}{\rule{1.350pt}{0.400pt}}
\put(1165,491.67){\rule{3.132pt}{0.400pt}}
\multiput(1165.00,491.17)(6.500,1.000){2}{\rule{1.566pt}{0.400pt}}
\put(1178,493.17){\rule{2.700pt}{0.400pt}}
\multiput(1178.00,492.17)(7.396,2.000){2}{\rule{1.350pt}{0.400pt}}
\put(1191,494.67){\rule{3.132pt}{0.400pt}}
\multiput(1191.00,494.17)(6.500,1.000){2}{\rule{1.566pt}{0.400pt}}
\put(1204,496.17){\rule{2.700pt}{0.400pt}}
\multiput(1204.00,495.17)(7.396,2.000){2}{\rule{1.350pt}{0.400pt}}
\put(1217,497.67){\rule{3.132pt}{0.400pt}}
\multiput(1217.00,497.17)(6.500,1.000){2}{\rule{1.566pt}{0.400pt}}
\put(1230,498.67){\rule{3.132pt}{0.400pt}}
\multiput(1230.00,498.17)(6.500,1.000){2}{\rule{1.566pt}{0.400pt}}
\put(1243,500.17){\rule{2.700pt}{0.400pt}}
\multiput(1243.00,499.17)(7.396,2.000){2}{\rule{1.350pt}{0.400pt}}
\put(1256,501.67){\rule{3.132pt}{0.400pt}}
\multiput(1256.00,501.17)(6.500,1.000){2}{\rule{1.566pt}{0.400pt}}
\put(1269,502.67){\rule{3.132pt}{0.400pt}}
\multiput(1269.00,502.17)(6.500,1.000){2}{\rule{1.566pt}{0.400pt}}
\put(1282,504.17){\rule{2.700pt}{0.400pt}}
\multiput(1282.00,503.17)(7.396,2.000){2}{\rule{1.350pt}{0.400pt}}
\put(1295,505.67){\rule{3.132pt}{0.400pt}}
\multiput(1295.00,505.17)(6.500,1.000){2}{\rule{1.566pt}{0.400pt}}
\put(1308,507.17){\rule{2.700pt}{0.400pt}}
\multiput(1308.00,506.17)(7.396,2.000){2}{\rule{1.350pt}{0.400pt}}
\put(1321,508.67){\rule{3.132pt}{0.400pt}}
\multiput(1321.00,508.17)(6.500,1.000){2}{\rule{1.566pt}{0.400pt}}
\put(1334,509.67){\rule{3.132pt}{0.400pt}}
\multiput(1334.00,509.17)(6.500,1.000){2}{\rule{1.566pt}{0.400pt}}
\put(1347,511.17){\rule{2.700pt}{0.400pt}}
\multiput(1347.00,510.17)(7.396,2.000){2}{\rule{1.350pt}{0.400pt}}
\put(1360,513.17){\rule{2.700pt}{0.400pt}}
\multiput(1360.00,512.17)(7.396,2.000){2}{\rule{1.350pt}{0.400pt}}
\put(1373,514.67){\rule{3.132pt}{0.400pt}}
\multiput(1373.00,514.17)(6.500,1.000){2}{\rule{1.566pt}{0.400pt}}
\put(1386,515.67){\rule{3.132pt}{0.400pt}}
\multiput(1386.00,515.17)(6.500,1.000){2}{\rule{1.566pt}{0.400pt}}
\put(1399,516.67){\rule{3.132pt}{0.400pt}}
\multiput(1399.00,516.17)(6.500,1.000){2}{\rule{1.566pt}{0.400pt}}
\put(1412,518.17){\rule{2.700pt}{0.400pt}}
\multiput(1412.00,517.17)(7.396,2.000){2}{\rule{1.350pt}{0.400pt}}
\put(1425,519.67){\rule{3.132pt}{0.400pt}}
\multiput(1425.00,519.17)(6.500,1.000){2}{\rule{1.566pt}{0.400pt}}
\put(1438.0,521.0){\usebox{\plotpoint}}
\sbox{\plotpoint}{\rule[-0.500pt]{1.000pt}{1.000pt}}%
\sbox{\plotpoint}{\rule[-0.200pt]{0.400pt}{0.400pt}}%
\put(1279,123){\makebox(0,0)[r]{algorytm Strassena}}
\sbox{\plotpoint}{\rule[-0.500pt]{1.000pt}{1.000pt}}%
\multiput(1299,123)(20.756,0.000){5}{\usebox{\plotpoint}}
\put(1399,123){\usebox{\plotpoint}}
\put(175,236){\usebox{\plotpoint}}
\multiput(175,236)(2.235,20.635){6}{\usebox{\plotpoint}}
\multiput(188,356)(4.466,20.269){3}{\usebox{\plotpoint}}
\multiput(201,415)(7.413,19.387){2}{\usebox{\plotpoint}}
\put(218.83,458.65){\usebox{\plotpoint}}
\multiput(227,475)(10.925,17.648){2}{\usebox{\plotpoint}}
\put(250.89,511.92){\usebox{\plotpoint}}
\put(263.74,528.21){\usebox{\plotpoint}}
\put(278.14,543.14){\usebox{\plotpoint}}
\put(293.99,556.53){\usebox{\plotpoint}}
\put(310.04,569.65){\usebox{\plotpoint}}
\put(325.58,583.41){\usebox{\plotpoint}}
\put(341.82,596.32){\usebox{\plotpoint}}
\put(358.93,608.04){\usebox{\plotpoint}}
\put(377.64,616.94){\usebox{\plotpoint}}
\put(397.37,623.32){\usebox{\plotpoint}}
\put(417.48,628.42){\usebox{\plotpoint}}
\put(437.57,633.60){\usebox{\plotpoint}}
\put(456.72,641.56){\usebox{\plotpoint}}
\put(475.15,651.09){\usebox{\plotpoint}}
\put(492.99,661.69){\usebox{\plotpoint}}
\put(510.93,672.11){\usebox{\plotpoint}}
\put(529.23,681.86){\usebox{\plotpoint}}
\put(548.60,689.31){\usebox{\plotpoint}}
\put(568.67,694.41){\usebox{\plotpoint}}
\put(589.18,697.57){\usebox{\plotpoint}}
\put(609.85,699.37){\usebox{\plotpoint}}
\put(630.58,700.00){\usebox{\plotpoint}}
\put(651.34,700.00){\usebox{\plotpoint}}
\put(672.06,701.00){\usebox{\plotpoint}}
\put(692.81,701.00){\usebox{\plotpoint}}
\put(713.48,702.69){\usebox{\plotpoint}}
\put(734.09,704.93){\usebox{\plotpoint}}
\put(754.34,709.46){\usebox{\plotpoint}}
\put(773.99,716.00){\usebox{\plotpoint}}
\put(793.19,723.86){\usebox{\plotpoint}}
\put(812.04,732.55){\usebox{\plotpoint}}
\put(830.48,742.07){\usebox{\plotpoint}}
\put(849.32,750.76){\usebox{\plotpoint}}
\put(868.62,758.39){\usebox{\plotpoint}}
\put(888.43,764.41){\usebox{\plotpoint}}
\put(908.72,768.73){\usebox{\plotpoint}}
\put(929.34,770.95){\usebox{\plotpoint}}
\put(950.06,772.00){\usebox{\plotpoint}}
\put(970.81,772.00){\usebox{\plotpoint}}
\put(991.54,772.66){\usebox{\plotpoint}}
\put(1012.28,773.00){\usebox{\plotpoint}}
\put(1033.04,773.00){\usebox{\plotpoint}}
\put(1053.80,773.00){\usebox{\plotpoint}}
\put(1074.55,773.00){\usebox{\plotpoint}}
\put(1095.31,773.00){\usebox{\plotpoint}}
\put(1116.06,773.00){\usebox{\plotpoint}}
\put(1136.82,773.00){\usebox{\plotpoint}}
\put(1157.57,773.00){\usebox{\plotpoint}}
\put(1178.33,773.00){\usebox{\plotpoint}}
\put(1199.08,773.00){\usebox{\plotpoint}}
\put(1219.84,773.00){\usebox{\plotpoint}}
\put(1240.59,773.00){\usebox{\plotpoint}}
\put(1261.35,773.00){\usebox{\plotpoint}}
\put(1282.11,773.00){\usebox{\plotpoint}}
\put(1302.86,773.00){\usebox{\plotpoint}}
\put(1323.62,773.00){\usebox{\plotpoint}}
\put(1344.37,773.00){\usebox{\plotpoint}}
\put(1365.13,773.00){\usebox{\plotpoint}}
\put(1385.88,773.00){\usebox{\plotpoint}}
\put(1406.64,773.00){\usebox{\plotpoint}}
\put(1427.39,773.00){\usebox{\plotpoint}}
\put(1439,773){\usebox{\plotpoint}}
\sbox{\plotpoint}{\rule[-0.200pt]{0.400pt}{0.400pt}}%
\put(170.0,82.0){\rule[-0.200pt]{0.400pt}{187.179pt}}
\put(170.0,82.0){\rule[-0.200pt]{305.702pt}{0.400pt}}
\put(1439.0,82.0){\rule[-0.200pt]{0.400pt}{187.179pt}}
\put(170.0,859.0){\rule[-0.200pt]{305.702pt}{0.400pt}}
\end{picture}

\caption{Zależność czasu działania od wielkości macierzy dla metody naturalnej i Strassena}
\end{center}
\end{figure}
\begin{figure}[h!tb]
\begin{center}
% GNUPLOT: LaTeX picture
\setlength{\unitlength}{0.240900pt}
\ifx\plotpoint\undefined\newsavebox{\plotpoint}\fi
\sbox{\plotpoint}{\rule[-0.200pt]{0.400pt}{0.400pt}}%
\begin{picture}(1500,900)(0,0)
\sbox{\plotpoint}{\rule[-0.200pt]{0.400pt}{0.400pt}}%
\put(170.0,82.0){\rule[-0.200pt]{4.818pt}{0.400pt}}
\put(150,82){\makebox(0,0)[r]{ 1}}
\put(1419.0,82.0){\rule[-0.200pt]{4.818pt}{0.400pt}}
\put(170.0,108.0){\rule[-0.200pt]{2.409pt}{0.400pt}}
\put(1429.0,108.0){\rule[-0.200pt]{2.409pt}{0.400pt}}
\put(170.0,123.0){\rule[-0.200pt]{2.409pt}{0.400pt}}
\put(1429.0,123.0){\rule[-0.200pt]{2.409pt}{0.400pt}}
\put(170.0,134.0){\rule[-0.200pt]{2.409pt}{0.400pt}}
\put(1429.0,134.0){\rule[-0.200pt]{2.409pt}{0.400pt}}
\put(170.0,142.0){\rule[-0.200pt]{2.409pt}{0.400pt}}
\put(1429.0,142.0){\rule[-0.200pt]{2.409pt}{0.400pt}}
\put(170.0,149.0){\rule[-0.200pt]{2.409pt}{0.400pt}}
\put(1429.0,149.0){\rule[-0.200pt]{2.409pt}{0.400pt}}
\put(170.0,155.0){\rule[-0.200pt]{2.409pt}{0.400pt}}
\put(1429.0,155.0){\rule[-0.200pt]{2.409pt}{0.400pt}}
\put(170.0,160.0){\rule[-0.200pt]{2.409pt}{0.400pt}}
\put(1429.0,160.0){\rule[-0.200pt]{2.409pt}{0.400pt}}
\put(170.0,164.0){\rule[-0.200pt]{2.409pt}{0.400pt}}
\put(1429.0,164.0){\rule[-0.200pt]{2.409pt}{0.400pt}}
\put(170.0,168.0){\rule[-0.200pt]{2.409pt}{0.400pt}}
\put(1429.0,168.0){\rule[-0.200pt]{2.409pt}{0.400pt}}
\put(170.0,172.0){\rule[-0.200pt]{2.409pt}{0.400pt}}
\put(1429.0,172.0){\rule[-0.200pt]{2.409pt}{0.400pt}}
\put(170.0,175.0){\rule[-0.200pt]{2.409pt}{0.400pt}}
\put(1429.0,175.0){\rule[-0.200pt]{2.409pt}{0.400pt}}
\put(170.0,178.0){\rule[-0.200pt]{2.409pt}{0.400pt}}
\put(1429.0,178.0){\rule[-0.200pt]{2.409pt}{0.400pt}}
\put(170.0,181.0){\rule[-0.200pt]{2.409pt}{0.400pt}}
\put(1429.0,181.0){\rule[-0.200pt]{2.409pt}{0.400pt}}
\put(170.0,184.0){\rule[-0.200pt]{2.409pt}{0.400pt}}
\put(1429.0,184.0){\rule[-0.200pt]{2.409pt}{0.400pt}}
\put(170.0,186.0){\rule[-0.200pt]{2.409pt}{0.400pt}}
\put(1429.0,186.0){\rule[-0.200pt]{2.409pt}{0.400pt}}
\put(170.0,188.0){\rule[-0.200pt]{2.409pt}{0.400pt}}
\put(1429.0,188.0){\rule[-0.200pt]{2.409pt}{0.400pt}}
\put(170.0,190.0){\rule[-0.200pt]{2.409pt}{0.400pt}}
\put(1429.0,190.0){\rule[-0.200pt]{2.409pt}{0.400pt}}
\put(170.0,192.0){\rule[-0.200pt]{2.409pt}{0.400pt}}
\put(1429.0,192.0){\rule[-0.200pt]{2.409pt}{0.400pt}}
\put(170.0,194.0){\rule[-0.200pt]{2.409pt}{0.400pt}}
\put(1429.0,194.0){\rule[-0.200pt]{2.409pt}{0.400pt}}
\put(170.0,196.0){\rule[-0.200pt]{2.409pt}{0.400pt}}
\put(1429.0,196.0){\rule[-0.200pt]{2.409pt}{0.400pt}}
\put(170.0,198.0){\rule[-0.200pt]{2.409pt}{0.400pt}}
\put(1429.0,198.0){\rule[-0.200pt]{2.409pt}{0.400pt}}
\put(170.0,200.0){\rule[-0.200pt]{2.409pt}{0.400pt}}
\put(1429.0,200.0){\rule[-0.200pt]{2.409pt}{0.400pt}}
\put(170.0,201.0){\rule[-0.200pt]{2.409pt}{0.400pt}}
\put(1429.0,201.0){\rule[-0.200pt]{2.409pt}{0.400pt}}
\put(170.0,203.0){\rule[-0.200pt]{2.409pt}{0.400pt}}
\put(1429.0,203.0){\rule[-0.200pt]{2.409pt}{0.400pt}}
\put(170.0,204.0){\rule[-0.200pt]{2.409pt}{0.400pt}}
\put(1429.0,204.0){\rule[-0.200pt]{2.409pt}{0.400pt}}
\put(170.0,206.0){\rule[-0.200pt]{2.409pt}{0.400pt}}
\put(1429.0,206.0){\rule[-0.200pt]{2.409pt}{0.400pt}}
\put(170.0,207.0){\rule[-0.200pt]{2.409pt}{0.400pt}}
\put(1429.0,207.0){\rule[-0.200pt]{2.409pt}{0.400pt}}
\put(170.0,208.0){\rule[-0.200pt]{2.409pt}{0.400pt}}
\put(1429.0,208.0){\rule[-0.200pt]{2.409pt}{0.400pt}}
\put(170.0,210.0){\rule[-0.200pt]{2.409pt}{0.400pt}}
\put(1429.0,210.0){\rule[-0.200pt]{2.409pt}{0.400pt}}
\put(170.0,211.0){\rule[-0.200pt]{2.409pt}{0.400pt}}
\put(1429.0,211.0){\rule[-0.200pt]{2.409pt}{0.400pt}}
\put(170.0,212.0){\rule[-0.200pt]{2.409pt}{0.400pt}}
\put(1429.0,212.0){\rule[-0.200pt]{2.409pt}{0.400pt}}
\put(170.0,213.0){\rule[-0.200pt]{2.409pt}{0.400pt}}
\put(1429.0,213.0){\rule[-0.200pt]{2.409pt}{0.400pt}}
\put(170.0,214.0){\rule[-0.200pt]{2.409pt}{0.400pt}}
\put(1429.0,214.0){\rule[-0.200pt]{2.409pt}{0.400pt}}
\put(170.0,215.0){\rule[-0.200pt]{2.409pt}{0.400pt}}
\put(1429.0,215.0){\rule[-0.200pt]{2.409pt}{0.400pt}}
\put(170.0,216.0){\rule[-0.200pt]{2.409pt}{0.400pt}}
\put(1429.0,216.0){\rule[-0.200pt]{2.409pt}{0.400pt}}
\put(170.0,217.0){\rule[-0.200pt]{2.409pt}{0.400pt}}
\put(1429.0,217.0){\rule[-0.200pt]{2.409pt}{0.400pt}}
\put(170.0,218.0){\rule[-0.200pt]{2.409pt}{0.400pt}}
\put(1429.0,218.0){\rule[-0.200pt]{2.409pt}{0.400pt}}
\put(170.0,219.0){\rule[-0.200pt]{2.409pt}{0.400pt}}
\put(1429.0,219.0){\rule[-0.200pt]{2.409pt}{0.400pt}}
\put(170.0,220.0){\rule[-0.200pt]{2.409pt}{0.400pt}}
\put(1429.0,220.0){\rule[-0.200pt]{2.409pt}{0.400pt}}
\put(170.0,221.0){\rule[-0.200pt]{2.409pt}{0.400pt}}
\put(1429.0,221.0){\rule[-0.200pt]{2.409pt}{0.400pt}}
\put(170.0,222.0){\rule[-0.200pt]{2.409pt}{0.400pt}}
\put(1429.0,222.0){\rule[-0.200pt]{2.409pt}{0.400pt}}
\put(170.0,223.0){\rule[-0.200pt]{2.409pt}{0.400pt}}
\put(1429.0,223.0){\rule[-0.200pt]{2.409pt}{0.400pt}}
\put(170.0,224.0){\rule[-0.200pt]{2.409pt}{0.400pt}}
\put(1429.0,224.0){\rule[-0.200pt]{2.409pt}{0.400pt}}
\put(170.0,225.0){\rule[-0.200pt]{2.409pt}{0.400pt}}
\put(1429.0,225.0){\rule[-0.200pt]{2.409pt}{0.400pt}}
\put(170.0,226.0){\rule[-0.200pt]{2.409pt}{0.400pt}}
\put(1429.0,226.0){\rule[-0.200pt]{2.409pt}{0.400pt}}
\put(170.0,226.0){\rule[-0.200pt]{2.409pt}{0.400pt}}
\put(1429.0,226.0){\rule[-0.200pt]{2.409pt}{0.400pt}}
\put(170.0,227.0){\rule[-0.200pt]{2.409pt}{0.400pt}}
\put(1429.0,227.0){\rule[-0.200pt]{2.409pt}{0.400pt}}
\put(170.0,228.0){\rule[-0.200pt]{2.409pt}{0.400pt}}
\put(1429.0,228.0){\rule[-0.200pt]{2.409pt}{0.400pt}}
\put(170.0,229.0){\rule[-0.200pt]{2.409pt}{0.400pt}}
\put(1429.0,229.0){\rule[-0.200pt]{2.409pt}{0.400pt}}
\put(170.0,229.0){\rule[-0.200pt]{2.409pt}{0.400pt}}
\put(1429.0,229.0){\rule[-0.200pt]{2.409pt}{0.400pt}}
\put(170.0,230.0){\rule[-0.200pt]{2.409pt}{0.400pt}}
\put(1429.0,230.0){\rule[-0.200pt]{2.409pt}{0.400pt}}
\put(170.0,231.0){\rule[-0.200pt]{2.409pt}{0.400pt}}
\put(1429.0,231.0){\rule[-0.200pt]{2.409pt}{0.400pt}}
\put(170.0,232.0){\rule[-0.200pt]{2.409pt}{0.400pt}}
\put(1429.0,232.0){\rule[-0.200pt]{2.409pt}{0.400pt}}
\put(170.0,232.0){\rule[-0.200pt]{2.409pt}{0.400pt}}
\put(1429.0,232.0){\rule[-0.200pt]{2.409pt}{0.400pt}}
\put(170.0,233.0){\rule[-0.200pt]{2.409pt}{0.400pt}}
\put(1429.0,233.0){\rule[-0.200pt]{2.409pt}{0.400pt}}
\put(170.0,234.0){\rule[-0.200pt]{2.409pt}{0.400pt}}
\put(1429.0,234.0){\rule[-0.200pt]{2.409pt}{0.400pt}}
\put(170.0,234.0){\rule[-0.200pt]{2.409pt}{0.400pt}}
\put(1429.0,234.0){\rule[-0.200pt]{2.409pt}{0.400pt}}
\put(170.0,235.0){\rule[-0.200pt]{2.409pt}{0.400pt}}
\put(1429.0,235.0){\rule[-0.200pt]{2.409pt}{0.400pt}}
\put(170.0,236.0){\rule[-0.200pt]{2.409pt}{0.400pt}}
\put(1429.0,236.0){\rule[-0.200pt]{2.409pt}{0.400pt}}
\put(170.0,236.0){\rule[-0.200pt]{2.409pt}{0.400pt}}
\put(1429.0,236.0){\rule[-0.200pt]{2.409pt}{0.400pt}}
\put(170.0,237.0){\rule[-0.200pt]{2.409pt}{0.400pt}}
\put(1429.0,237.0){\rule[-0.200pt]{2.409pt}{0.400pt}}
\put(170.0,237.0){\rule[-0.200pt]{2.409pt}{0.400pt}}
\put(1429.0,237.0){\rule[-0.200pt]{2.409pt}{0.400pt}}
\put(170.0,238.0){\rule[-0.200pt]{2.409pt}{0.400pt}}
\put(1429.0,238.0){\rule[-0.200pt]{2.409pt}{0.400pt}}
\put(170.0,239.0){\rule[-0.200pt]{2.409pt}{0.400pt}}
\put(1429.0,239.0){\rule[-0.200pt]{2.409pt}{0.400pt}}
\put(170.0,239.0){\rule[-0.200pt]{2.409pt}{0.400pt}}
\put(1429.0,239.0){\rule[-0.200pt]{2.409pt}{0.400pt}}
\put(170.0,240.0){\rule[-0.200pt]{2.409pt}{0.400pt}}
\put(1429.0,240.0){\rule[-0.200pt]{2.409pt}{0.400pt}}
\put(170.0,240.0){\rule[-0.200pt]{2.409pt}{0.400pt}}
\put(1429.0,240.0){\rule[-0.200pt]{2.409pt}{0.400pt}}
\put(170.0,241.0){\rule[-0.200pt]{2.409pt}{0.400pt}}
\put(1429.0,241.0){\rule[-0.200pt]{2.409pt}{0.400pt}}
\put(170.0,241.0){\rule[-0.200pt]{2.409pt}{0.400pt}}
\put(1429.0,241.0){\rule[-0.200pt]{2.409pt}{0.400pt}}
\put(170.0,242.0){\rule[-0.200pt]{2.409pt}{0.400pt}}
\put(1429.0,242.0){\rule[-0.200pt]{2.409pt}{0.400pt}}
\put(170.0,242.0){\rule[-0.200pt]{2.409pt}{0.400pt}}
\put(1429.0,242.0){\rule[-0.200pt]{2.409pt}{0.400pt}}
\put(170.0,243.0){\rule[-0.200pt]{2.409pt}{0.400pt}}
\put(1429.0,243.0){\rule[-0.200pt]{2.409pt}{0.400pt}}
\put(170.0,243.0){\rule[-0.200pt]{2.409pt}{0.400pt}}
\put(1429.0,243.0){\rule[-0.200pt]{2.409pt}{0.400pt}}
\put(170.0,244.0){\rule[-0.200pt]{2.409pt}{0.400pt}}
\put(1429.0,244.0){\rule[-0.200pt]{2.409pt}{0.400pt}}
\put(170.0,244.0){\rule[-0.200pt]{2.409pt}{0.400pt}}
\put(1429.0,244.0){\rule[-0.200pt]{2.409pt}{0.400pt}}
\put(170.0,245.0){\rule[-0.200pt]{2.409pt}{0.400pt}}
\put(1429.0,245.0){\rule[-0.200pt]{2.409pt}{0.400pt}}
\put(170.0,245.0){\rule[-0.200pt]{2.409pt}{0.400pt}}
\put(1429.0,245.0){\rule[-0.200pt]{2.409pt}{0.400pt}}
\put(170.0,246.0){\rule[-0.200pt]{2.409pt}{0.400pt}}
\put(1429.0,246.0){\rule[-0.200pt]{2.409pt}{0.400pt}}
\put(170.0,246.0){\rule[-0.200pt]{2.409pt}{0.400pt}}
\put(1429.0,246.0){\rule[-0.200pt]{2.409pt}{0.400pt}}
\put(170.0,247.0){\rule[-0.200pt]{2.409pt}{0.400pt}}
\put(1429.0,247.0){\rule[-0.200pt]{2.409pt}{0.400pt}}
\put(170.0,247.0){\rule[-0.200pt]{2.409pt}{0.400pt}}
\put(1429.0,247.0){\rule[-0.200pt]{2.409pt}{0.400pt}}
\put(170.0,248.0){\rule[-0.200pt]{2.409pt}{0.400pt}}
\put(1429.0,248.0){\rule[-0.200pt]{2.409pt}{0.400pt}}
\put(170.0,248.0){\rule[-0.200pt]{2.409pt}{0.400pt}}
\put(1429.0,248.0){\rule[-0.200pt]{2.409pt}{0.400pt}}
\put(170.0,249.0){\rule[-0.200pt]{2.409pt}{0.400pt}}
\put(1429.0,249.0){\rule[-0.200pt]{2.409pt}{0.400pt}}
\put(170.0,249.0){\rule[-0.200pt]{2.409pt}{0.400pt}}
\put(1429.0,249.0){\rule[-0.200pt]{2.409pt}{0.400pt}}
\put(170.0,249.0){\rule[-0.200pt]{2.409pt}{0.400pt}}
\put(1429.0,249.0){\rule[-0.200pt]{2.409pt}{0.400pt}}
\put(170.0,250.0){\rule[-0.200pt]{2.409pt}{0.400pt}}
\put(1429.0,250.0){\rule[-0.200pt]{2.409pt}{0.400pt}}
\put(170.0,250.0){\rule[-0.200pt]{2.409pt}{0.400pt}}
\put(1429.0,250.0){\rule[-0.200pt]{2.409pt}{0.400pt}}
\put(170.0,251.0){\rule[-0.200pt]{2.409pt}{0.400pt}}
\put(1429.0,251.0){\rule[-0.200pt]{2.409pt}{0.400pt}}
\put(170.0,251.0){\rule[-0.200pt]{2.409pt}{0.400pt}}
\put(1429.0,251.0){\rule[-0.200pt]{2.409pt}{0.400pt}}
\put(170.0,252.0){\rule[-0.200pt]{2.409pt}{0.400pt}}
\put(1429.0,252.0){\rule[-0.200pt]{2.409pt}{0.400pt}}
\put(170.0,252.0){\rule[-0.200pt]{2.409pt}{0.400pt}}
\put(1429.0,252.0){\rule[-0.200pt]{2.409pt}{0.400pt}}
\put(170.0,252.0){\rule[-0.200pt]{2.409pt}{0.400pt}}
\put(1429.0,252.0){\rule[-0.200pt]{2.409pt}{0.400pt}}
\put(170.0,253.0){\rule[-0.200pt]{2.409pt}{0.400pt}}
\put(1429.0,253.0){\rule[-0.200pt]{2.409pt}{0.400pt}}
\put(170.0,253.0){\rule[-0.200pt]{2.409pt}{0.400pt}}
\put(1429.0,253.0){\rule[-0.200pt]{2.409pt}{0.400pt}}
\put(170.0,254.0){\rule[-0.200pt]{2.409pt}{0.400pt}}
\put(1429.0,254.0){\rule[-0.200pt]{2.409pt}{0.400pt}}
\put(170.0,254.0){\rule[-0.200pt]{2.409pt}{0.400pt}}
\put(1429.0,254.0){\rule[-0.200pt]{2.409pt}{0.400pt}}
\put(170.0,254.0){\rule[-0.200pt]{2.409pt}{0.400pt}}
\put(1429.0,254.0){\rule[-0.200pt]{2.409pt}{0.400pt}}
\put(170.0,255.0){\rule[-0.200pt]{2.409pt}{0.400pt}}
\put(1429.0,255.0){\rule[-0.200pt]{2.409pt}{0.400pt}}
\put(170.0,255.0){\rule[-0.200pt]{2.409pt}{0.400pt}}
\put(1429.0,255.0){\rule[-0.200pt]{2.409pt}{0.400pt}}
\put(170.0,255.0){\rule[-0.200pt]{2.409pt}{0.400pt}}
\put(1429.0,255.0){\rule[-0.200pt]{2.409pt}{0.400pt}}
\put(170.0,256.0){\rule[-0.200pt]{2.409pt}{0.400pt}}
\put(1429.0,256.0){\rule[-0.200pt]{2.409pt}{0.400pt}}
\put(170.0,256.0){\rule[-0.200pt]{2.409pt}{0.400pt}}
\put(1429.0,256.0){\rule[-0.200pt]{2.409pt}{0.400pt}}
\put(170.0,256.0){\rule[-0.200pt]{2.409pt}{0.400pt}}
\put(1429.0,256.0){\rule[-0.200pt]{2.409pt}{0.400pt}}
\put(170.0,257.0){\rule[-0.200pt]{2.409pt}{0.400pt}}
\put(1429.0,257.0){\rule[-0.200pt]{2.409pt}{0.400pt}}
\put(170.0,257.0){\rule[-0.200pt]{2.409pt}{0.400pt}}
\put(1429.0,257.0){\rule[-0.200pt]{2.409pt}{0.400pt}}
\put(170.0,258.0){\rule[-0.200pt]{2.409pt}{0.400pt}}
\put(1429.0,258.0){\rule[-0.200pt]{2.409pt}{0.400pt}}
\put(170.0,258.0){\rule[-0.200pt]{2.409pt}{0.400pt}}
\put(1429.0,258.0){\rule[-0.200pt]{2.409pt}{0.400pt}}
\put(170.0,258.0){\rule[-0.200pt]{2.409pt}{0.400pt}}
\put(1429.0,258.0){\rule[-0.200pt]{2.409pt}{0.400pt}}
\put(170.0,259.0){\rule[-0.200pt]{2.409pt}{0.400pt}}
\put(1429.0,259.0){\rule[-0.200pt]{2.409pt}{0.400pt}}
\put(170.0,259.0){\rule[-0.200pt]{2.409pt}{0.400pt}}
\put(1429.0,259.0){\rule[-0.200pt]{2.409pt}{0.400pt}}
\put(170.0,259.0){\rule[-0.200pt]{2.409pt}{0.400pt}}
\put(1429.0,259.0){\rule[-0.200pt]{2.409pt}{0.400pt}}
\put(170.0,260.0){\rule[-0.200pt]{2.409pt}{0.400pt}}
\put(1429.0,260.0){\rule[-0.200pt]{2.409pt}{0.400pt}}
\put(170.0,260.0){\rule[-0.200pt]{2.409pt}{0.400pt}}
\put(1429.0,260.0){\rule[-0.200pt]{2.409pt}{0.400pt}}
\put(170.0,260.0){\rule[-0.200pt]{2.409pt}{0.400pt}}
\put(1429.0,260.0){\rule[-0.200pt]{2.409pt}{0.400pt}}
\put(170.0,261.0){\rule[-0.200pt]{2.409pt}{0.400pt}}
\put(1429.0,261.0){\rule[-0.200pt]{2.409pt}{0.400pt}}
\put(170.0,261.0){\rule[-0.200pt]{2.409pt}{0.400pt}}
\put(1429.0,261.0){\rule[-0.200pt]{2.409pt}{0.400pt}}
\put(170.0,261.0){\rule[-0.200pt]{2.409pt}{0.400pt}}
\put(1429.0,261.0){\rule[-0.200pt]{2.409pt}{0.400pt}}
\put(170.0,262.0){\rule[-0.200pt]{2.409pt}{0.400pt}}
\put(1429.0,262.0){\rule[-0.200pt]{2.409pt}{0.400pt}}
\put(170.0,262.0){\rule[-0.200pt]{2.409pt}{0.400pt}}
\put(1429.0,262.0){\rule[-0.200pt]{2.409pt}{0.400pt}}
\put(170.0,262.0){\rule[-0.200pt]{2.409pt}{0.400pt}}
\put(1429.0,262.0){\rule[-0.200pt]{2.409pt}{0.400pt}}
\put(170.0,262.0){\rule[-0.200pt]{2.409pt}{0.400pt}}
\put(1429.0,262.0){\rule[-0.200pt]{2.409pt}{0.400pt}}
\put(170.0,263.0){\rule[-0.200pt]{2.409pt}{0.400pt}}
\put(1429.0,263.0){\rule[-0.200pt]{2.409pt}{0.400pt}}
\put(170.0,263.0){\rule[-0.200pt]{2.409pt}{0.400pt}}
\put(1429.0,263.0){\rule[-0.200pt]{2.409pt}{0.400pt}}
\put(170.0,263.0){\rule[-0.200pt]{2.409pt}{0.400pt}}
\put(1429.0,263.0){\rule[-0.200pt]{2.409pt}{0.400pt}}
\put(170.0,264.0){\rule[-0.200pt]{2.409pt}{0.400pt}}
\put(1429.0,264.0){\rule[-0.200pt]{2.409pt}{0.400pt}}
\put(170.0,264.0){\rule[-0.200pt]{2.409pt}{0.400pt}}
\put(1429.0,264.0){\rule[-0.200pt]{2.409pt}{0.400pt}}
\put(170.0,264.0){\rule[-0.200pt]{2.409pt}{0.400pt}}
\put(1429.0,264.0){\rule[-0.200pt]{2.409pt}{0.400pt}}
\put(170.0,265.0){\rule[-0.200pt]{2.409pt}{0.400pt}}
\put(1429.0,265.0){\rule[-0.200pt]{2.409pt}{0.400pt}}
\put(170.0,265.0){\rule[-0.200pt]{2.409pt}{0.400pt}}
\put(1429.0,265.0){\rule[-0.200pt]{2.409pt}{0.400pt}}
\put(170.0,265.0){\rule[-0.200pt]{2.409pt}{0.400pt}}
\put(1429.0,265.0){\rule[-0.200pt]{2.409pt}{0.400pt}}
\put(170.0,265.0){\rule[-0.200pt]{2.409pt}{0.400pt}}
\put(1429.0,265.0){\rule[-0.200pt]{2.409pt}{0.400pt}}
\put(170.0,266.0){\rule[-0.200pt]{2.409pt}{0.400pt}}
\put(1429.0,266.0){\rule[-0.200pt]{2.409pt}{0.400pt}}
\put(170.0,266.0){\rule[-0.200pt]{2.409pt}{0.400pt}}
\put(1429.0,266.0){\rule[-0.200pt]{2.409pt}{0.400pt}}
\put(170.0,266.0){\rule[-0.200pt]{2.409pt}{0.400pt}}
\put(1429.0,266.0){\rule[-0.200pt]{2.409pt}{0.400pt}}
\put(170.0,266.0){\rule[-0.200pt]{2.409pt}{0.400pt}}
\put(1429.0,266.0){\rule[-0.200pt]{2.409pt}{0.400pt}}
\put(170.0,267.0){\rule[-0.200pt]{2.409pt}{0.400pt}}
\put(1429.0,267.0){\rule[-0.200pt]{2.409pt}{0.400pt}}
\put(170.0,267.0){\rule[-0.200pt]{2.409pt}{0.400pt}}
\put(1429.0,267.0){\rule[-0.200pt]{2.409pt}{0.400pt}}
\put(170.0,267.0){\rule[-0.200pt]{2.409pt}{0.400pt}}
\put(1429.0,267.0){\rule[-0.200pt]{2.409pt}{0.400pt}}
\put(170.0,268.0){\rule[-0.200pt]{2.409pt}{0.400pt}}
\put(1429.0,268.0){\rule[-0.200pt]{2.409pt}{0.400pt}}
\put(170.0,268.0){\rule[-0.200pt]{2.409pt}{0.400pt}}
\put(1429.0,268.0){\rule[-0.200pt]{2.409pt}{0.400pt}}
\put(170.0,268.0){\rule[-0.200pt]{2.409pt}{0.400pt}}
\put(1429.0,268.0){\rule[-0.200pt]{2.409pt}{0.400pt}}
\put(170.0,268.0){\rule[-0.200pt]{2.409pt}{0.400pt}}
\put(1429.0,268.0){\rule[-0.200pt]{2.409pt}{0.400pt}}
\put(170.0,269.0){\rule[-0.200pt]{2.409pt}{0.400pt}}
\put(1429.0,269.0){\rule[-0.200pt]{2.409pt}{0.400pt}}
\put(170.0,269.0){\rule[-0.200pt]{2.409pt}{0.400pt}}
\put(1429.0,269.0){\rule[-0.200pt]{2.409pt}{0.400pt}}
\put(170.0,269.0){\rule[-0.200pt]{2.409pt}{0.400pt}}
\put(1429.0,269.0){\rule[-0.200pt]{2.409pt}{0.400pt}}
\put(170.0,269.0){\rule[-0.200pt]{2.409pt}{0.400pt}}
\put(1429.0,269.0){\rule[-0.200pt]{2.409pt}{0.400pt}}
\put(170.0,270.0){\rule[-0.200pt]{2.409pt}{0.400pt}}
\put(1429.0,270.0){\rule[-0.200pt]{2.409pt}{0.400pt}}
\put(170.0,270.0){\rule[-0.200pt]{2.409pt}{0.400pt}}
\put(1429.0,270.0){\rule[-0.200pt]{2.409pt}{0.400pt}}
\put(170.0,270.0){\rule[-0.200pt]{2.409pt}{0.400pt}}
\put(1429.0,270.0){\rule[-0.200pt]{2.409pt}{0.400pt}}
\put(170.0,270.0){\rule[-0.200pt]{2.409pt}{0.400pt}}
\put(1429.0,270.0){\rule[-0.200pt]{2.409pt}{0.400pt}}
\put(170.0,271.0){\rule[-0.200pt]{2.409pt}{0.400pt}}
\put(1429.0,271.0){\rule[-0.200pt]{2.409pt}{0.400pt}}
\put(170.0,271.0){\rule[-0.200pt]{2.409pt}{0.400pt}}
\put(1429.0,271.0){\rule[-0.200pt]{2.409pt}{0.400pt}}
\put(170.0,271.0){\rule[-0.200pt]{2.409pt}{0.400pt}}
\put(1429.0,271.0){\rule[-0.200pt]{2.409pt}{0.400pt}}
\put(170.0,271.0){\rule[-0.200pt]{2.409pt}{0.400pt}}
\put(1429.0,271.0){\rule[-0.200pt]{2.409pt}{0.400pt}}
\put(170.0,272.0){\rule[-0.200pt]{2.409pt}{0.400pt}}
\put(1429.0,272.0){\rule[-0.200pt]{2.409pt}{0.400pt}}
\put(170.0,272.0){\rule[-0.200pt]{2.409pt}{0.400pt}}
\put(1429.0,272.0){\rule[-0.200pt]{2.409pt}{0.400pt}}
\put(170.0,272.0){\rule[-0.200pt]{2.409pt}{0.400pt}}
\put(1429.0,272.0){\rule[-0.200pt]{2.409pt}{0.400pt}}
\put(170.0,272.0){\rule[-0.200pt]{2.409pt}{0.400pt}}
\put(1429.0,272.0){\rule[-0.200pt]{2.409pt}{0.400pt}}
\put(170.0,273.0){\rule[-0.200pt]{2.409pt}{0.400pt}}
\put(1429.0,273.0){\rule[-0.200pt]{2.409pt}{0.400pt}}
\put(170.0,273.0){\rule[-0.200pt]{2.409pt}{0.400pt}}
\put(1429.0,273.0){\rule[-0.200pt]{2.409pt}{0.400pt}}
\put(170.0,273.0){\rule[-0.200pt]{2.409pt}{0.400pt}}
\put(1429.0,273.0){\rule[-0.200pt]{2.409pt}{0.400pt}}
\put(170.0,273.0){\rule[-0.200pt]{2.409pt}{0.400pt}}
\put(1429.0,273.0){\rule[-0.200pt]{2.409pt}{0.400pt}}
\put(170.0,273.0){\rule[-0.200pt]{2.409pt}{0.400pt}}
\put(1429.0,273.0){\rule[-0.200pt]{2.409pt}{0.400pt}}
\put(170.0,274.0){\rule[-0.200pt]{2.409pt}{0.400pt}}
\put(1429.0,274.0){\rule[-0.200pt]{2.409pt}{0.400pt}}
\put(170.0,274.0){\rule[-0.200pt]{2.409pt}{0.400pt}}
\put(1429.0,274.0){\rule[-0.200pt]{2.409pt}{0.400pt}}
\put(170.0,274.0){\rule[-0.200pt]{2.409pt}{0.400pt}}
\put(1429.0,274.0){\rule[-0.200pt]{2.409pt}{0.400pt}}
\put(170.0,274.0){\rule[-0.200pt]{2.409pt}{0.400pt}}
\put(1429.0,274.0){\rule[-0.200pt]{2.409pt}{0.400pt}}
\put(170.0,275.0){\rule[-0.200pt]{2.409pt}{0.400pt}}
\put(1429.0,275.0){\rule[-0.200pt]{2.409pt}{0.400pt}}
\put(170.0,275.0){\rule[-0.200pt]{2.409pt}{0.400pt}}
\put(1429.0,275.0){\rule[-0.200pt]{2.409pt}{0.400pt}}
\put(170.0,275.0){\rule[-0.200pt]{2.409pt}{0.400pt}}
\put(1429.0,275.0){\rule[-0.200pt]{2.409pt}{0.400pt}}
\put(170.0,275.0){\rule[-0.200pt]{2.409pt}{0.400pt}}
\put(1429.0,275.0){\rule[-0.200pt]{2.409pt}{0.400pt}}
\put(170.0,275.0){\rule[-0.200pt]{2.409pt}{0.400pt}}
\put(1429.0,275.0){\rule[-0.200pt]{2.409pt}{0.400pt}}
\put(170.0,276.0){\rule[-0.200pt]{2.409pt}{0.400pt}}
\put(1429.0,276.0){\rule[-0.200pt]{2.409pt}{0.400pt}}
\put(170.0,276.0){\rule[-0.200pt]{2.409pt}{0.400pt}}
\put(1429.0,276.0){\rule[-0.200pt]{2.409pt}{0.400pt}}
\put(170.0,276.0){\rule[-0.200pt]{2.409pt}{0.400pt}}
\put(1429.0,276.0){\rule[-0.200pt]{2.409pt}{0.400pt}}
\put(170.0,276.0){\rule[-0.200pt]{2.409pt}{0.400pt}}
\put(1429.0,276.0){\rule[-0.200pt]{2.409pt}{0.400pt}}
\put(170.0,276.0){\rule[-0.200pt]{2.409pt}{0.400pt}}
\put(1429.0,276.0){\rule[-0.200pt]{2.409pt}{0.400pt}}
\put(170.0,277.0){\rule[-0.200pt]{2.409pt}{0.400pt}}
\put(1429.0,277.0){\rule[-0.200pt]{2.409pt}{0.400pt}}
\put(170.0,277.0){\rule[-0.200pt]{2.409pt}{0.400pt}}
\put(1429.0,277.0){\rule[-0.200pt]{2.409pt}{0.400pt}}
\put(170.0,277.0){\rule[-0.200pt]{2.409pt}{0.400pt}}
\put(1429.0,277.0){\rule[-0.200pt]{2.409pt}{0.400pt}}
\put(170.0,277.0){\rule[-0.200pt]{2.409pt}{0.400pt}}
\put(1429.0,277.0){\rule[-0.200pt]{2.409pt}{0.400pt}}
\put(170.0,278.0){\rule[-0.200pt]{2.409pt}{0.400pt}}
\put(1429.0,278.0){\rule[-0.200pt]{2.409pt}{0.400pt}}
\put(170.0,278.0){\rule[-0.200pt]{2.409pt}{0.400pt}}
\put(1429.0,278.0){\rule[-0.200pt]{2.409pt}{0.400pt}}
\put(170.0,278.0){\rule[-0.200pt]{2.409pt}{0.400pt}}
\put(1429.0,278.0){\rule[-0.200pt]{2.409pt}{0.400pt}}
\put(170.0,278.0){\rule[-0.200pt]{2.409pt}{0.400pt}}
\put(1429.0,278.0){\rule[-0.200pt]{2.409pt}{0.400pt}}
\put(170.0,278.0){\rule[-0.200pt]{2.409pt}{0.400pt}}
\put(1429.0,278.0){\rule[-0.200pt]{2.409pt}{0.400pt}}
\put(170.0,279.0){\rule[-0.200pt]{2.409pt}{0.400pt}}
\put(1429.0,279.0){\rule[-0.200pt]{2.409pt}{0.400pt}}
\put(170.0,279.0){\rule[-0.200pt]{2.409pt}{0.400pt}}
\put(1429.0,279.0){\rule[-0.200pt]{2.409pt}{0.400pt}}
\put(170.0,279.0){\rule[-0.200pt]{2.409pt}{0.400pt}}
\put(1429.0,279.0){\rule[-0.200pt]{2.409pt}{0.400pt}}
\put(170.0,279.0){\rule[-0.200pt]{2.409pt}{0.400pt}}
\put(1429.0,279.0){\rule[-0.200pt]{2.409pt}{0.400pt}}
\put(170.0,279.0){\rule[-0.200pt]{2.409pt}{0.400pt}}
\put(1429.0,279.0){\rule[-0.200pt]{2.409pt}{0.400pt}}
\put(170.0,280.0){\rule[-0.200pt]{2.409pt}{0.400pt}}
\put(1429.0,280.0){\rule[-0.200pt]{2.409pt}{0.400pt}}
\put(170.0,280.0){\rule[-0.200pt]{2.409pt}{0.400pt}}
\put(1429.0,280.0){\rule[-0.200pt]{2.409pt}{0.400pt}}
\put(170.0,280.0){\rule[-0.200pt]{2.409pt}{0.400pt}}
\put(1429.0,280.0){\rule[-0.200pt]{2.409pt}{0.400pt}}
\put(170.0,280.0){\rule[-0.200pt]{2.409pt}{0.400pt}}
\put(1429.0,280.0){\rule[-0.200pt]{2.409pt}{0.400pt}}
\put(170.0,280.0){\rule[-0.200pt]{2.409pt}{0.400pt}}
\put(1429.0,280.0){\rule[-0.200pt]{2.409pt}{0.400pt}}
\put(170.0,280.0){\rule[-0.200pt]{2.409pt}{0.400pt}}
\put(1429.0,280.0){\rule[-0.200pt]{2.409pt}{0.400pt}}
\put(170.0,281.0){\rule[-0.200pt]{2.409pt}{0.400pt}}
\put(1429.0,281.0){\rule[-0.200pt]{2.409pt}{0.400pt}}
\put(170.0,281.0){\rule[-0.200pt]{2.409pt}{0.400pt}}
\put(1429.0,281.0){\rule[-0.200pt]{2.409pt}{0.400pt}}
\put(170.0,281.0){\rule[-0.200pt]{2.409pt}{0.400pt}}
\put(1429.0,281.0){\rule[-0.200pt]{2.409pt}{0.400pt}}
\put(170.0,281.0){\rule[-0.200pt]{2.409pt}{0.400pt}}
\put(1429.0,281.0){\rule[-0.200pt]{2.409pt}{0.400pt}}
\put(170.0,281.0){\rule[-0.200pt]{2.409pt}{0.400pt}}
\put(1429.0,281.0){\rule[-0.200pt]{2.409pt}{0.400pt}}
\put(170.0,282.0){\rule[-0.200pt]{2.409pt}{0.400pt}}
\put(1429.0,282.0){\rule[-0.200pt]{2.409pt}{0.400pt}}
\put(170.0,282.0){\rule[-0.200pt]{2.409pt}{0.400pt}}
\put(1429.0,282.0){\rule[-0.200pt]{2.409pt}{0.400pt}}
\put(170.0,282.0){\rule[-0.200pt]{2.409pt}{0.400pt}}
\put(1429.0,282.0){\rule[-0.200pt]{2.409pt}{0.400pt}}
\put(170.0,282.0){\rule[-0.200pt]{2.409pt}{0.400pt}}
\put(1429.0,282.0){\rule[-0.200pt]{2.409pt}{0.400pt}}
\put(170.0,282.0){\rule[-0.200pt]{2.409pt}{0.400pt}}
\put(1429.0,282.0){\rule[-0.200pt]{2.409pt}{0.400pt}}
\put(170.0,282.0){\rule[-0.200pt]{2.409pt}{0.400pt}}
\put(1429.0,282.0){\rule[-0.200pt]{2.409pt}{0.400pt}}
\put(170.0,283.0){\rule[-0.200pt]{2.409pt}{0.400pt}}
\put(1429.0,283.0){\rule[-0.200pt]{2.409pt}{0.400pt}}
\put(170.0,283.0){\rule[-0.200pt]{2.409pt}{0.400pt}}
\put(1429.0,283.0){\rule[-0.200pt]{2.409pt}{0.400pt}}
\put(170.0,283.0){\rule[-0.200pt]{2.409pt}{0.400pt}}
\put(1429.0,283.0){\rule[-0.200pt]{2.409pt}{0.400pt}}
\put(170.0,283.0){\rule[-0.200pt]{2.409pt}{0.400pt}}
\put(1429.0,283.0){\rule[-0.200pt]{2.409pt}{0.400pt}}
\put(170.0,283.0){\rule[-0.200pt]{2.409pt}{0.400pt}}
\put(1429.0,283.0){\rule[-0.200pt]{2.409pt}{0.400pt}}
\put(170.0,284.0){\rule[-0.200pt]{2.409pt}{0.400pt}}
\put(1429.0,284.0){\rule[-0.200pt]{2.409pt}{0.400pt}}
\put(170.0,284.0){\rule[-0.200pt]{2.409pt}{0.400pt}}
\put(1429.0,284.0){\rule[-0.200pt]{2.409pt}{0.400pt}}
\put(170.0,284.0){\rule[-0.200pt]{2.409pt}{0.400pt}}
\put(1429.0,284.0){\rule[-0.200pt]{2.409pt}{0.400pt}}
\put(170.0,284.0){\rule[-0.200pt]{2.409pt}{0.400pt}}
\put(1429.0,284.0){\rule[-0.200pt]{2.409pt}{0.400pt}}
\put(170.0,284.0){\rule[-0.200pt]{2.409pt}{0.400pt}}
\put(1429.0,284.0){\rule[-0.200pt]{2.409pt}{0.400pt}}
\put(170.0,284.0){\rule[-0.200pt]{2.409pt}{0.400pt}}
\put(1429.0,284.0){\rule[-0.200pt]{2.409pt}{0.400pt}}
\put(170.0,285.0){\rule[-0.200pt]{2.409pt}{0.400pt}}
\put(1429.0,285.0){\rule[-0.200pt]{2.409pt}{0.400pt}}
\put(170.0,285.0){\rule[-0.200pt]{2.409pt}{0.400pt}}
\put(1429.0,285.0){\rule[-0.200pt]{2.409pt}{0.400pt}}
\put(170.0,285.0){\rule[-0.200pt]{2.409pt}{0.400pt}}
\put(1429.0,285.0){\rule[-0.200pt]{2.409pt}{0.400pt}}
\put(170.0,285.0){\rule[-0.200pt]{2.409pt}{0.400pt}}
\put(1429.0,285.0){\rule[-0.200pt]{2.409pt}{0.400pt}}
\put(170.0,285.0){\rule[-0.200pt]{2.409pt}{0.400pt}}
\put(1429.0,285.0){\rule[-0.200pt]{2.409pt}{0.400pt}}
\put(170.0,285.0){\rule[-0.200pt]{2.409pt}{0.400pt}}
\put(1429.0,285.0){\rule[-0.200pt]{2.409pt}{0.400pt}}
\put(170.0,286.0){\rule[-0.200pt]{2.409pt}{0.400pt}}
\put(1429.0,286.0){\rule[-0.200pt]{2.409pt}{0.400pt}}
\put(170.0,286.0){\rule[-0.200pt]{2.409pt}{0.400pt}}
\put(1429.0,286.0){\rule[-0.200pt]{2.409pt}{0.400pt}}
\put(170.0,286.0){\rule[-0.200pt]{2.409pt}{0.400pt}}
\put(1429.0,286.0){\rule[-0.200pt]{2.409pt}{0.400pt}}
\put(170.0,286.0){\rule[-0.200pt]{2.409pt}{0.400pt}}
\put(1429.0,286.0){\rule[-0.200pt]{2.409pt}{0.400pt}}
\put(170.0,286.0){\rule[-0.200pt]{2.409pt}{0.400pt}}
\put(1429.0,286.0){\rule[-0.200pt]{2.409pt}{0.400pt}}
\put(170.0,286.0){\rule[-0.200pt]{2.409pt}{0.400pt}}
\put(1429.0,286.0){\rule[-0.200pt]{2.409pt}{0.400pt}}
\put(170.0,287.0){\rule[-0.200pt]{2.409pt}{0.400pt}}
\put(1429.0,287.0){\rule[-0.200pt]{2.409pt}{0.400pt}}
\put(170.0,287.0){\rule[-0.200pt]{2.409pt}{0.400pt}}
\put(1429.0,287.0){\rule[-0.200pt]{2.409pt}{0.400pt}}
\put(170.0,287.0){\rule[-0.200pt]{2.409pt}{0.400pt}}
\put(1429.0,287.0){\rule[-0.200pt]{2.409pt}{0.400pt}}
\put(170.0,287.0){\rule[-0.200pt]{2.409pt}{0.400pt}}
\put(1429.0,287.0){\rule[-0.200pt]{2.409pt}{0.400pt}}
\put(170.0,287.0){\rule[-0.200pt]{2.409pt}{0.400pt}}
\put(1429.0,287.0){\rule[-0.200pt]{2.409pt}{0.400pt}}
\put(170.0,287.0){\rule[-0.200pt]{2.409pt}{0.400pt}}
\put(1429.0,287.0){\rule[-0.200pt]{2.409pt}{0.400pt}}
\put(170.0,287.0){\rule[-0.200pt]{2.409pt}{0.400pt}}
\put(1429.0,287.0){\rule[-0.200pt]{2.409pt}{0.400pt}}
\put(170.0,288.0){\rule[-0.200pt]{2.409pt}{0.400pt}}
\put(1429.0,288.0){\rule[-0.200pt]{2.409pt}{0.400pt}}
\put(170.0,288.0){\rule[-0.200pt]{2.409pt}{0.400pt}}
\put(1429.0,288.0){\rule[-0.200pt]{2.409pt}{0.400pt}}
\put(170.0,288.0){\rule[-0.200pt]{2.409pt}{0.400pt}}
\put(1429.0,288.0){\rule[-0.200pt]{2.409pt}{0.400pt}}
\put(170.0,288.0){\rule[-0.200pt]{2.409pt}{0.400pt}}
\put(1429.0,288.0){\rule[-0.200pt]{2.409pt}{0.400pt}}
\put(170.0,288.0){\rule[-0.200pt]{2.409pt}{0.400pt}}
\put(1429.0,288.0){\rule[-0.200pt]{2.409pt}{0.400pt}}
\put(170.0,288.0){\rule[-0.200pt]{2.409pt}{0.400pt}}
\put(1429.0,288.0){\rule[-0.200pt]{2.409pt}{0.400pt}}
\put(170.0,289.0){\rule[-0.200pt]{2.409pt}{0.400pt}}
\put(1429.0,289.0){\rule[-0.200pt]{2.409pt}{0.400pt}}
\put(170.0,289.0){\rule[-0.200pt]{2.409pt}{0.400pt}}
\put(1429.0,289.0){\rule[-0.200pt]{2.409pt}{0.400pt}}
\put(170.0,289.0){\rule[-0.200pt]{2.409pt}{0.400pt}}
\put(1429.0,289.0){\rule[-0.200pt]{2.409pt}{0.400pt}}
\put(170.0,289.0){\rule[-0.200pt]{2.409pt}{0.400pt}}
\put(1429.0,289.0){\rule[-0.200pt]{2.409pt}{0.400pt}}
\put(170.0,289.0){\rule[-0.200pt]{2.409pt}{0.400pt}}
\put(1429.0,289.0){\rule[-0.200pt]{2.409pt}{0.400pt}}
\put(170.0,289.0){\rule[-0.200pt]{2.409pt}{0.400pt}}
\put(1429.0,289.0){\rule[-0.200pt]{2.409pt}{0.400pt}}
\put(170.0,289.0){\rule[-0.200pt]{2.409pt}{0.400pt}}
\put(1429.0,289.0){\rule[-0.200pt]{2.409pt}{0.400pt}}
\put(170.0,290.0){\rule[-0.200pt]{2.409pt}{0.400pt}}
\put(1429.0,290.0){\rule[-0.200pt]{2.409pt}{0.400pt}}
\put(170.0,290.0){\rule[-0.200pt]{2.409pt}{0.400pt}}
\put(1429.0,290.0){\rule[-0.200pt]{2.409pt}{0.400pt}}
\put(170.0,290.0){\rule[-0.200pt]{2.409pt}{0.400pt}}
\put(1429.0,290.0){\rule[-0.200pt]{2.409pt}{0.400pt}}
\put(170.0,290.0){\rule[-0.200pt]{2.409pt}{0.400pt}}
\put(1429.0,290.0){\rule[-0.200pt]{2.409pt}{0.400pt}}
\put(170.0,290.0){\rule[-0.200pt]{2.409pt}{0.400pt}}
\put(1429.0,290.0){\rule[-0.200pt]{2.409pt}{0.400pt}}
\put(170.0,290.0){\rule[-0.200pt]{2.409pt}{0.400pt}}
\put(1429.0,290.0){\rule[-0.200pt]{2.409pt}{0.400pt}}
\put(170.0,290.0){\rule[-0.200pt]{2.409pt}{0.400pt}}
\put(1429.0,290.0){\rule[-0.200pt]{2.409pt}{0.400pt}}
\put(170.0,291.0){\rule[-0.200pt]{2.409pt}{0.400pt}}
\put(1429.0,291.0){\rule[-0.200pt]{2.409pt}{0.400pt}}
\put(170.0,291.0){\rule[-0.200pt]{2.409pt}{0.400pt}}
\put(1429.0,291.0){\rule[-0.200pt]{2.409pt}{0.400pt}}
\put(170.0,291.0){\rule[-0.200pt]{2.409pt}{0.400pt}}
\put(1429.0,291.0){\rule[-0.200pt]{2.409pt}{0.400pt}}
\put(170.0,291.0){\rule[-0.200pt]{2.409pt}{0.400pt}}
\put(1429.0,291.0){\rule[-0.200pt]{2.409pt}{0.400pt}}
\put(170.0,291.0){\rule[-0.200pt]{2.409pt}{0.400pt}}
\put(1429.0,291.0){\rule[-0.200pt]{2.409pt}{0.400pt}}
\put(170.0,291.0){\rule[-0.200pt]{2.409pt}{0.400pt}}
\put(1429.0,291.0){\rule[-0.200pt]{2.409pt}{0.400pt}}
\put(170.0,291.0){\rule[-0.200pt]{2.409pt}{0.400pt}}
\put(1429.0,291.0){\rule[-0.200pt]{2.409pt}{0.400pt}}
\put(170.0,292.0){\rule[-0.200pt]{2.409pt}{0.400pt}}
\put(1429.0,292.0){\rule[-0.200pt]{2.409pt}{0.400pt}}
\put(170.0,292.0){\rule[-0.200pt]{2.409pt}{0.400pt}}
\put(1429.0,292.0){\rule[-0.200pt]{2.409pt}{0.400pt}}
\put(170.0,292.0){\rule[-0.200pt]{2.409pt}{0.400pt}}
\put(1429.0,292.0){\rule[-0.200pt]{2.409pt}{0.400pt}}
\put(170.0,292.0){\rule[-0.200pt]{2.409pt}{0.400pt}}
\put(1429.0,292.0){\rule[-0.200pt]{2.409pt}{0.400pt}}
\put(170.0,292.0){\rule[-0.200pt]{2.409pt}{0.400pt}}
\put(1429.0,292.0){\rule[-0.200pt]{2.409pt}{0.400pt}}
\put(170.0,292.0){\rule[-0.200pt]{2.409pt}{0.400pt}}
\put(1429.0,292.0){\rule[-0.200pt]{2.409pt}{0.400pt}}
\put(170.0,292.0){\rule[-0.200pt]{2.409pt}{0.400pt}}
\put(1429.0,292.0){\rule[-0.200pt]{2.409pt}{0.400pt}}
\put(170.0,293.0){\rule[-0.200pt]{2.409pt}{0.400pt}}
\put(1429.0,293.0){\rule[-0.200pt]{2.409pt}{0.400pt}}
\put(170.0,293.0){\rule[-0.200pt]{2.409pt}{0.400pt}}
\put(1429.0,293.0){\rule[-0.200pt]{2.409pt}{0.400pt}}
\put(170.0,293.0){\rule[-0.200pt]{2.409pt}{0.400pt}}
\put(1429.0,293.0){\rule[-0.200pt]{2.409pt}{0.400pt}}
\put(170.0,293.0){\rule[-0.200pt]{2.409pt}{0.400pt}}
\put(1429.0,293.0){\rule[-0.200pt]{2.409pt}{0.400pt}}
\put(170.0,293.0){\rule[-0.200pt]{2.409pt}{0.400pt}}
\put(1429.0,293.0){\rule[-0.200pt]{2.409pt}{0.400pt}}
\put(170.0,293.0){\rule[-0.200pt]{2.409pt}{0.400pt}}
\put(1429.0,293.0){\rule[-0.200pt]{2.409pt}{0.400pt}}
\put(170.0,293.0){\rule[-0.200pt]{2.409pt}{0.400pt}}
\put(1429.0,293.0){\rule[-0.200pt]{2.409pt}{0.400pt}}
\put(170.0,294.0){\rule[-0.200pt]{2.409pt}{0.400pt}}
\put(1429.0,294.0){\rule[-0.200pt]{2.409pt}{0.400pt}}
\put(170.0,294.0){\rule[-0.200pt]{2.409pt}{0.400pt}}
\put(1429.0,294.0){\rule[-0.200pt]{2.409pt}{0.400pt}}
\put(170.0,294.0){\rule[-0.200pt]{2.409pt}{0.400pt}}
\put(1429.0,294.0){\rule[-0.200pt]{2.409pt}{0.400pt}}
\put(170.0,294.0){\rule[-0.200pt]{2.409pt}{0.400pt}}
\put(1429.0,294.0){\rule[-0.200pt]{2.409pt}{0.400pt}}
\put(170.0,294.0){\rule[-0.200pt]{2.409pt}{0.400pt}}
\put(1429.0,294.0){\rule[-0.200pt]{2.409pt}{0.400pt}}
\put(170.0,294.0){\rule[-0.200pt]{2.409pt}{0.400pt}}
\put(1429.0,294.0){\rule[-0.200pt]{2.409pt}{0.400pt}}
\put(170.0,294.0){\rule[-0.200pt]{2.409pt}{0.400pt}}
\put(1429.0,294.0){\rule[-0.200pt]{2.409pt}{0.400pt}}
\put(170.0,294.0){\rule[-0.200pt]{2.409pt}{0.400pt}}
\put(1429.0,294.0){\rule[-0.200pt]{2.409pt}{0.400pt}}
\put(170.0,295.0){\rule[-0.200pt]{2.409pt}{0.400pt}}
\put(1429.0,295.0){\rule[-0.200pt]{2.409pt}{0.400pt}}
\put(170.0,295.0){\rule[-0.200pt]{2.409pt}{0.400pt}}
\put(1429.0,295.0){\rule[-0.200pt]{2.409pt}{0.400pt}}
\put(170.0,295.0){\rule[-0.200pt]{2.409pt}{0.400pt}}
\put(1429.0,295.0){\rule[-0.200pt]{2.409pt}{0.400pt}}
\put(170.0,295.0){\rule[-0.200pt]{2.409pt}{0.400pt}}
\put(1429.0,295.0){\rule[-0.200pt]{2.409pt}{0.400pt}}
\put(170.0,295.0){\rule[-0.200pt]{2.409pt}{0.400pt}}
\put(1429.0,295.0){\rule[-0.200pt]{2.409pt}{0.400pt}}
\put(170.0,295.0){\rule[-0.200pt]{2.409pt}{0.400pt}}
\put(1429.0,295.0){\rule[-0.200pt]{2.409pt}{0.400pt}}
\put(170.0,295.0){\rule[-0.200pt]{2.409pt}{0.400pt}}
\put(1429.0,295.0){\rule[-0.200pt]{2.409pt}{0.400pt}}
\put(170.0,295.0){\rule[-0.200pt]{2.409pt}{0.400pt}}
\put(1429.0,295.0){\rule[-0.200pt]{2.409pt}{0.400pt}}
\put(170.0,296.0){\rule[-0.200pt]{2.409pt}{0.400pt}}
\put(1429.0,296.0){\rule[-0.200pt]{2.409pt}{0.400pt}}
\put(170.0,296.0){\rule[-0.200pt]{2.409pt}{0.400pt}}
\put(1429.0,296.0){\rule[-0.200pt]{2.409pt}{0.400pt}}
\put(170.0,296.0){\rule[-0.200pt]{2.409pt}{0.400pt}}
\put(1429.0,296.0){\rule[-0.200pt]{2.409pt}{0.400pt}}
\put(170.0,296.0){\rule[-0.200pt]{2.409pt}{0.400pt}}
\put(1429.0,296.0){\rule[-0.200pt]{2.409pt}{0.400pt}}
\put(170.0,296.0){\rule[-0.200pt]{2.409pt}{0.400pt}}
\put(1429.0,296.0){\rule[-0.200pt]{2.409pt}{0.400pt}}
\put(170.0,296.0){\rule[-0.200pt]{2.409pt}{0.400pt}}
\put(1429.0,296.0){\rule[-0.200pt]{2.409pt}{0.400pt}}
\put(170.0,296.0){\rule[-0.200pt]{2.409pt}{0.400pt}}
\put(1429.0,296.0){\rule[-0.200pt]{2.409pt}{0.400pt}}
\put(170.0,296.0){\rule[-0.200pt]{2.409pt}{0.400pt}}
\put(1429.0,296.0){\rule[-0.200pt]{2.409pt}{0.400pt}}
\put(170.0,297.0){\rule[-0.200pt]{2.409pt}{0.400pt}}
\put(1429.0,297.0){\rule[-0.200pt]{2.409pt}{0.400pt}}
\put(170.0,297.0){\rule[-0.200pt]{2.409pt}{0.400pt}}
\put(1429.0,297.0){\rule[-0.200pt]{2.409pt}{0.400pt}}
\put(170.0,297.0){\rule[-0.200pt]{2.409pt}{0.400pt}}
\put(1429.0,297.0){\rule[-0.200pt]{2.409pt}{0.400pt}}
\put(170.0,297.0){\rule[-0.200pt]{2.409pt}{0.400pt}}
\put(1429.0,297.0){\rule[-0.200pt]{2.409pt}{0.400pt}}
\put(170.0,297.0){\rule[-0.200pt]{2.409pt}{0.400pt}}
\put(1429.0,297.0){\rule[-0.200pt]{2.409pt}{0.400pt}}
\put(170.0,297.0){\rule[-0.200pt]{2.409pt}{0.400pt}}
\put(1429.0,297.0){\rule[-0.200pt]{2.409pt}{0.400pt}}
\put(170.0,297.0){\rule[-0.200pt]{2.409pt}{0.400pt}}
\put(1429.0,297.0){\rule[-0.200pt]{2.409pt}{0.400pt}}
\put(170.0,297.0){\rule[-0.200pt]{2.409pt}{0.400pt}}
\put(1429.0,297.0){\rule[-0.200pt]{2.409pt}{0.400pt}}
\put(170.0,298.0){\rule[-0.200pt]{2.409pt}{0.400pt}}
\put(1429.0,298.0){\rule[-0.200pt]{2.409pt}{0.400pt}}
\put(170.0,298.0){\rule[-0.200pt]{2.409pt}{0.400pt}}
\put(1429.0,298.0){\rule[-0.200pt]{2.409pt}{0.400pt}}
\put(170.0,298.0){\rule[-0.200pt]{2.409pt}{0.400pt}}
\put(1429.0,298.0){\rule[-0.200pt]{2.409pt}{0.400pt}}
\put(170.0,298.0){\rule[-0.200pt]{2.409pt}{0.400pt}}
\put(1429.0,298.0){\rule[-0.200pt]{2.409pt}{0.400pt}}
\put(170.0,298.0){\rule[-0.200pt]{2.409pt}{0.400pt}}
\put(1429.0,298.0){\rule[-0.200pt]{2.409pt}{0.400pt}}
\put(170.0,298.0){\rule[-0.200pt]{2.409pt}{0.400pt}}
\put(1429.0,298.0){\rule[-0.200pt]{2.409pt}{0.400pt}}
\put(170.0,298.0){\rule[-0.200pt]{2.409pt}{0.400pt}}
\put(1429.0,298.0){\rule[-0.200pt]{2.409pt}{0.400pt}}
\put(170.0,298.0){\rule[-0.200pt]{2.409pt}{0.400pt}}
\put(1429.0,298.0){\rule[-0.200pt]{2.409pt}{0.400pt}}
\put(170.0,299.0){\rule[-0.200pt]{2.409pt}{0.400pt}}
\put(1429.0,299.0){\rule[-0.200pt]{2.409pt}{0.400pt}}
\put(170.0,299.0){\rule[-0.200pt]{2.409pt}{0.400pt}}
\put(1429.0,299.0){\rule[-0.200pt]{2.409pt}{0.400pt}}
\put(170.0,299.0){\rule[-0.200pt]{2.409pt}{0.400pt}}
\put(1429.0,299.0){\rule[-0.200pt]{2.409pt}{0.400pt}}
\put(170.0,299.0){\rule[-0.200pt]{2.409pt}{0.400pt}}
\put(1429.0,299.0){\rule[-0.200pt]{2.409pt}{0.400pt}}
\put(170.0,299.0){\rule[-0.200pt]{2.409pt}{0.400pt}}
\put(1429.0,299.0){\rule[-0.200pt]{2.409pt}{0.400pt}}
\put(170.0,299.0){\rule[-0.200pt]{2.409pt}{0.400pt}}
\put(1429.0,299.0){\rule[-0.200pt]{2.409pt}{0.400pt}}
\put(170.0,299.0){\rule[-0.200pt]{2.409pt}{0.400pt}}
\put(1429.0,299.0){\rule[-0.200pt]{2.409pt}{0.400pt}}
\put(170.0,299.0){\rule[-0.200pt]{2.409pt}{0.400pt}}
\put(1429.0,299.0){\rule[-0.200pt]{2.409pt}{0.400pt}}
\put(170.0,299.0){\rule[-0.200pt]{2.409pt}{0.400pt}}
\put(1429.0,299.0){\rule[-0.200pt]{2.409pt}{0.400pt}}
\put(170.0,300.0){\rule[-0.200pt]{2.409pt}{0.400pt}}
\put(1429.0,300.0){\rule[-0.200pt]{2.409pt}{0.400pt}}
\put(170.0,300.0){\rule[-0.200pt]{2.409pt}{0.400pt}}
\put(1429.0,300.0){\rule[-0.200pt]{2.409pt}{0.400pt}}
\put(170.0,300.0){\rule[-0.200pt]{2.409pt}{0.400pt}}
\put(1429.0,300.0){\rule[-0.200pt]{2.409pt}{0.400pt}}
\put(170.0,300.0){\rule[-0.200pt]{2.409pt}{0.400pt}}
\put(1429.0,300.0){\rule[-0.200pt]{2.409pt}{0.400pt}}
\put(170.0,300.0){\rule[-0.200pt]{2.409pt}{0.400pt}}
\put(1429.0,300.0){\rule[-0.200pt]{2.409pt}{0.400pt}}
\put(170.0,300.0){\rule[-0.200pt]{2.409pt}{0.400pt}}
\put(1429.0,300.0){\rule[-0.200pt]{2.409pt}{0.400pt}}
\put(170.0,300.0){\rule[-0.200pt]{2.409pt}{0.400pt}}
\put(1429.0,300.0){\rule[-0.200pt]{2.409pt}{0.400pt}}
\put(170.0,300.0){\rule[-0.200pt]{2.409pt}{0.400pt}}
\put(1429.0,300.0){\rule[-0.200pt]{2.409pt}{0.400pt}}
\put(170.0,300.0){\rule[-0.200pt]{2.409pt}{0.400pt}}
\put(1429.0,300.0){\rule[-0.200pt]{2.409pt}{0.400pt}}
\put(170.0,301.0){\rule[-0.200pt]{2.409pt}{0.400pt}}
\put(1429.0,301.0){\rule[-0.200pt]{2.409pt}{0.400pt}}
\put(170.0,301.0){\rule[-0.200pt]{2.409pt}{0.400pt}}
\put(1429.0,301.0){\rule[-0.200pt]{2.409pt}{0.400pt}}
\put(170.0,301.0){\rule[-0.200pt]{2.409pt}{0.400pt}}
\put(1429.0,301.0){\rule[-0.200pt]{2.409pt}{0.400pt}}
\put(170.0,301.0){\rule[-0.200pt]{2.409pt}{0.400pt}}
\put(1429.0,301.0){\rule[-0.200pt]{2.409pt}{0.400pt}}
\put(170.0,301.0){\rule[-0.200pt]{2.409pt}{0.400pt}}
\put(1429.0,301.0){\rule[-0.200pt]{2.409pt}{0.400pt}}
\put(170.0,301.0){\rule[-0.200pt]{2.409pt}{0.400pt}}
\put(1429.0,301.0){\rule[-0.200pt]{2.409pt}{0.400pt}}
\put(170.0,301.0){\rule[-0.200pt]{2.409pt}{0.400pt}}
\put(1429.0,301.0){\rule[-0.200pt]{2.409pt}{0.400pt}}
\put(170.0,301.0){\rule[-0.200pt]{2.409pt}{0.400pt}}
\put(1429.0,301.0){\rule[-0.200pt]{2.409pt}{0.400pt}}
\put(170.0,301.0){\rule[-0.200pt]{2.409pt}{0.400pt}}
\put(1429.0,301.0){\rule[-0.200pt]{2.409pt}{0.400pt}}
\put(170.0,302.0){\rule[-0.200pt]{2.409pt}{0.400pt}}
\put(1429.0,302.0){\rule[-0.200pt]{2.409pt}{0.400pt}}
\put(170.0,302.0){\rule[-0.200pt]{2.409pt}{0.400pt}}
\put(1429.0,302.0){\rule[-0.200pt]{2.409pt}{0.400pt}}
\put(170.0,302.0){\rule[-0.200pt]{2.409pt}{0.400pt}}
\put(1429.0,302.0){\rule[-0.200pt]{2.409pt}{0.400pt}}
\put(170.0,302.0){\rule[-0.200pt]{2.409pt}{0.400pt}}
\put(1429.0,302.0){\rule[-0.200pt]{2.409pt}{0.400pt}}
\put(170.0,302.0){\rule[-0.200pt]{2.409pt}{0.400pt}}
\put(1429.0,302.0){\rule[-0.200pt]{2.409pt}{0.400pt}}
\put(170.0,302.0){\rule[-0.200pt]{2.409pt}{0.400pt}}
\put(1429.0,302.0){\rule[-0.200pt]{2.409pt}{0.400pt}}
\put(170.0,302.0){\rule[-0.200pt]{2.409pt}{0.400pt}}
\put(1429.0,302.0){\rule[-0.200pt]{2.409pt}{0.400pt}}
\put(170.0,302.0){\rule[-0.200pt]{2.409pt}{0.400pt}}
\put(1429.0,302.0){\rule[-0.200pt]{2.409pt}{0.400pt}}
\put(170.0,302.0){\rule[-0.200pt]{2.409pt}{0.400pt}}
\put(1429.0,302.0){\rule[-0.200pt]{2.409pt}{0.400pt}}
\put(170.0,302.0){\rule[-0.200pt]{2.409pt}{0.400pt}}
\put(1429.0,302.0){\rule[-0.200pt]{2.409pt}{0.400pt}}
\put(170.0,303.0){\rule[-0.200pt]{2.409pt}{0.400pt}}
\put(1429.0,303.0){\rule[-0.200pt]{2.409pt}{0.400pt}}
\put(170.0,303.0){\rule[-0.200pt]{2.409pt}{0.400pt}}
\put(1429.0,303.0){\rule[-0.200pt]{2.409pt}{0.400pt}}
\put(170.0,303.0){\rule[-0.200pt]{2.409pt}{0.400pt}}
\put(1429.0,303.0){\rule[-0.200pt]{2.409pt}{0.400pt}}
\put(170.0,303.0){\rule[-0.200pt]{2.409pt}{0.400pt}}
\put(1429.0,303.0){\rule[-0.200pt]{2.409pt}{0.400pt}}
\put(170.0,303.0){\rule[-0.200pt]{2.409pt}{0.400pt}}
\put(1429.0,303.0){\rule[-0.200pt]{2.409pt}{0.400pt}}
\put(170.0,303.0){\rule[-0.200pt]{2.409pt}{0.400pt}}
\put(1429.0,303.0){\rule[-0.200pt]{2.409pt}{0.400pt}}
\put(170.0,303.0){\rule[-0.200pt]{2.409pt}{0.400pt}}
\put(1429.0,303.0){\rule[-0.200pt]{2.409pt}{0.400pt}}
\put(170.0,303.0){\rule[-0.200pt]{2.409pt}{0.400pt}}
\put(1429.0,303.0){\rule[-0.200pt]{2.409pt}{0.400pt}}
\put(170.0,303.0){\rule[-0.200pt]{2.409pt}{0.400pt}}
\put(1429.0,303.0){\rule[-0.200pt]{2.409pt}{0.400pt}}
\put(170.0,304.0){\rule[-0.200pt]{2.409pt}{0.400pt}}
\put(1429.0,304.0){\rule[-0.200pt]{2.409pt}{0.400pt}}
\put(170.0,304.0){\rule[-0.200pt]{2.409pt}{0.400pt}}
\put(1429.0,304.0){\rule[-0.200pt]{2.409pt}{0.400pt}}
\put(170.0,304.0){\rule[-0.200pt]{2.409pt}{0.400pt}}
\put(1429.0,304.0){\rule[-0.200pt]{2.409pt}{0.400pt}}
\put(170.0,304.0){\rule[-0.200pt]{2.409pt}{0.400pt}}
\put(1429.0,304.0){\rule[-0.200pt]{2.409pt}{0.400pt}}
\put(170.0,304.0){\rule[-0.200pt]{2.409pt}{0.400pt}}
\put(1429.0,304.0){\rule[-0.200pt]{2.409pt}{0.400pt}}
\put(170.0,304.0){\rule[-0.200pt]{2.409pt}{0.400pt}}
\put(1429.0,304.0){\rule[-0.200pt]{2.409pt}{0.400pt}}
\put(170.0,304.0){\rule[-0.200pt]{2.409pt}{0.400pt}}
\put(1429.0,304.0){\rule[-0.200pt]{2.409pt}{0.400pt}}
\put(170.0,304.0){\rule[-0.200pt]{2.409pt}{0.400pt}}
\put(1429.0,304.0){\rule[-0.200pt]{2.409pt}{0.400pt}}
\put(170.0,304.0){\rule[-0.200pt]{2.409pt}{0.400pt}}
\put(1429.0,304.0){\rule[-0.200pt]{2.409pt}{0.400pt}}
\put(170.0,304.0){\rule[-0.200pt]{2.409pt}{0.400pt}}
\put(1429.0,304.0){\rule[-0.200pt]{2.409pt}{0.400pt}}
\put(170.0,305.0){\rule[-0.200pt]{2.409pt}{0.400pt}}
\put(1429.0,305.0){\rule[-0.200pt]{2.409pt}{0.400pt}}
\put(170.0,305.0){\rule[-0.200pt]{2.409pt}{0.400pt}}
\put(1429.0,305.0){\rule[-0.200pt]{2.409pt}{0.400pt}}
\put(170.0,305.0){\rule[-0.200pt]{2.409pt}{0.400pt}}
\put(1429.0,305.0){\rule[-0.200pt]{2.409pt}{0.400pt}}
\put(170.0,305.0){\rule[-0.200pt]{2.409pt}{0.400pt}}
\put(1429.0,305.0){\rule[-0.200pt]{2.409pt}{0.400pt}}
\put(170.0,305.0){\rule[-0.200pt]{2.409pt}{0.400pt}}
\put(1429.0,305.0){\rule[-0.200pt]{2.409pt}{0.400pt}}
\put(170.0,305.0){\rule[-0.200pt]{2.409pt}{0.400pt}}
\put(1429.0,305.0){\rule[-0.200pt]{2.409pt}{0.400pt}}
\put(170.0,305.0){\rule[-0.200pt]{2.409pt}{0.400pt}}
\put(1429.0,305.0){\rule[-0.200pt]{2.409pt}{0.400pt}}
\put(170.0,305.0){\rule[-0.200pt]{2.409pt}{0.400pt}}
\put(1429.0,305.0){\rule[-0.200pt]{2.409pt}{0.400pt}}
\put(170.0,305.0){\rule[-0.200pt]{2.409pt}{0.400pt}}
\put(1429.0,305.0){\rule[-0.200pt]{2.409pt}{0.400pt}}
\put(170.0,305.0){\rule[-0.200pt]{2.409pt}{0.400pt}}
\put(1429.0,305.0){\rule[-0.200pt]{2.409pt}{0.400pt}}
\put(170.0,306.0){\rule[-0.200pt]{2.409pt}{0.400pt}}
\put(1429.0,306.0){\rule[-0.200pt]{2.409pt}{0.400pt}}
\put(170.0,306.0){\rule[-0.200pt]{2.409pt}{0.400pt}}
\put(1429.0,306.0){\rule[-0.200pt]{2.409pt}{0.400pt}}
\put(170.0,306.0){\rule[-0.200pt]{2.409pt}{0.400pt}}
\put(1429.0,306.0){\rule[-0.200pt]{2.409pt}{0.400pt}}
\put(170.0,306.0){\rule[-0.200pt]{2.409pt}{0.400pt}}
\put(1429.0,306.0){\rule[-0.200pt]{2.409pt}{0.400pt}}
\put(170.0,306.0){\rule[-0.200pt]{2.409pt}{0.400pt}}
\put(1429.0,306.0){\rule[-0.200pt]{2.409pt}{0.400pt}}
\put(170.0,306.0){\rule[-0.200pt]{2.409pt}{0.400pt}}
\put(1429.0,306.0){\rule[-0.200pt]{2.409pt}{0.400pt}}
\put(170.0,306.0){\rule[-0.200pt]{2.409pt}{0.400pt}}
\put(1429.0,306.0){\rule[-0.200pt]{2.409pt}{0.400pt}}
\put(170.0,306.0){\rule[-0.200pt]{2.409pt}{0.400pt}}
\put(1429.0,306.0){\rule[-0.200pt]{2.409pt}{0.400pt}}
\put(170.0,306.0){\rule[-0.200pt]{2.409pt}{0.400pt}}
\put(1429.0,306.0){\rule[-0.200pt]{2.409pt}{0.400pt}}
\put(170.0,306.0){\rule[-0.200pt]{2.409pt}{0.400pt}}
\put(1429.0,306.0){\rule[-0.200pt]{2.409pt}{0.400pt}}
\put(170.0,306.0){\rule[-0.200pt]{2.409pt}{0.400pt}}
\put(1429.0,306.0){\rule[-0.200pt]{2.409pt}{0.400pt}}
\put(170.0,307.0){\rule[-0.200pt]{2.409pt}{0.400pt}}
\put(1429.0,307.0){\rule[-0.200pt]{2.409pt}{0.400pt}}
\put(170.0,307.0){\rule[-0.200pt]{2.409pt}{0.400pt}}
\put(1429.0,307.0){\rule[-0.200pt]{2.409pt}{0.400pt}}
\put(170.0,307.0){\rule[-0.200pt]{2.409pt}{0.400pt}}
\put(1429.0,307.0){\rule[-0.200pt]{2.409pt}{0.400pt}}
\put(170.0,307.0){\rule[-0.200pt]{2.409pt}{0.400pt}}
\put(1429.0,307.0){\rule[-0.200pt]{2.409pt}{0.400pt}}
\put(170.0,307.0){\rule[-0.200pt]{2.409pt}{0.400pt}}
\put(1429.0,307.0){\rule[-0.200pt]{2.409pt}{0.400pt}}
\put(170.0,307.0){\rule[-0.200pt]{2.409pt}{0.400pt}}
\put(1429.0,307.0){\rule[-0.200pt]{2.409pt}{0.400pt}}
\put(170.0,307.0){\rule[-0.200pt]{2.409pt}{0.400pt}}
\put(1429.0,307.0){\rule[-0.200pt]{2.409pt}{0.400pt}}
\put(170.0,307.0){\rule[-0.200pt]{2.409pt}{0.400pt}}
\put(1429.0,307.0){\rule[-0.200pt]{2.409pt}{0.400pt}}
\put(170.0,307.0){\rule[-0.200pt]{2.409pt}{0.400pt}}
\put(1429.0,307.0){\rule[-0.200pt]{2.409pt}{0.400pt}}
\put(170.0,307.0){\rule[-0.200pt]{2.409pt}{0.400pt}}
\put(1429.0,307.0){\rule[-0.200pt]{2.409pt}{0.400pt}}
\put(170.0,307.0){\rule[-0.200pt]{2.409pt}{0.400pt}}
\put(1429.0,307.0){\rule[-0.200pt]{2.409pt}{0.400pt}}
\put(170.0,308.0){\rule[-0.200pt]{2.409pt}{0.400pt}}
\put(1429.0,308.0){\rule[-0.200pt]{2.409pt}{0.400pt}}
\put(170.0,308.0){\rule[-0.200pt]{2.409pt}{0.400pt}}
\put(1429.0,308.0){\rule[-0.200pt]{2.409pt}{0.400pt}}
\put(170.0,308.0){\rule[-0.200pt]{2.409pt}{0.400pt}}
\put(1429.0,308.0){\rule[-0.200pt]{2.409pt}{0.400pt}}
\put(170.0,308.0){\rule[-0.200pt]{2.409pt}{0.400pt}}
\put(1429.0,308.0){\rule[-0.200pt]{2.409pt}{0.400pt}}
\put(170.0,308.0){\rule[-0.200pt]{2.409pt}{0.400pt}}
\put(1429.0,308.0){\rule[-0.200pt]{2.409pt}{0.400pt}}
\put(170.0,308.0){\rule[-0.200pt]{2.409pt}{0.400pt}}
\put(1429.0,308.0){\rule[-0.200pt]{2.409pt}{0.400pt}}
\put(170.0,308.0){\rule[-0.200pt]{2.409pt}{0.400pt}}
\put(1429.0,308.0){\rule[-0.200pt]{2.409pt}{0.400pt}}
\put(170.0,308.0){\rule[-0.200pt]{2.409pt}{0.400pt}}
\put(1429.0,308.0){\rule[-0.200pt]{2.409pt}{0.400pt}}
\put(170.0,308.0){\rule[-0.200pt]{2.409pt}{0.400pt}}
\put(1429.0,308.0){\rule[-0.200pt]{2.409pt}{0.400pt}}
\put(170.0,308.0){\rule[-0.200pt]{2.409pt}{0.400pt}}
\put(1429.0,308.0){\rule[-0.200pt]{2.409pt}{0.400pt}}
\put(170.0,308.0){\rule[-0.200pt]{2.409pt}{0.400pt}}
\put(1429.0,308.0){\rule[-0.200pt]{2.409pt}{0.400pt}}
\put(170.0,309.0){\rule[-0.200pt]{2.409pt}{0.400pt}}
\put(1429.0,309.0){\rule[-0.200pt]{2.409pt}{0.400pt}}
\put(170.0,309.0){\rule[-0.200pt]{2.409pt}{0.400pt}}
\put(1429.0,309.0){\rule[-0.200pt]{2.409pt}{0.400pt}}
\put(170.0,309.0){\rule[-0.200pt]{2.409pt}{0.400pt}}
\put(1429.0,309.0){\rule[-0.200pt]{2.409pt}{0.400pt}}
\put(170.0,309.0){\rule[-0.200pt]{2.409pt}{0.400pt}}
\put(1429.0,309.0){\rule[-0.200pt]{2.409pt}{0.400pt}}
\put(170.0,309.0){\rule[-0.200pt]{2.409pt}{0.400pt}}
\put(1429.0,309.0){\rule[-0.200pt]{2.409pt}{0.400pt}}
\put(170.0,309.0){\rule[-0.200pt]{2.409pt}{0.400pt}}
\put(1429.0,309.0){\rule[-0.200pt]{2.409pt}{0.400pt}}
\put(170.0,309.0){\rule[-0.200pt]{2.409pt}{0.400pt}}
\put(1429.0,309.0){\rule[-0.200pt]{2.409pt}{0.400pt}}
\put(170.0,309.0){\rule[-0.200pt]{2.409pt}{0.400pt}}
\put(1429.0,309.0){\rule[-0.200pt]{2.409pt}{0.400pt}}
\put(170.0,309.0){\rule[-0.200pt]{2.409pt}{0.400pt}}
\put(1429.0,309.0){\rule[-0.200pt]{2.409pt}{0.400pt}}
\put(170.0,309.0){\rule[-0.200pt]{2.409pt}{0.400pt}}
\put(1429.0,309.0){\rule[-0.200pt]{2.409pt}{0.400pt}}
\put(170.0,309.0){\rule[-0.200pt]{2.409pt}{0.400pt}}
\put(1429.0,309.0){\rule[-0.200pt]{2.409pt}{0.400pt}}
\put(170.0,310.0){\rule[-0.200pt]{2.409pt}{0.400pt}}
\put(1429.0,310.0){\rule[-0.200pt]{2.409pt}{0.400pt}}
\put(170.0,310.0){\rule[-0.200pt]{2.409pt}{0.400pt}}
\put(1429.0,310.0){\rule[-0.200pt]{2.409pt}{0.400pt}}
\put(170.0,310.0){\rule[-0.200pt]{2.409pt}{0.400pt}}
\put(1429.0,310.0){\rule[-0.200pt]{2.409pt}{0.400pt}}
\put(170.0,310.0){\rule[-0.200pt]{2.409pt}{0.400pt}}
\put(1429.0,310.0){\rule[-0.200pt]{2.409pt}{0.400pt}}
\put(170.0,310.0){\rule[-0.200pt]{2.409pt}{0.400pt}}
\put(1429.0,310.0){\rule[-0.200pt]{2.409pt}{0.400pt}}
\put(170.0,310.0){\rule[-0.200pt]{2.409pt}{0.400pt}}
\put(1429.0,310.0){\rule[-0.200pt]{2.409pt}{0.400pt}}
\put(170.0,310.0){\rule[-0.200pt]{2.409pt}{0.400pt}}
\put(1429.0,310.0){\rule[-0.200pt]{2.409pt}{0.400pt}}
\put(170.0,310.0){\rule[-0.200pt]{2.409pt}{0.400pt}}
\put(1429.0,310.0){\rule[-0.200pt]{2.409pt}{0.400pt}}
\put(170.0,310.0){\rule[-0.200pt]{2.409pt}{0.400pt}}
\put(1429.0,310.0){\rule[-0.200pt]{2.409pt}{0.400pt}}
\put(170.0,310.0){\rule[-0.200pt]{2.409pt}{0.400pt}}
\put(1429.0,310.0){\rule[-0.200pt]{2.409pt}{0.400pt}}
\put(170.0,310.0){\rule[-0.200pt]{2.409pt}{0.400pt}}
\put(1429.0,310.0){\rule[-0.200pt]{2.409pt}{0.400pt}}
\put(170.0,310.0){\rule[-0.200pt]{2.409pt}{0.400pt}}
\put(1429.0,310.0){\rule[-0.200pt]{2.409pt}{0.400pt}}
\put(170.0,311.0){\rule[-0.200pt]{2.409pt}{0.400pt}}
\put(1429.0,311.0){\rule[-0.200pt]{2.409pt}{0.400pt}}
\put(170.0,311.0){\rule[-0.200pt]{2.409pt}{0.400pt}}
\put(1429.0,311.0){\rule[-0.200pt]{2.409pt}{0.400pt}}
\put(170.0,311.0){\rule[-0.200pt]{2.409pt}{0.400pt}}
\put(1429.0,311.0){\rule[-0.200pt]{2.409pt}{0.400pt}}
\put(170.0,311.0){\rule[-0.200pt]{2.409pt}{0.400pt}}
\put(1429.0,311.0){\rule[-0.200pt]{2.409pt}{0.400pt}}
\put(170.0,311.0){\rule[-0.200pt]{2.409pt}{0.400pt}}
\put(1429.0,311.0){\rule[-0.200pt]{2.409pt}{0.400pt}}
\put(170.0,311.0){\rule[-0.200pt]{2.409pt}{0.400pt}}
\put(1429.0,311.0){\rule[-0.200pt]{2.409pt}{0.400pt}}
\put(170.0,311.0){\rule[-0.200pt]{2.409pt}{0.400pt}}
\put(1429.0,311.0){\rule[-0.200pt]{2.409pt}{0.400pt}}
\put(170.0,311.0){\rule[-0.200pt]{2.409pt}{0.400pt}}
\put(1429.0,311.0){\rule[-0.200pt]{2.409pt}{0.400pt}}
\put(170.0,311.0){\rule[-0.200pt]{2.409pt}{0.400pt}}
\put(1429.0,311.0){\rule[-0.200pt]{2.409pt}{0.400pt}}
\put(170.0,311.0){\rule[-0.200pt]{2.409pt}{0.400pt}}
\put(1429.0,311.0){\rule[-0.200pt]{2.409pt}{0.400pt}}
\put(170.0,311.0){\rule[-0.200pt]{2.409pt}{0.400pt}}
\put(1429.0,311.0){\rule[-0.200pt]{2.409pt}{0.400pt}}
\put(170.0,311.0){\rule[-0.200pt]{2.409pt}{0.400pt}}
\put(1429.0,311.0){\rule[-0.200pt]{2.409pt}{0.400pt}}
\put(170.0,312.0){\rule[-0.200pt]{2.409pt}{0.400pt}}
\put(1429.0,312.0){\rule[-0.200pt]{2.409pt}{0.400pt}}
\put(170.0,312.0){\rule[-0.200pt]{2.409pt}{0.400pt}}
\put(1429.0,312.0){\rule[-0.200pt]{2.409pt}{0.400pt}}
\put(170.0,312.0){\rule[-0.200pt]{2.409pt}{0.400pt}}
\put(1429.0,312.0){\rule[-0.200pt]{2.409pt}{0.400pt}}
\put(170.0,312.0){\rule[-0.200pt]{2.409pt}{0.400pt}}
\put(1429.0,312.0){\rule[-0.200pt]{2.409pt}{0.400pt}}
\put(170.0,312.0){\rule[-0.200pt]{2.409pt}{0.400pt}}
\put(1429.0,312.0){\rule[-0.200pt]{2.409pt}{0.400pt}}
\put(170.0,312.0){\rule[-0.200pt]{2.409pt}{0.400pt}}
\put(1429.0,312.0){\rule[-0.200pt]{2.409pt}{0.400pt}}
\put(170.0,312.0){\rule[-0.200pt]{2.409pt}{0.400pt}}
\put(1429.0,312.0){\rule[-0.200pt]{2.409pt}{0.400pt}}
\put(170.0,312.0){\rule[-0.200pt]{2.409pt}{0.400pt}}
\put(1429.0,312.0){\rule[-0.200pt]{2.409pt}{0.400pt}}
\put(170.0,312.0){\rule[-0.200pt]{2.409pt}{0.400pt}}
\put(1429.0,312.0){\rule[-0.200pt]{2.409pt}{0.400pt}}
\put(170.0,312.0){\rule[-0.200pt]{2.409pt}{0.400pt}}
\put(1429.0,312.0){\rule[-0.200pt]{2.409pt}{0.400pt}}
\put(170.0,312.0){\rule[-0.200pt]{2.409pt}{0.400pt}}
\put(1429.0,312.0){\rule[-0.200pt]{2.409pt}{0.400pt}}
\put(170.0,312.0){\rule[-0.200pt]{2.409pt}{0.400pt}}
\put(1429.0,312.0){\rule[-0.200pt]{2.409pt}{0.400pt}}
\put(170.0,313.0){\rule[-0.200pt]{2.409pt}{0.400pt}}
\put(1429.0,313.0){\rule[-0.200pt]{2.409pt}{0.400pt}}
\put(170.0,313.0){\rule[-0.200pt]{2.409pt}{0.400pt}}
\put(1429.0,313.0){\rule[-0.200pt]{2.409pt}{0.400pt}}
\put(170.0,313.0){\rule[-0.200pt]{2.409pt}{0.400pt}}
\put(1429.0,313.0){\rule[-0.200pt]{2.409pt}{0.400pt}}
\put(170.0,313.0){\rule[-0.200pt]{2.409pt}{0.400pt}}
\put(1429.0,313.0){\rule[-0.200pt]{2.409pt}{0.400pt}}
\put(170.0,313.0){\rule[-0.200pt]{2.409pt}{0.400pt}}
\put(1429.0,313.0){\rule[-0.200pt]{2.409pt}{0.400pt}}
\put(170.0,313.0){\rule[-0.200pt]{2.409pt}{0.400pt}}
\put(1429.0,313.0){\rule[-0.200pt]{2.409pt}{0.400pt}}
\put(170.0,313.0){\rule[-0.200pt]{2.409pt}{0.400pt}}
\put(1429.0,313.0){\rule[-0.200pt]{2.409pt}{0.400pt}}
\put(170.0,313.0){\rule[-0.200pt]{2.409pt}{0.400pt}}
\put(1429.0,313.0){\rule[-0.200pt]{2.409pt}{0.400pt}}
\put(170.0,313.0){\rule[-0.200pt]{2.409pt}{0.400pt}}
\put(1429.0,313.0){\rule[-0.200pt]{2.409pt}{0.400pt}}
\put(170.0,313.0){\rule[-0.200pt]{2.409pt}{0.400pt}}
\put(1429.0,313.0){\rule[-0.200pt]{2.409pt}{0.400pt}}
\put(170.0,313.0){\rule[-0.200pt]{2.409pt}{0.400pt}}
\put(1429.0,313.0){\rule[-0.200pt]{2.409pt}{0.400pt}}
\put(170.0,313.0){\rule[-0.200pt]{2.409pt}{0.400pt}}
\put(1429.0,313.0){\rule[-0.200pt]{2.409pt}{0.400pt}}
\put(170.0,313.0){\rule[-0.200pt]{2.409pt}{0.400pt}}
\put(1429.0,313.0){\rule[-0.200pt]{2.409pt}{0.400pt}}
\put(170.0,314.0){\rule[-0.200pt]{2.409pt}{0.400pt}}
\put(1429.0,314.0){\rule[-0.200pt]{2.409pt}{0.400pt}}
\put(170.0,314.0){\rule[-0.200pt]{2.409pt}{0.400pt}}
\put(1429.0,314.0){\rule[-0.200pt]{2.409pt}{0.400pt}}
\put(170.0,314.0){\rule[-0.200pt]{2.409pt}{0.400pt}}
\put(1429.0,314.0){\rule[-0.200pt]{2.409pt}{0.400pt}}
\put(170.0,314.0){\rule[-0.200pt]{2.409pt}{0.400pt}}
\put(1429.0,314.0){\rule[-0.200pt]{2.409pt}{0.400pt}}
\put(170.0,314.0){\rule[-0.200pt]{2.409pt}{0.400pt}}
\put(1429.0,314.0){\rule[-0.200pt]{2.409pt}{0.400pt}}
\put(170.0,314.0){\rule[-0.200pt]{2.409pt}{0.400pt}}
\put(1429.0,314.0){\rule[-0.200pt]{2.409pt}{0.400pt}}
\put(170.0,314.0){\rule[-0.200pt]{2.409pt}{0.400pt}}
\put(1429.0,314.0){\rule[-0.200pt]{2.409pt}{0.400pt}}
\put(170.0,314.0){\rule[-0.200pt]{2.409pt}{0.400pt}}
\put(1429.0,314.0){\rule[-0.200pt]{2.409pt}{0.400pt}}
\put(170.0,314.0){\rule[-0.200pt]{2.409pt}{0.400pt}}
\put(1429.0,314.0){\rule[-0.200pt]{2.409pt}{0.400pt}}
\put(170.0,314.0){\rule[-0.200pt]{2.409pt}{0.400pt}}
\put(1429.0,314.0){\rule[-0.200pt]{2.409pt}{0.400pt}}
\put(170.0,314.0){\rule[-0.200pt]{2.409pt}{0.400pt}}
\put(1429.0,314.0){\rule[-0.200pt]{2.409pt}{0.400pt}}
\put(170.0,314.0){\rule[-0.200pt]{2.409pt}{0.400pt}}
\put(1429.0,314.0){\rule[-0.200pt]{2.409pt}{0.400pt}}
\put(170.0,314.0){\rule[-0.200pt]{2.409pt}{0.400pt}}
\put(1429.0,314.0){\rule[-0.200pt]{2.409pt}{0.400pt}}
\put(170.0,315.0){\rule[-0.200pt]{2.409pt}{0.400pt}}
\put(1429.0,315.0){\rule[-0.200pt]{2.409pt}{0.400pt}}
\put(170.0,315.0){\rule[-0.200pt]{2.409pt}{0.400pt}}
\put(1429.0,315.0){\rule[-0.200pt]{2.409pt}{0.400pt}}
\put(170.0,315.0){\rule[-0.200pt]{2.409pt}{0.400pt}}
\put(1429.0,315.0){\rule[-0.200pt]{2.409pt}{0.400pt}}
\put(170.0,315.0){\rule[-0.200pt]{2.409pt}{0.400pt}}
\put(1429.0,315.0){\rule[-0.200pt]{2.409pt}{0.400pt}}
\put(170.0,315.0){\rule[-0.200pt]{2.409pt}{0.400pt}}
\put(1429.0,315.0){\rule[-0.200pt]{2.409pt}{0.400pt}}
\put(170.0,315.0){\rule[-0.200pt]{2.409pt}{0.400pt}}
\put(1429.0,315.0){\rule[-0.200pt]{2.409pt}{0.400pt}}
\put(170.0,315.0){\rule[-0.200pt]{2.409pt}{0.400pt}}
\put(1429.0,315.0){\rule[-0.200pt]{2.409pt}{0.400pt}}
\put(170.0,315.0){\rule[-0.200pt]{2.409pt}{0.400pt}}
\put(1429.0,315.0){\rule[-0.200pt]{2.409pt}{0.400pt}}
\put(170.0,315.0){\rule[-0.200pt]{2.409pt}{0.400pt}}
\put(1429.0,315.0){\rule[-0.200pt]{2.409pt}{0.400pt}}
\put(170.0,315.0){\rule[-0.200pt]{2.409pt}{0.400pt}}
\put(1429.0,315.0){\rule[-0.200pt]{2.409pt}{0.400pt}}
\put(170.0,315.0){\rule[-0.200pt]{2.409pt}{0.400pt}}
\put(1429.0,315.0){\rule[-0.200pt]{2.409pt}{0.400pt}}
\put(170.0,315.0){\rule[-0.200pt]{2.409pt}{0.400pt}}
\put(1429.0,315.0){\rule[-0.200pt]{2.409pt}{0.400pt}}
\put(170.0,315.0){\rule[-0.200pt]{2.409pt}{0.400pt}}
\put(1429.0,315.0){\rule[-0.200pt]{2.409pt}{0.400pt}}
\put(170.0,316.0){\rule[-0.200pt]{2.409pt}{0.400pt}}
\put(1429.0,316.0){\rule[-0.200pt]{2.409pt}{0.400pt}}
\put(170.0,316.0){\rule[-0.200pt]{2.409pt}{0.400pt}}
\put(1429.0,316.0){\rule[-0.200pt]{2.409pt}{0.400pt}}
\put(170.0,316.0){\rule[-0.200pt]{2.409pt}{0.400pt}}
\put(1429.0,316.0){\rule[-0.200pt]{2.409pt}{0.400pt}}
\put(170.0,316.0){\rule[-0.200pt]{2.409pt}{0.400pt}}
\put(1429.0,316.0){\rule[-0.200pt]{2.409pt}{0.400pt}}
\put(170.0,316.0){\rule[-0.200pt]{2.409pt}{0.400pt}}
\put(1429.0,316.0){\rule[-0.200pt]{2.409pt}{0.400pt}}
\put(170.0,316.0){\rule[-0.200pt]{2.409pt}{0.400pt}}
\put(1429.0,316.0){\rule[-0.200pt]{2.409pt}{0.400pt}}
\put(170.0,316.0){\rule[-0.200pt]{2.409pt}{0.400pt}}
\put(1429.0,316.0){\rule[-0.200pt]{2.409pt}{0.400pt}}
\put(170.0,316.0){\rule[-0.200pt]{2.409pt}{0.400pt}}
\put(1429.0,316.0){\rule[-0.200pt]{2.409pt}{0.400pt}}
\put(170.0,316.0){\rule[-0.200pt]{2.409pt}{0.400pt}}
\put(1429.0,316.0){\rule[-0.200pt]{2.409pt}{0.400pt}}
\put(170.0,316.0){\rule[-0.200pt]{2.409pt}{0.400pt}}
\put(1429.0,316.0){\rule[-0.200pt]{2.409pt}{0.400pt}}
\put(170.0,316.0){\rule[-0.200pt]{2.409pt}{0.400pt}}
\put(1429.0,316.0){\rule[-0.200pt]{2.409pt}{0.400pt}}
\put(170.0,316.0){\rule[-0.200pt]{2.409pt}{0.400pt}}
\put(1429.0,316.0){\rule[-0.200pt]{2.409pt}{0.400pt}}
\put(170.0,316.0){\rule[-0.200pt]{2.409pt}{0.400pt}}
\put(1429.0,316.0){\rule[-0.200pt]{2.409pt}{0.400pt}}
\put(170.0,316.0){\rule[-0.200pt]{2.409pt}{0.400pt}}
\put(1429.0,316.0){\rule[-0.200pt]{2.409pt}{0.400pt}}
\put(170.0,317.0){\rule[-0.200pt]{2.409pt}{0.400pt}}
\put(1429.0,317.0){\rule[-0.200pt]{2.409pt}{0.400pt}}
\put(170.0,317.0){\rule[-0.200pt]{2.409pt}{0.400pt}}
\put(1429.0,317.0){\rule[-0.200pt]{2.409pt}{0.400pt}}
\put(170.0,317.0){\rule[-0.200pt]{2.409pt}{0.400pt}}
\put(1429.0,317.0){\rule[-0.200pt]{2.409pt}{0.400pt}}
\put(170.0,317.0){\rule[-0.200pt]{2.409pt}{0.400pt}}
\put(1429.0,317.0){\rule[-0.200pt]{2.409pt}{0.400pt}}
\put(170.0,317.0){\rule[-0.200pt]{2.409pt}{0.400pt}}
\put(1429.0,317.0){\rule[-0.200pt]{2.409pt}{0.400pt}}
\put(170.0,317.0){\rule[-0.200pt]{2.409pt}{0.400pt}}
\put(1429.0,317.0){\rule[-0.200pt]{2.409pt}{0.400pt}}
\put(170.0,317.0){\rule[-0.200pt]{2.409pt}{0.400pt}}
\put(1429.0,317.0){\rule[-0.200pt]{2.409pt}{0.400pt}}
\put(170.0,317.0){\rule[-0.200pt]{2.409pt}{0.400pt}}
\put(1429.0,317.0){\rule[-0.200pt]{2.409pt}{0.400pt}}
\put(170.0,317.0){\rule[-0.200pt]{2.409pt}{0.400pt}}
\put(1429.0,317.0){\rule[-0.200pt]{2.409pt}{0.400pt}}
\put(170.0,317.0){\rule[-0.200pt]{2.409pt}{0.400pt}}
\put(1429.0,317.0){\rule[-0.200pt]{2.409pt}{0.400pt}}
\put(170.0,317.0){\rule[-0.200pt]{2.409pt}{0.400pt}}
\put(1429.0,317.0){\rule[-0.200pt]{2.409pt}{0.400pt}}
\put(170.0,317.0){\rule[-0.200pt]{2.409pt}{0.400pt}}
\put(1429.0,317.0){\rule[-0.200pt]{2.409pt}{0.400pt}}
\put(170.0,317.0){\rule[-0.200pt]{2.409pt}{0.400pt}}
\put(1429.0,317.0){\rule[-0.200pt]{2.409pt}{0.400pt}}
\put(170.0,317.0){\rule[-0.200pt]{2.409pt}{0.400pt}}
\put(1429.0,317.0){\rule[-0.200pt]{2.409pt}{0.400pt}}
\put(170.0,318.0){\rule[-0.200pt]{2.409pt}{0.400pt}}
\put(1429.0,318.0){\rule[-0.200pt]{2.409pt}{0.400pt}}
\put(170.0,318.0){\rule[-0.200pt]{2.409pt}{0.400pt}}
\put(1429.0,318.0){\rule[-0.200pt]{2.409pt}{0.400pt}}
\put(170.0,318.0){\rule[-0.200pt]{2.409pt}{0.400pt}}
\put(1429.0,318.0){\rule[-0.200pt]{2.409pt}{0.400pt}}
\put(170.0,318.0){\rule[-0.200pt]{2.409pt}{0.400pt}}
\put(1429.0,318.0){\rule[-0.200pt]{2.409pt}{0.400pt}}
\put(170.0,318.0){\rule[-0.200pt]{2.409pt}{0.400pt}}
\put(1429.0,318.0){\rule[-0.200pt]{2.409pt}{0.400pt}}
\put(170.0,318.0){\rule[-0.200pt]{2.409pt}{0.400pt}}
\put(1429.0,318.0){\rule[-0.200pt]{2.409pt}{0.400pt}}
\put(170.0,318.0){\rule[-0.200pt]{2.409pt}{0.400pt}}
\put(1429.0,318.0){\rule[-0.200pt]{2.409pt}{0.400pt}}
\put(170.0,318.0){\rule[-0.200pt]{2.409pt}{0.400pt}}
\put(1429.0,318.0){\rule[-0.200pt]{2.409pt}{0.400pt}}
\put(170.0,318.0){\rule[-0.200pt]{2.409pt}{0.400pt}}
\put(1429.0,318.0){\rule[-0.200pt]{2.409pt}{0.400pt}}
\put(170.0,318.0){\rule[-0.200pt]{2.409pt}{0.400pt}}
\put(1429.0,318.0){\rule[-0.200pt]{2.409pt}{0.400pt}}
\put(170.0,318.0){\rule[-0.200pt]{2.409pt}{0.400pt}}
\put(1429.0,318.0){\rule[-0.200pt]{2.409pt}{0.400pt}}
\put(170.0,318.0){\rule[-0.200pt]{2.409pt}{0.400pt}}
\put(1429.0,318.0){\rule[-0.200pt]{2.409pt}{0.400pt}}
\put(170.0,318.0){\rule[-0.200pt]{2.409pt}{0.400pt}}
\put(1429.0,318.0){\rule[-0.200pt]{2.409pt}{0.400pt}}
\put(170.0,318.0){\rule[-0.200pt]{2.409pt}{0.400pt}}
\put(1429.0,318.0){\rule[-0.200pt]{2.409pt}{0.400pt}}
\put(170.0,319.0){\rule[-0.200pt]{2.409pt}{0.400pt}}
\put(1429.0,319.0){\rule[-0.200pt]{2.409pt}{0.400pt}}
\put(170.0,319.0){\rule[-0.200pt]{2.409pt}{0.400pt}}
\put(1429.0,319.0){\rule[-0.200pt]{2.409pt}{0.400pt}}
\put(170.0,319.0){\rule[-0.200pt]{2.409pt}{0.400pt}}
\put(1429.0,319.0){\rule[-0.200pt]{2.409pt}{0.400pt}}
\put(170.0,319.0){\rule[-0.200pt]{2.409pt}{0.400pt}}
\put(1429.0,319.0){\rule[-0.200pt]{2.409pt}{0.400pt}}
\put(170.0,319.0){\rule[-0.200pt]{2.409pt}{0.400pt}}
\put(1429.0,319.0){\rule[-0.200pt]{2.409pt}{0.400pt}}
\put(170.0,319.0){\rule[-0.200pt]{2.409pt}{0.400pt}}
\put(1429.0,319.0){\rule[-0.200pt]{2.409pt}{0.400pt}}
\put(170.0,319.0){\rule[-0.200pt]{2.409pt}{0.400pt}}
\put(1429.0,319.0){\rule[-0.200pt]{2.409pt}{0.400pt}}
\put(170.0,319.0){\rule[-0.200pt]{2.409pt}{0.400pt}}
\put(1429.0,319.0){\rule[-0.200pt]{2.409pt}{0.400pt}}
\put(170.0,319.0){\rule[-0.200pt]{2.409pt}{0.400pt}}
\put(1429.0,319.0){\rule[-0.200pt]{2.409pt}{0.400pt}}
\put(170.0,319.0){\rule[-0.200pt]{2.409pt}{0.400pt}}
\put(1429.0,319.0){\rule[-0.200pt]{2.409pt}{0.400pt}}
\put(170.0,319.0){\rule[-0.200pt]{2.409pt}{0.400pt}}
\put(1429.0,319.0){\rule[-0.200pt]{2.409pt}{0.400pt}}
\put(170.0,319.0){\rule[-0.200pt]{2.409pt}{0.400pt}}
\put(1429.0,319.0){\rule[-0.200pt]{2.409pt}{0.400pt}}
\put(170.0,319.0){\rule[-0.200pt]{2.409pt}{0.400pt}}
\put(1429.0,319.0){\rule[-0.200pt]{2.409pt}{0.400pt}}
\put(170.0,319.0){\rule[-0.200pt]{2.409pt}{0.400pt}}
\put(1429.0,319.0){\rule[-0.200pt]{2.409pt}{0.400pt}}
\put(170.0,319.0){\rule[-0.200pt]{2.409pt}{0.400pt}}
\put(1429.0,319.0){\rule[-0.200pt]{2.409pt}{0.400pt}}
\put(170.0,320.0){\rule[-0.200pt]{2.409pt}{0.400pt}}
\put(1429.0,320.0){\rule[-0.200pt]{2.409pt}{0.400pt}}
\put(170.0,320.0){\rule[-0.200pt]{2.409pt}{0.400pt}}
\put(1429.0,320.0){\rule[-0.200pt]{2.409pt}{0.400pt}}
\put(170.0,320.0){\rule[-0.200pt]{2.409pt}{0.400pt}}
\put(1429.0,320.0){\rule[-0.200pt]{2.409pt}{0.400pt}}
\put(170.0,320.0){\rule[-0.200pt]{2.409pt}{0.400pt}}
\put(1429.0,320.0){\rule[-0.200pt]{2.409pt}{0.400pt}}
\put(170.0,320.0){\rule[-0.200pt]{2.409pt}{0.400pt}}
\put(1429.0,320.0){\rule[-0.200pt]{2.409pt}{0.400pt}}
\put(170.0,320.0){\rule[-0.200pt]{2.409pt}{0.400pt}}
\put(1429.0,320.0){\rule[-0.200pt]{2.409pt}{0.400pt}}
\put(170.0,320.0){\rule[-0.200pt]{2.409pt}{0.400pt}}
\put(1429.0,320.0){\rule[-0.200pt]{2.409pt}{0.400pt}}
\put(170.0,320.0){\rule[-0.200pt]{2.409pt}{0.400pt}}
\put(1429.0,320.0){\rule[-0.200pt]{2.409pt}{0.400pt}}
\put(170.0,320.0){\rule[-0.200pt]{2.409pt}{0.400pt}}
\put(1429.0,320.0){\rule[-0.200pt]{2.409pt}{0.400pt}}
\put(170.0,320.0){\rule[-0.200pt]{2.409pt}{0.400pt}}
\put(1429.0,320.0){\rule[-0.200pt]{2.409pt}{0.400pt}}
\put(170.0,320.0){\rule[-0.200pt]{2.409pt}{0.400pt}}
\put(1429.0,320.0){\rule[-0.200pt]{2.409pt}{0.400pt}}
\put(170.0,320.0){\rule[-0.200pt]{2.409pt}{0.400pt}}
\put(1429.0,320.0){\rule[-0.200pt]{2.409pt}{0.400pt}}
\put(170.0,320.0){\rule[-0.200pt]{2.409pt}{0.400pt}}
\put(1429.0,320.0){\rule[-0.200pt]{2.409pt}{0.400pt}}
\put(170.0,320.0){\rule[-0.200pt]{2.409pt}{0.400pt}}
\put(1429.0,320.0){\rule[-0.200pt]{2.409pt}{0.400pt}}
\put(170.0,320.0){\rule[-0.200pt]{2.409pt}{0.400pt}}
\put(1429.0,320.0){\rule[-0.200pt]{2.409pt}{0.400pt}}
\put(170.0,321.0){\rule[-0.200pt]{2.409pt}{0.400pt}}
\put(1429.0,321.0){\rule[-0.200pt]{2.409pt}{0.400pt}}
\put(170.0,321.0){\rule[-0.200pt]{2.409pt}{0.400pt}}
\put(1429.0,321.0){\rule[-0.200pt]{2.409pt}{0.400pt}}
\put(170.0,321.0){\rule[-0.200pt]{2.409pt}{0.400pt}}
\put(1429.0,321.0){\rule[-0.200pt]{2.409pt}{0.400pt}}
\put(170.0,321.0){\rule[-0.200pt]{2.409pt}{0.400pt}}
\put(1429.0,321.0){\rule[-0.200pt]{2.409pt}{0.400pt}}
\put(170.0,321.0){\rule[-0.200pt]{2.409pt}{0.400pt}}
\put(1429.0,321.0){\rule[-0.200pt]{2.409pt}{0.400pt}}
\put(170.0,321.0){\rule[-0.200pt]{2.409pt}{0.400pt}}
\put(1429.0,321.0){\rule[-0.200pt]{2.409pt}{0.400pt}}
\put(170.0,321.0){\rule[-0.200pt]{2.409pt}{0.400pt}}
\put(1429.0,321.0){\rule[-0.200pt]{2.409pt}{0.400pt}}
\put(170.0,321.0){\rule[-0.200pt]{2.409pt}{0.400pt}}
\put(1429.0,321.0){\rule[-0.200pt]{2.409pt}{0.400pt}}
\put(170.0,321.0){\rule[-0.200pt]{2.409pt}{0.400pt}}
\put(1429.0,321.0){\rule[-0.200pt]{2.409pt}{0.400pt}}
\put(170.0,321.0){\rule[-0.200pt]{2.409pt}{0.400pt}}
\put(1429.0,321.0){\rule[-0.200pt]{2.409pt}{0.400pt}}
\put(170.0,321.0){\rule[-0.200pt]{2.409pt}{0.400pt}}
\put(1429.0,321.0){\rule[-0.200pt]{2.409pt}{0.400pt}}
\put(170.0,321.0){\rule[-0.200pt]{2.409pt}{0.400pt}}
\put(1429.0,321.0){\rule[-0.200pt]{2.409pt}{0.400pt}}
\put(170.0,321.0){\rule[-0.200pt]{2.409pt}{0.400pt}}
\put(1429.0,321.0){\rule[-0.200pt]{2.409pt}{0.400pt}}
\put(170.0,321.0){\rule[-0.200pt]{2.409pt}{0.400pt}}
\put(1429.0,321.0){\rule[-0.200pt]{2.409pt}{0.400pt}}
\put(170.0,321.0){\rule[-0.200pt]{2.409pt}{0.400pt}}
\put(1429.0,321.0){\rule[-0.200pt]{2.409pt}{0.400pt}}
\put(170.0,321.0){\rule[-0.200pt]{2.409pt}{0.400pt}}
\put(1429.0,321.0){\rule[-0.200pt]{2.409pt}{0.400pt}}
\put(170.0,322.0){\rule[-0.200pt]{2.409pt}{0.400pt}}
\put(1429.0,322.0){\rule[-0.200pt]{2.409pt}{0.400pt}}
\put(170.0,322.0){\rule[-0.200pt]{2.409pt}{0.400pt}}
\put(1429.0,322.0){\rule[-0.200pt]{2.409pt}{0.400pt}}
\put(170.0,322.0){\rule[-0.200pt]{2.409pt}{0.400pt}}
\put(1429.0,322.0){\rule[-0.200pt]{2.409pt}{0.400pt}}
\put(170.0,322.0){\rule[-0.200pt]{2.409pt}{0.400pt}}
\put(1429.0,322.0){\rule[-0.200pt]{2.409pt}{0.400pt}}
\put(170.0,322.0){\rule[-0.200pt]{2.409pt}{0.400pt}}
\put(1429.0,322.0){\rule[-0.200pt]{2.409pt}{0.400pt}}
\put(170.0,322.0){\rule[-0.200pt]{2.409pt}{0.400pt}}
\put(1429.0,322.0){\rule[-0.200pt]{2.409pt}{0.400pt}}
\put(170.0,322.0){\rule[-0.200pt]{2.409pt}{0.400pt}}
\put(1429.0,322.0){\rule[-0.200pt]{2.409pt}{0.400pt}}
\put(170.0,322.0){\rule[-0.200pt]{2.409pt}{0.400pt}}
\put(1429.0,322.0){\rule[-0.200pt]{2.409pt}{0.400pt}}
\put(170.0,322.0){\rule[-0.200pt]{2.409pt}{0.400pt}}
\put(1429.0,322.0){\rule[-0.200pt]{2.409pt}{0.400pt}}
\put(170.0,322.0){\rule[-0.200pt]{2.409pt}{0.400pt}}
\put(1429.0,322.0){\rule[-0.200pt]{2.409pt}{0.400pt}}
\put(170.0,322.0){\rule[-0.200pt]{2.409pt}{0.400pt}}
\put(1429.0,322.0){\rule[-0.200pt]{2.409pt}{0.400pt}}
\put(170.0,322.0){\rule[-0.200pt]{2.409pt}{0.400pt}}
\put(1429.0,322.0){\rule[-0.200pt]{2.409pt}{0.400pt}}
\put(170.0,322.0){\rule[-0.200pt]{2.409pt}{0.400pt}}
\put(1429.0,322.0){\rule[-0.200pt]{2.409pt}{0.400pt}}
\put(170.0,322.0){\rule[-0.200pt]{2.409pt}{0.400pt}}
\put(1429.0,322.0){\rule[-0.200pt]{2.409pt}{0.400pt}}
\put(170.0,322.0){\rule[-0.200pt]{2.409pt}{0.400pt}}
\put(1429.0,322.0){\rule[-0.200pt]{2.409pt}{0.400pt}}
\put(170.0,322.0){\rule[-0.200pt]{2.409pt}{0.400pt}}
\put(1429.0,322.0){\rule[-0.200pt]{2.409pt}{0.400pt}}
\put(170.0,323.0){\rule[-0.200pt]{2.409pt}{0.400pt}}
\put(1429.0,323.0){\rule[-0.200pt]{2.409pt}{0.400pt}}
\put(170.0,323.0){\rule[-0.200pt]{2.409pt}{0.400pt}}
\put(1429.0,323.0){\rule[-0.200pt]{2.409pt}{0.400pt}}
\put(170.0,323.0){\rule[-0.200pt]{2.409pt}{0.400pt}}
\put(1429.0,323.0){\rule[-0.200pt]{2.409pt}{0.400pt}}
\put(170.0,323.0){\rule[-0.200pt]{2.409pt}{0.400pt}}
\put(1429.0,323.0){\rule[-0.200pt]{2.409pt}{0.400pt}}
\put(170.0,323.0){\rule[-0.200pt]{2.409pt}{0.400pt}}
\put(1429.0,323.0){\rule[-0.200pt]{2.409pt}{0.400pt}}
\put(170.0,323.0){\rule[-0.200pt]{2.409pt}{0.400pt}}
\put(1429.0,323.0){\rule[-0.200pt]{2.409pt}{0.400pt}}
\put(170.0,323.0){\rule[-0.200pt]{2.409pt}{0.400pt}}
\put(1429.0,323.0){\rule[-0.200pt]{2.409pt}{0.400pt}}
\put(170.0,323.0){\rule[-0.200pt]{2.409pt}{0.400pt}}
\put(1429.0,323.0){\rule[-0.200pt]{2.409pt}{0.400pt}}
\put(170.0,323.0){\rule[-0.200pt]{2.409pt}{0.400pt}}
\put(1429.0,323.0){\rule[-0.200pt]{2.409pt}{0.400pt}}
\put(170.0,323.0){\rule[-0.200pt]{2.409pt}{0.400pt}}
\put(1429.0,323.0){\rule[-0.200pt]{2.409pt}{0.400pt}}
\put(170.0,323.0){\rule[-0.200pt]{2.409pt}{0.400pt}}
\put(1429.0,323.0){\rule[-0.200pt]{2.409pt}{0.400pt}}
\put(170.0,323.0){\rule[-0.200pt]{2.409pt}{0.400pt}}
\put(1429.0,323.0){\rule[-0.200pt]{2.409pt}{0.400pt}}
\put(170.0,323.0){\rule[-0.200pt]{2.409pt}{0.400pt}}
\put(1429.0,323.0){\rule[-0.200pt]{2.409pt}{0.400pt}}
\put(170.0,323.0){\rule[-0.200pt]{2.409pt}{0.400pt}}
\put(1429.0,323.0){\rule[-0.200pt]{2.409pt}{0.400pt}}
\put(170.0,323.0){\rule[-0.200pt]{2.409pt}{0.400pt}}
\put(1429.0,323.0){\rule[-0.200pt]{2.409pt}{0.400pt}}
\put(170.0,323.0){\rule[-0.200pt]{2.409pt}{0.400pt}}
\put(1429.0,323.0){\rule[-0.200pt]{2.409pt}{0.400pt}}
\put(170.0,323.0){\rule[-0.200pt]{2.409pt}{0.400pt}}
\put(1429.0,323.0){\rule[-0.200pt]{2.409pt}{0.400pt}}
\put(170.0,324.0){\rule[-0.200pt]{2.409pt}{0.400pt}}
\put(1429.0,324.0){\rule[-0.200pt]{2.409pt}{0.400pt}}
\put(170.0,324.0){\rule[-0.200pt]{2.409pt}{0.400pt}}
\put(1429.0,324.0){\rule[-0.200pt]{2.409pt}{0.400pt}}
\put(170.0,324.0){\rule[-0.200pt]{2.409pt}{0.400pt}}
\put(1429.0,324.0){\rule[-0.200pt]{2.409pt}{0.400pt}}
\put(170.0,324.0){\rule[-0.200pt]{2.409pt}{0.400pt}}
\put(1429.0,324.0){\rule[-0.200pt]{2.409pt}{0.400pt}}
\put(170.0,324.0){\rule[-0.200pt]{2.409pt}{0.400pt}}
\put(1429.0,324.0){\rule[-0.200pt]{2.409pt}{0.400pt}}
\put(170.0,324.0){\rule[-0.200pt]{2.409pt}{0.400pt}}
\put(1429.0,324.0){\rule[-0.200pt]{2.409pt}{0.400pt}}
\put(170.0,324.0){\rule[-0.200pt]{2.409pt}{0.400pt}}
\put(1429.0,324.0){\rule[-0.200pt]{2.409pt}{0.400pt}}
\put(170.0,324.0){\rule[-0.200pt]{2.409pt}{0.400pt}}
\put(1429.0,324.0){\rule[-0.200pt]{2.409pt}{0.400pt}}
\put(170.0,324.0){\rule[-0.200pt]{2.409pt}{0.400pt}}
\put(1429.0,324.0){\rule[-0.200pt]{2.409pt}{0.400pt}}
\put(170.0,324.0){\rule[-0.200pt]{2.409pt}{0.400pt}}
\put(1429.0,324.0){\rule[-0.200pt]{2.409pt}{0.400pt}}
\put(170.0,324.0){\rule[-0.200pt]{2.409pt}{0.400pt}}
\put(1429.0,324.0){\rule[-0.200pt]{2.409pt}{0.400pt}}
\put(170.0,324.0){\rule[-0.200pt]{2.409pt}{0.400pt}}
\put(1429.0,324.0){\rule[-0.200pt]{2.409pt}{0.400pt}}
\put(170.0,324.0){\rule[-0.200pt]{2.409pt}{0.400pt}}
\put(1429.0,324.0){\rule[-0.200pt]{2.409pt}{0.400pt}}
\put(170.0,324.0){\rule[-0.200pt]{2.409pt}{0.400pt}}
\put(1429.0,324.0){\rule[-0.200pt]{2.409pt}{0.400pt}}
\put(170.0,324.0){\rule[-0.200pt]{2.409pt}{0.400pt}}
\put(1429.0,324.0){\rule[-0.200pt]{2.409pt}{0.400pt}}
\put(170.0,324.0){\rule[-0.200pt]{2.409pt}{0.400pt}}
\put(1429.0,324.0){\rule[-0.200pt]{2.409pt}{0.400pt}}
\put(170.0,325.0){\rule[-0.200pt]{2.409pt}{0.400pt}}
\put(1429.0,325.0){\rule[-0.200pt]{2.409pt}{0.400pt}}
\put(170.0,325.0){\rule[-0.200pt]{2.409pt}{0.400pt}}
\put(1429.0,325.0){\rule[-0.200pt]{2.409pt}{0.400pt}}
\put(170.0,325.0){\rule[-0.200pt]{2.409pt}{0.400pt}}
\put(1429.0,325.0){\rule[-0.200pt]{2.409pt}{0.400pt}}
\put(170.0,325.0){\rule[-0.200pt]{2.409pt}{0.400pt}}
\put(1429.0,325.0){\rule[-0.200pt]{2.409pt}{0.400pt}}
\put(170.0,325.0){\rule[-0.200pt]{2.409pt}{0.400pt}}
\put(1429.0,325.0){\rule[-0.200pt]{2.409pt}{0.400pt}}
\put(170.0,325.0){\rule[-0.200pt]{2.409pt}{0.400pt}}
\put(1429.0,325.0){\rule[-0.200pt]{2.409pt}{0.400pt}}
\put(170.0,325.0){\rule[-0.200pt]{2.409pt}{0.400pt}}
\put(1429.0,325.0){\rule[-0.200pt]{2.409pt}{0.400pt}}
\put(170.0,325.0){\rule[-0.200pt]{2.409pt}{0.400pt}}
\put(1429.0,325.0){\rule[-0.200pt]{2.409pt}{0.400pt}}
\put(170.0,325.0){\rule[-0.200pt]{2.409pt}{0.400pt}}
\put(1429.0,325.0){\rule[-0.200pt]{2.409pt}{0.400pt}}
\put(170.0,325.0){\rule[-0.200pt]{2.409pt}{0.400pt}}
\put(1429.0,325.0){\rule[-0.200pt]{2.409pt}{0.400pt}}
\put(170.0,325.0){\rule[-0.200pt]{2.409pt}{0.400pt}}
\put(1429.0,325.0){\rule[-0.200pt]{2.409pt}{0.400pt}}
\put(170.0,325.0){\rule[-0.200pt]{2.409pt}{0.400pt}}
\put(1429.0,325.0){\rule[-0.200pt]{2.409pt}{0.400pt}}
\put(170.0,325.0){\rule[-0.200pt]{2.409pt}{0.400pt}}
\put(1429.0,325.0){\rule[-0.200pt]{2.409pt}{0.400pt}}
\put(170.0,325.0){\rule[-0.200pt]{2.409pt}{0.400pt}}
\put(1429.0,325.0){\rule[-0.200pt]{2.409pt}{0.400pt}}
\put(170.0,325.0){\rule[-0.200pt]{2.409pt}{0.400pt}}
\put(1429.0,325.0){\rule[-0.200pt]{2.409pt}{0.400pt}}
\put(170.0,325.0){\rule[-0.200pt]{2.409pt}{0.400pt}}
\put(1429.0,325.0){\rule[-0.200pt]{2.409pt}{0.400pt}}
\put(170.0,325.0){\rule[-0.200pt]{2.409pt}{0.400pt}}
\put(1429.0,325.0){\rule[-0.200pt]{2.409pt}{0.400pt}}
\put(170.0,325.0){\rule[-0.200pt]{2.409pt}{0.400pt}}
\put(1429.0,325.0){\rule[-0.200pt]{2.409pt}{0.400pt}}
\put(170.0,326.0){\rule[-0.200pt]{2.409pt}{0.400pt}}
\put(1429.0,326.0){\rule[-0.200pt]{2.409pt}{0.400pt}}
\put(170.0,326.0){\rule[-0.200pt]{2.409pt}{0.400pt}}
\put(1429.0,326.0){\rule[-0.200pt]{2.409pt}{0.400pt}}
\put(170.0,326.0){\rule[-0.200pt]{2.409pt}{0.400pt}}
\put(1429.0,326.0){\rule[-0.200pt]{2.409pt}{0.400pt}}
\put(170.0,326.0){\rule[-0.200pt]{2.409pt}{0.400pt}}
\put(1429.0,326.0){\rule[-0.200pt]{2.409pt}{0.400pt}}
\put(170.0,326.0){\rule[-0.200pt]{2.409pt}{0.400pt}}
\put(1429.0,326.0){\rule[-0.200pt]{2.409pt}{0.400pt}}
\put(170.0,326.0){\rule[-0.200pt]{2.409pt}{0.400pt}}
\put(1429.0,326.0){\rule[-0.200pt]{2.409pt}{0.400pt}}
\put(170.0,326.0){\rule[-0.200pt]{2.409pt}{0.400pt}}
\put(1429.0,326.0){\rule[-0.200pt]{2.409pt}{0.400pt}}
\put(170.0,326.0){\rule[-0.200pt]{2.409pt}{0.400pt}}
\put(1429.0,326.0){\rule[-0.200pt]{2.409pt}{0.400pt}}
\put(170.0,326.0){\rule[-0.200pt]{2.409pt}{0.400pt}}
\put(1429.0,326.0){\rule[-0.200pt]{2.409pt}{0.400pt}}
\put(170.0,326.0){\rule[-0.200pt]{2.409pt}{0.400pt}}
\put(1429.0,326.0){\rule[-0.200pt]{2.409pt}{0.400pt}}
\put(170.0,326.0){\rule[-0.200pt]{2.409pt}{0.400pt}}
\put(1429.0,326.0){\rule[-0.200pt]{2.409pt}{0.400pt}}
\put(170.0,326.0){\rule[-0.200pt]{2.409pt}{0.400pt}}
\put(1429.0,326.0){\rule[-0.200pt]{2.409pt}{0.400pt}}
\put(170.0,326.0){\rule[-0.200pt]{2.409pt}{0.400pt}}
\put(1429.0,326.0){\rule[-0.200pt]{2.409pt}{0.400pt}}
\put(170.0,326.0){\rule[-0.200pt]{2.409pt}{0.400pt}}
\put(1429.0,326.0){\rule[-0.200pt]{2.409pt}{0.400pt}}
\put(170.0,326.0){\rule[-0.200pt]{2.409pt}{0.400pt}}
\put(1429.0,326.0){\rule[-0.200pt]{2.409pt}{0.400pt}}
\put(170.0,326.0){\rule[-0.200pt]{2.409pt}{0.400pt}}
\put(1429.0,326.0){\rule[-0.200pt]{2.409pt}{0.400pt}}
\put(170.0,326.0){\rule[-0.200pt]{2.409pt}{0.400pt}}
\put(1429.0,326.0){\rule[-0.200pt]{2.409pt}{0.400pt}}
\put(170.0,326.0){\rule[-0.200pt]{2.409pt}{0.400pt}}
\put(1429.0,326.0){\rule[-0.200pt]{2.409pt}{0.400pt}}
\put(170.0,327.0){\rule[-0.200pt]{2.409pt}{0.400pt}}
\put(1429.0,327.0){\rule[-0.200pt]{2.409pt}{0.400pt}}
\put(170.0,327.0){\rule[-0.200pt]{2.409pt}{0.400pt}}
\put(1429.0,327.0){\rule[-0.200pt]{2.409pt}{0.400pt}}
\put(170.0,327.0){\rule[-0.200pt]{2.409pt}{0.400pt}}
\put(1429.0,327.0){\rule[-0.200pt]{2.409pt}{0.400pt}}
\put(170.0,327.0){\rule[-0.200pt]{2.409pt}{0.400pt}}
\put(1429.0,327.0){\rule[-0.200pt]{2.409pt}{0.400pt}}
\put(170.0,327.0){\rule[-0.200pt]{2.409pt}{0.400pt}}
\put(1429.0,327.0){\rule[-0.200pt]{2.409pt}{0.400pt}}
\put(170.0,327.0){\rule[-0.200pt]{2.409pt}{0.400pt}}
\put(1429.0,327.0){\rule[-0.200pt]{2.409pt}{0.400pt}}
\put(170.0,327.0){\rule[-0.200pt]{2.409pt}{0.400pt}}
\put(1429.0,327.0){\rule[-0.200pt]{2.409pt}{0.400pt}}
\put(170.0,327.0){\rule[-0.200pt]{2.409pt}{0.400pt}}
\put(1429.0,327.0){\rule[-0.200pt]{2.409pt}{0.400pt}}
\put(170.0,327.0){\rule[-0.200pt]{2.409pt}{0.400pt}}
\put(1429.0,327.0){\rule[-0.200pt]{2.409pt}{0.400pt}}
\put(170.0,327.0){\rule[-0.200pt]{2.409pt}{0.400pt}}
\put(1429.0,327.0){\rule[-0.200pt]{2.409pt}{0.400pt}}
\put(170.0,327.0){\rule[-0.200pt]{2.409pt}{0.400pt}}
\put(1429.0,327.0){\rule[-0.200pt]{2.409pt}{0.400pt}}
\put(170.0,327.0){\rule[-0.200pt]{2.409pt}{0.400pt}}
\put(1429.0,327.0){\rule[-0.200pt]{2.409pt}{0.400pt}}
\put(170.0,327.0){\rule[-0.200pt]{2.409pt}{0.400pt}}
\put(1429.0,327.0){\rule[-0.200pt]{2.409pt}{0.400pt}}
\put(170.0,327.0){\rule[-0.200pt]{2.409pt}{0.400pt}}
\put(1429.0,327.0){\rule[-0.200pt]{2.409pt}{0.400pt}}
\put(170.0,327.0){\rule[-0.200pt]{2.409pt}{0.400pt}}
\put(1429.0,327.0){\rule[-0.200pt]{2.409pt}{0.400pt}}
\put(170.0,327.0){\rule[-0.200pt]{2.409pt}{0.400pt}}
\put(1429.0,327.0){\rule[-0.200pt]{2.409pt}{0.400pt}}
\put(170.0,327.0){\rule[-0.200pt]{2.409pt}{0.400pt}}
\put(1429.0,327.0){\rule[-0.200pt]{2.409pt}{0.400pt}}
\put(170.0,327.0){\rule[-0.200pt]{2.409pt}{0.400pt}}
\put(1429.0,327.0){\rule[-0.200pt]{2.409pt}{0.400pt}}
\put(170.0,328.0){\rule[-0.200pt]{2.409pt}{0.400pt}}
\put(1429.0,328.0){\rule[-0.200pt]{2.409pt}{0.400pt}}
\put(170.0,328.0){\rule[-0.200pt]{2.409pt}{0.400pt}}
\put(1429.0,328.0){\rule[-0.200pt]{2.409pt}{0.400pt}}
\put(170.0,328.0){\rule[-0.200pt]{2.409pt}{0.400pt}}
\put(1429.0,328.0){\rule[-0.200pt]{2.409pt}{0.400pt}}
\put(170.0,328.0){\rule[-0.200pt]{2.409pt}{0.400pt}}
\put(1429.0,328.0){\rule[-0.200pt]{2.409pt}{0.400pt}}
\put(170.0,328.0){\rule[-0.200pt]{2.409pt}{0.400pt}}
\put(1429.0,328.0){\rule[-0.200pt]{2.409pt}{0.400pt}}
\put(170.0,328.0){\rule[-0.200pt]{2.409pt}{0.400pt}}
\put(1429.0,328.0){\rule[-0.200pt]{2.409pt}{0.400pt}}
\put(170.0,328.0){\rule[-0.200pt]{2.409pt}{0.400pt}}
\put(1429.0,328.0){\rule[-0.200pt]{2.409pt}{0.400pt}}
\put(170.0,328.0){\rule[-0.200pt]{2.409pt}{0.400pt}}
\put(1429.0,328.0){\rule[-0.200pt]{2.409pt}{0.400pt}}
\put(170.0,328.0){\rule[-0.200pt]{2.409pt}{0.400pt}}
\put(1429.0,328.0){\rule[-0.200pt]{2.409pt}{0.400pt}}
\put(170.0,328.0){\rule[-0.200pt]{2.409pt}{0.400pt}}
\put(1429.0,328.0){\rule[-0.200pt]{2.409pt}{0.400pt}}
\put(170.0,328.0){\rule[-0.200pt]{2.409pt}{0.400pt}}
\put(1429.0,328.0){\rule[-0.200pt]{2.409pt}{0.400pt}}
\put(170.0,328.0){\rule[-0.200pt]{2.409pt}{0.400pt}}
\put(1429.0,328.0){\rule[-0.200pt]{2.409pt}{0.400pt}}
\put(170.0,328.0){\rule[-0.200pt]{2.409pt}{0.400pt}}
\put(1429.0,328.0){\rule[-0.200pt]{2.409pt}{0.400pt}}
\put(170.0,328.0){\rule[-0.200pt]{2.409pt}{0.400pt}}
\put(1429.0,328.0){\rule[-0.200pt]{2.409pt}{0.400pt}}
\put(170.0,328.0){\rule[-0.200pt]{2.409pt}{0.400pt}}
\put(1429.0,328.0){\rule[-0.200pt]{2.409pt}{0.400pt}}
\put(170.0,328.0){\rule[-0.200pt]{2.409pt}{0.400pt}}
\put(1429.0,328.0){\rule[-0.200pt]{2.409pt}{0.400pt}}
\put(170.0,328.0){\rule[-0.200pt]{2.409pt}{0.400pt}}
\put(1429.0,328.0){\rule[-0.200pt]{2.409pt}{0.400pt}}
\put(170.0,328.0){\rule[-0.200pt]{2.409pt}{0.400pt}}
\put(1429.0,328.0){\rule[-0.200pt]{2.409pt}{0.400pt}}
\put(170.0,328.0){\rule[-0.200pt]{2.409pt}{0.400pt}}
\put(1429.0,328.0){\rule[-0.200pt]{2.409pt}{0.400pt}}
\put(170.0,329.0){\rule[-0.200pt]{2.409pt}{0.400pt}}
\put(1429.0,329.0){\rule[-0.200pt]{2.409pt}{0.400pt}}
\put(170.0,329.0){\rule[-0.200pt]{2.409pt}{0.400pt}}
\put(1429.0,329.0){\rule[-0.200pt]{2.409pt}{0.400pt}}
\put(170.0,329.0){\rule[-0.200pt]{2.409pt}{0.400pt}}
\put(1429.0,329.0){\rule[-0.200pt]{2.409pt}{0.400pt}}
\put(170.0,329.0){\rule[-0.200pt]{2.409pt}{0.400pt}}
\put(1429.0,329.0){\rule[-0.200pt]{2.409pt}{0.400pt}}
\put(170.0,329.0){\rule[-0.200pt]{2.409pt}{0.400pt}}
\put(1429.0,329.0){\rule[-0.200pt]{2.409pt}{0.400pt}}
\put(170.0,329.0){\rule[-0.200pt]{2.409pt}{0.400pt}}
\put(1429.0,329.0){\rule[-0.200pt]{2.409pt}{0.400pt}}
\put(170.0,329.0){\rule[-0.200pt]{2.409pt}{0.400pt}}
\put(1429.0,329.0){\rule[-0.200pt]{2.409pt}{0.400pt}}
\put(170.0,329.0){\rule[-0.200pt]{2.409pt}{0.400pt}}
\put(1429.0,329.0){\rule[-0.200pt]{2.409pt}{0.400pt}}
\put(170.0,329.0){\rule[-0.200pt]{2.409pt}{0.400pt}}
\put(1429.0,329.0){\rule[-0.200pt]{2.409pt}{0.400pt}}
\put(170.0,329.0){\rule[-0.200pt]{2.409pt}{0.400pt}}
\put(1429.0,329.0){\rule[-0.200pt]{2.409pt}{0.400pt}}
\put(170.0,329.0){\rule[-0.200pt]{2.409pt}{0.400pt}}
\put(1429.0,329.0){\rule[-0.200pt]{2.409pt}{0.400pt}}
\put(170.0,329.0){\rule[-0.200pt]{2.409pt}{0.400pt}}
\put(1429.0,329.0){\rule[-0.200pt]{2.409pt}{0.400pt}}
\put(170.0,329.0){\rule[-0.200pt]{2.409pt}{0.400pt}}
\put(1429.0,329.0){\rule[-0.200pt]{2.409pt}{0.400pt}}
\put(170.0,329.0){\rule[-0.200pt]{2.409pt}{0.400pt}}
\put(1429.0,329.0){\rule[-0.200pt]{2.409pt}{0.400pt}}
\put(170.0,329.0){\rule[-0.200pt]{2.409pt}{0.400pt}}
\put(1429.0,329.0){\rule[-0.200pt]{2.409pt}{0.400pt}}
\put(170.0,329.0){\rule[-0.200pt]{2.409pt}{0.400pt}}
\put(1429.0,329.0){\rule[-0.200pt]{2.409pt}{0.400pt}}
\put(170.0,329.0){\rule[-0.200pt]{2.409pt}{0.400pt}}
\put(1429.0,329.0){\rule[-0.200pt]{2.409pt}{0.400pt}}
\put(170.0,329.0){\rule[-0.200pt]{2.409pt}{0.400pt}}
\put(1429.0,329.0){\rule[-0.200pt]{2.409pt}{0.400pt}}
\put(170.0,329.0){\rule[-0.200pt]{2.409pt}{0.400pt}}
\put(1429.0,329.0){\rule[-0.200pt]{2.409pt}{0.400pt}}
\put(170.0,330.0){\rule[-0.200pt]{2.409pt}{0.400pt}}
\put(1429.0,330.0){\rule[-0.200pt]{2.409pt}{0.400pt}}
\put(170.0,330.0){\rule[-0.200pt]{2.409pt}{0.400pt}}
\put(1429.0,330.0){\rule[-0.200pt]{2.409pt}{0.400pt}}
\put(170.0,330.0){\rule[-0.200pt]{2.409pt}{0.400pt}}
\put(1429.0,330.0){\rule[-0.200pt]{2.409pt}{0.400pt}}
\put(170.0,330.0){\rule[-0.200pt]{2.409pt}{0.400pt}}
\put(1429.0,330.0){\rule[-0.200pt]{2.409pt}{0.400pt}}
\put(170.0,330.0){\rule[-0.200pt]{2.409pt}{0.400pt}}
\put(1429.0,330.0){\rule[-0.200pt]{2.409pt}{0.400pt}}
\put(170.0,330.0){\rule[-0.200pt]{2.409pt}{0.400pt}}
\put(1429.0,330.0){\rule[-0.200pt]{2.409pt}{0.400pt}}
\put(170.0,330.0){\rule[-0.200pt]{2.409pt}{0.400pt}}
\put(1429.0,330.0){\rule[-0.200pt]{2.409pt}{0.400pt}}
\put(170.0,330.0){\rule[-0.200pt]{2.409pt}{0.400pt}}
\put(1429.0,330.0){\rule[-0.200pt]{2.409pt}{0.400pt}}
\put(170.0,330.0){\rule[-0.200pt]{2.409pt}{0.400pt}}
\put(1429.0,330.0){\rule[-0.200pt]{2.409pt}{0.400pt}}
\put(170.0,330.0){\rule[-0.200pt]{2.409pt}{0.400pt}}
\put(1429.0,330.0){\rule[-0.200pt]{2.409pt}{0.400pt}}
\put(170.0,330.0){\rule[-0.200pt]{2.409pt}{0.400pt}}
\put(1429.0,330.0){\rule[-0.200pt]{2.409pt}{0.400pt}}
\put(170.0,330.0){\rule[-0.200pt]{2.409pt}{0.400pt}}
\put(1429.0,330.0){\rule[-0.200pt]{2.409pt}{0.400pt}}
\put(170.0,330.0){\rule[-0.200pt]{2.409pt}{0.400pt}}
\put(1429.0,330.0){\rule[-0.200pt]{2.409pt}{0.400pt}}
\put(170.0,330.0){\rule[-0.200pt]{2.409pt}{0.400pt}}
\put(1429.0,330.0){\rule[-0.200pt]{2.409pt}{0.400pt}}
\put(170.0,330.0){\rule[-0.200pt]{2.409pt}{0.400pt}}
\put(1429.0,330.0){\rule[-0.200pt]{2.409pt}{0.400pt}}
\put(170.0,330.0){\rule[-0.200pt]{2.409pt}{0.400pt}}
\put(1429.0,330.0){\rule[-0.200pt]{2.409pt}{0.400pt}}
\put(170.0,330.0){\rule[-0.200pt]{2.409pt}{0.400pt}}
\put(1429.0,330.0){\rule[-0.200pt]{2.409pt}{0.400pt}}
\put(170.0,330.0){\rule[-0.200pt]{2.409pt}{0.400pt}}
\put(1429.0,330.0){\rule[-0.200pt]{2.409pt}{0.400pt}}
\put(170.0,330.0){\rule[-0.200pt]{2.409pt}{0.400pt}}
\put(1429.0,330.0){\rule[-0.200pt]{2.409pt}{0.400pt}}
\put(170.0,330.0){\rule[-0.200pt]{2.409pt}{0.400pt}}
\put(1429.0,330.0){\rule[-0.200pt]{2.409pt}{0.400pt}}
\put(170.0,331.0){\rule[-0.200pt]{2.409pt}{0.400pt}}
\put(1429.0,331.0){\rule[-0.200pt]{2.409pt}{0.400pt}}
\put(170.0,331.0){\rule[-0.200pt]{2.409pt}{0.400pt}}
\put(1429.0,331.0){\rule[-0.200pt]{2.409pt}{0.400pt}}
\put(170.0,331.0){\rule[-0.200pt]{2.409pt}{0.400pt}}
\put(1429.0,331.0){\rule[-0.200pt]{2.409pt}{0.400pt}}
\put(170.0,331.0){\rule[-0.200pt]{2.409pt}{0.400pt}}
\put(1429.0,331.0){\rule[-0.200pt]{2.409pt}{0.400pt}}
\put(170.0,331.0){\rule[-0.200pt]{2.409pt}{0.400pt}}
\put(1429.0,331.0){\rule[-0.200pt]{2.409pt}{0.400pt}}
\put(170.0,331.0){\rule[-0.200pt]{2.409pt}{0.400pt}}
\put(1429.0,331.0){\rule[-0.200pt]{2.409pt}{0.400pt}}
\put(170.0,331.0){\rule[-0.200pt]{2.409pt}{0.400pt}}
\put(1429.0,331.0){\rule[-0.200pt]{2.409pt}{0.400pt}}
\put(170.0,331.0){\rule[-0.200pt]{2.409pt}{0.400pt}}
\put(1429.0,331.0){\rule[-0.200pt]{2.409pt}{0.400pt}}
\put(170.0,331.0){\rule[-0.200pt]{2.409pt}{0.400pt}}
\put(1429.0,331.0){\rule[-0.200pt]{2.409pt}{0.400pt}}
\put(170.0,331.0){\rule[-0.200pt]{2.409pt}{0.400pt}}
\put(1429.0,331.0){\rule[-0.200pt]{2.409pt}{0.400pt}}
\put(170.0,331.0){\rule[-0.200pt]{2.409pt}{0.400pt}}
\put(1429.0,331.0){\rule[-0.200pt]{2.409pt}{0.400pt}}
\put(170.0,331.0){\rule[-0.200pt]{2.409pt}{0.400pt}}
\put(1429.0,331.0){\rule[-0.200pt]{2.409pt}{0.400pt}}
\put(170.0,331.0){\rule[-0.200pt]{2.409pt}{0.400pt}}
\put(1429.0,331.0){\rule[-0.200pt]{2.409pt}{0.400pt}}
\put(170.0,331.0){\rule[-0.200pt]{2.409pt}{0.400pt}}
\put(1429.0,331.0){\rule[-0.200pt]{2.409pt}{0.400pt}}
\put(170.0,331.0){\rule[-0.200pt]{2.409pt}{0.400pt}}
\put(1429.0,331.0){\rule[-0.200pt]{2.409pt}{0.400pt}}
\put(170.0,331.0){\rule[-0.200pt]{2.409pt}{0.400pt}}
\put(1429.0,331.0){\rule[-0.200pt]{2.409pt}{0.400pt}}
\put(170.0,331.0){\rule[-0.200pt]{2.409pt}{0.400pt}}
\put(1429.0,331.0){\rule[-0.200pt]{2.409pt}{0.400pt}}
\put(170.0,331.0){\rule[-0.200pt]{2.409pt}{0.400pt}}
\put(1429.0,331.0){\rule[-0.200pt]{2.409pt}{0.400pt}}
\put(170.0,331.0){\rule[-0.200pt]{2.409pt}{0.400pt}}
\put(1429.0,331.0){\rule[-0.200pt]{2.409pt}{0.400pt}}
\put(170.0,331.0){\rule[-0.200pt]{2.409pt}{0.400pt}}
\put(1429.0,331.0){\rule[-0.200pt]{2.409pt}{0.400pt}}
\put(170.0,331.0){\rule[-0.200pt]{2.409pt}{0.400pt}}
\put(1429.0,331.0){\rule[-0.200pt]{2.409pt}{0.400pt}}
\put(170.0,332.0){\rule[-0.200pt]{2.409pt}{0.400pt}}
\put(1429.0,332.0){\rule[-0.200pt]{2.409pt}{0.400pt}}
\put(170.0,332.0){\rule[-0.200pt]{2.409pt}{0.400pt}}
\put(1429.0,332.0){\rule[-0.200pt]{2.409pt}{0.400pt}}
\put(170.0,332.0){\rule[-0.200pt]{2.409pt}{0.400pt}}
\put(1429.0,332.0){\rule[-0.200pt]{2.409pt}{0.400pt}}
\put(170.0,332.0){\rule[-0.200pt]{2.409pt}{0.400pt}}
\put(1429.0,332.0){\rule[-0.200pt]{2.409pt}{0.400pt}}
\put(170.0,332.0){\rule[-0.200pt]{2.409pt}{0.400pt}}
\put(1429.0,332.0){\rule[-0.200pt]{2.409pt}{0.400pt}}
\put(170.0,332.0){\rule[-0.200pt]{2.409pt}{0.400pt}}
\put(1429.0,332.0){\rule[-0.200pt]{2.409pt}{0.400pt}}
\put(170.0,332.0){\rule[-0.200pt]{2.409pt}{0.400pt}}
\put(1429.0,332.0){\rule[-0.200pt]{2.409pt}{0.400pt}}
\put(170.0,332.0){\rule[-0.200pt]{2.409pt}{0.400pt}}
\put(1429.0,332.0){\rule[-0.200pt]{2.409pt}{0.400pt}}
\put(170.0,332.0){\rule[-0.200pt]{2.409pt}{0.400pt}}
\put(1429.0,332.0){\rule[-0.200pt]{2.409pt}{0.400pt}}
\put(170.0,332.0){\rule[-0.200pt]{2.409pt}{0.400pt}}
\put(1429.0,332.0){\rule[-0.200pt]{2.409pt}{0.400pt}}
\put(170.0,332.0){\rule[-0.200pt]{2.409pt}{0.400pt}}
\put(1429.0,332.0){\rule[-0.200pt]{2.409pt}{0.400pt}}
\put(170.0,332.0){\rule[-0.200pt]{2.409pt}{0.400pt}}
\put(1429.0,332.0){\rule[-0.200pt]{2.409pt}{0.400pt}}
\put(170.0,332.0){\rule[-0.200pt]{2.409pt}{0.400pt}}
\put(1429.0,332.0){\rule[-0.200pt]{2.409pt}{0.400pt}}
\put(170.0,332.0){\rule[-0.200pt]{2.409pt}{0.400pt}}
\put(1429.0,332.0){\rule[-0.200pt]{2.409pt}{0.400pt}}
\put(170.0,332.0){\rule[-0.200pt]{2.409pt}{0.400pt}}
\put(1429.0,332.0){\rule[-0.200pt]{2.409pt}{0.400pt}}
\put(170.0,332.0){\rule[-0.200pt]{2.409pt}{0.400pt}}
\put(1429.0,332.0){\rule[-0.200pt]{2.409pt}{0.400pt}}
\put(170.0,332.0){\rule[-0.200pt]{2.409pt}{0.400pt}}
\put(1429.0,332.0){\rule[-0.200pt]{2.409pt}{0.400pt}}
\put(170.0,332.0){\rule[-0.200pt]{2.409pt}{0.400pt}}
\put(1429.0,332.0){\rule[-0.200pt]{2.409pt}{0.400pt}}
\put(170.0,332.0){\rule[-0.200pt]{2.409pt}{0.400pt}}
\put(1429.0,332.0){\rule[-0.200pt]{2.409pt}{0.400pt}}
\put(170.0,332.0){\rule[-0.200pt]{2.409pt}{0.400pt}}
\put(1429.0,332.0){\rule[-0.200pt]{2.409pt}{0.400pt}}
\put(170.0,332.0){\rule[-0.200pt]{2.409pt}{0.400pt}}
\put(1429.0,332.0){\rule[-0.200pt]{2.409pt}{0.400pt}}
\put(170.0,333.0){\rule[-0.200pt]{2.409pt}{0.400pt}}
\put(1429.0,333.0){\rule[-0.200pt]{2.409pt}{0.400pt}}
\put(170.0,333.0){\rule[-0.200pt]{2.409pt}{0.400pt}}
\put(1429.0,333.0){\rule[-0.200pt]{2.409pt}{0.400pt}}
\put(170.0,333.0){\rule[-0.200pt]{2.409pt}{0.400pt}}
\put(1429.0,333.0){\rule[-0.200pt]{2.409pt}{0.400pt}}
\put(170.0,333.0){\rule[-0.200pt]{2.409pt}{0.400pt}}
\put(1429.0,333.0){\rule[-0.200pt]{2.409pt}{0.400pt}}
\put(170.0,333.0){\rule[-0.200pt]{2.409pt}{0.400pt}}
\put(1429.0,333.0){\rule[-0.200pt]{2.409pt}{0.400pt}}
\put(170.0,333.0){\rule[-0.200pt]{2.409pt}{0.400pt}}
\put(1429.0,333.0){\rule[-0.200pt]{2.409pt}{0.400pt}}
\put(170.0,333.0){\rule[-0.200pt]{2.409pt}{0.400pt}}
\put(1429.0,333.0){\rule[-0.200pt]{2.409pt}{0.400pt}}
\put(170.0,333.0){\rule[-0.200pt]{2.409pt}{0.400pt}}
\put(1429.0,333.0){\rule[-0.200pt]{2.409pt}{0.400pt}}
\put(170.0,333.0){\rule[-0.200pt]{2.409pt}{0.400pt}}
\put(1429.0,333.0){\rule[-0.200pt]{2.409pt}{0.400pt}}
\put(170.0,333.0){\rule[-0.200pt]{2.409pt}{0.400pt}}
\put(1429.0,333.0){\rule[-0.200pt]{2.409pt}{0.400pt}}
\put(170.0,333.0){\rule[-0.200pt]{2.409pt}{0.400pt}}
\put(1429.0,333.0){\rule[-0.200pt]{2.409pt}{0.400pt}}
\put(170.0,333.0){\rule[-0.200pt]{2.409pt}{0.400pt}}
\put(1429.0,333.0){\rule[-0.200pt]{2.409pt}{0.400pt}}
\put(170.0,333.0){\rule[-0.200pt]{2.409pt}{0.400pt}}
\put(1429.0,333.0){\rule[-0.200pt]{2.409pt}{0.400pt}}
\put(170.0,333.0){\rule[-0.200pt]{2.409pt}{0.400pt}}
\put(1429.0,333.0){\rule[-0.200pt]{2.409pt}{0.400pt}}
\put(170.0,333.0){\rule[-0.200pt]{2.409pt}{0.400pt}}
\put(1429.0,333.0){\rule[-0.200pt]{2.409pt}{0.400pt}}
\put(170.0,333.0){\rule[-0.200pt]{2.409pt}{0.400pt}}
\put(1429.0,333.0){\rule[-0.200pt]{2.409pt}{0.400pt}}
\put(170.0,333.0){\rule[-0.200pt]{2.409pt}{0.400pt}}
\put(1429.0,333.0){\rule[-0.200pt]{2.409pt}{0.400pt}}
\put(170.0,333.0){\rule[-0.200pt]{2.409pt}{0.400pt}}
\put(1429.0,333.0){\rule[-0.200pt]{2.409pt}{0.400pt}}
\put(170.0,333.0){\rule[-0.200pt]{2.409pt}{0.400pt}}
\put(1429.0,333.0){\rule[-0.200pt]{2.409pt}{0.400pt}}
\put(170.0,333.0){\rule[-0.200pt]{2.409pt}{0.400pt}}
\put(1429.0,333.0){\rule[-0.200pt]{2.409pt}{0.400pt}}
\put(170.0,333.0){\rule[-0.200pt]{2.409pt}{0.400pt}}
\put(1429.0,333.0){\rule[-0.200pt]{2.409pt}{0.400pt}}
\put(170.0,334.0){\rule[-0.200pt]{2.409pt}{0.400pt}}
\put(1429.0,334.0){\rule[-0.200pt]{2.409pt}{0.400pt}}
\put(170.0,334.0){\rule[-0.200pt]{2.409pt}{0.400pt}}
\put(1429.0,334.0){\rule[-0.200pt]{2.409pt}{0.400pt}}
\put(170.0,334.0){\rule[-0.200pt]{2.409pt}{0.400pt}}
\put(1429.0,334.0){\rule[-0.200pt]{2.409pt}{0.400pt}}
\put(170.0,334.0){\rule[-0.200pt]{2.409pt}{0.400pt}}
\put(1429.0,334.0){\rule[-0.200pt]{2.409pt}{0.400pt}}
\put(170.0,334.0){\rule[-0.200pt]{2.409pt}{0.400pt}}
\put(1429.0,334.0){\rule[-0.200pt]{2.409pt}{0.400pt}}
\put(170.0,334.0){\rule[-0.200pt]{2.409pt}{0.400pt}}
\put(1429.0,334.0){\rule[-0.200pt]{2.409pt}{0.400pt}}
\put(170.0,334.0){\rule[-0.200pt]{2.409pt}{0.400pt}}
\put(1429.0,334.0){\rule[-0.200pt]{2.409pt}{0.400pt}}
\put(170.0,334.0){\rule[-0.200pt]{2.409pt}{0.400pt}}
\put(1429.0,334.0){\rule[-0.200pt]{2.409pt}{0.400pt}}
\put(170.0,334.0){\rule[-0.200pt]{2.409pt}{0.400pt}}
\put(1429.0,334.0){\rule[-0.200pt]{2.409pt}{0.400pt}}
\put(170.0,334.0){\rule[-0.200pt]{2.409pt}{0.400pt}}
\put(1429.0,334.0){\rule[-0.200pt]{2.409pt}{0.400pt}}
\put(170.0,334.0){\rule[-0.200pt]{2.409pt}{0.400pt}}
\put(1429.0,334.0){\rule[-0.200pt]{2.409pt}{0.400pt}}
\put(170.0,334.0){\rule[-0.200pt]{2.409pt}{0.400pt}}
\put(1429.0,334.0){\rule[-0.200pt]{2.409pt}{0.400pt}}
\put(170.0,334.0){\rule[-0.200pt]{2.409pt}{0.400pt}}
\put(1429.0,334.0){\rule[-0.200pt]{2.409pt}{0.400pt}}
\put(170.0,334.0){\rule[-0.200pt]{2.409pt}{0.400pt}}
\put(1429.0,334.0){\rule[-0.200pt]{2.409pt}{0.400pt}}
\put(170.0,334.0){\rule[-0.200pt]{2.409pt}{0.400pt}}
\put(1429.0,334.0){\rule[-0.200pt]{2.409pt}{0.400pt}}
\put(170.0,334.0){\rule[-0.200pt]{2.409pt}{0.400pt}}
\put(1429.0,334.0){\rule[-0.200pt]{2.409pt}{0.400pt}}
\put(170.0,334.0){\rule[-0.200pt]{2.409pt}{0.400pt}}
\put(1429.0,334.0){\rule[-0.200pt]{2.409pt}{0.400pt}}
\put(170.0,334.0){\rule[-0.200pt]{2.409pt}{0.400pt}}
\put(1429.0,334.0){\rule[-0.200pt]{2.409pt}{0.400pt}}
\put(170.0,334.0){\rule[-0.200pt]{2.409pt}{0.400pt}}
\put(1429.0,334.0){\rule[-0.200pt]{2.409pt}{0.400pt}}
\put(170.0,334.0){\rule[-0.200pt]{2.409pt}{0.400pt}}
\put(1429.0,334.0){\rule[-0.200pt]{2.409pt}{0.400pt}}
\put(170.0,334.0){\rule[-0.200pt]{2.409pt}{0.400pt}}
\put(1429.0,334.0){\rule[-0.200pt]{2.409pt}{0.400pt}}
\put(170.0,334.0){\rule[-0.200pt]{2.409pt}{0.400pt}}
\put(1429.0,334.0){\rule[-0.200pt]{2.409pt}{0.400pt}}
\put(170.0,335.0){\rule[-0.200pt]{2.409pt}{0.400pt}}
\put(1429.0,335.0){\rule[-0.200pt]{2.409pt}{0.400pt}}
\put(170.0,335.0){\rule[-0.200pt]{2.409pt}{0.400pt}}
\put(1429.0,335.0){\rule[-0.200pt]{2.409pt}{0.400pt}}
\put(170.0,335.0){\rule[-0.200pt]{2.409pt}{0.400pt}}
\put(1429.0,335.0){\rule[-0.200pt]{2.409pt}{0.400pt}}
\put(170.0,335.0){\rule[-0.200pt]{2.409pt}{0.400pt}}
\put(1429.0,335.0){\rule[-0.200pt]{2.409pt}{0.400pt}}
\put(170.0,335.0){\rule[-0.200pt]{2.409pt}{0.400pt}}
\put(1429.0,335.0){\rule[-0.200pt]{2.409pt}{0.400pt}}
\put(170.0,335.0){\rule[-0.200pt]{2.409pt}{0.400pt}}
\put(1429.0,335.0){\rule[-0.200pt]{2.409pt}{0.400pt}}
\put(170.0,335.0){\rule[-0.200pt]{2.409pt}{0.400pt}}
\put(1429.0,335.0){\rule[-0.200pt]{2.409pt}{0.400pt}}
\put(170.0,335.0){\rule[-0.200pt]{2.409pt}{0.400pt}}
\put(1429.0,335.0){\rule[-0.200pt]{2.409pt}{0.400pt}}
\put(170.0,335.0){\rule[-0.200pt]{2.409pt}{0.400pt}}
\put(1429.0,335.0){\rule[-0.200pt]{2.409pt}{0.400pt}}
\put(170.0,335.0){\rule[-0.200pt]{2.409pt}{0.400pt}}
\put(1429.0,335.0){\rule[-0.200pt]{2.409pt}{0.400pt}}
\put(170.0,335.0){\rule[-0.200pt]{2.409pt}{0.400pt}}
\put(1429.0,335.0){\rule[-0.200pt]{2.409pt}{0.400pt}}
\put(170.0,335.0){\rule[-0.200pt]{2.409pt}{0.400pt}}
\put(1429.0,335.0){\rule[-0.200pt]{2.409pt}{0.400pt}}
\put(170.0,335.0){\rule[-0.200pt]{2.409pt}{0.400pt}}
\put(1429.0,335.0){\rule[-0.200pt]{2.409pt}{0.400pt}}
\put(170.0,335.0){\rule[-0.200pt]{2.409pt}{0.400pt}}
\put(1429.0,335.0){\rule[-0.200pt]{2.409pt}{0.400pt}}
\put(170.0,335.0){\rule[-0.200pt]{2.409pt}{0.400pt}}
\put(1429.0,335.0){\rule[-0.200pt]{2.409pt}{0.400pt}}
\put(170.0,335.0){\rule[-0.200pt]{2.409pt}{0.400pt}}
\put(1429.0,335.0){\rule[-0.200pt]{2.409pt}{0.400pt}}
\put(170.0,335.0){\rule[-0.200pt]{2.409pt}{0.400pt}}
\put(1429.0,335.0){\rule[-0.200pt]{2.409pt}{0.400pt}}
\put(170.0,335.0){\rule[-0.200pt]{2.409pt}{0.400pt}}
\put(1429.0,335.0){\rule[-0.200pt]{2.409pt}{0.400pt}}
\put(170.0,335.0){\rule[-0.200pt]{2.409pt}{0.400pt}}
\put(1429.0,335.0){\rule[-0.200pt]{2.409pt}{0.400pt}}
\put(170.0,335.0){\rule[-0.200pt]{2.409pt}{0.400pt}}
\put(1429.0,335.0){\rule[-0.200pt]{2.409pt}{0.400pt}}
\put(170.0,335.0){\rule[-0.200pt]{2.409pt}{0.400pt}}
\put(1429.0,335.0){\rule[-0.200pt]{2.409pt}{0.400pt}}
\put(170.0,335.0){\rule[-0.200pt]{2.409pt}{0.400pt}}
\put(1429.0,335.0){\rule[-0.200pt]{2.409pt}{0.400pt}}
\put(170.0,335.0){\rule[-0.200pt]{2.409pt}{0.400pt}}
\put(1429.0,335.0){\rule[-0.200pt]{2.409pt}{0.400pt}}
\put(170.0,336.0){\rule[-0.200pt]{2.409pt}{0.400pt}}
\put(1429.0,336.0){\rule[-0.200pt]{2.409pt}{0.400pt}}
\put(170.0,336.0){\rule[-0.200pt]{2.409pt}{0.400pt}}
\put(1429.0,336.0){\rule[-0.200pt]{2.409pt}{0.400pt}}
\put(170.0,336.0){\rule[-0.200pt]{2.409pt}{0.400pt}}
\put(1429.0,336.0){\rule[-0.200pt]{2.409pt}{0.400pt}}
\put(170.0,336.0){\rule[-0.200pt]{2.409pt}{0.400pt}}
\put(1429.0,336.0){\rule[-0.200pt]{2.409pt}{0.400pt}}
\put(170.0,336.0){\rule[-0.200pt]{2.409pt}{0.400pt}}
\put(1429.0,336.0){\rule[-0.200pt]{2.409pt}{0.400pt}}
\put(170.0,336.0){\rule[-0.200pt]{2.409pt}{0.400pt}}
\put(1429.0,336.0){\rule[-0.200pt]{2.409pt}{0.400pt}}
\put(170.0,336.0){\rule[-0.200pt]{2.409pt}{0.400pt}}
\put(1429.0,336.0){\rule[-0.200pt]{2.409pt}{0.400pt}}
\put(170.0,336.0){\rule[-0.200pt]{2.409pt}{0.400pt}}
\put(1429.0,336.0){\rule[-0.200pt]{2.409pt}{0.400pt}}
\put(170.0,336.0){\rule[-0.200pt]{2.409pt}{0.400pt}}
\put(1429.0,336.0){\rule[-0.200pt]{2.409pt}{0.400pt}}
\put(170.0,336.0){\rule[-0.200pt]{2.409pt}{0.400pt}}
\put(1429.0,336.0){\rule[-0.200pt]{2.409pt}{0.400pt}}
\put(170.0,336.0){\rule[-0.200pt]{2.409pt}{0.400pt}}
\put(1429.0,336.0){\rule[-0.200pt]{2.409pt}{0.400pt}}
\put(170.0,336.0){\rule[-0.200pt]{2.409pt}{0.400pt}}
\put(1429.0,336.0){\rule[-0.200pt]{2.409pt}{0.400pt}}
\put(170.0,336.0){\rule[-0.200pt]{2.409pt}{0.400pt}}
\put(1429.0,336.0){\rule[-0.200pt]{2.409pt}{0.400pt}}
\put(170.0,336.0){\rule[-0.200pt]{2.409pt}{0.400pt}}
\put(1429.0,336.0){\rule[-0.200pt]{2.409pt}{0.400pt}}
\put(170.0,336.0){\rule[-0.200pt]{2.409pt}{0.400pt}}
\put(1429.0,336.0){\rule[-0.200pt]{2.409pt}{0.400pt}}
\put(170.0,336.0){\rule[-0.200pt]{2.409pt}{0.400pt}}
\put(1429.0,336.0){\rule[-0.200pt]{2.409pt}{0.400pt}}
\put(170.0,336.0){\rule[-0.200pt]{2.409pt}{0.400pt}}
\put(1429.0,336.0){\rule[-0.200pt]{2.409pt}{0.400pt}}
\put(170.0,336.0){\rule[-0.200pt]{2.409pt}{0.400pt}}
\put(1429.0,336.0){\rule[-0.200pt]{2.409pt}{0.400pt}}
\put(170.0,336.0){\rule[-0.200pt]{2.409pt}{0.400pt}}
\put(1429.0,336.0){\rule[-0.200pt]{2.409pt}{0.400pt}}
\put(170.0,336.0){\rule[-0.200pt]{2.409pt}{0.400pt}}
\put(1429.0,336.0){\rule[-0.200pt]{2.409pt}{0.400pt}}
\put(170.0,336.0){\rule[-0.200pt]{2.409pt}{0.400pt}}
\put(1429.0,336.0){\rule[-0.200pt]{2.409pt}{0.400pt}}
\put(170.0,336.0){\rule[-0.200pt]{2.409pt}{0.400pt}}
\put(1429.0,336.0){\rule[-0.200pt]{2.409pt}{0.400pt}}
\put(170.0,336.0){\rule[-0.200pt]{2.409pt}{0.400pt}}
\put(1429.0,336.0){\rule[-0.200pt]{2.409pt}{0.400pt}}
\put(170.0,337.0){\rule[-0.200pt]{2.409pt}{0.400pt}}
\put(1429.0,337.0){\rule[-0.200pt]{2.409pt}{0.400pt}}
\put(170.0,337.0){\rule[-0.200pt]{2.409pt}{0.400pt}}
\put(1429.0,337.0){\rule[-0.200pt]{2.409pt}{0.400pt}}
\put(170.0,337.0){\rule[-0.200pt]{2.409pt}{0.400pt}}
\put(1429.0,337.0){\rule[-0.200pt]{2.409pt}{0.400pt}}
\put(170.0,337.0){\rule[-0.200pt]{2.409pt}{0.400pt}}
\put(1429.0,337.0){\rule[-0.200pt]{2.409pt}{0.400pt}}
\put(170.0,337.0){\rule[-0.200pt]{2.409pt}{0.400pt}}
\put(1429.0,337.0){\rule[-0.200pt]{2.409pt}{0.400pt}}
\put(170.0,337.0){\rule[-0.200pt]{2.409pt}{0.400pt}}
\put(1429.0,337.0){\rule[-0.200pt]{2.409pt}{0.400pt}}
\put(170.0,337.0){\rule[-0.200pt]{2.409pt}{0.400pt}}
\put(1429.0,337.0){\rule[-0.200pt]{2.409pt}{0.400pt}}
\put(170.0,337.0){\rule[-0.200pt]{2.409pt}{0.400pt}}
\put(1429.0,337.0){\rule[-0.200pt]{2.409pt}{0.400pt}}
\put(170.0,337.0){\rule[-0.200pt]{2.409pt}{0.400pt}}
\put(1429.0,337.0){\rule[-0.200pt]{2.409pt}{0.400pt}}
\put(170.0,337.0){\rule[-0.200pt]{2.409pt}{0.400pt}}
\put(1429.0,337.0){\rule[-0.200pt]{2.409pt}{0.400pt}}
\put(170.0,337.0){\rule[-0.200pt]{2.409pt}{0.400pt}}
\put(1429.0,337.0){\rule[-0.200pt]{2.409pt}{0.400pt}}
\put(170.0,337.0){\rule[-0.200pt]{2.409pt}{0.400pt}}
\put(1429.0,337.0){\rule[-0.200pt]{2.409pt}{0.400pt}}
\put(170.0,337.0){\rule[-0.200pt]{2.409pt}{0.400pt}}
\put(1429.0,337.0){\rule[-0.200pt]{2.409pt}{0.400pt}}
\put(170.0,337.0){\rule[-0.200pt]{2.409pt}{0.400pt}}
\put(1429.0,337.0){\rule[-0.200pt]{2.409pt}{0.400pt}}
\put(170.0,337.0){\rule[-0.200pt]{2.409pt}{0.400pt}}
\put(1429.0,337.0){\rule[-0.200pt]{2.409pt}{0.400pt}}
\put(170.0,337.0){\rule[-0.200pt]{2.409pt}{0.400pt}}
\put(1429.0,337.0){\rule[-0.200pt]{2.409pt}{0.400pt}}
\put(170.0,337.0){\rule[-0.200pt]{2.409pt}{0.400pt}}
\put(1429.0,337.0){\rule[-0.200pt]{2.409pt}{0.400pt}}
\put(170.0,337.0){\rule[-0.200pt]{2.409pt}{0.400pt}}
\put(1429.0,337.0){\rule[-0.200pt]{2.409pt}{0.400pt}}
\put(170.0,337.0){\rule[-0.200pt]{2.409pt}{0.400pt}}
\put(1429.0,337.0){\rule[-0.200pt]{2.409pt}{0.400pt}}
\put(170.0,337.0){\rule[-0.200pt]{2.409pt}{0.400pt}}
\put(1429.0,337.0){\rule[-0.200pt]{2.409pt}{0.400pt}}
\put(170.0,337.0){\rule[-0.200pt]{2.409pt}{0.400pt}}
\put(1429.0,337.0){\rule[-0.200pt]{2.409pt}{0.400pt}}
\put(170.0,337.0){\rule[-0.200pt]{2.409pt}{0.400pt}}
\put(1429.0,337.0){\rule[-0.200pt]{2.409pt}{0.400pt}}
\put(170.0,337.0){\rule[-0.200pt]{2.409pt}{0.400pt}}
\put(1429.0,337.0){\rule[-0.200pt]{2.409pt}{0.400pt}}
\put(170.0,337.0){\rule[-0.200pt]{2.409pt}{0.400pt}}
\put(1429.0,337.0){\rule[-0.200pt]{2.409pt}{0.400pt}}
\put(170.0,338.0){\rule[-0.200pt]{2.409pt}{0.400pt}}
\put(1429.0,338.0){\rule[-0.200pt]{2.409pt}{0.400pt}}
\put(170.0,338.0){\rule[-0.200pt]{2.409pt}{0.400pt}}
\put(1429.0,338.0){\rule[-0.200pt]{2.409pt}{0.400pt}}
\put(170.0,338.0){\rule[-0.200pt]{2.409pt}{0.400pt}}
\put(1429.0,338.0){\rule[-0.200pt]{2.409pt}{0.400pt}}
\put(170.0,338.0){\rule[-0.200pt]{2.409pt}{0.400pt}}
\put(1429.0,338.0){\rule[-0.200pt]{2.409pt}{0.400pt}}
\put(170.0,338.0){\rule[-0.200pt]{2.409pt}{0.400pt}}
\put(1429.0,338.0){\rule[-0.200pt]{2.409pt}{0.400pt}}
\put(170.0,338.0){\rule[-0.200pt]{2.409pt}{0.400pt}}
\put(1429.0,338.0){\rule[-0.200pt]{2.409pt}{0.400pt}}
\put(170.0,338.0){\rule[-0.200pt]{2.409pt}{0.400pt}}
\put(1429.0,338.0){\rule[-0.200pt]{2.409pt}{0.400pt}}
\put(170.0,338.0){\rule[-0.200pt]{2.409pt}{0.400pt}}
\put(1429.0,338.0){\rule[-0.200pt]{2.409pt}{0.400pt}}
\put(170.0,338.0){\rule[-0.200pt]{2.409pt}{0.400pt}}
\put(1429.0,338.0){\rule[-0.200pt]{2.409pt}{0.400pt}}
\put(170.0,338.0){\rule[-0.200pt]{2.409pt}{0.400pt}}
\put(1429.0,338.0){\rule[-0.200pt]{2.409pt}{0.400pt}}
\put(170.0,338.0){\rule[-0.200pt]{2.409pt}{0.400pt}}
\put(1429.0,338.0){\rule[-0.200pt]{2.409pt}{0.400pt}}
\put(170.0,338.0){\rule[-0.200pt]{2.409pt}{0.400pt}}
\put(1429.0,338.0){\rule[-0.200pt]{2.409pt}{0.400pt}}
\put(170.0,338.0){\rule[-0.200pt]{2.409pt}{0.400pt}}
\put(1429.0,338.0){\rule[-0.200pt]{2.409pt}{0.400pt}}
\put(170.0,338.0){\rule[-0.200pt]{2.409pt}{0.400pt}}
\put(1429.0,338.0){\rule[-0.200pt]{2.409pt}{0.400pt}}
\put(170.0,338.0){\rule[-0.200pt]{2.409pt}{0.400pt}}
\put(1429.0,338.0){\rule[-0.200pt]{2.409pt}{0.400pt}}
\put(170.0,338.0){\rule[-0.200pt]{2.409pt}{0.400pt}}
\put(1429.0,338.0){\rule[-0.200pt]{2.409pt}{0.400pt}}
\put(170.0,338.0){\rule[-0.200pt]{2.409pt}{0.400pt}}
\put(1429.0,338.0){\rule[-0.200pt]{2.409pt}{0.400pt}}
\put(170.0,338.0){\rule[-0.200pt]{2.409pt}{0.400pt}}
\put(1429.0,338.0){\rule[-0.200pt]{2.409pt}{0.400pt}}
\put(170.0,338.0){\rule[-0.200pt]{2.409pt}{0.400pt}}
\put(1429.0,338.0){\rule[-0.200pt]{2.409pt}{0.400pt}}
\put(170.0,338.0){\rule[-0.200pt]{2.409pt}{0.400pt}}
\put(1429.0,338.0){\rule[-0.200pt]{2.409pt}{0.400pt}}
\put(170.0,338.0){\rule[-0.200pt]{2.409pt}{0.400pt}}
\put(1429.0,338.0){\rule[-0.200pt]{2.409pt}{0.400pt}}
\put(170.0,338.0){\rule[-0.200pt]{2.409pt}{0.400pt}}
\put(1429.0,338.0){\rule[-0.200pt]{2.409pt}{0.400pt}}
\put(170.0,338.0){\rule[-0.200pt]{2.409pt}{0.400pt}}
\put(1429.0,338.0){\rule[-0.200pt]{2.409pt}{0.400pt}}
\put(170.0,338.0){\rule[-0.200pt]{2.409pt}{0.400pt}}
\put(1429.0,338.0){\rule[-0.200pt]{2.409pt}{0.400pt}}
\put(170.0,338.0){\rule[-0.200pt]{2.409pt}{0.400pt}}
\put(1429.0,338.0){\rule[-0.200pt]{2.409pt}{0.400pt}}
\put(170.0,339.0){\rule[-0.200pt]{2.409pt}{0.400pt}}
\put(1429.0,339.0){\rule[-0.200pt]{2.409pt}{0.400pt}}
\put(170.0,339.0){\rule[-0.200pt]{2.409pt}{0.400pt}}
\put(1429.0,339.0){\rule[-0.200pt]{2.409pt}{0.400pt}}
\put(170.0,339.0){\rule[-0.200pt]{2.409pt}{0.400pt}}
\put(1429.0,339.0){\rule[-0.200pt]{2.409pt}{0.400pt}}
\put(170.0,339.0){\rule[-0.200pt]{2.409pt}{0.400pt}}
\put(1429.0,339.0){\rule[-0.200pt]{2.409pt}{0.400pt}}
\put(170.0,339.0){\rule[-0.200pt]{2.409pt}{0.400pt}}
\put(1429.0,339.0){\rule[-0.200pt]{2.409pt}{0.400pt}}
\put(170.0,339.0){\rule[-0.200pt]{2.409pt}{0.400pt}}
\put(1429.0,339.0){\rule[-0.200pt]{2.409pt}{0.400pt}}
\put(170.0,339.0){\rule[-0.200pt]{2.409pt}{0.400pt}}
\put(1429.0,339.0){\rule[-0.200pt]{2.409pt}{0.400pt}}
\put(170.0,339.0){\rule[-0.200pt]{2.409pt}{0.400pt}}
\put(1429.0,339.0){\rule[-0.200pt]{2.409pt}{0.400pt}}
\put(170.0,339.0){\rule[-0.200pt]{2.409pt}{0.400pt}}
\put(1429.0,339.0){\rule[-0.200pt]{2.409pt}{0.400pt}}
\put(170.0,339.0){\rule[-0.200pt]{2.409pt}{0.400pt}}
\put(1429.0,339.0){\rule[-0.200pt]{2.409pt}{0.400pt}}
\put(170.0,339.0){\rule[-0.200pt]{2.409pt}{0.400pt}}
\put(1429.0,339.0){\rule[-0.200pt]{2.409pt}{0.400pt}}
\put(170.0,339.0){\rule[-0.200pt]{2.409pt}{0.400pt}}
\put(1429.0,339.0){\rule[-0.200pt]{2.409pt}{0.400pt}}
\put(170.0,339.0){\rule[-0.200pt]{2.409pt}{0.400pt}}
\put(1429.0,339.0){\rule[-0.200pt]{2.409pt}{0.400pt}}
\put(170.0,339.0){\rule[-0.200pt]{2.409pt}{0.400pt}}
\put(1429.0,339.0){\rule[-0.200pt]{2.409pt}{0.400pt}}
\put(170.0,339.0){\rule[-0.200pt]{2.409pt}{0.400pt}}
\put(1429.0,339.0){\rule[-0.200pt]{2.409pt}{0.400pt}}
\put(170.0,339.0){\rule[-0.200pt]{2.409pt}{0.400pt}}
\put(1429.0,339.0){\rule[-0.200pt]{2.409pt}{0.400pt}}
\put(170.0,339.0){\rule[-0.200pt]{2.409pt}{0.400pt}}
\put(1429.0,339.0){\rule[-0.200pt]{2.409pt}{0.400pt}}
\put(170.0,339.0){\rule[-0.200pt]{2.409pt}{0.400pt}}
\put(1429.0,339.0){\rule[-0.200pt]{2.409pt}{0.400pt}}
\put(170.0,339.0){\rule[-0.200pt]{2.409pt}{0.400pt}}
\put(1429.0,339.0){\rule[-0.200pt]{2.409pt}{0.400pt}}
\put(170.0,339.0){\rule[-0.200pt]{2.409pt}{0.400pt}}
\put(1429.0,339.0){\rule[-0.200pt]{2.409pt}{0.400pt}}
\put(170.0,339.0){\rule[-0.200pt]{2.409pt}{0.400pt}}
\put(1429.0,339.0){\rule[-0.200pt]{2.409pt}{0.400pt}}
\put(170.0,339.0){\rule[-0.200pt]{2.409pt}{0.400pt}}
\put(1429.0,339.0){\rule[-0.200pt]{2.409pt}{0.400pt}}
\put(170.0,339.0){\rule[-0.200pt]{2.409pt}{0.400pt}}
\put(1429.0,339.0){\rule[-0.200pt]{2.409pt}{0.400pt}}
\put(170.0,339.0){\rule[-0.200pt]{2.409pt}{0.400pt}}
\put(1429.0,339.0){\rule[-0.200pt]{2.409pt}{0.400pt}}
\put(170.0,339.0){\rule[-0.200pt]{2.409pt}{0.400pt}}
\put(1429.0,339.0){\rule[-0.200pt]{2.409pt}{0.400pt}}
\put(170.0,340.0){\rule[-0.200pt]{2.409pt}{0.400pt}}
\put(1429.0,340.0){\rule[-0.200pt]{2.409pt}{0.400pt}}
\put(170.0,340.0){\rule[-0.200pt]{2.409pt}{0.400pt}}
\put(1429.0,340.0){\rule[-0.200pt]{2.409pt}{0.400pt}}
\put(170.0,340.0){\rule[-0.200pt]{2.409pt}{0.400pt}}
\put(1429.0,340.0){\rule[-0.200pt]{2.409pt}{0.400pt}}
\put(170.0,340.0){\rule[-0.200pt]{2.409pt}{0.400pt}}
\put(1429.0,340.0){\rule[-0.200pt]{2.409pt}{0.400pt}}
\put(170.0,340.0){\rule[-0.200pt]{2.409pt}{0.400pt}}
\put(1429.0,340.0){\rule[-0.200pt]{2.409pt}{0.400pt}}
\put(170.0,340.0){\rule[-0.200pt]{2.409pt}{0.400pt}}
\put(1429.0,340.0){\rule[-0.200pt]{2.409pt}{0.400pt}}
\put(170.0,340.0){\rule[-0.200pt]{2.409pt}{0.400pt}}
\put(1429.0,340.0){\rule[-0.200pt]{2.409pt}{0.400pt}}
\put(170.0,340.0){\rule[-0.200pt]{2.409pt}{0.400pt}}
\put(1429.0,340.0){\rule[-0.200pt]{2.409pt}{0.400pt}}
\put(170.0,340.0){\rule[-0.200pt]{2.409pt}{0.400pt}}
\put(1429.0,340.0){\rule[-0.200pt]{2.409pt}{0.400pt}}
\put(170.0,340.0){\rule[-0.200pt]{2.409pt}{0.400pt}}
\put(1429.0,340.0){\rule[-0.200pt]{2.409pt}{0.400pt}}
\put(170.0,340.0){\rule[-0.200pt]{2.409pt}{0.400pt}}
\put(1429.0,340.0){\rule[-0.200pt]{2.409pt}{0.400pt}}
\put(170.0,340.0){\rule[-0.200pt]{2.409pt}{0.400pt}}
\put(1429.0,340.0){\rule[-0.200pt]{2.409pt}{0.400pt}}
\put(170.0,340.0){\rule[-0.200pt]{2.409pt}{0.400pt}}
\put(1429.0,340.0){\rule[-0.200pt]{2.409pt}{0.400pt}}
\put(170.0,340.0){\rule[-0.200pt]{2.409pt}{0.400pt}}
\put(1429.0,340.0){\rule[-0.200pt]{2.409pt}{0.400pt}}
\put(170.0,340.0){\rule[-0.200pt]{2.409pt}{0.400pt}}
\put(1429.0,340.0){\rule[-0.200pt]{2.409pt}{0.400pt}}
\put(170.0,340.0){\rule[-0.200pt]{2.409pt}{0.400pt}}
\put(1429.0,340.0){\rule[-0.200pt]{2.409pt}{0.400pt}}
\put(170.0,340.0){\rule[-0.200pt]{2.409pt}{0.400pt}}
\put(1429.0,340.0){\rule[-0.200pt]{2.409pt}{0.400pt}}
\put(170.0,340.0){\rule[-0.200pt]{2.409pt}{0.400pt}}
\put(1429.0,340.0){\rule[-0.200pt]{2.409pt}{0.400pt}}
\put(170.0,340.0){\rule[-0.200pt]{2.409pt}{0.400pt}}
\put(1429.0,340.0){\rule[-0.200pt]{2.409pt}{0.400pt}}
\put(170.0,340.0){\rule[-0.200pt]{2.409pt}{0.400pt}}
\put(1429.0,340.0){\rule[-0.200pt]{2.409pt}{0.400pt}}
\put(170.0,340.0){\rule[-0.200pt]{2.409pt}{0.400pt}}
\put(1429.0,340.0){\rule[-0.200pt]{2.409pt}{0.400pt}}
\put(170.0,340.0){\rule[-0.200pt]{2.409pt}{0.400pt}}
\put(1429.0,340.0){\rule[-0.200pt]{2.409pt}{0.400pt}}
\put(170.0,340.0){\rule[-0.200pt]{2.409pt}{0.400pt}}
\put(1429.0,340.0){\rule[-0.200pt]{2.409pt}{0.400pt}}
\put(170.0,340.0){\rule[-0.200pt]{2.409pt}{0.400pt}}
\put(1429.0,340.0){\rule[-0.200pt]{2.409pt}{0.400pt}}
\put(170.0,340.0){\rule[-0.200pt]{2.409pt}{0.400pt}}
\put(1429.0,340.0){\rule[-0.200pt]{2.409pt}{0.400pt}}
\put(170.0,340.0){\rule[-0.200pt]{2.409pt}{0.400pt}}
\put(1429.0,340.0){\rule[-0.200pt]{2.409pt}{0.400pt}}
\put(170.0,341.0){\rule[-0.200pt]{2.409pt}{0.400pt}}
\put(1429.0,341.0){\rule[-0.200pt]{2.409pt}{0.400pt}}
\put(170.0,341.0){\rule[-0.200pt]{2.409pt}{0.400pt}}
\put(1429.0,341.0){\rule[-0.200pt]{2.409pt}{0.400pt}}
\put(170.0,341.0){\rule[-0.200pt]{2.409pt}{0.400pt}}
\put(1429.0,341.0){\rule[-0.200pt]{2.409pt}{0.400pt}}
\put(170.0,341.0){\rule[-0.200pt]{2.409pt}{0.400pt}}
\put(1429.0,341.0){\rule[-0.200pt]{2.409pt}{0.400pt}}
\put(170.0,341.0){\rule[-0.200pt]{2.409pt}{0.400pt}}
\put(1429.0,341.0){\rule[-0.200pt]{2.409pt}{0.400pt}}
\put(170.0,341.0){\rule[-0.200pt]{2.409pt}{0.400pt}}
\put(1429.0,341.0){\rule[-0.200pt]{2.409pt}{0.400pt}}
\put(170.0,341.0){\rule[-0.200pt]{2.409pt}{0.400pt}}
\put(1429.0,341.0){\rule[-0.200pt]{2.409pt}{0.400pt}}
\put(170.0,341.0){\rule[-0.200pt]{2.409pt}{0.400pt}}
\put(1429.0,341.0){\rule[-0.200pt]{2.409pt}{0.400pt}}
\put(170.0,341.0){\rule[-0.200pt]{2.409pt}{0.400pt}}
\put(1429.0,341.0){\rule[-0.200pt]{2.409pt}{0.400pt}}
\put(170.0,341.0){\rule[-0.200pt]{2.409pt}{0.400pt}}
\put(1429.0,341.0){\rule[-0.200pt]{2.409pt}{0.400pt}}
\put(170.0,341.0){\rule[-0.200pt]{2.409pt}{0.400pt}}
\put(1429.0,341.0){\rule[-0.200pt]{2.409pt}{0.400pt}}
\put(170.0,341.0){\rule[-0.200pt]{2.409pt}{0.400pt}}
\put(1429.0,341.0){\rule[-0.200pt]{2.409pt}{0.400pt}}
\put(170.0,341.0){\rule[-0.200pt]{2.409pt}{0.400pt}}
\put(1429.0,341.0){\rule[-0.200pt]{2.409pt}{0.400pt}}
\put(170.0,341.0){\rule[-0.200pt]{4.818pt}{0.400pt}}
\put(150,341){\makebox(0,0)[r]{ 1000}}
\put(1419.0,341.0){\rule[-0.200pt]{4.818pt}{0.400pt}}
\put(170.0,367.0){\rule[-0.200pt]{2.409pt}{0.400pt}}
\put(1429.0,367.0){\rule[-0.200pt]{2.409pt}{0.400pt}}
\put(170.0,382.0){\rule[-0.200pt]{2.409pt}{0.400pt}}
\put(1429.0,382.0){\rule[-0.200pt]{2.409pt}{0.400pt}}
\put(170.0,393.0){\rule[-0.200pt]{2.409pt}{0.400pt}}
\put(1429.0,393.0){\rule[-0.200pt]{2.409pt}{0.400pt}}
\put(170.0,401.0){\rule[-0.200pt]{2.409pt}{0.400pt}}
\put(1429.0,401.0){\rule[-0.200pt]{2.409pt}{0.400pt}}
\put(170.0,408.0){\rule[-0.200pt]{2.409pt}{0.400pt}}
\put(1429.0,408.0){\rule[-0.200pt]{2.409pt}{0.400pt}}
\put(170.0,414.0){\rule[-0.200pt]{2.409pt}{0.400pt}}
\put(1429.0,414.0){\rule[-0.200pt]{2.409pt}{0.400pt}}
\put(170.0,419.0){\rule[-0.200pt]{2.409pt}{0.400pt}}
\put(1429.0,419.0){\rule[-0.200pt]{2.409pt}{0.400pt}}
\put(170.0,423.0){\rule[-0.200pt]{2.409pt}{0.400pt}}
\put(1429.0,423.0){\rule[-0.200pt]{2.409pt}{0.400pt}}
\put(170.0,427.0){\rule[-0.200pt]{2.409pt}{0.400pt}}
\put(1429.0,427.0){\rule[-0.200pt]{2.409pt}{0.400pt}}
\put(170.0,431.0){\rule[-0.200pt]{2.409pt}{0.400pt}}
\put(1429.0,431.0){\rule[-0.200pt]{2.409pt}{0.400pt}}
\put(170.0,434.0){\rule[-0.200pt]{2.409pt}{0.400pt}}
\put(1429.0,434.0){\rule[-0.200pt]{2.409pt}{0.400pt}}
\put(170.0,437.0){\rule[-0.200pt]{2.409pt}{0.400pt}}
\put(1429.0,437.0){\rule[-0.200pt]{2.409pt}{0.400pt}}
\put(170.0,440.0){\rule[-0.200pt]{2.409pt}{0.400pt}}
\put(1429.0,440.0){\rule[-0.200pt]{2.409pt}{0.400pt}}
\put(170.0,443.0){\rule[-0.200pt]{2.409pt}{0.400pt}}
\put(1429.0,443.0){\rule[-0.200pt]{2.409pt}{0.400pt}}
\put(170.0,445.0){\rule[-0.200pt]{2.409pt}{0.400pt}}
\put(1429.0,445.0){\rule[-0.200pt]{2.409pt}{0.400pt}}
\put(170.0,447.0){\rule[-0.200pt]{2.409pt}{0.400pt}}
\put(1429.0,447.0){\rule[-0.200pt]{2.409pt}{0.400pt}}
\put(170.0,449.0){\rule[-0.200pt]{2.409pt}{0.400pt}}
\put(1429.0,449.0){\rule[-0.200pt]{2.409pt}{0.400pt}}
\put(170.0,451.0){\rule[-0.200pt]{2.409pt}{0.400pt}}
\put(1429.0,451.0){\rule[-0.200pt]{2.409pt}{0.400pt}}
\put(170.0,453.0){\rule[-0.200pt]{2.409pt}{0.400pt}}
\put(1429.0,453.0){\rule[-0.200pt]{2.409pt}{0.400pt}}
\put(170.0,455.0){\rule[-0.200pt]{2.409pt}{0.400pt}}
\put(1429.0,455.0){\rule[-0.200pt]{2.409pt}{0.400pt}}
\put(170.0,457.0){\rule[-0.200pt]{2.409pt}{0.400pt}}
\put(1429.0,457.0){\rule[-0.200pt]{2.409pt}{0.400pt}}
\put(170.0,459.0){\rule[-0.200pt]{2.409pt}{0.400pt}}
\put(1429.0,459.0){\rule[-0.200pt]{2.409pt}{0.400pt}}
\put(170.0,460.0){\rule[-0.200pt]{2.409pt}{0.400pt}}
\put(1429.0,460.0){\rule[-0.200pt]{2.409pt}{0.400pt}}
\put(170.0,462.0){\rule[-0.200pt]{2.409pt}{0.400pt}}
\put(1429.0,462.0){\rule[-0.200pt]{2.409pt}{0.400pt}}
\put(170.0,463.0){\rule[-0.200pt]{2.409pt}{0.400pt}}
\put(1429.0,463.0){\rule[-0.200pt]{2.409pt}{0.400pt}}
\put(170.0,465.0){\rule[-0.200pt]{2.409pt}{0.400pt}}
\put(1429.0,465.0){\rule[-0.200pt]{2.409pt}{0.400pt}}
\put(170.0,466.0){\rule[-0.200pt]{2.409pt}{0.400pt}}
\put(1429.0,466.0){\rule[-0.200pt]{2.409pt}{0.400pt}}
\put(170.0,467.0){\rule[-0.200pt]{2.409pt}{0.400pt}}
\put(1429.0,467.0){\rule[-0.200pt]{2.409pt}{0.400pt}}
\put(170.0,469.0){\rule[-0.200pt]{2.409pt}{0.400pt}}
\put(1429.0,469.0){\rule[-0.200pt]{2.409pt}{0.400pt}}
\put(170.0,470.0){\rule[-0.200pt]{2.409pt}{0.400pt}}
\put(1429.0,470.0){\rule[-0.200pt]{2.409pt}{0.400pt}}
\put(170.0,471.0){\rule[-0.200pt]{2.409pt}{0.400pt}}
\put(1429.0,471.0){\rule[-0.200pt]{2.409pt}{0.400pt}}
\put(170.0,472.0){\rule[-0.200pt]{2.409pt}{0.400pt}}
\put(1429.0,472.0){\rule[-0.200pt]{2.409pt}{0.400pt}}
\put(170.0,473.0){\rule[-0.200pt]{2.409pt}{0.400pt}}
\put(1429.0,473.0){\rule[-0.200pt]{2.409pt}{0.400pt}}
\put(170.0,474.0){\rule[-0.200pt]{2.409pt}{0.400pt}}
\put(1429.0,474.0){\rule[-0.200pt]{2.409pt}{0.400pt}}
\put(170.0,475.0){\rule[-0.200pt]{2.409pt}{0.400pt}}
\put(1429.0,475.0){\rule[-0.200pt]{2.409pt}{0.400pt}}
\put(170.0,476.0){\rule[-0.200pt]{2.409pt}{0.400pt}}
\put(1429.0,476.0){\rule[-0.200pt]{2.409pt}{0.400pt}}
\put(170.0,477.0){\rule[-0.200pt]{2.409pt}{0.400pt}}
\put(1429.0,477.0){\rule[-0.200pt]{2.409pt}{0.400pt}}
\put(170.0,478.0){\rule[-0.200pt]{2.409pt}{0.400pt}}
\put(1429.0,478.0){\rule[-0.200pt]{2.409pt}{0.400pt}}
\put(170.0,479.0){\rule[-0.200pt]{2.409pt}{0.400pt}}
\put(1429.0,479.0){\rule[-0.200pt]{2.409pt}{0.400pt}}
\put(170.0,480.0){\rule[-0.200pt]{2.409pt}{0.400pt}}
\put(1429.0,480.0){\rule[-0.200pt]{2.409pt}{0.400pt}}
\put(170.0,481.0){\rule[-0.200pt]{2.409pt}{0.400pt}}
\put(1429.0,481.0){\rule[-0.200pt]{2.409pt}{0.400pt}}
\put(170.0,482.0){\rule[-0.200pt]{2.409pt}{0.400pt}}
\put(1429.0,482.0){\rule[-0.200pt]{2.409pt}{0.400pt}}
\put(170.0,483.0){\rule[-0.200pt]{2.409pt}{0.400pt}}
\put(1429.0,483.0){\rule[-0.200pt]{2.409pt}{0.400pt}}
\put(170.0,484.0){\rule[-0.200pt]{2.409pt}{0.400pt}}
\put(1429.0,484.0){\rule[-0.200pt]{2.409pt}{0.400pt}}
\put(170.0,485.0){\rule[-0.200pt]{2.409pt}{0.400pt}}
\put(1429.0,485.0){\rule[-0.200pt]{2.409pt}{0.400pt}}
\put(170.0,485.0){\rule[-0.200pt]{2.409pt}{0.400pt}}
\put(1429.0,485.0){\rule[-0.200pt]{2.409pt}{0.400pt}}
\put(170.0,486.0){\rule[-0.200pt]{2.409pt}{0.400pt}}
\put(1429.0,486.0){\rule[-0.200pt]{2.409pt}{0.400pt}}
\put(170.0,487.0){\rule[-0.200pt]{2.409pt}{0.400pt}}
\put(1429.0,487.0){\rule[-0.200pt]{2.409pt}{0.400pt}}
\put(170.0,488.0){\rule[-0.200pt]{2.409pt}{0.400pt}}
\put(1429.0,488.0){\rule[-0.200pt]{2.409pt}{0.400pt}}
\put(170.0,488.0){\rule[-0.200pt]{2.409pt}{0.400pt}}
\put(1429.0,488.0){\rule[-0.200pt]{2.409pt}{0.400pt}}
\put(170.0,489.0){\rule[-0.200pt]{2.409pt}{0.400pt}}
\put(1429.0,489.0){\rule[-0.200pt]{2.409pt}{0.400pt}}
\put(170.0,490.0){\rule[-0.200pt]{2.409pt}{0.400pt}}
\put(1429.0,490.0){\rule[-0.200pt]{2.409pt}{0.400pt}}
\put(170.0,491.0){\rule[-0.200pt]{2.409pt}{0.400pt}}
\put(1429.0,491.0){\rule[-0.200pt]{2.409pt}{0.400pt}}
\put(170.0,491.0){\rule[-0.200pt]{2.409pt}{0.400pt}}
\put(1429.0,491.0){\rule[-0.200pt]{2.409pt}{0.400pt}}
\put(170.0,492.0){\rule[-0.200pt]{2.409pt}{0.400pt}}
\put(1429.0,492.0){\rule[-0.200pt]{2.409pt}{0.400pt}}
\put(170.0,493.0){\rule[-0.200pt]{2.409pt}{0.400pt}}
\put(1429.0,493.0){\rule[-0.200pt]{2.409pt}{0.400pt}}
\put(170.0,493.0){\rule[-0.200pt]{2.409pt}{0.400pt}}
\put(1429.0,493.0){\rule[-0.200pt]{2.409pt}{0.400pt}}
\put(170.0,494.0){\rule[-0.200pt]{2.409pt}{0.400pt}}
\put(1429.0,494.0){\rule[-0.200pt]{2.409pt}{0.400pt}}
\put(170.0,495.0){\rule[-0.200pt]{2.409pt}{0.400pt}}
\put(1429.0,495.0){\rule[-0.200pt]{2.409pt}{0.400pt}}
\put(170.0,495.0){\rule[-0.200pt]{2.409pt}{0.400pt}}
\put(1429.0,495.0){\rule[-0.200pt]{2.409pt}{0.400pt}}
\put(170.0,496.0){\rule[-0.200pt]{2.409pt}{0.400pt}}
\put(1429.0,496.0){\rule[-0.200pt]{2.409pt}{0.400pt}}
\put(170.0,496.0){\rule[-0.200pt]{2.409pt}{0.400pt}}
\put(1429.0,496.0){\rule[-0.200pt]{2.409pt}{0.400pt}}
\put(170.0,497.0){\rule[-0.200pt]{2.409pt}{0.400pt}}
\put(1429.0,497.0){\rule[-0.200pt]{2.409pt}{0.400pt}}
\put(170.0,498.0){\rule[-0.200pt]{2.409pt}{0.400pt}}
\put(1429.0,498.0){\rule[-0.200pt]{2.409pt}{0.400pt}}
\put(170.0,498.0){\rule[-0.200pt]{2.409pt}{0.400pt}}
\put(1429.0,498.0){\rule[-0.200pt]{2.409pt}{0.400pt}}
\put(170.0,499.0){\rule[-0.200pt]{2.409pt}{0.400pt}}
\put(1429.0,499.0){\rule[-0.200pt]{2.409pt}{0.400pt}}
\put(170.0,499.0){\rule[-0.200pt]{2.409pt}{0.400pt}}
\put(1429.0,499.0){\rule[-0.200pt]{2.409pt}{0.400pt}}
\put(170.0,500.0){\rule[-0.200pt]{2.409pt}{0.400pt}}
\put(1429.0,500.0){\rule[-0.200pt]{2.409pt}{0.400pt}}
\put(170.0,500.0){\rule[-0.200pt]{2.409pt}{0.400pt}}
\put(1429.0,500.0){\rule[-0.200pt]{2.409pt}{0.400pt}}
\put(170.0,501.0){\rule[-0.200pt]{2.409pt}{0.400pt}}
\put(1429.0,501.0){\rule[-0.200pt]{2.409pt}{0.400pt}}
\put(170.0,501.0){\rule[-0.200pt]{2.409pt}{0.400pt}}
\put(1429.0,501.0){\rule[-0.200pt]{2.409pt}{0.400pt}}
\put(170.0,502.0){\rule[-0.200pt]{2.409pt}{0.400pt}}
\put(1429.0,502.0){\rule[-0.200pt]{2.409pt}{0.400pt}}
\put(170.0,502.0){\rule[-0.200pt]{2.409pt}{0.400pt}}
\put(1429.0,502.0){\rule[-0.200pt]{2.409pt}{0.400pt}}
\put(170.0,503.0){\rule[-0.200pt]{2.409pt}{0.400pt}}
\put(1429.0,503.0){\rule[-0.200pt]{2.409pt}{0.400pt}}
\put(170.0,503.0){\rule[-0.200pt]{2.409pt}{0.400pt}}
\put(1429.0,503.0){\rule[-0.200pt]{2.409pt}{0.400pt}}
\put(170.0,504.0){\rule[-0.200pt]{2.409pt}{0.400pt}}
\put(1429.0,504.0){\rule[-0.200pt]{2.409pt}{0.400pt}}
\put(170.0,504.0){\rule[-0.200pt]{2.409pt}{0.400pt}}
\put(1429.0,504.0){\rule[-0.200pt]{2.409pt}{0.400pt}}
\put(170.0,505.0){\rule[-0.200pt]{2.409pt}{0.400pt}}
\put(1429.0,505.0){\rule[-0.200pt]{2.409pt}{0.400pt}}
\put(170.0,505.0){\rule[-0.200pt]{2.409pt}{0.400pt}}
\put(1429.0,505.0){\rule[-0.200pt]{2.409pt}{0.400pt}}
\put(170.0,506.0){\rule[-0.200pt]{2.409pt}{0.400pt}}
\put(1429.0,506.0){\rule[-0.200pt]{2.409pt}{0.400pt}}
\put(170.0,506.0){\rule[-0.200pt]{2.409pt}{0.400pt}}
\put(1429.0,506.0){\rule[-0.200pt]{2.409pt}{0.400pt}}
\put(170.0,507.0){\rule[-0.200pt]{2.409pt}{0.400pt}}
\put(1429.0,507.0){\rule[-0.200pt]{2.409pt}{0.400pt}}
\put(170.0,507.0){\rule[-0.200pt]{2.409pt}{0.400pt}}
\put(1429.0,507.0){\rule[-0.200pt]{2.409pt}{0.400pt}}
\put(170.0,508.0){\rule[-0.200pt]{2.409pt}{0.400pt}}
\put(1429.0,508.0){\rule[-0.200pt]{2.409pt}{0.400pt}}
\put(170.0,508.0){\rule[-0.200pt]{2.409pt}{0.400pt}}
\put(1429.0,508.0){\rule[-0.200pt]{2.409pt}{0.400pt}}
\put(170.0,508.0){\rule[-0.200pt]{2.409pt}{0.400pt}}
\put(1429.0,508.0){\rule[-0.200pt]{2.409pt}{0.400pt}}
\put(170.0,509.0){\rule[-0.200pt]{2.409pt}{0.400pt}}
\put(1429.0,509.0){\rule[-0.200pt]{2.409pt}{0.400pt}}
\put(170.0,509.0){\rule[-0.200pt]{2.409pt}{0.400pt}}
\put(1429.0,509.0){\rule[-0.200pt]{2.409pt}{0.400pt}}
\put(170.0,510.0){\rule[-0.200pt]{2.409pt}{0.400pt}}
\put(1429.0,510.0){\rule[-0.200pt]{2.409pt}{0.400pt}}
\put(170.0,510.0){\rule[-0.200pt]{2.409pt}{0.400pt}}
\put(1429.0,510.0){\rule[-0.200pt]{2.409pt}{0.400pt}}
\put(170.0,511.0){\rule[-0.200pt]{2.409pt}{0.400pt}}
\put(1429.0,511.0){\rule[-0.200pt]{2.409pt}{0.400pt}}
\put(170.0,511.0){\rule[-0.200pt]{2.409pt}{0.400pt}}
\put(1429.0,511.0){\rule[-0.200pt]{2.409pt}{0.400pt}}
\put(170.0,511.0){\rule[-0.200pt]{2.409pt}{0.400pt}}
\put(1429.0,511.0){\rule[-0.200pt]{2.409pt}{0.400pt}}
\put(170.0,512.0){\rule[-0.200pt]{2.409pt}{0.400pt}}
\put(1429.0,512.0){\rule[-0.200pt]{2.409pt}{0.400pt}}
\put(170.0,512.0){\rule[-0.200pt]{2.409pt}{0.400pt}}
\put(1429.0,512.0){\rule[-0.200pt]{2.409pt}{0.400pt}}
\put(170.0,513.0){\rule[-0.200pt]{2.409pt}{0.400pt}}
\put(1429.0,513.0){\rule[-0.200pt]{2.409pt}{0.400pt}}
\put(170.0,513.0){\rule[-0.200pt]{2.409pt}{0.400pt}}
\put(1429.0,513.0){\rule[-0.200pt]{2.409pt}{0.400pt}}
\put(170.0,513.0){\rule[-0.200pt]{2.409pt}{0.400pt}}
\put(1429.0,513.0){\rule[-0.200pt]{2.409pt}{0.400pt}}
\put(170.0,514.0){\rule[-0.200pt]{2.409pt}{0.400pt}}
\put(1429.0,514.0){\rule[-0.200pt]{2.409pt}{0.400pt}}
\put(170.0,514.0){\rule[-0.200pt]{2.409pt}{0.400pt}}
\put(1429.0,514.0){\rule[-0.200pt]{2.409pt}{0.400pt}}
\put(170.0,514.0){\rule[-0.200pt]{2.409pt}{0.400pt}}
\put(1429.0,514.0){\rule[-0.200pt]{2.409pt}{0.400pt}}
\put(170.0,515.0){\rule[-0.200pt]{2.409pt}{0.400pt}}
\put(1429.0,515.0){\rule[-0.200pt]{2.409pt}{0.400pt}}
\put(170.0,515.0){\rule[-0.200pt]{2.409pt}{0.400pt}}
\put(1429.0,515.0){\rule[-0.200pt]{2.409pt}{0.400pt}}
\put(170.0,515.0){\rule[-0.200pt]{2.409pt}{0.400pt}}
\put(1429.0,515.0){\rule[-0.200pt]{2.409pt}{0.400pt}}
\put(170.0,516.0){\rule[-0.200pt]{2.409pt}{0.400pt}}
\put(1429.0,516.0){\rule[-0.200pt]{2.409pt}{0.400pt}}
\put(170.0,516.0){\rule[-0.200pt]{2.409pt}{0.400pt}}
\put(1429.0,516.0){\rule[-0.200pt]{2.409pt}{0.400pt}}
\put(170.0,517.0){\rule[-0.200pt]{2.409pt}{0.400pt}}
\put(1429.0,517.0){\rule[-0.200pt]{2.409pt}{0.400pt}}
\put(170.0,517.0){\rule[-0.200pt]{2.409pt}{0.400pt}}
\put(1429.0,517.0){\rule[-0.200pt]{2.409pt}{0.400pt}}
\put(170.0,517.0){\rule[-0.200pt]{2.409pt}{0.400pt}}
\put(1429.0,517.0){\rule[-0.200pt]{2.409pt}{0.400pt}}
\put(170.0,518.0){\rule[-0.200pt]{2.409pt}{0.400pt}}
\put(1429.0,518.0){\rule[-0.200pt]{2.409pt}{0.400pt}}
\put(170.0,518.0){\rule[-0.200pt]{2.409pt}{0.400pt}}
\put(1429.0,518.0){\rule[-0.200pt]{2.409pt}{0.400pt}}
\put(170.0,518.0){\rule[-0.200pt]{2.409pt}{0.400pt}}
\put(1429.0,518.0){\rule[-0.200pt]{2.409pt}{0.400pt}}
\put(170.0,519.0){\rule[-0.200pt]{2.409pt}{0.400pt}}
\put(1429.0,519.0){\rule[-0.200pt]{2.409pt}{0.400pt}}
\put(170.0,519.0){\rule[-0.200pt]{2.409pt}{0.400pt}}
\put(1429.0,519.0){\rule[-0.200pt]{2.409pt}{0.400pt}}
\put(170.0,519.0){\rule[-0.200pt]{2.409pt}{0.400pt}}
\put(1429.0,519.0){\rule[-0.200pt]{2.409pt}{0.400pt}}
\put(170.0,520.0){\rule[-0.200pt]{2.409pt}{0.400pt}}
\put(1429.0,520.0){\rule[-0.200pt]{2.409pt}{0.400pt}}
\put(170.0,520.0){\rule[-0.200pt]{2.409pt}{0.400pt}}
\put(1429.0,520.0){\rule[-0.200pt]{2.409pt}{0.400pt}}
\put(170.0,520.0){\rule[-0.200pt]{2.409pt}{0.400pt}}
\put(1429.0,520.0){\rule[-0.200pt]{2.409pt}{0.400pt}}
\put(170.0,521.0){\rule[-0.200pt]{2.409pt}{0.400pt}}
\put(1429.0,521.0){\rule[-0.200pt]{2.409pt}{0.400pt}}
\put(170.0,521.0){\rule[-0.200pt]{2.409pt}{0.400pt}}
\put(1429.0,521.0){\rule[-0.200pt]{2.409pt}{0.400pt}}
\put(170.0,521.0){\rule[-0.200pt]{2.409pt}{0.400pt}}
\put(1429.0,521.0){\rule[-0.200pt]{2.409pt}{0.400pt}}
\put(170.0,521.0){\rule[-0.200pt]{2.409pt}{0.400pt}}
\put(1429.0,521.0){\rule[-0.200pt]{2.409pt}{0.400pt}}
\put(170.0,522.0){\rule[-0.200pt]{2.409pt}{0.400pt}}
\put(1429.0,522.0){\rule[-0.200pt]{2.409pt}{0.400pt}}
\put(170.0,522.0){\rule[-0.200pt]{2.409pt}{0.400pt}}
\put(1429.0,522.0){\rule[-0.200pt]{2.409pt}{0.400pt}}
\put(170.0,522.0){\rule[-0.200pt]{2.409pt}{0.400pt}}
\put(1429.0,522.0){\rule[-0.200pt]{2.409pt}{0.400pt}}
\put(170.0,523.0){\rule[-0.200pt]{2.409pt}{0.400pt}}
\put(1429.0,523.0){\rule[-0.200pt]{2.409pt}{0.400pt}}
\put(170.0,523.0){\rule[-0.200pt]{2.409pt}{0.400pt}}
\put(1429.0,523.0){\rule[-0.200pt]{2.409pt}{0.400pt}}
\put(170.0,523.0){\rule[-0.200pt]{2.409pt}{0.400pt}}
\put(1429.0,523.0){\rule[-0.200pt]{2.409pt}{0.400pt}}
\put(170.0,524.0){\rule[-0.200pt]{2.409pt}{0.400pt}}
\put(1429.0,524.0){\rule[-0.200pt]{2.409pt}{0.400pt}}
\put(170.0,524.0){\rule[-0.200pt]{2.409pt}{0.400pt}}
\put(1429.0,524.0){\rule[-0.200pt]{2.409pt}{0.400pt}}
\put(170.0,524.0){\rule[-0.200pt]{2.409pt}{0.400pt}}
\put(1429.0,524.0){\rule[-0.200pt]{2.409pt}{0.400pt}}
\put(170.0,524.0){\rule[-0.200pt]{2.409pt}{0.400pt}}
\put(1429.0,524.0){\rule[-0.200pt]{2.409pt}{0.400pt}}
\put(170.0,525.0){\rule[-0.200pt]{2.409pt}{0.400pt}}
\put(1429.0,525.0){\rule[-0.200pt]{2.409pt}{0.400pt}}
\put(170.0,525.0){\rule[-0.200pt]{2.409pt}{0.400pt}}
\put(1429.0,525.0){\rule[-0.200pt]{2.409pt}{0.400pt}}
\put(170.0,525.0){\rule[-0.200pt]{2.409pt}{0.400pt}}
\put(1429.0,525.0){\rule[-0.200pt]{2.409pt}{0.400pt}}
\put(170.0,525.0){\rule[-0.200pt]{2.409pt}{0.400pt}}
\put(1429.0,525.0){\rule[-0.200pt]{2.409pt}{0.400pt}}
\put(170.0,526.0){\rule[-0.200pt]{2.409pt}{0.400pt}}
\put(1429.0,526.0){\rule[-0.200pt]{2.409pt}{0.400pt}}
\put(170.0,526.0){\rule[-0.200pt]{2.409pt}{0.400pt}}
\put(1429.0,526.0){\rule[-0.200pt]{2.409pt}{0.400pt}}
\put(170.0,526.0){\rule[-0.200pt]{2.409pt}{0.400pt}}
\put(1429.0,526.0){\rule[-0.200pt]{2.409pt}{0.400pt}}
\put(170.0,527.0){\rule[-0.200pt]{2.409pt}{0.400pt}}
\put(1429.0,527.0){\rule[-0.200pt]{2.409pt}{0.400pt}}
\put(170.0,527.0){\rule[-0.200pt]{2.409pt}{0.400pt}}
\put(1429.0,527.0){\rule[-0.200pt]{2.409pt}{0.400pt}}
\put(170.0,527.0){\rule[-0.200pt]{2.409pt}{0.400pt}}
\put(1429.0,527.0){\rule[-0.200pt]{2.409pt}{0.400pt}}
\put(170.0,527.0){\rule[-0.200pt]{2.409pt}{0.400pt}}
\put(1429.0,527.0){\rule[-0.200pt]{2.409pt}{0.400pt}}
\put(170.0,528.0){\rule[-0.200pt]{2.409pt}{0.400pt}}
\put(1429.0,528.0){\rule[-0.200pt]{2.409pt}{0.400pt}}
\put(170.0,528.0){\rule[-0.200pt]{2.409pt}{0.400pt}}
\put(1429.0,528.0){\rule[-0.200pt]{2.409pt}{0.400pt}}
\put(170.0,528.0){\rule[-0.200pt]{2.409pt}{0.400pt}}
\put(1429.0,528.0){\rule[-0.200pt]{2.409pt}{0.400pt}}
\put(170.0,528.0){\rule[-0.200pt]{2.409pt}{0.400pt}}
\put(1429.0,528.0){\rule[-0.200pt]{2.409pt}{0.400pt}}
\put(170.0,529.0){\rule[-0.200pt]{2.409pt}{0.400pt}}
\put(1429.0,529.0){\rule[-0.200pt]{2.409pt}{0.400pt}}
\put(170.0,529.0){\rule[-0.200pt]{2.409pt}{0.400pt}}
\put(1429.0,529.0){\rule[-0.200pt]{2.409pt}{0.400pt}}
\put(170.0,529.0){\rule[-0.200pt]{2.409pt}{0.400pt}}
\put(1429.0,529.0){\rule[-0.200pt]{2.409pt}{0.400pt}}
\put(170.0,529.0){\rule[-0.200pt]{2.409pt}{0.400pt}}
\put(1429.0,529.0){\rule[-0.200pt]{2.409pt}{0.400pt}}
\put(170.0,530.0){\rule[-0.200pt]{2.409pt}{0.400pt}}
\put(1429.0,530.0){\rule[-0.200pt]{2.409pt}{0.400pt}}
\put(170.0,530.0){\rule[-0.200pt]{2.409pt}{0.400pt}}
\put(1429.0,530.0){\rule[-0.200pt]{2.409pt}{0.400pt}}
\put(170.0,530.0){\rule[-0.200pt]{2.409pt}{0.400pt}}
\put(1429.0,530.0){\rule[-0.200pt]{2.409pt}{0.400pt}}
\put(170.0,530.0){\rule[-0.200pt]{2.409pt}{0.400pt}}
\put(1429.0,530.0){\rule[-0.200pt]{2.409pt}{0.400pt}}
\put(170.0,531.0){\rule[-0.200pt]{2.409pt}{0.400pt}}
\put(1429.0,531.0){\rule[-0.200pt]{2.409pt}{0.400pt}}
\put(170.0,531.0){\rule[-0.200pt]{2.409pt}{0.400pt}}
\put(1429.0,531.0){\rule[-0.200pt]{2.409pt}{0.400pt}}
\put(170.0,531.0){\rule[-0.200pt]{2.409pt}{0.400pt}}
\put(1429.0,531.0){\rule[-0.200pt]{2.409pt}{0.400pt}}
\put(170.0,531.0){\rule[-0.200pt]{2.409pt}{0.400pt}}
\put(1429.0,531.0){\rule[-0.200pt]{2.409pt}{0.400pt}}
\put(170.0,532.0){\rule[-0.200pt]{2.409pt}{0.400pt}}
\put(1429.0,532.0){\rule[-0.200pt]{2.409pt}{0.400pt}}
\put(170.0,532.0){\rule[-0.200pt]{2.409pt}{0.400pt}}
\put(1429.0,532.0){\rule[-0.200pt]{2.409pt}{0.400pt}}
\put(170.0,532.0){\rule[-0.200pt]{2.409pt}{0.400pt}}
\put(1429.0,532.0){\rule[-0.200pt]{2.409pt}{0.400pt}}
\put(170.0,532.0){\rule[-0.200pt]{2.409pt}{0.400pt}}
\put(1429.0,532.0){\rule[-0.200pt]{2.409pt}{0.400pt}}
\put(170.0,532.0){\rule[-0.200pt]{2.409pt}{0.400pt}}
\put(1429.0,532.0){\rule[-0.200pt]{2.409pt}{0.400pt}}
\put(170.0,533.0){\rule[-0.200pt]{2.409pt}{0.400pt}}
\put(1429.0,533.0){\rule[-0.200pt]{2.409pt}{0.400pt}}
\put(170.0,533.0){\rule[-0.200pt]{2.409pt}{0.400pt}}
\put(1429.0,533.0){\rule[-0.200pt]{2.409pt}{0.400pt}}
\put(170.0,533.0){\rule[-0.200pt]{2.409pt}{0.400pt}}
\put(1429.0,533.0){\rule[-0.200pt]{2.409pt}{0.400pt}}
\put(170.0,533.0){\rule[-0.200pt]{2.409pt}{0.400pt}}
\put(1429.0,533.0){\rule[-0.200pt]{2.409pt}{0.400pt}}
\put(170.0,534.0){\rule[-0.200pt]{2.409pt}{0.400pt}}
\put(1429.0,534.0){\rule[-0.200pt]{2.409pt}{0.400pt}}
\put(170.0,534.0){\rule[-0.200pt]{2.409pt}{0.400pt}}
\put(1429.0,534.0){\rule[-0.200pt]{2.409pt}{0.400pt}}
\put(170.0,534.0){\rule[-0.200pt]{2.409pt}{0.400pt}}
\put(1429.0,534.0){\rule[-0.200pt]{2.409pt}{0.400pt}}
\put(170.0,534.0){\rule[-0.200pt]{2.409pt}{0.400pt}}
\put(1429.0,534.0){\rule[-0.200pt]{2.409pt}{0.400pt}}
\put(170.0,534.0){\rule[-0.200pt]{2.409pt}{0.400pt}}
\put(1429.0,534.0){\rule[-0.200pt]{2.409pt}{0.400pt}}
\put(170.0,535.0){\rule[-0.200pt]{2.409pt}{0.400pt}}
\put(1429.0,535.0){\rule[-0.200pt]{2.409pt}{0.400pt}}
\put(170.0,535.0){\rule[-0.200pt]{2.409pt}{0.400pt}}
\put(1429.0,535.0){\rule[-0.200pt]{2.409pt}{0.400pt}}
\put(170.0,535.0){\rule[-0.200pt]{2.409pt}{0.400pt}}
\put(1429.0,535.0){\rule[-0.200pt]{2.409pt}{0.400pt}}
\put(170.0,535.0){\rule[-0.200pt]{2.409pt}{0.400pt}}
\put(1429.0,535.0){\rule[-0.200pt]{2.409pt}{0.400pt}}
\put(170.0,535.0){\rule[-0.200pt]{2.409pt}{0.400pt}}
\put(1429.0,535.0){\rule[-0.200pt]{2.409pt}{0.400pt}}
\put(170.0,536.0){\rule[-0.200pt]{2.409pt}{0.400pt}}
\put(1429.0,536.0){\rule[-0.200pt]{2.409pt}{0.400pt}}
\put(170.0,536.0){\rule[-0.200pt]{2.409pt}{0.400pt}}
\put(1429.0,536.0){\rule[-0.200pt]{2.409pt}{0.400pt}}
\put(170.0,536.0){\rule[-0.200pt]{2.409pt}{0.400pt}}
\put(1429.0,536.0){\rule[-0.200pt]{2.409pt}{0.400pt}}
\put(170.0,536.0){\rule[-0.200pt]{2.409pt}{0.400pt}}
\put(1429.0,536.0){\rule[-0.200pt]{2.409pt}{0.400pt}}
\put(170.0,537.0){\rule[-0.200pt]{2.409pt}{0.400pt}}
\put(1429.0,537.0){\rule[-0.200pt]{2.409pt}{0.400pt}}
\put(170.0,537.0){\rule[-0.200pt]{2.409pt}{0.400pt}}
\put(1429.0,537.0){\rule[-0.200pt]{2.409pt}{0.400pt}}
\put(170.0,537.0){\rule[-0.200pt]{2.409pt}{0.400pt}}
\put(1429.0,537.0){\rule[-0.200pt]{2.409pt}{0.400pt}}
\put(170.0,537.0){\rule[-0.200pt]{2.409pt}{0.400pt}}
\put(1429.0,537.0){\rule[-0.200pt]{2.409pt}{0.400pt}}
\put(170.0,537.0){\rule[-0.200pt]{2.409pt}{0.400pt}}
\put(1429.0,537.0){\rule[-0.200pt]{2.409pt}{0.400pt}}
\put(170.0,538.0){\rule[-0.200pt]{2.409pt}{0.400pt}}
\put(1429.0,538.0){\rule[-0.200pt]{2.409pt}{0.400pt}}
\put(170.0,538.0){\rule[-0.200pt]{2.409pt}{0.400pt}}
\put(1429.0,538.0){\rule[-0.200pt]{2.409pt}{0.400pt}}
\put(170.0,538.0){\rule[-0.200pt]{2.409pt}{0.400pt}}
\put(1429.0,538.0){\rule[-0.200pt]{2.409pt}{0.400pt}}
\put(170.0,538.0){\rule[-0.200pt]{2.409pt}{0.400pt}}
\put(1429.0,538.0){\rule[-0.200pt]{2.409pt}{0.400pt}}
\put(170.0,538.0){\rule[-0.200pt]{2.409pt}{0.400pt}}
\put(1429.0,538.0){\rule[-0.200pt]{2.409pt}{0.400pt}}
\put(170.0,539.0){\rule[-0.200pt]{2.409pt}{0.400pt}}
\put(1429.0,539.0){\rule[-0.200pt]{2.409pt}{0.400pt}}
\put(170.0,539.0){\rule[-0.200pt]{2.409pt}{0.400pt}}
\put(1429.0,539.0){\rule[-0.200pt]{2.409pt}{0.400pt}}
\put(170.0,539.0){\rule[-0.200pt]{2.409pt}{0.400pt}}
\put(1429.0,539.0){\rule[-0.200pt]{2.409pt}{0.400pt}}
\put(170.0,539.0){\rule[-0.200pt]{2.409pt}{0.400pt}}
\put(1429.0,539.0){\rule[-0.200pt]{2.409pt}{0.400pt}}
\put(170.0,539.0){\rule[-0.200pt]{2.409pt}{0.400pt}}
\put(1429.0,539.0){\rule[-0.200pt]{2.409pt}{0.400pt}}
\put(170.0,539.0){\rule[-0.200pt]{2.409pt}{0.400pt}}
\put(1429.0,539.0){\rule[-0.200pt]{2.409pt}{0.400pt}}
\put(170.0,540.0){\rule[-0.200pt]{2.409pt}{0.400pt}}
\put(1429.0,540.0){\rule[-0.200pt]{2.409pt}{0.400pt}}
\put(170.0,540.0){\rule[-0.200pt]{2.409pt}{0.400pt}}
\put(1429.0,540.0){\rule[-0.200pt]{2.409pt}{0.400pt}}
\put(170.0,540.0){\rule[-0.200pt]{2.409pt}{0.400pt}}
\put(1429.0,540.0){\rule[-0.200pt]{2.409pt}{0.400pt}}
\put(170.0,540.0){\rule[-0.200pt]{2.409pt}{0.400pt}}
\put(1429.0,540.0){\rule[-0.200pt]{2.409pt}{0.400pt}}
\put(170.0,540.0){\rule[-0.200pt]{2.409pt}{0.400pt}}
\put(1429.0,540.0){\rule[-0.200pt]{2.409pt}{0.400pt}}
\put(170.0,541.0){\rule[-0.200pt]{2.409pt}{0.400pt}}
\put(1429.0,541.0){\rule[-0.200pt]{2.409pt}{0.400pt}}
\put(170.0,541.0){\rule[-0.200pt]{2.409pt}{0.400pt}}
\put(1429.0,541.0){\rule[-0.200pt]{2.409pt}{0.400pt}}
\put(170.0,541.0){\rule[-0.200pt]{2.409pt}{0.400pt}}
\put(1429.0,541.0){\rule[-0.200pt]{2.409pt}{0.400pt}}
\put(170.0,541.0){\rule[-0.200pt]{2.409pt}{0.400pt}}
\put(1429.0,541.0){\rule[-0.200pt]{2.409pt}{0.400pt}}
\put(170.0,541.0){\rule[-0.200pt]{2.409pt}{0.400pt}}
\put(1429.0,541.0){\rule[-0.200pt]{2.409pt}{0.400pt}}
\put(170.0,541.0){\rule[-0.200pt]{2.409pt}{0.400pt}}
\put(1429.0,541.0){\rule[-0.200pt]{2.409pt}{0.400pt}}
\put(170.0,542.0){\rule[-0.200pt]{2.409pt}{0.400pt}}
\put(1429.0,542.0){\rule[-0.200pt]{2.409pt}{0.400pt}}
\put(170.0,542.0){\rule[-0.200pt]{2.409pt}{0.400pt}}
\put(1429.0,542.0){\rule[-0.200pt]{2.409pt}{0.400pt}}
\put(170.0,542.0){\rule[-0.200pt]{2.409pt}{0.400pt}}
\put(1429.0,542.0){\rule[-0.200pt]{2.409pt}{0.400pt}}
\put(170.0,542.0){\rule[-0.200pt]{2.409pt}{0.400pt}}
\put(1429.0,542.0){\rule[-0.200pt]{2.409pt}{0.400pt}}
\put(170.0,542.0){\rule[-0.200pt]{2.409pt}{0.400pt}}
\put(1429.0,542.0){\rule[-0.200pt]{2.409pt}{0.400pt}}
\put(170.0,543.0){\rule[-0.200pt]{2.409pt}{0.400pt}}
\put(1429.0,543.0){\rule[-0.200pt]{2.409pt}{0.400pt}}
\put(170.0,543.0){\rule[-0.200pt]{2.409pt}{0.400pt}}
\put(1429.0,543.0){\rule[-0.200pt]{2.409pt}{0.400pt}}
\put(170.0,543.0){\rule[-0.200pt]{2.409pt}{0.400pt}}
\put(1429.0,543.0){\rule[-0.200pt]{2.409pt}{0.400pt}}
\put(170.0,543.0){\rule[-0.200pt]{2.409pt}{0.400pt}}
\put(1429.0,543.0){\rule[-0.200pt]{2.409pt}{0.400pt}}
\put(170.0,543.0){\rule[-0.200pt]{2.409pt}{0.400pt}}
\put(1429.0,543.0){\rule[-0.200pt]{2.409pt}{0.400pt}}
\put(170.0,543.0){\rule[-0.200pt]{2.409pt}{0.400pt}}
\put(1429.0,543.0){\rule[-0.200pt]{2.409pt}{0.400pt}}
\put(170.0,544.0){\rule[-0.200pt]{2.409pt}{0.400pt}}
\put(1429.0,544.0){\rule[-0.200pt]{2.409pt}{0.400pt}}
\put(170.0,544.0){\rule[-0.200pt]{2.409pt}{0.400pt}}
\put(1429.0,544.0){\rule[-0.200pt]{2.409pt}{0.400pt}}
\put(170.0,544.0){\rule[-0.200pt]{2.409pt}{0.400pt}}
\put(1429.0,544.0){\rule[-0.200pt]{2.409pt}{0.400pt}}
\put(170.0,544.0){\rule[-0.200pt]{2.409pt}{0.400pt}}
\put(1429.0,544.0){\rule[-0.200pt]{2.409pt}{0.400pt}}
\put(170.0,544.0){\rule[-0.200pt]{2.409pt}{0.400pt}}
\put(1429.0,544.0){\rule[-0.200pt]{2.409pt}{0.400pt}}
\put(170.0,544.0){\rule[-0.200pt]{2.409pt}{0.400pt}}
\put(1429.0,544.0){\rule[-0.200pt]{2.409pt}{0.400pt}}
\put(170.0,545.0){\rule[-0.200pt]{2.409pt}{0.400pt}}
\put(1429.0,545.0){\rule[-0.200pt]{2.409pt}{0.400pt}}
\put(170.0,545.0){\rule[-0.200pt]{2.409pt}{0.400pt}}
\put(1429.0,545.0){\rule[-0.200pt]{2.409pt}{0.400pt}}
\put(170.0,545.0){\rule[-0.200pt]{2.409pt}{0.400pt}}
\put(1429.0,545.0){\rule[-0.200pt]{2.409pt}{0.400pt}}
\put(170.0,545.0){\rule[-0.200pt]{2.409pt}{0.400pt}}
\put(1429.0,545.0){\rule[-0.200pt]{2.409pt}{0.400pt}}
\put(170.0,545.0){\rule[-0.200pt]{2.409pt}{0.400pt}}
\put(1429.0,545.0){\rule[-0.200pt]{2.409pt}{0.400pt}}
\put(170.0,545.0){\rule[-0.200pt]{2.409pt}{0.400pt}}
\put(1429.0,545.0){\rule[-0.200pt]{2.409pt}{0.400pt}}
\put(170.0,546.0){\rule[-0.200pt]{2.409pt}{0.400pt}}
\put(1429.0,546.0){\rule[-0.200pt]{2.409pt}{0.400pt}}
\put(170.0,546.0){\rule[-0.200pt]{2.409pt}{0.400pt}}
\put(1429.0,546.0){\rule[-0.200pt]{2.409pt}{0.400pt}}
\put(170.0,546.0){\rule[-0.200pt]{2.409pt}{0.400pt}}
\put(1429.0,546.0){\rule[-0.200pt]{2.409pt}{0.400pt}}
\put(170.0,546.0){\rule[-0.200pt]{2.409pt}{0.400pt}}
\put(1429.0,546.0){\rule[-0.200pt]{2.409pt}{0.400pt}}
\put(170.0,546.0){\rule[-0.200pt]{2.409pt}{0.400pt}}
\put(1429.0,546.0){\rule[-0.200pt]{2.409pt}{0.400pt}}
\put(170.0,546.0){\rule[-0.200pt]{2.409pt}{0.400pt}}
\put(1429.0,546.0){\rule[-0.200pt]{2.409pt}{0.400pt}}
\put(170.0,546.0){\rule[-0.200pt]{2.409pt}{0.400pt}}
\put(1429.0,546.0){\rule[-0.200pt]{2.409pt}{0.400pt}}
\put(170.0,547.0){\rule[-0.200pt]{2.409pt}{0.400pt}}
\put(1429.0,547.0){\rule[-0.200pt]{2.409pt}{0.400pt}}
\put(170.0,547.0){\rule[-0.200pt]{2.409pt}{0.400pt}}
\put(1429.0,547.0){\rule[-0.200pt]{2.409pt}{0.400pt}}
\put(170.0,547.0){\rule[-0.200pt]{2.409pt}{0.400pt}}
\put(1429.0,547.0){\rule[-0.200pt]{2.409pt}{0.400pt}}
\put(170.0,547.0){\rule[-0.200pt]{2.409pt}{0.400pt}}
\put(1429.0,547.0){\rule[-0.200pt]{2.409pt}{0.400pt}}
\put(170.0,547.0){\rule[-0.200pt]{2.409pt}{0.400pt}}
\put(1429.0,547.0){\rule[-0.200pt]{2.409pt}{0.400pt}}
\put(170.0,547.0){\rule[-0.200pt]{2.409pt}{0.400pt}}
\put(1429.0,547.0){\rule[-0.200pt]{2.409pt}{0.400pt}}
\put(170.0,548.0){\rule[-0.200pt]{2.409pt}{0.400pt}}
\put(1429.0,548.0){\rule[-0.200pt]{2.409pt}{0.400pt}}
\put(170.0,548.0){\rule[-0.200pt]{2.409pt}{0.400pt}}
\put(1429.0,548.0){\rule[-0.200pt]{2.409pt}{0.400pt}}
\put(170.0,548.0){\rule[-0.200pt]{2.409pt}{0.400pt}}
\put(1429.0,548.0){\rule[-0.200pt]{2.409pt}{0.400pt}}
\put(170.0,548.0){\rule[-0.200pt]{2.409pt}{0.400pt}}
\put(1429.0,548.0){\rule[-0.200pt]{2.409pt}{0.400pt}}
\put(170.0,548.0){\rule[-0.200pt]{2.409pt}{0.400pt}}
\put(1429.0,548.0){\rule[-0.200pt]{2.409pt}{0.400pt}}
\put(170.0,548.0){\rule[-0.200pt]{2.409pt}{0.400pt}}
\put(1429.0,548.0){\rule[-0.200pt]{2.409pt}{0.400pt}}
\put(170.0,548.0){\rule[-0.200pt]{2.409pt}{0.400pt}}
\put(1429.0,548.0){\rule[-0.200pt]{2.409pt}{0.400pt}}
\put(170.0,549.0){\rule[-0.200pt]{2.409pt}{0.400pt}}
\put(1429.0,549.0){\rule[-0.200pt]{2.409pt}{0.400pt}}
\put(170.0,549.0){\rule[-0.200pt]{2.409pt}{0.400pt}}
\put(1429.0,549.0){\rule[-0.200pt]{2.409pt}{0.400pt}}
\put(170.0,549.0){\rule[-0.200pt]{2.409pt}{0.400pt}}
\put(1429.0,549.0){\rule[-0.200pt]{2.409pt}{0.400pt}}
\put(170.0,549.0){\rule[-0.200pt]{2.409pt}{0.400pt}}
\put(1429.0,549.0){\rule[-0.200pt]{2.409pt}{0.400pt}}
\put(170.0,549.0){\rule[-0.200pt]{2.409pt}{0.400pt}}
\put(1429.0,549.0){\rule[-0.200pt]{2.409pt}{0.400pt}}
\put(170.0,549.0){\rule[-0.200pt]{2.409pt}{0.400pt}}
\put(1429.0,549.0){\rule[-0.200pt]{2.409pt}{0.400pt}}
\put(170.0,549.0){\rule[-0.200pt]{2.409pt}{0.400pt}}
\put(1429.0,549.0){\rule[-0.200pt]{2.409pt}{0.400pt}}
\put(170.0,550.0){\rule[-0.200pt]{2.409pt}{0.400pt}}
\put(1429.0,550.0){\rule[-0.200pt]{2.409pt}{0.400pt}}
\put(170.0,550.0){\rule[-0.200pt]{2.409pt}{0.400pt}}
\put(1429.0,550.0){\rule[-0.200pt]{2.409pt}{0.400pt}}
\put(170.0,550.0){\rule[-0.200pt]{2.409pt}{0.400pt}}
\put(1429.0,550.0){\rule[-0.200pt]{2.409pt}{0.400pt}}
\put(170.0,550.0){\rule[-0.200pt]{2.409pt}{0.400pt}}
\put(1429.0,550.0){\rule[-0.200pt]{2.409pt}{0.400pt}}
\put(170.0,550.0){\rule[-0.200pt]{2.409pt}{0.400pt}}
\put(1429.0,550.0){\rule[-0.200pt]{2.409pt}{0.400pt}}
\put(170.0,550.0){\rule[-0.200pt]{2.409pt}{0.400pt}}
\put(1429.0,550.0){\rule[-0.200pt]{2.409pt}{0.400pt}}
\put(170.0,550.0){\rule[-0.200pt]{2.409pt}{0.400pt}}
\put(1429.0,550.0){\rule[-0.200pt]{2.409pt}{0.400pt}}
\put(170.0,551.0){\rule[-0.200pt]{2.409pt}{0.400pt}}
\put(1429.0,551.0){\rule[-0.200pt]{2.409pt}{0.400pt}}
\put(170.0,551.0){\rule[-0.200pt]{2.409pt}{0.400pt}}
\put(1429.0,551.0){\rule[-0.200pt]{2.409pt}{0.400pt}}
\put(170.0,551.0){\rule[-0.200pt]{2.409pt}{0.400pt}}
\put(1429.0,551.0){\rule[-0.200pt]{2.409pt}{0.400pt}}
\put(170.0,551.0){\rule[-0.200pt]{2.409pt}{0.400pt}}
\put(1429.0,551.0){\rule[-0.200pt]{2.409pt}{0.400pt}}
\put(170.0,551.0){\rule[-0.200pt]{2.409pt}{0.400pt}}
\put(1429.0,551.0){\rule[-0.200pt]{2.409pt}{0.400pt}}
\put(170.0,551.0){\rule[-0.200pt]{2.409pt}{0.400pt}}
\put(1429.0,551.0){\rule[-0.200pt]{2.409pt}{0.400pt}}
\put(170.0,551.0){\rule[-0.200pt]{2.409pt}{0.400pt}}
\put(1429.0,551.0){\rule[-0.200pt]{2.409pt}{0.400pt}}
\put(170.0,552.0){\rule[-0.200pt]{2.409pt}{0.400pt}}
\put(1429.0,552.0){\rule[-0.200pt]{2.409pt}{0.400pt}}
\put(170.0,552.0){\rule[-0.200pt]{2.409pt}{0.400pt}}
\put(1429.0,552.0){\rule[-0.200pt]{2.409pt}{0.400pt}}
\put(170.0,552.0){\rule[-0.200pt]{2.409pt}{0.400pt}}
\put(1429.0,552.0){\rule[-0.200pt]{2.409pt}{0.400pt}}
\put(170.0,552.0){\rule[-0.200pt]{2.409pt}{0.400pt}}
\put(1429.0,552.0){\rule[-0.200pt]{2.409pt}{0.400pt}}
\put(170.0,552.0){\rule[-0.200pt]{2.409pt}{0.400pt}}
\put(1429.0,552.0){\rule[-0.200pt]{2.409pt}{0.400pt}}
\put(170.0,552.0){\rule[-0.200pt]{2.409pt}{0.400pt}}
\put(1429.0,552.0){\rule[-0.200pt]{2.409pt}{0.400pt}}
\put(170.0,552.0){\rule[-0.200pt]{2.409pt}{0.400pt}}
\put(1429.0,552.0){\rule[-0.200pt]{2.409pt}{0.400pt}}
\put(170.0,553.0){\rule[-0.200pt]{2.409pt}{0.400pt}}
\put(1429.0,553.0){\rule[-0.200pt]{2.409pt}{0.400pt}}
\put(170.0,553.0){\rule[-0.200pt]{2.409pt}{0.400pt}}
\put(1429.0,553.0){\rule[-0.200pt]{2.409pt}{0.400pt}}
\put(170.0,553.0){\rule[-0.200pt]{2.409pt}{0.400pt}}
\put(1429.0,553.0){\rule[-0.200pt]{2.409pt}{0.400pt}}
\put(170.0,553.0){\rule[-0.200pt]{2.409pt}{0.400pt}}
\put(1429.0,553.0){\rule[-0.200pt]{2.409pt}{0.400pt}}
\put(170.0,553.0){\rule[-0.200pt]{2.409pt}{0.400pt}}
\put(1429.0,553.0){\rule[-0.200pt]{2.409pt}{0.400pt}}
\put(170.0,553.0){\rule[-0.200pt]{2.409pt}{0.400pt}}
\put(1429.0,553.0){\rule[-0.200pt]{2.409pt}{0.400pt}}
\put(170.0,553.0){\rule[-0.200pt]{2.409pt}{0.400pt}}
\put(1429.0,553.0){\rule[-0.200pt]{2.409pt}{0.400pt}}
\put(170.0,553.0){\rule[-0.200pt]{2.409pt}{0.400pt}}
\put(1429.0,553.0){\rule[-0.200pt]{2.409pt}{0.400pt}}
\put(170.0,554.0){\rule[-0.200pt]{2.409pt}{0.400pt}}
\put(1429.0,554.0){\rule[-0.200pt]{2.409pt}{0.400pt}}
\put(170.0,554.0){\rule[-0.200pt]{2.409pt}{0.400pt}}
\put(1429.0,554.0){\rule[-0.200pt]{2.409pt}{0.400pt}}
\put(170.0,554.0){\rule[-0.200pt]{2.409pt}{0.400pt}}
\put(1429.0,554.0){\rule[-0.200pt]{2.409pt}{0.400pt}}
\put(170.0,554.0){\rule[-0.200pt]{2.409pt}{0.400pt}}
\put(1429.0,554.0){\rule[-0.200pt]{2.409pt}{0.400pt}}
\put(170.0,554.0){\rule[-0.200pt]{2.409pt}{0.400pt}}
\put(1429.0,554.0){\rule[-0.200pt]{2.409pt}{0.400pt}}
\put(170.0,554.0){\rule[-0.200pt]{2.409pt}{0.400pt}}
\put(1429.0,554.0){\rule[-0.200pt]{2.409pt}{0.400pt}}
\put(170.0,554.0){\rule[-0.200pt]{2.409pt}{0.400pt}}
\put(1429.0,554.0){\rule[-0.200pt]{2.409pt}{0.400pt}}
\put(170.0,554.0){\rule[-0.200pt]{2.409pt}{0.400pt}}
\put(1429.0,554.0){\rule[-0.200pt]{2.409pt}{0.400pt}}
\put(170.0,555.0){\rule[-0.200pt]{2.409pt}{0.400pt}}
\put(1429.0,555.0){\rule[-0.200pt]{2.409pt}{0.400pt}}
\put(170.0,555.0){\rule[-0.200pt]{2.409pt}{0.400pt}}
\put(1429.0,555.0){\rule[-0.200pt]{2.409pt}{0.400pt}}
\put(170.0,555.0){\rule[-0.200pt]{2.409pt}{0.400pt}}
\put(1429.0,555.0){\rule[-0.200pt]{2.409pt}{0.400pt}}
\put(170.0,555.0){\rule[-0.200pt]{2.409pt}{0.400pt}}
\put(1429.0,555.0){\rule[-0.200pt]{2.409pt}{0.400pt}}
\put(170.0,555.0){\rule[-0.200pt]{2.409pt}{0.400pt}}
\put(1429.0,555.0){\rule[-0.200pt]{2.409pt}{0.400pt}}
\put(170.0,555.0){\rule[-0.200pt]{2.409pt}{0.400pt}}
\put(1429.0,555.0){\rule[-0.200pt]{2.409pt}{0.400pt}}
\put(170.0,555.0){\rule[-0.200pt]{2.409pt}{0.400pt}}
\put(1429.0,555.0){\rule[-0.200pt]{2.409pt}{0.400pt}}
\put(170.0,555.0){\rule[-0.200pt]{2.409pt}{0.400pt}}
\put(1429.0,555.0){\rule[-0.200pt]{2.409pt}{0.400pt}}
\put(170.0,556.0){\rule[-0.200pt]{2.409pt}{0.400pt}}
\put(1429.0,556.0){\rule[-0.200pt]{2.409pt}{0.400pt}}
\put(170.0,556.0){\rule[-0.200pt]{2.409pt}{0.400pt}}
\put(1429.0,556.0){\rule[-0.200pt]{2.409pt}{0.400pt}}
\put(170.0,556.0){\rule[-0.200pt]{2.409pt}{0.400pt}}
\put(1429.0,556.0){\rule[-0.200pt]{2.409pt}{0.400pt}}
\put(170.0,556.0){\rule[-0.200pt]{2.409pt}{0.400pt}}
\put(1429.0,556.0){\rule[-0.200pt]{2.409pt}{0.400pt}}
\put(170.0,556.0){\rule[-0.200pt]{2.409pt}{0.400pt}}
\put(1429.0,556.0){\rule[-0.200pt]{2.409pt}{0.400pt}}
\put(170.0,556.0){\rule[-0.200pt]{2.409pt}{0.400pt}}
\put(1429.0,556.0){\rule[-0.200pt]{2.409pt}{0.400pt}}
\put(170.0,556.0){\rule[-0.200pt]{2.409pt}{0.400pt}}
\put(1429.0,556.0){\rule[-0.200pt]{2.409pt}{0.400pt}}
\put(170.0,556.0){\rule[-0.200pt]{2.409pt}{0.400pt}}
\put(1429.0,556.0){\rule[-0.200pt]{2.409pt}{0.400pt}}
\put(170.0,557.0){\rule[-0.200pt]{2.409pt}{0.400pt}}
\put(1429.0,557.0){\rule[-0.200pt]{2.409pt}{0.400pt}}
\put(170.0,557.0){\rule[-0.200pt]{2.409pt}{0.400pt}}
\put(1429.0,557.0){\rule[-0.200pt]{2.409pt}{0.400pt}}
\put(170.0,557.0){\rule[-0.200pt]{2.409pt}{0.400pt}}
\put(1429.0,557.0){\rule[-0.200pt]{2.409pt}{0.400pt}}
\put(170.0,557.0){\rule[-0.200pt]{2.409pt}{0.400pt}}
\put(1429.0,557.0){\rule[-0.200pt]{2.409pt}{0.400pt}}
\put(170.0,557.0){\rule[-0.200pt]{2.409pt}{0.400pt}}
\put(1429.0,557.0){\rule[-0.200pt]{2.409pt}{0.400pt}}
\put(170.0,557.0){\rule[-0.200pt]{2.409pt}{0.400pt}}
\put(1429.0,557.0){\rule[-0.200pt]{2.409pt}{0.400pt}}
\put(170.0,557.0){\rule[-0.200pt]{2.409pt}{0.400pt}}
\put(1429.0,557.0){\rule[-0.200pt]{2.409pt}{0.400pt}}
\put(170.0,557.0){\rule[-0.200pt]{2.409pt}{0.400pt}}
\put(1429.0,557.0){\rule[-0.200pt]{2.409pt}{0.400pt}}
\put(170.0,558.0){\rule[-0.200pt]{2.409pt}{0.400pt}}
\put(1429.0,558.0){\rule[-0.200pt]{2.409pt}{0.400pt}}
\put(170.0,558.0){\rule[-0.200pt]{2.409pt}{0.400pt}}
\put(1429.0,558.0){\rule[-0.200pt]{2.409pt}{0.400pt}}
\put(170.0,558.0){\rule[-0.200pt]{2.409pt}{0.400pt}}
\put(1429.0,558.0){\rule[-0.200pt]{2.409pt}{0.400pt}}
\put(170.0,558.0){\rule[-0.200pt]{2.409pt}{0.400pt}}
\put(1429.0,558.0){\rule[-0.200pt]{2.409pt}{0.400pt}}
\put(170.0,558.0){\rule[-0.200pt]{2.409pt}{0.400pt}}
\put(1429.0,558.0){\rule[-0.200pt]{2.409pt}{0.400pt}}
\put(170.0,558.0){\rule[-0.200pt]{2.409pt}{0.400pt}}
\put(1429.0,558.0){\rule[-0.200pt]{2.409pt}{0.400pt}}
\put(170.0,558.0){\rule[-0.200pt]{2.409pt}{0.400pt}}
\put(1429.0,558.0){\rule[-0.200pt]{2.409pt}{0.400pt}}
\put(170.0,558.0){\rule[-0.200pt]{2.409pt}{0.400pt}}
\put(1429.0,558.0){\rule[-0.200pt]{2.409pt}{0.400pt}}
\put(170.0,558.0){\rule[-0.200pt]{2.409pt}{0.400pt}}
\put(1429.0,558.0){\rule[-0.200pt]{2.409pt}{0.400pt}}
\put(170.0,559.0){\rule[-0.200pt]{2.409pt}{0.400pt}}
\put(1429.0,559.0){\rule[-0.200pt]{2.409pt}{0.400pt}}
\put(170.0,559.0){\rule[-0.200pt]{2.409pt}{0.400pt}}
\put(1429.0,559.0){\rule[-0.200pt]{2.409pt}{0.400pt}}
\put(170.0,559.0){\rule[-0.200pt]{2.409pt}{0.400pt}}
\put(1429.0,559.0){\rule[-0.200pt]{2.409pt}{0.400pt}}
\put(170.0,559.0){\rule[-0.200pt]{2.409pt}{0.400pt}}
\put(1429.0,559.0){\rule[-0.200pt]{2.409pt}{0.400pt}}
\put(170.0,559.0){\rule[-0.200pt]{2.409pt}{0.400pt}}
\put(1429.0,559.0){\rule[-0.200pt]{2.409pt}{0.400pt}}
\put(170.0,559.0){\rule[-0.200pt]{2.409pt}{0.400pt}}
\put(1429.0,559.0){\rule[-0.200pt]{2.409pt}{0.400pt}}
\put(170.0,559.0){\rule[-0.200pt]{2.409pt}{0.400pt}}
\put(1429.0,559.0){\rule[-0.200pt]{2.409pt}{0.400pt}}
\put(170.0,559.0){\rule[-0.200pt]{2.409pt}{0.400pt}}
\put(1429.0,559.0){\rule[-0.200pt]{2.409pt}{0.400pt}}
\put(170.0,559.0){\rule[-0.200pt]{2.409pt}{0.400pt}}
\put(1429.0,559.0){\rule[-0.200pt]{2.409pt}{0.400pt}}
\put(170.0,560.0){\rule[-0.200pt]{2.409pt}{0.400pt}}
\put(1429.0,560.0){\rule[-0.200pt]{2.409pt}{0.400pt}}
\put(170.0,560.0){\rule[-0.200pt]{2.409pt}{0.400pt}}
\put(1429.0,560.0){\rule[-0.200pt]{2.409pt}{0.400pt}}
\put(170.0,560.0){\rule[-0.200pt]{2.409pt}{0.400pt}}
\put(1429.0,560.0){\rule[-0.200pt]{2.409pt}{0.400pt}}
\put(170.0,560.0){\rule[-0.200pt]{2.409pt}{0.400pt}}
\put(1429.0,560.0){\rule[-0.200pt]{2.409pt}{0.400pt}}
\put(170.0,560.0){\rule[-0.200pt]{2.409pt}{0.400pt}}
\put(1429.0,560.0){\rule[-0.200pt]{2.409pt}{0.400pt}}
\put(170.0,560.0){\rule[-0.200pt]{2.409pt}{0.400pt}}
\put(1429.0,560.0){\rule[-0.200pt]{2.409pt}{0.400pt}}
\put(170.0,560.0){\rule[-0.200pt]{2.409pt}{0.400pt}}
\put(1429.0,560.0){\rule[-0.200pt]{2.409pt}{0.400pt}}
\put(170.0,560.0){\rule[-0.200pt]{2.409pt}{0.400pt}}
\put(1429.0,560.0){\rule[-0.200pt]{2.409pt}{0.400pt}}
\put(170.0,560.0){\rule[-0.200pt]{2.409pt}{0.400pt}}
\put(1429.0,560.0){\rule[-0.200pt]{2.409pt}{0.400pt}}
\put(170.0,561.0){\rule[-0.200pt]{2.409pt}{0.400pt}}
\put(1429.0,561.0){\rule[-0.200pt]{2.409pt}{0.400pt}}
\put(170.0,561.0){\rule[-0.200pt]{2.409pt}{0.400pt}}
\put(1429.0,561.0){\rule[-0.200pt]{2.409pt}{0.400pt}}
\put(170.0,561.0){\rule[-0.200pt]{2.409pt}{0.400pt}}
\put(1429.0,561.0){\rule[-0.200pt]{2.409pt}{0.400pt}}
\put(170.0,561.0){\rule[-0.200pt]{2.409pt}{0.400pt}}
\put(1429.0,561.0){\rule[-0.200pt]{2.409pt}{0.400pt}}
\put(170.0,561.0){\rule[-0.200pt]{2.409pt}{0.400pt}}
\put(1429.0,561.0){\rule[-0.200pt]{2.409pt}{0.400pt}}
\put(170.0,561.0){\rule[-0.200pt]{2.409pt}{0.400pt}}
\put(1429.0,561.0){\rule[-0.200pt]{2.409pt}{0.400pt}}
\put(170.0,561.0){\rule[-0.200pt]{2.409pt}{0.400pt}}
\put(1429.0,561.0){\rule[-0.200pt]{2.409pt}{0.400pt}}
\put(170.0,561.0){\rule[-0.200pt]{2.409pt}{0.400pt}}
\put(1429.0,561.0){\rule[-0.200pt]{2.409pt}{0.400pt}}
\put(170.0,561.0){\rule[-0.200pt]{2.409pt}{0.400pt}}
\put(1429.0,561.0){\rule[-0.200pt]{2.409pt}{0.400pt}}
\put(170.0,561.0){\rule[-0.200pt]{2.409pt}{0.400pt}}
\put(1429.0,561.0){\rule[-0.200pt]{2.409pt}{0.400pt}}
\put(170.0,562.0){\rule[-0.200pt]{2.409pt}{0.400pt}}
\put(1429.0,562.0){\rule[-0.200pt]{2.409pt}{0.400pt}}
\put(170.0,562.0){\rule[-0.200pt]{2.409pt}{0.400pt}}
\put(1429.0,562.0){\rule[-0.200pt]{2.409pt}{0.400pt}}
\put(170.0,562.0){\rule[-0.200pt]{2.409pt}{0.400pt}}
\put(1429.0,562.0){\rule[-0.200pt]{2.409pt}{0.400pt}}
\put(170.0,562.0){\rule[-0.200pt]{2.409pt}{0.400pt}}
\put(1429.0,562.0){\rule[-0.200pt]{2.409pt}{0.400pt}}
\put(170.0,562.0){\rule[-0.200pt]{2.409pt}{0.400pt}}
\put(1429.0,562.0){\rule[-0.200pt]{2.409pt}{0.400pt}}
\put(170.0,562.0){\rule[-0.200pt]{2.409pt}{0.400pt}}
\put(1429.0,562.0){\rule[-0.200pt]{2.409pt}{0.400pt}}
\put(170.0,562.0){\rule[-0.200pt]{2.409pt}{0.400pt}}
\put(1429.0,562.0){\rule[-0.200pt]{2.409pt}{0.400pt}}
\put(170.0,562.0){\rule[-0.200pt]{2.409pt}{0.400pt}}
\put(1429.0,562.0){\rule[-0.200pt]{2.409pt}{0.400pt}}
\put(170.0,562.0){\rule[-0.200pt]{2.409pt}{0.400pt}}
\put(1429.0,562.0){\rule[-0.200pt]{2.409pt}{0.400pt}}
\put(170.0,563.0){\rule[-0.200pt]{2.409pt}{0.400pt}}
\put(1429.0,563.0){\rule[-0.200pt]{2.409pt}{0.400pt}}
\put(170.0,563.0){\rule[-0.200pt]{2.409pt}{0.400pt}}
\put(1429.0,563.0){\rule[-0.200pt]{2.409pt}{0.400pt}}
\put(170.0,563.0){\rule[-0.200pt]{2.409pt}{0.400pt}}
\put(1429.0,563.0){\rule[-0.200pt]{2.409pt}{0.400pt}}
\put(170.0,563.0){\rule[-0.200pt]{2.409pt}{0.400pt}}
\put(1429.0,563.0){\rule[-0.200pt]{2.409pt}{0.400pt}}
\put(170.0,563.0){\rule[-0.200pt]{2.409pt}{0.400pt}}
\put(1429.0,563.0){\rule[-0.200pt]{2.409pt}{0.400pt}}
\put(170.0,563.0){\rule[-0.200pt]{2.409pt}{0.400pt}}
\put(1429.0,563.0){\rule[-0.200pt]{2.409pt}{0.400pt}}
\put(170.0,563.0){\rule[-0.200pt]{2.409pt}{0.400pt}}
\put(1429.0,563.0){\rule[-0.200pt]{2.409pt}{0.400pt}}
\put(170.0,563.0){\rule[-0.200pt]{2.409pt}{0.400pt}}
\put(1429.0,563.0){\rule[-0.200pt]{2.409pt}{0.400pt}}
\put(170.0,563.0){\rule[-0.200pt]{2.409pt}{0.400pt}}
\put(1429.0,563.0){\rule[-0.200pt]{2.409pt}{0.400pt}}
\put(170.0,563.0){\rule[-0.200pt]{2.409pt}{0.400pt}}
\put(1429.0,563.0){\rule[-0.200pt]{2.409pt}{0.400pt}}
\put(170.0,564.0){\rule[-0.200pt]{2.409pt}{0.400pt}}
\put(1429.0,564.0){\rule[-0.200pt]{2.409pt}{0.400pt}}
\put(170.0,564.0){\rule[-0.200pt]{2.409pt}{0.400pt}}
\put(1429.0,564.0){\rule[-0.200pt]{2.409pt}{0.400pt}}
\put(170.0,564.0){\rule[-0.200pt]{2.409pt}{0.400pt}}
\put(1429.0,564.0){\rule[-0.200pt]{2.409pt}{0.400pt}}
\put(170.0,564.0){\rule[-0.200pt]{2.409pt}{0.400pt}}
\put(1429.0,564.0){\rule[-0.200pt]{2.409pt}{0.400pt}}
\put(170.0,564.0){\rule[-0.200pt]{2.409pt}{0.400pt}}
\put(1429.0,564.0){\rule[-0.200pt]{2.409pt}{0.400pt}}
\put(170.0,564.0){\rule[-0.200pt]{2.409pt}{0.400pt}}
\put(1429.0,564.0){\rule[-0.200pt]{2.409pt}{0.400pt}}
\put(170.0,564.0){\rule[-0.200pt]{2.409pt}{0.400pt}}
\put(1429.0,564.0){\rule[-0.200pt]{2.409pt}{0.400pt}}
\put(170.0,564.0){\rule[-0.200pt]{2.409pt}{0.400pt}}
\put(1429.0,564.0){\rule[-0.200pt]{2.409pt}{0.400pt}}
\put(170.0,564.0){\rule[-0.200pt]{2.409pt}{0.400pt}}
\put(1429.0,564.0){\rule[-0.200pt]{2.409pt}{0.400pt}}
\put(170.0,564.0){\rule[-0.200pt]{2.409pt}{0.400pt}}
\put(1429.0,564.0){\rule[-0.200pt]{2.409pt}{0.400pt}}
\put(170.0,565.0){\rule[-0.200pt]{2.409pt}{0.400pt}}
\put(1429.0,565.0){\rule[-0.200pt]{2.409pt}{0.400pt}}
\put(170.0,565.0){\rule[-0.200pt]{2.409pt}{0.400pt}}
\put(1429.0,565.0){\rule[-0.200pt]{2.409pt}{0.400pt}}
\put(170.0,565.0){\rule[-0.200pt]{2.409pt}{0.400pt}}
\put(1429.0,565.0){\rule[-0.200pt]{2.409pt}{0.400pt}}
\put(170.0,565.0){\rule[-0.200pt]{2.409pt}{0.400pt}}
\put(1429.0,565.0){\rule[-0.200pt]{2.409pt}{0.400pt}}
\put(170.0,565.0){\rule[-0.200pt]{2.409pt}{0.400pt}}
\put(1429.0,565.0){\rule[-0.200pt]{2.409pt}{0.400pt}}
\put(170.0,565.0){\rule[-0.200pt]{2.409pt}{0.400pt}}
\put(1429.0,565.0){\rule[-0.200pt]{2.409pt}{0.400pt}}
\put(170.0,565.0){\rule[-0.200pt]{2.409pt}{0.400pt}}
\put(1429.0,565.0){\rule[-0.200pt]{2.409pt}{0.400pt}}
\put(170.0,565.0){\rule[-0.200pt]{2.409pt}{0.400pt}}
\put(1429.0,565.0){\rule[-0.200pt]{2.409pt}{0.400pt}}
\put(170.0,565.0){\rule[-0.200pt]{2.409pt}{0.400pt}}
\put(1429.0,565.0){\rule[-0.200pt]{2.409pt}{0.400pt}}
\put(170.0,565.0){\rule[-0.200pt]{2.409pt}{0.400pt}}
\put(1429.0,565.0){\rule[-0.200pt]{2.409pt}{0.400pt}}
\put(170.0,565.0){\rule[-0.200pt]{2.409pt}{0.400pt}}
\put(1429.0,565.0){\rule[-0.200pt]{2.409pt}{0.400pt}}
\put(170.0,566.0){\rule[-0.200pt]{2.409pt}{0.400pt}}
\put(1429.0,566.0){\rule[-0.200pt]{2.409pt}{0.400pt}}
\put(170.0,566.0){\rule[-0.200pt]{2.409pt}{0.400pt}}
\put(1429.0,566.0){\rule[-0.200pt]{2.409pt}{0.400pt}}
\put(170.0,566.0){\rule[-0.200pt]{2.409pt}{0.400pt}}
\put(1429.0,566.0){\rule[-0.200pt]{2.409pt}{0.400pt}}
\put(170.0,566.0){\rule[-0.200pt]{2.409pt}{0.400pt}}
\put(1429.0,566.0){\rule[-0.200pt]{2.409pt}{0.400pt}}
\put(170.0,566.0){\rule[-0.200pt]{2.409pt}{0.400pt}}
\put(1429.0,566.0){\rule[-0.200pt]{2.409pt}{0.400pt}}
\put(170.0,566.0){\rule[-0.200pt]{2.409pt}{0.400pt}}
\put(1429.0,566.0){\rule[-0.200pt]{2.409pt}{0.400pt}}
\put(170.0,566.0){\rule[-0.200pt]{2.409pt}{0.400pt}}
\put(1429.0,566.0){\rule[-0.200pt]{2.409pt}{0.400pt}}
\put(170.0,566.0){\rule[-0.200pt]{2.409pt}{0.400pt}}
\put(1429.0,566.0){\rule[-0.200pt]{2.409pt}{0.400pt}}
\put(170.0,566.0){\rule[-0.200pt]{2.409pt}{0.400pt}}
\put(1429.0,566.0){\rule[-0.200pt]{2.409pt}{0.400pt}}
\put(170.0,566.0){\rule[-0.200pt]{2.409pt}{0.400pt}}
\put(1429.0,566.0){\rule[-0.200pt]{2.409pt}{0.400pt}}
\put(170.0,566.0){\rule[-0.200pt]{2.409pt}{0.400pt}}
\put(1429.0,566.0){\rule[-0.200pt]{2.409pt}{0.400pt}}
\put(170.0,567.0){\rule[-0.200pt]{2.409pt}{0.400pt}}
\put(1429.0,567.0){\rule[-0.200pt]{2.409pt}{0.400pt}}
\put(170.0,567.0){\rule[-0.200pt]{2.409pt}{0.400pt}}
\put(1429.0,567.0){\rule[-0.200pt]{2.409pt}{0.400pt}}
\put(170.0,567.0){\rule[-0.200pt]{2.409pt}{0.400pt}}
\put(1429.0,567.0){\rule[-0.200pt]{2.409pt}{0.400pt}}
\put(170.0,567.0){\rule[-0.200pt]{2.409pt}{0.400pt}}
\put(1429.0,567.0){\rule[-0.200pt]{2.409pt}{0.400pt}}
\put(170.0,567.0){\rule[-0.200pt]{2.409pt}{0.400pt}}
\put(1429.0,567.0){\rule[-0.200pt]{2.409pt}{0.400pt}}
\put(170.0,567.0){\rule[-0.200pt]{2.409pt}{0.400pt}}
\put(1429.0,567.0){\rule[-0.200pt]{2.409pt}{0.400pt}}
\put(170.0,567.0){\rule[-0.200pt]{2.409pt}{0.400pt}}
\put(1429.0,567.0){\rule[-0.200pt]{2.409pt}{0.400pt}}
\put(170.0,567.0){\rule[-0.200pt]{2.409pt}{0.400pt}}
\put(1429.0,567.0){\rule[-0.200pt]{2.409pt}{0.400pt}}
\put(170.0,567.0){\rule[-0.200pt]{2.409pt}{0.400pt}}
\put(1429.0,567.0){\rule[-0.200pt]{2.409pt}{0.400pt}}
\put(170.0,567.0){\rule[-0.200pt]{2.409pt}{0.400pt}}
\put(1429.0,567.0){\rule[-0.200pt]{2.409pt}{0.400pt}}
\put(170.0,567.0){\rule[-0.200pt]{2.409pt}{0.400pt}}
\put(1429.0,567.0){\rule[-0.200pt]{2.409pt}{0.400pt}}
\put(170.0,568.0){\rule[-0.200pt]{2.409pt}{0.400pt}}
\put(1429.0,568.0){\rule[-0.200pt]{2.409pt}{0.400pt}}
\put(170.0,568.0){\rule[-0.200pt]{2.409pt}{0.400pt}}
\put(1429.0,568.0){\rule[-0.200pt]{2.409pt}{0.400pt}}
\put(170.0,568.0){\rule[-0.200pt]{2.409pt}{0.400pt}}
\put(1429.0,568.0){\rule[-0.200pt]{2.409pt}{0.400pt}}
\put(170.0,568.0){\rule[-0.200pt]{2.409pt}{0.400pt}}
\put(1429.0,568.0){\rule[-0.200pt]{2.409pt}{0.400pt}}
\put(170.0,568.0){\rule[-0.200pt]{2.409pt}{0.400pt}}
\put(1429.0,568.0){\rule[-0.200pt]{2.409pt}{0.400pt}}
\put(170.0,568.0){\rule[-0.200pt]{2.409pt}{0.400pt}}
\put(1429.0,568.0){\rule[-0.200pt]{2.409pt}{0.400pt}}
\put(170.0,568.0){\rule[-0.200pt]{2.409pt}{0.400pt}}
\put(1429.0,568.0){\rule[-0.200pt]{2.409pt}{0.400pt}}
\put(170.0,568.0){\rule[-0.200pt]{2.409pt}{0.400pt}}
\put(1429.0,568.0){\rule[-0.200pt]{2.409pt}{0.400pt}}
\put(170.0,568.0){\rule[-0.200pt]{2.409pt}{0.400pt}}
\put(1429.0,568.0){\rule[-0.200pt]{2.409pt}{0.400pt}}
\put(170.0,568.0){\rule[-0.200pt]{2.409pt}{0.400pt}}
\put(1429.0,568.0){\rule[-0.200pt]{2.409pt}{0.400pt}}
\put(170.0,568.0){\rule[-0.200pt]{2.409pt}{0.400pt}}
\put(1429.0,568.0){\rule[-0.200pt]{2.409pt}{0.400pt}}
\put(170.0,569.0){\rule[-0.200pt]{2.409pt}{0.400pt}}
\put(1429.0,569.0){\rule[-0.200pt]{2.409pt}{0.400pt}}
\put(170.0,569.0){\rule[-0.200pt]{2.409pt}{0.400pt}}
\put(1429.0,569.0){\rule[-0.200pt]{2.409pt}{0.400pt}}
\put(170.0,569.0){\rule[-0.200pt]{2.409pt}{0.400pt}}
\put(1429.0,569.0){\rule[-0.200pt]{2.409pt}{0.400pt}}
\put(170.0,569.0){\rule[-0.200pt]{2.409pt}{0.400pt}}
\put(1429.0,569.0){\rule[-0.200pt]{2.409pt}{0.400pt}}
\put(170.0,569.0){\rule[-0.200pt]{2.409pt}{0.400pt}}
\put(1429.0,569.0){\rule[-0.200pt]{2.409pt}{0.400pt}}
\put(170.0,569.0){\rule[-0.200pt]{2.409pt}{0.400pt}}
\put(1429.0,569.0){\rule[-0.200pt]{2.409pt}{0.400pt}}
\put(170.0,569.0){\rule[-0.200pt]{2.409pt}{0.400pt}}
\put(1429.0,569.0){\rule[-0.200pt]{2.409pt}{0.400pt}}
\put(170.0,569.0){\rule[-0.200pt]{2.409pt}{0.400pt}}
\put(1429.0,569.0){\rule[-0.200pt]{2.409pt}{0.400pt}}
\put(170.0,569.0){\rule[-0.200pt]{2.409pt}{0.400pt}}
\put(1429.0,569.0){\rule[-0.200pt]{2.409pt}{0.400pt}}
\put(170.0,569.0){\rule[-0.200pt]{2.409pt}{0.400pt}}
\put(1429.0,569.0){\rule[-0.200pt]{2.409pt}{0.400pt}}
\put(170.0,569.0){\rule[-0.200pt]{2.409pt}{0.400pt}}
\put(1429.0,569.0){\rule[-0.200pt]{2.409pt}{0.400pt}}
\put(170.0,569.0){\rule[-0.200pt]{2.409pt}{0.400pt}}
\put(1429.0,569.0){\rule[-0.200pt]{2.409pt}{0.400pt}}
\put(170.0,570.0){\rule[-0.200pt]{2.409pt}{0.400pt}}
\put(1429.0,570.0){\rule[-0.200pt]{2.409pt}{0.400pt}}
\put(170.0,570.0){\rule[-0.200pt]{2.409pt}{0.400pt}}
\put(1429.0,570.0){\rule[-0.200pt]{2.409pt}{0.400pt}}
\put(170.0,570.0){\rule[-0.200pt]{2.409pt}{0.400pt}}
\put(1429.0,570.0){\rule[-0.200pt]{2.409pt}{0.400pt}}
\put(170.0,570.0){\rule[-0.200pt]{2.409pt}{0.400pt}}
\put(1429.0,570.0){\rule[-0.200pt]{2.409pt}{0.400pt}}
\put(170.0,570.0){\rule[-0.200pt]{2.409pt}{0.400pt}}
\put(1429.0,570.0){\rule[-0.200pt]{2.409pt}{0.400pt}}
\put(170.0,570.0){\rule[-0.200pt]{2.409pt}{0.400pt}}
\put(1429.0,570.0){\rule[-0.200pt]{2.409pt}{0.400pt}}
\put(170.0,570.0){\rule[-0.200pt]{2.409pt}{0.400pt}}
\put(1429.0,570.0){\rule[-0.200pt]{2.409pt}{0.400pt}}
\put(170.0,570.0){\rule[-0.200pt]{2.409pt}{0.400pt}}
\put(1429.0,570.0){\rule[-0.200pt]{2.409pt}{0.400pt}}
\put(170.0,570.0){\rule[-0.200pt]{2.409pt}{0.400pt}}
\put(1429.0,570.0){\rule[-0.200pt]{2.409pt}{0.400pt}}
\put(170.0,570.0){\rule[-0.200pt]{2.409pt}{0.400pt}}
\put(1429.0,570.0){\rule[-0.200pt]{2.409pt}{0.400pt}}
\put(170.0,570.0){\rule[-0.200pt]{2.409pt}{0.400pt}}
\put(1429.0,570.0){\rule[-0.200pt]{2.409pt}{0.400pt}}
\put(170.0,570.0){\rule[-0.200pt]{2.409pt}{0.400pt}}
\put(1429.0,570.0){\rule[-0.200pt]{2.409pt}{0.400pt}}
\put(170.0,571.0){\rule[-0.200pt]{2.409pt}{0.400pt}}
\put(1429.0,571.0){\rule[-0.200pt]{2.409pt}{0.400pt}}
\put(170.0,571.0){\rule[-0.200pt]{2.409pt}{0.400pt}}
\put(1429.0,571.0){\rule[-0.200pt]{2.409pt}{0.400pt}}
\put(170.0,571.0){\rule[-0.200pt]{2.409pt}{0.400pt}}
\put(1429.0,571.0){\rule[-0.200pt]{2.409pt}{0.400pt}}
\put(170.0,571.0){\rule[-0.200pt]{2.409pt}{0.400pt}}
\put(1429.0,571.0){\rule[-0.200pt]{2.409pt}{0.400pt}}
\put(170.0,571.0){\rule[-0.200pt]{2.409pt}{0.400pt}}
\put(1429.0,571.0){\rule[-0.200pt]{2.409pt}{0.400pt}}
\put(170.0,571.0){\rule[-0.200pt]{2.409pt}{0.400pt}}
\put(1429.0,571.0){\rule[-0.200pt]{2.409pt}{0.400pt}}
\put(170.0,571.0){\rule[-0.200pt]{2.409pt}{0.400pt}}
\put(1429.0,571.0){\rule[-0.200pt]{2.409pt}{0.400pt}}
\put(170.0,571.0){\rule[-0.200pt]{2.409pt}{0.400pt}}
\put(1429.0,571.0){\rule[-0.200pt]{2.409pt}{0.400pt}}
\put(170.0,571.0){\rule[-0.200pt]{2.409pt}{0.400pt}}
\put(1429.0,571.0){\rule[-0.200pt]{2.409pt}{0.400pt}}
\put(170.0,571.0){\rule[-0.200pt]{2.409pt}{0.400pt}}
\put(1429.0,571.0){\rule[-0.200pt]{2.409pt}{0.400pt}}
\put(170.0,571.0){\rule[-0.200pt]{2.409pt}{0.400pt}}
\put(1429.0,571.0){\rule[-0.200pt]{2.409pt}{0.400pt}}
\put(170.0,571.0){\rule[-0.200pt]{2.409pt}{0.400pt}}
\put(1429.0,571.0){\rule[-0.200pt]{2.409pt}{0.400pt}}
\put(170.0,572.0){\rule[-0.200pt]{2.409pt}{0.400pt}}
\put(1429.0,572.0){\rule[-0.200pt]{2.409pt}{0.400pt}}
\put(170.0,572.0){\rule[-0.200pt]{2.409pt}{0.400pt}}
\put(1429.0,572.0){\rule[-0.200pt]{2.409pt}{0.400pt}}
\put(170.0,572.0){\rule[-0.200pt]{2.409pt}{0.400pt}}
\put(1429.0,572.0){\rule[-0.200pt]{2.409pt}{0.400pt}}
\put(170.0,572.0){\rule[-0.200pt]{2.409pt}{0.400pt}}
\put(1429.0,572.0){\rule[-0.200pt]{2.409pt}{0.400pt}}
\put(170.0,572.0){\rule[-0.200pt]{2.409pt}{0.400pt}}
\put(1429.0,572.0){\rule[-0.200pt]{2.409pt}{0.400pt}}
\put(170.0,572.0){\rule[-0.200pt]{2.409pt}{0.400pt}}
\put(1429.0,572.0){\rule[-0.200pt]{2.409pt}{0.400pt}}
\put(170.0,572.0){\rule[-0.200pt]{2.409pt}{0.400pt}}
\put(1429.0,572.0){\rule[-0.200pt]{2.409pt}{0.400pt}}
\put(170.0,572.0){\rule[-0.200pt]{2.409pt}{0.400pt}}
\put(1429.0,572.0){\rule[-0.200pt]{2.409pt}{0.400pt}}
\put(170.0,572.0){\rule[-0.200pt]{2.409pt}{0.400pt}}
\put(1429.0,572.0){\rule[-0.200pt]{2.409pt}{0.400pt}}
\put(170.0,572.0){\rule[-0.200pt]{2.409pt}{0.400pt}}
\put(1429.0,572.0){\rule[-0.200pt]{2.409pt}{0.400pt}}
\put(170.0,572.0){\rule[-0.200pt]{2.409pt}{0.400pt}}
\put(1429.0,572.0){\rule[-0.200pt]{2.409pt}{0.400pt}}
\put(170.0,572.0){\rule[-0.200pt]{2.409pt}{0.400pt}}
\put(1429.0,572.0){\rule[-0.200pt]{2.409pt}{0.400pt}}
\put(170.0,572.0){\rule[-0.200pt]{2.409pt}{0.400pt}}
\put(1429.0,572.0){\rule[-0.200pt]{2.409pt}{0.400pt}}
\put(170.0,573.0){\rule[-0.200pt]{2.409pt}{0.400pt}}
\put(1429.0,573.0){\rule[-0.200pt]{2.409pt}{0.400pt}}
\put(170.0,573.0){\rule[-0.200pt]{2.409pt}{0.400pt}}
\put(1429.0,573.0){\rule[-0.200pt]{2.409pt}{0.400pt}}
\put(170.0,573.0){\rule[-0.200pt]{2.409pt}{0.400pt}}
\put(1429.0,573.0){\rule[-0.200pt]{2.409pt}{0.400pt}}
\put(170.0,573.0){\rule[-0.200pt]{2.409pt}{0.400pt}}
\put(1429.0,573.0){\rule[-0.200pt]{2.409pt}{0.400pt}}
\put(170.0,573.0){\rule[-0.200pt]{2.409pt}{0.400pt}}
\put(1429.0,573.0){\rule[-0.200pt]{2.409pt}{0.400pt}}
\put(170.0,573.0){\rule[-0.200pt]{2.409pt}{0.400pt}}
\put(1429.0,573.0){\rule[-0.200pt]{2.409pt}{0.400pt}}
\put(170.0,573.0){\rule[-0.200pt]{2.409pt}{0.400pt}}
\put(1429.0,573.0){\rule[-0.200pt]{2.409pt}{0.400pt}}
\put(170.0,573.0){\rule[-0.200pt]{2.409pt}{0.400pt}}
\put(1429.0,573.0){\rule[-0.200pt]{2.409pt}{0.400pt}}
\put(170.0,573.0){\rule[-0.200pt]{2.409pt}{0.400pt}}
\put(1429.0,573.0){\rule[-0.200pt]{2.409pt}{0.400pt}}
\put(170.0,573.0){\rule[-0.200pt]{2.409pt}{0.400pt}}
\put(1429.0,573.0){\rule[-0.200pt]{2.409pt}{0.400pt}}
\put(170.0,573.0){\rule[-0.200pt]{2.409pt}{0.400pt}}
\put(1429.0,573.0){\rule[-0.200pt]{2.409pt}{0.400pt}}
\put(170.0,573.0){\rule[-0.200pt]{2.409pt}{0.400pt}}
\put(1429.0,573.0){\rule[-0.200pt]{2.409pt}{0.400pt}}
\put(170.0,573.0){\rule[-0.200pt]{2.409pt}{0.400pt}}
\put(1429.0,573.0){\rule[-0.200pt]{2.409pt}{0.400pt}}
\put(170.0,574.0){\rule[-0.200pt]{2.409pt}{0.400pt}}
\put(1429.0,574.0){\rule[-0.200pt]{2.409pt}{0.400pt}}
\put(170.0,574.0){\rule[-0.200pt]{2.409pt}{0.400pt}}
\put(1429.0,574.0){\rule[-0.200pt]{2.409pt}{0.400pt}}
\put(170.0,574.0){\rule[-0.200pt]{2.409pt}{0.400pt}}
\put(1429.0,574.0){\rule[-0.200pt]{2.409pt}{0.400pt}}
\put(170.0,574.0){\rule[-0.200pt]{2.409pt}{0.400pt}}
\put(1429.0,574.0){\rule[-0.200pt]{2.409pt}{0.400pt}}
\put(170.0,574.0){\rule[-0.200pt]{2.409pt}{0.400pt}}
\put(1429.0,574.0){\rule[-0.200pt]{2.409pt}{0.400pt}}
\put(170.0,574.0){\rule[-0.200pt]{2.409pt}{0.400pt}}
\put(1429.0,574.0){\rule[-0.200pt]{2.409pt}{0.400pt}}
\put(170.0,574.0){\rule[-0.200pt]{2.409pt}{0.400pt}}
\put(1429.0,574.0){\rule[-0.200pt]{2.409pt}{0.400pt}}
\put(170.0,574.0){\rule[-0.200pt]{2.409pt}{0.400pt}}
\put(1429.0,574.0){\rule[-0.200pt]{2.409pt}{0.400pt}}
\put(170.0,574.0){\rule[-0.200pt]{2.409pt}{0.400pt}}
\put(1429.0,574.0){\rule[-0.200pt]{2.409pt}{0.400pt}}
\put(170.0,574.0){\rule[-0.200pt]{2.409pt}{0.400pt}}
\put(1429.0,574.0){\rule[-0.200pt]{2.409pt}{0.400pt}}
\put(170.0,574.0){\rule[-0.200pt]{2.409pt}{0.400pt}}
\put(1429.0,574.0){\rule[-0.200pt]{2.409pt}{0.400pt}}
\put(170.0,574.0){\rule[-0.200pt]{2.409pt}{0.400pt}}
\put(1429.0,574.0){\rule[-0.200pt]{2.409pt}{0.400pt}}
\put(170.0,574.0){\rule[-0.200pt]{2.409pt}{0.400pt}}
\put(1429.0,574.0){\rule[-0.200pt]{2.409pt}{0.400pt}}
\put(170.0,575.0){\rule[-0.200pt]{2.409pt}{0.400pt}}
\put(1429.0,575.0){\rule[-0.200pt]{2.409pt}{0.400pt}}
\put(170.0,575.0){\rule[-0.200pt]{2.409pt}{0.400pt}}
\put(1429.0,575.0){\rule[-0.200pt]{2.409pt}{0.400pt}}
\put(170.0,575.0){\rule[-0.200pt]{2.409pt}{0.400pt}}
\put(1429.0,575.0){\rule[-0.200pt]{2.409pt}{0.400pt}}
\put(170.0,575.0){\rule[-0.200pt]{2.409pt}{0.400pt}}
\put(1429.0,575.0){\rule[-0.200pt]{2.409pt}{0.400pt}}
\put(170.0,575.0){\rule[-0.200pt]{2.409pt}{0.400pt}}
\put(1429.0,575.0){\rule[-0.200pt]{2.409pt}{0.400pt}}
\put(170.0,575.0){\rule[-0.200pt]{2.409pt}{0.400pt}}
\put(1429.0,575.0){\rule[-0.200pt]{2.409pt}{0.400pt}}
\put(170.0,575.0){\rule[-0.200pt]{2.409pt}{0.400pt}}
\put(1429.0,575.0){\rule[-0.200pt]{2.409pt}{0.400pt}}
\put(170.0,575.0){\rule[-0.200pt]{2.409pt}{0.400pt}}
\put(1429.0,575.0){\rule[-0.200pt]{2.409pt}{0.400pt}}
\put(170.0,575.0){\rule[-0.200pt]{2.409pt}{0.400pt}}
\put(1429.0,575.0){\rule[-0.200pt]{2.409pt}{0.400pt}}
\put(170.0,575.0){\rule[-0.200pt]{2.409pt}{0.400pt}}
\put(1429.0,575.0){\rule[-0.200pt]{2.409pt}{0.400pt}}
\put(170.0,575.0){\rule[-0.200pt]{2.409pt}{0.400pt}}
\put(1429.0,575.0){\rule[-0.200pt]{2.409pt}{0.400pt}}
\put(170.0,575.0){\rule[-0.200pt]{2.409pt}{0.400pt}}
\put(1429.0,575.0){\rule[-0.200pt]{2.409pt}{0.400pt}}
\put(170.0,575.0){\rule[-0.200pt]{2.409pt}{0.400pt}}
\put(1429.0,575.0){\rule[-0.200pt]{2.409pt}{0.400pt}}
\put(170.0,575.0){\rule[-0.200pt]{2.409pt}{0.400pt}}
\put(1429.0,575.0){\rule[-0.200pt]{2.409pt}{0.400pt}}
\put(170.0,576.0){\rule[-0.200pt]{2.409pt}{0.400pt}}
\put(1429.0,576.0){\rule[-0.200pt]{2.409pt}{0.400pt}}
\put(170.0,576.0){\rule[-0.200pt]{2.409pt}{0.400pt}}
\put(1429.0,576.0){\rule[-0.200pt]{2.409pt}{0.400pt}}
\put(170.0,576.0){\rule[-0.200pt]{2.409pt}{0.400pt}}
\put(1429.0,576.0){\rule[-0.200pt]{2.409pt}{0.400pt}}
\put(170.0,576.0){\rule[-0.200pt]{2.409pt}{0.400pt}}
\put(1429.0,576.0){\rule[-0.200pt]{2.409pt}{0.400pt}}
\put(170.0,576.0){\rule[-0.200pt]{2.409pt}{0.400pt}}
\put(1429.0,576.0){\rule[-0.200pt]{2.409pt}{0.400pt}}
\put(170.0,576.0){\rule[-0.200pt]{2.409pt}{0.400pt}}
\put(1429.0,576.0){\rule[-0.200pt]{2.409pt}{0.400pt}}
\put(170.0,576.0){\rule[-0.200pt]{2.409pt}{0.400pt}}
\put(1429.0,576.0){\rule[-0.200pt]{2.409pt}{0.400pt}}
\put(170.0,576.0){\rule[-0.200pt]{2.409pt}{0.400pt}}
\put(1429.0,576.0){\rule[-0.200pt]{2.409pt}{0.400pt}}
\put(170.0,576.0){\rule[-0.200pt]{2.409pt}{0.400pt}}
\put(1429.0,576.0){\rule[-0.200pt]{2.409pt}{0.400pt}}
\put(170.0,576.0){\rule[-0.200pt]{2.409pt}{0.400pt}}
\put(1429.0,576.0){\rule[-0.200pt]{2.409pt}{0.400pt}}
\put(170.0,576.0){\rule[-0.200pt]{2.409pt}{0.400pt}}
\put(1429.0,576.0){\rule[-0.200pt]{2.409pt}{0.400pt}}
\put(170.0,576.0){\rule[-0.200pt]{2.409pt}{0.400pt}}
\put(1429.0,576.0){\rule[-0.200pt]{2.409pt}{0.400pt}}
\put(170.0,576.0){\rule[-0.200pt]{2.409pt}{0.400pt}}
\put(1429.0,576.0){\rule[-0.200pt]{2.409pt}{0.400pt}}
\put(170.0,576.0){\rule[-0.200pt]{2.409pt}{0.400pt}}
\put(1429.0,576.0){\rule[-0.200pt]{2.409pt}{0.400pt}}
\put(170.0,577.0){\rule[-0.200pt]{2.409pt}{0.400pt}}
\put(1429.0,577.0){\rule[-0.200pt]{2.409pt}{0.400pt}}
\put(170.0,577.0){\rule[-0.200pt]{2.409pt}{0.400pt}}
\put(1429.0,577.0){\rule[-0.200pt]{2.409pt}{0.400pt}}
\put(170.0,577.0){\rule[-0.200pt]{2.409pt}{0.400pt}}
\put(1429.0,577.0){\rule[-0.200pt]{2.409pt}{0.400pt}}
\put(170.0,577.0){\rule[-0.200pt]{2.409pt}{0.400pt}}
\put(1429.0,577.0){\rule[-0.200pt]{2.409pt}{0.400pt}}
\put(170.0,577.0){\rule[-0.200pt]{2.409pt}{0.400pt}}
\put(1429.0,577.0){\rule[-0.200pt]{2.409pt}{0.400pt}}
\put(170.0,577.0){\rule[-0.200pt]{2.409pt}{0.400pt}}
\put(1429.0,577.0){\rule[-0.200pt]{2.409pt}{0.400pt}}
\put(170.0,577.0){\rule[-0.200pt]{2.409pt}{0.400pt}}
\put(1429.0,577.0){\rule[-0.200pt]{2.409pt}{0.400pt}}
\put(170.0,577.0){\rule[-0.200pt]{2.409pt}{0.400pt}}
\put(1429.0,577.0){\rule[-0.200pt]{2.409pt}{0.400pt}}
\put(170.0,577.0){\rule[-0.200pt]{2.409pt}{0.400pt}}
\put(1429.0,577.0){\rule[-0.200pt]{2.409pt}{0.400pt}}
\put(170.0,577.0){\rule[-0.200pt]{2.409pt}{0.400pt}}
\put(1429.0,577.0){\rule[-0.200pt]{2.409pt}{0.400pt}}
\put(170.0,577.0){\rule[-0.200pt]{2.409pt}{0.400pt}}
\put(1429.0,577.0){\rule[-0.200pt]{2.409pt}{0.400pt}}
\put(170.0,577.0){\rule[-0.200pt]{2.409pt}{0.400pt}}
\put(1429.0,577.0){\rule[-0.200pt]{2.409pt}{0.400pt}}
\put(170.0,577.0){\rule[-0.200pt]{2.409pt}{0.400pt}}
\put(1429.0,577.0){\rule[-0.200pt]{2.409pt}{0.400pt}}
\put(170.0,577.0){\rule[-0.200pt]{2.409pt}{0.400pt}}
\put(1429.0,577.0){\rule[-0.200pt]{2.409pt}{0.400pt}}
\put(170.0,578.0){\rule[-0.200pt]{2.409pt}{0.400pt}}
\put(1429.0,578.0){\rule[-0.200pt]{2.409pt}{0.400pt}}
\put(170.0,578.0){\rule[-0.200pt]{2.409pt}{0.400pt}}
\put(1429.0,578.0){\rule[-0.200pt]{2.409pt}{0.400pt}}
\put(170.0,578.0){\rule[-0.200pt]{2.409pt}{0.400pt}}
\put(1429.0,578.0){\rule[-0.200pt]{2.409pt}{0.400pt}}
\put(170.0,578.0){\rule[-0.200pt]{2.409pt}{0.400pt}}
\put(1429.0,578.0){\rule[-0.200pt]{2.409pt}{0.400pt}}
\put(170.0,578.0){\rule[-0.200pt]{2.409pt}{0.400pt}}
\put(1429.0,578.0){\rule[-0.200pt]{2.409pt}{0.400pt}}
\put(170.0,578.0){\rule[-0.200pt]{2.409pt}{0.400pt}}
\put(1429.0,578.0){\rule[-0.200pt]{2.409pt}{0.400pt}}
\put(170.0,578.0){\rule[-0.200pt]{2.409pt}{0.400pt}}
\put(1429.0,578.0){\rule[-0.200pt]{2.409pt}{0.400pt}}
\put(170.0,578.0){\rule[-0.200pt]{2.409pt}{0.400pt}}
\put(1429.0,578.0){\rule[-0.200pt]{2.409pt}{0.400pt}}
\put(170.0,578.0){\rule[-0.200pt]{2.409pt}{0.400pt}}
\put(1429.0,578.0){\rule[-0.200pt]{2.409pt}{0.400pt}}
\put(170.0,578.0){\rule[-0.200pt]{2.409pt}{0.400pt}}
\put(1429.0,578.0){\rule[-0.200pt]{2.409pt}{0.400pt}}
\put(170.0,578.0){\rule[-0.200pt]{2.409pt}{0.400pt}}
\put(1429.0,578.0){\rule[-0.200pt]{2.409pt}{0.400pt}}
\put(170.0,578.0){\rule[-0.200pt]{2.409pt}{0.400pt}}
\put(1429.0,578.0){\rule[-0.200pt]{2.409pt}{0.400pt}}
\put(170.0,578.0){\rule[-0.200pt]{2.409pt}{0.400pt}}
\put(1429.0,578.0){\rule[-0.200pt]{2.409pt}{0.400pt}}
\put(170.0,578.0){\rule[-0.200pt]{2.409pt}{0.400pt}}
\put(1429.0,578.0){\rule[-0.200pt]{2.409pt}{0.400pt}}
\put(170.0,578.0){\rule[-0.200pt]{2.409pt}{0.400pt}}
\put(1429.0,578.0){\rule[-0.200pt]{2.409pt}{0.400pt}}
\put(170.0,579.0){\rule[-0.200pt]{2.409pt}{0.400pt}}
\put(1429.0,579.0){\rule[-0.200pt]{2.409pt}{0.400pt}}
\put(170.0,579.0){\rule[-0.200pt]{2.409pt}{0.400pt}}
\put(1429.0,579.0){\rule[-0.200pt]{2.409pt}{0.400pt}}
\put(170.0,579.0){\rule[-0.200pt]{2.409pt}{0.400pt}}
\put(1429.0,579.0){\rule[-0.200pt]{2.409pt}{0.400pt}}
\put(170.0,579.0){\rule[-0.200pt]{2.409pt}{0.400pt}}
\put(1429.0,579.0){\rule[-0.200pt]{2.409pt}{0.400pt}}
\put(170.0,579.0){\rule[-0.200pt]{2.409pt}{0.400pt}}
\put(1429.0,579.0){\rule[-0.200pt]{2.409pt}{0.400pt}}
\put(170.0,579.0){\rule[-0.200pt]{2.409pt}{0.400pt}}
\put(1429.0,579.0){\rule[-0.200pt]{2.409pt}{0.400pt}}
\put(170.0,579.0){\rule[-0.200pt]{2.409pt}{0.400pt}}
\put(1429.0,579.0){\rule[-0.200pt]{2.409pt}{0.400pt}}
\put(170.0,579.0){\rule[-0.200pt]{2.409pt}{0.400pt}}
\put(1429.0,579.0){\rule[-0.200pt]{2.409pt}{0.400pt}}
\put(170.0,579.0){\rule[-0.200pt]{2.409pt}{0.400pt}}
\put(1429.0,579.0){\rule[-0.200pt]{2.409pt}{0.400pt}}
\put(170.0,579.0){\rule[-0.200pt]{2.409pt}{0.400pt}}
\put(1429.0,579.0){\rule[-0.200pt]{2.409pt}{0.400pt}}
\put(170.0,579.0){\rule[-0.200pt]{2.409pt}{0.400pt}}
\put(1429.0,579.0){\rule[-0.200pt]{2.409pt}{0.400pt}}
\put(170.0,579.0){\rule[-0.200pt]{2.409pt}{0.400pt}}
\put(1429.0,579.0){\rule[-0.200pt]{2.409pt}{0.400pt}}
\put(170.0,579.0){\rule[-0.200pt]{2.409pt}{0.400pt}}
\put(1429.0,579.0){\rule[-0.200pt]{2.409pt}{0.400pt}}
\put(170.0,579.0){\rule[-0.200pt]{2.409pt}{0.400pt}}
\put(1429.0,579.0){\rule[-0.200pt]{2.409pt}{0.400pt}}
\put(170.0,579.0){\rule[-0.200pt]{2.409pt}{0.400pt}}
\put(1429.0,579.0){\rule[-0.200pt]{2.409pt}{0.400pt}}
\put(170.0,580.0){\rule[-0.200pt]{2.409pt}{0.400pt}}
\put(1429.0,580.0){\rule[-0.200pt]{2.409pt}{0.400pt}}
\put(170.0,580.0){\rule[-0.200pt]{2.409pt}{0.400pt}}
\put(1429.0,580.0){\rule[-0.200pt]{2.409pt}{0.400pt}}
\put(170.0,580.0){\rule[-0.200pt]{2.409pt}{0.400pt}}
\put(1429.0,580.0){\rule[-0.200pt]{2.409pt}{0.400pt}}
\put(170.0,580.0){\rule[-0.200pt]{2.409pt}{0.400pt}}
\put(1429.0,580.0){\rule[-0.200pt]{2.409pt}{0.400pt}}
\put(170.0,580.0){\rule[-0.200pt]{2.409pt}{0.400pt}}
\put(1429.0,580.0){\rule[-0.200pt]{2.409pt}{0.400pt}}
\put(170.0,580.0){\rule[-0.200pt]{2.409pt}{0.400pt}}
\put(1429.0,580.0){\rule[-0.200pt]{2.409pt}{0.400pt}}
\put(170.0,580.0){\rule[-0.200pt]{2.409pt}{0.400pt}}
\put(1429.0,580.0){\rule[-0.200pt]{2.409pt}{0.400pt}}
\put(170.0,580.0){\rule[-0.200pt]{2.409pt}{0.400pt}}
\put(1429.0,580.0){\rule[-0.200pt]{2.409pt}{0.400pt}}
\put(170.0,580.0){\rule[-0.200pt]{2.409pt}{0.400pt}}
\put(1429.0,580.0){\rule[-0.200pt]{2.409pt}{0.400pt}}
\put(170.0,580.0){\rule[-0.200pt]{2.409pt}{0.400pt}}
\put(1429.0,580.0){\rule[-0.200pt]{2.409pt}{0.400pt}}
\put(170.0,580.0){\rule[-0.200pt]{2.409pt}{0.400pt}}
\put(1429.0,580.0){\rule[-0.200pt]{2.409pt}{0.400pt}}
\put(170.0,580.0){\rule[-0.200pt]{2.409pt}{0.400pt}}
\put(1429.0,580.0){\rule[-0.200pt]{2.409pt}{0.400pt}}
\put(170.0,580.0){\rule[-0.200pt]{2.409pt}{0.400pt}}
\put(1429.0,580.0){\rule[-0.200pt]{2.409pt}{0.400pt}}
\put(170.0,580.0){\rule[-0.200pt]{2.409pt}{0.400pt}}
\put(1429.0,580.0){\rule[-0.200pt]{2.409pt}{0.400pt}}
\put(170.0,580.0){\rule[-0.200pt]{2.409pt}{0.400pt}}
\put(1429.0,580.0){\rule[-0.200pt]{2.409pt}{0.400pt}}
\put(170.0,580.0){\rule[-0.200pt]{2.409pt}{0.400pt}}
\put(1429.0,580.0){\rule[-0.200pt]{2.409pt}{0.400pt}}
\put(170.0,581.0){\rule[-0.200pt]{2.409pt}{0.400pt}}
\put(1429.0,581.0){\rule[-0.200pt]{2.409pt}{0.400pt}}
\put(170.0,581.0){\rule[-0.200pt]{2.409pt}{0.400pt}}
\put(1429.0,581.0){\rule[-0.200pt]{2.409pt}{0.400pt}}
\put(170.0,581.0){\rule[-0.200pt]{2.409pt}{0.400pt}}
\put(1429.0,581.0){\rule[-0.200pt]{2.409pt}{0.400pt}}
\put(170.0,581.0){\rule[-0.200pt]{2.409pt}{0.400pt}}
\put(1429.0,581.0){\rule[-0.200pt]{2.409pt}{0.400pt}}
\put(170.0,581.0){\rule[-0.200pt]{2.409pt}{0.400pt}}
\put(1429.0,581.0){\rule[-0.200pt]{2.409pt}{0.400pt}}
\put(170.0,581.0){\rule[-0.200pt]{2.409pt}{0.400pt}}
\put(1429.0,581.0){\rule[-0.200pt]{2.409pt}{0.400pt}}
\put(170.0,581.0){\rule[-0.200pt]{2.409pt}{0.400pt}}
\put(1429.0,581.0){\rule[-0.200pt]{2.409pt}{0.400pt}}
\put(170.0,581.0){\rule[-0.200pt]{2.409pt}{0.400pt}}
\put(1429.0,581.0){\rule[-0.200pt]{2.409pt}{0.400pt}}
\put(170.0,581.0){\rule[-0.200pt]{2.409pt}{0.400pt}}
\put(1429.0,581.0){\rule[-0.200pt]{2.409pt}{0.400pt}}
\put(170.0,581.0){\rule[-0.200pt]{2.409pt}{0.400pt}}
\put(1429.0,581.0){\rule[-0.200pt]{2.409pt}{0.400pt}}
\put(170.0,581.0){\rule[-0.200pt]{2.409pt}{0.400pt}}
\put(1429.0,581.0){\rule[-0.200pt]{2.409pt}{0.400pt}}
\put(170.0,581.0){\rule[-0.200pt]{2.409pt}{0.400pt}}
\put(1429.0,581.0){\rule[-0.200pt]{2.409pt}{0.400pt}}
\put(170.0,581.0){\rule[-0.200pt]{2.409pt}{0.400pt}}
\put(1429.0,581.0){\rule[-0.200pt]{2.409pt}{0.400pt}}
\put(170.0,581.0){\rule[-0.200pt]{2.409pt}{0.400pt}}
\put(1429.0,581.0){\rule[-0.200pt]{2.409pt}{0.400pt}}
\put(170.0,581.0){\rule[-0.200pt]{2.409pt}{0.400pt}}
\put(1429.0,581.0){\rule[-0.200pt]{2.409pt}{0.400pt}}
\put(170.0,581.0){\rule[-0.200pt]{2.409pt}{0.400pt}}
\put(1429.0,581.0){\rule[-0.200pt]{2.409pt}{0.400pt}}
\put(170.0,582.0){\rule[-0.200pt]{2.409pt}{0.400pt}}
\put(1429.0,582.0){\rule[-0.200pt]{2.409pt}{0.400pt}}
\put(170.0,582.0){\rule[-0.200pt]{2.409pt}{0.400pt}}
\put(1429.0,582.0){\rule[-0.200pt]{2.409pt}{0.400pt}}
\put(170.0,582.0){\rule[-0.200pt]{2.409pt}{0.400pt}}
\put(1429.0,582.0){\rule[-0.200pt]{2.409pt}{0.400pt}}
\put(170.0,582.0){\rule[-0.200pt]{2.409pt}{0.400pt}}
\put(1429.0,582.0){\rule[-0.200pt]{2.409pt}{0.400pt}}
\put(170.0,582.0){\rule[-0.200pt]{2.409pt}{0.400pt}}
\put(1429.0,582.0){\rule[-0.200pt]{2.409pt}{0.400pt}}
\put(170.0,582.0){\rule[-0.200pt]{2.409pt}{0.400pt}}
\put(1429.0,582.0){\rule[-0.200pt]{2.409pt}{0.400pt}}
\put(170.0,582.0){\rule[-0.200pt]{2.409pt}{0.400pt}}
\put(1429.0,582.0){\rule[-0.200pt]{2.409pt}{0.400pt}}
\put(170.0,582.0){\rule[-0.200pt]{2.409pt}{0.400pt}}
\put(1429.0,582.0){\rule[-0.200pt]{2.409pt}{0.400pt}}
\put(170.0,582.0){\rule[-0.200pt]{2.409pt}{0.400pt}}
\put(1429.0,582.0){\rule[-0.200pt]{2.409pt}{0.400pt}}
\put(170.0,582.0){\rule[-0.200pt]{2.409pt}{0.400pt}}
\put(1429.0,582.0){\rule[-0.200pt]{2.409pt}{0.400pt}}
\put(170.0,582.0){\rule[-0.200pt]{2.409pt}{0.400pt}}
\put(1429.0,582.0){\rule[-0.200pt]{2.409pt}{0.400pt}}
\put(170.0,582.0){\rule[-0.200pt]{2.409pt}{0.400pt}}
\put(1429.0,582.0){\rule[-0.200pt]{2.409pt}{0.400pt}}
\put(170.0,582.0){\rule[-0.200pt]{2.409pt}{0.400pt}}
\put(1429.0,582.0){\rule[-0.200pt]{2.409pt}{0.400pt}}
\put(170.0,582.0){\rule[-0.200pt]{2.409pt}{0.400pt}}
\put(1429.0,582.0){\rule[-0.200pt]{2.409pt}{0.400pt}}
\put(170.0,582.0){\rule[-0.200pt]{2.409pt}{0.400pt}}
\put(1429.0,582.0){\rule[-0.200pt]{2.409pt}{0.400pt}}
\put(170.0,582.0){\rule[-0.200pt]{2.409pt}{0.400pt}}
\put(1429.0,582.0){\rule[-0.200pt]{2.409pt}{0.400pt}}
\put(170.0,582.0){\rule[-0.200pt]{2.409pt}{0.400pt}}
\put(1429.0,582.0){\rule[-0.200pt]{2.409pt}{0.400pt}}
\put(170.0,583.0){\rule[-0.200pt]{2.409pt}{0.400pt}}
\put(1429.0,583.0){\rule[-0.200pt]{2.409pt}{0.400pt}}
\put(170.0,583.0){\rule[-0.200pt]{2.409pt}{0.400pt}}
\put(1429.0,583.0){\rule[-0.200pt]{2.409pt}{0.400pt}}
\put(170.0,583.0){\rule[-0.200pt]{2.409pt}{0.400pt}}
\put(1429.0,583.0){\rule[-0.200pt]{2.409pt}{0.400pt}}
\put(170.0,583.0){\rule[-0.200pt]{2.409pt}{0.400pt}}
\put(1429.0,583.0){\rule[-0.200pt]{2.409pt}{0.400pt}}
\put(170.0,583.0){\rule[-0.200pt]{2.409pt}{0.400pt}}
\put(1429.0,583.0){\rule[-0.200pt]{2.409pt}{0.400pt}}
\put(170.0,583.0){\rule[-0.200pt]{2.409pt}{0.400pt}}
\put(1429.0,583.0){\rule[-0.200pt]{2.409pt}{0.400pt}}
\put(170.0,583.0){\rule[-0.200pt]{2.409pt}{0.400pt}}
\put(1429.0,583.0){\rule[-0.200pt]{2.409pt}{0.400pt}}
\put(170.0,583.0){\rule[-0.200pt]{2.409pt}{0.400pt}}
\put(1429.0,583.0){\rule[-0.200pt]{2.409pt}{0.400pt}}
\put(170.0,583.0){\rule[-0.200pt]{2.409pt}{0.400pt}}
\put(1429.0,583.0){\rule[-0.200pt]{2.409pt}{0.400pt}}
\put(170.0,583.0){\rule[-0.200pt]{2.409pt}{0.400pt}}
\put(1429.0,583.0){\rule[-0.200pt]{2.409pt}{0.400pt}}
\put(170.0,583.0){\rule[-0.200pt]{2.409pt}{0.400pt}}
\put(1429.0,583.0){\rule[-0.200pt]{2.409pt}{0.400pt}}
\put(170.0,583.0){\rule[-0.200pt]{2.409pt}{0.400pt}}
\put(1429.0,583.0){\rule[-0.200pt]{2.409pt}{0.400pt}}
\put(170.0,583.0){\rule[-0.200pt]{2.409pt}{0.400pt}}
\put(1429.0,583.0){\rule[-0.200pt]{2.409pt}{0.400pt}}
\put(170.0,583.0){\rule[-0.200pt]{2.409pt}{0.400pt}}
\put(1429.0,583.0){\rule[-0.200pt]{2.409pt}{0.400pt}}
\put(170.0,583.0){\rule[-0.200pt]{2.409pt}{0.400pt}}
\put(1429.0,583.0){\rule[-0.200pt]{2.409pt}{0.400pt}}
\put(170.0,583.0){\rule[-0.200pt]{2.409pt}{0.400pt}}
\put(1429.0,583.0){\rule[-0.200pt]{2.409pt}{0.400pt}}
\put(170.0,584.0){\rule[-0.200pt]{2.409pt}{0.400pt}}
\put(1429.0,584.0){\rule[-0.200pt]{2.409pt}{0.400pt}}
\put(170.0,584.0){\rule[-0.200pt]{2.409pt}{0.400pt}}
\put(1429.0,584.0){\rule[-0.200pt]{2.409pt}{0.400pt}}
\put(170.0,584.0){\rule[-0.200pt]{2.409pt}{0.400pt}}
\put(1429.0,584.0){\rule[-0.200pt]{2.409pt}{0.400pt}}
\put(170.0,584.0){\rule[-0.200pt]{2.409pt}{0.400pt}}
\put(1429.0,584.0){\rule[-0.200pt]{2.409pt}{0.400pt}}
\put(170.0,584.0){\rule[-0.200pt]{2.409pt}{0.400pt}}
\put(1429.0,584.0){\rule[-0.200pt]{2.409pt}{0.400pt}}
\put(170.0,584.0){\rule[-0.200pt]{2.409pt}{0.400pt}}
\put(1429.0,584.0){\rule[-0.200pt]{2.409pt}{0.400pt}}
\put(170.0,584.0){\rule[-0.200pt]{2.409pt}{0.400pt}}
\put(1429.0,584.0){\rule[-0.200pt]{2.409pt}{0.400pt}}
\put(170.0,584.0){\rule[-0.200pt]{2.409pt}{0.400pt}}
\put(1429.0,584.0){\rule[-0.200pt]{2.409pt}{0.400pt}}
\put(170.0,584.0){\rule[-0.200pt]{2.409pt}{0.400pt}}
\put(1429.0,584.0){\rule[-0.200pt]{2.409pt}{0.400pt}}
\put(170.0,584.0){\rule[-0.200pt]{2.409pt}{0.400pt}}
\put(1429.0,584.0){\rule[-0.200pt]{2.409pt}{0.400pt}}
\put(170.0,584.0){\rule[-0.200pt]{2.409pt}{0.400pt}}
\put(1429.0,584.0){\rule[-0.200pt]{2.409pt}{0.400pt}}
\put(170.0,584.0){\rule[-0.200pt]{2.409pt}{0.400pt}}
\put(1429.0,584.0){\rule[-0.200pt]{2.409pt}{0.400pt}}
\put(170.0,584.0){\rule[-0.200pt]{2.409pt}{0.400pt}}
\put(1429.0,584.0){\rule[-0.200pt]{2.409pt}{0.400pt}}
\put(170.0,584.0){\rule[-0.200pt]{2.409pt}{0.400pt}}
\put(1429.0,584.0){\rule[-0.200pt]{2.409pt}{0.400pt}}
\put(170.0,584.0){\rule[-0.200pt]{2.409pt}{0.400pt}}
\put(1429.0,584.0){\rule[-0.200pt]{2.409pt}{0.400pt}}
\put(170.0,584.0){\rule[-0.200pt]{2.409pt}{0.400pt}}
\put(1429.0,584.0){\rule[-0.200pt]{2.409pt}{0.400pt}}
\put(170.0,584.0){\rule[-0.200pt]{2.409pt}{0.400pt}}
\put(1429.0,584.0){\rule[-0.200pt]{2.409pt}{0.400pt}}
\put(170.0,584.0){\rule[-0.200pt]{2.409pt}{0.400pt}}
\put(1429.0,584.0){\rule[-0.200pt]{2.409pt}{0.400pt}}
\put(170.0,585.0){\rule[-0.200pt]{2.409pt}{0.400pt}}
\put(1429.0,585.0){\rule[-0.200pt]{2.409pt}{0.400pt}}
\put(170.0,585.0){\rule[-0.200pt]{2.409pt}{0.400pt}}
\put(1429.0,585.0){\rule[-0.200pt]{2.409pt}{0.400pt}}
\put(170.0,585.0){\rule[-0.200pt]{2.409pt}{0.400pt}}
\put(1429.0,585.0){\rule[-0.200pt]{2.409pt}{0.400pt}}
\put(170.0,585.0){\rule[-0.200pt]{2.409pt}{0.400pt}}
\put(1429.0,585.0){\rule[-0.200pt]{2.409pt}{0.400pt}}
\put(170.0,585.0){\rule[-0.200pt]{2.409pt}{0.400pt}}
\put(1429.0,585.0){\rule[-0.200pt]{2.409pt}{0.400pt}}
\put(170.0,585.0){\rule[-0.200pt]{2.409pt}{0.400pt}}
\put(1429.0,585.0){\rule[-0.200pt]{2.409pt}{0.400pt}}
\put(170.0,585.0){\rule[-0.200pt]{2.409pt}{0.400pt}}
\put(1429.0,585.0){\rule[-0.200pt]{2.409pt}{0.400pt}}
\put(170.0,585.0){\rule[-0.200pt]{2.409pt}{0.400pt}}
\put(1429.0,585.0){\rule[-0.200pt]{2.409pt}{0.400pt}}
\put(170.0,585.0){\rule[-0.200pt]{2.409pt}{0.400pt}}
\put(1429.0,585.0){\rule[-0.200pt]{2.409pt}{0.400pt}}
\put(170.0,585.0){\rule[-0.200pt]{2.409pt}{0.400pt}}
\put(1429.0,585.0){\rule[-0.200pt]{2.409pt}{0.400pt}}
\put(170.0,585.0){\rule[-0.200pt]{2.409pt}{0.400pt}}
\put(1429.0,585.0){\rule[-0.200pt]{2.409pt}{0.400pt}}
\put(170.0,585.0){\rule[-0.200pt]{2.409pt}{0.400pt}}
\put(1429.0,585.0){\rule[-0.200pt]{2.409pt}{0.400pt}}
\put(170.0,585.0){\rule[-0.200pt]{2.409pt}{0.400pt}}
\put(1429.0,585.0){\rule[-0.200pt]{2.409pt}{0.400pt}}
\put(170.0,585.0){\rule[-0.200pt]{2.409pt}{0.400pt}}
\put(1429.0,585.0){\rule[-0.200pt]{2.409pt}{0.400pt}}
\put(170.0,585.0){\rule[-0.200pt]{2.409pt}{0.400pt}}
\put(1429.0,585.0){\rule[-0.200pt]{2.409pt}{0.400pt}}
\put(170.0,585.0){\rule[-0.200pt]{2.409pt}{0.400pt}}
\put(1429.0,585.0){\rule[-0.200pt]{2.409pt}{0.400pt}}
\put(170.0,585.0){\rule[-0.200pt]{2.409pt}{0.400pt}}
\put(1429.0,585.0){\rule[-0.200pt]{2.409pt}{0.400pt}}
\put(170.0,585.0){\rule[-0.200pt]{2.409pt}{0.400pt}}
\put(1429.0,585.0){\rule[-0.200pt]{2.409pt}{0.400pt}}
\put(170.0,586.0){\rule[-0.200pt]{2.409pt}{0.400pt}}
\put(1429.0,586.0){\rule[-0.200pt]{2.409pt}{0.400pt}}
\put(170.0,586.0){\rule[-0.200pt]{2.409pt}{0.400pt}}
\put(1429.0,586.0){\rule[-0.200pt]{2.409pt}{0.400pt}}
\put(170.0,586.0){\rule[-0.200pt]{2.409pt}{0.400pt}}
\put(1429.0,586.0){\rule[-0.200pt]{2.409pt}{0.400pt}}
\put(170.0,586.0){\rule[-0.200pt]{2.409pt}{0.400pt}}
\put(1429.0,586.0){\rule[-0.200pt]{2.409pt}{0.400pt}}
\put(170.0,586.0){\rule[-0.200pt]{2.409pt}{0.400pt}}
\put(1429.0,586.0){\rule[-0.200pt]{2.409pt}{0.400pt}}
\put(170.0,586.0){\rule[-0.200pt]{2.409pt}{0.400pt}}
\put(1429.0,586.0){\rule[-0.200pt]{2.409pt}{0.400pt}}
\put(170.0,586.0){\rule[-0.200pt]{2.409pt}{0.400pt}}
\put(1429.0,586.0){\rule[-0.200pt]{2.409pt}{0.400pt}}
\put(170.0,586.0){\rule[-0.200pt]{2.409pt}{0.400pt}}
\put(1429.0,586.0){\rule[-0.200pt]{2.409pt}{0.400pt}}
\put(170.0,586.0){\rule[-0.200pt]{2.409pt}{0.400pt}}
\put(1429.0,586.0){\rule[-0.200pt]{2.409pt}{0.400pt}}
\put(170.0,586.0){\rule[-0.200pt]{2.409pt}{0.400pt}}
\put(1429.0,586.0){\rule[-0.200pt]{2.409pt}{0.400pt}}
\put(170.0,586.0){\rule[-0.200pt]{2.409pt}{0.400pt}}
\put(1429.0,586.0){\rule[-0.200pt]{2.409pt}{0.400pt}}
\put(170.0,586.0){\rule[-0.200pt]{2.409pt}{0.400pt}}
\put(1429.0,586.0){\rule[-0.200pt]{2.409pt}{0.400pt}}
\put(170.0,586.0){\rule[-0.200pt]{2.409pt}{0.400pt}}
\put(1429.0,586.0){\rule[-0.200pt]{2.409pt}{0.400pt}}
\put(170.0,586.0){\rule[-0.200pt]{2.409pt}{0.400pt}}
\put(1429.0,586.0){\rule[-0.200pt]{2.409pt}{0.400pt}}
\put(170.0,586.0){\rule[-0.200pt]{2.409pt}{0.400pt}}
\put(1429.0,586.0){\rule[-0.200pt]{2.409pt}{0.400pt}}
\put(170.0,586.0){\rule[-0.200pt]{2.409pt}{0.400pt}}
\put(1429.0,586.0){\rule[-0.200pt]{2.409pt}{0.400pt}}
\put(170.0,586.0){\rule[-0.200pt]{2.409pt}{0.400pt}}
\put(1429.0,586.0){\rule[-0.200pt]{2.409pt}{0.400pt}}
\put(170.0,586.0){\rule[-0.200pt]{2.409pt}{0.400pt}}
\put(1429.0,586.0){\rule[-0.200pt]{2.409pt}{0.400pt}}
\put(170.0,587.0){\rule[-0.200pt]{2.409pt}{0.400pt}}
\put(1429.0,587.0){\rule[-0.200pt]{2.409pt}{0.400pt}}
\put(170.0,587.0){\rule[-0.200pt]{2.409pt}{0.400pt}}
\put(1429.0,587.0){\rule[-0.200pt]{2.409pt}{0.400pt}}
\put(170.0,587.0){\rule[-0.200pt]{2.409pt}{0.400pt}}
\put(1429.0,587.0){\rule[-0.200pt]{2.409pt}{0.400pt}}
\put(170.0,587.0){\rule[-0.200pt]{2.409pt}{0.400pt}}
\put(1429.0,587.0){\rule[-0.200pt]{2.409pt}{0.400pt}}
\put(170.0,587.0){\rule[-0.200pt]{2.409pt}{0.400pt}}
\put(1429.0,587.0){\rule[-0.200pt]{2.409pt}{0.400pt}}
\put(170.0,587.0){\rule[-0.200pt]{2.409pt}{0.400pt}}
\put(1429.0,587.0){\rule[-0.200pt]{2.409pt}{0.400pt}}
\put(170.0,587.0){\rule[-0.200pt]{2.409pt}{0.400pt}}
\put(1429.0,587.0){\rule[-0.200pt]{2.409pt}{0.400pt}}
\put(170.0,587.0){\rule[-0.200pt]{2.409pt}{0.400pt}}
\put(1429.0,587.0){\rule[-0.200pt]{2.409pt}{0.400pt}}
\put(170.0,587.0){\rule[-0.200pt]{2.409pt}{0.400pt}}
\put(1429.0,587.0){\rule[-0.200pt]{2.409pt}{0.400pt}}
\put(170.0,587.0){\rule[-0.200pt]{2.409pt}{0.400pt}}
\put(1429.0,587.0){\rule[-0.200pt]{2.409pt}{0.400pt}}
\put(170.0,587.0){\rule[-0.200pt]{2.409pt}{0.400pt}}
\put(1429.0,587.0){\rule[-0.200pt]{2.409pt}{0.400pt}}
\put(170.0,587.0){\rule[-0.200pt]{2.409pt}{0.400pt}}
\put(1429.0,587.0){\rule[-0.200pt]{2.409pt}{0.400pt}}
\put(170.0,587.0){\rule[-0.200pt]{2.409pt}{0.400pt}}
\put(1429.0,587.0){\rule[-0.200pt]{2.409pt}{0.400pt}}
\put(170.0,587.0){\rule[-0.200pt]{2.409pt}{0.400pt}}
\put(1429.0,587.0){\rule[-0.200pt]{2.409pt}{0.400pt}}
\put(170.0,587.0){\rule[-0.200pt]{2.409pt}{0.400pt}}
\put(1429.0,587.0){\rule[-0.200pt]{2.409pt}{0.400pt}}
\put(170.0,587.0){\rule[-0.200pt]{2.409pt}{0.400pt}}
\put(1429.0,587.0){\rule[-0.200pt]{2.409pt}{0.400pt}}
\put(170.0,587.0){\rule[-0.200pt]{2.409pt}{0.400pt}}
\put(1429.0,587.0){\rule[-0.200pt]{2.409pt}{0.400pt}}
\put(170.0,587.0){\rule[-0.200pt]{2.409pt}{0.400pt}}
\put(1429.0,587.0){\rule[-0.200pt]{2.409pt}{0.400pt}}
\put(170.0,587.0){\rule[-0.200pt]{2.409pt}{0.400pt}}
\put(1429.0,587.0){\rule[-0.200pt]{2.409pt}{0.400pt}}
\put(170.0,588.0){\rule[-0.200pt]{2.409pt}{0.400pt}}
\put(1429.0,588.0){\rule[-0.200pt]{2.409pt}{0.400pt}}
\put(170.0,588.0){\rule[-0.200pt]{2.409pt}{0.400pt}}
\put(1429.0,588.0){\rule[-0.200pt]{2.409pt}{0.400pt}}
\put(170.0,588.0){\rule[-0.200pt]{2.409pt}{0.400pt}}
\put(1429.0,588.0){\rule[-0.200pt]{2.409pt}{0.400pt}}
\put(170.0,588.0){\rule[-0.200pt]{2.409pt}{0.400pt}}
\put(1429.0,588.0){\rule[-0.200pt]{2.409pt}{0.400pt}}
\put(170.0,588.0){\rule[-0.200pt]{2.409pt}{0.400pt}}
\put(1429.0,588.0){\rule[-0.200pt]{2.409pt}{0.400pt}}
\put(170.0,588.0){\rule[-0.200pt]{2.409pt}{0.400pt}}
\put(1429.0,588.0){\rule[-0.200pt]{2.409pt}{0.400pt}}
\put(170.0,588.0){\rule[-0.200pt]{2.409pt}{0.400pt}}
\put(1429.0,588.0){\rule[-0.200pt]{2.409pt}{0.400pt}}
\put(170.0,588.0){\rule[-0.200pt]{2.409pt}{0.400pt}}
\put(1429.0,588.0){\rule[-0.200pt]{2.409pt}{0.400pt}}
\put(170.0,588.0){\rule[-0.200pt]{2.409pt}{0.400pt}}
\put(1429.0,588.0){\rule[-0.200pt]{2.409pt}{0.400pt}}
\put(170.0,588.0){\rule[-0.200pt]{2.409pt}{0.400pt}}
\put(1429.0,588.0){\rule[-0.200pt]{2.409pt}{0.400pt}}
\put(170.0,588.0){\rule[-0.200pt]{2.409pt}{0.400pt}}
\put(1429.0,588.0){\rule[-0.200pt]{2.409pt}{0.400pt}}
\put(170.0,588.0){\rule[-0.200pt]{2.409pt}{0.400pt}}
\put(1429.0,588.0){\rule[-0.200pt]{2.409pt}{0.400pt}}
\put(170.0,588.0){\rule[-0.200pt]{2.409pt}{0.400pt}}
\put(1429.0,588.0){\rule[-0.200pt]{2.409pt}{0.400pt}}
\put(170.0,588.0){\rule[-0.200pt]{2.409pt}{0.400pt}}
\put(1429.0,588.0){\rule[-0.200pt]{2.409pt}{0.400pt}}
\put(170.0,588.0){\rule[-0.200pt]{2.409pt}{0.400pt}}
\put(1429.0,588.0){\rule[-0.200pt]{2.409pt}{0.400pt}}
\put(170.0,588.0){\rule[-0.200pt]{2.409pt}{0.400pt}}
\put(1429.0,588.0){\rule[-0.200pt]{2.409pt}{0.400pt}}
\put(170.0,588.0){\rule[-0.200pt]{2.409pt}{0.400pt}}
\put(1429.0,588.0){\rule[-0.200pt]{2.409pt}{0.400pt}}
\put(170.0,588.0){\rule[-0.200pt]{2.409pt}{0.400pt}}
\put(1429.0,588.0){\rule[-0.200pt]{2.409pt}{0.400pt}}
\put(170.0,588.0){\rule[-0.200pt]{2.409pt}{0.400pt}}
\put(1429.0,588.0){\rule[-0.200pt]{2.409pt}{0.400pt}}
\put(170.0,589.0){\rule[-0.200pt]{2.409pt}{0.400pt}}
\put(1429.0,589.0){\rule[-0.200pt]{2.409pt}{0.400pt}}
\put(170.0,589.0){\rule[-0.200pt]{2.409pt}{0.400pt}}
\put(1429.0,589.0){\rule[-0.200pt]{2.409pt}{0.400pt}}
\put(170.0,589.0){\rule[-0.200pt]{2.409pt}{0.400pt}}
\put(1429.0,589.0){\rule[-0.200pt]{2.409pt}{0.400pt}}
\put(170.0,589.0){\rule[-0.200pt]{2.409pt}{0.400pt}}
\put(1429.0,589.0){\rule[-0.200pt]{2.409pt}{0.400pt}}
\put(170.0,589.0){\rule[-0.200pt]{2.409pt}{0.400pt}}
\put(1429.0,589.0){\rule[-0.200pt]{2.409pt}{0.400pt}}
\put(170.0,589.0){\rule[-0.200pt]{2.409pt}{0.400pt}}
\put(1429.0,589.0){\rule[-0.200pt]{2.409pt}{0.400pt}}
\put(170.0,589.0){\rule[-0.200pt]{2.409pt}{0.400pt}}
\put(1429.0,589.0){\rule[-0.200pt]{2.409pt}{0.400pt}}
\put(170.0,589.0){\rule[-0.200pt]{2.409pt}{0.400pt}}
\put(1429.0,589.0){\rule[-0.200pt]{2.409pt}{0.400pt}}
\put(170.0,589.0){\rule[-0.200pt]{2.409pt}{0.400pt}}
\put(1429.0,589.0){\rule[-0.200pt]{2.409pt}{0.400pt}}
\put(170.0,589.0){\rule[-0.200pt]{2.409pt}{0.400pt}}
\put(1429.0,589.0){\rule[-0.200pt]{2.409pt}{0.400pt}}
\put(170.0,589.0){\rule[-0.200pt]{2.409pt}{0.400pt}}
\put(1429.0,589.0){\rule[-0.200pt]{2.409pt}{0.400pt}}
\put(170.0,589.0){\rule[-0.200pt]{2.409pt}{0.400pt}}
\put(1429.0,589.0){\rule[-0.200pt]{2.409pt}{0.400pt}}
\put(170.0,589.0){\rule[-0.200pt]{2.409pt}{0.400pt}}
\put(1429.0,589.0){\rule[-0.200pt]{2.409pt}{0.400pt}}
\put(170.0,589.0){\rule[-0.200pt]{2.409pt}{0.400pt}}
\put(1429.0,589.0){\rule[-0.200pt]{2.409pt}{0.400pt}}
\put(170.0,589.0){\rule[-0.200pt]{2.409pt}{0.400pt}}
\put(1429.0,589.0){\rule[-0.200pt]{2.409pt}{0.400pt}}
\put(170.0,589.0){\rule[-0.200pt]{2.409pt}{0.400pt}}
\put(1429.0,589.0){\rule[-0.200pt]{2.409pt}{0.400pt}}
\put(170.0,589.0){\rule[-0.200pt]{2.409pt}{0.400pt}}
\put(1429.0,589.0){\rule[-0.200pt]{2.409pt}{0.400pt}}
\put(170.0,589.0){\rule[-0.200pt]{2.409pt}{0.400pt}}
\put(1429.0,589.0){\rule[-0.200pt]{2.409pt}{0.400pt}}
\put(170.0,589.0){\rule[-0.200pt]{2.409pt}{0.400pt}}
\put(1429.0,589.0){\rule[-0.200pt]{2.409pt}{0.400pt}}
\put(170.0,589.0){\rule[-0.200pt]{2.409pt}{0.400pt}}
\put(1429.0,589.0){\rule[-0.200pt]{2.409pt}{0.400pt}}
\put(170.0,590.0){\rule[-0.200pt]{2.409pt}{0.400pt}}
\put(1429.0,590.0){\rule[-0.200pt]{2.409pt}{0.400pt}}
\put(170.0,590.0){\rule[-0.200pt]{2.409pt}{0.400pt}}
\put(1429.0,590.0){\rule[-0.200pt]{2.409pt}{0.400pt}}
\put(170.0,590.0){\rule[-0.200pt]{2.409pt}{0.400pt}}
\put(1429.0,590.0){\rule[-0.200pt]{2.409pt}{0.400pt}}
\put(170.0,590.0){\rule[-0.200pt]{2.409pt}{0.400pt}}
\put(1429.0,590.0){\rule[-0.200pt]{2.409pt}{0.400pt}}
\put(170.0,590.0){\rule[-0.200pt]{2.409pt}{0.400pt}}
\put(1429.0,590.0){\rule[-0.200pt]{2.409pt}{0.400pt}}
\put(170.0,590.0){\rule[-0.200pt]{2.409pt}{0.400pt}}
\put(1429.0,590.0){\rule[-0.200pt]{2.409pt}{0.400pt}}
\put(170.0,590.0){\rule[-0.200pt]{2.409pt}{0.400pt}}
\put(1429.0,590.0){\rule[-0.200pt]{2.409pt}{0.400pt}}
\put(170.0,590.0){\rule[-0.200pt]{2.409pt}{0.400pt}}
\put(1429.0,590.0){\rule[-0.200pt]{2.409pt}{0.400pt}}
\put(170.0,590.0){\rule[-0.200pt]{2.409pt}{0.400pt}}
\put(1429.0,590.0){\rule[-0.200pt]{2.409pt}{0.400pt}}
\put(170.0,590.0){\rule[-0.200pt]{2.409pt}{0.400pt}}
\put(1429.0,590.0){\rule[-0.200pt]{2.409pt}{0.400pt}}
\put(170.0,590.0){\rule[-0.200pt]{2.409pt}{0.400pt}}
\put(1429.0,590.0){\rule[-0.200pt]{2.409pt}{0.400pt}}
\put(170.0,590.0){\rule[-0.200pt]{2.409pt}{0.400pt}}
\put(1429.0,590.0){\rule[-0.200pt]{2.409pt}{0.400pt}}
\put(170.0,590.0){\rule[-0.200pt]{2.409pt}{0.400pt}}
\put(1429.0,590.0){\rule[-0.200pt]{2.409pt}{0.400pt}}
\put(170.0,590.0){\rule[-0.200pt]{2.409pt}{0.400pt}}
\put(1429.0,590.0){\rule[-0.200pt]{2.409pt}{0.400pt}}
\put(170.0,590.0){\rule[-0.200pt]{2.409pt}{0.400pt}}
\put(1429.0,590.0){\rule[-0.200pt]{2.409pt}{0.400pt}}
\put(170.0,590.0){\rule[-0.200pt]{2.409pt}{0.400pt}}
\put(1429.0,590.0){\rule[-0.200pt]{2.409pt}{0.400pt}}
\put(170.0,590.0){\rule[-0.200pt]{2.409pt}{0.400pt}}
\put(1429.0,590.0){\rule[-0.200pt]{2.409pt}{0.400pt}}
\put(170.0,590.0){\rule[-0.200pt]{2.409pt}{0.400pt}}
\put(1429.0,590.0){\rule[-0.200pt]{2.409pt}{0.400pt}}
\put(170.0,590.0){\rule[-0.200pt]{2.409pt}{0.400pt}}
\put(1429.0,590.0){\rule[-0.200pt]{2.409pt}{0.400pt}}
\put(170.0,590.0){\rule[-0.200pt]{2.409pt}{0.400pt}}
\put(1429.0,590.0){\rule[-0.200pt]{2.409pt}{0.400pt}}
\put(170.0,590.0){\rule[-0.200pt]{2.409pt}{0.400pt}}
\put(1429.0,590.0){\rule[-0.200pt]{2.409pt}{0.400pt}}
\put(170.0,591.0){\rule[-0.200pt]{2.409pt}{0.400pt}}
\put(1429.0,591.0){\rule[-0.200pt]{2.409pt}{0.400pt}}
\put(170.0,591.0){\rule[-0.200pt]{2.409pt}{0.400pt}}
\put(1429.0,591.0){\rule[-0.200pt]{2.409pt}{0.400pt}}
\put(170.0,591.0){\rule[-0.200pt]{2.409pt}{0.400pt}}
\put(1429.0,591.0){\rule[-0.200pt]{2.409pt}{0.400pt}}
\put(170.0,591.0){\rule[-0.200pt]{2.409pt}{0.400pt}}
\put(1429.0,591.0){\rule[-0.200pt]{2.409pt}{0.400pt}}
\put(170.0,591.0){\rule[-0.200pt]{2.409pt}{0.400pt}}
\put(1429.0,591.0){\rule[-0.200pt]{2.409pt}{0.400pt}}
\put(170.0,591.0){\rule[-0.200pt]{2.409pt}{0.400pt}}
\put(1429.0,591.0){\rule[-0.200pt]{2.409pt}{0.400pt}}
\put(170.0,591.0){\rule[-0.200pt]{2.409pt}{0.400pt}}
\put(1429.0,591.0){\rule[-0.200pt]{2.409pt}{0.400pt}}
\put(170.0,591.0){\rule[-0.200pt]{2.409pt}{0.400pt}}
\put(1429.0,591.0){\rule[-0.200pt]{2.409pt}{0.400pt}}
\put(170.0,591.0){\rule[-0.200pt]{2.409pt}{0.400pt}}
\put(1429.0,591.0){\rule[-0.200pt]{2.409pt}{0.400pt}}
\put(170.0,591.0){\rule[-0.200pt]{2.409pt}{0.400pt}}
\put(1429.0,591.0){\rule[-0.200pt]{2.409pt}{0.400pt}}
\put(170.0,591.0){\rule[-0.200pt]{2.409pt}{0.400pt}}
\put(1429.0,591.0){\rule[-0.200pt]{2.409pt}{0.400pt}}
\put(170.0,591.0){\rule[-0.200pt]{2.409pt}{0.400pt}}
\put(1429.0,591.0){\rule[-0.200pt]{2.409pt}{0.400pt}}
\put(170.0,591.0){\rule[-0.200pt]{2.409pt}{0.400pt}}
\put(1429.0,591.0){\rule[-0.200pt]{2.409pt}{0.400pt}}
\put(170.0,591.0){\rule[-0.200pt]{2.409pt}{0.400pt}}
\put(1429.0,591.0){\rule[-0.200pt]{2.409pt}{0.400pt}}
\put(170.0,591.0){\rule[-0.200pt]{2.409pt}{0.400pt}}
\put(1429.0,591.0){\rule[-0.200pt]{2.409pt}{0.400pt}}
\put(170.0,591.0){\rule[-0.200pt]{2.409pt}{0.400pt}}
\put(1429.0,591.0){\rule[-0.200pt]{2.409pt}{0.400pt}}
\put(170.0,591.0){\rule[-0.200pt]{2.409pt}{0.400pt}}
\put(1429.0,591.0){\rule[-0.200pt]{2.409pt}{0.400pt}}
\put(170.0,591.0){\rule[-0.200pt]{2.409pt}{0.400pt}}
\put(1429.0,591.0){\rule[-0.200pt]{2.409pt}{0.400pt}}
\put(170.0,591.0){\rule[-0.200pt]{2.409pt}{0.400pt}}
\put(1429.0,591.0){\rule[-0.200pt]{2.409pt}{0.400pt}}
\put(170.0,591.0){\rule[-0.200pt]{2.409pt}{0.400pt}}
\put(1429.0,591.0){\rule[-0.200pt]{2.409pt}{0.400pt}}
\put(170.0,591.0){\rule[-0.200pt]{2.409pt}{0.400pt}}
\put(1429.0,591.0){\rule[-0.200pt]{2.409pt}{0.400pt}}
\put(170.0,592.0){\rule[-0.200pt]{2.409pt}{0.400pt}}
\put(1429.0,592.0){\rule[-0.200pt]{2.409pt}{0.400pt}}
\put(170.0,592.0){\rule[-0.200pt]{2.409pt}{0.400pt}}
\put(1429.0,592.0){\rule[-0.200pt]{2.409pt}{0.400pt}}
\put(170.0,592.0){\rule[-0.200pt]{2.409pt}{0.400pt}}
\put(1429.0,592.0){\rule[-0.200pt]{2.409pt}{0.400pt}}
\put(170.0,592.0){\rule[-0.200pt]{2.409pt}{0.400pt}}
\put(1429.0,592.0){\rule[-0.200pt]{2.409pt}{0.400pt}}
\put(170.0,592.0){\rule[-0.200pt]{2.409pt}{0.400pt}}
\put(1429.0,592.0){\rule[-0.200pt]{2.409pt}{0.400pt}}
\put(170.0,592.0){\rule[-0.200pt]{2.409pt}{0.400pt}}
\put(1429.0,592.0){\rule[-0.200pt]{2.409pt}{0.400pt}}
\put(170.0,592.0){\rule[-0.200pt]{2.409pt}{0.400pt}}
\put(1429.0,592.0){\rule[-0.200pt]{2.409pt}{0.400pt}}
\put(170.0,592.0){\rule[-0.200pt]{2.409pt}{0.400pt}}
\put(1429.0,592.0){\rule[-0.200pt]{2.409pt}{0.400pt}}
\put(170.0,592.0){\rule[-0.200pt]{2.409pt}{0.400pt}}
\put(1429.0,592.0){\rule[-0.200pt]{2.409pt}{0.400pt}}
\put(170.0,592.0){\rule[-0.200pt]{2.409pt}{0.400pt}}
\put(1429.0,592.0){\rule[-0.200pt]{2.409pt}{0.400pt}}
\put(170.0,592.0){\rule[-0.200pt]{2.409pt}{0.400pt}}
\put(1429.0,592.0){\rule[-0.200pt]{2.409pt}{0.400pt}}
\put(170.0,592.0){\rule[-0.200pt]{2.409pt}{0.400pt}}
\put(1429.0,592.0){\rule[-0.200pt]{2.409pt}{0.400pt}}
\put(170.0,592.0){\rule[-0.200pt]{2.409pt}{0.400pt}}
\put(1429.0,592.0){\rule[-0.200pt]{2.409pt}{0.400pt}}
\put(170.0,592.0){\rule[-0.200pt]{2.409pt}{0.400pt}}
\put(1429.0,592.0){\rule[-0.200pt]{2.409pt}{0.400pt}}
\put(170.0,592.0){\rule[-0.200pt]{2.409pt}{0.400pt}}
\put(1429.0,592.0){\rule[-0.200pt]{2.409pt}{0.400pt}}
\put(170.0,592.0){\rule[-0.200pt]{2.409pt}{0.400pt}}
\put(1429.0,592.0){\rule[-0.200pt]{2.409pt}{0.400pt}}
\put(170.0,592.0){\rule[-0.200pt]{2.409pt}{0.400pt}}
\put(1429.0,592.0){\rule[-0.200pt]{2.409pt}{0.400pt}}
\put(170.0,592.0){\rule[-0.200pt]{2.409pt}{0.400pt}}
\put(1429.0,592.0){\rule[-0.200pt]{2.409pt}{0.400pt}}
\put(170.0,592.0){\rule[-0.200pt]{2.409pt}{0.400pt}}
\put(1429.0,592.0){\rule[-0.200pt]{2.409pt}{0.400pt}}
\put(170.0,592.0){\rule[-0.200pt]{2.409pt}{0.400pt}}
\put(1429.0,592.0){\rule[-0.200pt]{2.409pt}{0.400pt}}
\put(170.0,592.0){\rule[-0.200pt]{2.409pt}{0.400pt}}
\put(1429.0,592.0){\rule[-0.200pt]{2.409pt}{0.400pt}}
\put(170.0,593.0){\rule[-0.200pt]{2.409pt}{0.400pt}}
\put(1429.0,593.0){\rule[-0.200pt]{2.409pt}{0.400pt}}
\put(170.0,593.0){\rule[-0.200pt]{2.409pt}{0.400pt}}
\put(1429.0,593.0){\rule[-0.200pt]{2.409pt}{0.400pt}}
\put(170.0,593.0){\rule[-0.200pt]{2.409pt}{0.400pt}}
\put(1429.0,593.0){\rule[-0.200pt]{2.409pt}{0.400pt}}
\put(170.0,593.0){\rule[-0.200pt]{2.409pt}{0.400pt}}
\put(1429.0,593.0){\rule[-0.200pt]{2.409pt}{0.400pt}}
\put(170.0,593.0){\rule[-0.200pt]{2.409pt}{0.400pt}}
\put(1429.0,593.0){\rule[-0.200pt]{2.409pt}{0.400pt}}
\put(170.0,593.0){\rule[-0.200pt]{2.409pt}{0.400pt}}
\put(1429.0,593.0){\rule[-0.200pt]{2.409pt}{0.400pt}}
\put(170.0,593.0){\rule[-0.200pt]{2.409pt}{0.400pt}}
\put(1429.0,593.0){\rule[-0.200pt]{2.409pt}{0.400pt}}
\put(170.0,593.0){\rule[-0.200pt]{2.409pt}{0.400pt}}
\put(1429.0,593.0){\rule[-0.200pt]{2.409pt}{0.400pt}}
\put(170.0,593.0){\rule[-0.200pt]{2.409pt}{0.400pt}}
\put(1429.0,593.0){\rule[-0.200pt]{2.409pt}{0.400pt}}
\put(170.0,593.0){\rule[-0.200pt]{2.409pt}{0.400pt}}
\put(1429.0,593.0){\rule[-0.200pt]{2.409pt}{0.400pt}}
\put(170.0,593.0){\rule[-0.200pt]{2.409pt}{0.400pt}}
\put(1429.0,593.0){\rule[-0.200pt]{2.409pt}{0.400pt}}
\put(170.0,593.0){\rule[-0.200pt]{2.409pt}{0.400pt}}
\put(1429.0,593.0){\rule[-0.200pt]{2.409pt}{0.400pt}}
\put(170.0,593.0){\rule[-0.200pt]{2.409pt}{0.400pt}}
\put(1429.0,593.0){\rule[-0.200pt]{2.409pt}{0.400pt}}
\put(170.0,593.0){\rule[-0.200pt]{2.409pt}{0.400pt}}
\put(1429.0,593.0){\rule[-0.200pt]{2.409pt}{0.400pt}}
\put(170.0,593.0){\rule[-0.200pt]{2.409pt}{0.400pt}}
\put(1429.0,593.0){\rule[-0.200pt]{2.409pt}{0.400pt}}
\put(170.0,593.0){\rule[-0.200pt]{2.409pt}{0.400pt}}
\put(1429.0,593.0){\rule[-0.200pt]{2.409pt}{0.400pt}}
\put(170.0,593.0){\rule[-0.200pt]{2.409pt}{0.400pt}}
\put(1429.0,593.0){\rule[-0.200pt]{2.409pt}{0.400pt}}
\put(170.0,593.0){\rule[-0.200pt]{2.409pt}{0.400pt}}
\put(1429.0,593.0){\rule[-0.200pt]{2.409pt}{0.400pt}}
\put(170.0,593.0){\rule[-0.200pt]{2.409pt}{0.400pt}}
\put(1429.0,593.0){\rule[-0.200pt]{2.409pt}{0.400pt}}
\put(170.0,593.0){\rule[-0.200pt]{2.409pt}{0.400pt}}
\put(1429.0,593.0){\rule[-0.200pt]{2.409pt}{0.400pt}}
\put(170.0,593.0){\rule[-0.200pt]{2.409pt}{0.400pt}}
\put(1429.0,593.0){\rule[-0.200pt]{2.409pt}{0.400pt}}
\put(170.0,593.0){\rule[-0.200pt]{2.409pt}{0.400pt}}
\put(1429.0,593.0){\rule[-0.200pt]{2.409pt}{0.400pt}}
\put(170.0,594.0){\rule[-0.200pt]{2.409pt}{0.400pt}}
\put(1429.0,594.0){\rule[-0.200pt]{2.409pt}{0.400pt}}
\put(170.0,594.0){\rule[-0.200pt]{2.409pt}{0.400pt}}
\put(1429.0,594.0){\rule[-0.200pt]{2.409pt}{0.400pt}}
\put(170.0,594.0){\rule[-0.200pt]{2.409pt}{0.400pt}}
\put(1429.0,594.0){\rule[-0.200pt]{2.409pt}{0.400pt}}
\put(170.0,594.0){\rule[-0.200pt]{2.409pt}{0.400pt}}
\put(1429.0,594.0){\rule[-0.200pt]{2.409pt}{0.400pt}}
\put(170.0,594.0){\rule[-0.200pt]{2.409pt}{0.400pt}}
\put(1429.0,594.0){\rule[-0.200pt]{2.409pt}{0.400pt}}
\put(170.0,594.0){\rule[-0.200pt]{2.409pt}{0.400pt}}
\put(1429.0,594.0){\rule[-0.200pt]{2.409pt}{0.400pt}}
\put(170.0,594.0){\rule[-0.200pt]{2.409pt}{0.400pt}}
\put(1429.0,594.0){\rule[-0.200pt]{2.409pt}{0.400pt}}
\put(170.0,594.0){\rule[-0.200pt]{2.409pt}{0.400pt}}
\put(1429.0,594.0){\rule[-0.200pt]{2.409pt}{0.400pt}}
\put(170.0,594.0){\rule[-0.200pt]{2.409pt}{0.400pt}}
\put(1429.0,594.0){\rule[-0.200pt]{2.409pt}{0.400pt}}
\put(170.0,594.0){\rule[-0.200pt]{2.409pt}{0.400pt}}
\put(1429.0,594.0){\rule[-0.200pt]{2.409pt}{0.400pt}}
\put(170.0,594.0){\rule[-0.200pt]{2.409pt}{0.400pt}}
\put(1429.0,594.0){\rule[-0.200pt]{2.409pt}{0.400pt}}
\put(170.0,594.0){\rule[-0.200pt]{2.409pt}{0.400pt}}
\put(1429.0,594.0){\rule[-0.200pt]{2.409pt}{0.400pt}}
\put(170.0,594.0){\rule[-0.200pt]{2.409pt}{0.400pt}}
\put(1429.0,594.0){\rule[-0.200pt]{2.409pt}{0.400pt}}
\put(170.0,594.0){\rule[-0.200pt]{2.409pt}{0.400pt}}
\put(1429.0,594.0){\rule[-0.200pt]{2.409pt}{0.400pt}}
\put(170.0,594.0){\rule[-0.200pt]{2.409pt}{0.400pt}}
\put(1429.0,594.0){\rule[-0.200pt]{2.409pt}{0.400pt}}
\put(170.0,594.0){\rule[-0.200pt]{2.409pt}{0.400pt}}
\put(1429.0,594.0){\rule[-0.200pt]{2.409pt}{0.400pt}}
\put(170.0,594.0){\rule[-0.200pt]{2.409pt}{0.400pt}}
\put(1429.0,594.0){\rule[-0.200pt]{2.409pt}{0.400pt}}
\put(170.0,594.0){\rule[-0.200pt]{2.409pt}{0.400pt}}
\put(1429.0,594.0){\rule[-0.200pt]{2.409pt}{0.400pt}}
\put(170.0,594.0){\rule[-0.200pt]{2.409pt}{0.400pt}}
\put(1429.0,594.0){\rule[-0.200pt]{2.409pt}{0.400pt}}
\put(170.0,594.0){\rule[-0.200pt]{2.409pt}{0.400pt}}
\put(1429.0,594.0){\rule[-0.200pt]{2.409pt}{0.400pt}}
\put(170.0,594.0){\rule[-0.200pt]{2.409pt}{0.400pt}}
\put(1429.0,594.0){\rule[-0.200pt]{2.409pt}{0.400pt}}
\put(170.0,594.0){\rule[-0.200pt]{2.409pt}{0.400pt}}
\put(1429.0,594.0){\rule[-0.200pt]{2.409pt}{0.400pt}}
\put(170.0,594.0){\rule[-0.200pt]{2.409pt}{0.400pt}}
\put(1429.0,594.0){\rule[-0.200pt]{2.409pt}{0.400pt}}
\put(170.0,595.0){\rule[-0.200pt]{2.409pt}{0.400pt}}
\put(1429.0,595.0){\rule[-0.200pt]{2.409pt}{0.400pt}}
\put(170.0,595.0){\rule[-0.200pt]{2.409pt}{0.400pt}}
\put(1429.0,595.0){\rule[-0.200pt]{2.409pt}{0.400pt}}
\put(170.0,595.0){\rule[-0.200pt]{2.409pt}{0.400pt}}
\put(1429.0,595.0){\rule[-0.200pt]{2.409pt}{0.400pt}}
\put(170.0,595.0){\rule[-0.200pt]{2.409pt}{0.400pt}}
\put(1429.0,595.0){\rule[-0.200pt]{2.409pt}{0.400pt}}
\put(170.0,595.0){\rule[-0.200pt]{2.409pt}{0.400pt}}
\put(1429.0,595.0){\rule[-0.200pt]{2.409pt}{0.400pt}}
\put(170.0,595.0){\rule[-0.200pt]{2.409pt}{0.400pt}}
\put(1429.0,595.0){\rule[-0.200pt]{2.409pt}{0.400pt}}
\put(170.0,595.0){\rule[-0.200pt]{2.409pt}{0.400pt}}
\put(1429.0,595.0){\rule[-0.200pt]{2.409pt}{0.400pt}}
\put(170.0,595.0){\rule[-0.200pt]{2.409pt}{0.400pt}}
\put(1429.0,595.0){\rule[-0.200pt]{2.409pt}{0.400pt}}
\put(170.0,595.0){\rule[-0.200pt]{2.409pt}{0.400pt}}
\put(1429.0,595.0){\rule[-0.200pt]{2.409pt}{0.400pt}}
\put(170.0,595.0){\rule[-0.200pt]{2.409pt}{0.400pt}}
\put(1429.0,595.0){\rule[-0.200pt]{2.409pt}{0.400pt}}
\put(170.0,595.0){\rule[-0.200pt]{2.409pt}{0.400pt}}
\put(1429.0,595.0){\rule[-0.200pt]{2.409pt}{0.400pt}}
\put(170.0,595.0){\rule[-0.200pt]{2.409pt}{0.400pt}}
\put(1429.0,595.0){\rule[-0.200pt]{2.409pt}{0.400pt}}
\put(170.0,595.0){\rule[-0.200pt]{2.409pt}{0.400pt}}
\put(1429.0,595.0){\rule[-0.200pt]{2.409pt}{0.400pt}}
\put(170.0,595.0){\rule[-0.200pt]{2.409pt}{0.400pt}}
\put(1429.0,595.0){\rule[-0.200pt]{2.409pt}{0.400pt}}
\put(170.0,595.0){\rule[-0.200pt]{2.409pt}{0.400pt}}
\put(1429.0,595.0){\rule[-0.200pt]{2.409pt}{0.400pt}}
\put(170.0,595.0){\rule[-0.200pt]{2.409pt}{0.400pt}}
\put(1429.0,595.0){\rule[-0.200pt]{2.409pt}{0.400pt}}
\put(170.0,595.0){\rule[-0.200pt]{2.409pt}{0.400pt}}
\put(1429.0,595.0){\rule[-0.200pt]{2.409pt}{0.400pt}}
\put(170.0,595.0){\rule[-0.200pt]{2.409pt}{0.400pt}}
\put(1429.0,595.0){\rule[-0.200pt]{2.409pt}{0.400pt}}
\put(170.0,595.0){\rule[-0.200pt]{2.409pt}{0.400pt}}
\put(1429.0,595.0){\rule[-0.200pt]{2.409pt}{0.400pt}}
\put(170.0,595.0){\rule[-0.200pt]{2.409pt}{0.400pt}}
\put(1429.0,595.0){\rule[-0.200pt]{2.409pt}{0.400pt}}
\put(170.0,595.0){\rule[-0.200pt]{2.409pt}{0.400pt}}
\put(1429.0,595.0){\rule[-0.200pt]{2.409pt}{0.400pt}}
\put(170.0,595.0){\rule[-0.200pt]{2.409pt}{0.400pt}}
\put(1429.0,595.0){\rule[-0.200pt]{2.409pt}{0.400pt}}
\put(170.0,595.0){\rule[-0.200pt]{2.409pt}{0.400pt}}
\put(1429.0,595.0){\rule[-0.200pt]{2.409pt}{0.400pt}}
\put(170.0,596.0){\rule[-0.200pt]{2.409pt}{0.400pt}}
\put(1429.0,596.0){\rule[-0.200pt]{2.409pt}{0.400pt}}
\put(170.0,596.0){\rule[-0.200pt]{2.409pt}{0.400pt}}
\put(1429.0,596.0){\rule[-0.200pt]{2.409pt}{0.400pt}}
\put(170.0,596.0){\rule[-0.200pt]{2.409pt}{0.400pt}}
\put(1429.0,596.0){\rule[-0.200pt]{2.409pt}{0.400pt}}
\put(170.0,596.0){\rule[-0.200pt]{2.409pt}{0.400pt}}
\put(1429.0,596.0){\rule[-0.200pt]{2.409pt}{0.400pt}}
\put(170.0,596.0){\rule[-0.200pt]{2.409pt}{0.400pt}}
\put(1429.0,596.0){\rule[-0.200pt]{2.409pt}{0.400pt}}
\put(170.0,596.0){\rule[-0.200pt]{2.409pt}{0.400pt}}
\put(1429.0,596.0){\rule[-0.200pt]{2.409pt}{0.400pt}}
\put(170.0,596.0){\rule[-0.200pt]{2.409pt}{0.400pt}}
\put(1429.0,596.0){\rule[-0.200pt]{2.409pt}{0.400pt}}
\put(170.0,596.0){\rule[-0.200pt]{2.409pt}{0.400pt}}
\put(1429.0,596.0){\rule[-0.200pt]{2.409pt}{0.400pt}}
\put(170.0,596.0){\rule[-0.200pt]{2.409pt}{0.400pt}}
\put(1429.0,596.0){\rule[-0.200pt]{2.409pt}{0.400pt}}
\put(170.0,596.0){\rule[-0.200pt]{2.409pt}{0.400pt}}
\put(1429.0,596.0){\rule[-0.200pt]{2.409pt}{0.400pt}}
\put(170.0,596.0){\rule[-0.200pt]{2.409pt}{0.400pt}}
\put(1429.0,596.0){\rule[-0.200pt]{2.409pt}{0.400pt}}
\put(170.0,596.0){\rule[-0.200pt]{2.409pt}{0.400pt}}
\put(1429.0,596.0){\rule[-0.200pt]{2.409pt}{0.400pt}}
\put(170.0,596.0){\rule[-0.200pt]{2.409pt}{0.400pt}}
\put(1429.0,596.0){\rule[-0.200pt]{2.409pt}{0.400pt}}
\put(170.0,596.0){\rule[-0.200pt]{2.409pt}{0.400pt}}
\put(1429.0,596.0){\rule[-0.200pt]{2.409pt}{0.400pt}}
\put(170.0,596.0){\rule[-0.200pt]{2.409pt}{0.400pt}}
\put(1429.0,596.0){\rule[-0.200pt]{2.409pt}{0.400pt}}
\put(170.0,596.0){\rule[-0.200pt]{2.409pt}{0.400pt}}
\put(1429.0,596.0){\rule[-0.200pt]{2.409pt}{0.400pt}}
\put(170.0,596.0){\rule[-0.200pt]{2.409pt}{0.400pt}}
\put(1429.0,596.0){\rule[-0.200pt]{2.409pt}{0.400pt}}
\put(170.0,596.0){\rule[-0.200pt]{2.409pt}{0.400pt}}
\put(1429.0,596.0){\rule[-0.200pt]{2.409pt}{0.400pt}}
\put(170.0,596.0){\rule[-0.200pt]{2.409pt}{0.400pt}}
\put(1429.0,596.0){\rule[-0.200pt]{2.409pt}{0.400pt}}
\put(170.0,596.0){\rule[-0.200pt]{2.409pt}{0.400pt}}
\put(1429.0,596.0){\rule[-0.200pt]{2.409pt}{0.400pt}}
\put(170.0,596.0){\rule[-0.200pt]{2.409pt}{0.400pt}}
\put(1429.0,596.0){\rule[-0.200pt]{2.409pt}{0.400pt}}
\put(170.0,596.0){\rule[-0.200pt]{2.409pt}{0.400pt}}
\put(1429.0,596.0){\rule[-0.200pt]{2.409pt}{0.400pt}}
\put(170.0,596.0){\rule[-0.200pt]{2.409pt}{0.400pt}}
\put(1429.0,596.0){\rule[-0.200pt]{2.409pt}{0.400pt}}
\put(170.0,596.0){\rule[-0.200pt]{2.409pt}{0.400pt}}
\put(1429.0,596.0){\rule[-0.200pt]{2.409pt}{0.400pt}}
\put(170.0,597.0){\rule[-0.200pt]{2.409pt}{0.400pt}}
\put(1429.0,597.0){\rule[-0.200pt]{2.409pt}{0.400pt}}
\put(170.0,597.0){\rule[-0.200pt]{2.409pt}{0.400pt}}
\put(1429.0,597.0){\rule[-0.200pt]{2.409pt}{0.400pt}}
\put(170.0,597.0){\rule[-0.200pt]{2.409pt}{0.400pt}}
\put(1429.0,597.0){\rule[-0.200pt]{2.409pt}{0.400pt}}
\put(170.0,597.0){\rule[-0.200pt]{2.409pt}{0.400pt}}
\put(1429.0,597.0){\rule[-0.200pt]{2.409pt}{0.400pt}}
\put(170.0,597.0){\rule[-0.200pt]{2.409pt}{0.400pt}}
\put(1429.0,597.0){\rule[-0.200pt]{2.409pt}{0.400pt}}
\put(170.0,597.0){\rule[-0.200pt]{2.409pt}{0.400pt}}
\put(1429.0,597.0){\rule[-0.200pt]{2.409pt}{0.400pt}}
\put(170.0,597.0){\rule[-0.200pt]{2.409pt}{0.400pt}}
\put(1429.0,597.0){\rule[-0.200pt]{2.409pt}{0.400pt}}
\put(170.0,597.0){\rule[-0.200pt]{2.409pt}{0.400pt}}
\put(1429.0,597.0){\rule[-0.200pt]{2.409pt}{0.400pt}}
\put(170.0,597.0){\rule[-0.200pt]{2.409pt}{0.400pt}}
\put(1429.0,597.0){\rule[-0.200pt]{2.409pt}{0.400pt}}
\put(170.0,597.0){\rule[-0.200pt]{2.409pt}{0.400pt}}
\put(1429.0,597.0){\rule[-0.200pt]{2.409pt}{0.400pt}}
\put(170.0,597.0){\rule[-0.200pt]{2.409pt}{0.400pt}}
\put(1429.0,597.0){\rule[-0.200pt]{2.409pt}{0.400pt}}
\put(170.0,597.0){\rule[-0.200pt]{2.409pt}{0.400pt}}
\put(1429.0,597.0){\rule[-0.200pt]{2.409pt}{0.400pt}}
\put(170.0,597.0){\rule[-0.200pt]{2.409pt}{0.400pt}}
\put(1429.0,597.0){\rule[-0.200pt]{2.409pt}{0.400pt}}
\put(170.0,597.0){\rule[-0.200pt]{2.409pt}{0.400pt}}
\put(1429.0,597.0){\rule[-0.200pt]{2.409pt}{0.400pt}}
\put(170.0,597.0){\rule[-0.200pt]{2.409pt}{0.400pt}}
\put(1429.0,597.0){\rule[-0.200pt]{2.409pt}{0.400pt}}
\put(170.0,597.0){\rule[-0.200pt]{2.409pt}{0.400pt}}
\put(1429.0,597.0){\rule[-0.200pt]{2.409pt}{0.400pt}}
\put(170.0,597.0){\rule[-0.200pt]{2.409pt}{0.400pt}}
\put(1429.0,597.0){\rule[-0.200pt]{2.409pt}{0.400pt}}
\put(170.0,597.0){\rule[-0.200pt]{2.409pt}{0.400pt}}
\put(1429.0,597.0){\rule[-0.200pt]{2.409pt}{0.400pt}}
\put(170.0,597.0){\rule[-0.200pt]{2.409pt}{0.400pt}}
\put(1429.0,597.0){\rule[-0.200pt]{2.409pt}{0.400pt}}
\put(170.0,597.0){\rule[-0.200pt]{2.409pt}{0.400pt}}
\put(1429.0,597.0){\rule[-0.200pt]{2.409pt}{0.400pt}}
\put(170.0,597.0){\rule[-0.200pt]{2.409pt}{0.400pt}}
\put(1429.0,597.0){\rule[-0.200pt]{2.409pt}{0.400pt}}
\put(170.0,597.0){\rule[-0.200pt]{2.409pt}{0.400pt}}
\put(1429.0,597.0){\rule[-0.200pt]{2.409pt}{0.400pt}}
\put(170.0,597.0){\rule[-0.200pt]{2.409pt}{0.400pt}}
\put(1429.0,597.0){\rule[-0.200pt]{2.409pt}{0.400pt}}
\put(170.0,597.0){\rule[-0.200pt]{2.409pt}{0.400pt}}
\put(1429.0,597.0){\rule[-0.200pt]{2.409pt}{0.400pt}}
\put(170.0,597.0){\rule[-0.200pt]{2.409pt}{0.400pt}}
\put(1429.0,597.0){\rule[-0.200pt]{2.409pt}{0.400pt}}
\put(170.0,598.0){\rule[-0.200pt]{2.409pt}{0.400pt}}
\put(1429.0,598.0){\rule[-0.200pt]{2.409pt}{0.400pt}}
\put(170.0,598.0){\rule[-0.200pt]{2.409pt}{0.400pt}}
\put(1429.0,598.0){\rule[-0.200pt]{2.409pt}{0.400pt}}
\put(170.0,598.0){\rule[-0.200pt]{2.409pt}{0.400pt}}
\put(1429.0,598.0){\rule[-0.200pt]{2.409pt}{0.400pt}}
\put(170.0,598.0){\rule[-0.200pt]{2.409pt}{0.400pt}}
\put(1429.0,598.0){\rule[-0.200pt]{2.409pt}{0.400pt}}
\put(170.0,598.0){\rule[-0.200pt]{2.409pt}{0.400pt}}
\put(1429.0,598.0){\rule[-0.200pt]{2.409pt}{0.400pt}}
\put(170.0,598.0){\rule[-0.200pt]{2.409pt}{0.400pt}}
\put(1429.0,598.0){\rule[-0.200pt]{2.409pt}{0.400pt}}
\put(170.0,598.0){\rule[-0.200pt]{2.409pt}{0.400pt}}
\put(1429.0,598.0){\rule[-0.200pt]{2.409pt}{0.400pt}}
\put(170.0,598.0){\rule[-0.200pt]{2.409pt}{0.400pt}}
\put(1429.0,598.0){\rule[-0.200pt]{2.409pt}{0.400pt}}
\put(170.0,598.0){\rule[-0.200pt]{2.409pt}{0.400pt}}
\put(1429.0,598.0){\rule[-0.200pt]{2.409pt}{0.400pt}}
\put(170.0,598.0){\rule[-0.200pt]{2.409pt}{0.400pt}}
\put(1429.0,598.0){\rule[-0.200pt]{2.409pt}{0.400pt}}
\put(170.0,598.0){\rule[-0.200pt]{2.409pt}{0.400pt}}
\put(1429.0,598.0){\rule[-0.200pt]{2.409pt}{0.400pt}}
\put(170.0,598.0){\rule[-0.200pt]{2.409pt}{0.400pt}}
\put(1429.0,598.0){\rule[-0.200pt]{2.409pt}{0.400pt}}
\put(170.0,598.0){\rule[-0.200pt]{2.409pt}{0.400pt}}
\put(1429.0,598.0){\rule[-0.200pt]{2.409pt}{0.400pt}}
\put(170.0,598.0){\rule[-0.200pt]{2.409pt}{0.400pt}}
\put(1429.0,598.0){\rule[-0.200pt]{2.409pt}{0.400pt}}
\put(170.0,598.0){\rule[-0.200pt]{2.409pt}{0.400pt}}
\put(1429.0,598.0){\rule[-0.200pt]{2.409pt}{0.400pt}}
\put(170.0,598.0){\rule[-0.200pt]{2.409pt}{0.400pt}}
\put(1429.0,598.0){\rule[-0.200pt]{2.409pt}{0.400pt}}
\put(170.0,598.0){\rule[-0.200pt]{2.409pt}{0.400pt}}
\put(1429.0,598.0){\rule[-0.200pt]{2.409pt}{0.400pt}}
\put(170.0,598.0){\rule[-0.200pt]{2.409pt}{0.400pt}}
\put(1429.0,598.0){\rule[-0.200pt]{2.409pt}{0.400pt}}
\put(170.0,598.0){\rule[-0.200pt]{2.409pt}{0.400pt}}
\put(1429.0,598.0){\rule[-0.200pt]{2.409pt}{0.400pt}}
\put(170.0,598.0){\rule[-0.200pt]{2.409pt}{0.400pt}}
\put(1429.0,598.0){\rule[-0.200pt]{2.409pt}{0.400pt}}
\put(170.0,598.0){\rule[-0.200pt]{2.409pt}{0.400pt}}
\put(1429.0,598.0){\rule[-0.200pt]{2.409pt}{0.400pt}}
\put(170.0,598.0){\rule[-0.200pt]{2.409pt}{0.400pt}}
\put(1429.0,598.0){\rule[-0.200pt]{2.409pt}{0.400pt}}
\put(170.0,598.0){\rule[-0.200pt]{2.409pt}{0.400pt}}
\put(1429.0,598.0){\rule[-0.200pt]{2.409pt}{0.400pt}}
\put(170.0,598.0){\rule[-0.200pt]{2.409pt}{0.400pt}}
\put(1429.0,598.0){\rule[-0.200pt]{2.409pt}{0.400pt}}
\put(170.0,598.0){\rule[-0.200pt]{2.409pt}{0.400pt}}
\put(1429.0,598.0){\rule[-0.200pt]{2.409pt}{0.400pt}}
\put(170.0,599.0){\rule[-0.200pt]{2.409pt}{0.400pt}}
\put(1429.0,599.0){\rule[-0.200pt]{2.409pt}{0.400pt}}
\put(170.0,599.0){\rule[-0.200pt]{2.409pt}{0.400pt}}
\put(1429.0,599.0){\rule[-0.200pt]{2.409pt}{0.400pt}}
\put(170.0,599.0){\rule[-0.200pt]{2.409pt}{0.400pt}}
\put(1429.0,599.0){\rule[-0.200pt]{2.409pt}{0.400pt}}
\put(170.0,599.0){\rule[-0.200pt]{2.409pt}{0.400pt}}
\put(1429.0,599.0){\rule[-0.200pt]{2.409pt}{0.400pt}}
\put(170.0,599.0){\rule[-0.200pt]{2.409pt}{0.400pt}}
\put(1429.0,599.0){\rule[-0.200pt]{2.409pt}{0.400pt}}
\put(170.0,599.0){\rule[-0.200pt]{2.409pt}{0.400pt}}
\put(1429.0,599.0){\rule[-0.200pt]{2.409pt}{0.400pt}}
\put(170.0,599.0){\rule[-0.200pt]{2.409pt}{0.400pt}}
\put(1429.0,599.0){\rule[-0.200pt]{2.409pt}{0.400pt}}
\put(170.0,599.0){\rule[-0.200pt]{2.409pt}{0.400pt}}
\put(1429.0,599.0){\rule[-0.200pt]{2.409pt}{0.400pt}}
\put(170.0,599.0){\rule[-0.200pt]{2.409pt}{0.400pt}}
\put(1429.0,599.0){\rule[-0.200pt]{2.409pt}{0.400pt}}
\put(170.0,599.0){\rule[-0.200pt]{2.409pt}{0.400pt}}
\put(1429.0,599.0){\rule[-0.200pt]{2.409pt}{0.400pt}}
\put(170.0,599.0){\rule[-0.200pt]{2.409pt}{0.400pt}}
\put(1429.0,599.0){\rule[-0.200pt]{2.409pt}{0.400pt}}
\put(170.0,599.0){\rule[-0.200pt]{2.409pt}{0.400pt}}
\put(1429.0,599.0){\rule[-0.200pt]{2.409pt}{0.400pt}}
\put(170.0,599.0){\rule[-0.200pt]{2.409pt}{0.400pt}}
\put(1429.0,599.0){\rule[-0.200pt]{2.409pt}{0.400pt}}
\put(170.0,599.0){\rule[-0.200pt]{2.409pt}{0.400pt}}
\put(1429.0,599.0){\rule[-0.200pt]{2.409pt}{0.400pt}}
\put(170.0,599.0){\rule[-0.200pt]{2.409pt}{0.400pt}}
\put(1429.0,599.0){\rule[-0.200pt]{2.409pt}{0.400pt}}
\put(170.0,599.0){\rule[-0.200pt]{2.409pt}{0.400pt}}
\put(1429.0,599.0){\rule[-0.200pt]{2.409pt}{0.400pt}}
\put(170.0,599.0){\rule[-0.200pt]{2.409pt}{0.400pt}}
\put(1429.0,599.0){\rule[-0.200pt]{2.409pt}{0.400pt}}
\put(170.0,599.0){\rule[-0.200pt]{2.409pt}{0.400pt}}
\put(1429.0,599.0){\rule[-0.200pt]{2.409pt}{0.400pt}}
\put(170.0,599.0){\rule[-0.200pt]{2.409pt}{0.400pt}}
\put(1429.0,599.0){\rule[-0.200pt]{2.409pt}{0.400pt}}
\put(170.0,599.0){\rule[-0.200pt]{2.409pt}{0.400pt}}
\put(1429.0,599.0){\rule[-0.200pt]{2.409pt}{0.400pt}}
\put(170.0,599.0){\rule[-0.200pt]{2.409pt}{0.400pt}}
\put(1429.0,599.0){\rule[-0.200pt]{2.409pt}{0.400pt}}
\put(170.0,599.0){\rule[-0.200pt]{2.409pt}{0.400pt}}
\put(1429.0,599.0){\rule[-0.200pt]{2.409pt}{0.400pt}}
\put(170.0,599.0){\rule[-0.200pt]{2.409pt}{0.400pt}}
\put(1429.0,599.0){\rule[-0.200pt]{2.409pt}{0.400pt}}
\put(170.0,599.0){\rule[-0.200pt]{2.409pt}{0.400pt}}
\put(1429.0,599.0){\rule[-0.200pt]{2.409pt}{0.400pt}}
\put(170.0,599.0){\rule[-0.200pt]{2.409pt}{0.400pt}}
\put(1429.0,599.0){\rule[-0.200pt]{2.409pt}{0.400pt}}
\put(170.0,599.0){\rule[-0.200pt]{2.409pt}{0.400pt}}
\put(1429.0,599.0){\rule[-0.200pt]{2.409pt}{0.400pt}}
\put(170.0,600.0){\rule[-0.200pt]{2.409pt}{0.400pt}}
\put(1429.0,600.0){\rule[-0.200pt]{2.409pt}{0.400pt}}
\put(170.0,600.0){\rule[-0.200pt]{2.409pt}{0.400pt}}
\put(1429.0,600.0){\rule[-0.200pt]{2.409pt}{0.400pt}}
\put(170.0,600.0){\rule[-0.200pt]{2.409pt}{0.400pt}}
\put(1429.0,600.0){\rule[-0.200pt]{2.409pt}{0.400pt}}
\put(170.0,600.0){\rule[-0.200pt]{2.409pt}{0.400pt}}
\put(1429.0,600.0){\rule[-0.200pt]{2.409pt}{0.400pt}}
\put(170.0,600.0){\rule[-0.200pt]{2.409pt}{0.400pt}}
\put(1429.0,600.0){\rule[-0.200pt]{2.409pt}{0.400pt}}
\put(170.0,600.0){\rule[-0.200pt]{2.409pt}{0.400pt}}
\put(1429.0,600.0){\rule[-0.200pt]{2.409pt}{0.400pt}}
\put(170.0,600.0){\rule[-0.200pt]{2.409pt}{0.400pt}}
\put(1429.0,600.0){\rule[-0.200pt]{2.409pt}{0.400pt}}
\put(170.0,600.0){\rule[-0.200pt]{2.409pt}{0.400pt}}
\put(1429.0,600.0){\rule[-0.200pt]{2.409pt}{0.400pt}}
\put(170.0,600.0){\rule[-0.200pt]{2.409pt}{0.400pt}}
\put(1429.0,600.0){\rule[-0.200pt]{2.409pt}{0.400pt}}
\put(170.0,600.0){\rule[-0.200pt]{2.409pt}{0.400pt}}
\put(1429.0,600.0){\rule[-0.200pt]{2.409pt}{0.400pt}}
\put(170.0,600.0){\rule[-0.200pt]{2.409pt}{0.400pt}}
\put(1429.0,600.0){\rule[-0.200pt]{2.409pt}{0.400pt}}
\put(170.0,600.0){\rule[-0.200pt]{2.409pt}{0.400pt}}
\put(1429.0,600.0){\rule[-0.200pt]{2.409pt}{0.400pt}}
\put(170.0,600.0){\rule[-0.200pt]{2.409pt}{0.400pt}}
\put(1429.0,600.0){\rule[-0.200pt]{2.409pt}{0.400pt}}
\put(170.0,600.0){\rule[-0.200pt]{4.818pt}{0.400pt}}
\put(150,600){\makebox(0,0)[r]{ 1e+06}}
\put(1419.0,600.0){\rule[-0.200pt]{4.818pt}{0.400pt}}
\put(170.0,626.0){\rule[-0.200pt]{2.409pt}{0.400pt}}
\put(1429.0,626.0){\rule[-0.200pt]{2.409pt}{0.400pt}}
\put(170.0,641.0){\rule[-0.200pt]{2.409pt}{0.400pt}}
\put(1429.0,641.0){\rule[-0.200pt]{2.409pt}{0.400pt}}
\put(170.0,652.0){\rule[-0.200pt]{2.409pt}{0.400pt}}
\put(1429.0,652.0){\rule[-0.200pt]{2.409pt}{0.400pt}}
\put(170.0,660.0){\rule[-0.200pt]{2.409pt}{0.400pt}}
\put(1429.0,660.0){\rule[-0.200pt]{2.409pt}{0.400pt}}
\put(170.0,667.0){\rule[-0.200pt]{2.409pt}{0.400pt}}
\put(1429.0,667.0){\rule[-0.200pt]{2.409pt}{0.400pt}}
\put(170.0,673.0){\rule[-0.200pt]{2.409pt}{0.400pt}}
\put(1429.0,673.0){\rule[-0.200pt]{2.409pt}{0.400pt}}
\put(170.0,678.0){\rule[-0.200pt]{2.409pt}{0.400pt}}
\put(1429.0,678.0){\rule[-0.200pt]{2.409pt}{0.400pt}}
\put(170.0,682.0){\rule[-0.200pt]{2.409pt}{0.400pt}}
\put(1429.0,682.0){\rule[-0.200pt]{2.409pt}{0.400pt}}
\put(170.0,686.0){\rule[-0.200pt]{2.409pt}{0.400pt}}
\put(1429.0,686.0){\rule[-0.200pt]{2.409pt}{0.400pt}}
\put(170.0,690.0){\rule[-0.200pt]{2.409pt}{0.400pt}}
\put(1429.0,690.0){\rule[-0.200pt]{2.409pt}{0.400pt}}
\put(170.0,693.0){\rule[-0.200pt]{2.409pt}{0.400pt}}
\put(1429.0,693.0){\rule[-0.200pt]{2.409pt}{0.400pt}}
\put(170.0,696.0){\rule[-0.200pt]{2.409pt}{0.400pt}}
\put(1429.0,696.0){\rule[-0.200pt]{2.409pt}{0.400pt}}
\put(170.0,699.0){\rule[-0.200pt]{2.409pt}{0.400pt}}
\put(1429.0,699.0){\rule[-0.200pt]{2.409pt}{0.400pt}}
\put(170.0,702.0){\rule[-0.200pt]{2.409pt}{0.400pt}}
\put(1429.0,702.0){\rule[-0.200pt]{2.409pt}{0.400pt}}
\put(170.0,704.0){\rule[-0.200pt]{2.409pt}{0.400pt}}
\put(1429.0,704.0){\rule[-0.200pt]{2.409pt}{0.400pt}}
\put(170.0,706.0){\rule[-0.200pt]{2.409pt}{0.400pt}}
\put(1429.0,706.0){\rule[-0.200pt]{2.409pt}{0.400pt}}
\put(170.0,708.0){\rule[-0.200pt]{2.409pt}{0.400pt}}
\put(1429.0,708.0){\rule[-0.200pt]{2.409pt}{0.400pt}}
\put(170.0,710.0){\rule[-0.200pt]{2.409pt}{0.400pt}}
\put(1429.0,710.0){\rule[-0.200pt]{2.409pt}{0.400pt}}
\put(170.0,712.0){\rule[-0.200pt]{2.409pt}{0.400pt}}
\put(1429.0,712.0){\rule[-0.200pt]{2.409pt}{0.400pt}}
\put(170.0,714.0){\rule[-0.200pt]{2.409pt}{0.400pt}}
\put(1429.0,714.0){\rule[-0.200pt]{2.409pt}{0.400pt}}
\put(170.0,716.0){\rule[-0.200pt]{2.409pt}{0.400pt}}
\put(1429.0,716.0){\rule[-0.200pt]{2.409pt}{0.400pt}}
\put(170.0,718.0){\rule[-0.200pt]{2.409pt}{0.400pt}}
\put(1429.0,718.0){\rule[-0.200pt]{2.409pt}{0.400pt}}
\put(170.0,719.0){\rule[-0.200pt]{2.409pt}{0.400pt}}
\put(1429.0,719.0){\rule[-0.200pt]{2.409pt}{0.400pt}}
\put(170.0,721.0){\rule[-0.200pt]{2.409pt}{0.400pt}}
\put(1429.0,721.0){\rule[-0.200pt]{2.409pt}{0.400pt}}
\put(170.0,722.0){\rule[-0.200pt]{2.409pt}{0.400pt}}
\put(1429.0,722.0){\rule[-0.200pt]{2.409pt}{0.400pt}}
\put(170.0,724.0){\rule[-0.200pt]{2.409pt}{0.400pt}}
\put(1429.0,724.0){\rule[-0.200pt]{2.409pt}{0.400pt}}
\put(170.0,725.0){\rule[-0.200pt]{2.409pt}{0.400pt}}
\put(1429.0,725.0){\rule[-0.200pt]{2.409pt}{0.400pt}}
\put(170.0,726.0){\rule[-0.200pt]{2.409pt}{0.400pt}}
\put(1429.0,726.0){\rule[-0.200pt]{2.409pt}{0.400pt}}
\put(170.0,728.0){\rule[-0.200pt]{2.409pt}{0.400pt}}
\put(1429.0,728.0){\rule[-0.200pt]{2.409pt}{0.400pt}}
\put(170.0,729.0){\rule[-0.200pt]{2.409pt}{0.400pt}}
\put(1429.0,729.0){\rule[-0.200pt]{2.409pt}{0.400pt}}
\put(170.0,730.0){\rule[-0.200pt]{2.409pt}{0.400pt}}
\put(1429.0,730.0){\rule[-0.200pt]{2.409pt}{0.400pt}}
\put(170.0,731.0){\rule[-0.200pt]{2.409pt}{0.400pt}}
\put(1429.0,731.0){\rule[-0.200pt]{2.409pt}{0.400pt}}
\put(170.0,732.0){\rule[-0.200pt]{2.409pt}{0.400pt}}
\put(1429.0,732.0){\rule[-0.200pt]{2.409pt}{0.400pt}}
\put(170.0,733.0){\rule[-0.200pt]{2.409pt}{0.400pt}}
\put(1429.0,733.0){\rule[-0.200pt]{2.409pt}{0.400pt}}
\put(170.0,734.0){\rule[-0.200pt]{2.409pt}{0.400pt}}
\put(1429.0,734.0){\rule[-0.200pt]{2.409pt}{0.400pt}}
\put(170.0,735.0){\rule[-0.200pt]{2.409pt}{0.400pt}}
\put(1429.0,735.0){\rule[-0.200pt]{2.409pt}{0.400pt}}
\put(170.0,736.0){\rule[-0.200pt]{2.409pt}{0.400pt}}
\put(1429.0,736.0){\rule[-0.200pt]{2.409pt}{0.400pt}}
\put(170.0,737.0){\rule[-0.200pt]{2.409pt}{0.400pt}}
\put(1429.0,737.0){\rule[-0.200pt]{2.409pt}{0.400pt}}
\put(170.0,738.0){\rule[-0.200pt]{2.409pt}{0.400pt}}
\put(1429.0,738.0){\rule[-0.200pt]{2.409pt}{0.400pt}}
\put(170.0,739.0){\rule[-0.200pt]{2.409pt}{0.400pt}}
\put(1429.0,739.0){\rule[-0.200pt]{2.409pt}{0.400pt}}
\put(170.0,740.0){\rule[-0.200pt]{2.409pt}{0.400pt}}
\put(1429.0,740.0){\rule[-0.200pt]{2.409pt}{0.400pt}}
\put(170.0,741.0){\rule[-0.200pt]{2.409pt}{0.400pt}}
\put(1429.0,741.0){\rule[-0.200pt]{2.409pt}{0.400pt}}
\put(170.0,742.0){\rule[-0.200pt]{2.409pt}{0.400pt}}
\put(1429.0,742.0){\rule[-0.200pt]{2.409pt}{0.400pt}}
\put(170.0,743.0){\rule[-0.200pt]{2.409pt}{0.400pt}}
\put(1429.0,743.0){\rule[-0.200pt]{2.409pt}{0.400pt}}
\put(170.0,744.0){\rule[-0.200pt]{2.409pt}{0.400pt}}
\put(1429.0,744.0){\rule[-0.200pt]{2.409pt}{0.400pt}}
\put(170.0,744.0){\rule[-0.200pt]{2.409pt}{0.400pt}}
\put(1429.0,744.0){\rule[-0.200pt]{2.409pt}{0.400pt}}
\put(170.0,745.0){\rule[-0.200pt]{2.409pt}{0.400pt}}
\put(1429.0,745.0){\rule[-0.200pt]{2.409pt}{0.400pt}}
\put(170.0,746.0){\rule[-0.200pt]{2.409pt}{0.400pt}}
\put(1429.0,746.0){\rule[-0.200pt]{2.409pt}{0.400pt}}
\put(170.0,747.0){\rule[-0.200pt]{2.409pt}{0.400pt}}
\put(1429.0,747.0){\rule[-0.200pt]{2.409pt}{0.400pt}}
\put(170.0,747.0){\rule[-0.200pt]{2.409pt}{0.400pt}}
\put(1429.0,747.0){\rule[-0.200pt]{2.409pt}{0.400pt}}
\put(170.0,748.0){\rule[-0.200pt]{2.409pt}{0.400pt}}
\put(1429.0,748.0){\rule[-0.200pt]{2.409pt}{0.400pt}}
\put(170.0,749.0){\rule[-0.200pt]{2.409pt}{0.400pt}}
\put(1429.0,749.0){\rule[-0.200pt]{2.409pt}{0.400pt}}
\put(170.0,750.0){\rule[-0.200pt]{2.409pt}{0.400pt}}
\put(1429.0,750.0){\rule[-0.200pt]{2.409pt}{0.400pt}}
\put(170.0,750.0){\rule[-0.200pt]{2.409pt}{0.400pt}}
\put(1429.0,750.0){\rule[-0.200pt]{2.409pt}{0.400pt}}
\put(170.0,751.0){\rule[-0.200pt]{2.409pt}{0.400pt}}
\put(1429.0,751.0){\rule[-0.200pt]{2.409pt}{0.400pt}}
\put(170.0,752.0){\rule[-0.200pt]{2.409pt}{0.400pt}}
\put(1429.0,752.0){\rule[-0.200pt]{2.409pt}{0.400pt}}
\put(170.0,752.0){\rule[-0.200pt]{2.409pt}{0.400pt}}
\put(1429.0,752.0){\rule[-0.200pt]{2.409pt}{0.400pt}}
\put(170.0,753.0){\rule[-0.200pt]{2.409pt}{0.400pt}}
\put(1429.0,753.0){\rule[-0.200pt]{2.409pt}{0.400pt}}
\put(170.0,754.0){\rule[-0.200pt]{2.409pt}{0.400pt}}
\put(1429.0,754.0){\rule[-0.200pt]{2.409pt}{0.400pt}}
\put(170.0,754.0){\rule[-0.200pt]{2.409pt}{0.400pt}}
\put(1429.0,754.0){\rule[-0.200pt]{2.409pt}{0.400pt}}
\put(170.0,755.0){\rule[-0.200pt]{2.409pt}{0.400pt}}
\put(1429.0,755.0){\rule[-0.200pt]{2.409pt}{0.400pt}}
\put(170.0,755.0){\rule[-0.200pt]{2.409pt}{0.400pt}}
\put(1429.0,755.0){\rule[-0.200pt]{2.409pt}{0.400pt}}
\put(170.0,756.0){\rule[-0.200pt]{2.409pt}{0.400pt}}
\put(1429.0,756.0){\rule[-0.200pt]{2.409pt}{0.400pt}}
\put(170.0,757.0){\rule[-0.200pt]{2.409pt}{0.400pt}}
\put(1429.0,757.0){\rule[-0.200pt]{2.409pt}{0.400pt}}
\put(170.0,757.0){\rule[-0.200pt]{2.409pt}{0.400pt}}
\put(1429.0,757.0){\rule[-0.200pt]{2.409pt}{0.400pt}}
\put(170.0,758.0){\rule[-0.200pt]{2.409pt}{0.400pt}}
\put(1429.0,758.0){\rule[-0.200pt]{2.409pt}{0.400pt}}
\put(170.0,758.0){\rule[-0.200pt]{2.409pt}{0.400pt}}
\put(1429.0,758.0){\rule[-0.200pt]{2.409pt}{0.400pt}}
\put(170.0,759.0){\rule[-0.200pt]{2.409pt}{0.400pt}}
\put(1429.0,759.0){\rule[-0.200pt]{2.409pt}{0.400pt}}
\put(170.0,759.0){\rule[-0.200pt]{2.409pt}{0.400pt}}
\put(1429.0,759.0){\rule[-0.200pt]{2.409pt}{0.400pt}}
\put(170.0,760.0){\rule[-0.200pt]{2.409pt}{0.400pt}}
\put(1429.0,760.0){\rule[-0.200pt]{2.409pt}{0.400pt}}
\put(170.0,760.0){\rule[-0.200pt]{2.409pt}{0.400pt}}
\put(1429.0,760.0){\rule[-0.200pt]{2.409pt}{0.400pt}}
\put(170.0,761.0){\rule[-0.200pt]{2.409pt}{0.400pt}}
\put(1429.0,761.0){\rule[-0.200pt]{2.409pt}{0.400pt}}
\put(170.0,761.0){\rule[-0.200pt]{2.409pt}{0.400pt}}
\put(1429.0,761.0){\rule[-0.200pt]{2.409pt}{0.400pt}}
\put(170.0,762.0){\rule[-0.200pt]{2.409pt}{0.400pt}}
\put(1429.0,762.0){\rule[-0.200pt]{2.409pt}{0.400pt}}
\put(170.0,762.0){\rule[-0.200pt]{2.409pt}{0.400pt}}
\put(1429.0,762.0){\rule[-0.200pt]{2.409pt}{0.400pt}}
\put(170.0,763.0){\rule[-0.200pt]{2.409pt}{0.400pt}}
\put(1429.0,763.0){\rule[-0.200pt]{2.409pt}{0.400pt}}
\put(170.0,763.0){\rule[-0.200pt]{2.409pt}{0.400pt}}
\put(1429.0,763.0){\rule[-0.200pt]{2.409pt}{0.400pt}}
\put(170.0,764.0){\rule[-0.200pt]{2.409pt}{0.400pt}}
\put(1429.0,764.0){\rule[-0.200pt]{2.409pt}{0.400pt}}
\put(170.0,764.0){\rule[-0.200pt]{2.409pt}{0.400pt}}
\put(1429.0,764.0){\rule[-0.200pt]{2.409pt}{0.400pt}}
\put(170.0,765.0){\rule[-0.200pt]{2.409pt}{0.400pt}}
\put(1429.0,765.0){\rule[-0.200pt]{2.409pt}{0.400pt}}
\put(170.0,765.0){\rule[-0.200pt]{2.409pt}{0.400pt}}
\put(1429.0,765.0){\rule[-0.200pt]{2.409pt}{0.400pt}}
\put(170.0,766.0){\rule[-0.200pt]{2.409pt}{0.400pt}}
\put(1429.0,766.0){\rule[-0.200pt]{2.409pt}{0.400pt}}
\put(170.0,766.0){\rule[-0.200pt]{2.409pt}{0.400pt}}
\put(1429.0,766.0){\rule[-0.200pt]{2.409pt}{0.400pt}}
\put(170.0,767.0){\rule[-0.200pt]{2.409pt}{0.400pt}}
\put(1429.0,767.0){\rule[-0.200pt]{2.409pt}{0.400pt}}
\put(170.0,767.0){\rule[-0.200pt]{2.409pt}{0.400pt}}
\put(1429.0,767.0){\rule[-0.200pt]{2.409pt}{0.400pt}}
\put(170.0,767.0){\rule[-0.200pt]{2.409pt}{0.400pt}}
\put(1429.0,767.0){\rule[-0.200pt]{2.409pt}{0.400pt}}
\put(170.0,768.0){\rule[-0.200pt]{2.409pt}{0.400pt}}
\put(1429.0,768.0){\rule[-0.200pt]{2.409pt}{0.400pt}}
\put(170.0,768.0){\rule[-0.200pt]{2.409pt}{0.400pt}}
\put(1429.0,768.0){\rule[-0.200pt]{2.409pt}{0.400pt}}
\put(170.0,769.0){\rule[-0.200pt]{2.409pt}{0.400pt}}
\put(1429.0,769.0){\rule[-0.200pt]{2.409pt}{0.400pt}}
\put(170.0,769.0){\rule[-0.200pt]{2.409pt}{0.400pt}}
\put(1429.0,769.0){\rule[-0.200pt]{2.409pt}{0.400pt}}
\put(170.0,770.0){\rule[-0.200pt]{2.409pt}{0.400pt}}
\put(1429.0,770.0){\rule[-0.200pt]{2.409pt}{0.400pt}}
\put(170.0,770.0){\rule[-0.200pt]{2.409pt}{0.400pt}}
\put(1429.0,770.0){\rule[-0.200pt]{2.409pt}{0.400pt}}
\put(170.0,770.0){\rule[-0.200pt]{2.409pt}{0.400pt}}
\put(1429.0,770.0){\rule[-0.200pt]{2.409pt}{0.400pt}}
\put(170.0,771.0){\rule[-0.200pt]{2.409pt}{0.400pt}}
\put(1429.0,771.0){\rule[-0.200pt]{2.409pt}{0.400pt}}
\put(170.0,771.0){\rule[-0.200pt]{2.409pt}{0.400pt}}
\put(1429.0,771.0){\rule[-0.200pt]{2.409pt}{0.400pt}}
\put(170.0,772.0){\rule[-0.200pt]{2.409pt}{0.400pt}}
\put(1429.0,772.0){\rule[-0.200pt]{2.409pt}{0.400pt}}
\put(170.0,772.0){\rule[-0.200pt]{2.409pt}{0.400pt}}
\put(1429.0,772.0){\rule[-0.200pt]{2.409pt}{0.400pt}}
\put(170.0,772.0){\rule[-0.200pt]{2.409pt}{0.400pt}}
\put(1429.0,772.0){\rule[-0.200pt]{2.409pt}{0.400pt}}
\put(170.0,773.0){\rule[-0.200pt]{2.409pt}{0.400pt}}
\put(1429.0,773.0){\rule[-0.200pt]{2.409pt}{0.400pt}}
\put(170.0,773.0){\rule[-0.200pt]{2.409pt}{0.400pt}}
\put(1429.0,773.0){\rule[-0.200pt]{2.409pt}{0.400pt}}
\put(170.0,773.0){\rule[-0.200pt]{2.409pt}{0.400pt}}
\put(1429.0,773.0){\rule[-0.200pt]{2.409pt}{0.400pt}}
\put(170.0,774.0){\rule[-0.200pt]{2.409pt}{0.400pt}}
\put(1429.0,774.0){\rule[-0.200pt]{2.409pt}{0.400pt}}
\put(170.0,774.0){\rule[-0.200pt]{2.409pt}{0.400pt}}
\put(1429.0,774.0){\rule[-0.200pt]{2.409pt}{0.400pt}}
\put(170.0,774.0){\rule[-0.200pt]{2.409pt}{0.400pt}}
\put(1429.0,774.0){\rule[-0.200pt]{2.409pt}{0.400pt}}
\put(170.0,775.0){\rule[-0.200pt]{2.409pt}{0.400pt}}
\put(1429.0,775.0){\rule[-0.200pt]{2.409pt}{0.400pt}}
\put(170.0,775.0){\rule[-0.200pt]{2.409pt}{0.400pt}}
\put(1429.0,775.0){\rule[-0.200pt]{2.409pt}{0.400pt}}
\put(170.0,776.0){\rule[-0.200pt]{2.409pt}{0.400pt}}
\put(1429.0,776.0){\rule[-0.200pt]{2.409pt}{0.400pt}}
\put(170.0,776.0){\rule[-0.200pt]{2.409pt}{0.400pt}}
\put(1429.0,776.0){\rule[-0.200pt]{2.409pt}{0.400pt}}
\put(170.0,776.0){\rule[-0.200pt]{2.409pt}{0.400pt}}
\put(1429.0,776.0){\rule[-0.200pt]{2.409pt}{0.400pt}}
\put(170.0,777.0){\rule[-0.200pt]{2.409pt}{0.400pt}}
\put(1429.0,777.0){\rule[-0.200pt]{2.409pt}{0.400pt}}
\put(170.0,777.0){\rule[-0.200pt]{2.409pt}{0.400pt}}
\put(1429.0,777.0){\rule[-0.200pt]{2.409pt}{0.400pt}}
\put(170.0,777.0){\rule[-0.200pt]{2.409pt}{0.400pt}}
\put(1429.0,777.0){\rule[-0.200pt]{2.409pt}{0.400pt}}
\put(170.0,778.0){\rule[-0.200pt]{2.409pt}{0.400pt}}
\put(1429.0,778.0){\rule[-0.200pt]{2.409pt}{0.400pt}}
\put(170.0,778.0){\rule[-0.200pt]{2.409pt}{0.400pt}}
\put(1429.0,778.0){\rule[-0.200pt]{2.409pt}{0.400pt}}
\put(170.0,778.0){\rule[-0.200pt]{2.409pt}{0.400pt}}
\put(1429.0,778.0){\rule[-0.200pt]{2.409pt}{0.400pt}}
\put(170.0,779.0){\rule[-0.200pt]{2.409pt}{0.400pt}}
\put(1429.0,779.0){\rule[-0.200pt]{2.409pt}{0.400pt}}
\put(170.0,779.0){\rule[-0.200pt]{2.409pt}{0.400pt}}
\put(1429.0,779.0){\rule[-0.200pt]{2.409pt}{0.400pt}}
\put(170.0,779.0){\rule[-0.200pt]{2.409pt}{0.400pt}}
\put(1429.0,779.0){\rule[-0.200pt]{2.409pt}{0.400pt}}
\put(170.0,780.0){\rule[-0.200pt]{2.409pt}{0.400pt}}
\put(1429.0,780.0){\rule[-0.200pt]{2.409pt}{0.400pt}}
\put(170.0,780.0){\rule[-0.200pt]{2.409pt}{0.400pt}}
\put(1429.0,780.0){\rule[-0.200pt]{2.409pt}{0.400pt}}
\put(170.0,780.0){\rule[-0.200pt]{2.409pt}{0.400pt}}
\put(1429.0,780.0){\rule[-0.200pt]{2.409pt}{0.400pt}}
\put(170.0,780.0){\rule[-0.200pt]{2.409pt}{0.400pt}}
\put(1429.0,780.0){\rule[-0.200pt]{2.409pt}{0.400pt}}
\put(170.0,781.0){\rule[-0.200pt]{2.409pt}{0.400pt}}
\put(1429.0,781.0){\rule[-0.200pt]{2.409pt}{0.400pt}}
\put(170.0,781.0){\rule[-0.200pt]{2.409pt}{0.400pt}}
\put(1429.0,781.0){\rule[-0.200pt]{2.409pt}{0.400pt}}
\put(170.0,781.0){\rule[-0.200pt]{2.409pt}{0.400pt}}
\put(1429.0,781.0){\rule[-0.200pt]{2.409pt}{0.400pt}}
\put(170.0,782.0){\rule[-0.200pt]{2.409pt}{0.400pt}}
\put(1429.0,782.0){\rule[-0.200pt]{2.409pt}{0.400pt}}
\put(170.0,782.0){\rule[-0.200pt]{2.409pt}{0.400pt}}
\put(1429.0,782.0){\rule[-0.200pt]{2.409pt}{0.400pt}}
\put(170.0,782.0){\rule[-0.200pt]{2.409pt}{0.400pt}}
\put(1429.0,782.0){\rule[-0.200pt]{2.409pt}{0.400pt}}
\put(170.0,783.0){\rule[-0.200pt]{2.409pt}{0.400pt}}
\put(1429.0,783.0){\rule[-0.200pt]{2.409pt}{0.400pt}}
\put(170.0,783.0){\rule[-0.200pt]{2.409pt}{0.400pt}}
\put(1429.0,783.0){\rule[-0.200pt]{2.409pt}{0.400pt}}
\put(170.0,783.0){\rule[-0.200pt]{2.409pt}{0.400pt}}
\put(1429.0,783.0){\rule[-0.200pt]{2.409pt}{0.400pt}}
\put(170.0,783.0){\rule[-0.200pt]{2.409pt}{0.400pt}}
\put(1429.0,783.0){\rule[-0.200pt]{2.409pt}{0.400pt}}
\put(170.0,784.0){\rule[-0.200pt]{2.409pt}{0.400pt}}
\put(1429.0,784.0){\rule[-0.200pt]{2.409pt}{0.400pt}}
\put(170.0,784.0){\rule[-0.200pt]{2.409pt}{0.400pt}}
\put(1429.0,784.0){\rule[-0.200pt]{2.409pt}{0.400pt}}
\put(170.0,784.0){\rule[-0.200pt]{2.409pt}{0.400pt}}
\put(1429.0,784.0){\rule[-0.200pt]{2.409pt}{0.400pt}}
\put(170.0,784.0){\rule[-0.200pt]{2.409pt}{0.400pt}}
\put(1429.0,784.0){\rule[-0.200pt]{2.409pt}{0.400pt}}
\put(170.0,785.0){\rule[-0.200pt]{2.409pt}{0.400pt}}
\put(1429.0,785.0){\rule[-0.200pt]{2.409pt}{0.400pt}}
\put(170.0,785.0){\rule[-0.200pt]{2.409pt}{0.400pt}}
\put(1429.0,785.0){\rule[-0.200pt]{2.409pt}{0.400pt}}
\put(170.0,785.0){\rule[-0.200pt]{2.409pt}{0.400pt}}
\put(1429.0,785.0){\rule[-0.200pt]{2.409pt}{0.400pt}}
\put(170.0,786.0){\rule[-0.200pt]{2.409pt}{0.400pt}}
\put(1429.0,786.0){\rule[-0.200pt]{2.409pt}{0.400pt}}
\put(170.0,786.0){\rule[-0.200pt]{2.409pt}{0.400pt}}
\put(1429.0,786.0){\rule[-0.200pt]{2.409pt}{0.400pt}}
\put(170.0,786.0){\rule[-0.200pt]{2.409pt}{0.400pt}}
\put(1429.0,786.0){\rule[-0.200pt]{2.409pt}{0.400pt}}
\put(170.0,786.0){\rule[-0.200pt]{2.409pt}{0.400pt}}
\put(1429.0,786.0){\rule[-0.200pt]{2.409pt}{0.400pt}}
\put(170.0,787.0){\rule[-0.200pt]{2.409pt}{0.400pt}}
\put(1429.0,787.0){\rule[-0.200pt]{2.409pt}{0.400pt}}
\put(170.0,787.0){\rule[-0.200pt]{2.409pt}{0.400pt}}
\put(1429.0,787.0){\rule[-0.200pt]{2.409pt}{0.400pt}}
\put(170.0,787.0){\rule[-0.200pt]{2.409pt}{0.400pt}}
\put(1429.0,787.0){\rule[-0.200pt]{2.409pt}{0.400pt}}
\put(170.0,787.0){\rule[-0.200pt]{2.409pt}{0.400pt}}
\put(1429.0,787.0){\rule[-0.200pt]{2.409pt}{0.400pt}}
\put(170.0,788.0){\rule[-0.200pt]{2.409pt}{0.400pt}}
\put(1429.0,788.0){\rule[-0.200pt]{2.409pt}{0.400pt}}
\put(170.0,788.0){\rule[-0.200pt]{2.409pt}{0.400pt}}
\put(1429.0,788.0){\rule[-0.200pt]{2.409pt}{0.400pt}}
\put(170.0,788.0){\rule[-0.200pt]{2.409pt}{0.400pt}}
\put(1429.0,788.0){\rule[-0.200pt]{2.409pt}{0.400pt}}
\put(170.0,788.0){\rule[-0.200pt]{2.409pt}{0.400pt}}
\put(1429.0,788.0){\rule[-0.200pt]{2.409pt}{0.400pt}}
\put(170.0,789.0){\rule[-0.200pt]{2.409pt}{0.400pt}}
\put(1429.0,789.0){\rule[-0.200pt]{2.409pt}{0.400pt}}
\put(170.0,789.0){\rule[-0.200pt]{2.409pt}{0.400pt}}
\put(1429.0,789.0){\rule[-0.200pt]{2.409pt}{0.400pt}}
\put(170.0,789.0){\rule[-0.200pt]{2.409pt}{0.400pt}}
\put(1429.0,789.0){\rule[-0.200pt]{2.409pt}{0.400pt}}
\put(170.0,789.0){\rule[-0.200pt]{2.409pt}{0.400pt}}
\put(1429.0,789.0){\rule[-0.200pt]{2.409pt}{0.400pt}}
\put(170.0,790.0){\rule[-0.200pt]{2.409pt}{0.400pt}}
\put(1429.0,790.0){\rule[-0.200pt]{2.409pt}{0.400pt}}
\put(170.0,790.0){\rule[-0.200pt]{2.409pt}{0.400pt}}
\put(1429.0,790.0){\rule[-0.200pt]{2.409pt}{0.400pt}}
\put(170.0,790.0){\rule[-0.200pt]{2.409pt}{0.400pt}}
\put(1429.0,790.0){\rule[-0.200pt]{2.409pt}{0.400pt}}
\put(170.0,790.0){\rule[-0.200pt]{2.409pt}{0.400pt}}
\put(1429.0,790.0){\rule[-0.200pt]{2.409pt}{0.400pt}}
\put(170.0,791.0){\rule[-0.200pt]{2.409pt}{0.400pt}}
\put(1429.0,791.0){\rule[-0.200pt]{2.409pt}{0.400pt}}
\put(170.0,791.0){\rule[-0.200pt]{2.409pt}{0.400pt}}
\put(1429.0,791.0){\rule[-0.200pt]{2.409pt}{0.400pt}}
\put(170.0,791.0){\rule[-0.200pt]{2.409pt}{0.400pt}}
\put(1429.0,791.0){\rule[-0.200pt]{2.409pt}{0.400pt}}
\put(170.0,791.0){\rule[-0.200pt]{2.409pt}{0.400pt}}
\put(1429.0,791.0){\rule[-0.200pt]{2.409pt}{0.400pt}}
\put(170.0,791.0){\rule[-0.200pt]{2.409pt}{0.400pt}}
\put(1429.0,791.0){\rule[-0.200pt]{2.409pt}{0.400pt}}
\put(170.0,792.0){\rule[-0.200pt]{2.409pt}{0.400pt}}
\put(1429.0,792.0){\rule[-0.200pt]{2.409pt}{0.400pt}}
\put(170.0,792.0){\rule[-0.200pt]{2.409pt}{0.400pt}}
\put(1429.0,792.0){\rule[-0.200pt]{2.409pt}{0.400pt}}
\put(170.0,792.0){\rule[-0.200pt]{2.409pt}{0.400pt}}
\put(1429.0,792.0){\rule[-0.200pt]{2.409pt}{0.400pt}}
\put(170.0,792.0){\rule[-0.200pt]{2.409pt}{0.400pt}}
\put(1429.0,792.0){\rule[-0.200pt]{2.409pt}{0.400pt}}
\put(170.0,793.0){\rule[-0.200pt]{2.409pt}{0.400pt}}
\put(1429.0,793.0){\rule[-0.200pt]{2.409pt}{0.400pt}}
\put(170.0,793.0){\rule[-0.200pt]{2.409pt}{0.400pt}}
\put(1429.0,793.0){\rule[-0.200pt]{2.409pt}{0.400pt}}
\put(170.0,793.0){\rule[-0.200pt]{2.409pt}{0.400pt}}
\put(1429.0,793.0){\rule[-0.200pt]{2.409pt}{0.400pt}}
\put(170.0,793.0){\rule[-0.200pt]{2.409pt}{0.400pt}}
\put(1429.0,793.0){\rule[-0.200pt]{2.409pt}{0.400pt}}
\put(170.0,793.0){\rule[-0.200pt]{2.409pt}{0.400pt}}
\put(1429.0,793.0){\rule[-0.200pt]{2.409pt}{0.400pt}}
\put(170.0,794.0){\rule[-0.200pt]{2.409pt}{0.400pt}}
\put(1429.0,794.0){\rule[-0.200pt]{2.409pt}{0.400pt}}
\put(170.0,794.0){\rule[-0.200pt]{2.409pt}{0.400pt}}
\put(1429.0,794.0){\rule[-0.200pt]{2.409pt}{0.400pt}}
\put(170.0,794.0){\rule[-0.200pt]{2.409pt}{0.400pt}}
\put(1429.0,794.0){\rule[-0.200pt]{2.409pt}{0.400pt}}
\put(170.0,794.0){\rule[-0.200pt]{2.409pt}{0.400pt}}
\put(1429.0,794.0){\rule[-0.200pt]{2.409pt}{0.400pt}}
\put(170.0,794.0){\rule[-0.200pt]{2.409pt}{0.400pt}}
\put(1429.0,794.0){\rule[-0.200pt]{2.409pt}{0.400pt}}
\put(170.0,795.0){\rule[-0.200pt]{2.409pt}{0.400pt}}
\put(1429.0,795.0){\rule[-0.200pt]{2.409pt}{0.400pt}}
\put(170.0,795.0){\rule[-0.200pt]{2.409pt}{0.400pt}}
\put(1429.0,795.0){\rule[-0.200pt]{2.409pt}{0.400pt}}
\put(170.0,795.0){\rule[-0.200pt]{2.409pt}{0.400pt}}
\put(1429.0,795.0){\rule[-0.200pt]{2.409pt}{0.400pt}}
\put(170.0,795.0){\rule[-0.200pt]{2.409pt}{0.400pt}}
\put(1429.0,795.0){\rule[-0.200pt]{2.409pt}{0.400pt}}
\put(170.0,796.0){\rule[-0.200pt]{2.409pt}{0.400pt}}
\put(1429.0,796.0){\rule[-0.200pt]{2.409pt}{0.400pt}}
\put(170.0,796.0){\rule[-0.200pt]{2.409pt}{0.400pt}}
\put(1429.0,796.0){\rule[-0.200pt]{2.409pt}{0.400pt}}
\put(170.0,796.0){\rule[-0.200pt]{2.409pt}{0.400pt}}
\put(1429.0,796.0){\rule[-0.200pt]{2.409pt}{0.400pt}}
\put(170.0,796.0){\rule[-0.200pt]{2.409pt}{0.400pt}}
\put(1429.0,796.0){\rule[-0.200pt]{2.409pt}{0.400pt}}
\put(170.0,796.0){\rule[-0.200pt]{2.409pt}{0.400pt}}
\put(1429.0,796.0){\rule[-0.200pt]{2.409pt}{0.400pt}}
\put(170.0,797.0){\rule[-0.200pt]{2.409pt}{0.400pt}}
\put(1429.0,797.0){\rule[-0.200pt]{2.409pt}{0.400pt}}
\put(170.0,797.0){\rule[-0.200pt]{2.409pt}{0.400pt}}
\put(1429.0,797.0){\rule[-0.200pt]{2.409pt}{0.400pt}}
\put(170.0,797.0){\rule[-0.200pt]{2.409pt}{0.400pt}}
\put(1429.0,797.0){\rule[-0.200pt]{2.409pt}{0.400pt}}
\put(170.0,797.0){\rule[-0.200pt]{2.409pt}{0.400pt}}
\put(1429.0,797.0){\rule[-0.200pt]{2.409pt}{0.400pt}}
\put(170.0,797.0){\rule[-0.200pt]{2.409pt}{0.400pt}}
\put(1429.0,797.0){\rule[-0.200pt]{2.409pt}{0.400pt}}
\put(170.0,798.0){\rule[-0.200pt]{2.409pt}{0.400pt}}
\put(1429.0,798.0){\rule[-0.200pt]{2.409pt}{0.400pt}}
\put(170.0,798.0){\rule[-0.200pt]{2.409pt}{0.400pt}}
\put(1429.0,798.0){\rule[-0.200pt]{2.409pt}{0.400pt}}
\put(170.0,798.0){\rule[-0.200pt]{2.409pt}{0.400pt}}
\put(1429.0,798.0){\rule[-0.200pt]{2.409pt}{0.400pt}}
\put(170.0,798.0){\rule[-0.200pt]{2.409pt}{0.400pt}}
\put(1429.0,798.0){\rule[-0.200pt]{2.409pt}{0.400pt}}
\put(170.0,798.0){\rule[-0.200pt]{2.409pt}{0.400pt}}
\put(1429.0,798.0){\rule[-0.200pt]{2.409pt}{0.400pt}}
\put(170.0,798.0){\rule[-0.200pt]{2.409pt}{0.400pt}}
\put(1429.0,798.0){\rule[-0.200pt]{2.409pt}{0.400pt}}
\put(170.0,799.0){\rule[-0.200pt]{2.409pt}{0.400pt}}
\put(1429.0,799.0){\rule[-0.200pt]{2.409pt}{0.400pt}}
\put(170.0,799.0){\rule[-0.200pt]{2.409pt}{0.400pt}}
\put(1429.0,799.0){\rule[-0.200pt]{2.409pt}{0.400pt}}
\put(170.0,799.0){\rule[-0.200pt]{2.409pt}{0.400pt}}
\put(1429.0,799.0){\rule[-0.200pt]{2.409pt}{0.400pt}}
\put(170.0,799.0){\rule[-0.200pt]{2.409pt}{0.400pt}}
\put(1429.0,799.0){\rule[-0.200pt]{2.409pt}{0.400pt}}
\put(170.0,799.0){\rule[-0.200pt]{2.409pt}{0.400pt}}
\put(1429.0,799.0){\rule[-0.200pt]{2.409pt}{0.400pt}}
\put(170.0,800.0){\rule[-0.200pt]{2.409pt}{0.400pt}}
\put(1429.0,800.0){\rule[-0.200pt]{2.409pt}{0.400pt}}
\put(170.0,800.0){\rule[-0.200pt]{2.409pt}{0.400pt}}
\put(1429.0,800.0){\rule[-0.200pt]{2.409pt}{0.400pt}}
\put(170.0,800.0){\rule[-0.200pt]{2.409pt}{0.400pt}}
\put(1429.0,800.0){\rule[-0.200pt]{2.409pt}{0.400pt}}
\put(170.0,800.0){\rule[-0.200pt]{2.409pt}{0.400pt}}
\put(1429.0,800.0){\rule[-0.200pt]{2.409pt}{0.400pt}}
\put(170.0,800.0){\rule[-0.200pt]{2.409pt}{0.400pt}}
\put(1429.0,800.0){\rule[-0.200pt]{2.409pt}{0.400pt}}
\put(170.0,800.0){\rule[-0.200pt]{2.409pt}{0.400pt}}
\put(1429.0,800.0){\rule[-0.200pt]{2.409pt}{0.400pt}}
\put(170.0,801.0){\rule[-0.200pt]{2.409pt}{0.400pt}}
\put(1429.0,801.0){\rule[-0.200pt]{2.409pt}{0.400pt}}
\put(170.0,801.0){\rule[-0.200pt]{2.409pt}{0.400pt}}
\put(1429.0,801.0){\rule[-0.200pt]{2.409pt}{0.400pt}}
\put(170.0,801.0){\rule[-0.200pt]{2.409pt}{0.400pt}}
\put(1429.0,801.0){\rule[-0.200pt]{2.409pt}{0.400pt}}
\put(170.0,801.0){\rule[-0.200pt]{2.409pt}{0.400pt}}
\put(1429.0,801.0){\rule[-0.200pt]{2.409pt}{0.400pt}}
\put(170.0,801.0){\rule[-0.200pt]{2.409pt}{0.400pt}}
\put(1429.0,801.0){\rule[-0.200pt]{2.409pt}{0.400pt}}
\put(170.0,802.0){\rule[-0.200pt]{2.409pt}{0.400pt}}
\put(1429.0,802.0){\rule[-0.200pt]{2.409pt}{0.400pt}}
\put(170.0,802.0){\rule[-0.200pt]{2.409pt}{0.400pt}}
\put(1429.0,802.0){\rule[-0.200pt]{2.409pt}{0.400pt}}
\put(170.0,802.0){\rule[-0.200pt]{2.409pt}{0.400pt}}
\put(1429.0,802.0){\rule[-0.200pt]{2.409pt}{0.400pt}}
\put(170.0,802.0){\rule[-0.200pt]{2.409pt}{0.400pt}}
\put(1429.0,802.0){\rule[-0.200pt]{2.409pt}{0.400pt}}
\put(170.0,802.0){\rule[-0.200pt]{2.409pt}{0.400pt}}
\put(1429.0,802.0){\rule[-0.200pt]{2.409pt}{0.400pt}}
\put(170.0,802.0){\rule[-0.200pt]{2.409pt}{0.400pt}}
\put(1429.0,802.0){\rule[-0.200pt]{2.409pt}{0.400pt}}
\put(170.0,803.0){\rule[-0.200pt]{2.409pt}{0.400pt}}
\put(1429.0,803.0){\rule[-0.200pt]{2.409pt}{0.400pt}}
\put(170.0,803.0){\rule[-0.200pt]{2.409pt}{0.400pt}}
\put(1429.0,803.0){\rule[-0.200pt]{2.409pt}{0.400pt}}
\put(170.0,803.0){\rule[-0.200pt]{2.409pt}{0.400pt}}
\put(1429.0,803.0){\rule[-0.200pt]{2.409pt}{0.400pt}}
\put(170.0,803.0){\rule[-0.200pt]{2.409pt}{0.400pt}}
\put(1429.0,803.0){\rule[-0.200pt]{2.409pt}{0.400pt}}
\put(170.0,803.0){\rule[-0.200pt]{2.409pt}{0.400pt}}
\put(1429.0,803.0){\rule[-0.200pt]{2.409pt}{0.400pt}}
\put(170.0,803.0){\rule[-0.200pt]{2.409pt}{0.400pt}}
\put(1429.0,803.0){\rule[-0.200pt]{2.409pt}{0.400pt}}
\put(170.0,804.0){\rule[-0.200pt]{2.409pt}{0.400pt}}
\put(1429.0,804.0){\rule[-0.200pt]{2.409pt}{0.400pt}}
\put(170.0,804.0){\rule[-0.200pt]{2.409pt}{0.400pt}}
\put(1429.0,804.0){\rule[-0.200pt]{2.409pt}{0.400pt}}
\put(170.0,804.0){\rule[-0.200pt]{2.409pt}{0.400pt}}
\put(1429.0,804.0){\rule[-0.200pt]{2.409pt}{0.400pt}}
\put(170.0,804.0){\rule[-0.200pt]{2.409pt}{0.400pt}}
\put(1429.0,804.0){\rule[-0.200pt]{2.409pt}{0.400pt}}
\put(170.0,804.0){\rule[-0.200pt]{2.409pt}{0.400pt}}
\put(1429.0,804.0){\rule[-0.200pt]{2.409pt}{0.400pt}}
\put(170.0,804.0){\rule[-0.200pt]{2.409pt}{0.400pt}}
\put(1429.0,804.0){\rule[-0.200pt]{2.409pt}{0.400pt}}
\put(170.0,805.0){\rule[-0.200pt]{2.409pt}{0.400pt}}
\put(1429.0,805.0){\rule[-0.200pt]{2.409pt}{0.400pt}}
\put(170.0,805.0){\rule[-0.200pt]{2.409pt}{0.400pt}}
\put(1429.0,805.0){\rule[-0.200pt]{2.409pt}{0.400pt}}
\put(170.0,805.0){\rule[-0.200pt]{2.409pt}{0.400pt}}
\put(1429.0,805.0){\rule[-0.200pt]{2.409pt}{0.400pt}}
\put(170.0,805.0){\rule[-0.200pt]{2.409pt}{0.400pt}}
\put(1429.0,805.0){\rule[-0.200pt]{2.409pt}{0.400pt}}
\put(170.0,805.0){\rule[-0.200pt]{2.409pt}{0.400pt}}
\put(1429.0,805.0){\rule[-0.200pt]{2.409pt}{0.400pt}}
\put(170.0,805.0){\rule[-0.200pt]{2.409pt}{0.400pt}}
\put(1429.0,805.0){\rule[-0.200pt]{2.409pt}{0.400pt}}
\put(170.0,805.0){\rule[-0.200pt]{2.409pt}{0.400pt}}
\put(1429.0,805.0){\rule[-0.200pt]{2.409pt}{0.400pt}}
\put(170.0,806.0){\rule[-0.200pt]{2.409pt}{0.400pt}}
\put(1429.0,806.0){\rule[-0.200pt]{2.409pt}{0.400pt}}
\put(170.0,806.0){\rule[-0.200pt]{2.409pt}{0.400pt}}
\put(1429.0,806.0){\rule[-0.200pt]{2.409pt}{0.400pt}}
\put(170.0,806.0){\rule[-0.200pt]{2.409pt}{0.400pt}}
\put(1429.0,806.0){\rule[-0.200pt]{2.409pt}{0.400pt}}
\put(170.0,806.0){\rule[-0.200pt]{2.409pt}{0.400pt}}
\put(1429.0,806.0){\rule[-0.200pt]{2.409pt}{0.400pt}}
\put(170.0,806.0){\rule[-0.200pt]{2.409pt}{0.400pt}}
\put(1429.0,806.0){\rule[-0.200pt]{2.409pt}{0.400pt}}
\put(170.0,806.0){\rule[-0.200pt]{2.409pt}{0.400pt}}
\put(1429.0,806.0){\rule[-0.200pt]{2.409pt}{0.400pt}}
\put(170.0,807.0){\rule[-0.200pt]{2.409pt}{0.400pt}}
\put(1429.0,807.0){\rule[-0.200pt]{2.409pt}{0.400pt}}
\put(170.0,807.0){\rule[-0.200pt]{2.409pt}{0.400pt}}
\put(1429.0,807.0){\rule[-0.200pt]{2.409pt}{0.400pt}}
\put(170.0,807.0){\rule[-0.200pt]{2.409pt}{0.400pt}}
\put(1429.0,807.0){\rule[-0.200pt]{2.409pt}{0.400pt}}
\put(170.0,807.0){\rule[-0.200pt]{2.409pt}{0.400pt}}
\put(1429.0,807.0){\rule[-0.200pt]{2.409pt}{0.400pt}}
\put(170.0,807.0){\rule[-0.200pt]{2.409pt}{0.400pt}}
\put(1429.0,807.0){\rule[-0.200pt]{2.409pt}{0.400pt}}
\put(170.0,807.0){\rule[-0.200pt]{2.409pt}{0.400pt}}
\put(1429.0,807.0){\rule[-0.200pt]{2.409pt}{0.400pt}}
\put(170.0,807.0){\rule[-0.200pt]{2.409pt}{0.400pt}}
\put(1429.0,807.0){\rule[-0.200pt]{2.409pt}{0.400pt}}
\put(170.0,808.0){\rule[-0.200pt]{2.409pt}{0.400pt}}
\put(1429.0,808.0){\rule[-0.200pt]{2.409pt}{0.400pt}}
\put(170.0,808.0){\rule[-0.200pt]{2.409pt}{0.400pt}}
\put(1429.0,808.0){\rule[-0.200pt]{2.409pt}{0.400pt}}
\put(170.0,808.0){\rule[-0.200pt]{2.409pt}{0.400pt}}
\put(1429.0,808.0){\rule[-0.200pt]{2.409pt}{0.400pt}}
\put(170.0,808.0){\rule[-0.200pt]{2.409pt}{0.400pt}}
\put(1429.0,808.0){\rule[-0.200pt]{2.409pt}{0.400pt}}
\put(170.0,808.0){\rule[-0.200pt]{2.409pt}{0.400pt}}
\put(1429.0,808.0){\rule[-0.200pt]{2.409pt}{0.400pt}}
\put(170.0,808.0){\rule[-0.200pt]{2.409pt}{0.400pt}}
\put(1429.0,808.0){\rule[-0.200pt]{2.409pt}{0.400pt}}
\put(170.0,808.0){\rule[-0.200pt]{2.409pt}{0.400pt}}
\put(1429.0,808.0){\rule[-0.200pt]{2.409pt}{0.400pt}}
\put(170.0,809.0){\rule[-0.200pt]{2.409pt}{0.400pt}}
\put(1429.0,809.0){\rule[-0.200pt]{2.409pt}{0.400pt}}
\put(170.0,809.0){\rule[-0.200pt]{2.409pt}{0.400pt}}
\put(1429.0,809.0){\rule[-0.200pt]{2.409pt}{0.400pt}}
\put(170.0,809.0){\rule[-0.200pt]{2.409pt}{0.400pt}}
\put(1429.0,809.0){\rule[-0.200pt]{2.409pt}{0.400pt}}
\put(170.0,809.0){\rule[-0.200pt]{2.409pt}{0.400pt}}
\put(1429.0,809.0){\rule[-0.200pt]{2.409pt}{0.400pt}}
\put(170.0,809.0){\rule[-0.200pt]{2.409pt}{0.400pt}}
\put(1429.0,809.0){\rule[-0.200pt]{2.409pt}{0.400pt}}
\put(170.0,809.0){\rule[-0.200pt]{2.409pt}{0.400pt}}
\put(1429.0,809.0){\rule[-0.200pt]{2.409pt}{0.400pt}}
\put(170.0,809.0){\rule[-0.200pt]{2.409pt}{0.400pt}}
\put(1429.0,809.0){\rule[-0.200pt]{2.409pt}{0.400pt}}
\put(170.0,810.0){\rule[-0.200pt]{2.409pt}{0.400pt}}
\put(1429.0,810.0){\rule[-0.200pt]{2.409pt}{0.400pt}}
\put(170.0,810.0){\rule[-0.200pt]{2.409pt}{0.400pt}}
\put(1429.0,810.0){\rule[-0.200pt]{2.409pt}{0.400pt}}
\put(170.0,810.0){\rule[-0.200pt]{2.409pt}{0.400pt}}
\put(1429.0,810.0){\rule[-0.200pt]{2.409pt}{0.400pt}}
\put(170.0,810.0){\rule[-0.200pt]{2.409pt}{0.400pt}}
\put(1429.0,810.0){\rule[-0.200pt]{2.409pt}{0.400pt}}
\put(170.0,810.0){\rule[-0.200pt]{2.409pt}{0.400pt}}
\put(1429.0,810.0){\rule[-0.200pt]{2.409pt}{0.400pt}}
\put(170.0,810.0){\rule[-0.200pt]{2.409pt}{0.400pt}}
\put(1429.0,810.0){\rule[-0.200pt]{2.409pt}{0.400pt}}
\put(170.0,810.0){\rule[-0.200pt]{2.409pt}{0.400pt}}
\put(1429.0,810.0){\rule[-0.200pt]{2.409pt}{0.400pt}}
\put(170.0,811.0){\rule[-0.200pt]{2.409pt}{0.400pt}}
\put(1429.0,811.0){\rule[-0.200pt]{2.409pt}{0.400pt}}
\put(170.0,811.0){\rule[-0.200pt]{2.409pt}{0.400pt}}
\put(1429.0,811.0){\rule[-0.200pt]{2.409pt}{0.400pt}}
\put(170.0,811.0){\rule[-0.200pt]{2.409pt}{0.400pt}}
\put(1429.0,811.0){\rule[-0.200pt]{2.409pt}{0.400pt}}
\put(170.0,811.0){\rule[-0.200pt]{2.409pt}{0.400pt}}
\put(1429.0,811.0){\rule[-0.200pt]{2.409pt}{0.400pt}}
\put(170.0,811.0){\rule[-0.200pt]{2.409pt}{0.400pt}}
\put(1429.0,811.0){\rule[-0.200pt]{2.409pt}{0.400pt}}
\put(170.0,811.0){\rule[-0.200pt]{2.409pt}{0.400pt}}
\put(1429.0,811.0){\rule[-0.200pt]{2.409pt}{0.400pt}}
\put(170.0,811.0){\rule[-0.200pt]{2.409pt}{0.400pt}}
\put(1429.0,811.0){\rule[-0.200pt]{2.409pt}{0.400pt}}
\put(170.0,812.0){\rule[-0.200pt]{2.409pt}{0.400pt}}
\put(1429.0,812.0){\rule[-0.200pt]{2.409pt}{0.400pt}}
\put(170.0,812.0){\rule[-0.200pt]{2.409pt}{0.400pt}}
\put(1429.0,812.0){\rule[-0.200pt]{2.409pt}{0.400pt}}
\put(170.0,812.0){\rule[-0.200pt]{2.409pt}{0.400pt}}
\put(1429.0,812.0){\rule[-0.200pt]{2.409pt}{0.400pt}}
\put(170.0,812.0){\rule[-0.200pt]{2.409pt}{0.400pt}}
\put(1429.0,812.0){\rule[-0.200pt]{2.409pt}{0.400pt}}
\put(170.0,812.0){\rule[-0.200pt]{2.409pt}{0.400pt}}
\put(1429.0,812.0){\rule[-0.200pt]{2.409pt}{0.400pt}}
\put(170.0,812.0){\rule[-0.200pt]{2.409pt}{0.400pt}}
\put(1429.0,812.0){\rule[-0.200pt]{2.409pt}{0.400pt}}
\put(170.0,812.0){\rule[-0.200pt]{2.409pt}{0.400pt}}
\put(1429.0,812.0){\rule[-0.200pt]{2.409pt}{0.400pt}}
\put(170.0,812.0){\rule[-0.200pt]{2.409pt}{0.400pt}}
\put(1429.0,812.0){\rule[-0.200pt]{2.409pt}{0.400pt}}
\put(170.0,813.0){\rule[-0.200pt]{2.409pt}{0.400pt}}
\put(1429.0,813.0){\rule[-0.200pt]{2.409pt}{0.400pt}}
\put(170.0,813.0){\rule[-0.200pt]{2.409pt}{0.400pt}}
\put(1429.0,813.0){\rule[-0.200pt]{2.409pt}{0.400pt}}
\put(170.0,813.0){\rule[-0.200pt]{2.409pt}{0.400pt}}
\put(1429.0,813.0){\rule[-0.200pt]{2.409pt}{0.400pt}}
\put(170.0,813.0){\rule[-0.200pt]{2.409pt}{0.400pt}}
\put(1429.0,813.0){\rule[-0.200pt]{2.409pt}{0.400pt}}
\put(170.0,813.0){\rule[-0.200pt]{2.409pt}{0.400pt}}
\put(1429.0,813.0){\rule[-0.200pt]{2.409pt}{0.400pt}}
\put(170.0,813.0){\rule[-0.200pt]{2.409pt}{0.400pt}}
\put(1429.0,813.0){\rule[-0.200pt]{2.409pt}{0.400pt}}
\put(170.0,813.0){\rule[-0.200pt]{2.409pt}{0.400pt}}
\put(1429.0,813.0){\rule[-0.200pt]{2.409pt}{0.400pt}}
\put(170.0,813.0){\rule[-0.200pt]{2.409pt}{0.400pt}}
\put(1429.0,813.0){\rule[-0.200pt]{2.409pt}{0.400pt}}
\put(170.0,814.0){\rule[-0.200pt]{2.409pt}{0.400pt}}
\put(1429.0,814.0){\rule[-0.200pt]{2.409pt}{0.400pt}}
\put(170.0,814.0){\rule[-0.200pt]{2.409pt}{0.400pt}}
\put(1429.0,814.0){\rule[-0.200pt]{2.409pt}{0.400pt}}
\put(170.0,814.0){\rule[-0.200pt]{2.409pt}{0.400pt}}
\put(1429.0,814.0){\rule[-0.200pt]{2.409pt}{0.400pt}}
\put(170.0,814.0){\rule[-0.200pt]{2.409pt}{0.400pt}}
\put(1429.0,814.0){\rule[-0.200pt]{2.409pt}{0.400pt}}
\put(170.0,814.0){\rule[-0.200pt]{2.409pt}{0.400pt}}
\put(1429.0,814.0){\rule[-0.200pt]{2.409pt}{0.400pt}}
\put(170.0,814.0){\rule[-0.200pt]{2.409pt}{0.400pt}}
\put(1429.0,814.0){\rule[-0.200pt]{2.409pt}{0.400pt}}
\put(170.0,814.0){\rule[-0.200pt]{2.409pt}{0.400pt}}
\put(1429.0,814.0){\rule[-0.200pt]{2.409pt}{0.400pt}}
\put(170.0,814.0){\rule[-0.200pt]{2.409pt}{0.400pt}}
\put(1429.0,814.0){\rule[-0.200pt]{2.409pt}{0.400pt}}
\put(170.0,815.0){\rule[-0.200pt]{2.409pt}{0.400pt}}
\put(1429.0,815.0){\rule[-0.200pt]{2.409pt}{0.400pt}}
\put(170.0,815.0){\rule[-0.200pt]{2.409pt}{0.400pt}}
\put(1429.0,815.0){\rule[-0.200pt]{2.409pt}{0.400pt}}
\put(170.0,815.0){\rule[-0.200pt]{2.409pt}{0.400pt}}
\put(1429.0,815.0){\rule[-0.200pt]{2.409pt}{0.400pt}}
\put(170.0,815.0){\rule[-0.200pt]{2.409pt}{0.400pt}}
\put(1429.0,815.0){\rule[-0.200pt]{2.409pt}{0.400pt}}
\put(170.0,815.0){\rule[-0.200pt]{2.409pt}{0.400pt}}
\put(1429.0,815.0){\rule[-0.200pt]{2.409pt}{0.400pt}}
\put(170.0,815.0){\rule[-0.200pt]{2.409pt}{0.400pt}}
\put(1429.0,815.0){\rule[-0.200pt]{2.409pt}{0.400pt}}
\put(170.0,815.0){\rule[-0.200pt]{2.409pt}{0.400pt}}
\put(1429.0,815.0){\rule[-0.200pt]{2.409pt}{0.400pt}}
\put(170.0,815.0){\rule[-0.200pt]{2.409pt}{0.400pt}}
\put(1429.0,815.0){\rule[-0.200pt]{2.409pt}{0.400pt}}
\put(170.0,816.0){\rule[-0.200pt]{2.409pt}{0.400pt}}
\put(1429.0,816.0){\rule[-0.200pt]{2.409pt}{0.400pt}}
\put(170.0,816.0){\rule[-0.200pt]{2.409pt}{0.400pt}}
\put(1429.0,816.0){\rule[-0.200pt]{2.409pt}{0.400pt}}
\put(170.0,816.0){\rule[-0.200pt]{2.409pt}{0.400pt}}
\put(1429.0,816.0){\rule[-0.200pt]{2.409pt}{0.400pt}}
\put(170.0,816.0){\rule[-0.200pt]{2.409pt}{0.400pt}}
\put(1429.0,816.0){\rule[-0.200pt]{2.409pt}{0.400pt}}
\put(170.0,816.0){\rule[-0.200pt]{2.409pt}{0.400pt}}
\put(1429.0,816.0){\rule[-0.200pt]{2.409pt}{0.400pt}}
\put(170.0,816.0){\rule[-0.200pt]{2.409pt}{0.400pt}}
\put(1429.0,816.0){\rule[-0.200pt]{2.409pt}{0.400pt}}
\put(170.0,816.0){\rule[-0.200pt]{2.409pt}{0.400pt}}
\put(1429.0,816.0){\rule[-0.200pt]{2.409pt}{0.400pt}}
\put(170.0,816.0){\rule[-0.200pt]{2.409pt}{0.400pt}}
\put(1429.0,816.0){\rule[-0.200pt]{2.409pt}{0.400pt}}
\put(170.0,817.0){\rule[-0.200pt]{2.409pt}{0.400pt}}
\put(1429.0,817.0){\rule[-0.200pt]{2.409pt}{0.400pt}}
\put(170.0,817.0){\rule[-0.200pt]{2.409pt}{0.400pt}}
\put(1429.0,817.0){\rule[-0.200pt]{2.409pt}{0.400pt}}
\put(170.0,817.0){\rule[-0.200pt]{2.409pt}{0.400pt}}
\put(1429.0,817.0){\rule[-0.200pt]{2.409pt}{0.400pt}}
\put(170.0,817.0){\rule[-0.200pt]{2.409pt}{0.400pt}}
\put(1429.0,817.0){\rule[-0.200pt]{2.409pt}{0.400pt}}
\put(170.0,817.0){\rule[-0.200pt]{2.409pt}{0.400pt}}
\put(1429.0,817.0){\rule[-0.200pt]{2.409pt}{0.400pt}}
\put(170.0,817.0){\rule[-0.200pt]{2.409pt}{0.400pt}}
\put(1429.0,817.0){\rule[-0.200pt]{2.409pt}{0.400pt}}
\put(170.0,817.0){\rule[-0.200pt]{2.409pt}{0.400pt}}
\put(1429.0,817.0){\rule[-0.200pt]{2.409pt}{0.400pt}}
\put(170.0,817.0){\rule[-0.200pt]{2.409pt}{0.400pt}}
\put(1429.0,817.0){\rule[-0.200pt]{2.409pt}{0.400pt}}
\put(170.0,817.0){\rule[-0.200pt]{2.409pt}{0.400pt}}
\put(1429.0,817.0){\rule[-0.200pt]{2.409pt}{0.400pt}}
\put(170.0,818.0){\rule[-0.200pt]{2.409pt}{0.400pt}}
\put(1429.0,818.0){\rule[-0.200pt]{2.409pt}{0.400pt}}
\put(170.0,818.0){\rule[-0.200pt]{2.409pt}{0.400pt}}
\put(1429.0,818.0){\rule[-0.200pt]{2.409pt}{0.400pt}}
\put(170.0,818.0){\rule[-0.200pt]{2.409pt}{0.400pt}}
\put(1429.0,818.0){\rule[-0.200pt]{2.409pt}{0.400pt}}
\put(170.0,818.0){\rule[-0.200pt]{2.409pt}{0.400pt}}
\put(1429.0,818.0){\rule[-0.200pt]{2.409pt}{0.400pt}}
\put(170.0,818.0){\rule[-0.200pt]{2.409pt}{0.400pt}}
\put(1429.0,818.0){\rule[-0.200pt]{2.409pt}{0.400pt}}
\put(170.0,818.0){\rule[-0.200pt]{2.409pt}{0.400pt}}
\put(1429.0,818.0){\rule[-0.200pt]{2.409pt}{0.400pt}}
\put(170.0,818.0){\rule[-0.200pt]{2.409pt}{0.400pt}}
\put(1429.0,818.0){\rule[-0.200pt]{2.409pt}{0.400pt}}
\put(170.0,818.0){\rule[-0.200pt]{2.409pt}{0.400pt}}
\put(1429.0,818.0){\rule[-0.200pt]{2.409pt}{0.400pt}}
\put(170.0,818.0){\rule[-0.200pt]{2.409pt}{0.400pt}}
\put(1429.0,818.0){\rule[-0.200pt]{2.409pt}{0.400pt}}
\put(170.0,819.0){\rule[-0.200pt]{2.409pt}{0.400pt}}
\put(1429.0,819.0){\rule[-0.200pt]{2.409pt}{0.400pt}}
\put(170.0,819.0){\rule[-0.200pt]{2.409pt}{0.400pt}}
\put(1429.0,819.0){\rule[-0.200pt]{2.409pt}{0.400pt}}
\put(170.0,819.0){\rule[-0.200pt]{2.409pt}{0.400pt}}
\put(1429.0,819.0){\rule[-0.200pt]{2.409pt}{0.400pt}}
\put(170.0,819.0){\rule[-0.200pt]{2.409pt}{0.400pt}}
\put(1429.0,819.0){\rule[-0.200pt]{2.409pt}{0.400pt}}
\put(170.0,819.0){\rule[-0.200pt]{2.409pt}{0.400pt}}
\put(1429.0,819.0){\rule[-0.200pt]{2.409pt}{0.400pt}}
\put(170.0,819.0){\rule[-0.200pt]{2.409pt}{0.400pt}}
\put(1429.0,819.0){\rule[-0.200pt]{2.409pt}{0.400pt}}
\put(170.0,819.0){\rule[-0.200pt]{2.409pt}{0.400pt}}
\put(1429.0,819.0){\rule[-0.200pt]{2.409pt}{0.400pt}}
\put(170.0,819.0){\rule[-0.200pt]{2.409pt}{0.400pt}}
\put(1429.0,819.0){\rule[-0.200pt]{2.409pt}{0.400pt}}
\put(170.0,819.0){\rule[-0.200pt]{2.409pt}{0.400pt}}
\put(1429.0,819.0){\rule[-0.200pt]{2.409pt}{0.400pt}}
\put(170.0,820.0){\rule[-0.200pt]{2.409pt}{0.400pt}}
\put(1429.0,820.0){\rule[-0.200pt]{2.409pt}{0.400pt}}
\put(170.0,820.0){\rule[-0.200pt]{2.409pt}{0.400pt}}
\put(1429.0,820.0){\rule[-0.200pt]{2.409pt}{0.400pt}}
\put(170.0,820.0){\rule[-0.200pt]{2.409pt}{0.400pt}}
\put(1429.0,820.0){\rule[-0.200pt]{2.409pt}{0.400pt}}
\put(170.0,820.0){\rule[-0.200pt]{2.409pt}{0.400pt}}
\put(1429.0,820.0){\rule[-0.200pt]{2.409pt}{0.400pt}}
\put(170.0,820.0){\rule[-0.200pt]{2.409pt}{0.400pt}}
\put(1429.0,820.0){\rule[-0.200pt]{2.409pt}{0.400pt}}
\put(170.0,820.0){\rule[-0.200pt]{2.409pt}{0.400pt}}
\put(1429.0,820.0){\rule[-0.200pt]{2.409pt}{0.400pt}}
\put(170.0,820.0){\rule[-0.200pt]{2.409pt}{0.400pt}}
\put(1429.0,820.0){\rule[-0.200pt]{2.409pt}{0.400pt}}
\put(170.0,820.0){\rule[-0.200pt]{2.409pt}{0.400pt}}
\put(1429.0,820.0){\rule[-0.200pt]{2.409pt}{0.400pt}}
\put(170.0,820.0){\rule[-0.200pt]{2.409pt}{0.400pt}}
\put(1429.0,820.0){\rule[-0.200pt]{2.409pt}{0.400pt}}
\put(170.0,820.0){\rule[-0.200pt]{2.409pt}{0.400pt}}
\put(1429.0,820.0){\rule[-0.200pt]{2.409pt}{0.400pt}}
\put(170.0,821.0){\rule[-0.200pt]{2.409pt}{0.400pt}}
\put(1429.0,821.0){\rule[-0.200pt]{2.409pt}{0.400pt}}
\put(170.0,821.0){\rule[-0.200pt]{2.409pt}{0.400pt}}
\put(1429.0,821.0){\rule[-0.200pt]{2.409pt}{0.400pt}}
\put(170.0,821.0){\rule[-0.200pt]{2.409pt}{0.400pt}}
\put(1429.0,821.0){\rule[-0.200pt]{2.409pt}{0.400pt}}
\put(170.0,821.0){\rule[-0.200pt]{2.409pt}{0.400pt}}
\put(1429.0,821.0){\rule[-0.200pt]{2.409pt}{0.400pt}}
\put(170.0,821.0){\rule[-0.200pt]{2.409pt}{0.400pt}}
\put(1429.0,821.0){\rule[-0.200pt]{2.409pt}{0.400pt}}
\put(170.0,821.0){\rule[-0.200pt]{2.409pt}{0.400pt}}
\put(1429.0,821.0){\rule[-0.200pt]{2.409pt}{0.400pt}}
\put(170.0,821.0){\rule[-0.200pt]{2.409pt}{0.400pt}}
\put(1429.0,821.0){\rule[-0.200pt]{2.409pt}{0.400pt}}
\put(170.0,821.0){\rule[-0.200pt]{2.409pt}{0.400pt}}
\put(1429.0,821.0){\rule[-0.200pt]{2.409pt}{0.400pt}}
\put(170.0,821.0){\rule[-0.200pt]{2.409pt}{0.400pt}}
\put(1429.0,821.0){\rule[-0.200pt]{2.409pt}{0.400pt}}
\put(170.0,822.0){\rule[-0.200pt]{2.409pt}{0.400pt}}
\put(1429.0,822.0){\rule[-0.200pt]{2.409pt}{0.400pt}}
\put(170.0,822.0){\rule[-0.200pt]{2.409pt}{0.400pt}}
\put(1429.0,822.0){\rule[-0.200pt]{2.409pt}{0.400pt}}
\put(170.0,822.0){\rule[-0.200pt]{2.409pt}{0.400pt}}
\put(1429.0,822.0){\rule[-0.200pt]{2.409pt}{0.400pt}}
\put(170.0,822.0){\rule[-0.200pt]{2.409pt}{0.400pt}}
\put(1429.0,822.0){\rule[-0.200pt]{2.409pt}{0.400pt}}
\put(170.0,822.0){\rule[-0.200pt]{2.409pt}{0.400pt}}
\put(1429.0,822.0){\rule[-0.200pt]{2.409pt}{0.400pt}}
\put(170.0,822.0){\rule[-0.200pt]{2.409pt}{0.400pt}}
\put(1429.0,822.0){\rule[-0.200pt]{2.409pt}{0.400pt}}
\put(170.0,822.0){\rule[-0.200pt]{2.409pt}{0.400pt}}
\put(1429.0,822.0){\rule[-0.200pt]{2.409pt}{0.400pt}}
\put(170.0,822.0){\rule[-0.200pt]{2.409pt}{0.400pt}}
\put(1429.0,822.0){\rule[-0.200pt]{2.409pt}{0.400pt}}
\put(170.0,822.0){\rule[-0.200pt]{2.409pt}{0.400pt}}
\put(1429.0,822.0){\rule[-0.200pt]{2.409pt}{0.400pt}}
\put(170.0,822.0){\rule[-0.200pt]{2.409pt}{0.400pt}}
\put(1429.0,822.0){\rule[-0.200pt]{2.409pt}{0.400pt}}
\put(170.0,823.0){\rule[-0.200pt]{2.409pt}{0.400pt}}
\put(1429.0,823.0){\rule[-0.200pt]{2.409pt}{0.400pt}}
\put(170.0,823.0){\rule[-0.200pt]{2.409pt}{0.400pt}}
\put(1429.0,823.0){\rule[-0.200pt]{2.409pt}{0.400pt}}
\put(170.0,823.0){\rule[-0.200pt]{2.409pt}{0.400pt}}
\put(1429.0,823.0){\rule[-0.200pt]{2.409pt}{0.400pt}}
\put(170.0,823.0){\rule[-0.200pt]{2.409pt}{0.400pt}}
\put(1429.0,823.0){\rule[-0.200pt]{2.409pt}{0.400pt}}
\put(170.0,823.0){\rule[-0.200pt]{2.409pt}{0.400pt}}
\put(1429.0,823.0){\rule[-0.200pt]{2.409pt}{0.400pt}}
\put(170.0,823.0){\rule[-0.200pt]{2.409pt}{0.400pt}}
\put(1429.0,823.0){\rule[-0.200pt]{2.409pt}{0.400pt}}
\put(170.0,823.0){\rule[-0.200pt]{2.409pt}{0.400pt}}
\put(1429.0,823.0){\rule[-0.200pt]{2.409pt}{0.400pt}}
\put(170.0,823.0){\rule[-0.200pt]{2.409pt}{0.400pt}}
\put(1429.0,823.0){\rule[-0.200pt]{2.409pt}{0.400pt}}
\put(170.0,823.0){\rule[-0.200pt]{2.409pt}{0.400pt}}
\put(1429.0,823.0){\rule[-0.200pt]{2.409pt}{0.400pt}}
\put(170.0,823.0){\rule[-0.200pt]{2.409pt}{0.400pt}}
\put(1429.0,823.0){\rule[-0.200pt]{2.409pt}{0.400pt}}
\put(170.0,824.0){\rule[-0.200pt]{2.409pt}{0.400pt}}
\put(1429.0,824.0){\rule[-0.200pt]{2.409pt}{0.400pt}}
\put(170.0,824.0){\rule[-0.200pt]{2.409pt}{0.400pt}}
\put(1429.0,824.0){\rule[-0.200pt]{2.409pt}{0.400pt}}
\put(170.0,824.0){\rule[-0.200pt]{2.409pt}{0.400pt}}
\put(1429.0,824.0){\rule[-0.200pt]{2.409pt}{0.400pt}}
\put(170.0,824.0){\rule[-0.200pt]{2.409pt}{0.400pt}}
\put(1429.0,824.0){\rule[-0.200pt]{2.409pt}{0.400pt}}
\put(170.0,824.0){\rule[-0.200pt]{2.409pt}{0.400pt}}
\put(1429.0,824.0){\rule[-0.200pt]{2.409pt}{0.400pt}}
\put(170.0,824.0){\rule[-0.200pt]{2.409pt}{0.400pt}}
\put(1429.0,824.0){\rule[-0.200pt]{2.409pt}{0.400pt}}
\put(170.0,824.0){\rule[-0.200pt]{2.409pt}{0.400pt}}
\put(1429.0,824.0){\rule[-0.200pt]{2.409pt}{0.400pt}}
\put(170.0,824.0){\rule[-0.200pt]{2.409pt}{0.400pt}}
\put(1429.0,824.0){\rule[-0.200pt]{2.409pt}{0.400pt}}
\put(170.0,824.0){\rule[-0.200pt]{2.409pt}{0.400pt}}
\put(1429.0,824.0){\rule[-0.200pt]{2.409pt}{0.400pt}}
\put(170.0,824.0){\rule[-0.200pt]{2.409pt}{0.400pt}}
\put(1429.0,824.0){\rule[-0.200pt]{2.409pt}{0.400pt}}
\put(170.0,824.0){\rule[-0.200pt]{2.409pt}{0.400pt}}
\put(1429.0,824.0){\rule[-0.200pt]{2.409pt}{0.400pt}}
\put(170.0,825.0){\rule[-0.200pt]{2.409pt}{0.400pt}}
\put(1429.0,825.0){\rule[-0.200pt]{2.409pt}{0.400pt}}
\put(170.0,825.0){\rule[-0.200pt]{2.409pt}{0.400pt}}
\put(1429.0,825.0){\rule[-0.200pt]{2.409pt}{0.400pt}}
\put(170.0,825.0){\rule[-0.200pt]{2.409pt}{0.400pt}}
\put(1429.0,825.0){\rule[-0.200pt]{2.409pt}{0.400pt}}
\put(170.0,825.0){\rule[-0.200pt]{2.409pt}{0.400pt}}
\put(1429.0,825.0){\rule[-0.200pt]{2.409pt}{0.400pt}}
\put(170.0,825.0){\rule[-0.200pt]{2.409pt}{0.400pt}}
\put(1429.0,825.0){\rule[-0.200pt]{2.409pt}{0.400pt}}
\put(170.0,825.0){\rule[-0.200pt]{2.409pt}{0.400pt}}
\put(1429.0,825.0){\rule[-0.200pt]{2.409pt}{0.400pt}}
\put(170.0,825.0){\rule[-0.200pt]{2.409pt}{0.400pt}}
\put(1429.0,825.0){\rule[-0.200pt]{2.409pt}{0.400pt}}
\put(170.0,825.0){\rule[-0.200pt]{2.409pt}{0.400pt}}
\put(1429.0,825.0){\rule[-0.200pt]{2.409pt}{0.400pt}}
\put(170.0,825.0){\rule[-0.200pt]{2.409pt}{0.400pt}}
\put(1429.0,825.0){\rule[-0.200pt]{2.409pt}{0.400pt}}
\put(170.0,825.0){\rule[-0.200pt]{2.409pt}{0.400pt}}
\put(1429.0,825.0){\rule[-0.200pt]{2.409pt}{0.400pt}}
\put(170.0,825.0){\rule[-0.200pt]{2.409pt}{0.400pt}}
\put(1429.0,825.0){\rule[-0.200pt]{2.409pt}{0.400pt}}
\put(170.0,826.0){\rule[-0.200pt]{2.409pt}{0.400pt}}
\put(1429.0,826.0){\rule[-0.200pt]{2.409pt}{0.400pt}}
\put(170.0,826.0){\rule[-0.200pt]{2.409pt}{0.400pt}}
\put(1429.0,826.0){\rule[-0.200pt]{2.409pt}{0.400pt}}
\put(170.0,826.0){\rule[-0.200pt]{2.409pt}{0.400pt}}
\put(1429.0,826.0){\rule[-0.200pt]{2.409pt}{0.400pt}}
\put(170.0,826.0){\rule[-0.200pt]{2.409pt}{0.400pt}}
\put(1429.0,826.0){\rule[-0.200pt]{2.409pt}{0.400pt}}
\put(170.0,826.0){\rule[-0.200pt]{2.409pt}{0.400pt}}
\put(1429.0,826.0){\rule[-0.200pt]{2.409pt}{0.400pt}}
\put(170.0,826.0){\rule[-0.200pt]{2.409pt}{0.400pt}}
\put(1429.0,826.0){\rule[-0.200pt]{2.409pt}{0.400pt}}
\put(170.0,826.0){\rule[-0.200pt]{2.409pt}{0.400pt}}
\put(1429.0,826.0){\rule[-0.200pt]{2.409pt}{0.400pt}}
\put(170.0,826.0){\rule[-0.200pt]{2.409pt}{0.400pt}}
\put(1429.0,826.0){\rule[-0.200pt]{2.409pt}{0.400pt}}
\put(170.0,826.0){\rule[-0.200pt]{2.409pt}{0.400pt}}
\put(1429.0,826.0){\rule[-0.200pt]{2.409pt}{0.400pt}}
\put(170.0,826.0){\rule[-0.200pt]{2.409pt}{0.400pt}}
\put(1429.0,826.0){\rule[-0.200pt]{2.409pt}{0.400pt}}
\put(170.0,826.0){\rule[-0.200pt]{2.409pt}{0.400pt}}
\put(1429.0,826.0){\rule[-0.200pt]{2.409pt}{0.400pt}}
\put(170.0,827.0){\rule[-0.200pt]{2.409pt}{0.400pt}}
\put(1429.0,827.0){\rule[-0.200pt]{2.409pt}{0.400pt}}
\put(170.0,827.0){\rule[-0.200pt]{2.409pt}{0.400pt}}
\put(1429.0,827.0){\rule[-0.200pt]{2.409pt}{0.400pt}}
\put(170.0,827.0){\rule[-0.200pt]{2.409pt}{0.400pt}}
\put(1429.0,827.0){\rule[-0.200pt]{2.409pt}{0.400pt}}
\put(170.0,827.0){\rule[-0.200pt]{2.409pt}{0.400pt}}
\put(1429.0,827.0){\rule[-0.200pt]{2.409pt}{0.400pt}}
\put(170.0,827.0){\rule[-0.200pt]{2.409pt}{0.400pt}}
\put(1429.0,827.0){\rule[-0.200pt]{2.409pt}{0.400pt}}
\put(170.0,827.0){\rule[-0.200pt]{2.409pt}{0.400pt}}
\put(1429.0,827.0){\rule[-0.200pt]{2.409pt}{0.400pt}}
\put(170.0,827.0){\rule[-0.200pt]{2.409pt}{0.400pt}}
\put(1429.0,827.0){\rule[-0.200pt]{2.409pt}{0.400pt}}
\put(170.0,827.0){\rule[-0.200pt]{2.409pt}{0.400pt}}
\put(1429.0,827.0){\rule[-0.200pt]{2.409pt}{0.400pt}}
\put(170.0,827.0){\rule[-0.200pt]{2.409pt}{0.400pt}}
\put(1429.0,827.0){\rule[-0.200pt]{2.409pt}{0.400pt}}
\put(170.0,827.0){\rule[-0.200pt]{2.409pt}{0.400pt}}
\put(1429.0,827.0){\rule[-0.200pt]{2.409pt}{0.400pt}}
\put(170.0,827.0){\rule[-0.200pt]{2.409pt}{0.400pt}}
\put(1429.0,827.0){\rule[-0.200pt]{2.409pt}{0.400pt}}
\put(170.0,828.0){\rule[-0.200pt]{2.409pt}{0.400pt}}
\put(1429.0,828.0){\rule[-0.200pt]{2.409pt}{0.400pt}}
\put(170.0,828.0){\rule[-0.200pt]{2.409pt}{0.400pt}}
\put(1429.0,828.0){\rule[-0.200pt]{2.409pt}{0.400pt}}
\put(170.0,828.0){\rule[-0.200pt]{2.409pt}{0.400pt}}
\put(1429.0,828.0){\rule[-0.200pt]{2.409pt}{0.400pt}}
\put(170.0,828.0){\rule[-0.200pt]{2.409pt}{0.400pt}}
\put(1429.0,828.0){\rule[-0.200pt]{2.409pt}{0.400pt}}
\put(170.0,828.0){\rule[-0.200pt]{2.409pt}{0.400pt}}
\put(1429.0,828.0){\rule[-0.200pt]{2.409pt}{0.400pt}}
\put(170.0,828.0){\rule[-0.200pt]{2.409pt}{0.400pt}}
\put(1429.0,828.0){\rule[-0.200pt]{2.409pt}{0.400pt}}
\put(170.0,828.0){\rule[-0.200pt]{2.409pt}{0.400pt}}
\put(1429.0,828.0){\rule[-0.200pt]{2.409pt}{0.400pt}}
\put(170.0,828.0){\rule[-0.200pt]{2.409pt}{0.400pt}}
\put(1429.0,828.0){\rule[-0.200pt]{2.409pt}{0.400pt}}
\put(170.0,828.0){\rule[-0.200pt]{2.409pt}{0.400pt}}
\put(1429.0,828.0){\rule[-0.200pt]{2.409pt}{0.400pt}}
\put(170.0,828.0){\rule[-0.200pt]{2.409pt}{0.400pt}}
\put(1429.0,828.0){\rule[-0.200pt]{2.409pt}{0.400pt}}
\put(170.0,828.0){\rule[-0.200pt]{2.409pt}{0.400pt}}
\put(1429.0,828.0){\rule[-0.200pt]{2.409pt}{0.400pt}}
\put(170.0,828.0){\rule[-0.200pt]{2.409pt}{0.400pt}}
\put(1429.0,828.0){\rule[-0.200pt]{2.409pt}{0.400pt}}
\put(170.0,829.0){\rule[-0.200pt]{2.409pt}{0.400pt}}
\put(1429.0,829.0){\rule[-0.200pt]{2.409pt}{0.400pt}}
\put(170.0,829.0){\rule[-0.200pt]{2.409pt}{0.400pt}}
\put(1429.0,829.0){\rule[-0.200pt]{2.409pt}{0.400pt}}
\put(170.0,829.0){\rule[-0.200pt]{2.409pt}{0.400pt}}
\put(1429.0,829.0){\rule[-0.200pt]{2.409pt}{0.400pt}}
\put(170.0,829.0){\rule[-0.200pt]{2.409pt}{0.400pt}}
\put(1429.0,829.0){\rule[-0.200pt]{2.409pt}{0.400pt}}
\put(170.0,829.0){\rule[-0.200pt]{2.409pt}{0.400pt}}
\put(1429.0,829.0){\rule[-0.200pt]{2.409pt}{0.400pt}}
\put(170.0,829.0){\rule[-0.200pt]{2.409pt}{0.400pt}}
\put(1429.0,829.0){\rule[-0.200pt]{2.409pt}{0.400pt}}
\put(170.0,829.0){\rule[-0.200pt]{2.409pt}{0.400pt}}
\put(1429.0,829.0){\rule[-0.200pt]{2.409pt}{0.400pt}}
\put(170.0,829.0){\rule[-0.200pt]{2.409pt}{0.400pt}}
\put(1429.0,829.0){\rule[-0.200pt]{2.409pt}{0.400pt}}
\put(170.0,829.0){\rule[-0.200pt]{2.409pt}{0.400pt}}
\put(1429.0,829.0){\rule[-0.200pt]{2.409pt}{0.400pt}}
\put(170.0,829.0){\rule[-0.200pt]{2.409pt}{0.400pt}}
\put(1429.0,829.0){\rule[-0.200pt]{2.409pt}{0.400pt}}
\put(170.0,829.0){\rule[-0.200pt]{2.409pt}{0.400pt}}
\put(1429.0,829.0){\rule[-0.200pt]{2.409pt}{0.400pt}}
\put(170.0,829.0){\rule[-0.200pt]{2.409pt}{0.400pt}}
\put(1429.0,829.0){\rule[-0.200pt]{2.409pt}{0.400pt}}
\put(170.0,830.0){\rule[-0.200pt]{2.409pt}{0.400pt}}
\put(1429.0,830.0){\rule[-0.200pt]{2.409pt}{0.400pt}}
\put(170.0,830.0){\rule[-0.200pt]{2.409pt}{0.400pt}}
\put(1429.0,830.0){\rule[-0.200pt]{2.409pt}{0.400pt}}
\put(170.0,830.0){\rule[-0.200pt]{2.409pt}{0.400pt}}
\put(1429.0,830.0){\rule[-0.200pt]{2.409pt}{0.400pt}}
\put(170.0,830.0){\rule[-0.200pt]{2.409pt}{0.400pt}}
\put(1429.0,830.0){\rule[-0.200pt]{2.409pt}{0.400pt}}
\put(170.0,830.0){\rule[-0.200pt]{2.409pt}{0.400pt}}
\put(1429.0,830.0){\rule[-0.200pt]{2.409pt}{0.400pt}}
\put(170.0,830.0){\rule[-0.200pt]{2.409pt}{0.400pt}}
\put(1429.0,830.0){\rule[-0.200pt]{2.409pt}{0.400pt}}
\put(170.0,830.0){\rule[-0.200pt]{2.409pt}{0.400pt}}
\put(1429.0,830.0){\rule[-0.200pt]{2.409pt}{0.400pt}}
\put(170.0,830.0){\rule[-0.200pt]{2.409pt}{0.400pt}}
\put(1429.0,830.0){\rule[-0.200pt]{2.409pt}{0.400pt}}
\put(170.0,830.0){\rule[-0.200pt]{2.409pt}{0.400pt}}
\put(1429.0,830.0){\rule[-0.200pt]{2.409pt}{0.400pt}}
\put(170.0,830.0){\rule[-0.200pt]{2.409pt}{0.400pt}}
\put(1429.0,830.0){\rule[-0.200pt]{2.409pt}{0.400pt}}
\put(170.0,830.0){\rule[-0.200pt]{2.409pt}{0.400pt}}
\put(1429.0,830.0){\rule[-0.200pt]{2.409pt}{0.400pt}}
\put(170.0,830.0){\rule[-0.200pt]{2.409pt}{0.400pt}}
\put(1429.0,830.0){\rule[-0.200pt]{2.409pt}{0.400pt}}
\put(170.0,831.0){\rule[-0.200pt]{2.409pt}{0.400pt}}
\put(1429.0,831.0){\rule[-0.200pt]{2.409pt}{0.400pt}}
\put(170.0,831.0){\rule[-0.200pt]{2.409pt}{0.400pt}}
\put(1429.0,831.0){\rule[-0.200pt]{2.409pt}{0.400pt}}
\put(170.0,831.0){\rule[-0.200pt]{2.409pt}{0.400pt}}
\put(1429.0,831.0){\rule[-0.200pt]{2.409pt}{0.400pt}}
\put(170.0,831.0){\rule[-0.200pt]{2.409pt}{0.400pt}}
\put(1429.0,831.0){\rule[-0.200pt]{2.409pt}{0.400pt}}
\put(170.0,831.0){\rule[-0.200pt]{2.409pt}{0.400pt}}
\put(1429.0,831.0){\rule[-0.200pt]{2.409pt}{0.400pt}}
\put(170.0,831.0){\rule[-0.200pt]{2.409pt}{0.400pt}}
\put(1429.0,831.0){\rule[-0.200pt]{2.409pt}{0.400pt}}
\put(170.0,831.0){\rule[-0.200pt]{2.409pt}{0.400pt}}
\put(1429.0,831.0){\rule[-0.200pt]{2.409pt}{0.400pt}}
\put(170.0,831.0){\rule[-0.200pt]{2.409pt}{0.400pt}}
\put(1429.0,831.0){\rule[-0.200pt]{2.409pt}{0.400pt}}
\put(170.0,831.0){\rule[-0.200pt]{2.409pt}{0.400pt}}
\put(1429.0,831.0){\rule[-0.200pt]{2.409pt}{0.400pt}}
\put(170.0,831.0){\rule[-0.200pt]{2.409pt}{0.400pt}}
\put(1429.0,831.0){\rule[-0.200pt]{2.409pt}{0.400pt}}
\put(170.0,831.0){\rule[-0.200pt]{2.409pt}{0.400pt}}
\put(1429.0,831.0){\rule[-0.200pt]{2.409pt}{0.400pt}}
\put(170.0,831.0){\rule[-0.200pt]{2.409pt}{0.400pt}}
\put(1429.0,831.0){\rule[-0.200pt]{2.409pt}{0.400pt}}
\put(170.0,831.0){\rule[-0.200pt]{2.409pt}{0.400pt}}
\put(1429.0,831.0){\rule[-0.200pt]{2.409pt}{0.400pt}}
\put(170.0,832.0){\rule[-0.200pt]{2.409pt}{0.400pt}}
\put(1429.0,832.0){\rule[-0.200pt]{2.409pt}{0.400pt}}
\put(170.0,832.0){\rule[-0.200pt]{2.409pt}{0.400pt}}
\put(1429.0,832.0){\rule[-0.200pt]{2.409pt}{0.400pt}}
\put(170.0,832.0){\rule[-0.200pt]{2.409pt}{0.400pt}}
\put(1429.0,832.0){\rule[-0.200pt]{2.409pt}{0.400pt}}
\put(170.0,832.0){\rule[-0.200pt]{2.409pt}{0.400pt}}
\put(1429.0,832.0){\rule[-0.200pt]{2.409pt}{0.400pt}}
\put(170.0,832.0){\rule[-0.200pt]{2.409pt}{0.400pt}}
\put(1429.0,832.0){\rule[-0.200pt]{2.409pt}{0.400pt}}
\put(170.0,832.0){\rule[-0.200pt]{2.409pt}{0.400pt}}
\put(1429.0,832.0){\rule[-0.200pt]{2.409pt}{0.400pt}}
\put(170.0,832.0){\rule[-0.200pt]{2.409pt}{0.400pt}}
\put(1429.0,832.0){\rule[-0.200pt]{2.409pt}{0.400pt}}
\put(170.0,832.0){\rule[-0.200pt]{2.409pt}{0.400pt}}
\put(1429.0,832.0){\rule[-0.200pt]{2.409pt}{0.400pt}}
\put(170.0,832.0){\rule[-0.200pt]{2.409pt}{0.400pt}}
\put(1429.0,832.0){\rule[-0.200pt]{2.409pt}{0.400pt}}
\put(170.0,832.0){\rule[-0.200pt]{2.409pt}{0.400pt}}
\put(1429.0,832.0){\rule[-0.200pt]{2.409pt}{0.400pt}}
\put(170.0,832.0){\rule[-0.200pt]{2.409pt}{0.400pt}}
\put(1429.0,832.0){\rule[-0.200pt]{2.409pt}{0.400pt}}
\put(170.0,832.0){\rule[-0.200pt]{2.409pt}{0.400pt}}
\put(1429.0,832.0){\rule[-0.200pt]{2.409pt}{0.400pt}}
\put(170.0,832.0){\rule[-0.200pt]{2.409pt}{0.400pt}}
\put(1429.0,832.0){\rule[-0.200pt]{2.409pt}{0.400pt}}
\put(170.0,833.0){\rule[-0.200pt]{2.409pt}{0.400pt}}
\put(1429.0,833.0){\rule[-0.200pt]{2.409pt}{0.400pt}}
\put(170.0,833.0){\rule[-0.200pt]{2.409pt}{0.400pt}}
\put(1429.0,833.0){\rule[-0.200pt]{2.409pt}{0.400pt}}
\put(170.0,833.0){\rule[-0.200pt]{2.409pt}{0.400pt}}
\put(1429.0,833.0){\rule[-0.200pt]{2.409pt}{0.400pt}}
\put(170.0,833.0){\rule[-0.200pt]{2.409pt}{0.400pt}}
\put(1429.0,833.0){\rule[-0.200pt]{2.409pt}{0.400pt}}
\put(170.0,833.0){\rule[-0.200pt]{2.409pt}{0.400pt}}
\put(1429.0,833.0){\rule[-0.200pt]{2.409pt}{0.400pt}}
\put(170.0,833.0){\rule[-0.200pt]{2.409pt}{0.400pt}}
\put(1429.0,833.0){\rule[-0.200pt]{2.409pt}{0.400pt}}
\put(170.0,833.0){\rule[-0.200pt]{2.409pt}{0.400pt}}
\put(1429.0,833.0){\rule[-0.200pt]{2.409pt}{0.400pt}}
\put(170.0,833.0){\rule[-0.200pt]{2.409pt}{0.400pt}}
\put(1429.0,833.0){\rule[-0.200pt]{2.409pt}{0.400pt}}
\put(170.0,833.0){\rule[-0.200pt]{2.409pt}{0.400pt}}
\put(1429.0,833.0){\rule[-0.200pt]{2.409pt}{0.400pt}}
\put(170.0,833.0){\rule[-0.200pt]{2.409pt}{0.400pt}}
\put(1429.0,833.0){\rule[-0.200pt]{2.409pt}{0.400pt}}
\put(170.0,833.0){\rule[-0.200pt]{2.409pt}{0.400pt}}
\put(1429.0,833.0){\rule[-0.200pt]{2.409pt}{0.400pt}}
\put(170.0,833.0){\rule[-0.200pt]{2.409pt}{0.400pt}}
\put(1429.0,833.0){\rule[-0.200pt]{2.409pt}{0.400pt}}
\put(170.0,833.0){\rule[-0.200pt]{2.409pt}{0.400pt}}
\put(1429.0,833.0){\rule[-0.200pt]{2.409pt}{0.400pt}}
\put(170.0,834.0){\rule[-0.200pt]{2.409pt}{0.400pt}}
\put(1429.0,834.0){\rule[-0.200pt]{2.409pt}{0.400pt}}
\put(170.0,834.0){\rule[-0.200pt]{2.409pt}{0.400pt}}
\put(1429.0,834.0){\rule[-0.200pt]{2.409pt}{0.400pt}}
\put(170.0,834.0){\rule[-0.200pt]{2.409pt}{0.400pt}}
\put(1429.0,834.0){\rule[-0.200pt]{2.409pt}{0.400pt}}
\put(170.0,834.0){\rule[-0.200pt]{2.409pt}{0.400pt}}
\put(1429.0,834.0){\rule[-0.200pt]{2.409pt}{0.400pt}}
\put(170.0,834.0){\rule[-0.200pt]{2.409pt}{0.400pt}}
\put(1429.0,834.0){\rule[-0.200pt]{2.409pt}{0.400pt}}
\put(170.0,834.0){\rule[-0.200pt]{2.409pt}{0.400pt}}
\put(1429.0,834.0){\rule[-0.200pt]{2.409pt}{0.400pt}}
\put(170.0,834.0){\rule[-0.200pt]{2.409pt}{0.400pt}}
\put(1429.0,834.0){\rule[-0.200pt]{2.409pt}{0.400pt}}
\put(170.0,834.0){\rule[-0.200pt]{2.409pt}{0.400pt}}
\put(1429.0,834.0){\rule[-0.200pt]{2.409pt}{0.400pt}}
\put(170.0,834.0){\rule[-0.200pt]{2.409pt}{0.400pt}}
\put(1429.0,834.0){\rule[-0.200pt]{2.409pt}{0.400pt}}
\put(170.0,834.0){\rule[-0.200pt]{2.409pt}{0.400pt}}
\put(1429.0,834.0){\rule[-0.200pt]{2.409pt}{0.400pt}}
\put(170.0,834.0){\rule[-0.200pt]{2.409pt}{0.400pt}}
\put(1429.0,834.0){\rule[-0.200pt]{2.409pt}{0.400pt}}
\put(170.0,834.0){\rule[-0.200pt]{2.409pt}{0.400pt}}
\put(1429.0,834.0){\rule[-0.200pt]{2.409pt}{0.400pt}}
\put(170.0,834.0){\rule[-0.200pt]{2.409pt}{0.400pt}}
\put(1429.0,834.0){\rule[-0.200pt]{2.409pt}{0.400pt}}
\put(170.0,834.0){\rule[-0.200pt]{2.409pt}{0.400pt}}
\put(1429.0,834.0){\rule[-0.200pt]{2.409pt}{0.400pt}}
\put(170.0,835.0){\rule[-0.200pt]{2.409pt}{0.400pt}}
\put(1429.0,835.0){\rule[-0.200pt]{2.409pt}{0.400pt}}
\put(170.0,835.0){\rule[-0.200pt]{2.409pt}{0.400pt}}
\put(1429.0,835.0){\rule[-0.200pt]{2.409pt}{0.400pt}}
\put(170.0,835.0){\rule[-0.200pt]{2.409pt}{0.400pt}}
\put(1429.0,835.0){\rule[-0.200pt]{2.409pt}{0.400pt}}
\put(170.0,835.0){\rule[-0.200pt]{2.409pt}{0.400pt}}
\put(1429.0,835.0){\rule[-0.200pt]{2.409pt}{0.400pt}}
\put(170.0,835.0){\rule[-0.200pt]{2.409pt}{0.400pt}}
\put(1429.0,835.0){\rule[-0.200pt]{2.409pt}{0.400pt}}
\put(170.0,835.0){\rule[-0.200pt]{2.409pt}{0.400pt}}
\put(1429.0,835.0){\rule[-0.200pt]{2.409pt}{0.400pt}}
\put(170.0,835.0){\rule[-0.200pt]{2.409pt}{0.400pt}}
\put(1429.0,835.0){\rule[-0.200pt]{2.409pt}{0.400pt}}
\put(170.0,835.0){\rule[-0.200pt]{2.409pt}{0.400pt}}
\put(1429.0,835.0){\rule[-0.200pt]{2.409pt}{0.400pt}}
\put(170.0,835.0){\rule[-0.200pt]{2.409pt}{0.400pt}}
\put(1429.0,835.0){\rule[-0.200pt]{2.409pt}{0.400pt}}
\put(170.0,835.0){\rule[-0.200pt]{2.409pt}{0.400pt}}
\put(1429.0,835.0){\rule[-0.200pt]{2.409pt}{0.400pt}}
\put(170.0,835.0){\rule[-0.200pt]{2.409pt}{0.400pt}}
\put(1429.0,835.0){\rule[-0.200pt]{2.409pt}{0.400pt}}
\put(170.0,835.0){\rule[-0.200pt]{2.409pt}{0.400pt}}
\put(1429.0,835.0){\rule[-0.200pt]{2.409pt}{0.400pt}}
\put(170.0,835.0){\rule[-0.200pt]{2.409pt}{0.400pt}}
\put(1429.0,835.0){\rule[-0.200pt]{2.409pt}{0.400pt}}
\put(170.0,835.0){\rule[-0.200pt]{2.409pt}{0.400pt}}
\put(1429.0,835.0){\rule[-0.200pt]{2.409pt}{0.400pt}}
\put(170.0,836.0){\rule[-0.200pt]{2.409pt}{0.400pt}}
\put(1429.0,836.0){\rule[-0.200pt]{2.409pt}{0.400pt}}
\put(170.0,836.0){\rule[-0.200pt]{2.409pt}{0.400pt}}
\put(1429.0,836.0){\rule[-0.200pt]{2.409pt}{0.400pt}}
\put(170.0,836.0){\rule[-0.200pt]{2.409pt}{0.400pt}}
\put(1429.0,836.0){\rule[-0.200pt]{2.409pt}{0.400pt}}
\put(170.0,836.0){\rule[-0.200pt]{2.409pt}{0.400pt}}
\put(1429.0,836.0){\rule[-0.200pt]{2.409pt}{0.400pt}}
\put(170.0,836.0){\rule[-0.200pt]{2.409pt}{0.400pt}}
\put(1429.0,836.0){\rule[-0.200pt]{2.409pt}{0.400pt}}
\put(170.0,836.0){\rule[-0.200pt]{2.409pt}{0.400pt}}
\put(1429.0,836.0){\rule[-0.200pt]{2.409pt}{0.400pt}}
\put(170.0,836.0){\rule[-0.200pt]{2.409pt}{0.400pt}}
\put(1429.0,836.0){\rule[-0.200pt]{2.409pt}{0.400pt}}
\put(170.0,836.0){\rule[-0.200pt]{2.409pt}{0.400pt}}
\put(1429.0,836.0){\rule[-0.200pt]{2.409pt}{0.400pt}}
\put(170.0,836.0){\rule[-0.200pt]{2.409pt}{0.400pt}}
\put(1429.0,836.0){\rule[-0.200pt]{2.409pt}{0.400pt}}
\put(170.0,836.0){\rule[-0.200pt]{2.409pt}{0.400pt}}
\put(1429.0,836.0){\rule[-0.200pt]{2.409pt}{0.400pt}}
\put(170.0,836.0){\rule[-0.200pt]{2.409pt}{0.400pt}}
\put(1429.0,836.0){\rule[-0.200pt]{2.409pt}{0.400pt}}
\put(170.0,836.0){\rule[-0.200pt]{2.409pt}{0.400pt}}
\put(1429.0,836.0){\rule[-0.200pt]{2.409pt}{0.400pt}}
\put(170.0,836.0){\rule[-0.200pt]{2.409pt}{0.400pt}}
\put(1429.0,836.0){\rule[-0.200pt]{2.409pt}{0.400pt}}
\put(170.0,836.0){\rule[-0.200pt]{2.409pt}{0.400pt}}
\put(1429.0,836.0){\rule[-0.200pt]{2.409pt}{0.400pt}}
\put(170.0,837.0){\rule[-0.200pt]{2.409pt}{0.400pt}}
\put(1429.0,837.0){\rule[-0.200pt]{2.409pt}{0.400pt}}
\put(170.0,837.0){\rule[-0.200pt]{2.409pt}{0.400pt}}
\put(1429.0,837.0){\rule[-0.200pt]{2.409pt}{0.400pt}}
\put(170.0,837.0){\rule[-0.200pt]{2.409pt}{0.400pt}}
\put(1429.0,837.0){\rule[-0.200pt]{2.409pt}{0.400pt}}
\put(170.0,837.0){\rule[-0.200pt]{2.409pt}{0.400pt}}
\put(1429.0,837.0){\rule[-0.200pt]{2.409pt}{0.400pt}}
\put(170.0,837.0){\rule[-0.200pt]{2.409pt}{0.400pt}}
\put(1429.0,837.0){\rule[-0.200pt]{2.409pt}{0.400pt}}
\put(170.0,837.0){\rule[-0.200pt]{2.409pt}{0.400pt}}
\put(1429.0,837.0){\rule[-0.200pt]{2.409pt}{0.400pt}}
\put(170.0,837.0){\rule[-0.200pt]{2.409pt}{0.400pt}}
\put(1429.0,837.0){\rule[-0.200pt]{2.409pt}{0.400pt}}
\put(170.0,837.0){\rule[-0.200pt]{2.409pt}{0.400pt}}
\put(1429.0,837.0){\rule[-0.200pt]{2.409pt}{0.400pt}}
\put(170.0,837.0){\rule[-0.200pt]{2.409pt}{0.400pt}}
\put(1429.0,837.0){\rule[-0.200pt]{2.409pt}{0.400pt}}
\put(170.0,837.0){\rule[-0.200pt]{2.409pt}{0.400pt}}
\put(1429.0,837.0){\rule[-0.200pt]{2.409pt}{0.400pt}}
\put(170.0,837.0){\rule[-0.200pt]{2.409pt}{0.400pt}}
\put(1429.0,837.0){\rule[-0.200pt]{2.409pt}{0.400pt}}
\put(170.0,837.0){\rule[-0.200pt]{2.409pt}{0.400pt}}
\put(1429.0,837.0){\rule[-0.200pt]{2.409pt}{0.400pt}}
\put(170.0,837.0){\rule[-0.200pt]{2.409pt}{0.400pt}}
\put(1429.0,837.0){\rule[-0.200pt]{2.409pt}{0.400pt}}
\put(170.0,837.0){\rule[-0.200pt]{2.409pt}{0.400pt}}
\put(1429.0,837.0){\rule[-0.200pt]{2.409pt}{0.400pt}}
\put(170.0,837.0){\rule[-0.200pt]{2.409pt}{0.400pt}}
\put(1429.0,837.0){\rule[-0.200pt]{2.409pt}{0.400pt}}
\put(170.0,838.0){\rule[-0.200pt]{2.409pt}{0.400pt}}
\put(1429.0,838.0){\rule[-0.200pt]{2.409pt}{0.400pt}}
\put(170.0,838.0){\rule[-0.200pt]{2.409pt}{0.400pt}}
\put(1429.0,838.0){\rule[-0.200pt]{2.409pt}{0.400pt}}
\put(170.0,838.0){\rule[-0.200pt]{2.409pt}{0.400pt}}
\put(1429.0,838.0){\rule[-0.200pt]{2.409pt}{0.400pt}}
\put(170.0,838.0){\rule[-0.200pt]{2.409pt}{0.400pt}}
\put(1429.0,838.0){\rule[-0.200pt]{2.409pt}{0.400pt}}
\put(170.0,838.0){\rule[-0.200pt]{2.409pt}{0.400pt}}
\put(1429.0,838.0){\rule[-0.200pt]{2.409pt}{0.400pt}}
\put(170.0,838.0){\rule[-0.200pt]{2.409pt}{0.400pt}}
\put(1429.0,838.0){\rule[-0.200pt]{2.409pt}{0.400pt}}
\put(170.0,838.0){\rule[-0.200pt]{2.409pt}{0.400pt}}
\put(1429.0,838.0){\rule[-0.200pt]{2.409pt}{0.400pt}}
\put(170.0,838.0){\rule[-0.200pt]{2.409pt}{0.400pt}}
\put(1429.0,838.0){\rule[-0.200pt]{2.409pt}{0.400pt}}
\put(170.0,838.0){\rule[-0.200pt]{2.409pt}{0.400pt}}
\put(1429.0,838.0){\rule[-0.200pt]{2.409pt}{0.400pt}}
\put(170.0,838.0){\rule[-0.200pt]{2.409pt}{0.400pt}}
\put(1429.0,838.0){\rule[-0.200pt]{2.409pt}{0.400pt}}
\put(170.0,838.0){\rule[-0.200pt]{2.409pt}{0.400pt}}
\put(1429.0,838.0){\rule[-0.200pt]{2.409pt}{0.400pt}}
\put(170.0,838.0){\rule[-0.200pt]{2.409pt}{0.400pt}}
\put(1429.0,838.0){\rule[-0.200pt]{2.409pt}{0.400pt}}
\put(170.0,838.0){\rule[-0.200pt]{2.409pt}{0.400pt}}
\put(1429.0,838.0){\rule[-0.200pt]{2.409pt}{0.400pt}}
\put(170.0,838.0){\rule[-0.200pt]{2.409pt}{0.400pt}}
\put(1429.0,838.0){\rule[-0.200pt]{2.409pt}{0.400pt}}
\put(170.0,838.0){\rule[-0.200pt]{2.409pt}{0.400pt}}
\put(1429.0,838.0){\rule[-0.200pt]{2.409pt}{0.400pt}}
\put(170.0,839.0){\rule[-0.200pt]{2.409pt}{0.400pt}}
\put(1429.0,839.0){\rule[-0.200pt]{2.409pt}{0.400pt}}
\put(170.0,839.0){\rule[-0.200pt]{2.409pt}{0.400pt}}
\put(1429.0,839.0){\rule[-0.200pt]{2.409pt}{0.400pt}}
\put(170.0,839.0){\rule[-0.200pt]{2.409pt}{0.400pt}}
\put(1429.0,839.0){\rule[-0.200pt]{2.409pt}{0.400pt}}
\put(170.0,839.0){\rule[-0.200pt]{2.409pt}{0.400pt}}
\put(1429.0,839.0){\rule[-0.200pt]{2.409pt}{0.400pt}}
\put(170.0,839.0){\rule[-0.200pt]{2.409pt}{0.400pt}}
\put(1429.0,839.0){\rule[-0.200pt]{2.409pt}{0.400pt}}
\put(170.0,839.0){\rule[-0.200pt]{2.409pt}{0.400pt}}
\put(1429.0,839.0){\rule[-0.200pt]{2.409pt}{0.400pt}}
\put(170.0,839.0){\rule[-0.200pt]{2.409pt}{0.400pt}}
\put(1429.0,839.0){\rule[-0.200pt]{2.409pt}{0.400pt}}
\put(170.0,839.0){\rule[-0.200pt]{2.409pt}{0.400pt}}
\put(1429.0,839.0){\rule[-0.200pt]{2.409pt}{0.400pt}}
\put(170.0,839.0){\rule[-0.200pt]{2.409pt}{0.400pt}}
\put(1429.0,839.0){\rule[-0.200pt]{2.409pt}{0.400pt}}
\put(170.0,839.0){\rule[-0.200pt]{2.409pt}{0.400pt}}
\put(1429.0,839.0){\rule[-0.200pt]{2.409pt}{0.400pt}}
\put(170.0,839.0){\rule[-0.200pt]{2.409pt}{0.400pt}}
\put(1429.0,839.0){\rule[-0.200pt]{2.409pt}{0.400pt}}
\put(170.0,839.0){\rule[-0.200pt]{2.409pt}{0.400pt}}
\put(1429.0,839.0){\rule[-0.200pt]{2.409pt}{0.400pt}}
\put(170.0,839.0){\rule[-0.200pt]{2.409pt}{0.400pt}}
\put(1429.0,839.0){\rule[-0.200pt]{2.409pt}{0.400pt}}
\put(170.0,839.0){\rule[-0.200pt]{2.409pt}{0.400pt}}
\put(1429.0,839.0){\rule[-0.200pt]{2.409pt}{0.400pt}}
\put(170.0,839.0){\rule[-0.200pt]{2.409pt}{0.400pt}}
\put(1429.0,839.0){\rule[-0.200pt]{2.409pt}{0.400pt}}
\put(170.0,839.0){\rule[-0.200pt]{2.409pt}{0.400pt}}
\put(1429.0,839.0){\rule[-0.200pt]{2.409pt}{0.400pt}}
\put(170.0,840.0){\rule[-0.200pt]{2.409pt}{0.400pt}}
\put(1429.0,840.0){\rule[-0.200pt]{2.409pt}{0.400pt}}
\put(170.0,840.0){\rule[-0.200pt]{2.409pt}{0.400pt}}
\put(1429.0,840.0){\rule[-0.200pt]{2.409pt}{0.400pt}}
\put(170.0,840.0){\rule[-0.200pt]{2.409pt}{0.400pt}}
\put(1429.0,840.0){\rule[-0.200pt]{2.409pt}{0.400pt}}
\put(170.0,840.0){\rule[-0.200pt]{2.409pt}{0.400pt}}
\put(1429.0,840.0){\rule[-0.200pt]{2.409pt}{0.400pt}}
\put(170.0,840.0){\rule[-0.200pt]{2.409pt}{0.400pt}}
\put(1429.0,840.0){\rule[-0.200pt]{2.409pt}{0.400pt}}
\put(170.0,840.0){\rule[-0.200pt]{2.409pt}{0.400pt}}
\put(1429.0,840.0){\rule[-0.200pt]{2.409pt}{0.400pt}}
\put(170.0,840.0){\rule[-0.200pt]{2.409pt}{0.400pt}}
\put(1429.0,840.0){\rule[-0.200pt]{2.409pt}{0.400pt}}
\put(170.0,840.0){\rule[-0.200pt]{2.409pt}{0.400pt}}
\put(1429.0,840.0){\rule[-0.200pt]{2.409pt}{0.400pt}}
\put(170.0,840.0){\rule[-0.200pt]{2.409pt}{0.400pt}}
\put(1429.0,840.0){\rule[-0.200pt]{2.409pt}{0.400pt}}
\put(170.0,840.0){\rule[-0.200pt]{2.409pt}{0.400pt}}
\put(1429.0,840.0){\rule[-0.200pt]{2.409pt}{0.400pt}}
\put(170.0,840.0){\rule[-0.200pt]{2.409pt}{0.400pt}}
\put(1429.0,840.0){\rule[-0.200pt]{2.409pt}{0.400pt}}
\put(170.0,840.0){\rule[-0.200pt]{2.409pt}{0.400pt}}
\put(1429.0,840.0){\rule[-0.200pt]{2.409pt}{0.400pt}}
\put(170.0,840.0){\rule[-0.200pt]{2.409pt}{0.400pt}}
\put(1429.0,840.0){\rule[-0.200pt]{2.409pt}{0.400pt}}
\put(170.0,840.0){\rule[-0.200pt]{2.409pt}{0.400pt}}
\put(1429.0,840.0){\rule[-0.200pt]{2.409pt}{0.400pt}}
\put(170.0,840.0){\rule[-0.200pt]{2.409pt}{0.400pt}}
\put(1429.0,840.0){\rule[-0.200pt]{2.409pt}{0.400pt}}
\put(170.0,840.0){\rule[-0.200pt]{2.409pt}{0.400pt}}
\put(1429.0,840.0){\rule[-0.200pt]{2.409pt}{0.400pt}}
\put(170.0,841.0){\rule[-0.200pt]{2.409pt}{0.400pt}}
\put(1429.0,841.0){\rule[-0.200pt]{2.409pt}{0.400pt}}
\put(170.0,841.0){\rule[-0.200pt]{2.409pt}{0.400pt}}
\put(1429.0,841.0){\rule[-0.200pt]{2.409pt}{0.400pt}}
\put(170.0,841.0){\rule[-0.200pt]{2.409pt}{0.400pt}}
\put(1429.0,841.0){\rule[-0.200pt]{2.409pt}{0.400pt}}
\put(170.0,841.0){\rule[-0.200pt]{2.409pt}{0.400pt}}
\put(1429.0,841.0){\rule[-0.200pt]{2.409pt}{0.400pt}}
\put(170.0,841.0){\rule[-0.200pt]{2.409pt}{0.400pt}}
\put(1429.0,841.0){\rule[-0.200pt]{2.409pt}{0.400pt}}
\put(170.0,841.0){\rule[-0.200pt]{2.409pt}{0.400pt}}
\put(1429.0,841.0){\rule[-0.200pt]{2.409pt}{0.400pt}}
\put(170.0,841.0){\rule[-0.200pt]{2.409pt}{0.400pt}}
\put(1429.0,841.0){\rule[-0.200pt]{2.409pt}{0.400pt}}
\put(170.0,841.0){\rule[-0.200pt]{2.409pt}{0.400pt}}
\put(1429.0,841.0){\rule[-0.200pt]{2.409pt}{0.400pt}}
\put(170.0,841.0){\rule[-0.200pt]{2.409pt}{0.400pt}}
\put(1429.0,841.0){\rule[-0.200pt]{2.409pt}{0.400pt}}
\put(170.0,841.0){\rule[-0.200pt]{2.409pt}{0.400pt}}
\put(1429.0,841.0){\rule[-0.200pt]{2.409pt}{0.400pt}}
\put(170.0,841.0){\rule[-0.200pt]{2.409pt}{0.400pt}}
\put(1429.0,841.0){\rule[-0.200pt]{2.409pt}{0.400pt}}
\put(170.0,841.0){\rule[-0.200pt]{2.409pt}{0.400pt}}
\put(1429.0,841.0){\rule[-0.200pt]{2.409pt}{0.400pt}}
\put(170.0,841.0){\rule[-0.200pt]{2.409pt}{0.400pt}}
\put(1429.0,841.0){\rule[-0.200pt]{2.409pt}{0.400pt}}
\put(170.0,841.0){\rule[-0.200pt]{2.409pt}{0.400pt}}
\put(1429.0,841.0){\rule[-0.200pt]{2.409pt}{0.400pt}}
\put(170.0,841.0){\rule[-0.200pt]{2.409pt}{0.400pt}}
\put(1429.0,841.0){\rule[-0.200pt]{2.409pt}{0.400pt}}
\put(170.0,841.0){\rule[-0.200pt]{2.409pt}{0.400pt}}
\put(1429.0,841.0){\rule[-0.200pt]{2.409pt}{0.400pt}}
\put(170.0,841.0){\rule[-0.200pt]{2.409pt}{0.400pt}}
\put(1429.0,841.0){\rule[-0.200pt]{2.409pt}{0.400pt}}
\put(170.0,842.0){\rule[-0.200pt]{2.409pt}{0.400pt}}
\put(1429.0,842.0){\rule[-0.200pt]{2.409pt}{0.400pt}}
\put(170.0,842.0){\rule[-0.200pt]{2.409pt}{0.400pt}}
\put(1429.0,842.0){\rule[-0.200pt]{2.409pt}{0.400pt}}
\put(170.0,842.0){\rule[-0.200pt]{2.409pt}{0.400pt}}
\put(1429.0,842.0){\rule[-0.200pt]{2.409pt}{0.400pt}}
\put(170.0,842.0){\rule[-0.200pt]{2.409pt}{0.400pt}}
\put(1429.0,842.0){\rule[-0.200pt]{2.409pt}{0.400pt}}
\put(170.0,842.0){\rule[-0.200pt]{2.409pt}{0.400pt}}
\put(1429.0,842.0){\rule[-0.200pt]{2.409pt}{0.400pt}}
\put(170.0,842.0){\rule[-0.200pt]{2.409pt}{0.400pt}}
\put(1429.0,842.0){\rule[-0.200pt]{2.409pt}{0.400pt}}
\put(170.0,842.0){\rule[-0.200pt]{2.409pt}{0.400pt}}
\put(1429.0,842.0){\rule[-0.200pt]{2.409pt}{0.400pt}}
\put(170.0,842.0){\rule[-0.200pt]{2.409pt}{0.400pt}}
\put(1429.0,842.0){\rule[-0.200pt]{2.409pt}{0.400pt}}
\put(170.0,842.0){\rule[-0.200pt]{2.409pt}{0.400pt}}
\put(1429.0,842.0){\rule[-0.200pt]{2.409pt}{0.400pt}}
\put(170.0,842.0){\rule[-0.200pt]{2.409pt}{0.400pt}}
\put(1429.0,842.0){\rule[-0.200pt]{2.409pt}{0.400pt}}
\put(170.0,842.0){\rule[-0.200pt]{2.409pt}{0.400pt}}
\put(1429.0,842.0){\rule[-0.200pt]{2.409pt}{0.400pt}}
\put(170.0,842.0){\rule[-0.200pt]{2.409pt}{0.400pt}}
\put(1429.0,842.0){\rule[-0.200pt]{2.409pt}{0.400pt}}
\put(170.0,842.0){\rule[-0.200pt]{2.409pt}{0.400pt}}
\put(1429.0,842.0){\rule[-0.200pt]{2.409pt}{0.400pt}}
\put(170.0,842.0){\rule[-0.200pt]{2.409pt}{0.400pt}}
\put(1429.0,842.0){\rule[-0.200pt]{2.409pt}{0.400pt}}
\put(170.0,842.0){\rule[-0.200pt]{2.409pt}{0.400pt}}
\put(1429.0,842.0){\rule[-0.200pt]{2.409pt}{0.400pt}}
\put(170.0,842.0){\rule[-0.200pt]{2.409pt}{0.400pt}}
\put(1429.0,842.0){\rule[-0.200pt]{2.409pt}{0.400pt}}
\put(170.0,843.0){\rule[-0.200pt]{2.409pt}{0.400pt}}
\put(1429.0,843.0){\rule[-0.200pt]{2.409pt}{0.400pt}}
\put(170.0,843.0){\rule[-0.200pt]{2.409pt}{0.400pt}}
\put(1429.0,843.0){\rule[-0.200pt]{2.409pt}{0.400pt}}
\put(170.0,843.0){\rule[-0.200pt]{2.409pt}{0.400pt}}
\put(1429.0,843.0){\rule[-0.200pt]{2.409pt}{0.400pt}}
\put(170.0,843.0){\rule[-0.200pt]{2.409pt}{0.400pt}}
\put(1429.0,843.0){\rule[-0.200pt]{2.409pt}{0.400pt}}
\put(170.0,843.0){\rule[-0.200pt]{2.409pt}{0.400pt}}
\put(1429.0,843.0){\rule[-0.200pt]{2.409pt}{0.400pt}}
\put(170.0,843.0){\rule[-0.200pt]{2.409pt}{0.400pt}}
\put(1429.0,843.0){\rule[-0.200pt]{2.409pt}{0.400pt}}
\put(170.0,843.0){\rule[-0.200pt]{2.409pt}{0.400pt}}
\put(1429.0,843.0){\rule[-0.200pt]{2.409pt}{0.400pt}}
\put(170.0,843.0){\rule[-0.200pt]{2.409pt}{0.400pt}}
\put(1429.0,843.0){\rule[-0.200pt]{2.409pt}{0.400pt}}
\put(170.0,843.0){\rule[-0.200pt]{2.409pt}{0.400pt}}
\put(1429.0,843.0){\rule[-0.200pt]{2.409pt}{0.400pt}}
\put(170.0,843.0){\rule[-0.200pt]{2.409pt}{0.400pt}}
\put(1429.0,843.0){\rule[-0.200pt]{2.409pt}{0.400pt}}
\put(170.0,843.0){\rule[-0.200pt]{2.409pt}{0.400pt}}
\put(1429.0,843.0){\rule[-0.200pt]{2.409pt}{0.400pt}}
\put(170.0,843.0){\rule[-0.200pt]{2.409pt}{0.400pt}}
\put(1429.0,843.0){\rule[-0.200pt]{2.409pt}{0.400pt}}
\put(170.0,843.0){\rule[-0.200pt]{2.409pt}{0.400pt}}
\put(1429.0,843.0){\rule[-0.200pt]{2.409pt}{0.400pt}}
\put(170.0,843.0){\rule[-0.200pt]{2.409pt}{0.400pt}}
\put(1429.0,843.0){\rule[-0.200pt]{2.409pt}{0.400pt}}
\put(170.0,843.0){\rule[-0.200pt]{2.409pt}{0.400pt}}
\put(1429.0,843.0){\rule[-0.200pt]{2.409pt}{0.400pt}}
\put(170.0,843.0){\rule[-0.200pt]{2.409pt}{0.400pt}}
\put(1429.0,843.0){\rule[-0.200pt]{2.409pt}{0.400pt}}
\put(170.0,843.0){\rule[-0.200pt]{2.409pt}{0.400pt}}
\put(1429.0,843.0){\rule[-0.200pt]{2.409pt}{0.400pt}}
\put(170.0,843.0){\rule[-0.200pt]{2.409pt}{0.400pt}}
\put(1429.0,843.0){\rule[-0.200pt]{2.409pt}{0.400pt}}
\put(170.0,844.0){\rule[-0.200pt]{2.409pt}{0.400pt}}
\put(1429.0,844.0){\rule[-0.200pt]{2.409pt}{0.400pt}}
\put(170.0,844.0){\rule[-0.200pt]{2.409pt}{0.400pt}}
\put(1429.0,844.0){\rule[-0.200pt]{2.409pt}{0.400pt}}
\put(170.0,844.0){\rule[-0.200pt]{2.409pt}{0.400pt}}
\put(1429.0,844.0){\rule[-0.200pt]{2.409pt}{0.400pt}}
\put(170.0,844.0){\rule[-0.200pt]{2.409pt}{0.400pt}}
\put(1429.0,844.0){\rule[-0.200pt]{2.409pt}{0.400pt}}
\put(170.0,844.0){\rule[-0.200pt]{2.409pt}{0.400pt}}
\put(1429.0,844.0){\rule[-0.200pt]{2.409pt}{0.400pt}}
\put(170.0,844.0){\rule[-0.200pt]{2.409pt}{0.400pt}}
\put(1429.0,844.0){\rule[-0.200pt]{2.409pt}{0.400pt}}
\put(170.0,844.0){\rule[-0.200pt]{2.409pt}{0.400pt}}
\put(1429.0,844.0){\rule[-0.200pt]{2.409pt}{0.400pt}}
\put(170.0,844.0){\rule[-0.200pt]{2.409pt}{0.400pt}}
\put(1429.0,844.0){\rule[-0.200pt]{2.409pt}{0.400pt}}
\put(170.0,844.0){\rule[-0.200pt]{2.409pt}{0.400pt}}
\put(1429.0,844.0){\rule[-0.200pt]{2.409pt}{0.400pt}}
\put(170.0,844.0){\rule[-0.200pt]{2.409pt}{0.400pt}}
\put(1429.0,844.0){\rule[-0.200pt]{2.409pt}{0.400pt}}
\put(170.0,844.0){\rule[-0.200pt]{2.409pt}{0.400pt}}
\put(1429.0,844.0){\rule[-0.200pt]{2.409pt}{0.400pt}}
\put(170.0,844.0){\rule[-0.200pt]{2.409pt}{0.400pt}}
\put(1429.0,844.0){\rule[-0.200pt]{2.409pt}{0.400pt}}
\put(170.0,844.0){\rule[-0.200pt]{2.409pt}{0.400pt}}
\put(1429.0,844.0){\rule[-0.200pt]{2.409pt}{0.400pt}}
\put(170.0,844.0){\rule[-0.200pt]{2.409pt}{0.400pt}}
\put(1429.0,844.0){\rule[-0.200pt]{2.409pt}{0.400pt}}
\put(170.0,844.0){\rule[-0.200pt]{2.409pt}{0.400pt}}
\put(1429.0,844.0){\rule[-0.200pt]{2.409pt}{0.400pt}}
\put(170.0,844.0){\rule[-0.200pt]{2.409pt}{0.400pt}}
\put(1429.0,844.0){\rule[-0.200pt]{2.409pt}{0.400pt}}
\put(170.0,844.0){\rule[-0.200pt]{2.409pt}{0.400pt}}
\put(1429.0,844.0){\rule[-0.200pt]{2.409pt}{0.400pt}}
\put(170.0,844.0){\rule[-0.200pt]{2.409pt}{0.400pt}}
\put(1429.0,844.0){\rule[-0.200pt]{2.409pt}{0.400pt}}
\put(170.0,845.0){\rule[-0.200pt]{2.409pt}{0.400pt}}
\put(1429.0,845.0){\rule[-0.200pt]{2.409pt}{0.400pt}}
\put(170.0,845.0){\rule[-0.200pt]{2.409pt}{0.400pt}}
\put(1429.0,845.0){\rule[-0.200pt]{2.409pt}{0.400pt}}
\put(170.0,845.0){\rule[-0.200pt]{2.409pt}{0.400pt}}
\put(1429.0,845.0){\rule[-0.200pt]{2.409pt}{0.400pt}}
\put(170.0,845.0){\rule[-0.200pt]{2.409pt}{0.400pt}}
\put(1429.0,845.0){\rule[-0.200pt]{2.409pt}{0.400pt}}
\put(170.0,845.0){\rule[-0.200pt]{2.409pt}{0.400pt}}
\put(1429.0,845.0){\rule[-0.200pt]{2.409pt}{0.400pt}}
\put(170.0,845.0){\rule[-0.200pt]{2.409pt}{0.400pt}}
\put(1429.0,845.0){\rule[-0.200pt]{2.409pt}{0.400pt}}
\put(170.0,845.0){\rule[-0.200pt]{2.409pt}{0.400pt}}
\put(1429.0,845.0){\rule[-0.200pt]{2.409pt}{0.400pt}}
\put(170.0,845.0){\rule[-0.200pt]{2.409pt}{0.400pt}}
\put(1429.0,845.0){\rule[-0.200pt]{2.409pt}{0.400pt}}
\put(170.0,845.0){\rule[-0.200pt]{2.409pt}{0.400pt}}
\put(1429.0,845.0){\rule[-0.200pt]{2.409pt}{0.400pt}}
\put(170.0,845.0){\rule[-0.200pt]{2.409pt}{0.400pt}}
\put(1429.0,845.0){\rule[-0.200pt]{2.409pt}{0.400pt}}
\put(170.0,845.0){\rule[-0.200pt]{2.409pt}{0.400pt}}
\put(1429.0,845.0){\rule[-0.200pt]{2.409pt}{0.400pt}}
\put(170.0,845.0){\rule[-0.200pt]{2.409pt}{0.400pt}}
\put(1429.0,845.0){\rule[-0.200pt]{2.409pt}{0.400pt}}
\put(170.0,845.0){\rule[-0.200pt]{2.409pt}{0.400pt}}
\put(1429.0,845.0){\rule[-0.200pt]{2.409pt}{0.400pt}}
\put(170.0,845.0){\rule[-0.200pt]{2.409pt}{0.400pt}}
\put(1429.0,845.0){\rule[-0.200pt]{2.409pt}{0.400pt}}
\put(170.0,845.0){\rule[-0.200pt]{2.409pt}{0.400pt}}
\put(1429.0,845.0){\rule[-0.200pt]{2.409pt}{0.400pt}}
\put(170.0,845.0){\rule[-0.200pt]{2.409pt}{0.400pt}}
\put(1429.0,845.0){\rule[-0.200pt]{2.409pt}{0.400pt}}
\put(170.0,845.0){\rule[-0.200pt]{2.409pt}{0.400pt}}
\put(1429.0,845.0){\rule[-0.200pt]{2.409pt}{0.400pt}}
\put(170.0,845.0){\rule[-0.200pt]{2.409pt}{0.400pt}}
\put(1429.0,845.0){\rule[-0.200pt]{2.409pt}{0.400pt}}
\put(170.0,846.0){\rule[-0.200pt]{2.409pt}{0.400pt}}
\put(1429.0,846.0){\rule[-0.200pt]{2.409pt}{0.400pt}}
\put(170.0,846.0){\rule[-0.200pt]{2.409pt}{0.400pt}}
\put(1429.0,846.0){\rule[-0.200pt]{2.409pt}{0.400pt}}
\put(170.0,846.0){\rule[-0.200pt]{2.409pt}{0.400pt}}
\put(1429.0,846.0){\rule[-0.200pt]{2.409pt}{0.400pt}}
\put(170.0,846.0){\rule[-0.200pt]{2.409pt}{0.400pt}}
\put(1429.0,846.0){\rule[-0.200pt]{2.409pt}{0.400pt}}
\put(170.0,846.0){\rule[-0.200pt]{2.409pt}{0.400pt}}
\put(1429.0,846.0){\rule[-0.200pt]{2.409pt}{0.400pt}}
\put(170.0,846.0){\rule[-0.200pt]{2.409pt}{0.400pt}}
\put(1429.0,846.0){\rule[-0.200pt]{2.409pt}{0.400pt}}
\put(170.0,846.0){\rule[-0.200pt]{2.409pt}{0.400pt}}
\put(1429.0,846.0){\rule[-0.200pt]{2.409pt}{0.400pt}}
\put(170.0,846.0){\rule[-0.200pt]{2.409pt}{0.400pt}}
\put(1429.0,846.0){\rule[-0.200pt]{2.409pt}{0.400pt}}
\put(170.0,846.0){\rule[-0.200pt]{2.409pt}{0.400pt}}
\put(1429.0,846.0){\rule[-0.200pt]{2.409pt}{0.400pt}}
\put(170.0,846.0){\rule[-0.200pt]{2.409pt}{0.400pt}}
\put(1429.0,846.0){\rule[-0.200pt]{2.409pt}{0.400pt}}
\put(170.0,846.0){\rule[-0.200pt]{2.409pt}{0.400pt}}
\put(1429.0,846.0){\rule[-0.200pt]{2.409pt}{0.400pt}}
\put(170.0,846.0){\rule[-0.200pt]{2.409pt}{0.400pt}}
\put(1429.0,846.0){\rule[-0.200pt]{2.409pt}{0.400pt}}
\put(170.0,846.0){\rule[-0.200pt]{2.409pt}{0.400pt}}
\put(1429.0,846.0){\rule[-0.200pt]{2.409pt}{0.400pt}}
\put(170.0,846.0){\rule[-0.200pt]{2.409pt}{0.400pt}}
\put(1429.0,846.0){\rule[-0.200pt]{2.409pt}{0.400pt}}
\put(170.0,846.0){\rule[-0.200pt]{2.409pt}{0.400pt}}
\put(1429.0,846.0){\rule[-0.200pt]{2.409pt}{0.400pt}}
\put(170.0,846.0){\rule[-0.200pt]{2.409pt}{0.400pt}}
\put(1429.0,846.0){\rule[-0.200pt]{2.409pt}{0.400pt}}
\put(170.0,846.0){\rule[-0.200pt]{2.409pt}{0.400pt}}
\put(1429.0,846.0){\rule[-0.200pt]{2.409pt}{0.400pt}}
\put(170.0,846.0){\rule[-0.200pt]{2.409pt}{0.400pt}}
\put(1429.0,846.0){\rule[-0.200pt]{2.409pt}{0.400pt}}
\put(170.0,846.0){\rule[-0.200pt]{2.409pt}{0.400pt}}
\put(1429.0,846.0){\rule[-0.200pt]{2.409pt}{0.400pt}}
\put(170.0,847.0){\rule[-0.200pt]{2.409pt}{0.400pt}}
\put(1429.0,847.0){\rule[-0.200pt]{2.409pt}{0.400pt}}
\put(170.0,847.0){\rule[-0.200pt]{2.409pt}{0.400pt}}
\put(1429.0,847.0){\rule[-0.200pt]{2.409pt}{0.400pt}}
\put(170.0,847.0){\rule[-0.200pt]{2.409pt}{0.400pt}}
\put(1429.0,847.0){\rule[-0.200pt]{2.409pt}{0.400pt}}
\put(170.0,847.0){\rule[-0.200pt]{2.409pt}{0.400pt}}
\put(1429.0,847.0){\rule[-0.200pt]{2.409pt}{0.400pt}}
\put(170.0,847.0){\rule[-0.200pt]{2.409pt}{0.400pt}}
\put(1429.0,847.0){\rule[-0.200pt]{2.409pt}{0.400pt}}
\put(170.0,847.0){\rule[-0.200pt]{2.409pt}{0.400pt}}
\put(1429.0,847.0){\rule[-0.200pt]{2.409pt}{0.400pt}}
\put(170.0,847.0){\rule[-0.200pt]{2.409pt}{0.400pt}}
\put(1429.0,847.0){\rule[-0.200pt]{2.409pt}{0.400pt}}
\put(170.0,847.0){\rule[-0.200pt]{2.409pt}{0.400pt}}
\put(1429.0,847.0){\rule[-0.200pt]{2.409pt}{0.400pt}}
\put(170.0,847.0){\rule[-0.200pt]{2.409pt}{0.400pt}}
\put(1429.0,847.0){\rule[-0.200pt]{2.409pt}{0.400pt}}
\put(170.0,847.0){\rule[-0.200pt]{2.409pt}{0.400pt}}
\put(1429.0,847.0){\rule[-0.200pt]{2.409pt}{0.400pt}}
\put(170.0,847.0){\rule[-0.200pt]{2.409pt}{0.400pt}}
\put(1429.0,847.0){\rule[-0.200pt]{2.409pt}{0.400pt}}
\put(170.0,847.0){\rule[-0.200pt]{2.409pt}{0.400pt}}
\put(1429.0,847.0){\rule[-0.200pt]{2.409pt}{0.400pt}}
\put(170.0,847.0){\rule[-0.200pt]{2.409pt}{0.400pt}}
\put(1429.0,847.0){\rule[-0.200pt]{2.409pt}{0.400pt}}
\put(170.0,847.0){\rule[-0.200pt]{2.409pt}{0.400pt}}
\put(1429.0,847.0){\rule[-0.200pt]{2.409pt}{0.400pt}}
\put(170.0,847.0){\rule[-0.200pt]{2.409pt}{0.400pt}}
\put(1429.0,847.0){\rule[-0.200pt]{2.409pt}{0.400pt}}
\put(170.0,847.0){\rule[-0.200pt]{2.409pt}{0.400pt}}
\put(1429.0,847.0){\rule[-0.200pt]{2.409pt}{0.400pt}}
\put(170.0,847.0){\rule[-0.200pt]{2.409pt}{0.400pt}}
\put(1429.0,847.0){\rule[-0.200pt]{2.409pt}{0.400pt}}
\put(170.0,847.0){\rule[-0.200pt]{2.409pt}{0.400pt}}
\put(1429.0,847.0){\rule[-0.200pt]{2.409pt}{0.400pt}}
\put(170.0,847.0){\rule[-0.200pt]{2.409pt}{0.400pt}}
\put(1429.0,847.0){\rule[-0.200pt]{2.409pt}{0.400pt}}
\put(170.0,848.0){\rule[-0.200pt]{2.409pt}{0.400pt}}
\put(1429.0,848.0){\rule[-0.200pt]{2.409pt}{0.400pt}}
\put(170.0,848.0){\rule[-0.200pt]{2.409pt}{0.400pt}}
\put(1429.0,848.0){\rule[-0.200pt]{2.409pt}{0.400pt}}
\put(170.0,848.0){\rule[-0.200pt]{2.409pt}{0.400pt}}
\put(1429.0,848.0){\rule[-0.200pt]{2.409pt}{0.400pt}}
\put(170.0,848.0){\rule[-0.200pt]{2.409pt}{0.400pt}}
\put(1429.0,848.0){\rule[-0.200pt]{2.409pt}{0.400pt}}
\put(170.0,848.0){\rule[-0.200pt]{2.409pt}{0.400pt}}
\put(1429.0,848.0){\rule[-0.200pt]{2.409pt}{0.400pt}}
\put(170.0,848.0){\rule[-0.200pt]{2.409pt}{0.400pt}}
\put(1429.0,848.0){\rule[-0.200pt]{2.409pt}{0.400pt}}
\put(170.0,848.0){\rule[-0.200pt]{2.409pt}{0.400pt}}
\put(1429.0,848.0){\rule[-0.200pt]{2.409pt}{0.400pt}}
\put(170.0,848.0){\rule[-0.200pt]{2.409pt}{0.400pt}}
\put(1429.0,848.0){\rule[-0.200pt]{2.409pt}{0.400pt}}
\put(170.0,848.0){\rule[-0.200pt]{2.409pt}{0.400pt}}
\put(1429.0,848.0){\rule[-0.200pt]{2.409pt}{0.400pt}}
\put(170.0,848.0){\rule[-0.200pt]{2.409pt}{0.400pt}}
\put(1429.0,848.0){\rule[-0.200pt]{2.409pt}{0.400pt}}
\put(170.0,848.0){\rule[-0.200pt]{2.409pt}{0.400pt}}
\put(1429.0,848.0){\rule[-0.200pt]{2.409pt}{0.400pt}}
\put(170.0,848.0){\rule[-0.200pt]{2.409pt}{0.400pt}}
\put(1429.0,848.0){\rule[-0.200pt]{2.409pt}{0.400pt}}
\put(170.0,848.0){\rule[-0.200pt]{2.409pt}{0.400pt}}
\put(1429.0,848.0){\rule[-0.200pt]{2.409pt}{0.400pt}}
\put(170.0,848.0){\rule[-0.200pt]{2.409pt}{0.400pt}}
\put(1429.0,848.0){\rule[-0.200pt]{2.409pt}{0.400pt}}
\put(170.0,848.0){\rule[-0.200pt]{2.409pt}{0.400pt}}
\put(1429.0,848.0){\rule[-0.200pt]{2.409pt}{0.400pt}}
\put(170.0,848.0){\rule[-0.200pt]{2.409pt}{0.400pt}}
\put(1429.0,848.0){\rule[-0.200pt]{2.409pt}{0.400pt}}
\put(170.0,848.0){\rule[-0.200pt]{2.409pt}{0.400pt}}
\put(1429.0,848.0){\rule[-0.200pt]{2.409pt}{0.400pt}}
\put(170.0,848.0){\rule[-0.200pt]{2.409pt}{0.400pt}}
\put(1429.0,848.0){\rule[-0.200pt]{2.409pt}{0.400pt}}
\put(170.0,848.0){\rule[-0.200pt]{2.409pt}{0.400pt}}
\put(1429.0,848.0){\rule[-0.200pt]{2.409pt}{0.400pt}}
\put(170.0,848.0){\rule[-0.200pt]{2.409pt}{0.400pt}}
\put(1429.0,848.0){\rule[-0.200pt]{2.409pt}{0.400pt}}
\put(170.0,849.0){\rule[-0.200pt]{2.409pt}{0.400pt}}
\put(1429.0,849.0){\rule[-0.200pt]{2.409pt}{0.400pt}}
\put(170.0,849.0){\rule[-0.200pt]{2.409pt}{0.400pt}}
\put(1429.0,849.0){\rule[-0.200pt]{2.409pt}{0.400pt}}
\put(170.0,849.0){\rule[-0.200pt]{2.409pt}{0.400pt}}
\put(1429.0,849.0){\rule[-0.200pt]{2.409pt}{0.400pt}}
\put(170.0,849.0){\rule[-0.200pt]{2.409pt}{0.400pt}}
\put(1429.0,849.0){\rule[-0.200pt]{2.409pt}{0.400pt}}
\put(170.0,849.0){\rule[-0.200pt]{2.409pt}{0.400pt}}
\put(1429.0,849.0){\rule[-0.200pt]{2.409pt}{0.400pt}}
\put(170.0,849.0){\rule[-0.200pt]{2.409pt}{0.400pt}}
\put(1429.0,849.0){\rule[-0.200pt]{2.409pt}{0.400pt}}
\put(170.0,849.0){\rule[-0.200pt]{2.409pt}{0.400pt}}
\put(1429.0,849.0){\rule[-0.200pt]{2.409pt}{0.400pt}}
\put(170.0,849.0){\rule[-0.200pt]{2.409pt}{0.400pt}}
\put(1429.0,849.0){\rule[-0.200pt]{2.409pt}{0.400pt}}
\put(170.0,849.0){\rule[-0.200pt]{2.409pt}{0.400pt}}
\put(1429.0,849.0){\rule[-0.200pt]{2.409pt}{0.400pt}}
\put(170.0,849.0){\rule[-0.200pt]{2.409pt}{0.400pt}}
\put(1429.0,849.0){\rule[-0.200pt]{2.409pt}{0.400pt}}
\put(170.0,849.0){\rule[-0.200pt]{2.409pt}{0.400pt}}
\put(1429.0,849.0){\rule[-0.200pt]{2.409pt}{0.400pt}}
\put(170.0,849.0){\rule[-0.200pt]{2.409pt}{0.400pt}}
\put(1429.0,849.0){\rule[-0.200pt]{2.409pt}{0.400pt}}
\put(170.0,849.0){\rule[-0.200pt]{2.409pt}{0.400pt}}
\put(1429.0,849.0){\rule[-0.200pt]{2.409pt}{0.400pt}}
\put(170.0,849.0){\rule[-0.200pt]{2.409pt}{0.400pt}}
\put(1429.0,849.0){\rule[-0.200pt]{2.409pt}{0.400pt}}
\put(170.0,849.0){\rule[-0.200pt]{2.409pt}{0.400pt}}
\put(1429.0,849.0){\rule[-0.200pt]{2.409pt}{0.400pt}}
\put(170.0,849.0){\rule[-0.200pt]{2.409pt}{0.400pt}}
\put(1429.0,849.0){\rule[-0.200pt]{2.409pt}{0.400pt}}
\put(170.0,849.0){\rule[-0.200pt]{2.409pt}{0.400pt}}
\put(1429.0,849.0){\rule[-0.200pt]{2.409pt}{0.400pt}}
\put(170.0,849.0){\rule[-0.200pt]{2.409pt}{0.400pt}}
\put(1429.0,849.0){\rule[-0.200pt]{2.409pt}{0.400pt}}
\put(170.0,849.0){\rule[-0.200pt]{2.409pt}{0.400pt}}
\put(1429.0,849.0){\rule[-0.200pt]{2.409pt}{0.400pt}}
\put(170.0,849.0){\rule[-0.200pt]{2.409pt}{0.400pt}}
\put(1429.0,849.0){\rule[-0.200pt]{2.409pt}{0.400pt}}
\put(170.0,849.0){\rule[-0.200pt]{2.409pt}{0.400pt}}
\put(1429.0,849.0){\rule[-0.200pt]{2.409pt}{0.400pt}}
\put(170.0,850.0){\rule[-0.200pt]{2.409pt}{0.400pt}}
\put(1429.0,850.0){\rule[-0.200pt]{2.409pt}{0.400pt}}
\put(170.0,850.0){\rule[-0.200pt]{2.409pt}{0.400pt}}
\put(1429.0,850.0){\rule[-0.200pt]{2.409pt}{0.400pt}}
\put(170.0,850.0){\rule[-0.200pt]{2.409pt}{0.400pt}}
\put(1429.0,850.0){\rule[-0.200pt]{2.409pt}{0.400pt}}
\put(170.0,850.0){\rule[-0.200pt]{2.409pt}{0.400pt}}
\put(1429.0,850.0){\rule[-0.200pt]{2.409pt}{0.400pt}}
\put(170.0,850.0){\rule[-0.200pt]{2.409pt}{0.400pt}}
\put(1429.0,850.0){\rule[-0.200pt]{2.409pt}{0.400pt}}
\put(170.0,850.0){\rule[-0.200pt]{2.409pt}{0.400pt}}
\put(1429.0,850.0){\rule[-0.200pt]{2.409pt}{0.400pt}}
\put(170.0,850.0){\rule[-0.200pt]{2.409pt}{0.400pt}}
\put(1429.0,850.0){\rule[-0.200pt]{2.409pt}{0.400pt}}
\put(170.0,850.0){\rule[-0.200pt]{2.409pt}{0.400pt}}
\put(1429.0,850.0){\rule[-0.200pt]{2.409pt}{0.400pt}}
\put(170.0,850.0){\rule[-0.200pt]{2.409pt}{0.400pt}}
\put(1429.0,850.0){\rule[-0.200pt]{2.409pt}{0.400pt}}
\put(170.0,850.0){\rule[-0.200pt]{2.409pt}{0.400pt}}
\put(1429.0,850.0){\rule[-0.200pt]{2.409pt}{0.400pt}}
\put(170.0,850.0){\rule[-0.200pt]{2.409pt}{0.400pt}}
\put(1429.0,850.0){\rule[-0.200pt]{2.409pt}{0.400pt}}
\put(170.0,850.0){\rule[-0.200pt]{2.409pt}{0.400pt}}
\put(1429.0,850.0){\rule[-0.200pt]{2.409pt}{0.400pt}}
\put(170.0,850.0){\rule[-0.200pt]{2.409pt}{0.400pt}}
\put(1429.0,850.0){\rule[-0.200pt]{2.409pt}{0.400pt}}
\put(170.0,850.0){\rule[-0.200pt]{2.409pt}{0.400pt}}
\put(1429.0,850.0){\rule[-0.200pt]{2.409pt}{0.400pt}}
\put(170.0,850.0){\rule[-0.200pt]{2.409pt}{0.400pt}}
\put(1429.0,850.0){\rule[-0.200pt]{2.409pt}{0.400pt}}
\put(170.0,850.0){\rule[-0.200pt]{2.409pt}{0.400pt}}
\put(1429.0,850.0){\rule[-0.200pt]{2.409pt}{0.400pt}}
\put(170.0,850.0){\rule[-0.200pt]{2.409pt}{0.400pt}}
\put(1429.0,850.0){\rule[-0.200pt]{2.409pt}{0.400pt}}
\put(170.0,850.0){\rule[-0.200pt]{2.409pt}{0.400pt}}
\put(1429.0,850.0){\rule[-0.200pt]{2.409pt}{0.400pt}}
\put(170.0,850.0){\rule[-0.200pt]{2.409pt}{0.400pt}}
\put(1429.0,850.0){\rule[-0.200pt]{2.409pt}{0.400pt}}
\put(170.0,850.0){\rule[-0.200pt]{2.409pt}{0.400pt}}
\put(1429.0,850.0){\rule[-0.200pt]{2.409pt}{0.400pt}}
\put(170.0,850.0){\rule[-0.200pt]{2.409pt}{0.400pt}}
\put(1429.0,850.0){\rule[-0.200pt]{2.409pt}{0.400pt}}
\put(170.0,851.0){\rule[-0.200pt]{2.409pt}{0.400pt}}
\put(1429.0,851.0){\rule[-0.200pt]{2.409pt}{0.400pt}}
\put(170.0,851.0){\rule[-0.200pt]{2.409pt}{0.400pt}}
\put(1429.0,851.0){\rule[-0.200pt]{2.409pt}{0.400pt}}
\put(170.0,851.0){\rule[-0.200pt]{2.409pt}{0.400pt}}
\put(1429.0,851.0){\rule[-0.200pt]{2.409pt}{0.400pt}}
\put(170.0,851.0){\rule[-0.200pt]{2.409pt}{0.400pt}}
\put(1429.0,851.0){\rule[-0.200pt]{2.409pt}{0.400pt}}
\put(170.0,851.0){\rule[-0.200pt]{2.409pt}{0.400pt}}
\put(1429.0,851.0){\rule[-0.200pt]{2.409pt}{0.400pt}}
\put(170.0,851.0){\rule[-0.200pt]{2.409pt}{0.400pt}}
\put(1429.0,851.0){\rule[-0.200pt]{2.409pt}{0.400pt}}
\put(170.0,851.0){\rule[-0.200pt]{2.409pt}{0.400pt}}
\put(1429.0,851.0){\rule[-0.200pt]{2.409pt}{0.400pt}}
\put(170.0,851.0){\rule[-0.200pt]{2.409pt}{0.400pt}}
\put(1429.0,851.0){\rule[-0.200pt]{2.409pt}{0.400pt}}
\put(170.0,851.0){\rule[-0.200pt]{2.409pt}{0.400pt}}
\put(1429.0,851.0){\rule[-0.200pt]{2.409pt}{0.400pt}}
\put(170.0,851.0){\rule[-0.200pt]{2.409pt}{0.400pt}}
\put(1429.0,851.0){\rule[-0.200pt]{2.409pt}{0.400pt}}
\put(170.0,851.0){\rule[-0.200pt]{2.409pt}{0.400pt}}
\put(1429.0,851.0){\rule[-0.200pt]{2.409pt}{0.400pt}}
\put(170.0,851.0){\rule[-0.200pt]{2.409pt}{0.400pt}}
\put(1429.0,851.0){\rule[-0.200pt]{2.409pt}{0.400pt}}
\put(170.0,851.0){\rule[-0.200pt]{2.409pt}{0.400pt}}
\put(1429.0,851.0){\rule[-0.200pt]{2.409pt}{0.400pt}}
\put(170.0,851.0){\rule[-0.200pt]{2.409pt}{0.400pt}}
\put(1429.0,851.0){\rule[-0.200pt]{2.409pt}{0.400pt}}
\put(170.0,851.0){\rule[-0.200pt]{2.409pt}{0.400pt}}
\put(1429.0,851.0){\rule[-0.200pt]{2.409pt}{0.400pt}}
\put(170.0,851.0){\rule[-0.200pt]{2.409pt}{0.400pt}}
\put(1429.0,851.0){\rule[-0.200pt]{2.409pt}{0.400pt}}
\put(170.0,851.0){\rule[-0.200pt]{2.409pt}{0.400pt}}
\put(1429.0,851.0){\rule[-0.200pt]{2.409pt}{0.400pt}}
\put(170.0,851.0){\rule[-0.200pt]{2.409pt}{0.400pt}}
\put(1429.0,851.0){\rule[-0.200pt]{2.409pt}{0.400pt}}
\put(170.0,851.0){\rule[-0.200pt]{2.409pt}{0.400pt}}
\put(1429.0,851.0){\rule[-0.200pt]{2.409pt}{0.400pt}}
\put(170.0,851.0){\rule[-0.200pt]{2.409pt}{0.400pt}}
\put(1429.0,851.0){\rule[-0.200pt]{2.409pt}{0.400pt}}
\put(170.0,851.0){\rule[-0.200pt]{2.409pt}{0.400pt}}
\put(1429.0,851.0){\rule[-0.200pt]{2.409pt}{0.400pt}}
\put(170.0,852.0){\rule[-0.200pt]{2.409pt}{0.400pt}}
\put(1429.0,852.0){\rule[-0.200pt]{2.409pt}{0.400pt}}
\put(170.0,852.0){\rule[-0.200pt]{2.409pt}{0.400pt}}
\put(1429.0,852.0){\rule[-0.200pt]{2.409pt}{0.400pt}}
\put(170.0,852.0){\rule[-0.200pt]{2.409pt}{0.400pt}}
\put(1429.0,852.0){\rule[-0.200pt]{2.409pt}{0.400pt}}
\put(170.0,852.0){\rule[-0.200pt]{2.409pt}{0.400pt}}
\put(1429.0,852.0){\rule[-0.200pt]{2.409pt}{0.400pt}}
\put(170.0,852.0){\rule[-0.200pt]{2.409pt}{0.400pt}}
\put(1429.0,852.0){\rule[-0.200pt]{2.409pt}{0.400pt}}
\put(170.0,852.0){\rule[-0.200pt]{2.409pt}{0.400pt}}
\put(1429.0,852.0){\rule[-0.200pt]{2.409pt}{0.400pt}}
\put(170.0,852.0){\rule[-0.200pt]{2.409pt}{0.400pt}}
\put(1429.0,852.0){\rule[-0.200pt]{2.409pt}{0.400pt}}
\put(170.0,852.0){\rule[-0.200pt]{2.409pt}{0.400pt}}
\put(1429.0,852.0){\rule[-0.200pt]{2.409pt}{0.400pt}}
\put(170.0,852.0){\rule[-0.200pt]{2.409pt}{0.400pt}}
\put(1429.0,852.0){\rule[-0.200pt]{2.409pt}{0.400pt}}
\put(170.0,852.0){\rule[-0.200pt]{2.409pt}{0.400pt}}
\put(1429.0,852.0){\rule[-0.200pt]{2.409pt}{0.400pt}}
\put(170.0,852.0){\rule[-0.200pt]{2.409pt}{0.400pt}}
\put(1429.0,852.0){\rule[-0.200pt]{2.409pt}{0.400pt}}
\put(170.0,852.0){\rule[-0.200pt]{2.409pt}{0.400pt}}
\put(1429.0,852.0){\rule[-0.200pt]{2.409pt}{0.400pt}}
\put(170.0,852.0){\rule[-0.200pt]{2.409pt}{0.400pt}}
\put(1429.0,852.0){\rule[-0.200pt]{2.409pt}{0.400pt}}
\put(170.0,852.0){\rule[-0.200pt]{2.409pt}{0.400pt}}
\put(1429.0,852.0){\rule[-0.200pt]{2.409pt}{0.400pt}}
\put(170.0,852.0){\rule[-0.200pt]{2.409pt}{0.400pt}}
\put(1429.0,852.0){\rule[-0.200pt]{2.409pt}{0.400pt}}
\put(170.0,852.0){\rule[-0.200pt]{2.409pt}{0.400pt}}
\put(1429.0,852.0){\rule[-0.200pt]{2.409pt}{0.400pt}}
\put(170.0,852.0){\rule[-0.200pt]{2.409pt}{0.400pt}}
\put(1429.0,852.0){\rule[-0.200pt]{2.409pt}{0.400pt}}
\put(170.0,852.0){\rule[-0.200pt]{2.409pt}{0.400pt}}
\put(1429.0,852.0){\rule[-0.200pt]{2.409pt}{0.400pt}}
\put(170.0,852.0){\rule[-0.200pt]{2.409pt}{0.400pt}}
\put(1429.0,852.0){\rule[-0.200pt]{2.409pt}{0.400pt}}
\put(170.0,852.0){\rule[-0.200pt]{2.409pt}{0.400pt}}
\put(1429.0,852.0){\rule[-0.200pt]{2.409pt}{0.400pt}}
\put(170.0,852.0){\rule[-0.200pt]{2.409pt}{0.400pt}}
\put(1429.0,852.0){\rule[-0.200pt]{2.409pt}{0.400pt}}
\put(170.0,852.0){\rule[-0.200pt]{2.409pt}{0.400pt}}
\put(1429.0,852.0){\rule[-0.200pt]{2.409pt}{0.400pt}}
\put(170.0,853.0){\rule[-0.200pt]{2.409pt}{0.400pt}}
\put(1429.0,853.0){\rule[-0.200pt]{2.409pt}{0.400pt}}
\put(170.0,853.0){\rule[-0.200pt]{2.409pt}{0.400pt}}
\put(1429.0,853.0){\rule[-0.200pt]{2.409pt}{0.400pt}}
\put(170.0,853.0){\rule[-0.200pt]{2.409pt}{0.400pt}}
\put(1429.0,853.0){\rule[-0.200pt]{2.409pt}{0.400pt}}
\put(170.0,853.0){\rule[-0.200pt]{2.409pt}{0.400pt}}
\put(1429.0,853.0){\rule[-0.200pt]{2.409pt}{0.400pt}}
\put(170.0,853.0){\rule[-0.200pt]{2.409pt}{0.400pt}}
\put(1429.0,853.0){\rule[-0.200pt]{2.409pt}{0.400pt}}
\put(170.0,853.0){\rule[-0.200pt]{2.409pt}{0.400pt}}
\put(1429.0,853.0){\rule[-0.200pt]{2.409pt}{0.400pt}}
\put(170.0,853.0){\rule[-0.200pt]{2.409pt}{0.400pt}}
\put(1429.0,853.0){\rule[-0.200pt]{2.409pt}{0.400pt}}
\put(170.0,853.0){\rule[-0.200pt]{2.409pt}{0.400pt}}
\put(1429.0,853.0){\rule[-0.200pt]{2.409pt}{0.400pt}}
\put(170.0,853.0){\rule[-0.200pt]{2.409pt}{0.400pt}}
\put(1429.0,853.0){\rule[-0.200pt]{2.409pt}{0.400pt}}
\put(170.0,853.0){\rule[-0.200pt]{2.409pt}{0.400pt}}
\put(1429.0,853.0){\rule[-0.200pt]{2.409pt}{0.400pt}}
\put(170.0,853.0){\rule[-0.200pt]{2.409pt}{0.400pt}}
\put(1429.0,853.0){\rule[-0.200pt]{2.409pt}{0.400pt}}
\put(170.0,853.0){\rule[-0.200pt]{2.409pt}{0.400pt}}
\put(1429.0,853.0){\rule[-0.200pt]{2.409pt}{0.400pt}}
\put(170.0,853.0){\rule[-0.200pt]{2.409pt}{0.400pt}}
\put(1429.0,853.0){\rule[-0.200pt]{2.409pt}{0.400pt}}
\put(170.0,853.0){\rule[-0.200pt]{2.409pt}{0.400pt}}
\put(1429.0,853.0){\rule[-0.200pt]{2.409pt}{0.400pt}}
\put(170.0,853.0){\rule[-0.200pt]{2.409pt}{0.400pt}}
\put(1429.0,853.0){\rule[-0.200pt]{2.409pt}{0.400pt}}
\put(170.0,853.0){\rule[-0.200pt]{2.409pt}{0.400pt}}
\put(1429.0,853.0){\rule[-0.200pt]{2.409pt}{0.400pt}}
\put(170.0,853.0){\rule[-0.200pt]{2.409pt}{0.400pt}}
\put(1429.0,853.0){\rule[-0.200pt]{2.409pt}{0.400pt}}
\put(170.0,853.0){\rule[-0.200pt]{2.409pt}{0.400pt}}
\put(1429.0,853.0){\rule[-0.200pt]{2.409pt}{0.400pt}}
\put(170.0,853.0){\rule[-0.200pt]{2.409pt}{0.400pt}}
\put(1429.0,853.0){\rule[-0.200pt]{2.409pt}{0.400pt}}
\put(170.0,853.0){\rule[-0.200pt]{2.409pt}{0.400pt}}
\put(1429.0,853.0){\rule[-0.200pt]{2.409pt}{0.400pt}}
\put(170.0,853.0){\rule[-0.200pt]{2.409pt}{0.400pt}}
\put(1429.0,853.0){\rule[-0.200pt]{2.409pt}{0.400pt}}
\put(170.0,853.0){\rule[-0.200pt]{2.409pt}{0.400pt}}
\put(1429.0,853.0){\rule[-0.200pt]{2.409pt}{0.400pt}}
\put(170.0,853.0){\rule[-0.200pt]{2.409pt}{0.400pt}}
\put(1429.0,853.0){\rule[-0.200pt]{2.409pt}{0.400pt}}
\put(170.0,854.0){\rule[-0.200pt]{2.409pt}{0.400pt}}
\put(1429.0,854.0){\rule[-0.200pt]{2.409pt}{0.400pt}}
\put(170.0,854.0){\rule[-0.200pt]{2.409pt}{0.400pt}}
\put(1429.0,854.0){\rule[-0.200pt]{2.409pt}{0.400pt}}
\put(170.0,854.0){\rule[-0.200pt]{2.409pt}{0.400pt}}
\put(1429.0,854.0){\rule[-0.200pt]{2.409pt}{0.400pt}}
\put(170.0,854.0){\rule[-0.200pt]{2.409pt}{0.400pt}}
\put(1429.0,854.0){\rule[-0.200pt]{2.409pt}{0.400pt}}
\put(170.0,854.0){\rule[-0.200pt]{2.409pt}{0.400pt}}
\put(1429.0,854.0){\rule[-0.200pt]{2.409pt}{0.400pt}}
\put(170.0,854.0){\rule[-0.200pt]{2.409pt}{0.400pt}}
\put(1429.0,854.0){\rule[-0.200pt]{2.409pt}{0.400pt}}
\put(170.0,854.0){\rule[-0.200pt]{2.409pt}{0.400pt}}
\put(1429.0,854.0){\rule[-0.200pt]{2.409pt}{0.400pt}}
\put(170.0,854.0){\rule[-0.200pt]{2.409pt}{0.400pt}}
\put(1429.0,854.0){\rule[-0.200pt]{2.409pt}{0.400pt}}
\put(170.0,854.0){\rule[-0.200pt]{2.409pt}{0.400pt}}
\put(1429.0,854.0){\rule[-0.200pt]{2.409pt}{0.400pt}}
\put(170.0,854.0){\rule[-0.200pt]{2.409pt}{0.400pt}}
\put(1429.0,854.0){\rule[-0.200pt]{2.409pt}{0.400pt}}
\put(170.0,854.0){\rule[-0.200pt]{2.409pt}{0.400pt}}
\put(1429.0,854.0){\rule[-0.200pt]{2.409pt}{0.400pt}}
\put(170.0,854.0){\rule[-0.200pt]{2.409pt}{0.400pt}}
\put(1429.0,854.0){\rule[-0.200pt]{2.409pt}{0.400pt}}
\put(170.0,854.0){\rule[-0.200pt]{2.409pt}{0.400pt}}
\put(1429.0,854.0){\rule[-0.200pt]{2.409pt}{0.400pt}}
\put(170.0,854.0){\rule[-0.200pt]{2.409pt}{0.400pt}}
\put(1429.0,854.0){\rule[-0.200pt]{2.409pt}{0.400pt}}
\put(170.0,854.0){\rule[-0.200pt]{2.409pt}{0.400pt}}
\put(1429.0,854.0){\rule[-0.200pt]{2.409pt}{0.400pt}}
\put(170.0,854.0){\rule[-0.200pt]{2.409pt}{0.400pt}}
\put(1429.0,854.0){\rule[-0.200pt]{2.409pt}{0.400pt}}
\put(170.0,854.0){\rule[-0.200pt]{2.409pt}{0.400pt}}
\put(1429.0,854.0){\rule[-0.200pt]{2.409pt}{0.400pt}}
\put(170.0,854.0){\rule[-0.200pt]{2.409pt}{0.400pt}}
\put(1429.0,854.0){\rule[-0.200pt]{2.409pt}{0.400pt}}
\put(170.0,854.0){\rule[-0.200pt]{2.409pt}{0.400pt}}
\put(1429.0,854.0){\rule[-0.200pt]{2.409pt}{0.400pt}}
\put(170.0,854.0){\rule[-0.200pt]{2.409pt}{0.400pt}}
\put(1429.0,854.0){\rule[-0.200pt]{2.409pt}{0.400pt}}
\put(170.0,854.0){\rule[-0.200pt]{2.409pt}{0.400pt}}
\put(1429.0,854.0){\rule[-0.200pt]{2.409pt}{0.400pt}}
\put(170.0,854.0){\rule[-0.200pt]{2.409pt}{0.400pt}}
\put(1429.0,854.0){\rule[-0.200pt]{2.409pt}{0.400pt}}
\put(170.0,854.0){\rule[-0.200pt]{2.409pt}{0.400pt}}
\put(1429.0,854.0){\rule[-0.200pt]{2.409pt}{0.400pt}}
\put(170.0,855.0){\rule[-0.200pt]{2.409pt}{0.400pt}}
\put(1429.0,855.0){\rule[-0.200pt]{2.409pt}{0.400pt}}
\put(170.0,855.0){\rule[-0.200pt]{2.409pt}{0.400pt}}
\put(1429.0,855.0){\rule[-0.200pt]{2.409pt}{0.400pt}}
\put(170.0,855.0){\rule[-0.200pt]{2.409pt}{0.400pt}}
\put(1429.0,855.0){\rule[-0.200pt]{2.409pt}{0.400pt}}
\put(170.0,855.0){\rule[-0.200pt]{2.409pt}{0.400pt}}
\put(1429.0,855.0){\rule[-0.200pt]{2.409pt}{0.400pt}}
\put(170.0,855.0){\rule[-0.200pt]{2.409pt}{0.400pt}}
\put(1429.0,855.0){\rule[-0.200pt]{2.409pt}{0.400pt}}
\put(170.0,855.0){\rule[-0.200pt]{2.409pt}{0.400pt}}
\put(1429.0,855.0){\rule[-0.200pt]{2.409pt}{0.400pt}}
\put(170.0,855.0){\rule[-0.200pt]{2.409pt}{0.400pt}}
\put(1429.0,855.0){\rule[-0.200pt]{2.409pt}{0.400pt}}
\put(170.0,855.0){\rule[-0.200pt]{2.409pt}{0.400pt}}
\put(1429.0,855.0){\rule[-0.200pt]{2.409pt}{0.400pt}}
\put(170.0,855.0){\rule[-0.200pt]{2.409pt}{0.400pt}}
\put(1429.0,855.0){\rule[-0.200pt]{2.409pt}{0.400pt}}
\put(170.0,855.0){\rule[-0.200pt]{2.409pt}{0.400pt}}
\put(1429.0,855.0){\rule[-0.200pt]{2.409pt}{0.400pt}}
\put(170.0,855.0){\rule[-0.200pt]{2.409pt}{0.400pt}}
\put(1429.0,855.0){\rule[-0.200pt]{2.409pt}{0.400pt}}
\put(170.0,855.0){\rule[-0.200pt]{2.409pt}{0.400pt}}
\put(1429.0,855.0){\rule[-0.200pt]{2.409pt}{0.400pt}}
\put(170.0,855.0){\rule[-0.200pt]{2.409pt}{0.400pt}}
\put(1429.0,855.0){\rule[-0.200pt]{2.409pt}{0.400pt}}
\put(170.0,855.0){\rule[-0.200pt]{2.409pt}{0.400pt}}
\put(1429.0,855.0){\rule[-0.200pt]{2.409pt}{0.400pt}}
\put(170.0,855.0){\rule[-0.200pt]{2.409pt}{0.400pt}}
\put(1429.0,855.0){\rule[-0.200pt]{2.409pt}{0.400pt}}
\put(170.0,855.0){\rule[-0.200pt]{2.409pt}{0.400pt}}
\put(1429.0,855.0){\rule[-0.200pt]{2.409pt}{0.400pt}}
\put(170.0,855.0){\rule[-0.200pt]{2.409pt}{0.400pt}}
\put(1429.0,855.0){\rule[-0.200pt]{2.409pt}{0.400pt}}
\put(170.0,855.0){\rule[-0.200pt]{2.409pt}{0.400pt}}
\put(1429.0,855.0){\rule[-0.200pt]{2.409pt}{0.400pt}}
\put(170.0,855.0){\rule[-0.200pt]{2.409pt}{0.400pt}}
\put(1429.0,855.0){\rule[-0.200pt]{2.409pt}{0.400pt}}
\put(170.0,855.0){\rule[-0.200pt]{2.409pt}{0.400pt}}
\put(1429.0,855.0){\rule[-0.200pt]{2.409pt}{0.400pt}}
\put(170.0,855.0){\rule[-0.200pt]{2.409pt}{0.400pt}}
\put(1429.0,855.0){\rule[-0.200pt]{2.409pt}{0.400pt}}
\put(170.0,855.0){\rule[-0.200pt]{2.409pt}{0.400pt}}
\put(1429.0,855.0){\rule[-0.200pt]{2.409pt}{0.400pt}}
\put(170.0,855.0){\rule[-0.200pt]{2.409pt}{0.400pt}}
\put(1429.0,855.0){\rule[-0.200pt]{2.409pt}{0.400pt}}
\put(170.0,855.0){\rule[-0.200pt]{2.409pt}{0.400pt}}
\put(1429.0,855.0){\rule[-0.200pt]{2.409pt}{0.400pt}}
\put(170.0,856.0){\rule[-0.200pt]{2.409pt}{0.400pt}}
\put(1429.0,856.0){\rule[-0.200pt]{2.409pt}{0.400pt}}
\put(170.0,856.0){\rule[-0.200pt]{2.409pt}{0.400pt}}
\put(1429.0,856.0){\rule[-0.200pt]{2.409pt}{0.400pt}}
\put(170.0,856.0){\rule[-0.200pt]{2.409pt}{0.400pt}}
\put(1429.0,856.0){\rule[-0.200pt]{2.409pt}{0.400pt}}
\put(170.0,856.0){\rule[-0.200pt]{2.409pt}{0.400pt}}
\put(1429.0,856.0){\rule[-0.200pt]{2.409pt}{0.400pt}}
\put(170.0,856.0){\rule[-0.200pt]{2.409pt}{0.400pt}}
\put(1429.0,856.0){\rule[-0.200pt]{2.409pt}{0.400pt}}
\put(170.0,856.0){\rule[-0.200pt]{2.409pt}{0.400pt}}
\put(1429.0,856.0){\rule[-0.200pt]{2.409pt}{0.400pt}}
\put(170.0,856.0){\rule[-0.200pt]{2.409pt}{0.400pt}}
\put(1429.0,856.0){\rule[-0.200pt]{2.409pt}{0.400pt}}
\put(170.0,856.0){\rule[-0.200pt]{2.409pt}{0.400pt}}
\put(1429.0,856.0){\rule[-0.200pt]{2.409pt}{0.400pt}}
\put(170.0,856.0){\rule[-0.200pt]{2.409pt}{0.400pt}}
\put(1429.0,856.0){\rule[-0.200pt]{2.409pt}{0.400pt}}
\put(170.0,856.0){\rule[-0.200pt]{2.409pt}{0.400pt}}
\put(1429.0,856.0){\rule[-0.200pt]{2.409pt}{0.400pt}}
\put(170.0,856.0){\rule[-0.200pt]{2.409pt}{0.400pt}}
\put(1429.0,856.0){\rule[-0.200pt]{2.409pt}{0.400pt}}
\put(170.0,856.0){\rule[-0.200pt]{2.409pt}{0.400pt}}
\put(1429.0,856.0){\rule[-0.200pt]{2.409pt}{0.400pt}}
\put(170.0,856.0){\rule[-0.200pt]{2.409pt}{0.400pt}}
\put(1429.0,856.0){\rule[-0.200pt]{2.409pt}{0.400pt}}
\put(170.0,856.0){\rule[-0.200pt]{2.409pt}{0.400pt}}
\put(1429.0,856.0){\rule[-0.200pt]{2.409pt}{0.400pt}}
\put(170.0,856.0){\rule[-0.200pt]{2.409pt}{0.400pt}}
\put(1429.0,856.0){\rule[-0.200pt]{2.409pt}{0.400pt}}
\put(170.0,856.0){\rule[-0.200pt]{2.409pt}{0.400pt}}
\put(1429.0,856.0){\rule[-0.200pt]{2.409pt}{0.400pt}}
\put(170.0,856.0){\rule[-0.200pt]{2.409pt}{0.400pt}}
\put(1429.0,856.0){\rule[-0.200pt]{2.409pt}{0.400pt}}
\put(170.0,856.0){\rule[-0.200pt]{2.409pt}{0.400pt}}
\put(1429.0,856.0){\rule[-0.200pt]{2.409pt}{0.400pt}}
\put(170.0,856.0){\rule[-0.200pt]{2.409pt}{0.400pt}}
\put(1429.0,856.0){\rule[-0.200pt]{2.409pt}{0.400pt}}
\put(170.0,856.0){\rule[-0.200pt]{2.409pt}{0.400pt}}
\put(1429.0,856.0){\rule[-0.200pt]{2.409pt}{0.400pt}}
\put(170.0,856.0){\rule[-0.200pt]{2.409pt}{0.400pt}}
\put(1429.0,856.0){\rule[-0.200pt]{2.409pt}{0.400pt}}
\put(170.0,856.0){\rule[-0.200pt]{2.409pt}{0.400pt}}
\put(1429.0,856.0){\rule[-0.200pt]{2.409pt}{0.400pt}}
\put(170.0,856.0){\rule[-0.200pt]{2.409pt}{0.400pt}}
\put(1429.0,856.0){\rule[-0.200pt]{2.409pt}{0.400pt}}
\put(170.0,856.0){\rule[-0.200pt]{2.409pt}{0.400pt}}
\put(1429.0,856.0){\rule[-0.200pt]{2.409pt}{0.400pt}}
\put(170.0,856.0){\rule[-0.200pt]{2.409pt}{0.400pt}}
\put(1429.0,856.0){\rule[-0.200pt]{2.409pt}{0.400pt}}
\put(170.0,857.0){\rule[-0.200pt]{2.409pt}{0.400pt}}
\put(1429.0,857.0){\rule[-0.200pt]{2.409pt}{0.400pt}}
\put(170.0,857.0){\rule[-0.200pt]{2.409pt}{0.400pt}}
\put(1429.0,857.0){\rule[-0.200pt]{2.409pt}{0.400pt}}
\put(170.0,857.0){\rule[-0.200pt]{2.409pt}{0.400pt}}
\put(1429.0,857.0){\rule[-0.200pt]{2.409pt}{0.400pt}}
\put(170.0,857.0){\rule[-0.200pt]{2.409pt}{0.400pt}}
\put(1429.0,857.0){\rule[-0.200pt]{2.409pt}{0.400pt}}
\put(170.0,857.0){\rule[-0.200pt]{2.409pt}{0.400pt}}
\put(1429.0,857.0){\rule[-0.200pt]{2.409pt}{0.400pt}}
\put(170.0,857.0){\rule[-0.200pt]{2.409pt}{0.400pt}}
\put(1429.0,857.0){\rule[-0.200pt]{2.409pt}{0.400pt}}
\put(170.0,857.0){\rule[-0.200pt]{2.409pt}{0.400pt}}
\put(1429.0,857.0){\rule[-0.200pt]{2.409pt}{0.400pt}}
\put(170.0,857.0){\rule[-0.200pt]{2.409pt}{0.400pt}}
\put(1429.0,857.0){\rule[-0.200pt]{2.409pt}{0.400pt}}
\put(170.0,857.0){\rule[-0.200pt]{2.409pt}{0.400pt}}
\put(1429.0,857.0){\rule[-0.200pt]{2.409pt}{0.400pt}}
\put(170.0,857.0){\rule[-0.200pt]{2.409pt}{0.400pt}}
\put(1429.0,857.0){\rule[-0.200pt]{2.409pt}{0.400pt}}
\put(170.0,857.0){\rule[-0.200pt]{2.409pt}{0.400pt}}
\put(1429.0,857.0){\rule[-0.200pt]{2.409pt}{0.400pt}}
\put(170.0,857.0){\rule[-0.200pt]{2.409pt}{0.400pt}}
\put(1429.0,857.0){\rule[-0.200pt]{2.409pt}{0.400pt}}
\put(170.0,857.0){\rule[-0.200pt]{2.409pt}{0.400pt}}
\put(1429.0,857.0){\rule[-0.200pt]{2.409pt}{0.400pt}}
\put(170.0,857.0){\rule[-0.200pt]{2.409pt}{0.400pt}}
\put(1429.0,857.0){\rule[-0.200pt]{2.409pt}{0.400pt}}
\put(170.0,857.0){\rule[-0.200pt]{2.409pt}{0.400pt}}
\put(1429.0,857.0){\rule[-0.200pt]{2.409pt}{0.400pt}}
\put(170.0,857.0){\rule[-0.200pt]{2.409pt}{0.400pt}}
\put(1429.0,857.0){\rule[-0.200pt]{2.409pt}{0.400pt}}
\put(170.0,857.0){\rule[-0.200pt]{2.409pt}{0.400pt}}
\put(1429.0,857.0){\rule[-0.200pt]{2.409pt}{0.400pt}}
\put(170.0,857.0){\rule[-0.200pt]{2.409pt}{0.400pt}}
\put(1429.0,857.0){\rule[-0.200pt]{2.409pt}{0.400pt}}
\put(170.0,857.0){\rule[-0.200pt]{2.409pt}{0.400pt}}
\put(1429.0,857.0){\rule[-0.200pt]{2.409pt}{0.400pt}}
\put(170.0,857.0){\rule[-0.200pt]{2.409pt}{0.400pt}}
\put(1429.0,857.0){\rule[-0.200pt]{2.409pt}{0.400pt}}
\put(170.0,857.0){\rule[-0.200pt]{2.409pt}{0.400pt}}
\put(1429.0,857.0){\rule[-0.200pt]{2.409pt}{0.400pt}}
\put(170.0,857.0){\rule[-0.200pt]{2.409pt}{0.400pt}}
\put(1429.0,857.0){\rule[-0.200pt]{2.409pt}{0.400pt}}
\put(170.0,857.0){\rule[-0.200pt]{2.409pt}{0.400pt}}
\put(1429.0,857.0){\rule[-0.200pt]{2.409pt}{0.400pt}}
\put(170.0,857.0){\rule[-0.200pt]{2.409pt}{0.400pt}}
\put(1429.0,857.0){\rule[-0.200pt]{2.409pt}{0.400pt}}
\put(170.0,857.0){\rule[-0.200pt]{2.409pt}{0.400pt}}
\put(1429.0,857.0){\rule[-0.200pt]{2.409pt}{0.400pt}}
\put(170.0,858.0){\rule[-0.200pt]{2.409pt}{0.400pt}}
\put(1429.0,858.0){\rule[-0.200pt]{2.409pt}{0.400pt}}
\put(170.0,858.0){\rule[-0.200pt]{2.409pt}{0.400pt}}
\put(1429.0,858.0){\rule[-0.200pt]{2.409pt}{0.400pt}}
\put(170.0,858.0){\rule[-0.200pt]{2.409pt}{0.400pt}}
\put(1429.0,858.0){\rule[-0.200pt]{2.409pt}{0.400pt}}
\put(170.0,858.0){\rule[-0.200pt]{2.409pt}{0.400pt}}
\put(1429.0,858.0){\rule[-0.200pt]{2.409pt}{0.400pt}}
\put(170.0,858.0){\rule[-0.200pt]{2.409pt}{0.400pt}}
\put(1429.0,858.0){\rule[-0.200pt]{2.409pt}{0.400pt}}
\put(170.0,858.0){\rule[-0.200pt]{2.409pt}{0.400pt}}
\put(1429.0,858.0){\rule[-0.200pt]{2.409pt}{0.400pt}}
\put(170.0,858.0){\rule[-0.200pt]{2.409pt}{0.400pt}}
\put(1429.0,858.0){\rule[-0.200pt]{2.409pt}{0.400pt}}
\put(170.0,858.0){\rule[-0.200pt]{2.409pt}{0.400pt}}
\put(1429.0,858.0){\rule[-0.200pt]{2.409pt}{0.400pt}}
\put(170.0,858.0){\rule[-0.200pt]{2.409pt}{0.400pt}}
\put(1429.0,858.0){\rule[-0.200pt]{2.409pt}{0.400pt}}
\put(170.0,858.0){\rule[-0.200pt]{2.409pt}{0.400pt}}
\put(1429.0,858.0){\rule[-0.200pt]{2.409pt}{0.400pt}}
\put(170.0,858.0){\rule[-0.200pt]{2.409pt}{0.400pt}}
\put(1429.0,858.0){\rule[-0.200pt]{2.409pt}{0.400pt}}
\put(170.0,858.0){\rule[-0.200pt]{2.409pt}{0.400pt}}
\put(1429.0,858.0){\rule[-0.200pt]{2.409pt}{0.400pt}}
\put(170.0,858.0){\rule[-0.200pt]{2.409pt}{0.400pt}}
\put(1429.0,858.0){\rule[-0.200pt]{2.409pt}{0.400pt}}
\put(170.0,858.0){\rule[-0.200pt]{2.409pt}{0.400pt}}
\put(1429.0,858.0){\rule[-0.200pt]{2.409pt}{0.400pt}}
\put(170.0,858.0){\rule[-0.200pt]{2.409pt}{0.400pt}}
\put(1429.0,858.0){\rule[-0.200pt]{2.409pt}{0.400pt}}
\put(170.0,858.0){\rule[-0.200pt]{2.409pt}{0.400pt}}
\put(1429.0,858.0){\rule[-0.200pt]{2.409pt}{0.400pt}}
\put(170.0,858.0){\rule[-0.200pt]{2.409pt}{0.400pt}}
\put(1429.0,858.0){\rule[-0.200pt]{2.409pt}{0.400pt}}
\put(170.0,858.0){\rule[-0.200pt]{2.409pt}{0.400pt}}
\put(1429.0,858.0){\rule[-0.200pt]{2.409pt}{0.400pt}}
\put(170.0,858.0){\rule[-0.200pt]{2.409pt}{0.400pt}}
\put(1429.0,858.0){\rule[-0.200pt]{2.409pt}{0.400pt}}
\put(170.0,858.0){\rule[-0.200pt]{2.409pt}{0.400pt}}
\put(1429.0,858.0){\rule[-0.200pt]{2.409pt}{0.400pt}}
\put(170.0,858.0){\rule[-0.200pt]{2.409pt}{0.400pt}}
\put(1429.0,858.0){\rule[-0.200pt]{2.409pt}{0.400pt}}
\put(170.0,858.0){\rule[-0.200pt]{2.409pt}{0.400pt}}
\put(1429.0,858.0){\rule[-0.200pt]{2.409pt}{0.400pt}}
\put(170.0,858.0){\rule[-0.200pt]{2.409pt}{0.400pt}}
\put(1429.0,858.0){\rule[-0.200pt]{2.409pt}{0.400pt}}
\put(170.0,858.0){\rule[-0.200pt]{2.409pt}{0.400pt}}
\put(1429.0,858.0){\rule[-0.200pt]{2.409pt}{0.400pt}}
\put(170.0,858.0){\rule[-0.200pt]{2.409pt}{0.400pt}}
\put(1429.0,858.0){\rule[-0.200pt]{2.409pt}{0.400pt}}
\put(170.0,858.0){\rule[-0.200pt]{2.409pt}{0.400pt}}
\put(1429.0,858.0){\rule[-0.200pt]{2.409pt}{0.400pt}}
\put(170.0,859.0){\rule[-0.200pt]{2.409pt}{0.400pt}}
\put(1429.0,859.0){\rule[-0.200pt]{2.409pt}{0.400pt}}
\put(170.0,859.0){\rule[-0.200pt]{2.409pt}{0.400pt}}
\put(1429.0,859.0){\rule[-0.200pt]{2.409pt}{0.400pt}}
\put(170.0,859.0){\rule[-0.200pt]{2.409pt}{0.400pt}}
\put(1429.0,859.0){\rule[-0.200pt]{2.409pt}{0.400pt}}
\put(170.0,859.0){\rule[-0.200pt]{2.409pt}{0.400pt}}
\put(1429.0,859.0){\rule[-0.200pt]{2.409pt}{0.400pt}}
\put(170.0,859.0){\rule[-0.200pt]{2.409pt}{0.400pt}}
\put(1429.0,859.0){\rule[-0.200pt]{2.409pt}{0.400pt}}
\put(170.0,859.0){\rule[-0.200pt]{2.409pt}{0.400pt}}
\put(1429.0,859.0){\rule[-0.200pt]{2.409pt}{0.400pt}}
\put(170.0,859.0){\rule[-0.200pt]{2.409pt}{0.400pt}}
\put(1429.0,859.0){\rule[-0.200pt]{2.409pt}{0.400pt}}
\put(170.0,859.0){\rule[-0.200pt]{2.409pt}{0.400pt}}
\put(1429.0,859.0){\rule[-0.200pt]{2.409pt}{0.400pt}}
\put(170.0,859.0){\rule[-0.200pt]{2.409pt}{0.400pt}}
\put(1429.0,859.0){\rule[-0.200pt]{2.409pt}{0.400pt}}
\put(170.0,859.0){\rule[-0.200pt]{2.409pt}{0.400pt}}
\put(1429.0,859.0){\rule[-0.200pt]{2.409pt}{0.400pt}}
\put(170.0,859.0){\rule[-0.200pt]{2.409pt}{0.400pt}}
\put(1429.0,859.0){\rule[-0.200pt]{2.409pt}{0.400pt}}
\put(170.0,859.0){\rule[-0.200pt]{2.409pt}{0.400pt}}
\put(1429.0,859.0){\rule[-0.200pt]{2.409pt}{0.400pt}}
\put(170.0,859.0){\rule[-0.200pt]{2.409pt}{0.400pt}}
\put(1429.0,859.0){\rule[-0.200pt]{2.409pt}{0.400pt}}
\put(170.0,859.0){\rule[-0.200pt]{4.818pt}{0.400pt}}
\put(150,859){\makebox(0,0)[r]{ 1e+09}}
\put(1419.0,859.0){\rule[-0.200pt]{4.818pt}{0.400pt}}
\put(170.0,82.0){\rule[-0.200pt]{0.400pt}{4.818pt}}
\put(170,41){\makebox(0,0){ 0}}
\put(170.0,839.0){\rule[-0.200pt]{0.400pt}{4.818pt}}
\put(487.0,82.0){\rule[-0.200pt]{0.400pt}{4.818pt}}
\put(487,41){\makebox(0,0){ 500}}
\put(487.0,839.0){\rule[-0.200pt]{0.400pt}{4.818pt}}
\put(804.0,82.0){\rule[-0.200pt]{0.400pt}{4.818pt}}
\put(804,41){\makebox(0,0){ 1000}}
\put(804.0,839.0){\rule[-0.200pt]{0.400pt}{4.818pt}}
\put(1122.0,82.0){\rule[-0.200pt]{0.400pt}{4.818pt}}
\put(1122,41){\makebox(0,0){ 1500}}
\put(1122.0,839.0){\rule[-0.200pt]{0.400pt}{4.818pt}}
\put(1439.0,82.0){\rule[-0.200pt]{0.400pt}{4.818pt}}
\put(1439,41){\makebox(0,0){ 2000}}
\put(1439.0,839.0){\rule[-0.200pt]{0.400pt}{4.818pt}}
\put(170.0,82.0){\rule[-0.200pt]{0.400pt}{187.179pt}}
\put(170.0,82.0){\rule[-0.200pt]{305.702pt}{0.400pt}}
\put(1439.0,82.0){\rule[-0.200pt]{0.400pt}{187.179pt}}
\put(170.0,859.0){\rule[-0.200pt]{305.702pt}{0.400pt}}
\put(1279,164){\makebox(0,0)[r]{algorytm naturalny}}
\put(1299.0,164.0){\rule[-0.200pt]{24.090pt}{0.400pt}}
\put(171,82){\usebox{\plotpoint}}
\multiput(171.58,82.00)(0.493,4.224){23}{\rule{0.119pt}{3.392pt}}
\multiput(170.17,82.00)(13.000,99.959){2}{\rule{0.400pt}{1.696pt}}
\multiput(184.58,189.00)(0.493,2.994){23}{\rule{0.119pt}{2.438pt}}
\multiput(183.17,189.00)(13.000,70.939){2}{\rule{0.400pt}{1.219pt}}
\multiput(197.58,265.00)(0.494,1.488){25}{\rule{0.119pt}{1.271pt}}
\multiput(196.17,265.00)(14.000,38.361){2}{\rule{0.400pt}{0.636pt}}
\multiput(211.58,306.00)(0.493,0.734){23}{\rule{0.119pt}{0.685pt}}
\multiput(210.17,306.00)(13.000,17.579){2}{\rule{0.400pt}{0.342pt}}
\multiput(224.58,325.00)(0.493,0.616){23}{\rule{0.119pt}{0.592pt}}
\multiput(223.17,325.00)(13.000,14.771){2}{\rule{0.400pt}{0.296pt}}
\multiput(237.58,341.00)(0.493,0.695){23}{\rule{0.119pt}{0.654pt}}
\multiput(236.17,341.00)(13.000,16.643){2}{\rule{0.400pt}{0.327pt}}
\multiput(250.58,359.00)(0.493,0.655){23}{\rule{0.119pt}{0.623pt}}
\multiput(249.17,359.00)(13.000,15.707){2}{\rule{0.400pt}{0.312pt}}
\multiput(263.58,376.00)(0.493,0.576){23}{\rule{0.119pt}{0.562pt}}
\multiput(262.17,376.00)(13.000,13.834){2}{\rule{0.400pt}{0.281pt}}
\multiput(276.00,391.58)(0.497,0.493){23}{\rule{0.500pt}{0.119pt}}
\multiput(276.00,390.17)(11.962,13.000){2}{\rule{0.250pt}{0.400pt}}
\multiput(289.00,404.58)(0.539,0.492){21}{\rule{0.533pt}{0.119pt}}
\multiput(289.00,403.17)(11.893,12.000){2}{\rule{0.267pt}{0.400pt}}
\multiput(302.00,416.58)(0.590,0.492){19}{\rule{0.573pt}{0.118pt}}
\multiput(302.00,415.17)(11.811,11.000){2}{\rule{0.286pt}{0.400pt}}
\multiput(315.00,427.58)(0.652,0.491){17}{\rule{0.620pt}{0.118pt}}
\multiput(315.00,426.17)(11.713,10.000){2}{\rule{0.310pt}{0.400pt}}
\multiput(328.00,437.58)(0.704,0.491){17}{\rule{0.660pt}{0.118pt}}
\multiput(328.00,436.17)(12.630,10.000){2}{\rule{0.330pt}{0.400pt}}
\multiput(342.00,447.59)(0.728,0.489){15}{\rule{0.678pt}{0.118pt}}
\multiput(342.00,446.17)(11.593,9.000){2}{\rule{0.339pt}{0.400pt}}
\multiput(355.00,456.59)(0.824,0.488){13}{\rule{0.750pt}{0.117pt}}
\multiput(355.00,455.17)(11.443,8.000){2}{\rule{0.375pt}{0.400pt}}
\multiput(368.00,464.59)(0.950,0.485){11}{\rule{0.843pt}{0.117pt}}
\multiput(368.00,463.17)(11.251,7.000){2}{\rule{0.421pt}{0.400pt}}
\multiput(381.00,471.59)(0.824,0.488){13}{\rule{0.750pt}{0.117pt}}
\multiput(381.00,470.17)(11.443,8.000){2}{\rule{0.375pt}{0.400pt}}
\multiput(394.00,479.59)(1.123,0.482){9}{\rule{0.967pt}{0.116pt}}
\multiput(394.00,478.17)(10.994,6.000){2}{\rule{0.483pt}{0.400pt}}
\multiput(407.00,485.59)(0.950,0.485){11}{\rule{0.843pt}{0.117pt}}
\multiput(407.00,484.17)(11.251,7.000){2}{\rule{0.421pt}{0.400pt}}
\multiput(420.00,492.59)(1.123,0.482){9}{\rule{0.967pt}{0.116pt}}
\multiput(420.00,491.17)(10.994,6.000){2}{\rule{0.483pt}{0.400pt}}
\multiput(433.00,498.59)(1.123,0.482){9}{\rule{0.967pt}{0.116pt}}
\multiput(433.00,497.17)(10.994,6.000){2}{\rule{0.483pt}{0.400pt}}
\multiput(446.00,504.59)(1.123,0.482){9}{\rule{0.967pt}{0.116pt}}
\multiput(446.00,503.17)(10.994,6.000){2}{\rule{0.483pt}{0.400pt}}
\multiput(459.00,510.59)(1.489,0.477){7}{\rule{1.220pt}{0.115pt}}
\multiput(459.00,509.17)(11.468,5.000){2}{\rule{0.610pt}{0.400pt}}
\multiput(473.00,515.59)(1.123,0.482){9}{\rule{0.967pt}{0.116pt}}
\multiput(473.00,514.17)(10.994,6.000){2}{\rule{0.483pt}{0.400pt}}
\multiput(486.00,521.59)(1.378,0.477){7}{\rule{1.140pt}{0.115pt}}
\multiput(486.00,520.17)(10.634,5.000){2}{\rule{0.570pt}{0.400pt}}
\multiput(499.00,526.59)(1.378,0.477){7}{\rule{1.140pt}{0.115pt}}
\multiput(499.00,525.17)(10.634,5.000){2}{\rule{0.570pt}{0.400pt}}
\multiput(512.00,531.59)(1.378,0.477){7}{\rule{1.140pt}{0.115pt}}
\multiput(512.00,530.17)(10.634,5.000){2}{\rule{0.570pt}{0.400pt}}
\multiput(525.00,536.60)(1.797,0.468){5}{\rule{1.400pt}{0.113pt}}
\multiput(525.00,535.17)(10.094,4.000){2}{\rule{0.700pt}{0.400pt}}
\multiput(538.00,540.59)(1.378,0.477){7}{\rule{1.140pt}{0.115pt}}
\multiput(538.00,539.17)(10.634,5.000){2}{\rule{0.570pt}{0.400pt}}
\multiput(551.00,545.60)(1.797,0.468){5}{\rule{1.400pt}{0.113pt}}
\multiput(551.00,544.17)(10.094,4.000){2}{\rule{0.700pt}{0.400pt}}
\multiput(564.00,549.60)(1.797,0.468){5}{\rule{1.400pt}{0.113pt}}
\multiput(564.00,548.17)(10.094,4.000){2}{\rule{0.700pt}{0.400pt}}
\multiput(577.00,553.59)(1.378,0.477){7}{\rule{1.140pt}{0.115pt}}
\multiput(577.00,552.17)(10.634,5.000){2}{\rule{0.570pt}{0.400pt}}
\multiput(590.00,558.59)(1.214,0.482){9}{\rule{1.033pt}{0.116pt}}
\multiput(590.00,557.17)(11.855,6.000){2}{\rule{0.517pt}{0.400pt}}
\multiput(604.00,564.59)(0.824,0.488){13}{\rule{0.750pt}{0.117pt}}
\multiput(604.00,563.17)(11.443,8.000){2}{\rule{0.375pt}{0.400pt}}
\multiput(617.00,572.58)(0.590,0.492){19}{\rule{0.573pt}{0.118pt}}
\multiput(617.00,571.17)(11.811,11.000){2}{\rule{0.286pt}{0.400pt}}
\multiput(630.00,583.58)(0.539,0.492){21}{\rule{0.533pt}{0.119pt}}
\multiput(630.00,582.17)(11.893,12.000){2}{\rule{0.267pt}{0.400pt}}
\multiput(643.00,595.58)(0.497,0.493){23}{\rule{0.500pt}{0.119pt}}
\multiput(643.00,594.17)(11.962,13.000){2}{\rule{0.250pt}{0.400pt}}
\multiput(656.00,608.58)(0.590,0.492){19}{\rule{0.573pt}{0.118pt}}
\multiput(656.00,607.17)(11.811,11.000){2}{\rule{0.286pt}{0.400pt}}
\multiput(669.00,619.58)(0.652,0.491){17}{\rule{0.620pt}{0.118pt}}
\multiput(669.00,618.17)(11.713,10.000){2}{\rule{0.310pt}{0.400pt}}
\multiput(682.00,629.59)(0.824,0.488){13}{\rule{0.750pt}{0.117pt}}
\multiput(682.00,628.17)(11.443,8.000){2}{\rule{0.375pt}{0.400pt}}
\multiput(695.00,637.59)(1.123,0.482){9}{\rule{0.967pt}{0.116pt}}
\multiput(695.00,636.17)(10.994,6.000){2}{\rule{0.483pt}{0.400pt}}
\multiput(708.00,643.60)(1.797,0.468){5}{\rule{1.400pt}{0.113pt}}
\multiput(708.00,642.17)(10.094,4.000){2}{\rule{0.700pt}{0.400pt}}
\multiput(721.00,647.60)(1.943,0.468){5}{\rule{1.500pt}{0.113pt}}
\multiput(721.00,646.17)(10.887,4.000){2}{\rule{0.750pt}{0.400pt}}
\multiput(735.00,651.60)(1.797,0.468){5}{\rule{1.400pt}{0.113pt}}
\multiput(735.00,650.17)(10.094,4.000){2}{\rule{0.700pt}{0.400pt}}
\multiput(748.00,655.61)(2.695,0.447){3}{\rule{1.833pt}{0.108pt}}
\multiput(748.00,654.17)(9.195,3.000){2}{\rule{0.917pt}{0.400pt}}
\multiput(761.00,658.61)(2.695,0.447){3}{\rule{1.833pt}{0.108pt}}
\multiput(761.00,657.17)(9.195,3.000){2}{\rule{0.917pt}{0.400pt}}
\multiput(774.00,661.61)(2.695,0.447){3}{\rule{1.833pt}{0.108pt}}
\multiput(774.00,660.17)(9.195,3.000){2}{\rule{0.917pt}{0.400pt}}
\multiput(787.00,664.61)(2.695,0.447){3}{\rule{1.833pt}{0.108pt}}
\multiput(787.00,663.17)(9.195,3.000){2}{\rule{0.917pt}{0.400pt}}
\multiput(800.00,667.61)(2.695,0.447){3}{\rule{1.833pt}{0.108pt}}
\multiput(800.00,666.17)(9.195,3.000){2}{\rule{0.917pt}{0.400pt}}
\multiput(813.00,670.61)(2.695,0.447){3}{\rule{1.833pt}{0.108pt}}
\multiput(813.00,669.17)(9.195,3.000){2}{\rule{0.917pt}{0.400pt}}
\put(826,673.17){\rule{2.700pt}{0.400pt}}
\multiput(826.00,672.17)(7.396,2.000){2}{\rule{1.350pt}{0.400pt}}
\multiput(839.00,675.61)(2.695,0.447){3}{\rule{1.833pt}{0.108pt}}
\multiput(839.00,674.17)(9.195,3.000){2}{\rule{0.917pt}{0.400pt}}
\put(852,678.17){\rule{2.900pt}{0.400pt}}
\multiput(852.00,677.17)(7.981,2.000){2}{\rule{1.450pt}{0.400pt}}
\put(866,680.17){\rule{2.700pt}{0.400pt}}
\multiput(866.00,679.17)(7.396,2.000){2}{\rule{1.350pt}{0.400pt}}
\put(879,682.17){\rule{2.700pt}{0.400pt}}
\multiput(879.00,681.17)(7.396,2.000){2}{\rule{1.350pt}{0.400pt}}
\put(892,684.17){\rule{2.700pt}{0.400pt}}
\multiput(892.00,683.17)(7.396,2.000){2}{\rule{1.350pt}{0.400pt}}
\put(905,686.17){\rule{2.700pt}{0.400pt}}
\multiput(905.00,685.17)(7.396,2.000){2}{\rule{1.350pt}{0.400pt}}
\put(918,688.17){\rule{2.700pt}{0.400pt}}
\multiput(918.00,687.17)(7.396,2.000){2}{\rule{1.350pt}{0.400pt}}
\put(931,690.17){\rule{2.700pt}{0.400pt}}
\multiput(931.00,689.17)(7.396,2.000){2}{\rule{1.350pt}{0.400pt}}
\put(944,692.17){\rule{2.700pt}{0.400pt}}
\multiput(944.00,691.17)(7.396,2.000){2}{\rule{1.350pt}{0.400pt}}
\put(957,694.17){\rule{2.700pt}{0.400pt}}
\multiput(957.00,693.17)(7.396,2.000){2}{\rule{1.350pt}{0.400pt}}
\put(970,696.17){\rule{2.700pt}{0.400pt}}
\multiput(970.00,695.17)(7.396,2.000){2}{\rule{1.350pt}{0.400pt}}
\put(983,698.17){\rule{2.900pt}{0.400pt}}
\multiput(983.00,697.17)(7.981,2.000){2}{\rule{1.450pt}{0.400pt}}
\put(997,699.67){\rule{3.132pt}{0.400pt}}
\multiput(997.00,699.17)(6.500,1.000){2}{\rule{1.566pt}{0.400pt}}
\put(1010,701.17){\rule{2.700pt}{0.400pt}}
\multiput(1010.00,700.17)(7.396,2.000){2}{\rule{1.350pt}{0.400pt}}
\put(1023,703.17){\rule{2.700pt}{0.400pt}}
\multiput(1023.00,702.17)(7.396,2.000){2}{\rule{1.350pt}{0.400pt}}
\put(1036,705.17){\rule{2.700pt}{0.400pt}}
\multiput(1036.00,704.17)(7.396,2.000){2}{\rule{1.350pt}{0.400pt}}
\put(1049,706.67){\rule{3.132pt}{0.400pt}}
\multiput(1049.00,706.17)(6.500,1.000){2}{\rule{1.566pt}{0.400pt}}
\put(1062,708.17){\rule{2.700pt}{0.400pt}}
\multiput(1062.00,707.17)(7.396,2.000){2}{\rule{1.350pt}{0.400pt}}
\put(1075,710.17){\rule{2.700pt}{0.400pt}}
\multiput(1075.00,709.17)(7.396,2.000){2}{\rule{1.350pt}{0.400pt}}
\put(1088,711.67){\rule{3.132pt}{0.400pt}}
\multiput(1088.00,711.17)(6.500,1.000){2}{\rule{1.566pt}{0.400pt}}
\put(1101,713.17){\rule{2.700pt}{0.400pt}}
\multiput(1101.00,712.17)(7.396,2.000){2}{\rule{1.350pt}{0.400pt}}
\put(1114,715.17){\rule{2.900pt}{0.400pt}}
\multiput(1114.00,714.17)(7.981,2.000){2}{\rule{1.450pt}{0.400pt}}
\put(1128,716.67){\rule{3.132pt}{0.400pt}}
\multiput(1128.00,716.17)(6.500,1.000){2}{\rule{1.566pt}{0.400pt}}
\put(1141,718.17){\rule{2.700pt}{0.400pt}}
\multiput(1141.00,717.17)(7.396,2.000){2}{\rule{1.350pt}{0.400pt}}
\put(1154,719.67){\rule{3.132pt}{0.400pt}}
\multiput(1154.00,719.17)(6.500,1.000){2}{\rule{1.566pt}{0.400pt}}
\put(1167,721.17){\rule{2.700pt}{0.400pt}}
\multiput(1167.00,720.17)(7.396,2.000){2}{\rule{1.350pt}{0.400pt}}
\put(1180,722.67){\rule{3.132pt}{0.400pt}}
\multiput(1180.00,722.17)(6.500,1.000){2}{\rule{1.566pt}{0.400pt}}
\put(1193,723.67){\rule{3.132pt}{0.400pt}}
\multiput(1193.00,723.17)(6.500,1.000){2}{\rule{1.566pt}{0.400pt}}
\put(1206,725.17){\rule{2.700pt}{0.400pt}}
\multiput(1206.00,724.17)(7.396,2.000){2}{\rule{1.350pt}{0.400pt}}
\put(1219,726.67){\rule{3.132pt}{0.400pt}}
\multiput(1219.00,726.17)(6.500,1.000){2}{\rule{1.566pt}{0.400pt}}
\put(1232,728.17){\rule{2.700pt}{0.400pt}}
\multiput(1232.00,727.17)(7.396,2.000){2}{\rule{1.350pt}{0.400pt}}
\put(1245,729.67){\rule{3.373pt}{0.400pt}}
\multiput(1245.00,729.17)(7.000,1.000){2}{\rule{1.686pt}{0.400pt}}
\put(1259,731.17){\rule{2.700pt}{0.400pt}}
\multiput(1259.00,730.17)(7.396,2.000){2}{\rule{1.350pt}{0.400pt}}
\put(1272,732.67){\rule{3.132pt}{0.400pt}}
\multiput(1272.00,732.17)(6.500,1.000){2}{\rule{1.566pt}{0.400pt}}
\put(1285,733.67){\rule{3.132pt}{0.400pt}}
\multiput(1285.00,733.17)(6.500,1.000){2}{\rule{1.566pt}{0.400pt}}
\put(1298,735.17){\rule{2.700pt}{0.400pt}}
\multiput(1298.00,734.17)(7.396,2.000){2}{\rule{1.350pt}{0.400pt}}
\put(1311,736.67){\rule{3.132pt}{0.400pt}}
\multiput(1311.00,736.17)(6.500,1.000){2}{\rule{1.566pt}{0.400pt}}
\put(1324,737.67){\rule{3.132pt}{0.400pt}}
\multiput(1324.00,737.17)(6.500,1.000){2}{\rule{1.566pt}{0.400pt}}
\put(1337,739.17){\rule{2.700pt}{0.400pt}}
\multiput(1337.00,738.17)(7.396,2.000){2}{\rule{1.350pt}{0.400pt}}
\put(1350,740.67){\rule{3.132pt}{0.400pt}}
\multiput(1350.00,740.17)(6.500,1.000){2}{\rule{1.566pt}{0.400pt}}
\put(1363,741.67){\rule{3.132pt}{0.400pt}}
\multiput(1363.00,741.17)(6.500,1.000){2}{\rule{1.566pt}{0.400pt}}
\put(1376,743.17){\rule{2.900pt}{0.400pt}}
\multiput(1376.00,742.17)(7.981,2.000){2}{\rule{1.450pt}{0.400pt}}
\put(1390,744.67){\rule{3.132pt}{0.400pt}}
\multiput(1390.00,744.17)(6.500,1.000){2}{\rule{1.566pt}{0.400pt}}
\put(1403,745.67){\rule{3.132pt}{0.400pt}}
\multiput(1403.00,745.17)(6.500,1.000){2}{\rule{1.566pt}{0.400pt}}
\put(1416,746.67){\rule{3.132pt}{0.400pt}}
\multiput(1416.00,746.17)(6.500,1.000){2}{\rule{1.566pt}{0.400pt}}
\put(1429,747.67){\rule{2.409pt}{0.400pt}}
\multiput(1429.00,747.17)(5.000,1.000){2}{\rule{1.204pt}{0.400pt}}
\sbox{\plotpoint}{\rule[-0.500pt]{1.000pt}{1.000pt}}%
\sbox{\plotpoint}{\rule[-0.200pt]{0.400pt}{0.400pt}}%
\put(1279,123){\makebox(0,0)[r]{algorytm z progiem}}
\sbox{\plotpoint}{\rule[-0.500pt]{1.000pt}{1.000pt}}%
\multiput(1299,123)(20.756,0.000){5}{\usebox{\plotpoint}}
\put(1399,123){\usebox{\plotpoint}}
\put(171,134){\usebox{\plotpoint}}
\multiput(171,134)(4.466,20.269){3}{\usebox{\plotpoint}}
\multiput(184,193)(3.591,20.442){4}{\usebox{\plotpoint}}
\multiput(197,267)(6.857,19.590){2}{\usebox{\plotpoint}}
\put(216.01,314.32){\usebox{\plotpoint}}
\put(227.25,331.75){\usebox{\plotpoint}}
\multiput(237,349)(7.227,19.457){2}{\usebox{\plotpoint}}
\multiput(250,384)(8.253,19.044){2}{\usebox{\plotpoint}}
\put(271.79,424.81){\usebox{\plotpoint}}
\put(288.78,435.90){\usebox{\plotpoint}}
\put(306.22,446.90){\usebox{\plotpoint}}
\put(319.97,462.27){\usebox{\plotpoint}}
\put(331.63,479.44){\usebox{\plotpoint}}
\put(343.35,496.56){\usebox{\plotpoint}}
\put(357.61,511.41){\usebox{\plotpoint}}
\put(376.73,519.01){\usebox{\plotpoint}}
\put(397.32,521.26){\usebox{\plotpoint}}
\put(418.05,522.00){\usebox{\plotpoint}}
\put(438.70,523.88){\usebox{\plotpoint}}
\put(458.78,528.93){\usebox{\plotpoint}}
\put(475.74,540.74){\usebox{\plotpoint}}
\put(489.94,555.85){\usebox{\plotpoint}}
\put(503.35,571.68){\usebox{\plotpoint}}
\put(518.37,585.90){\usebox{\plotpoint}}
\put(536.57,595.45){\usebox{\plotpoint}}
\put(557.05,598.47){\usebox{\plotpoint}}
\put(577.75,600.00){\usebox{\plotpoint}}
\put(598.50,600.00){\usebox{\plotpoint}}
\put(619.26,600.00){\usebox{\plotpoint}}
\put(639.98,600.77){\usebox{\plotpoint}}
\put(660.73,601.00){\usebox{\plotpoint}}
\put(681.49,601.00){\usebox{\plotpoint}}
\put(702.24,601.00){\usebox{\plotpoint}}
\put(722.96,602.00){\usebox{\plotpoint}}
\put(743.71,602.00){\usebox{\plotpoint}}
\put(764.28,604.50){\usebox{\plotpoint}}
\put(783.91,610.58){\usebox{\plotpoint}}
\put(801.54,621.42){\usebox{\plotpoint}}
\put(816.51,635.78){\usebox{\plotpoint}}
\put(831.00,650.62){\usebox{\plotpoint}}
\put(847.41,663.17){\usebox{\plotpoint}}
\put(866.13,672.02){\usebox{\plotpoint}}
\put(886.71,674.59){\usebox{\plotpoint}}
\put(907.41,676.00){\usebox{\plotpoint}}
\put(928.17,676.00){\usebox{\plotpoint}}
\put(948.92,676.00){\usebox{\plotpoint}}
\put(969.68,676.00){\usebox{\plotpoint}}
\put(990.43,676.00){\usebox{\plotpoint}}
\put(1011.19,676.00){\usebox{\plotpoint}}
\put(1031.94,676.00){\usebox{\plotpoint}}
\put(1052.70,676.00){\usebox{\plotpoint}}
\put(1073.45,676.00){\usebox{\plotpoint}}
\put(1094.19,676.48){\usebox{\plotpoint}}
\put(1114.93,677.00){\usebox{\plotpoint}}
\put(1135.68,677.00){\usebox{\plotpoint}}
\put(1156.44,677.00){\usebox{\plotpoint}}
\put(1177.19,677.00){\usebox{\plotpoint}}
\put(1197.95,677.00){\usebox{\plotpoint}}
\put(1218.70,677.00){\usebox{\plotpoint}}
\put(1239.46,677.00){\usebox{\plotpoint}}
\put(1260.22,677.00){\usebox{\plotpoint}}
\put(1280.97,677.00){\usebox{\plotpoint}}
\put(1301.73,677.00){\usebox{\plotpoint}}
\put(1322.48,677.00){\usebox{\plotpoint}}
\put(1343.20,678.00){\usebox{\plotpoint}}
\put(1363.95,678.00){\usebox{\plotpoint}}
\put(1384.71,678.00){\usebox{\plotpoint}}
\put(1405.47,678.00){\usebox{\plotpoint}}
\put(1426.22,678.00){\usebox{\plotpoint}}
\put(1439,678){\usebox{\plotpoint}}
\sbox{\plotpoint}{\rule[-0.200pt]{0.400pt}{0.400pt}}%
\put(170.0,82.0){\rule[-0.200pt]{0.400pt}{187.179pt}}
\put(170.0,82.0){\rule[-0.200pt]{305.702pt}{0.400pt}}
\put(1439.0,82.0){\rule[-0.200pt]{0.400pt}{187.179pt}}
\put(170.0,859.0){\rule[-0.200pt]{305.702pt}{0.400pt}}
\end{picture}

\caption{Zależność czasu działania od wielkości macierzy dla metody naturalnej i połączonej}
\end{center}
\end{figure}
Wykres przedstawiony na rysunku \textbf{3.1} pozwolił nam na wyciągnięcie wniosku, że dla badanego przez nas
zakresu wielkości macierzy, mimo że algorytm Strassena ma mniejszą złożoność
obliczeniową, jest wolniejszy od algorytmu naturalnego niemalże tysiąckrotnie. W związku z tym
postanowiliśmy zbadać, czy połączenie obydwu algorytmów nie dałoby znacznie
lepszych wyników. Na podstawie przeprowadzonych przez nas doświadczeń wyznaczyliśmy eksperymentalnie
próg na $n=128$. Oznacza to, że dla mniejszych macierzy (z zakresu
$n \in [2, 128]$) do obliczeń wykorzystujemy naturalny algorytm mnożenia macierzy,
dla reszty zaś algorytm Strassena. Porównaliśmy czas pracy programu będącego
implementują otrzymanej metody z czasem działania programu realizującego metodę
mnożenia macierzy z definicji dla zakresu danych $n \in [2, 2000]$. Nasze przypuszczenia okazały się być słuszne.
Połączenie obydwu algorytmów wymaga krótszego czasu wykonywania obliczeń, co obrazuje wykres z rysunku \textbf{3.2}.
\subsection{Porównanie dokładności algorytmów}
W celu zbadania dokładności metod, obliczyliśmy wartości odpowiednich
współczynników. Pierwszy z nich zadany jest wzorem:
$$\Delta(XX^{-1}-I),$$
gdzie $I$ jest macierzą jednostkową, natomiast $\Delta(X) = \sum_{i=1}^{n}
\sum_{j=1}^{n} x_{ij}^2.$

Aby obliczyć ten wartość zaprezentowanego powyżej współczynnika potrzebowaliśmy macierzy o znanej odwrotności, która zadana jest wzorem. Wykorzystaliśmy jeden z typów macierzy trójdiagonalnych. Niech więc $B = (b_{ij})$ będzie macierzą $n \times n$ zadaną przez
%$$b_{ij}=b_i,  i=j,$$
%$$b_{ij}=\delta_{i, j-1}, i<j,$$
%$$b_{ij}=\delta_{i-1, j}, \and i>j,$$
$$b_{ij} = \left\{\begin{matrix}b_i & \mbox{jeśli } i=j \\\delta_{i, j-1} & \mbox{jeśli } i<j \\\delta_{i-1,j} & \mbox{jeśli } i>j \end{matrix}\right.$$
gdzie $b_i=b_{n-i+1}$ oraz $\delta_{ij}$ to tzw. delta Korneckera. Definiując
$b_kr_{k-1}+r_{k-2}, k = 2, ... , n-1$ i $r=(b_nr_{n-1}+r_{n-2})$, gdzie
$r_0=1, r_1=-b_1$ i określając macierz $C=(c_{ij})$ rozmiaru $n \times n$, gdzie
%$$c_{ij}=r^{-1}r_{i-1}r_{n-j}, \and i \leq j,$$
%$$c_{ij}=c_{ji}, \and i>j,$$
$$c_{ij} = \left\{\begin{matrix}r^{-1}r_{i-j}r_{n-j} & \mbox{jeśli } i \leq j  \\c_{ji} & \mbox{jeśli } i > j \end{matrix}\right.$$
dostajemy macierz odwrotną do macierzy $B$ ($B^{-1}=C$).

Obliczenia przeprowadziliśmy posługując się arytmetykami single oraz double. Rysunki \textbf{3.3} oraz \textbf{3.4} obrazują wyniki naszych doświadczeń. Macierze trójdiagonalne rozmiarów z przedziału $n \in [2,500]$  generowaliśmy losowo. 
\begin{figure}[h!tb]
\begin{center}
% GNUPLOT: LaTeX picture
\setlength{\unitlength}{0.240900pt}
\ifx\plotpoint\undefined\newsavebox{\plotpoint}\fi
\sbox{\plotpoint}{\rule[-0.200pt]{0.400pt}{0.400pt}}%
\begin{picture}(1500,900)(0,0)
\sbox{\plotpoint}{\rule[-0.200pt]{0.400pt}{0.400pt}}%
\put(170.0,82.0){\rule[-0.200pt]{4.818pt}{0.400pt}}
\put(150,82){\makebox(0,0)[r]{ 0}}
\put(1419.0,82.0){\rule[-0.200pt]{4.818pt}{0.400pt}}
\put(170.0,168.0){\rule[-0.200pt]{4.818pt}{0.400pt}}
\put(150,168){\makebox(0,0)[r]{ 1e-16}}
\put(1419.0,168.0){\rule[-0.200pt]{4.818pt}{0.400pt}}
\put(170.0,255.0){\rule[-0.200pt]{4.818pt}{0.400pt}}
\put(150,255){\makebox(0,0)[r]{ 2e-16}}
\put(1419.0,255.0){\rule[-0.200pt]{4.818pt}{0.400pt}}
\put(170.0,341.0){\rule[-0.200pt]{4.818pt}{0.400pt}}
\put(150,341){\makebox(0,0)[r]{ 3e-16}}
\put(1419.0,341.0){\rule[-0.200pt]{4.818pt}{0.400pt}}
\put(170.0,427.0){\rule[-0.200pt]{4.818pt}{0.400pt}}
\put(150,427){\makebox(0,0)[r]{ 4e-16}}
\put(1419.0,427.0){\rule[-0.200pt]{4.818pt}{0.400pt}}
\put(170.0,514.0){\rule[-0.200pt]{4.818pt}{0.400pt}}
\put(150,514){\makebox(0,0)[r]{ 5e-16}}
\put(1419.0,514.0){\rule[-0.200pt]{4.818pt}{0.400pt}}
\put(170.0,600.0){\rule[-0.200pt]{4.818pt}{0.400pt}}
\put(150,600){\makebox(0,0)[r]{ 6e-16}}
\put(1419.0,600.0){\rule[-0.200pt]{4.818pt}{0.400pt}}
\put(170.0,686.0){\rule[-0.200pt]{4.818pt}{0.400pt}}
\put(150,686){\makebox(0,0)[r]{ 7e-16}}
\put(1419.0,686.0){\rule[-0.200pt]{4.818pt}{0.400pt}}
\put(170.0,773.0){\rule[-0.200pt]{4.818pt}{0.400pt}}
\put(150,773){\makebox(0,0)[r]{ 8e-16}}
\put(1419.0,773.0){\rule[-0.200pt]{4.818pt}{0.400pt}}
\put(170.0,859.0){\rule[-0.200pt]{4.818pt}{0.400pt}}
\put(150,859){\makebox(0,0)[r]{ 9e-16}}
\put(1419.0,859.0){\rule[-0.200pt]{4.818pt}{0.400pt}}
\put(170.0,82.0){\rule[-0.200pt]{0.400pt}{4.818pt}}
\put(170,41){\makebox(0,0){ 0}}
\put(170.0,839.0){\rule[-0.200pt]{0.400pt}{4.818pt}}
\put(487.0,82.0){\rule[-0.200pt]{0.400pt}{4.818pt}}
\put(487,41){\makebox(0,0){ 500}}
\put(487.0,839.0){\rule[-0.200pt]{0.400pt}{4.818pt}}
\put(804.0,82.0){\rule[-0.200pt]{0.400pt}{4.818pt}}
\put(804,41){\makebox(0,0){ 1000}}
\put(804.0,839.0){\rule[-0.200pt]{0.400pt}{4.818pt}}
\put(1122.0,82.0){\rule[-0.200pt]{0.400pt}{4.818pt}}
\put(1122,41){\makebox(0,0){ 1500}}
\put(1122.0,839.0){\rule[-0.200pt]{0.400pt}{4.818pt}}
\put(1439.0,82.0){\rule[-0.200pt]{0.400pt}{4.818pt}}
\put(1439,41){\makebox(0,0){ 2000}}
\put(1439.0,839.0){\rule[-0.200pt]{0.400pt}{4.818pt}}
\put(170.0,82.0){\rule[-0.200pt]{0.400pt}{187.179pt}}
\put(170.0,82.0){\rule[-0.200pt]{305.702pt}{0.400pt}}
\put(1439.0,82.0){\rule[-0.200pt]{0.400pt}{187.179pt}}
\put(170.0,859.0){\rule[-0.200pt]{305.702pt}{0.400pt}}
\put(1279,205){\makebox(0,0)[r]{algorytm naturalny}}
\put(1299.0,205.0){\rule[-0.200pt]{24.090pt}{0.400pt}}
\put(171,82){\usebox{\plotpoint}}
\put(171.0,82.0){\rule[-0.200pt]{77.811pt}{0.400pt}}
\sbox{\plotpoint}{\rule[-0.500pt]{1.000pt}{1.000pt}}%
\sbox{\plotpoint}{\rule[-0.200pt]{0.400pt}{0.400pt}}%
\put(1279,164){\makebox(0,0)[r]{algorytm Strassena}}
\sbox{\plotpoint}{\rule[-0.500pt]{1.000pt}{1.000pt}}%
\multiput(1299,164)(20.756,0.000){5}{\usebox{\plotpoint}}
\put(1399,164){\usebox{\plotpoint}}
\put(171,82){\usebox{\plotpoint}}
\put(171.00,82.00){\usebox{\plotpoint}}
\put(191.76,82.00){\usebox{\plotpoint}}
\put(212.51,82.00){\usebox{\plotpoint}}
\put(233.27,82.00){\usebox{\plotpoint}}
\put(253.97,82.32){\usebox{\plotpoint}}
\put(271.24,91.99){\usebox{\plotpoint}}
\put(280.86,110.34){\usebox{\plotpoint}}
\put(289.86,128.86){\usebox{\plotpoint}}
\put(304.67,123.67){\usebox{\plotpoint}}
\put(313.82,105.06){\usebox{\plotpoint}}
\put(325.28,88.04){\usebox{\plotpoint}}
\put(344.87,83.00){\usebox{\plotpoint}}
\put(364.79,88.45){\usebox{\plotpoint}}
\put(385.05,90.32){\usebox{\plotpoint}}
\put(404.89,84.28){\usebox{\plotpoint}}
\put(425.29,84.00){\usebox{\plotpoint}}
\put(445.88,83.00){\usebox{\plotpoint}}
\put(466.40,84.60){\usebox{\plotpoint}}
\put(480.44,99.29){\usebox{\plotpoint}}
\put(491.45,86.19){\usebox{\plotpoint}}
\put(494,83){\usebox{\plotpoint}}
\sbox{\plotpoint}{\rule[-0.600pt]{1.200pt}{1.200pt}}%
\sbox{\plotpoint}{\rule[-0.200pt]{0.400pt}{0.400pt}}%
\put(1279,123){\makebox(0,0)[r]{algorytm z progiem}}
\sbox{\plotpoint}{\rule[-0.600pt]{1.200pt}{1.200pt}}%
\put(1299.0,123.0){\rule[-0.600pt]{24.090pt}{1.200pt}}
\put(171,82){\usebox{\plotpoint}}
\put(171.0,82.0){\rule[-0.600pt]{77.811pt}{1.200pt}}
\sbox{\plotpoint}{\rule[-0.200pt]{0.400pt}{0.400pt}}%
\put(170.0,82.0){\rule[-0.200pt]{0.400pt}{187.179pt}}
\put(170.0,82.0){\rule[-0.200pt]{305.702pt}{0.400pt}}
\put(1439.0,82.0){\rule[-0.200pt]{0.400pt}{187.179pt}}
\put(170.0,859.0){\rule[-0.200pt]{305.702pt}{0.400pt}}
\end{picture}

\caption{Wykres zależności wartości współczynnika $\Delta(XX^{-1}-I)$ od wielkości macierzy w arytmetyce double}
\end{center}
\end{figure}

Po przeanalizowaniu wykresów narzuca się fakt, że wartość wyznacznika dla metody Strassena jest dla niewielkich macierzy $10^3$ razy większa, a dla większych macierzy (od rozmiaru bliskiego $n=200$) nawet milionkrotnie większa od wartości tego samego współczynnika dla pozostałych metod. Świadczy to o tym, że algorytm Strassena jest mniej dokładany niż algorytm standardowy czy stworzony przez nas algorytm połączony.
Rząd błędów, który jest określany przez badany współczynnik odwrotności, dla algorytmu standardowego bowiem mieści się w przedziale $(10^{-33}, 10^{-24})$,
zaś dla algorytmu Strassena wartość wyrażenia $\Delta(XX^{-1}-I)$ osiąga wartości sięgające $10^{-16}$, gdy obliczenia wykonujemy przy użyciu arytmetyki double.
Za mniejszą złożoność obliczeniową płacimy więc sporymi stratami na dokładności.
%***********************************************************************
\begin{figure}[h!tb]
\begin{center}
% GNUPLOT: LaTeX picture
\setlength{\unitlength}{0.240900pt}
\ifx\plotpoint\undefined\newsavebox{\plotpoint}\fi
\sbox{\plotpoint}{\rule[-0.200pt]{0.400pt}{0.400pt}}%
\begin{picture}(1500,900)(0,0)
\sbox{\plotpoint}{\rule[-0.200pt]{0.400pt}{0.400pt}}%
\put(130.0,82.0){\rule[-0.200pt]{4.818pt}{0.400pt}}
\put(110,82){\makebox(0,0)[r]{ 0}}
\put(1419.0,82.0){\rule[-0.200pt]{4.818pt}{0.400pt}}
\put(130.0,179.0){\rule[-0.200pt]{4.818pt}{0.400pt}}
\put(110,179){\makebox(0,0)[r]{ 50}}
\put(1419.0,179.0){\rule[-0.200pt]{4.818pt}{0.400pt}}
\put(130.0,276.0){\rule[-0.200pt]{4.818pt}{0.400pt}}
\put(110,276){\makebox(0,0)[r]{ 100}}
\put(1419.0,276.0){\rule[-0.200pt]{4.818pt}{0.400pt}}
\put(130.0,373.0){\rule[-0.200pt]{4.818pt}{0.400pt}}
\put(110,373){\makebox(0,0)[r]{ 150}}
\put(1419.0,373.0){\rule[-0.200pt]{4.818pt}{0.400pt}}
\put(130.0,470.0){\rule[-0.200pt]{4.818pt}{0.400pt}}
\put(110,470){\makebox(0,0)[r]{ 200}}
\put(1419.0,470.0){\rule[-0.200pt]{4.818pt}{0.400pt}}
\put(130.0,568.0){\rule[-0.200pt]{4.818pt}{0.400pt}}
\put(110,568){\makebox(0,0)[r]{ 250}}
\put(1419.0,568.0){\rule[-0.200pt]{4.818pt}{0.400pt}}
\put(130.0,665.0){\rule[-0.200pt]{4.818pt}{0.400pt}}
\put(110,665){\makebox(0,0)[r]{ 300}}
\put(1419.0,665.0){\rule[-0.200pt]{4.818pt}{0.400pt}}
\put(130.0,762.0){\rule[-0.200pt]{4.818pt}{0.400pt}}
\put(110,762){\makebox(0,0)[r]{ 350}}
\put(1419.0,762.0){\rule[-0.200pt]{4.818pt}{0.400pt}}
\put(130.0,859.0){\rule[-0.200pt]{4.818pt}{0.400pt}}
\put(110,859){\makebox(0,0)[r]{ 400}}
\put(1419.0,859.0){\rule[-0.200pt]{4.818pt}{0.400pt}}
\put(130.0,82.0){\rule[-0.200pt]{0.400pt}{4.818pt}}
\put(130,41){\makebox(0,0){ 0}}
\put(130.0,839.0){\rule[-0.200pt]{0.400pt}{4.818pt}}
\put(457.0,82.0){\rule[-0.200pt]{0.400pt}{4.818pt}}
\put(457,41){\makebox(0,0){ 500}}
\put(457.0,839.0){\rule[-0.200pt]{0.400pt}{4.818pt}}
\put(784.0,82.0){\rule[-0.200pt]{0.400pt}{4.818pt}}
\put(784,41){\makebox(0,0){ 1000}}
\put(784.0,839.0){\rule[-0.200pt]{0.400pt}{4.818pt}}
\put(1112.0,82.0){\rule[-0.200pt]{0.400pt}{4.818pt}}
\put(1112,41){\makebox(0,0){ 1500}}
\put(1112.0,839.0){\rule[-0.200pt]{0.400pt}{4.818pt}}
\put(1439.0,82.0){\rule[-0.200pt]{0.400pt}{4.818pt}}
\put(1439,41){\makebox(0,0){ 2000}}
\put(1439.0,839.0){\rule[-0.200pt]{0.400pt}{4.818pt}}
\put(130.0,82.0){\rule[-0.200pt]{0.400pt}{187.179pt}}
\put(130.0,82.0){\rule[-0.200pt]{315.338pt}{0.400pt}}
\put(1439.0,82.0){\rule[-0.200pt]{0.400pt}{187.179pt}}
\put(130.0,859.0){\rule[-0.200pt]{315.338pt}{0.400pt}}
\put(1279,205){\makebox(0,0)[r]{algorytm naturalny}}
\put(1299.0,205.0){\rule[-0.200pt]{24.090pt}{0.400pt}}
\put(131,82){\usebox{\plotpoint}}
\put(131.0,82.0){\rule[-0.200pt]{80.220pt}{0.400pt}}
\sbox{\plotpoint}{\rule[-0.500pt]{1.000pt}{1.000pt}}%
\sbox{\plotpoint}{\rule[-0.200pt]{0.400pt}{0.400pt}}%
\put(1279,164){\makebox(0,0)[r]{algorytm Strassena}}
\sbox{\plotpoint}{\rule[-0.500pt]{1.000pt}{1.000pt}}%
\multiput(1299,164)(20.756,0.000){5}{\usebox{\plotpoint}}
\put(1399,164){\usebox{\plotpoint}}
\put(131,82){\usebox{\plotpoint}}
\put(131.00,82.00){\usebox{\plotpoint}}
\put(151.76,82.00){\usebox{\plotpoint}}
\put(172.51,82.00){\usebox{\plotpoint}}
\put(193.27,82.00){\usebox{\plotpoint}}
\put(214.02,82.00){\usebox{\plotpoint}}
\put(234.78,82.00){\usebox{\plotpoint}}
\put(255.53,82.00){\usebox{\plotpoint}}
\put(276.29,82.00){\usebox{\plotpoint}}
\put(297.04,82.00){\usebox{\plotpoint}}
\put(317.80,82.00){\usebox{\plotpoint}}
\put(338.48,82.62){\usebox{\plotpoint}}
\put(358.56,86.28){\usebox{\plotpoint}}
\put(371.23,102.28){\usebox{\plotpoint}}
\put(378.91,121.55){\usebox{\plotpoint}}
\put(388.64,139.64){\usebox{\plotpoint}}
\put(399.79,125.63){\usebox{\plotpoint}}
\put(406.23,105.93){\usebox{\plotpoint}}
\put(414.98,87.51){\usebox{\plotpoint}}
\put(434.44,83.15){\usebox{\plotpoint}}
\put(454.62,82.79){\usebox{\plotpoint}}
\put(464,83){\usebox{\plotpoint}}
\sbox{\plotpoint}{\rule[-0.600pt]{1.200pt}{1.200pt}}%
\sbox{\plotpoint}{\rule[-0.200pt]{0.400pt}{0.400pt}}%
\put(1279,123){\makebox(0,0)[r]{algorytm z progiem}}
\sbox{\plotpoint}{\rule[-0.600pt]{1.200pt}{1.200pt}}%
\put(1299.0,123.0){\rule[-0.600pt]{24.090pt}{1.200pt}}
\put(131,82){\usebox{\plotpoint}}
\put(131.0,82.0){\rule[-0.600pt]{80.220pt}{1.200pt}}
\sbox{\plotpoint}{\rule[-0.200pt]{0.400pt}{0.400pt}}%
\put(130.0,82.0){\rule[-0.200pt]{0.400pt}{187.179pt}}
\put(130.0,82.0){\rule[-0.200pt]{315.338pt}{0.400pt}}
\put(1439.0,82.0){\rule[-0.200pt]{0.400pt}{187.179pt}}
\put(130.0,859.0){\rule[-0.200pt]{315.338pt}{0.400pt}}
\end{picture}

\caption{Wykres zależności współczynnika $\Delta(XX^{-1}-I)$ od wielkości macierzy w arytmetyce single}
\end{center}
\end{figure}
Wykres zależności współczynnika $\Delta(XX^{-1}-I)$ dla arytmetyki single jest w kształcie podobny do wykresu z rysunku {\bf 3.3}. Różnica w rzędzie wielkości błędu nie jest więc zależna od wyboru arytmetyki. Zmniejszenie się wartości badanego wyrażenia wynika z tego, że arytmetyka single ma mniejszą precyzję od arytmetyki double.
%***********************************************************************
\begin{figure}[ht!b]
\begin{center}
% GNUPLOT: LaTeX picture
\setlength{\unitlength}{0.240900pt}
\ifx\plotpoint\undefined\newsavebox{\plotpoint}\fi
\sbox{\plotpoint}{\rule[-0.200pt]{0.400pt}{0.400pt}}%
\begin{picture}(1500,900)(0,0)
\sbox{\plotpoint}{\rule[-0.200pt]{0.400pt}{0.400pt}}%
\put(170.0,82.0){\rule[-0.200pt]{4.818pt}{0.400pt}}
\put(150,82){\makebox(0,0)[r]{ 0}}
\put(1419.0,82.0){\rule[-0.200pt]{4.818pt}{0.400pt}}
\put(170.0,168.0){\rule[-0.200pt]{4.818pt}{0.400pt}}
\put(150,168){\makebox(0,0)[r]{ 1e-16}}
\put(1419.0,168.0){\rule[-0.200pt]{4.818pt}{0.400pt}}
\put(170.0,255.0){\rule[-0.200pt]{4.818pt}{0.400pt}}
\put(150,255){\makebox(0,0)[r]{ 2e-16}}
\put(1419.0,255.0){\rule[-0.200pt]{4.818pt}{0.400pt}}
\put(170.0,341.0){\rule[-0.200pt]{4.818pt}{0.400pt}}
\put(150,341){\makebox(0,0)[r]{ 3e-16}}
\put(1419.0,341.0){\rule[-0.200pt]{4.818pt}{0.400pt}}
\put(170.0,427.0){\rule[-0.200pt]{4.818pt}{0.400pt}}
\put(150,427){\makebox(0,0)[r]{ 4e-16}}
\put(1419.0,427.0){\rule[-0.200pt]{4.818pt}{0.400pt}}
\put(170.0,514.0){\rule[-0.200pt]{4.818pt}{0.400pt}}
\put(150,514){\makebox(0,0)[r]{ 5e-16}}
\put(1419.0,514.0){\rule[-0.200pt]{4.818pt}{0.400pt}}
\put(170.0,600.0){\rule[-0.200pt]{4.818pt}{0.400pt}}
\put(150,600){\makebox(0,0)[r]{ 6e-16}}
\put(1419.0,600.0){\rule[-0.200pt]{4.818pt}{0.400pt}}
\put(170.0,686.0){\rule[-0.200pt]{4.818pt}{0.400pt}}
\put(150,686){\makebox(0,0)[r]{ 7e-16}}
\put(1419.0,686.0){\rule[-0.200pt]{4.818pt}{0.400pt}}
\put(170.0,773.0){\rule[-0.200pt]{4.818pt}{0.400pt}}
\put(150,773){\makebox(0,0)[r]{ 8e-16}}
\put(1419.0,773.0){\rule[-0.200pt]{4.818pt}{0.400pt}}
\put(170.0,859.0){\rule[-0.200pt]{4.818pt}{0.400pt}}
\put(150,859){\makebox(0,0)[r]{ 9e-16}}
\put(1419.0,859.0){\rule[-0.200pt]{4.818pt}{0.400pt}}
\put(170.0,82.0){\rule[-0.200pt]{0.400pt}{4.818pt}}
\put(170,41){\makebox(0,0){ 0}}
\put(170.0,839.0){\rule[-0.200pt]{0.400pt}{4.818pt}}
\put(487.0,82.0){\rule[-0.200pt]{0.400pt}{4.818pt}}
\put(487,41){\makebox(0,0){ 500}}
\put(487.0,839.0){\rule[-0.200pt]{0.400pt}{4.818pt}}
\put(804.0,82.0){\rule[-0.200pt]{0.400pt}{4.818pt}}
\put(804,41){\makebox(0,0){ 1000}}
\put(804.0,839.0){\rule[-0.200pt]{0.400pt}{4.818pt}}
\put(1122.0,82.0){\rule[-0.200pt]{0.400pt}{4.818pt}}
\put(1122,41){\makebox(0,0){ 1500}}
\put(1122.0,839.0){\rule[-0.200pt]{0.400pt}{4.818pt}}
\put(1439.0,82.0){\rule[-0.200pt]{0.400pt}{4.818pt}}
\put(1439,41){\makebox(0,0){ 2000}}
\put(1439.0,839.0){\rule[-0.200pt]{0.400pt}{4.818pt}}
\put(170.0,82.0){\rule[-0.200pt]{0.400pt}{187.179pt}}
\put(170.0,82.0){\rule[-0.200pt]{305.702pt}{0.400pt}}
\put(1439.0,82.0){\rule[-0.200pt]{0.400pt}{187.179pt}}
\put(170.0,859.0){\rule[-0.200pt]{305.702pt}{0.400pt}}
\put(1279,205){\makebox(0,0)[r]{algorytm naturlany}}
\put(1299.0,205.0){\rule[-0.200pt]{24.090pt}{0.400pt}}
\put(171,82){\usebox{\plotpoint}}
\put(171.0,82.0){\rule[-0.200pt]{77.811pt}{0.400pt}}
\sbox{\plotpoint}{\rule[-0.500pt]{1.000pt}{1.000pt}}%
\sbox{\plotpoint}{\rule[-0.200pt]{0.400pt}{0.400pt}}%
\put(1279,164){\makebox(0,0)[r]{algorytm Strassena}}
\sbox{\plotpoint}{\rule[-0.500pt]{1.000pt}{1.000pt}}%
\multiput(1299,164)(20.756,0.000){5}{\usebox{\plotpoint}}
\put(1399,164){\usebox{\plotpoint}}
\put(171,82){\usebox{\plotpoint}}
\put(171.00,82.00){\usebox{\plotpoint}}
\put(191.76,82.00){\usebox{\plotpoint}}
\put(212.51,82.00){\usebox{\plotpoint}}
\put(233.27,82.00){\usebox{\plotpoint}}
\put(253.97,82.32){\usebox{\plotpoint}}
\put(270.79,92.38){\usebox{\plotpoint}}
\put(280.39,110.71){\usebox{\plotpoint}}
\put(288.58,129.72){\usebox{\plotpoint}}
\put(303.57,128.86){\usebox{\plotpoint}}
\put(311.69,109.79){\usebox{\plotpoint}}
\put(321.93,91.76){\usebox{\plotpoint}}
\put(339.82,83.00){\usebox{\plotpoint}}
\put(359.97,86.99){\usebox{\plotpoint}}
\put(379.95,92.00){\usebox{\plotpoint}}
\put(399.88,86.71){\usebox{\plotpoint}}
\put(420.02,83.34){\usebox{\plotpoint}}
\put(440.67,84.00){\usebox{\plotpoint}}
\put(461.26,85.00){\usebox{\plotpoint}}
\put(475.93,97.85){\usebox{\plotpoint}}
\put(486.97,101.07){\usebox{\plotpoint}}
\put(494,84){\usebox{\plotpoint}}
\sbox{\plotpoint}{\rule[-0.400pt]{0.800pt}{0.800pt}}%
\sbox{\plotpoint}{\rule[-0.200pt]{0.400pt}{0.400pt}}%
\put(1279,123){\makebox(0,0)[r]{algorytm Strassena}}
\sbox{\plotpoint}{\rule[-0.400pt]{0.800pt}{0.800pt}}%
\put(1299.0,123.0){\rule[-0.400pt]{24.090pt}{0.800pt}}
\put(171,82){\usebox{\plotpoint}}
\put(171.0,82.0){\rule[-0.400pt]{77.811pt}{0.800pt}}
\sbox{\plotpoint}{\rule[-0.200pt]{0.400pt}{0.400pt}}%
\put(170.0,82.0){\rule[-0.200pt]{0.400pt}{187.179pt}}
\put(170.0,82.0){\rule[-0.200pt]{305.702pt}{0.400pt}}
\put(1439.0,82.0){\rule[-0.200pt]{0.400pt}{187.179pt}}
\put(170.0,859.0){\rule[-0.200pt]{305.702pt}{0.400pt}}
\end{picture}

\caption{Wykres zależności współczynnika $\Delta(X^{-1}X-I)$ od wielkości macierzy w arytmetyce double}
\end{center}
\end{figure}

Rysunki {\bf 3.5} oraz {\bf 3.6} stanowią graficzne przedstawienie obliczeń w odpowiednio arytmetyce single i double
współczynnika $\Delta (X^{-1}X - I)$. Porównanie ich z ich odpowiednikami dla wcześniejszego współczynnika
pozwalają nam na wyciągnięcie wniosku, że dokładność zaimplementowanych przez nas
algorytmów jest niezależna od zmiany kolejności obliczeń. Wynika to z faktu, że widocznym jest podobieństwo
powyższych wykresów, a rzędy wielkości  współczynników błędu dla danych arytmetyk nie zmieniły się po zastosowaniu przeciwnej kolejności
wykonywania mnożeń.
%***********************************************************************
\begin{figure}[h!tb]
\begin{center}
% GNUPLOT: LaTeX picture
\setlength{\unitlength}{0.240900pt}
\ifx\plotpoint\undefined\newsavebox{\plotpoint}\fi
\sbox{\plotpoint}{\rule[-0.200pt]{0.400pt}{0.400pt}}%
\begin{picture}(1500,900)(0,0)
\sbox{\plotpoint}{\rule[-0.200pt]{0.400pt}{0.400pt}}%
\put(170.0,82.0){\rule[-0.200pt]{4.818pt}{0.400pt}}
\put(150,82){\makebox(0,0)[r]{ 1e-15}}
\put(1419.0,82.0){\rule[-0.200pt]{4.818pt}{0.400pt}}
\put(170.0,95.0){\rule[-0.200pt]{2.409pt}{0.400pt}}
\put(1429.0,95.0){\rule[-0.200pt]{2.409pt}{0.400pt}}
\put(170.0,112.0){\rule[-0.200pt]{2.409pt}{0.400pt}}
\put(1429.0,112.0){\rule[-0.200pt]{2.409pt}{0.400pt}}
\put(170.0,121.0){\rule[-0.200pt]{2.409pt}{0.400pt}}
\put(1429.0,121.0){\rule[-0.200pt]{2.409pt}{0.400pt}}
\put(170.0,127.0){\rule[-0.200pt]{2.409pt}{0.400pt}}
\put(1429.0,127.0){\rule[-0.200pt]{2.409pt}{0.400pt}}
\put(170.0,131.0){\rule[-0.200pt]{2.409pt}{0.400pt}}
\put(1429.0,131.0){\rule[-0.200pt]{2.409pt}{0.400pt}}
\put(170.0,135.0){\rule[-0.200pt]{2.409pt}{0.400pt}}
\put(1429.0,135.0){\rule[-0.200pt]{2.409pt}{0.400pt}}
\put(170.0,138.0){\rule[-0.200pt]{2.409pt}{0.400pt}}
\put(1429.0,138.0){\rule[-0.200pt]{2.409pt}{0.400pt}}
\put(170.0,141.0){\rule[-0.200pt]{2.409pt}{0.400pt}}
\put(1429.0,141.0){\rule[-0.200pt]{2.409pt}{0.400pt}}
\put(170.0,143.0){\rule[-0.200pt]{2.409pt}{0.400pt}}
\put(1429.0,143.0){\rule[-0.200pt]{2.409pt}{0.400pt}}
\put(170.0,145.0){\rule[-0.200pt]{2.409pt}{0.400pt}}
\put(1429.0,145.0){\rule[-0.200pt]{2.409pt}{0.400pt}}
\put(170.0,147.0){\rule[-0.200pt]{2.409pt}{0.400pt}}
\put(1429.0,147.0){\rule[-0.200pt]{2.409pt}{0.400pt}}
\put(170.0,149.0){\rule[-0.200pt]{2.409pt}{0.400pt}}
\put(1429.0,149.0){\rule[-0.200pt]{2.409pt}{0.400pt}}
\put(170.0,150.0){\rule[-0.200pt]{2.409pt}{0.400pt}}
\put(1429.0,150.0){\rule[-0.200pt]{2.409pt}{0.400pt}}
\put(170.0,152.0){\rule[-0.200pt]{2.409pt}{0.400pt}}
\put(1429.0,152.0){\rule[-0.200pt]{2.409pt}{0.400pt}}
\put(170.0,153.0){\rule[-0.200pt]{2.409pt}{0.400pt}}
\put(1429.0,153.0){\rule[-0.200pt]{2.409pt}{0.400pt}}
\put(170.0,154.0){\rule[-0.200pt]{2.409pt}{0.400pt}}
\put(1429.0,154.0){\rule[-0.200pt]{2.409pt}{0.400pt}}
\put(170.0,155.0){\rule[-0.200pt]{2.409pt}{0.400pt}}
\put(1429.0,155.0){\rule[-0.200pt]{2.409pt}{0.400pt}}
\put(170.0,156.0){\rule[-0.200pt]{2.409pt}{0.400pt}}
\put(1429.0,156.0){\rule[-0.200pt]{2.409pt}{0.400pt}}
\put(170.0,157.0){\rule[-0.200pt]{2.409pt}{0.400pt}}
\put(1429.0,157.0){\rule[-0.200pt]{2.409pt}{0.400pt}}
\put(170.0,158.0){\rule[-0.200pt]{2.409pt}{0.400pt}}
\put(1429.0,158.0){\rule[-0.200pt]{2.409pt}{0.400pt}}
\put(170.0,159.0){\rule[-0.200pt]{2.409pt}{0.400pt}}
\put(1429.0,159.0){\rule[-0.200pt]{2.409pt}{0.400pt}}
\put(170.0,160.0){\rule[-0.200pt]{2.409pt}{0.400pt}}
\put(1429.0,160.0){\rule[-0.200pt]{2.409pt}{0.400pt}}
\put(170.0,161.0){\rule[-0.200pt]{2.409pt}{0.400pt}}
\put(1429.0,161.0){\rule[-0.200pt]{2.409pt}{0.400pt}}
\put(170.0,162.0){\rule[-0.200pt]{2.409pt}{0.400pt}}
\put(1429.0,162.0){\rule[-0.200pt]{2.409pt}{0.400pt}}
\put(170.0,163.0){\rule[-0.200pt]{2.409pt}{0.400pt}}
\put(1429.0,163.0){\rule[-0.200pt]{2.409pt}{0.400pt}}
\put(170.0,163.0){\rule[-0.200pt]{2.409pt}{0.400pt}}
\put(1429.0,163.0){\rule[-0.200pt]{2.409pt}{0.400pt}}
\put(170.0,164.0){\rule[-0.200pt]{2.409pt}{0.400pt}}
\put(1429.0,164.0){\rule[-0.200pt]{2.409pt}{0.400pt}}
\put(170.0,165.0){\rule[-0.200pt]{2.409pt}{0.400pt}}
\put(1429.0,165.0){\rule[-0.200pt]{2.409pt}{0.400pt}}
\put(170.0,166.0){\rule[-0.200pt]{2.409pt}{0.400pt}}
\put(1429.0,166.0){\rule[-0.200pt]{2.409pt}{0.400pt}}
\put(170.0,166.0){\rule[-0.200pt]{2.409pt}{0.400pt}}
\put(1429.0,166.0){\rule[-0.200pt]{2.409pt}{0.400pt}}
\put(170.0,167.0){\rule[-0.200pt]{2.409pt}{0.400pt}}
\put(1429.0,167.0){\rule[-0.200pt]{2.409pt}{0.400pt}}
\put(170.0,167.0){\rule[-0.200pt]{2.409pt}{0.400pt}}
\put(1429.0,167.0){\rule[-0.200pt]{2.409pt}{0.400pt}}
\put(170.0,168.0){\rule[-0.200pt]{2.409pt}{0.400pt}}
\put(1429.0,168.0){\rule[-0.200pt]{2.409pt}{0.400pt}}
\put(170.0,169.0){\rule[-0.200pt]{2.409pt}{0.400pt}}
\put(1429.0,169.0){\rule[-0.200pt]{2.409pt}{0.400pt}}
\put(170.0,169.0){\rule[-0.200pt]{2.409pt}{0.400pt}}
\put(1429.0,169.0){\rule[-0.200pt]{2.409pt}{0.400pt}}
\put(170.0,170.0){\rule[-0.200pt]{2.409pt}{0.400pt}}
\put(1429.0,170.0){\rule[-0.200pt]{2.409pt}{0.400pt}}
\put(170.0,170.0){\rule[-0.200pt]{2.409pt}{0.400pt}}
\put(1429.0,170.0){\rule[-0.200pt]{2.409pt}{0.400pt}}
\put(170.0,171.0){\rule[-0.200pt]{2.409pt}{0.400pt}}
\put(1429.0,171.0){\rule[-0.200pt]{2.409pt}{0.400pt}}
\put(170.0,171.0){\rule[-0.200pt]{2.409pt}{0.400pt}}
\put(1429.0,171.0){\rule[-0.200pt]{2.409pt}{0.400pt}}
\put(170.0,172.0){\rule[-0.200pt]{2.409pt}{0.400pt}}
\put(1429.0,172.0){\rule[-0.200pt]{2.409pt}{0.400pt}}
\put(170.0,172.0){\rule[-0.200pt]{2.409pt}{0.400pt}}
\put(1429.0,172.0){\rule[-0.200pt]{2.409pt}{0.400pt}}
\put(170.0,173.0){\rule[-0.200pt]{2.409pt}{0.400pt}}
\put(1429.0,173.0){\rule[-0.200pt]{2.409pt}{0.400pt}}
\put(170.0,173.0){\rule[-0.200pt]{2.409pt}{0.400pt}}
\put(1429.0,173.0){\rule[-0.200pt]{2.409pt}{0.400pt}}
\put(170.0,173.0){\rule[-0.200pt]{2.409pt}{0.400pt}}
\put(1429.0,173.0){\rule[-0.200pt]{2.409pt}{0.400pt}}
\put(170.0,174.0){\rule[-0.200pt]{2.409pt}{0.400pt}}
\put(1429.0,174.0){\rule[-0.200pt]{2.409pt}{0.400pt}}
\put(170.0,174.0){\rule[-0.200pt]{2.409pt}{0.400pt}}
\put(1429.0,174.0){\rule[-0.200pt]{2.409pt}{0.400pt}}
\put(170.0,175.0){\rule[-0.200pt]{2.409pt}{0.400pt}}
\put(1429.0,175.0){\rule[-0.200pt]{2.409pt}{0.400pt}}
\put(170.0,175.0){\rule[-0.200pt]{2.409pt}{0.400pt}}
\put(1429.0,175.0){\rule[-0.200pt]{2.409pt}{0.400pt}}
\put(170.0,175.0){\rule[-0.200pt]{2.409pt}{0.400pt}}
\put(1429.0,175.0){\rule[-0.200pt]{2.409pt}{0.400pt}}
\put(170.0,176.0){\rule[-0.200pt]{2.409pt}{0.400pt}}
\put(1429.0,176.0){\rule[-0.200pt]{2.409pt}{0.400pt}}
\put(170.0,176.0){\rule[-0.200pt]{2.409pt}{0.400pt}}
\put(1429.0,176.0){\rule[-0.200pt]{2.409pt}{0.400pt}}
\put(170.0,177.0){\rule[-0.200pt]{2.409pt}{0.400pt}}
\put(1429.0,177.0){\rule[-0.200pt]{2.409pt}{0.400pt}}
\put(170.0,177.0){\rule[-0.200pt]{2.409pt}{0.400pt}}
\put(1429.0,177.0){\rule[-0.200pt]{2.409pt}{0.400pt}}
\put(170.0,177.0){\rule[-0.200pt]{2.409pt}{0.400pt}}
\put(1429.0,177.0){\rule[-0.200pt]{2.409pt}{0.400pt}}
\put(170.0,178.0){\rule[-0.200pt]{2.409pt}{0.400pt}}
\put(1429.0,178.0){\rule[-0.200pt]{2.409pt}{0.400pt}}
\put(170.0,178.0){\rule[-0.200pt]{2.409pt}{0.400pt}}
\put(1429.0,178.0){\rule[-0.200pt]{2.409pt}{0.400pt}}
\put(170.0,178.0){\rule[-0.200pt]{2.409pt}{0.400pt}}
\put(1429.0,178.0){\rule[-0.200pt]{2.409pt}{0.400pt}}
\put(170.0,179.0){\rule[-0.200pt]{2.409pt}{0.400pt}}
\put(1429.0,179.0){\rule[-0.200pt]{2.409pt}{0.400pt}}
\put(170.0,179.0){\rule[-0.200pt]{2.409pt}{0.400pt}}
\put(1429.0,179.0){\rule[-0.200pt]{2.409pt}{0.400pt}}
\put(170.0,179.0){\rule[-0.200pt]{2.409pt}{0.400pt}}
\put(1429.0,179.0){\rule[-0.200pt]{2.409pt}{0.400pt}}
\put(170.0,180.0){\rule[-0.200pt]{2.409pt}{0.400pt}}
\put(1429.0,180.0){\rule[-0.200pt]{2.409pt}{0.400pt}}
\put(170.0,180.0){\rule[-0.200pt]{2.409pt}{0.400pt}}
\put(1429.0,180.0){\rule[-0.200pt]{2.409pt}{0.400pt}}
\put(170.0,180.0){\rule[-0.200pt]{2.409pt}{0.400pt}}
\put(1429.0,180.0){\rule[-0.200pt]{2.409pt}{0.400pt}}
\put(170.0,180.0){\rule[-0.200pt]{2.409pt}{0.400pt}}
\put(1429.0,180.0){\rule[-0.200pt]{2.409pt}{0.400pt}}
\put(170.0,181.0){\rule[-0.200pt]{2.409pt}{0.400pt}}
\put(1429.0,181.0){\rule[-0.200pt]{2.409pt}{0.400pt}}
\put(170.0,181.0){\rule[-0.200pt]{2.409pt}{0.400pt}}
\put(1429.0,181.0){\rule[-0.200pt]{2.409pt}{0.400pt}}
\put(170.0,181.0){\rule[-0.200pt]{2.409pt}{0.400pt}}
\put(1429.0,181.0){\rule[-0.200pt]{2.409pt}{0.400pt}}
\put(170.0,182.0){\rule[-0.200pt]{2.409pt}{0.400pt}}
\put(1429.0,182.0){\rule[-0.200pt]{2.409pt}{0.400pt}}
\put(170.0,182.0){\rule[-0.200pt]{2.409pt}{0.400pt}}
\put(1429.0,182.0){\rule[-0.200pt]{2.409pt}{0.400pt}}
\put(170.0,182.0){\rule[-0.200pt]{2.409pt}{0.400pt}}
\put(1429.0,182.0){\rule[-0.200pt]{2.409pt}{0.400pt}}
\put(170.0,182.0){\rule[-0.200pt]{2.409pt}{0.400pt}}
\put(1429.0,182.0){\rule[-0.200pt]{2.409pt}{0.400pt}}
\put(170.0,183.0){\rule[-0.200pt]{2.409pt}{0.400pt}}
\put(1429.0,183.0){\rule[-0.200pt]{2.409pt}{0.400pt}}
\put(170.0,183.0){\rule[-0.200pt]{2.409pt}{0.400pt}}
\put(1429.0,183.0){\rule[-0.200pt]{2.409pt}{0.400pt}}
\put(170.0,183.0){\rule[-0.200pt]{2.409pt}{0.400pt}}
\put(1429.0,183.0){\rule[-0.200pt]{2.409pt}{0.400pt}}
\put(170.0,183.0){\rule[-0.200pt]{2.409pt}{0.400pt}}
\put(1429.0,183.0){\rule[-0.200pt]{2.409pt}{0.400pt}}
\put(170.0,184.0){\rule[-0.200pt]{2.409pt}{0.400pt}}
\put(1429.0,184.0){\rule[-0.200pt]{2.409pt}{0.400pt}}
\put(170.0,184.0){\rule[-0.200pt]{2.409pt}{0.400pt}}
\put(1429.0,184.0){\rule[-0.200pt]{2.409pt}{0.400pt}}
\put(170.0,184.0){\rule[-0.200pt]{2.409pt}{0.400pt}}
\put(1429.0,184.0){\rule[-0.200pt]{2.409pt}{0.400pt}}
\put(170.0,184.0){\rule[-0.200pt]{2.409pt}{0.400pt}}
\put(1429.0,184.0){\rule[-0.200pt]{2.409pt}{0.400pt}}
\put(170.0,185.0){\rule[-0.200pt]{2.409pt}{0.400pt}}
\put(1429.0,185.0){\rule[-0.200pt]{2.409pt}{0.400pt}}
\put(170.0,185.0){\rule[-0.200pt]{2.409pt}{0.400pt}}
\put(1429.0,185.0){\rule[-0.200pt]{2.409pt}{0.400pt}}
\put(170.0,185.0){\rule[-0.200pt]{2.409pt}{0.400pt}}
\put(1429.0,185.0){\rule[-0.200pt]{2.409pt}{0.400pt}}
\put(170.0,185.0){\rule[-0.200pt]{2.409pt}{0.400pt}}
\put(1429.0,185.0){\rule[-0.200pt]{2.409pt}{0.400pt}}
\put(170.0,186.0){\rule[-0.200pt]{2.409pt}{0.400pt}}
\put(1429.0,186.0){\rule[-0.200pt]{2.409pt}{0.400pt}}
\put(170.0,186.0){\rule[-0.200pt]{2.409pt}{0.400pt}}
\put(1429.0,186.0){\rule[-0.200pt]{2.409pt}{0.400pt}}
\put(170.0,186.0){\rule[-0.200pt]{2.409pt}{0.400pt}}
\put(1429.0,186.0){\rule[-0.200pt]{2.409pt}{0.400pt}}
\put(170.0,186.0){\rule[-0.200pt]{2.409pt}{0.400pt}}
\put(1429.0,186.0){\rule[-0.200pt]{2.409pt}{0.400pt}}
\put(170.0,186.0){\rule[-0.200pt]{2.409pt}{0.400pt}}
\put(1429.0,186.0){\rule[-0.200pt]{2.409pt}{0.400pt}}
\put(170.0,187.0){\rule[-0.200pt]{2.409pt}{0.400pt}}
\put(1429.0,187.0){\rule[-0.200pt]{2.409pt}{0.400pt}}
\put(170.0,187.0){\rule[-0.200pt]{2.409pt}{0.400pt}}
\put(1429.0,187.0){\rule[-0.200pt]{2.409pt}{0.400pt}}
\put(170.0,187.0){\rule[-0.200pt]{2.409pt}{0.400pt}}
\put(1429.0,187.0){\rule[-0.200pt]{2.409pt}{0.400pt}}
\put(170.0,187.0){\rule[-0.200pt]{2.409pt}{0.400pt}}
\put(1429.0,187.0){\rule[-0.200pt]{2.409pt}{0.400pt}}
\put(170.0,188.0){\rule[-0.200pt]{2.409pt}{0.400pt}}
\put(1429.0,188.0){\rule[-0.200pt]{2.409pt}{0.400pt}}
\put(170.0,188.0){\rule[-0.200pt]{2.409pt}{0.400pt}}
\put(1429.0,188.0){\rule[-0.200pt]{2.409pt}{0.400pt}}
\put(170.0,188.0){\rule[-0.200pt]{2.409pt}{0.400pt}}
\put(1429.0,188.0){\rule[-0.200pt]{2.409pt}{0.400pt}}
\put(170.0,188.0){\rule[-0.200pt]{2.409pt}{0.400pt}}
\put(1429.0,188.0){\rule[-0.200pt]{2.409pt}{0.400pt}}
\put(170.0,188.0){\rule[-0.200pt]{2.409pt}{0.400pt}}
\put(1429.0,188.0){\rule[-0.200pt]{2.409pt}{0.400pt}}
\put(170.0,188.0){\rule[-0.200pt]{2.409pt}{0.400pt}}
\put(1429.0,188.0){\rule[-0.200pt]{2.409pt}{0.400pt}}
\put(170.0,189.0){\rule[-0.200pt]{2.409pt}{0.400pt}}
\put(1429.0,189.0){\rule[-0.200pt]{2.409pt}{0.400pt}}
\put(170.0,189.0){\rule[-0.200pt]{2.409pt}{0.400pt}}
\put(1429.0,189.0){\rule[-0.200pt]{2.409pt}{0.400pt}}
\put(170.0,189.0){\rule[-0.200pt]{2.409pt}{0.400pt}}
\put(1429.0,189.0){\rule[-0.200pt]{2.409pt}{0.400pt}}
\put(170.0,189.0){\rule[-0.200pt]{2.409pt}{0.400pt}}
\put(1429.0,189.0){\rule[-0.200pt]{2.409pt}{0.400pt}}
\put(170.0,189.0){\rule[-0.200pt]{2.409pt}{0.400pt}}
\put(1429.0,189.0){\rule[-0.200pt]{2.409pt}{0.400pt}}
\put(170.0,190.0){\rule[-0.200pt]{2.409pt}{0.400pt}}
\put(1429.0,190.0){\rule[-0.200pt]{2.409pt}{0.400pt}}
\put(170.0,190.0){\rule[-0.200pt]{2.409pt}{0.400pt}}
\put(1429.0,190.0){\rule[-0.200pt]{2.409pt}{0.400pt}}
\put(170.0,190.0){\rule[-0.200pt]{2.409pt}{0.400pt}}
\put(1429.0,190.0){\rule[-0.200pt]{2.409pt}{0.400pt}}
\put(170.0,190.0){\rule[-0.200pt]{2.409pt}{0.400pt}}
\put(1429.0,190.0){\rule[-0.200pt]{2.409pt}{0.400pt}}
\put(170.0,190.0){\rule[-0.200pt]{2.409pt}{0.400pt}}
\put(1429.0,190.0){\rule[-0.200pt]{2.409pt}{0.400pt}}
\put(170.0,190.0){\rule[-0.200pt]{2.409pt}{0.400pt}}
\put(1429.0,190.0){\rule[-0.200pt]{2.409pt}{0.400pt}}
\put(170.0,191.0){\rule[-0.200pt]{2.409pt}{0.400pt}}
\put(1429.0,191.0){\rule[-0.200pt]{2.409pt}{0.400pt}}
\put(170.0,191.0){\rule[-0.200pt]{2.409pt}{0.400pt}}
\put(1429.0,191.0){\rule[-0.200pt]{2.409pt}{0.400pt}}
\put(170.0,191.0){\rule[-0.200pt]{2.409pt}{0.400pt}}
\put(1429.0,191.0){\rule[-0.200pt]{2.409pt}{0.400pt}}
\put(170.0,191.0){\rule[-0.200pt]{2.409pt}{0.400pt}}
\put(1429.0,191.0){\rule[-0.200pt]{2.409pt}{0.400pt}}
\put(170.0,191.0){\rule[-0.200pt]{2.409pt}{0.400pt}}
\put(1429.0,191.0){\rule[-0.200pt]{2.409pt}{0.400pt}}
\put(170.0,191.0){\rule[-0.200pt]{2.409pt}{0.400pt}}
\put(1429.0,191.0){\rule[-0.200pt]{2.409pt}{0.400pt}}
\put(170.0,192.0){\rule[-0.200pt]{2.409pt}{0.400pt}}
\put(1429.0,192.0){\rule[-0.200pt]{2.409pt}{0.400pt}}
\put(170.0,192.0){\rule[-0.200pt]{2.409pt}{0.400pt}}
\put(1429.0,192.0){\rule[-0.200pt]{2.409pt}{0.400pt}}
\put(170.0,192.0){\rule[-0.200pt]{2.409pt}{0.400pt}}
\put(1429.0,192.0){\rule[-0.200pt]{2.409pt}{0.400pt}}
\put(170.0,192.0){\rule[-0.200pt]{2.409pt}{0.400pt}}
\put(1429.0,192.0){\rule[-0.200pt]{2.409pt}{0.400pt}}
\put(170.0,192.0){\rule[-0.200pt]{2.409pt}{0.400pt}}
\put(1429.0,192.0){\rule[-0.200pt]{2.409pt}{0.400pt}}
\put(170.0,192.0){\rule[-0.200pt]{2.409pt}{0.400pt}}
\put(1429.0,192.0){\rule[-0.200pt]{2.409pt}{0.400pt}}
\put(170.0,193.0){\rule[-0.200pt]{2.409pt}{0.400pt}}
\put(1429.0,193.0){\rule[-0.200pt]{2.409pt}{0.400pt}}
\put(170.0,193.0){\rule[-0.200pt]{2.409pt}{0.400pt}}
\put(1429.0,193.0){\rule[-0.200pt]{2.409pt}{0.400pt}}
\put(170.0,193.0){\rule[-0.200pt]{2.409pt}{0.400pt}}
\put(1429.0,193.0){\rule[-0.200pt]{2.409pt}{0.400pt}}
\put(170.0,193.0){\rule[-0.200pt]{2.409pt}{0.400pt}}
\put(1429.0,193.0){\rule[-0.200pt]{2.409pt}{0.400pt}}
\put(170.0,193.0){\rule[-0.200pt]{2.409pt}{0.400pt}}
\put(1429.0,193.0){\rule[-0.200pt]{2.409pt}{0.400pt}}
\put(170.0,193.0){\rule[-0.200pt]{2.409pt}{0.400pt}}
\put(1429.0,193.0){\rule[-0.200pt]{2.409pt}{0.400pt}}
\put(170.0,194.0){\rule[-0.200pt]{2.409pt}{0.400pt}}
\put(1429.0,194.0){\rule[-0.200pt]{2.409pt}{0.400pt}}
\put(170.0,194.0){\rule[-0.200pt]{2.409pt}{0.400pt}}
\put(1429.0,194.0){\rule[-0.200pt]{2.409pt}{0.400pt}}
\put(170.0,194.0){\rule[-0.200pt]{2.409pt}{0.400pt}}
\put(1429.0,194.0){\rule[-0.200pt]{2.409pt}{0.400pt}}
\put(170.0,194.0){\rule[-0.200pt]{2.409pt}{0.400pt}}
\put(1429.0,194.0){\rule[-0.200pt]{2.409pt}{0.400pt}}
\put(170.0,194.0){\rule[-0.200pt]{2.409pt}{0.400pt}}
\put(1429.0,194.0){\rule[-0.200pt]{2.409pt}{0.400pt}}
\put(170.0,194.0){\rule[-0.200pt]{2.409pt}{0.400pt}}
\put(1429.0,194.0){\rule[-0.200pt]{2.409pt}{0.400pt}}
\put(170.0,194.0){\rule[-0.200pt]{2.409pt}{0.400pt}}
\put(1429.0,194.0){\rule[-0.200pt]{2.409pt}{0.400pt}}
\put(170.0,195.0){\rule[-0.200pt]{2.409pt}{0.400pt}}
\put(1429.0,195.0){\rule[-0.200pt]{2.409pt}{0.400pt}}
\put(170.0,195.0){\rule[-0.200pt]{2.409pt}{0.400pt}}
\put(1429.0,195.0){\rule[-0.200pt]{2.409pt}{0.400pt}}
\put(170.0,195.0){\rule[-0.200pt]{2.409pt}{0.400pt}}
\put(1429.0,195.0){\rule[-0.200pt]{2.409pt}{0.400pt}}
\put(170.0,195.0){\rule[-0.200pt]{2.409pt}{0.400pt}}
\put(1429.0,195.0){\rule[-0.200pt]{2.409pt}{0.400pt}}
\put(170.0,195.0){\rule[-0.200pt]{2.409pt}{0.400pt}}
\put(1429.0,195.0){\rule[-0.200pt]{2.409pt}{0.400pt}}
\put(170.0,195.0){\rule[-0.200pt]{2.409pt}{0.400pt}}
\put(1429.0,195.0){\rule[-0.200pt]{2.409pt}{0.400pt}}
\put(170.0,195.0){\rule[-0.200pt]{2.409pt}{0.400pt}}
\put(1429.0,195.0){\rule[-0.200pt]{2.409pt}{0.400pt}}
\put(170.0,195.0){\rule[-0.200pt]{2.409pt}{0.400pt}}
\put(1429.0,195.0){\rule[-0.200pt]{2.409pt}{0.400pt}}
\put(170.0,196.0){\rule[-0.200pt]{2.409pt}{0.400pt}}
\put(1429.0,196.0){\rule[-0.200pt]{2.409pt}{0.400pt}}
\put(170.0,196.0){\rule[-0.200pt]{2.409pt}{0.400pt}}
\put(1429.0,196.0){\rule[-0.200pt]{2.409pt}{0.400pt}}
\put(170.0,196.0){\rule[-0.200pt]{2.409pt}{0.400pt}}
\put(1429.0,196.0){\rule[-0.200pt]{2.409pt}{0.400pt}}
\put(170.0,196.0){\rule[-0.200pt]{2.409pt}{0.400pt}}
\put(1429.0,196.0){\rule[-0.200pt]{2.409pt}{0.400pt}}
\put(170.0,196.0){\rule[-0.200pt]{2.409pt}{0.400pt}}
\put(1429.0,196.0){\rule[-0.200pt]{2.409pt}{0.400pt}}
\put(170.0,196.0){\rule[-0.200pt]{2.409pt}{0.400pt}}
\put(1429.0,196.0){\rule[-0.200pt]{2.409pt}{0.400pt}}
\put(170.0,196.0){\rule[-0.200pt]{2.409pt}{0.400pt}}
\put(1429.0,196.0){\rule[-0.200pt]{2.409pt}{0.400pt}}
\put(170.0,196.0){\rule[-0.200pt]{2.409pt}{0.400pt}}
\put(1429.0,196.0){\rule[-0.200pt]{2.409pt}{0.400pt}}
\put(170.0,197.0){\rule[-0.200pt]{2.409pt}{0.400pt}}
\put(1429.0,197.0){\rule[-0.200pt]{2.409pt}{0.400pt}}
\put(170.0,197.0){\rule[-0.200pt]{2.409pt}{0.400pt}}
\put(1429.0,197.0){\rule[-0.200pt]{2.409pt}{0.400pt}}
\put(170.0,197.0){\rule[-0.200pt]{2.409pt}{0.400pt}}
\put(1429.0,197.0){\rule[-0.200pt]{2.409pt}{0.400pt}}
\put(170.0,197.0){\rule[-0.200pt]{2.409pt}{0.400pt}}
\put(1429.0,197.0){\rule[-0.200pt]{2.409pt}{0.400pt}}
\put(170.0,197.0){\rule[-0.200pt]{2.409pt}{0.400pt}}
\put(1429.0,197.0){\rule[-0.200pt]{2.409pt}{0.400pt}}
\put(170.0,197.0){\rule[-0.200pt]{2.409pt}{0.400pt}}
\put(1429.0,197.0){\rule[-0.200pt]{2.409pt}{0.400pt}}
\put(170.0,197.0){\rule[-0.200pt]{2.409pt}{0.400pt}}
\put(1429.0,197.0){\rule[-0.200pt]{2.409pt}{0.400pt}}
\put(170.0,197.0){\rule[-0.200pt]{2.409pt}{0.400pt}}
\put(1429.0,197.0){\rule[-0.200pt]{2.409pt}{0.400pt}}
\put(170.0,198.0){\rule[-0.200pt]{2.409pt}{0.400pt}}
\put(1429.0,198.0){\rule[-0.200pt]{2.409pt}{0.400pt}}
\put(170.0,198.0){\rule[-0.200pt]{2.409pt}{0.400pt}}
\put(1429.0,198.0){\rule[-0.200pt]{2.409pt}{0.400pt}}
\put(170.0,198.0){\rule[-0.200pt]{2.409pt}{0.400pt}}
\put(1429.0,198.0){\rule[-0.200pt]{2.409pt}{0.400pt}}
\put(170.0,198.0){\rule[-0.200pt]{2.409pt}{0.400pt}}
\put(1429.0,198.0){\rule[-0.200pt]{2.409pt}{0.400pt}}
\put(170.0,198.0){\rule[-0.200pt]{2.409pt}{0.400pt}}
\put(1429.0,198.0){\rule[-0.200pt]{2.409pt}{0.400pt}}
\put(170.0,198.0){\rule[-0.200pt]{2.409pt}{0.400pt}}
\put(1429.0,198.0){\rule[-0.200pt]{2.409pt}{0.400pt}}
\put(170.0,198.0){\rule[-0.200pt]{2.409pt}{0.400pt}}
\put(1429.0,198.0){\rule[-0.200pt]{2.409pt}{0.400pt}}
\put(170.0,198.0){\rule[-0.200pt]{2.409pt}{0.400pt}}
\put(1429.0,198.0){\rule[-0.200pt]{2.409pt}{0.400pt}}
\put(170.0,199.0){\rule[-0.200pt]{2.409pt}{0.400pt}}
\put(1429.0,199.0){\rule[-0.200pt]{2.409pt}{0.400pt}}
\put(170.0,199.0){\rule[-0.200pt]{2.409pt}{0.400pt}}
\put(1429.0,199.0){\rule[-0.200pt]{2.409pt}{0.400pt}}
\put(170.0,199.0){\rule[-0.200pt]{2.409pt}{0.400pt}}
\put(1429.0,199.0){\rule[-0.200pt]{2.409pt}{0.400pt}}
\put(170.0,199.0){\rule[-0.200pt]{2.409pt}{0.400pt}}
\put(1429.0,199.0){\rule[-0.200pt]{2.409pt}{0.400pt}}
\put(170.0,199.0){\rule[-0.200pt]{2.409pt}{0.400pt}}
\put(1429.0,199.0){\rule[-0.200pt]{2.409pt}{0.400pt}}
\put(170.0,199.0){\rule[-0.200pt]{2.409pt}{0.400pt}}
\put(1429.0,199.0){\rule[-0.200pt]{2.409pt}{0.400pt}}
\put(170.0,199.0){\rule[-0.200pt]{2.409pt}{0.400pt}}
\put(1429.0,199.0){\rule[-0.200pt]{2.409pt}{0.400pt}}
\put(170.0,199.0){\rule[-0.200pt]{2.409pt}{0.400pt}}
\put(1429.0,199.0){\rule[-0.200pt]{2.409pt}{0.400pt}}
\put(170.0,199.0){\rule[-0.200pt]{2.409pt}{0.400pt}}
\put(1429.0,199.0){\rule[-0.200pt]{2.409pt}{0.400pt}}
\put(170.0,199.0){\rule[-0.200pt]{2.409pt}{0.400pt}}
\put(1429.0,199.0){\rule[-0.200pt]{2.409pt}{0.400pt}}
\put(170.0,200.0){\rule[-0.200pt]{2.409pt}{0.400pt}}
\put(1429.0,200.0){\rule[-0.200pt]{2.409pt}{0.400pt}}
\put(170.0,200.0){\rule[-0.200pt]{2.409pt}{0.400pt}}
\put(1429.0,200.0){\rule[-0.200pt]{2.409pt}{0.400pt}}
\put(170.0,200.0){\rule[-0.200pt]{2.409pt}{0.400pt}}
\put(1429.0,200.0){\rule[-0.200pt]{2.409pt}{0.400pt}}
\put(170.0,200.0){\rule[-0.200pt]{2.409pt}{0.400pt}}
\put(1429.0,200.0){\rule[-0.200pt]{2.409pt}{0.400pt}}
\put(170.0,200.0){\rule[-0.200pt]{2.409pt}{0.400pt}}
\put(1429.0,200.0){\rule[-0.200pt]{2.409pt}{0.400pt}}
\put(170.0,200.0){\rule[-0.200pt]{2.409pt}{0.400pt}}
\put(1429.0,200.0){\rule[-0.200pt]{2.409pt}{0.400pt}}
\put(170.0,200.0){\rule[-0.200pt]{2.409pt}{0.400pt}}
\put(1429.0,200.0){\rule[-0.200pt]{2.409pt}{0.400pt}}
\put(170.0,200.0){\rule[-0.200pt]{2.409pt}{0.400pt}}
\put(1429.0,200.0){\rule[-0.200pt]{2.409pt}{0.400pt}}
\put(170.0,200.0){\rule[-0.200pt]{2.409pt}{0.400pt}}
\put(1429.0,200.0){\rule[-0.200pt]{2.409pt}{0.400pt}}
\put(170.0,201.0){\rule[-0.200pt]{2.409pt}{0.400pt}}
\put(1429.0,201.0){\rule[-0.200pt]{2.409pt}{0.400pt}}
\put(170.0,201.0){\rule[-0.200pt]{2.409pt}{0.400pt}}
\put(1429.0,201.0){\rule[-0.200pt]{2.409pt}{0.400pt}}
\put(170.0,201.0){\rule[-0.200pt]{2.409pt}{0.400pt}}
\put(1429.0,201.0){\rule[-0.200pt]{2.409pt}{0.400pt}}
\put(170.0,201.0){\rule[-0.200pt]{2.409pt}{0.400pt}}
\put(1429.0,201.0){\rule[-0.200pt]{2.409pt}{0.400pt}}
\put(170.0,201.0){\rule[-0.200pt]{2.409pt}{0.400pt}}
\put(1429.0,201.0){\rule[-0.200pt]{2.409pt}{0.400pt}}
\put(170.0,201.0){\rule[-0.200pt]{2.409pt}{0.400pt}}
\put(1429.0,201.0){\rule[-0.200pt]{2.409pt}{0.400pt}}
\put(170.0,201.0){\rule[-0.200pt]{2.409pt}{0.400pt}}
\put(1429.0,201.0){\rule[-0.200pt]{2.409pt}{0.400pt}}
\put(170.0,201.0){\rule[-0.200pt]{2.409pt}{0.400pt}}
\put(1429.0,201.0){\rule[-0.200pt]{2.409pt}{0.400pt}}
\put(170.0,201.0){\rule[-0.200pt]{2.409pt}{0.400pt}}
\put(1429.0,201.0){\rule[-0.200pt]{2.409pt}{0.400pt}}
\put(170.0,201.0){\rule[-0.200pt]{2.409pt}{0.400pt}}
\put(1429.0,201.0){\rule[-0.200pt]{2.409pt}{0.400pt}}
\put(170.0,202.0){\rule[-0.200pt]{2.409pt}{0.400pt}}
\put(1429.0,202.0){\rule[-0.200pt]{2.409pt}{0.400pt}}
\put(170.0,202.0){\rule[-0.200pt]{2.409pt}{0.400pt}}
\put(1429.0,202.0){\rule[-0.200pt]{2.409pt}{0.400pt}}
\put(170.0,202.0){\rule[-0.200pt]{2.409pt}{0.400pt}}
\put(1429.0,202.0){\rule[-0.200pt]{2.409pt}{0.400pt}}
\put(170.0,202.0){\rule[-0.200pt]{2.409pt}{0.400pt}}
\put(1429.0,202.0){\rule[-0.200pt]{2.409pt}{0.400pt}}
\put(170.0,202.0){\rule[-0.200pt]{2.409pt}{0.400pt}}
\put(1429.0,202.0){\rule[-0.200pt]{2.409pt}{0.400pt}}
\put(170.0,202.0){\rule[-0.200pt]{2.409pt}{0.400pt}}
\put(1429.0,202.0){\rule[-0.200pt]{2.409pt}{0.400pt}}
\put(170.0,202.0){\rule[-0.200pt]{2.409pt}{0.400pt}}
\put(1429.0,202.0){\rule[-0.200pt]{2.409pt}{0.400pt}}
\put(170.0,202.0){\rule[-0.200pt]{2.409pt}{0.400pt}}
\put(1429.0,202.0){\rule[-0.200pt]{2.409pt}{0.400pt}}
\put(170.0,202.0){\rule[-0.200pt]{2.409pt}{0.400pt}}
\put(1429.0,202.0){\rule[-0.200pt]{2.409pt}{0.400pt}}
\put(170.0,202.0){\rule[-0.200pt]{2.409pt}{0.400pt}}
\put(1429.0,202.0){\rule[-0.200pt]{2.409pt}{0.400pt}}
\put(170.0,202.0){\rule[-0.200pt]{2.409pt}{0.400pt}}
\put(1429.0,202.0){\rule[-0.200pt]{2.409pt}{0.400pt}}
\put(170.0,203.0){\rule[-0.200pt]{2.409pt}{0.400pt}}
\put(1429.0,203.0){\rule[-0.200pt]{2.409pt}{0.400pt}}
\put(170.0,203.0){\rule[-0.200pt]{2.409pt}{0.400pt}}
\put(1429.0,203.0){\rule[-0.200pt]{2.409pt}{0.400pt}}
\put(170.0,203.0){\rule[-0.200pt]{2.409pt}{0.400pt}}
\put(1429.0,203.0){\rule[-0.200pt]{2.409pt}{0.400pt}}
\put(170.0,203.0){\rule[-0.200pt]{2.409pt}{0.400pt}}
\put(1429.0,203.0){\rule[-0.200pt]{2.409pt}{0.400pt}}
\put(170.0,203.0){\rule[-0.200pt]{2.409pt}{0.400pt}}
\put(1429.0,203.0){\rule[-0.200pt]{2.409pt}{0.400pt}}
\put(170.0,203.0){\rule[-0.200pt]{2.409pt}{0.400pt}}
\put(1429.0,203.0){\rule[-0.200pt]{2.409pt}{0.400pt}}
\put(170.0,203.0){\rule[-0.200pt]{2.409pt}{0.400pt}}
\put(1429.0,203.0){\rule[-0.200pt]{2.409pt}{0.400pt}}
\put(170.0,203.0){\rule[-0.200pt]{2.409pt}{0.400pt}}
\put(1429.0,203.0){\rule[-0.200pt]{2.409pt}{0.400pt}}
\put(170.0,203.0){\rule[-0.200pt]{2.409pt}{0.400pt}}
\put(1429.0,203.0){\rule[-0.200pt]{2.409pt}{0.400pt}}
\put(170.0,203.0){\rule[-0.200pt]{2.409pt}{0.400pt}}
\put(1429.0,203.0){\rule[-0.200pt]{2.409pt}{0.400pt}}
\put(170.0,203.0){\rule[-0.200pt]{2.409pt}{0.400pt}}
\put(1429.0,203.0){\rule[-0.200pt]{2.409pt}{0.400pt}}
\put(170.0,204.0){\rule[-0.200pt]{2.409pt}{0.400pt}}
\put(1429.0,204.0){\rule[-0.200pt]{2.409pt}{0.400pt}}
\put(170.0,204.0){\rule[-0.200pt]{2.409pt}{0.400pt}}
\put(1429.0,204.0){\rule[-0.200pt]{2.409pt}{0.400pt}}
\put(170.0,204.0){\rule[-0.200pt]{2.409pt}{0.400pt}}
\put(1429.0,204.0){\rule[-0.200pt]{2.409pt}{0.400pt}}
\put(170.0,204.0){\rule[-0.200pt]{2.409pt}{0.400pt}}
\put(1429.0,204.0){\rule[-0.200pt]{2.409pt}{0.400pt}}
\put(170.0,204.0){\rule[-0.200pt]{2.409pt}{0.400pt}}
\put(1429.0,204.0){\rule[-0.200pt]{2.409pt}{0.400pt}}
\put(170.0,204.0){\rule[-0.200pt]{2.409pt}{0.400pt}}
\put(1429.0,204.0){\rule[-0.200pt]{2.409pt}{0.400pt}}
\put(170.0,204.0){\rule[-0.200pt]{2.409pt}{0.400pt}}
\put(1429.0,204.0){\rule[-0.200pt]{2.409pt}{0.400pt}}
\put(170.0,204.0){\rule[-0.200pt]{2.409pt}{0.400pt}}
\put(1429.0,204.0){\rule[-0.200pt]{2.409pt}{0.400pt}}
\put(170.0,204.0){\rule[-0.200pt]{2.409pt}{0.400pt}}
\put(1429.0,204.0){\rule[-0.200pt]{2.409pt}{0.400pt}}
\put(170.0,204.0){\rule[-0.200pt]{2.409pt}{0.400pt}}
\put(1429.0,204.0){\rule[-0.200pt]{2.409pt}{0.400pt}}
\put(170.0,204.0){\rule[-0.200pt]{2.409pt}{0.400pt}}
\put(1429.0,204.0){\rule[-0.200pt]{2.409pt}{0.400pt}}
\put(170.0,204.0){\rule[-0.200pt]{2.409pt}{0.400pt}}
\put(1429.0,204.0){\rule[-0.200pt]{2.409pt}{0.400pt}}
\put(170.0,205.0){\rule[-0.200pt]{2.409pt}{0.400pt}}
\put(1429.0,205.0){\rule[-0.200pt]{2.409pt}{0.400pt}}
\put(170.0,205.0){\rule[-0.200pt]{2.409pt}{0.400pt}}
\put(1429.0,205.0){\rule[-0.200pt]{2.409pt}{0.400pt}}
\put(170.0,205.0){\rule[-0.200pt]{2.409pt}{0.400pt}}
\put(1429.0,205.0){\rule[-0.200pt]{2.409pt}{0.400pt}}
\put(170.0,205.0){\rule[-0.200pt]{2.409pt}{0.400pt}}
\put(1429.0,205.0){\rule[-0.200pt]{2.409pt}{0.400pt}}
\put(170.0,205.0){\rule[-0.200pt]{2.409pt}{0.400pt}}
\put(1429.0,205.0){\rule[-0.200pt]{2.409pt}{0.400pt}}
\put(170.0,205.0){\rule[-0.200pt]{2.409pt}{0.400pt}}
\put(1429.0,205.0){\rule[-0.200pt]{2.409pt}{0.400pt}}
\put(170.0,205.0){\rule[-0.200pt]{2.409pt}{0.400pt}}
\put(1429.0,205.0){\rule[-0.200pt]{2.409pt}{0.400pt}}
\put(170.0,205.0){\rule[-0.200pt]{2.409pt}{0.400pt}}
\put(1429.0,205.0){\rule[-0.200pt]{2.409pt}{0.400pt}}
\put(170.0,205.0){\rule[-0.200pt]{2.409pt}{0.400pt}}
\put(1429.0,205.0){\rule[-0.200pt]{2.409pt}{0.400pt}}
\put(170.0,205.0){\rule[-0.200pt]{2.409pt}{0.400pt}}
\put(1429.0,205.0){\rule[-0.200pt]{2.409pt}{0.400pt}}
\put(170.0,205.0){\rule[-0.200pt]{2.409pt}{0.400pt}}
\put(1429.0,205.0){\rule[-0.200pt]{2.409pt}{0.400pt}}
\put(170.0,205.0){\rule[-0.200pt]{2.409pt}{0.400pt}}
\put(1429.0,205.0){\rule[-0.200pt]{2.409pt}{0.400pt}}
\put(170.0,205.0){\rule[-0.200pt]{2.409pt}{0.400pt}}
\put(1429.0,205.0){\rule[-0.200pt]{2.409pt}{0.400pt}}
\put(170.0,206.0){\rule[-0.200pt]{2.409pt}{0.400pt}}
\put(1429.0,206.0){\rule[-0.200pt]{2.409pt}{0.400pt}}
\put(170.0,206.0){\rule[-0.200pt]{2.409pt}{0.400pt}}
\put(1429.0,206.0){\rule[-0.200pt]{2.409pt}{0.400pt}}
\put(170.0,206.0){\rule[-0.200pt]{2.409pt}{0.400pt}}
\put(1429.0,206.0){\rule[-0.200pt]{2.409pt}{0.400pt}}
\put(170.0,206.0){\rule[-0.200pt]{2.409pt}{0.400pt}}
\put(1429.0,206.0){\rule[-0.200pt]{2.409pt}{0.400pt}}
\put(170.0,206.0){\rule[-0.200pt]{2.409pt}{0.400pt}}
\put(1429.0,206.0){\rule[-0.200pt]{2.409pt}{0.400pt}}
\put(170.0,206.0){\rule[-0.200pt]{2.409pt}{0.400pt}}
\put(1429.0,206.0){\rule[-0.200pt]{2.409pt}{0.400pt}}
\put(170.0,206.0){\rule[-0.200pt]{2.409pt}{0.400pt}}
\put(1429.0,206.0){\rule[-0.200pt]{2.409pt}{0.400pt}}
\put(170.0,206.0){\rule[-0.200pt]{2.409pt}{0.400pt}}
\put(1429.0,206.0){\rule[-0.200pt]{2.409pt}{0.400pt}}
\put(170.0,206.0){\rule[-0.200pt]{2.409pt}{0.400pt}}
\put(1429.0,206.0){\rule[-0.200pt]{2.409pt}{0.400pt}}
\put(170.0,206.0){\rule[-0.200pt]{2.409pt}{0.400pt}}
\put(1429.0,206.0){\rule[-0.200pt]{2.409pt}{0.400pt}}
\put(170.0,206.0){\rule[-0.200pt]{2.409pt}{0.400pt}}
\put(1429.0,206.0){\rule[-0.200pt]{2.409pt}{0.400pt}}
\put(170.0,206.0){\rule[-0.200pt]{2.409pt}{0.400pt}}
\put(1429.0,206.0){\rule[-0.200pt]{2.409pt}{0.400pt}}
\put(170.0,206.0){\rule[-0.200pt]{2.409pt}{0.400pt}}
\put(1429.0,206.0){\rule[-0.200pt]{2.409pt}{0.400pt}}
\put(170.0,207.0){\rule[-0.200pt]{2.409pt}{0.400pt}}
\put(1429.0,207.0){\rule[-0.200pt]{2.409pt}{0.400pt}}
\put(170.0,207.0){\rule[-0.200pt]{2.409pt}{0.400pt}}
\put(1429.0,207.0){\rule[-0.200pt]{2.409pt}{0.400pt}}
\put(170.0,207.0){\rule[-0.200pt]{2.409pt}{0.400pt}}
\put(1429.0,207.0){\rule[-0.200pt]{2.409pt}{0.400pt}}
\put(170.0,207.0){\rule[-0.200pt]{2.409pt}{0.400pt}}
\put(1429.0,207.0){\rule[-0.200pt]{2.409pt}{0.400pt}}
\put(170.0,207.0){\rule[-0.200pt]{2.409pt}{0.400pt}}
\put(1429.0,207.0){\rule[-0.200pt]{2.409pt}{0.400pt}}
\put(170.0,207.0){\rule[-0.200pt]{2.409pt}{0.400pt}}
\put(1429.0,207.0){\rule[-0.200pt]{2.409pt}{0.400pt}}
\put(170.0,207.0){\rule[-0.200pt]{2.409pt}{0.400pt}}
\put(1429.0,207.0){\rule[-0.200pt]{2.409pt}{0.400pt}}
\put(170.0,207.0){\rule[-0.200pt]{2.409pt}{0.400pt}}
\put(1429.0,207.0){\rule[-0.200pt]{2.409pt}{0.400pt}}
\put(170.0,207.0){\rule[-0.200pt]{2.409pt}{0.400pt}}
\put(1429.0,207.0){\rule[-0.200pt]{2.409pt}{0.400pt}}
\put(170.0,207.0){\rule[-0.200pt]{2.409pt}{0.400pt}}
\put(1429.0,207.0){\rule[-0.200pt]{2.409pt}{0.400pt}}
\put(170.0,207.0){\rule[-0.200pt]{2.409pt}{0.400pt}}
\put(1429.0,207.0){\rule[-0.200pt]{2.409pt}{0.400pt}}
\put(170.0,207.0){\rule[-0.200pt]{2.409pt}{0.400pt}}
\put(1429.0,207.0){\rule[-0.200pt]{2.409pt}{0.400pt}}
\put(170.0,207.0){\rule[-0.200pt]{2.409pt}{0.400pt}}
\put(1429.0,207.0){\rule[-0.200pt]{2.409pt}{0.400pt}}
\put(170.0,207.0){\rule[-0.200pt]{2.409pt}{0.400pt}}
\put(1429.0,207.0){\rule[-0.200pt]{2.409pt}{0.400pt}}
\put(170.0,208.0){\rule[-0.200pt]{2.409pt}{0.400pt}}
\put(1429.0,208.0){\rule[-0.200pt]{2.409pt}{0.400pt}}
\put(170.0,208.0){\rule[-0.200pt]{2.409pt}{0.400pt}}
\put(1429.0,208.0){\rule[-0.200pt]{2.409pt}{0.400pt}}
\put(170.0,208.0){\rule[-0.200pt]{2.409pt}{0.400pt}}
\put(1429.0,208.0){\rule[-0.200pt]{2.409pt}{0.400pt}}
\put(170.0,208.0){\rule[-0.200pt]{2.409pt}{0.400pt}}
\put(1429.0,208.0){\rule[-0.200pt]{2.409pt}{0.400pt}}
\put(170.0,208.0){\rule[-0.200pt]{2.409pt}{0.400pt}}
\put(1429.0,208.0){\rule[-0.200pt]{2.409pt}{0.400pt}}
\put(170.0,208.0){\rule[-0.200pt]{2.409pt}{0.400pt}}
\put(1429.0,208.0){\rule[-0.200pt]{2.409pt}{0.400pt}}
\put(170.0,208.0){\rule[-0.200pt]{2.409pt}{0.400pt}}
\put(1429.0,208.0){\rule[-0.200pt]{2.409pt}{0.400pt}}
\put(170.0,208.0){\rule[-0.200pt]{2.409pt}{0.400pt}}
\put(1429.0,208.0){\rule[-0.200pt]{2.409pt}{0.400pt}}
\put(170.0,208.0){\rule[-0.200pt]{2.409pt}{0.400pt}}
\put(1429.0,208.0){\rule[-0.200pt]{2.409pt}{0.400pt}}
\put(170.0,208.0){\rule[-0.200pt]{2.409pt}{0.400pt}}
\put(1429.0,208.0){\rule[-0.200pt]{2.409pt}{0.400pt}}
\put(170.0,208.0){\rule[-0.200pt]{2.409pt}{0.400pt}}
\put(1429.0,208.0){\rule[-0.200pt]{2.409pt}{0.400pt}}
\put(170.0,208.0){\rule[-0.200pt]{2.409pt}{0.400pt}}
\put(1429.0,208.0){\rule[-0.200pt]{2.409pt}{0.400pt}}
\put(170.0,208.0){\rule[-0.200pt]{2.409pt}{0.400pt}}
\put(1429.0,208.0){\rule[-0.200pt]{2.409pt}{0.400pt}}
\put(170.0,208.0){\rule[-0.200pt]{2.409pt}{0.400pt}}
\put(1429.0,208.0){\rule[-0.200pt]{2.409pt}{0.400pt}}
\put(170.0,208.0){\rule[-0.200pt]{2.409pt}{0.400pt}}
\put(1429.0,208.0){\rule[-0.200pt]{2.409pt}{0.400pt}}
\put(170.0,209.0){\rule[-0.200pt]{2.409pt}{0.400pt}}
\put(1429.0,209.0){\rule[-0.200pt]{2.409pt}{0.400pt}}
\put(170.0,209.0){\rule[-0.200pt]{2.409pt}{0.400pt}}
\put(1429.0,209.0){\rule[-0.200pt]{2.409pt}{0.400pt}}
\put(170.0,209.0){\rule[-0.200pt]{2.409pt}{0.400pt}}
\put(1429.0,209.0){\rule[-0.200pt]{2.409pt}{0.400pt}}
\put(170.0,209.0){\rule[-0.200pt]{2.409pt}{0.400pt}}
\put(1429.0,209.0){\rule[-0.200pt]{2.409pt}{0.400pt}}
\put(170.0,209.0){\rule[-0.200pt]{2.409pt}{0.400pt}}
\put(1429.0,209.0){\rule[-0.200pt]{2.409pt}{0.400pt}}
\put(170.0,209.0){\rule[-0.200pt]{2.409pt}{0.400pt}}
\put(1429.0,209.0){\rule[-0.200pt]{2.409pt}{0.400pt}}
\put(170.0,209.0){\rule[-0.200pt]{2.409pt}{0.400pt}}
\put(1429.0,209.0){\rule[-0.200pt]{2.409pt}{0.400pt}}
\put(170.0,209.0){\rule[-0.200pt]{2.409pt}{0.400pt}}
\put(1429.0,209.0){\rule[-0.200pt]{2.409pt}{0.400pt}}
\put(170.0,209.0){\rule[-0.200pt]{2.409pt}{0.400pt}}
\put(1429.0,209.0){\rule[-0.200pt]{2.409pt}{0.400pt}}
\put(170.0,209.0){\rule[-0.200pt]{2.409pt}{0.400pt}}
\put(1429.0,209.0){\rule[-0.200pt]{2.409pt}{0.400pt}}
\put(170.0,209.0){\rule[-0.200pt]{2.409pt}{0.400pt}}
\put(1429.0,209.0){\rule[-0.200pt]{2.409pt}{0.400pt}}
\put(170.0,209.0){\rule[-0.200pt]{2.409pt}{0.400pt}}
\put(1429.0,209.0){\rule[-0.200pt]{2.409pt}{0.400pt}}
\put(170.0,209.0){\rule[-0.200pt]{2.409pt}{0.400pt}}
\put(1429.0,209.0){\rule[-0.200pt]{2.409pt}{0.400pt}}
\put(170.0,209.0){\rule[-0.200pt]{2.409pt}{0.400pt}}
\put(1429.0,209.0){\rule[-0.200pt]{2.409pt}{0.400pt}}
\put(170.0,209.0){\rule[-0.200pt]{2.409pt}{0.400pt}}
\put(1429.0,209.0){\rule[-0.200pt]{2.409pt}{0.400pt}}
\put(170.0,210.0){\rule[-0.200pt]{2.409pt}{0.400pt}}
\put(1429.0,210.0){\rule[-0.200pt]{2.409pt}{0.400pt}}
\put(170.0,210.0){\rule[-0.200pt]{2.409pt}{0.400pt}}
\put(1429.0,210.0){\rule[-0.200pt]{2.409pt}{0.400pt}}
\put(170.0,210.0){\rule[-0.200pt]{2.409pt}{0.400pt}}
\put(1429.0,210.0){\rule[-0.200pt]{2.409pt}{0.400pt}}
\put(170.0,210.0){\rule[-0.200pt]{2.409pt}{0.400pt}}
\put(1429.0,210.0){\rule[-0.200pt]{2.409pt}{0.400pt}}
\put(170.0,210.0){\rule[-0.200pt]{2.409pt}{0.400pt}}
\put(1429.0,210.0){\rule[-0.200pt]{2.409pt}{0.400pt}}
\put(170.0,210.0){\rule[-0.200pt]{2.409pt}{0.400pt}}
\put(1429.0,210.0){\rule[-0.200pt]{2.409pt}{0.400pt}}
\put(170.0,210.0){\rule[-0.200pt]{2.409pt}{0.400pt}}
\put(1429.0,210.0){\rule[-0.200pt]{2.409pt}{0.400pt}}
\put(170.0,210.0){\rule[-0.200pt]{2.409pt}{0.400pt}}
\put(1429.0,210.0){\rule[-0.200pt]{2.409pt}{0.400pt}}
\put(170.0,210.0){\rule[-0.200pt]{2.409pt}{0.400pt}}
\put(1429.0,210.0){\rule[-0.200pt]{2.409pt}{0.400pt}}
\put(170.0,210.0){\rule[-0.200pt]{2.409pt}{0.400pt}}
\put(1429.0,210.0){\rule[-0.200pt]{2.409pt}{0.400pt}}
\put(170.0,210.0){\rule[-0.200pt]{2.409pt}{0.400pt}}
\put(1429.0,210.0){\rule[-0.200pt]{2.409pt}{0.400pt}}
\put(170.0,210.0){\rule[-0.200pt]{2.409pt}{0.400pt}}
\put(1429.0,210.0){\rule[-0.200pt]{2.409pt}{0.400pt}}
\put(170.0,210.0){\rule[-0.200pt]{2.409pt}{0.400pt}}
\put(1429.0,210.0){\rule[-0.200pt]{2.409pt}{0.400pt}}
\put(170.0,210.0){\rule[-0.200pt]{2.409pt}{0.400pt}}
\put(1429.0,210.0){\rule[-0.200pt]{2.409pt}{0.400pt}}
\put(170.0,210.0){\rule[-0.200pt]{2.409pt}{0.400pt}}
\put(1429.0,210.0){\rule[-0.200pt]{2.409pt}{0.400pt}}
\put(170.0,210.0){\rule[-0.200pt]{2.409pt}{0.400pt}}
\put(1429.0,210.0){\rule[-0.200pt]{2.409pt}{0.400pt}}
\put(170.0,210.0){\rule[-0.200pt]{2.409pt}{0.400pt}}
\put(1429.0,210.0){\rule[-0.200pt]{2.409pt}{0.400pt}}
\put(170.0,211.0){\rule[-0.200pt]{2.409pt}{0.400pt}}
\put(1429.0,211.0){\rule[-0.200pt]{2.409pt}{0.400pt}}
\put(170.0,211.0){\rule[-0.200pt]{2.409pt}{0.400pt}}
\put(1429.0,211.0){\rule[-0.200pt]{2.409pt}{0.400pt}}
\put(170.0,211.0){\rule[-0.200pt]{2.409pt}{0.400pt}}
\put(1429.0,211.0){\rule[-0.200pt]{2.409pt}{0.400pt}}
\put(170.0,211.0){\rule[-0.200pt]{2.409pt}{0.400pt}}
\put(1429.0,211.0){\rule[-0.200pt]{2.409pt}{0.400pt}}
\put(170.0,211.0){\rule[-0.200pt]{2.409pt}{0.400pt}}
\put(1429.0,211.0){\rule[-0.200pt]{2.409pt}{0.400pt}}
\put(170.0,211.0){\rule[-0.200pt]{2.409pt}{0.400pt}}
\put(1429.0,211.0){\rule[-0.200pt]{2.409pt}{0.400pt}}
\put(170.0,211.0){\rule[-0.200pt]{2.409pt}{0.400pt}}
\put(1429.0,211.0){\rule[-0.200pt]{2.409pt}{0.400pt}}
\put(170.0,211.0){\rule[-0.200pt]{2.409pt}{0.400pt}}
\put(1429.0,211.0){\rule[-0.200pt]{2.409pt}{0.400pt}}
\put(170.0,211.0){\rule[-0.200pt]{2.409pt}{0.400pt}}
\put(1429.0,211.0){\rule[-0.200pt]{2.409pt}{0.400pt}}
\put(170.0,211.0){\rule[-0.200pt]{2.409pt}{0.400pt}}
\put(1429.0,211.0){\rule[-0.200pt]{2.409pt}{0.400pt}}
\put(170.0,211.0){\rule[-0.200pt]{2.409pt}{0.400pt}}
\put(1429.0,211.0){\rule[-0.200pt]{2.409pt}{0.400pt}}
\put(170.0,211.0){\rule[-0.200pt]{2.409pt}{0.400pt}}
\put(1429.0,211.0){\rule[-0.200pt]{2.409pt}{0.400pt}}
\put(170.0,211.0){\rule[-0.200pt]{2.409pt}{0.400pt}}
\put(1429.0,211.0){\rule[-0.200pt]{2.409pt}{0.400pt}}
\put(170.0,211.0){\rule[-0.200pt]{2.409pt}{0.400pt}}
\put(1429.0,211.0){\rule[-0.200pt]{2.409pt}{0.400pt}}
\put(170.0,211.0){\rule[-0.200pt]{2.409pt}{0.400pt}}
\put(1429.0,211.0){\rule[-0.200pt]{2.409pt}{0.400pt}}
\put(170.0,211.0){\rule[-0.200pt]{2.409pt}{0.400pt}}
\put(1429.0,211.0){\rule[-0.200pt]{2.409pt}{0.400pt}}
\put(170.0,211.0){\rule[-0.200pt]{2.409pt}{0.400pt}}
\put(1429.0,211.0){\rule[-0.200pt]{2.409pt}{0.400pt}}
\put(170.0,212.0){\rule[-0.200pt]{4.818pt}{0.400pt}}
\put(150,212){\makebox(0,0)[r]{ 1e-12}}
\put(1419.0,212.0){\rule[-0.200pt]{4.818pt}{0.400pt}}
\put(170.0,224.0){\rule[-0.200pt]{2.409pt}{0.400pt}}
\put(1429.0,224.0){\rule[-0.200pt]{2.409pt}{0.400pt}}
\put(170.0,242.0){\rule[-0.200pt]{2.409pt}{0.400pt}}
\put(1429.0,242.0){\rule[-0.200pt]{2.409pt}{0.400pt}}
\put(170.0,250.0){\rule[-0.200pt]{2.409pt}{0.400pt}}
\put(1429.0,250.0){\rule[-0.200pt]{2.409pt}{0.400pt}}
\put(170.0,256.0){\rule[-0.200pt]{2.409pt}{0.400pt}}
\put(1429.0,256.0){\rule[-0.200pt]{2.409pt}{0.400pt}}
\put(170.0,261.0){\rule[-0.200pt]{2.409pt}{0.400pt}}
\put(1429.0,261.0){\rule[-0.200pt]{2.409pt}{0.400pt}}
\put(170.0,265.0){\rule[-0.200pt]{2.409pt}{0.400pt}}
\put(1429.0,265.0){\rule[-0.200pt]{2.409pt}{0.400pt}}
\put(170.0,268.0){\rule[-0.200pt]{2.409pt}{0.400pt}}
\put(1429.0,268.0){\rule[-0.200pt]{2.409pt}{0.400pt}}
\put(170.0,270.0){\rule[-0.200pt]{2.409pt}{0.400pt}}
\put(1429.0,270.0){\rule[-0.200pt]{2.409pt}{0.400pt}}
\put(170.0,273.0){\rule[-0.200pt]{2.409pt}{0.400pt}}
\put(1429.0,273.0){\rule[-0.200pt]{2.409pt}{0.400pt}}
\put(170.0,275.0){\rule[-0.200pt]{2.409pt}{0.400pt}}
\put(1429.0,275.0){\rule[-0.200pt]{2.409pt}{0.400pt}}
\put(170.0,276.0){\rule[-0.200pt]{2.409pt}{0.400pt}}
\put(1429.0,276.0){\rule[-0.200pt]{2.409pt}{0.400pt}}
\put(170.0,278.0){\rule[-0.200pt]{2.409pt}{0.400pt}}
\put(1429.0,278.0){\rule[-0.200pt]{2.409pt}{0.400pt}}
\put(170.0,280.0){\rule[-0.200pt]{2.409pt}{0.400pt}}
\put(1429.0,280.0){\rule[-0.200pt]{2.409pt}{0.400pt}}
\put(170.0,281.0){\rule[-0.200pt]{2.409pt}{0.400pt}}
\put(1429.0,281.0){\rule[-0.200pt]{2.409pt}{0.400pt}}
\put(170.0,282.0){\rule[-0.200pt]{2.409pt}{0.400pt}}
\put(1429.0,282.0){\rule[-0.200pt]{2.409pt}{0.400pt}}
\put(170.0,284.0){\rule[-0.200pt]{2.409pt}{0.400pt}}
\put(1429.0,284.0){\rule[-0.200pt]{2.409pt}{0.400pt}}
\put(170.0,285.0){\rule[-0.200pt]{2.409pt}{0.400pt}}
\put(1429.0,285.0){\rule[-0.200pt]{2.409pt}{0.400pt}}
\put(170.0,286.0){\rule[-0.200pt]{2.409pt}{0.400pt}}
\put(1429.0,286.0){\rule[-0.200pt]{2.409pt}{0.400pt}}
\put(170.0,287.0){\rule[-0.200pt]{2.409pt}{0.400pt}}
\put(1429.0,287.0){\rule[-0.200pt]{2.409pt}{0.400pt}}
\put(170.0,288.0){\rule[-0.200pt]{2.409pt}{0.400pt}}
\put(1429.0,288.0){\rule[-0.200pt]{2.409pt}{0.400pt}}
\put(170.0,289.0){\rule[-0.200pt]{2.409pt}{0.400pt}}
\put(1429.0,289.0){\rule[-0.200pt]{2.409pt}{0.400pt}}
\put(170.0,290.0){\rule[-0.200pt]{2.409pt}{0.400pt}}
\put(1429.0,290.0){\rule[-0.200pt]{2.409pt}{0.400pt}}
\put(170.0,291.0){\rule[-0.200pt]{2.409pt}{0.400pt}}
\put(1429.0,291.0){\rule[-0.200pt]{2.409pt}{0.400pt}}
\put(170.0,291.0){\rule[-0.200pt]{2.409pt}{0.400pt}}
\put(1429.0,291.0){\rule[-0.200pt]{2.409pt}{0.400pt}}
\put(170.0,292.0){\rule[-0.200pt]{2.409pt}{0.400pt}}
\put(1429.0,292.0){\rule[-0.200pt]{2.409pt}{0.400pt}}
\put(170.0,293.0){\rule[-0.200pt]{2.409pt}{0.400pt}}
\put(1429.0,293.0){\rule[-0.200pt]{2.409pt}{0.400pt}}
\put(170.0,294.0){\rule[-0.200pt]{2.409pt}{0.400pt}}
\put(1429.0,294.0){\rule[-0.200pt]{2.409pt}{0.400pt}}
\put(170.0,294.0){\rule[-0.200pt]{2.409pt}{0.400pt}}
\put(1429.0,294.0){\rule[-0.200pt]{2.409pt}{0.400pt}}
\put(170.0,295.0){\rule[-0.200pt]{2.409pt}{0.400pt}}
\put(1429.0,295.0){\rule[-0.200pt]{2.409pt}{0.400pt}}
\put(170.0,296.0){\rule[-0.200pt]{2.409pt}{0.400pt}}
\put(1429.0,296.0){\rule[-0.200pt]{2.409pt}{0.400pt}}
\put(170.0,296.0){\rule[-0.200pt]{2.409pt}{0.400pt}}
\put(1429.0,296.0){\rule[-0.200pt]{2.409pt}{0.400pt}}
\put(170.0,297.0){\rule[-0.200pt]{2.409pt}{0.400pt}}
\put(1429.0,297.0){\rule[-0.200pt]{2.409pt}{0.400pt}}
\put(170.0,297.0){\rule[-0.200pt]{2.409pt}{0.400pt}}
\put(1429.0,297.0){\rule[-0.200pt]{2.409pt}{0.400pt}}
\put(170.0,298.0){\rule[-0.200pt]{2.409pt}{0.400pt}}
\put(1429.0,298.0){\rule[-0.200pt]{2.409pt}{0.400pt}}
\put(170.0,299.0){\rule[-0.200pt]{2.409pt}{0.400pt}}
\put(1429.0,299.0){\rule[-0.200pt]{2.409pt}{0.400pt}}
\put(170.0,299.0){\rule[-0.200pt]{2.409pt}{0.400pt}}
\put(1429.0,299.0){\rule[-0.200pt]{2.409pt}{0.400pt}}
\put(170.0,300.0){\rule[-0.200pt]{2.409pt}{0.400pt}}
\put(1429.0,300.0){\rule[-0.200pt]{2.409pt}{0.400pt}}
\put(170.0,300.0){\rule[-0.200pt]{2.409pt}{0.400pt}}
\put(1429.0,300.0){\rule[-0.200pt]{2.409pt}{0.400pt}}
\put(170.0,301.0){\rule[-0.200pt]{2.409pt}{0.400pt}}
\put(1429.0,301.0){\rule[-0.200pt]{2.409pt}{0.400pt}}
\put(170.0,301.0){\rule[-0.200pt]{2.409pt}{0.400pt}}
\put(1429.0,301.0){\rule[-0.200pt]{2.409pt}{0.400pt}}
\put(170.0,302.0){\rule[-0.200pt]{2.409pt}{0.400pt}}
\put(1429.0,302.0){\rule[-0.200pt]{2.409pt}{0.400pt}}
\put(170.0,302.0){\rule[-0.200pt]{2.409pt}{0.400pt}}
\put(1429.0,302.0){\rule[-0.200pt]{2.409pt}{0.400pt}}
\put(170.0,302.0){\rule[-0.200pt]{2.409pt}{0.400pt}}
\put(1429.0,302.0){\rule[-0.200pt]{2.409pt}{0.400pt}}
\put(170.0,303.0){\rule[-0.200pt]{2.409pt}{0.400pt}}
\put(1429.0,303.0){\rule[-0.200pt]{2.409pt}{0.400pt}}
\put(170.0,303.0){\rule[-0.200pt]{2.409pt}{0.400pt}}
\put(1429.0,303.0){\rule[-0.200pt]{2.409pt}{0.400pt}}
\put(170.0,304.0){\rule[-0.200pt]{2.409pt}{0.400pt}}
\put(1429.0,304.0){\rule[-0.200pt]{2.409pt}{0.400pt}}
\put(170.0,304.0){\rule[-0.200pt]{2.409pt}{0.400pt}}
\put(1429.0,304.0){\rule[-0.200pt]{2.409pt}{0.400pt}}
\put(170.0,305.0){\rule[-0.200pt]{2.409pt}{0.400pt}}
\put(1429.0,305.0){\rule[-0.200pt]{2.409pt}{0.400pt}}
\put(170.0,305.0){\rule[-0.200pt]{2.409pt}{0.400pt}}
\put(1429.0,305.0){\rule[-0.200pt]{2.409pt}{0.400pt}}
\put(170.0,305.0){\rule[-0.200pt]{2.409pt}{0.400pt}}
\put(1429.0,305.0){\rule[-0.200pt]{2.409pt}{0.400pt}}
\put(170.0,306.0){\rule[-0.200pt]{2.409pt}{0.400pt}}
\put(1429.0,306.0){\rule[-0.200pt]{2.409pt}{0.400pt}}
\put(170.0,306.0){\rule[-0.200pt]{2.409pt}{0.400pt}}
\put(1429.0,306.0){\rule[-0.200pt]{2.409pt}{0.400pt}}
\put(170.0,306.0){\rule[-0.200pt]{2.409pt}{0.400pt}}
\put(1429.0,306.0){\rule[-0.200pt]{2.409pt}{0.400pt}}
\put(170.0,307.0){\rule[-0.200pt]{2.409pt}{0.400pt}}
\put(1429.0,307.0){\rule[-0.200pt]{2.409pt}{0.400pt}}
\put(170.0,307.0){\rule[-0.200pt]{2.409pt}{0.400pt}}
\put(1429.0,307.0){\rule[-0.200pt]{2.409pt}{0.400pt}}
\put(170.0,307.0){\rule[-0.200pt]{2.409pt}{0.400pt}}
\put(1429.0,307.0){\rule[-0.200pt]{2.409pt}{0.400pt}}
\put(170.0,308.0){\rule[-0.200pt]{2.409pt}{0.400pt}}
\put(1429.0,308.0){\rule[-0.200pt]{2.409pt}{0.400pt}}
\put(170.0,308.0){\rule[-0.200pt]{2.409pt}{0.400pt}}
\put(1429.0,308.0){\rule[-0.200pt]{2.409pt}{0.400pt}}
\put(170.0,308.0){\rule[-0.200pt]{2.409pt}{0.400pt}}
\put(1429.0,308.0){\rule[-0.200pt]{2.409pt}{0.400pt}}
\put(170.0,309.0){\rule[-0.200pt]{2.409pt}{0.400pt}}
\put(1429.0,309.0){\rule[-0.200pt]{2.409pt}{0.400pt}}
\put(170.0,309.0){\rule[-0.200pt]{2.409pt}{0.400pt}}
\put(1429.0,309.0){\rule[-0.200pt]{2.409pt}{0.400pt}}
\put(170.0,309.0){\rule[-0.200pt]{2.409pt}{0.400pt}}
\put(1429.0,309.0){\rule[-0.200pt]{2.409pt}{0.400pt}}
\put(170.0,310.0){\rule[-0.200pt]{2.409pt}{0.400pt}}
\put(1429.0,310.0){\rule[-0.200pt]{2.409pt}{0.400pt}}
\put(170.0,310.0){\rule[-0.200pt]{2.409pt}{0.400pt}}
\put(1429.0,310.0){\rule[-0.200pt]{2.409pt}{0.400pt}}
\put(170.0,310.0){\rule[-0.200pt]{2.409pt}{0.400pt}}
\put(1429.0,310.0){\rule[-0.200pt]{2.409pt}{0.400pt}}
\put(170.0,311.0){\rule[-0.200pt]{2.409pt}{0.400pt}}
\put(1429.0,311.0){\rule[-0.200pt]{2.409pt}{0.400pt}}
\put(170.0,311.0){\rule[-0.200pt]{2.409pt}{0.400pt}}
\put(1429.0,311.0){\rule[-0.200pt]{2.409pt}{0.400pt}}
\put(170.0,311.0){\rule[-0.200pt]{2.409pt}{0.400pt}}
\put(1429.0,311.0){\rule[-0.200pt]{2.409pt}{0.400pt}}
\put(170.0,311.0){\rule[-0.200pt]{2.409pt}{0.400pt}}
\put(1429.0,311.0){\rule[-0.200pt]{2.409pt}{0.400pt}}
\put(170.0,312.0){\rule[-0.200pt]{2.409pt}{0.400pt}}
\put(1429.0,312.0){\rule[-0.200pt]{2.409pt}{0.400pt}}
\put(170.0,312.0){\rule[-0.200pt]{2.409pt}{0.400pt}}
\put(1429.0,312.0){\rule[-0.200pt]{2.409pt}{0.400pt}}
\put(170.0,312.0){\rule[-0.200pt]{2.409pt}{0.400pt}}
\put(1429.0,312.0){\rule[-0.200pt]{2.409pt}{0.400pt}}
\put(170.0,312.0){\rule[-0.200pt]{2.409pt}{0.400pt}}
\put(1429.0,312.0){\rule[-0.200pt]{2.409pt}{0.400pt}}
\put(170.0,313.0){\rule[-0.200pt]{2.409pt}{0.400pt}}
\put(1429.0,313.0){\rule[-0.200pt]{2.409pt}{0.400pt}}
\put(170.0,313.0){\rule[-0.200pt]{2.409pt}{0.400pt}}
\put(1429.0,313.0){\rule[-0.200pt]{2.409pt}{0.400pt}}
\put(170.0,313.0){\rule[-0.200pt]{2.409pt}{0.400pt}}
\put(1429.0,313.0){\rule[-0.200pt]{2.409pt}{0.400pt}}
\put(170.0,313.0){\rule[-0.200pt]{2.409pt}{0.400pt}}
\put(1429.0,313.0){\rule[-0.200pt]{2.409pt}{0.400pt}}
\put(170.0,314.0){\rule[-0.200pt]{2.409pt}{0.400pt}}
\put(1429.0,314.0){\rule[-0.200pt]{2.409pt}{0.400pt}}
\put(170.0,314.0){\rule[-0.200pt]{2.409pt}{0.400pt}}
\put(1429.0,314.0){\rule[-0.200pt]{2.409pt}{0.400pt}}
\put(170.0,314.0){\rule[-0.200pt]{2.409pt}{0.400pt}}
\put(1429.0,314.0){\rule[-0.200pt]{2.409pt}{0.400pt}}
\put(170.0,314.0){\rule[-0.200pt]{2.409pt}{0.400pt}}
\put(1429.0,314.0){\rule[-0.200pt]{2.409pt}{0.400pt}}
\put(170.0,315.0){\rule[-0.200pt]{2.409pt}{0.400pt}}
\put(1429.0,315.0){\rule[-0.200pt]{2.409pt}{0.400pt}}
\put(170.0,315.0){\rule[-0.200pt]{2.409pt}{0.400pt}}
\put(1429.0,315.0){\rule[-0.200pt]{2.409pt}{0.400pt}}
\put(170.0,315.0){\rule[-0.200pt]{2.409pt}{0.400pt}}
\put(1429.0,315.0){\rule[-0.200pt]{2.409pt}{0.400pt}}
\put(170.0,315.0){\rule[-0.200pt]{2.409pt}{0.400pt}}
\put(1429.0,315.0){\rule[-0.200pt]{2.409pt}{0.400pt}}
\put(170.0,316.0){\rule[-0.200pt]{2.409pt}{0.400pt}}
\put(1429.0,316.0){\rule[-0.200pt]{2.409pt}{0.400pt}}
\put(170.0,316.0){\rule[-0.200pt]{2.409pt}{0.400pt}}
\put(1429.0,316.0){\rule[-0.200pt]{2.409pt}{0.400pt}}
\put(170.0,316.0){\rule[-0.200pt]{2.409pt}{0.400pt}}
\put(1429.0,316.0){\rule[-0.200pt]{2.409pt}{0.400pt}}
\put(170.0,316.0){\rule[-0.200pt]{2.409pt}{0.400pt}}
\put(1429.0,316.0){\rule[-0.200pt]{2.409pt}{0.400pt}}
\put(170.0,316.0){\rule[-0.200pt]{2.409pt}{0.400pt}}
\put(1429.0,316.0){\rule[-0.200pt]{2.409pt}{0.400pt}}
\put(170.0,317.0){\rule[-0.200pt]{2.409pt}{0.400pt}}
\put(1429.0,317.0){\rule[-0.200pt]{2.409pt}{0.400pt}}
\put(170.0,317.0){\rule[-0.200pt]{2.409pt}{0.400pt}}
\put(1429.0,317.0){\rule[-0.200pt]{2.409pt}{0.400pt}}
\put(170.0,317.0){\rule[-0.200pt]{2.409pt}{0.400pt}}
\put(1429.0,317.0){\rule[-0.200pt]{2.409pt}{0.400pt}}
\put(170.0,317.0){\rule[-0.200pt]{2.409pt}{0.400pt}}
\put(1429.0,317.0){\rule[-0.200pt]{2.409pt}{0.400pt}}
\put(170.0,317.0){\rule[-0.200pt]{2.409pt}{0.400pt}}
\put(1429.0,317.0){\rule[-0.200pt]{2.409pt}{0.400pt}}
\put(170.0,318.0){\rule[-0.200pt]{2.409pt}{0.400pt}}
\put(1429.0,318.0){\rule[-0.200pt]{2.409pt}{0.400pt}}
\put(170.0,318.0){\rule[-0.200pt]{2.409pt}{0.400pt}}
\put(1429.0,318.0){\rule[-0.200pt]{2.409pt}{0.400pt}}
\put(170.0,318.0){\rule[-0.200pt]{2.409pt}{0.400pt}}
\put(1429.0,318.0){\rule[-0.200pt]{2.409pt}{0.400pt}}
\put(170.0,318.0){\rule[-0.200pt]{2.409pt}{0.400pt}}
\put(1429.0,318.0){\rule[-0.200pt]{2.409pt}{0.400pt}}
\put(170.0,318.0){\rule[-0.200pt]{2.409pt}{0.400pt}}
\put(1429.0,318.0){\rule[-0.200pt]{2.409pt}{0.400pt}}
\put(170.0,319.0){\rule[-0.200pt]{2.409pt}{0.400pt}}
\put(1429.0,319.0){\rule[-0.200pt]{2.409pt}{0.400pt}}
\put(170.0,319.0){\rule[-0.200pt]{2.409pt}{0.400pt}}
\put(1429.0,319.0){\rule[-0.200pt]{2.409pt}{0.400pt}}
\put(170.0,319.0){\rule[-0.200pt]{2.409pt}{0.400pt}}
\put(1429.0,319.0){\rule[-0.200pt]{2.409pt}{0.400pt}}
\put(170.0,319.0){\rule[-0.200pt]{2.409pt}{0.400pt}}
\put(1429.0,319.0){\rule[-0.200pt]{2.409pt}{0.400pt}}
\put(170.0,319.0){\rule[-0.200pt]{2.409pt}{0.400pt}}
\put(1429.0,319.0){\rule[-0.200pt]{2.409pt}{0.400pt}}
\put(170.0,319.0){\rule[-0.200pt]{2.409pt}{0.400pt}}
\put(1429.0,319.0){\rule[-0.200pt]{2.409pt}{0.400pt}}
\put(170.0,320.0){\rule[-0.200pt]{2.409pt}{0.400pt}}
\put(1429.0,320.0){\rule[-0.200pt]{2.409pt}{0.400pt}}
\put(170.0,320.0){\rule[-0.200pt]{2.409pt}{0.400pt}}
\put(1429.0,320.0){\rule[-0.200pt]{2.409pt}{0.400pt}}
\put(170.0,320.0){\rule[-0.200pt]{2.409pt}{0.400pt}}
\put(1429.0,320.0){\rule[-0.200pt]{2.409pt}{0.400pt}}
\put(170.0,320.0){\rule[-0.200pt]{2.409pt}{0.400pt}}
\put(1429.0,320.0){\rule[-0.200pt]{2.409pt}{0.400pt}}
\put(170.0,320.0){\rule[-0.200pt]{2.409pt}{0.400pt}}
\put(1429.0,320.0){\rule[-0.200pt]{2.409pt}{0.400pt}}
\put(170.0,320.0){\rule[-0.200pt]{2.409pt}{0.400pt}}
\put(1429.0,320.0){\rule[-0.200pt]{2.409pt}{0.400pt}}
\put(170.0,321.0){\rule[-0.200pt]{2.409pt}{0.400pt}}
\put(1429.0,321.0){\rule[-0.200pt]{2.409pt}{0.400pt}}
\put(170.0,321.0){\rule[-0.200pt]{2.409pt}{0.400pt}}
\put(1429.0,321.0){\rule[-0.200pt]{2.409pt}{0.400pt}}
\put(170.0,321.0){\rule[-0.200pt]{2.409pt}{0.400pt}}
\put(1429.0,321.0){\rule[-0.200pt]{2.409pt}{0.400pt}}
\put(170.0,321.0){\rule[-0.200pt]{2.409pt}{0.400pt}}
\put(1429.0,321.0){\rule[-0.200pt]{2.409pt}{0.400pt}}
\put(170.0,321.0){\rule[-0.200pt]{2.409pt}{0.400pt}}
\put(1429.0,321.0){\rule[-0.200pt]{2.409pt}{0.400pt}}
\put(170.0,321.0){\rule[-0.200pt]{2.409pt}{0.400pt}}
\put(1429.0,321.0){\rule[-0.200pt]{2.409pt}{0.400pt}}
\put(170.0,322.0){\rule[-0.200pt]{2.409pt}{0.400pt}}
\put(1429.0,322.0){\rule[-0.200pt]{2.409pt}{0.400pt}}
\put(170.0,322.0){\rule[-0.200pt]{2.409pt}{0.400pt}}
\put(1429.0,322.0){\rule[-0.200pt]{2.409pt}{0.400pt}}
\put(170.0,322.0){\rule[-0.200pt]{2.409pt}{0.400pt}}
\put(1429.0,322.0){\rule[-0.200pt]{2.409pt}{0.400pt}}
\put(170.0,322.0){\rule[-0.200pt]{2.409pt}{0.400pt}}
\put(1429.0,322.0){\rule[-0.200pt]{2.409pt}{0.400pt}}
\put(170.0,322.0){\rule[-0.200pt]{2.409pt}{0.400pt}}
\put(1429.0,322.0){\rule[-0.200pt]{2.409pt}{0.400pt}}
\put(170.0,322.0){\rule[-0.200pt]{2.409pt}{0.400pt}}
\put(1429.0,322.0){\rule[-0.200pt]{2.409pt}{0.400pt}}
\put(170.0,323.0){\rule[-0.200pt]{2.409pt}{0.400pt}}
\put(1429.0,323.0){\rule[-0.200pt]{2.409pt}{0.400pt}}
\put(170.0,323.0){\rule[-0.200pt]{2.409pt}{0.400pt}}
\put(1429.0,323.0){\rule[-0.200pt]{2.409pt}{0.400pt}}
\put(170.0,323.0){\rule[-0.200pt]{2.409pt}{0.400pt}}
\put(1429.0,323.0){\rule[-0.200pt]{2.409pt}{0.400pt}}
\put(170.0,323.0){\rule[-0.200pt]{2.409pt}{0.400pt}}
\put(1429.0,323.0){\rule[-0.200pt]{2.409pt}{0.400pt}}
\put(170.0,323.0){\rule[-0.200pt]{2.409pt}{0.400pt}}
\put(1429.0,323.0){\rule[-0.200pt]{2.409pt}{0.400pt}}
\put(170.0,323.0){\rule[-0.200pt]{2.409pt}{0.400pt}}
\put(1429.0,323.0){\rule[-0.200pt]{2.409pt}{0.400pt}}
\put(170.0,323.0){\rule[-0.200pt]{2.409pt}{0.400pt}}
\put(1429.0,323.0){\rule[-0.200pt]{2.409pt}{0.400pt}}
\put(170.0,324.0){\rule[-0.200pt]{2.409pt}{0.400pt}}
\put(1429.0,324.0){\rule[-0.200pt]{2.409pt}{0.400pt}}
\put(170.0,324.0){\rule[-0.200pt]{2.409pt}{0.400pt}}
\put(1429.0,324.0){\rule[-0.200pt]{2.409pt}{0.400pt}}
\put(170.0,324.0){\rule[-0.200pt]{2.409pt}{0.400pt}}
\put(1429.0,324.0){\rule[-0.200pt]{2.409pt}{0.400pt}}
\put(170.0,324.0){\rule[-0.200pt]{2.409pt}{0.400pt}}
\put(1429.0,324.0){\rule[-0.200pt]{2.409pt}{0.400pt}}
\put(170.0,324.0){\rule[-0.200pt]{2.409pt}{0.400pt}}
\put(1429.0,324.0){\rule[-0.200pt]{2.409pt}{0.400pt}}
\put(170.0,324.0){\rule[-0.200pt]{2.409pt}{0.400pt}}
\put(1429.0,324.0){\rule[-0.200pt]{2.409pt}{0.400pt}}
\put(170.0,324.0){\rule[-0.200pt]{2.409pt}{0.400pt}}
\put(1429.0,324.0){\rule[-0.200pt]{2.409pt}{0.400pt}}
\put(170.0,325.0){\rule[-0.200pt]{2.409pt}{0.400pt}}
\put(1429.0,325.0){\rule[-0.200pt]{2.409pt}{0.400pt}}
\put(170.0,325.0){\rule[-0.200pt]{2.409pt}{0.400pt}}
\put(1429.0,325.0){\rule[-0.200pt]{2.409pt}{0.400pt}}
\put(170.0,325.0){\rule[-0.200pt]{2.409pt}{0.400pt}}
\put(1429.0,325.0){\rule[-0.200pt]{2.409pt}{0.400pt}}
\put(170.0,325.0){\rule[-0.200pt]{2.409pt}{0.400pt}}
\put(1429.0,325.0){\rule[-0.200pt]{2.409pt}{0.400pt}}
\put(170.0,325.0){\rule[-0.200pt]{2.409pt}{0.400pt}}
\put(1429.0,325.0){\rule[-0.200pt]{2.409pt}{0.400pt}}
\put(170.0,325.0){\rule[-0.200pt]{2.409pt}{0.400pt}}
\put(1429.0,325.0){\rule[-0.200pt]{2.409pt}{0.400pt}}
\put(170.0,325.0){\rule[-0.200pt]{2.409pt}{0.400pt}}
\put(1429.0,325.0){\rule[-0.200pt]{2.409pt}{0.400pt}}
\put(170.0,325.0){\rule[-0.200pt]{2.409pt}{0.400pt}}
\put(1429.0,325.0){\rule[-0.200pt]{2.409pt}{0.400pt}}
\put(170.0,326.0){\rule[-0.200pt]{2.409pt}{0.400pt}}
\put(1429.0,326.0){\rule[-0.200pt]{2.409pt}{0.400pt}}
\put(170.0,326.0){\rule[-0.200pt]{2.409pt}{0.400pt}}
\put(1429.0,326.0){\rule[-0.200pt]{2.409pt}{0.400pt}}
\put(170.0,326.0){\rule[-0.200pt]{2.409pt}{0.400pt}}
\put(1429.0,326.0){\rule[-0.200pt]{2.409pt}{0.400pt}}
\put(170.0,326.0){\rule[-0.200pt]{2.409pt}{0.400pt}}
\put(1429.0,326.0){\rule[-0.200pt]{2.409pt}{0.400pt}}
\put(170.0,326.0){\rule[-0.200pt]{2.409pt}{0.400pt}}
\put(1429.0,326.0){\rule[-0.200pt]{2.409pt}{0.400pt}}
\put(170.0,326.0){\rule[-0.200pt]{2.409pt}{0.400pt}}
\put(1429.0,326.0){\rule[-0.200pt]{2.409pt}{0.400pt}}
\put(170.0,326.0){\rule[-0.200pt]{2.409pt}{0.400pt}}
\put(1429.0,326.0){\rule[-0.200pt]{2.409pt}{0.400pt}}
\put(170.0,326.0){\rule[-0.200pt]{2.409pt}{0.400pt}}
\put(1429.0,326.0){\rule[-0.200pt]{2.409pt}{0.400pt}}
\put(170.0,327.0){\rule[-0.200pt]{2.409pt}{0.400pt}}
\put(1429.0,327.0){\rule[-0.200pt]{2.409pt}{0.400pt}}
\put(170.0,327.0){\rule[-0.200pt]{2.409pt}{0.400pt}}
\put(1429.0,327.0){\rule[-0.200pt]{2.409pt}{0.400pt}}
\put(170.0,327.0){\rule[-0.200pt]{2.409pt}{0.400pt}}
\put(1429.0,327.0){\rule[-0.200pt]{2.409pt}{0.400pt}}
\put(170.0,327.0){\rule[-0.200pt]{2.409pt}{0.400pt}}
\put(1429.0,327.0){\rule[-0.200pt]{2.409pt}{0.400pt}}
\put(170.0,327.0){\rule[-0.200pt]{2.409pt}{0.400pt}}
\put(1429.0,327.0){\rule[-0.200pt]{2.409pt}{0.400pt}}
\put(170.0,327.0){\rule[-0.200pt]{2.409pt}{0.400pt}}
\put(1429.0,327.0){\rule[-0.200pt]{2.409pt}{0.400pt}}
\put(170.0,327.0){\rule[-0.200pt]{2.409pt}{0.400pt}}
\put(1429.0,327.0){\rule[-0.200pt]{2.409pt}{0.400pt}}
\put(170.0,327.0){\rule[-0.200pt]{2.409pt}{0.400pt}}
\put(1429.0,327.0){\rule[-0.200pt]{2.409pt}{0.400pt}}
\put(170.0,328.0){\rule[-0.200pt]{2.409pt}{0.400pt}}
\put(1429.0,328.0){\rule[-0.200pt]{2.409pt}{0.400pt}}
\put(170.0,328.0){\rule[-0.200pt]{2.409pt}{0.400pt}}
\put(1429.0,328.0){\rule[-0.200pt]{2.409pt}{0.400pt}}
\put(170.0,328.0){\rule[-0.200pt]{2.409pt}{0.400pt}}
\put(1429.0,328.0){\rule[-0.200pt]{2.409pt}{0.400pt}}
\put(170.0,328.0){\rule[-0.200pt]{2.409pt}{0.400pt}}
\put(1429.0,328.0){\rule[-0.200pt]{2.409pt}{0.400pt}}
\put(170.0,328.0){\rule[-0.200pt]{2.409pt}{0.400pt}}
\put(1429.0,328.0){\rule[-0.200pt]{2.409pt}{0.400pt}}
\put(170.0,328.0){\rule[-0.200pt]{2.409pt}{0.400pt}}
\put(1429.0,328.0){\rule[-0.200pt]{2.409pt}{0.400pt}}
\put(170.0,328.0){\rule[-0.200pt]{2.409pt}{0.400pt}}
\put(1429.0,328.0){\rule[-0.200pt]{2.409pt}{0.400pt}}
\put(170.0,328.0){\rule[-0.200pt]{2.409pt}{0.400pt}}
\put(1429.0,328.0){\rule[-0.200pt]{2.409pt}{0.400pt}}
\put(170.0,328.0){\rule[-0.200pt]{2.409pt}{0.400pt}}
\put(1429.0,328.0){\rule[-0.200pt]{2.409pt}{0.400pt}}
\put(170.0,329.0){\rule[-0.200pt]{2.409pt}{0.400pt}}
\put(1429.0,329.0){\rule[-0.200pt]{2.409pt}{0.400pt}}
\put(170.0,329.0){\rule[-0.200pt]{2.409pt}{0.400pt}}
\put(1429.0,329.0){\rule[-0.200pt]{2.409pt}{0.400pt}}
\put(170.0,329.0){\rule[-0.200pt]{2.409pt}{0.400pt}}
\put(1429.0,329.0){\rule[-0.200pt]{2.409pt}{0.400pt}}
\put(170.0,329.0){\rule[-0.200pt]{2.409pt}{0.400pt}}
\put(1429.0,329.0){\rule[-0.200pt]{2.409pt}{0.400pt}}
\put(170.0,329.0){\rule[-0.200pt]{2.409pt}{0.400pt}}
\put(1429.0,329.0){\rule[-0.200pt]{2.409pt}{0.400pt}}
\put(170.0,329.0){\rule[-0.200pt]{2.409pt}{0.400pt}}
\put(1429.0,329.0){\rule[-0.200pt]{2.409pt}{0.400pt}}
\put(170.0,329.0){\rule[-0.200pt]{2.409pt}{0.400pt}}
\put(1429.0,329.0){\rule[-0.200pt]{2.409pt}{0.400pt}}
\put(170.0,329.0){\rule[-0.200pt]{2.409pt}{0.400pt}}
\put(1429.0,329.0){\rule[-0.200pt]{2.409pt}{0.400pt}}
\put(170.0,329.0){\rule[-0.200pt]{2.409pt}{0.400pt}}
\put(1429.0,329.0){\rule[-0.200pt]{2.409pt}{0.400pt}}
\put(170.0,330.0){\rule[-0.200pt]{2.409pt}{0.400pt}}
\put(1429.0,330.0){\rule[-0.200pt]{2.409pt}{0.400pt}}
\put(170.0,330.0){\rule[-0.200pt]{2.409pt}{0.400pt}}
\put(1429.0,330.0){\rule[-0.200pt]{2.409pt}{0.400pt}}
\put(170.0,330.0){\rule[-0.200pt]{2.409pt}{0.400pt}}
\put(1429.0,330.0){\rule[-0.200pt]{2.409pt}{0.400pt}}
\put(170.0,330.0){\rule[-0.200pt]{2.409pt}{0.400pt}}
\put(1429.0,330.0){\rule[-0.200pt]{2.409pt}{0.400pt}}
\put(170.0,330.0){\rule[-0.200pt]{2.409pt}{0.400pt}}
\put(1429.0,330.0){\rule[-0.200pt]{2.409pt}{0.400pt}}
\put(170.0,330.0){\rule[-0.200pt]{2.409pt}{0.400pt}}
\put(1429.0,330.0){\rule[-0.200pt]{2.409pt}{0.400pt}}
\put(170.0,330.0){\rule[-0.200pt]{2.409pt}{0.400pt}}
\put(1429.0,330.0){\rule[-0.200pt]{2.409pt}{0.400pt}}
\put(170.0,330.0){\rule[-0.200pt]{2.409pt}{0.400pt}}
\put(1429.0,330.0){\rule[-0.200pt]{2.409pt}{0.400pt}}
\put(170.0,330.0){\rule[-0.200pt]{2.409pt}{0.400pt}}
\put(1429.0,330.0){\rule[-0.200pt]{2.409pt}{0.400pt}}
\put(170.0,330.0){\rule[-0.200pt]{2.409pt}{0.400pt}}
\put(1429.0,330.0){\rule[-0.200pt]{2.409pt}{0.400pt}}
\put(170.0,331.0){\rule[-0.200pt]{2.409pt}{0.400pt}}
\put(1429.0,331.0){\rule[-0.200pt]{2.409pt}{0.400pt}}
\put(170.0,331.0){\rule[-0.200pt]{2.409pt}{0.400pt}}
\put(1429.0,331.0){\rule[-0.200pt]{2.409pt}{0.400pt}}
\put(170.0,331.0){\rule[-0.200pt]{2.409pt}{0.400pt}}
\put(1429.0,331.0){\rule[-0.200pt]{2.409pt}{0.400pt}}
\put(170.0,331.0){\rule[-0.200pt]{2.409pt}{0.400pt}}
\put(1429.0,331.0){\rule[-0.200pt]{2.409pt}{0.400pt}}
\put(170.0,331.0){\rule[-0.200pt]{2.409pt}{0.400pt}}
\put(1429.0,331.0){\rule[-0.200pt]{2.409pt}{0.400pt}}
\put(170.0,331.0){\rule[-0.200pt]{2.409pt}{0.400pt}}
\put(1429.0,331.0){\rule[-0.200pt]{2.409pt}{0.400pt}}
\put(170.0,331.0){\rule[-0.200pt]{2.409pt}{0.400pt}}
\put(1429.0,331.0){\rule[-0.200pt]{2.409pt}{0.400pt}}
\put(170.0,331.0){\rule[-0.200pt]{2.409pt}{0.400pt}}
\put(1429.0,331.0){\rule[-0.200pt]{2.409pt}{0.400pt}}
\put(170.0,331.0){\rule[-0.200pt]{2.409pt}{0.400pt}}
\put(1429.0,331.0){\rule[-0.200pt]{2.409pt}{0.400pt}}
\put(170.0,331.0){\rule[-0.200pt]{2.409pt}{0.400pt}}
\put(1429.0,331.0){\rule[-0.200pt]{2.409pt}{0.400pt}}
\put(170.0,331.0){\rule[-0.200pt]{2.409pt}{0.400pt}}
\put(1429.0,331.0){\rule[-0.200pt]{2.409pt}{0.400pt}}
\put(170.0,332.0){\rule[-0.200pt]{2.409pt}{0.400pt}}
\put(1429.0,332.0){\rule[-0.200pt]{2.409pt}{0.400pt}}
\put(170.0,332.0){\rule[-0.200pt]{2.409pt}{0.400pt}}
\put(1429.0,332.0){\rule[-0.200pt]{2.409pt}{0.400pt}}
\put(170.0,332.0){\rule[-0.200pt]{2.409pt}{0.400pt}}
\put(1429.0,332.0){\rule[-0.200pt]{2.409pt}{0.400pt}}
\put(170.0,332.0){\rule[-0.200pt]{2.409pt}{0.400pt}}
\put(1429.0,332.0){\rule[-0.200pt]{2.409pt}{0.400pt}}
\put(170.0,332.0){\rule[-0.200pt]{2.409pt}{0.400pt}}
\put(1429.0,332.0){\rule[-0.200pt]{2.409pt}{0.400pt}}
\put(170.0,332.0){\rule[-0.200pt]{2.409pt}{0.400pt}}
\put(1429.0,332.0){\rule[-0.200pt]{2.409pt}{0.400pt}}
\put(170.0,332.0){\rule[-0.200pt]{2.409pt}{0.400pt}}
\put(1429.0,332.0){\rule[-0.200pt]{2.409pt}{0.400pt}}
\put(170.0,332.0){\rule[-0.200pt]{2.409pt}{0.400pt}}
\put(1429.0,332.0){\rule[-0.200pt]{2.409pt}{0.400pt}}
\put(170.0,332.0){\rule[-0.200pt]{2.409pt}{0.400pt}}
\put(1429.0,332.0){\rule[-0.200pt]{2.409pt}{0.400pt}}
\put(170.0,332.0){\rule[-0.200pt]{2.409pt}{0.400pt}}
\put(1429.0,332.0){\rule[-0.200pt]{2.409pt}{0.400pt}}
\put(170.0,332.0){\rule[-0.200pt]{2.409pt}{0.400pt}}
\put(1429.0,332.0){\rule[-0.200pt]{2.409pt}{0.400pt}}
\put(170.0,333.0){\rule[-0.200pt]{2.409pt}{0.400pt}}
\put(1429.0,333.0){\rule[-0.200pt]{2.409pt}{0.400pt}}
\put(170.0,333.0){\rule[-0.200pt]{2.409pt}{0.400pt}}
\put(1429.0,333.0){\rule[-0.200pt]{2.409pt}{0.400pt}}
\put(170.0,333.0){\rule[-0.200pt]{2.409pt}{0.400pt}}
\put(1429.0,333.0){\rule[-0.200pt]{2.409pt}{0.400pt}}
\put(170.0,333.0){\rule[-0.200pt]{2.409pt}{0.400pt}}
\put(1429.0,333.0){\rule[-0.200pt]{2.409pt}{0.400pt}}
\put(170.0,333.0){\rule[-0.200pt]{2.409pt}{0.400pt}}
\put(1429.0,333.0){\rule[-0.200pt]{2.409pt}{0.400pt}}
\put(170.0,333.0){\rule[-0.200pt]{2.409pt}{0.400pt}}
\put(1429.0,333.0){\rule[-0.200pt]{2.409pt}{0.400pt}}
\put(170.0,333.0){\rule[-0.200pt]{2.409pt}{0.400pt}}
\put(1429.0,333.0){\rule[-0.200pt]{2.409pt}{0.400pt}}
\put(170.0,333.0){\rule[-0.200pt]{2.409pt}{0.400pt}}
\put(1429.0,333.0){\rule[-0.200pt]{2.409pt}{0.400pt}}
\put(170.0,333.0){\rule[-0.200pt]{2.409pt}{0.400pt}}
\put(1429.0,333.0){\rule[-0.200pt]{2.409pt}{0.400pt}}
\put(170.0,333.0){\rule[-0.200pt]{2.409pt}{0.400pt}}
\put(1429.0,333.0){\rule[-0.200pt]{2.409pt}{0.400pt}}
\put(170.0,333.0){\rule[-0.200pt]{2.409pt}{0.400pt}}
\put(1429.0,333.0){\rule[-0.200pt]{2.409pt}{0.400pt}}
\put(170.0,334.0){\rule[-0.200pt]{2.409pt}{0.400pt}}
\put(1429.0,334.0){\rule[-0.200pt]{2.409pt}{0.400pt}}
\put(170.0,334.0){\rule[-0.200pt]{2.409pt}{0.400pt}}
\put(1429.0,334.0){\rule[-0.200pt]{2.409pt}{0.400pt}}
\put(170.0,334.0){\rule[-0.200pt]{2.409pt}{0.400pt}}
\put(1429.0,334.0){\rule[-0.200pt]{2.409pt}{0.400pt}}
\put(170.0,334.0){\rule[-0.200pt]{2.409pt}{0.400pt}}
\put(1429.0,334.0){\rule[-0.200pt]{2.409pt}{0.400pt}}
\put(170.0,334.0){\rule[-0.200pt]{2.409pt}{0.400pt}}
\put(1429.0,334.0){\rule[-0.200pt]{2.409pt}{0.400pt}}
\put(170.0,334.0){\rule[-0.200pt]{2.409pt}{0.400pt}}
\put(1429.0,334.0){\rule[-0.200pt]{2.409pt}{0.400pt}}
\put(170.0,334.0){\rule[-0.200pt]{2.409pt}{0.400pt}}
\put(1429.0,334.0){\rule[-0.200pt]{2.409pt}{0.400pt}}
\put(170.0,334.0){\rule[-0.200pt]{2.409pt}{0.400pt}}
\put(1429.0,334.0){\rule[-0.200pt]{2.409pt}{0.400pt}}
\put(170.0,334.0){\rule[-0.200pt]{2.409pt}{0.400pt}}
\put(1429.0,334.0){\rule[-0.200pt]{2.409pt}{0.400pt}}
\put(170.0,334.0){\rule[-0.200pt]{2.409pt}{0.400pt}}
\put(1429.0,334.0){\rule[-0.200pt]{2.409pt}{0.400pt}}
\put(170.0,334.0){\rule[-0.200pt]{2.409pt}{0.400pt}}
\put(1429.0,334.0){\rule[-0.200pt]{2.409pt}{0.400pt}}
\put(170.0,334.0){\rule[-0.200pt]{2.409pt}{0.400pt}}
\put(1429.0,334.0){\rule[-0.200pt]{2.409pt}{0.400pt}}
\put(170.0,334.0){\rule[-0.200pt]{2.409pt}{0.400pt}}
\put(1429.0,334.0){\rule[-0.200pt]{2.409pt}{0.400pt}}
\put(170.0,335.0){\rule[-0.200pt]{2.409pt}{0.400pt}}
\put(1429.0,335.0){\rule[-0.200pt]{2.409pt}{0.400pt}}
\put(170.0,335.0){\rule[-0.200pt]{2.409pt}{0.400pt}}
\put(1429.0,335.0){\rule[-0.200pt]{2.409pt}{0.400pt}}
\put(170.0,335.0){\rule[-0.200pt]{2.409pt}{0.400pt}}
\put(1429.0,335.0){\rule[-0.200pt]{2.409pt}{0.400pt}}
\put(170.0,335.0){\rule[-0.200pt]{2.409pt}{0.400pt}}
\put(1429.0,335.0){\rule[-0.200pt]{2.409pt}{0.400pt}}
\put(170.0,335.0){\rule[-0.200pt]{2.409pt}{0.400pt}}
\put(1429.0,335.0){\rule[-0.200pt]{2.409pt}{0.400pt}}
\put(170.0,335.0){\rule[-0.200pt]{2.409pt}{0.400pt}}
\put(1429.0,335.0){\rule[-0.200pt]{2.409pt}{0.400pt}}
\put(170.0,335.0){\rule[-0.200pt]{2.409pt}{0.400pt}}
\put(1429.0,335.0){\rule[-0.200pt]{2.409pt}{0.400pt}}
\put(170.0,335.0){\rule[-0.200pt]{2.409pt}{0.400pt}}
\put(1429.0,335.0){\rule[-0.200pt]{2.409pt}{0.400pt}}
\put(170.0,335.0){\rule[-0.200pt]{2.409pt}{0.400pt}}
\put(1429.0,335.0){\rule[-0.200pt]{2.409pt}{0.400pt}}
\put(170.0,335.0){\rule[-0.200pt]{2.409pt}{0.400pt}}
\put(1429.0,335.0){\rule[-0.200pt]{2.409pt}{0.400pt}}
\put(170.0,335.0){\rule[-0.200pt]{2.409pt}{0.400pt}}
\put(1429.0,335.0){\rule[-0.200pt]{2.409pt}{0.400pt}}
\put(170.0,335.0){\rule[-0.200pt]{2.409pt}{0.400pt}}
\put(1429.0,335.0){\rule[-0.200pt]{2.409pt}{0.400pt}}
\put(170.0,336.0){\rule[-0.200pt]{2.409pt}{0.400pt}}
\put(1429.0,336.0){\rule[-0.200pt]{2.409pt}{0.400pt}}
\put(170.0,336.0){\rule[-0.200pt]{2.409pt}{0.400pt}}
\put(1429.0,336.0){\rule[-0.200pt]{2.409pt}{0.400pt}}
\put(170.0,336.0){\rule[-0.200pt]{2.409pt}{0.400pt}}
\put(1429.0,336.0){\rule[-0.200pt]{2.409pt}{0.400pt}}
\put(170.0,336.0){\rule[-0.200pt]{2.409pt}{0.400pt}}
\put(1429.0,336.0){\rule[-0.200pt]{2.409pt}{0.400pt}}
\put(170.0,336.0){\rule[-0.200pt]{2.409pt}{0.400pt}}
\put(1429.0,336.0){\rule[-0.200pt]{2.409pt}{0.400pt}}
\put(170.0,336.0){\rule[-0.200pt]{2.409pt}{0.400pt}}
\put(1429.0,336.0){\rule[-0.200pt]{2.409pt}{0.400pt}}
\put(170.0,336.0){\rule[-0.200pt]{2.409pt}{0.400pt}}
\put(1429.0,336.0){\rule[-0.200pt]{2.409pt}{0.400pt}}
\put(170.0,336.0){\rule[-0.200pt]{2.409pt}{0.400pt}}
\put(1429.0,336.0){\rule[-0.200pt]{2.409pt}{0.400pt}}
\put(170.0,336.0){\rule[-0.200pt]{2.409pt}{0.400pt}}
\put(1429.0,336.0){\rule[-0.200pt]{2.409pt}{0.400pt}}
\put(170.0,336.0){\rule[-0.200pt]{2.409pt}{0.400pt}}
\put(1429.0,336.0){\rule[-0.200pt]{2.409pt}{0.400pt}}
\put(170.0,336.0){\rule[-0.200pt]{2.409pt}{0.400pt}}
\put(1429.0,336.0){\rule[-0.200pt]{2.409pt}{0.400pt}}
\put(170.0,336.0){\rule[-0.200pt]{2.409pt}{0.400pt}}
\put(1429.0,336.0){\rule[-0.200pt]{2.409pt}{0.400pt}}
\put(170.0,336.0){\rule[-0.200pt]{2.409pt}{0.400pt}}
\put(1429.0,336.0){\rule[-0.200pt]{2.409pt}{0.400pt}}
\put(170.0,336.0){\rule[-0.200pt]{2.409pt}{0.400pt}}
\put(1429.0,336.0){\rule[-0.200pt]{2.409pt}{0.400pt}}
\put(170.0,337.0){\rule[-0.200pt]{2.409pt}{0.400pt}}
\put(1429.0,337.0){\rule[-0.200pt]{2.409pt}{0.400pt}}
\put(170.0,337.0){\rule[-0.200pt]{2.409pt}{0.400pt}}
\put(1429.0,337.0){\rule[-0.200pt]{2.409pt}{0.400pt}}
\put(170.0,337.0){\rule[-0.200pt]{2.409pt}{0.400pt}}
\put(1429.0,337.0){\rule[-0.200pt]{2.409pt}{0.400pt}}
\put(170.0,337.0){\rule[-0.200pt]{2.409pt}{0.400pt}}
\put(1429.0,337.0){\rule[-0.200pt]{2.409pt}{0.400pt}}
\put(170.0,337.0){\rule[-0.200pt]{2.409pt}{0.400pt}}
\put(1429.0,337.0){\rule[-0.200pt]{2.409pt}{0.400pt}}
\put(170.0,337.0){\rule[-0.200pt]{2.409pt}{0.400pt}}
\put(1429.0,337.0){\rule[-0.200pt]{2.409pt}{0.400pt}}
\put(170.0,337.0){\rule[-0.200pt]{2.409pt}{0.400pt}}
\put(1429.0,337.0){\rule[-0.200pt]{2.409pt}{0.400pt}}
\put(170.0,337.0){\rule[-0.200pt]{2.409pt}{0.400pt}}
\put(1429.0,337.0){\rule[-0.200pt]{2.409pt}{0.400pt}}
\put(170.0,337.0){\rule[-0.200pt]{2.409pt}{0.400pt}}
\put(1429.0,337.0){\rule[-0.200pt]{2.409pt}{0.400pt}}
\put(170.0,337.0){\rule[-0.200pt]{2.409pt}{0.400pt}}
\put(1429.0,337.0){\rule[-0.200pt]{2.409pt}{0.400pt}}
\put(170.0,337.0){\rule[-0.200pt]{2.409pt}{0.400pt}}
\put(1429.0,337.0){\rule[-0.200pt]{2.409pt}{0.400pt}}
\put(170.0,337.0){\rule[-0.200pt]{2.409pt}{0.400pt}}
\put(1429.0,337.0){\rule[-0.200pt]{2.409pt}{0.400pt}}
\put(170.0,337.0){\rule[-0.200pt]{2.409pt}{0.400pt}}
\put(1429.0,337.0){\rule[-0.200pt]{2.409pt}{0.400pt}}
\put(170.0,337.0){\rule[-0.200pt]{2.409pt}{0.400pt}}
\put(1429.0,337.0){\rule[-0.200pt]{2.409pt}{0.400pt}}
\put(170.0,338.0){\rule[-0.200pt]{2.409pt}{0.400pt}}
\put(1429.0,338.0){\rule[-0.200pt]{2.409pt}{0.400pt}}
\put(170.0,338.0){\rule[-0.200pt]{2.409pt}{0.400pt}}
\put(1429.0,338.0){\rule[-0.200pt]{2.409pt}{0.400pt}}
\put(170.0,338.0){\rule[-0.200pt]{2.409pt}{0.400pt}}
\put(1429.0,338.0){\rule[-0.200pt]{2.409pt}{0.400pt}}
\put(170.0,338.0){\rule[-0.200pt]{2.409pt}{0.400pt}}
\put(1429.0,338.0){\rule[-0.200pt]{2.409pt}{0.400pt}}
\put(170.0,338.0){\rule[-0.200pt]{2.409pt}{0.400pt}}
\put(1429.0,338.0){\rule[-0.200pt]{2.409pt}{0.400pt}}
\put(170.0,338.0){\rule[-0.200pt]{2.409pt}{0.400pt}}
\put(1429.0,338.0){\rule[-0.200pt]{2.409pt}{0.400pt}}
\put(170.0,338.0){\rule[-0.200pt]{2.409pt}{0.400pt}}
\put(1429.0,338.0){\rule[-0.200pt]{2.409pt}{0.400pt}}
\put(170.0,338.0){\rule[-0.200pt]{2.409pt}{0.400pt}}
\put(1429.0,338.0){\rule[-0.200pt]{2.409pt}{0.400pt}}
\put(170.0,338.0){\rule[-0.200pt]{2.409pt}{0.400pt}}
\put(1429.0,338.0){\rule[-0.200pt]{2.409pt}{0.400pt}}
\put(170.0,338.0){\rule[-0.200pt]{2.409pt}{0.400pt}}
\put(1429.0,338.0){\rule[-0.200pt]{2.409pt}{0.400pt}}
\put(170.0,338.0){\rule[-0.200pt]{2.409pt}{0.400pt}}
\put(1429.0,338.0){\rule[-0.200pt]{2.409pt}{0.400pt}}
\put(170.0,338.0){\rule[-0.200pt]{2.409pt}{0.400pt}}
\put(1429.0,338.0){\rule[-0.200pt]{2.409pt}{0.400pt}}
\put(170.0,338.0){\rule[-0.200pt]{2.409pt}{0.400pt}}
\put(1429.0,338.0){\rule[-0.200pt]{2.409pt}{0.400pt}}
\put(170.0,338.0){\rule[-0.200pt]{2.409pt}{0.400pt}}
\put(1429.0,338.0){\rule[-0.200pt]{2.409pt}{0.400pt}}
\put(170.0,338.0){\rule[-0.200pt]{2.409pt}{0.400pt}}
\put(1429.0,338.0){\rule[-0.200pt]{2.409pt}{0.400pt}}
\put(170.0,338.0){\rule[-0.200pt]{2.409pt}{0.400pt}}
\put(1429.0,338.0){\rule[-0.200pt]{2.409pt}{0.400pt}}
\put(170.0,339.0){\rule[-0.200pt]{2.409pt}{0.400pt}}
\put(1429.0,339.0){\rule[-0.200pt]{2.409pt}{0.400pt}}
\put(170.0,339.0){\rule[-0.200pt]{2.409pt}{0.400pt}}
\put(1429.0,339.0){\rule[-0.200pt]{2.409pt}{0.400pt}}
\put(170.0,339.0){\rule[-0.200pt]{2.409pt}{0.400pt}}
\put(1429.0,339.0){\rule[-0.200pt]{2.409pt}{0.400pt}}
\put(170.0,339.0){\rule[-0.200pt]{2.409pt}{0.400pt}}
\put(1429.0,339.0){\rule[-0.200pt]{2.409pt}{0.400pt}}
\put(170.0,339.0){\rule[-0.200pt]{2.409pt}{0.400pt}}
\put(1429.0,339.0){\rule[-0.200pt]{2.409pt}{0.400pt}}
\put(170.0,339.0){\rule[-0.200pt]{2.409pt}{0.400pt}}
\put(1429.0,339.0){\rule[-0.200pt]{2.409pt}{0.400pt}}
\put(170.0,339.0){\rule[-0.200pt]{2.409pt}{0.400pt}}
\put(1429.0,339.0){\rule[-0.200pt]{2.409pt}{0.400pt}}
\put(170.0,339.0){\rule[-0.200pt]{2.409pt}{0.400pt}}
\put(1429.0,339.0){\rule[-0.200pt]{2.409pt}{0.400pt}}
\put(170.0,339.0){\rule[-0.200pt]{2.409pt}{0.400pt}}
\put(1429.0,339.0){\rule[-0.200pt]{2.409pt}{0.400pt}}
\put(170.0,339.0){\rule[-0.200pt]{2.409pt}{0.400pt}}
\put(1429.0,339.0){\rule[-0.200pt]{2.409pt}{0.400pt}}
\put(170.0,339.0){\rule[-0.200pt]{2.409pt}{0.400pt}}
\put(1429.0,339.0){\rule[-0.200pt]{2.409pt}{0.400pt}}
\put(170.0,339.0){\rule[-0.200pt]{2.409pt}{0.400pt}}
\put(1429.0,339.0){\rule[-0.200pt]{2.409pt}{0.400pt}}
\put(170.0,339.0){\rule[-0.200pt]{2.409pt}{0.400pt}}
\put(1429.0,339.0){\rule[-0.200pt]{2.409pt}{0.400pt}}
\put(170.0,339.0){\rule[-0.200pt]{2.409pt}{0.400pt}}
\put(1429.0,339.0){\rule[-0.200pt]{2.409pt}{0.400pt}}
\put(170.0,339.0){\rule[-0.200pt]{2.409pt}{0.400pt}}
\put(1429.0,339.0){\rule[-0.200pt]{2.409pt}{0.400pt}}
\put(170.0,339.0){\rule[-0.200pt]{2.409pt}{0.400pt}}
\put(1429.0,339.0){\rule[-0.200pt]{2.409pt}{0.400pt}}
\put(170.0,340.0){\rule[-0.200pt]{2.409pt}{0.400pt}}
\put(1429.0,340.0){\rule[-0.200pt]{2.409pt}{0.400pt}}
\put(170.0,340.0){\rule[-0.200pt]{2.409pt}{0.400pt}}
\put(1429.0,340.0){\rule[-0.200pt]{2.409pt}{0.400pt}}
\put(170.0,340.0){\rule[-0.200pt]{2.409pt}{0.400pt}}
\put(1429.0,340.0){\rule[-0.200pt]{2.409pt}{0.400pt}}
\put(170.0,340.0){\rule[-0.200pt]{2.409pt}{0.400pt}}
\put(1429.0,340.0){\rule[-0.200pt]{2.409pt}{0.400pt}}
\put(170.0,340.0){\rule[-0.200pt]{2.409pt}{0.400pt}}
\put(1429.0,340.0){\rule[-0.200pt]{2.409pt}{0.400pt}}
\put(170.0,340.0){\rule[-0.200pt]{2.409pt}{0.400pt}}
\put(1429.0,340.0){\rule[-0.200pt]{2.409pt}{0.400pt}}
\put(170.0,340.0){\rule[-0.200pt]{2.409pt}{0.400pt}}
\put(1429.0,340.0){\rule[-0.200pt]{2.409pt}{0.400pt}}
\put(170.0,340.0){\rule[-0.200pt]{2.409pt}{0.400pt}}
\put(1429.0,340.0){\rule[-0.200pt]{2.409pt}{0.400pt}}
\put(170.0,340.0){\rule[-0.200pt]{2.409pt}{0.400pt}}
\put(1429.0,340.0){\rule[-0.200pt]{2.409pt}{0.400pt}}
\put(170.0,340.0){\rule[-0.200pt]{2.409pt}{0.400pt}}
\put(1429.0,340.0){\rule[-0.200pt]{2.409pt}{0.400pt}}
\put(170.0,340.0){\rule[-0.200pt]{2.409pt}{0.400pt}}
\put(1429.0,340.0){\rule[-0.200pt]{2.409pt}{0.400pt}}
\put(170.0,340.0){\rule[-0.200pt]{2.409pt}{0.400pt}}
\put(1429.0,340.0){\rule[-0.200pt]{2.409pt}{0.400pt}}
\put(170.0,340.0){\rule[-0.200pt]{2.409pt}{0.400pt}}
\put(1429.0,340.0){\rule[-0.200pt]{2.409pt}{0.400pt}}
\put(170.0,340.0){\rule[-0.200pt]{2.409pt}{0.400pt}}
\put(1429.0,340.0){\rule[-0.200pt]{2.409pt}{0.400pt}}
\put(170.0,340.0){\rule[-0.200pt]{2.409pt}{0.400pt}}
\put(1429.0,340.0){\rule[-0.200pt]{2.409pt}{0.400pt}}
\put(170.0,340.0){\rule[-0.200pt]{2.409pt}{0.400pt}}
\put(1429.0,340.0){\rule[-0.200pt]{2.409pt}{0.400pt}}
\put(170.0,341.0){\rule[-0.200pt]{2.409pt}{0.400pt}}
\put(1429.0,341.0){\rule[-0.200pt]{2.409pt}{0.400pt}}
\put(170.0,341.0){\rule[-0.200pt]{2.409pt}{0.400pt}}
\put(1429.0,341.0){\rule[-0.200pt]{2.409pt}{0.400pt}}
\put(170.0,341.0){\rule[-0.200pt]{2.409pt}{0.400pt}}
\put(1429.0,341.0){\rule[-0.200pt]{2.409pt}{0.400pt}}
\put(170.0,341.0){\rule[-0.200pt]{2.409pt}{0.400pt}}
\put(1429.0,341.0){\rule[-0.200pt]{2.409pt}{0.400pt}}
\put(170.0,341.0){\rule[-0.200pt]{2.409pt}{0.400pt}}
\put(1429.0,341.0){\rule[-0.200pt]{2.409pt}{0.400pt}}
\put(170.0,341.0){\rule[-0.200pt]{2.409pt}{0.400pt}}
\put(1429.0,341.0){\rule[-0.200pt]{2.409pt}{0.400pt}}
\put(170.0,341.0){\rule[-0.200pt]{2.409pt}{0.400pt}}
\put(1429.0,341.0){\rule[-0.200pt]{2.409pt}{0.400pt}}
\put(170.0,341.0){\rule[-0.200pt]{2.409pt}{0.400pt}}
\put(1429.0,341.0){\rule[-0.200pt]{2.409pt}{0.400pt}}
\put(170.0,341.0){\rule[-0.200pt]{2.409pt}{0.400pt}}
\put(1429.0,341.0){\rule[-0.200pt]{2.409pt}{0.400pt}}
\put(170.0,341.0){\rule[-0.200pt]{4.818pt}{0.400pt}}
\put(150,341){\makebox(0,0)[r]{ 1e-09}}
\put(1419.0,341.0){\rule[-0.200pt]{4.818pt}{0.400pt}}
\put(170.0,354.0){\rule[-0.200pt]{2.409pt}{0.400pt}}
\put(1429.0,354.0){\rule[-0.200pt]{2.409pt}{0.400pt}}
\put(170.0,371.0){\rule[-0.200pt]{2.409pt}{0.400pt}}
\put(1429.0,371.0){\rule[-0.200pt]{2.409pt}{0.400pt}}
\put(170.0,380.0){\rule[-0.200pt]{2.409pt}{0.400pt}}
\put(1429.0,380.0){\rule[-0.200pt]{2.409pt}{0.400pt}}
\put(170.0,386.0){\rule[-0.200pt]{2.409pt}{0.400pt}}
\put(1429.0,386.0){\rule[-0.200pt]{2.409pt}{0.400pt}}
\put(170.0,390.0){\rule[-0.200pt]{2.409pt}{0.400pt}}
\put(1429.0,390.0){\rule[-0.200pt]{2.409pt}{0.400pt}}
\put(170.0,394.0){\rule[-0.200pt]{2.409pt}{0.400pt}}
\put(1429.0,394.0){\rule[-0.200pt]{2.409pt}{0.400pt}}
\put(170.0,397.0){\rule[-0.200pt]{2.409pt}{0.400pt}}
\put(1429.0,397.0){\rule[-0.200pt]{2.409pt}{0.400pt}}
\put(170.0,400.0){\rule[-0.200pt]{2.409pt}{0.400pt}}
\put(1429.0,400.0){\rule[-0.200pt]{2.409pt}{0.400pt}}
\put(170.0,402.0){\rule[-0.200pt]{2.409pt}{0.400pt}}
\put(1429.0,402.0){\rule[-0.200pt]{2.409pt}{0.400pt}}
\put(170.0,404.0){\rule[-0.200pt]{2.409pt}{0.400pt}}
\put(1429.0,404.0){\rule[-0.200pt]{2.409pt}{0.400pt}}
\put(170.0,406.0){\rule[-0.200pt]{2.409pt}{0.400pt}}
\put(1429.0,406.0){\rule[-0.200pt]{2.409pt}{0.400pt}}
\put(170.0,408.0){\rule[-0.200pt]{2.409pt}{0.400pt}}
\put(1429.0,408.0){\rule[-0.200pt]{2.409pt}{0.400pt}}
\put(170.0,409.0){\rule[-0.200pt]{2.409pt}{0.400pt}}
\put(1429.0,409.0){\rule[-0.200pt]{2.409pt}{0.400pt}}
\put(170.0,411.0){\rule[-0.200pt]{2.409pt}{0.400pt}}
\put(1429.0,411.0){\rule[-0.200pt]{2.409pt}{0.400pt}}
\put(170.0,412.0){\rule[-0.200pt]{2.409pt}{0.400pt}}
\put(1429.0,412.0){\rule[-0.200pt]{2.409pt}{0.400pt}}
\put(170.0,413.0){\rule[-0.200pt]{2.409pt}{0.400pt}}
\put(1429.0,413.0){\rule[-0.200pt]{2.409pt}{0.400pt}}
\put(170.0,414.0){\rule[-0.200pt]{2.409pt}{0.400pt}}
\put(1429.0,414.0){\rule[-0.200pt]{2.409pt}{0.400pt}}
\put(170.0,415.0){\rule[-0.200pt]{2.409pt}{0.400pt}}
\put(1429.0,415.0){\rule[-0.200pt]{2.409pt}{0.400pt}}
\put(170.0,416.0){\rule[-0.200pt]{2.409pt}{0.400pt}}
\put(1429.0,416.0){\rule[-0.200pt]{2.409pt}{0.400pt}}
\put(170.0,417.0){\rule[-0.200pt]{2.409pt}{0.400pt}}
\put(1429.0,417.0){\rule[-0.200pt]{2.409pt}{0.400pt}}
\put(170.0,418.0){\rule[-0.200pt]{2.409pt}{0.400pt}}
\put(1429.0,418.0){\rule[-0.200pt]{2.409pt}{0.400pt}}
\put(170.0,419.0){\rule[-0.200pt]{2.409pt}{0.400pt}}
\put(1429.0,419.0){\rule[-0.200pt]{2.409pt}{0.400pt}}
\put(170.0,420.0){\rule[-0.200pt]{2.409pt}{0.400pt}}
\put(1429.0,420.0){\rule[-0.200pt]{2.409pt}{0.400pt}}
\put(170.0,421.0){\rule[-0.200pt]{2.409pt}{0.400pt}}
\put(1429.0,421.0){\rule[-0.200pt]{2.409pt}{0.400pt}}
\put(170.0,422.0){\rule[-0.200pt]{2.409pt}{0.400pt}}
\put(1429.0,422.0){\rule[-0.200pt]{2.409pt}{0.400pt}}
\put(170.0,422.0){\rule[-0.200pt]{2.409pt}{0.400pt}}
\put(1429.0,422.0){\rule[-0.200pt]{2.409pt}{0.400pt}}
\put(170.0,423.0){\rule[-0.200pt]{2.409pt}{0.400pt}}
\put(1429.0,423.0){\rule[-0.200pt]{2.409pt}{0.400pt}}
\put(170.0,424.0){\rule[-0.200pt]{2.409pt}{0.400pt}}
\put(1429.0,424.0){\rule[-0.200pt]{2.409pt}{0.400pt}}
\put(170.0,425.0){\rule[-0.200pt]{2.409pt}{0.400pt}}
\put(1429.0,425.0){\rule[-0.200pt]{2.409pt}{0.400pt}}
\put(170.0,425.0){\rule[-0.200pt]{2.409pt}{0.400pt}}
\put(1429.0,425.0){\rule[-0.200pt]{2.409pt}{0.400pt}}
\put(170.0,426.0){\rule[-0.200pt]{2.409pt}{0.400pt}}
\put(1429.0,426.0){\rule[-0.200pt]{2.409pt}{0.400pt}}
\put(170.0,426.0){\rule[-0.200pt]{2.409pt}{0.400pt}}
\put(1429.0,426.0){\rule[-0.200pt]{2.409pt}{0.400pt}}
\put(170.0,427.0){\rule[-0.200pt]{2.409pt}{0.400pt}}
\put(1429.0,427.0){\rule[-0.200pt]{2.409pt}{0.400pt}}
\put(170.0,428.0){\rule[-0.200pt]{2.409pt}{0.400pt}}
\put(1429.0,428.0){\rule[-0.200pt]{2.409pt}{0.400pt}}
\put(170.0,428.0){\rule[-0.200pt]{2.409pt}{0.400pt}}
\put(1429.0,428.0){\rule[-0.200pt]{2.409pt}{0.400pt}}
\put(170.0,429.0){\rule[-0.200pt]{2.409pt}{0.400pt}}
\put(1429.0,429.0){\rule[-0.200pt]{2.409pt}{0.400pt}}
\put(170.0,429.0){\rule[-0.200pt]{2.409pt}{0.400pt}}
\put(1429.0,429.0){\rule[-0.200pt]{2.409pt}{0.400pt}}
\put(170.0,430.0){\rule[-0.200pt]{2.409pt}{0.400pt}}
\put(1429.0,430.0){\rule[-0.200pt]{2.409pt}{0.400pt}}
\put(170.0,430.0){\rule[-0.200pt]{2.409pt}{0.400pt}}
\put(1429.0,430.0){\rule[-0.200pt]{2.409pt}{0.400pt}}
\put(170.0,431.0){\rule[-0.200pt]{2.409pt}{0.400pt}}
\put(1429.0,431.0){\rule[-0.200pt]{2.409pt}{0.400pt}}
\put(170.0,431.0){\rule[-0.200pt]{2.409pt}{0.400pt}}
\put(1429.0,431.0){\rule[-0.200pt]{2.409pt}{0.400pt}}
\put(170.0,432.0){\rule[-0.200pt]{2.409pt}{0.400pt}}
\put(1429.0,432.0){\rule[-0.200pt]{2.409pt}{0.400pt}}
\put(170.0,432.0){\rule[-0.200pt]{2.409pt}{0.400pt}}
\put(1429.0,432.0){\rule[-0.200pt]{2.409pt}{0.400pt}}
\put(170.0,432.0){\rule[-0.200pt]{2.409pt}{0.400pt}}
\put(1429.0,432.0){\rule[-0.200pt]{2.409pt}{0.400pt}}
\put(170.0,433.0){\rule[-0.200pt]{2.409pt}{0.400pt}}
\put(1429.0,433.0){\rule[-0.200pt]{2.409pt}{0.400pt}}
\put(170.0,433.0){\rule[-0.200pt]{2.409pt}{0.400pt}}
\put(1429.0,433.0){\rule[-0.200pt]{2.409pt}{0.400pt}}
\put(170.0,434.0){\rule[-0.200pt]{2.409pt}{0.400pt}}
\put(1429.0,434.0){\rule[-0.200pt]{2.409pt}{0.400pt}}
\put(170.0,434.0){\rule[-0.200pt]{2.409pt}{0.400pt}}
\put(1429.0,434.0){\rule[-0.200pt]{2.409pt}{0.400pt}}
\put(170.0,434.0){\rule[-0.200pt]{2.409pt}{0.400pt}}
\put(1429.0,434.0){\rule[-0.200pt]{2.409pt}{0.400pt}}
\put(170.0,435.0){\rule[-0.200pt]{2.409pt}{0.400pt}}
\put(1429.0,435.0){\rule[-0.200pt]{2.409pt}{0.400pt}}
\put(170.0,435.0){\rule[-0.200pt]{2.409pt}{0.400pt}}
\put(1429.0,435.0){\rule[-0.200pt]{2.409pt}{0.400pt}}
\put(170.0,436.0){\rule[-0.200pt]{2.409pt}{0.400pt}}
\put(1429.0,436.0){\rule[-0.200pt]{2.409pt}{0.400pt}}
\put(170.0,436.0){\rule[-0.200pt]{2.409pt}{0.400pt}}
\put(1429.0,436.0){\rule[-0.200pt]{2.409pt}{0.400pt}}
\put(170.0,436.0){\rule[-0.200pt]{2.409pt}{0.400pt}}
\put(1429.0,436.0){\rule[-0.200pt]{2.409pt}{0.400pt}}
\put(170.0,437.0){\rule[-0.200pt]{2.409pt}{0.400pt}}
\put(1429.0,437.0){\rule[-0.200pt]{2.409pt}{0.400pt}}
\put(170.0,437.0){\rule[-0.200pt]{2.409pt}{0.400pt}}
\put(1429.0,437.0){\rule[-0.200pt]{2.409pt}{0.400pt}}
\put(170.0,437.0){\rule[-0.200pt]{2.409pt}{0.400pt}}
\put(1429.0,437.0){\rule[-0.200pt]{2.409pt}{0.400pt}}
\put(170.0,438.0){\rule[-0.200pt]{2.409pt}{0.400pt}}
\put(1429.0,438.0){\rule[-0.200pt]{2.409pt}{0.400pt}}
\put(170.0,438.0){\rule[-0.200pt]{2.409pt}{0.400pt}}
\put(1429.0,438.0){\rule[-0.200pt]{2.409pt}{0.400pt}}
\put(170.0,438.0){\rule[-0.200pt]{2.409pt}{0.400pt}}
\put(1429.0,438.0){\rule[-0.200pt]{2.409pt}{0.400pt}}
\put(170.0,439.0){\rule[-0.200pt]{2.409pt}{0.400pt}}
\put(1429.0,439.0){\rule[-0.200pt]{2.409pt}{0.400pt}}
\put(170.0,439.0){\rule[-0.200pt]{2.409pt}{0.400pt}}
\put(1429.0,439.0){\rule[-0.200pt]{2.409pt}{0.400pt}}
\put(170.0,439.0){\rule[-0.200pt]{2.409pt}{0.400pt}}
\put(1429.0,439.0){\rule[-0.200pt]{2.409pt}{0.400pt}}
\put(170.0,439.0){\rule[-0.200pt]{2.409pt}{0.400pt}}
\put(1429.0,439.0){\rule[-0.200pt]{2.409pt}{0.400pt}}
\put(170.0,440.0){\rule[-0.200pt]{2.409pt}{0.400pt}}
\put(1429.0,440.0){\rule[-0.200pt]{2.409pt}{0.400pt}}
\put(170.0,440.0){\rule[-0.200pt]{2.409pt}{0.400pt}}
\put(1429.0,440.0){\rule[-0.200pt]{2.409pt}{0.400pt}}
\put(170.0,440.0){\rule[-0.200pt]{2.409pt}{0.400pt}}
\put(1429.0,440.0){\rule[-0.200pt]{2.409pt}{0.400pt}}
\put(170.0,441.0){\rule[-0.200pt]{2.409pt}{0.400pt}}
\put(1429.0,441.0){\rule[-0.200pt]{2.409pt}{0.400pt}}
\put(170.0,441.0){\rule[-0.200pt]{2.409pt}{0.400pt}}
\put(1429.0,441.0){\rule[-0.200pt]{2.409pt}{0.400pt}}
\put(170.0,441.0){\rule[-0.200pt]{2.409pt}{0.400pt}}
\put(1429.0,441.0){\rule[-0.200pt]{2.409pt}{0.400pt}}
\put(170.0,441.0){\rule[-0.200pt]{2.409pt}{0.400pt}}
\put(1429.0,441.0){\rule[-0.200pt]{2.409pt}{0.400pt}}
\put(170.0,442.0){\rule[-0.200pt]{2.409pt}{0.400pt}}
\put(1429.0,442.0){\rule[-0.200pt]{2.409pt}{0.400pt}}
\put(170.0,442.0){\rule[-0.200pt]{2.409pt}{0.400pt}}
\put(1429.0,442.0){\rule[-0.200pt]{2.409pt}{0.400pt}}
\put(170.0,442.0){\rule[-0.200pt]{2.409pt}{0.400pt}}
\put(1429.0,442.0){\rule[-0.200pt]{2.409pt}{0.400pt}}
\put(170.0,442.0){\rule[-0.200pt]{2.409pt}{0.400pt}}
\put(1429.0,442.0){\rule[-0.200pt]{2.409pt}{0.400pt}}
\put(170.0,443.0){\rule[-0.200pt]{2.409pt}{0.400pt}}
\put(1429.0,443.0){\rule[-0.200pt]{2.409pt}{0.400pt}}
\put(170.0,443.0){\rule[-0.200pt]{2.409pt}{0.400pt}}
\put(1429.0,443.0){\rule[-0.200pt]{2.409pt}{0.400pt}}
\put(170.0,443.0){\rule[-0.200pt]{2.409pt}{0.400pt}}
\put(1429.0,443.0){\rule[-0.200pt]{2.409pt}{0.400pt}}
\put(170.0,443.0){\rule[-0.200pt]{2.409pt}{0.400pt}}
\put(1429.0,443.0){\rule[-0.200pt]{2.409pt}{0.400pt}}
\put(170.0,444.0){\rule[-0.200pt]{2.409pt}{0.400pt}}
\put(1429.0,444.0){\rule[-0.200pt]{2.409pt}{0.400pt}}
\put(170.0,444.0){\rule[-0.200pt]{2.409pt}{0.400pt}}
\put(1429.0,444.0){\rule[-0.200pt]{2.409pt}{0.400pt}}
\put(170.0,444.0){\rule[-0.200pt]{2.409pt}{0.400pt}}
\put(1429.0,444.0){\rule[-0.200pt]{2.409pt}{0.400pt}}
\put(170.0,444.0){\rule[-0.200pt]{2.409pt}{0.400pt}}
\put(1429.0,444.0){\rule[-0.200pt]{2.409pt}{0.400pt}}
\put(170.0,445.0){\rule[-0.200pt]{2.409pt}{0.400pt}}
\put(1429.0,445.0){\rule[-0.200pt]{2.409pt}{0.400pt}}
\put(170.0,445.0){\rule[-0.200pt]{2.409pt}{0.400pt}}
\put(1429.0,445.0){\rule[-0.200pt]{2.409pt}{0.400pt}}
\put(170.0,445.0){\rule[-0.200pt]{2.409pt}{0.400pt}}
\put(1429.0,445.0){\rule[-0.200pt]{2.409pt}{0.400pt}}
\put(170.0,445.0){\rule[-0.200pt]{2.409pt}{0.400pt}}
\put(1429.0,445.0){\rule[-0.200pt]{2.409pt}{0.400pt}}
\put(170.0,445.0){\rule[-0.200pt]{2.409pt}{0.400pt}}
\put(1429.0,445.0){\rule[-0.200pt]{2.409pt}{0.400pt}}
\put(170.0,446.0){\rule[-0.200pt]{2.409pt}{0.400pt}}
\put(1429.0,446.0){\rule[-0.200pt]{2.409pt}{0.400pt}}
\put(170.0,446.0){\rule[-0.200pt]{2.409pt}{0.400pt}}
\put(1429.0,446.0){\rule[-0.200pt]{2.409pt}{0.400pt}}
\put(170.0,446.0){\rule[-0.200pt]{2.409pt}{0.400pt}}
\put(1429.0,446.0){\rule[-0.200pt]{2.409pt}{0.400pt}}
\put(170.0,446.0){\rule[-0.200pt]{2.409pt}{0.400pt}}
\put(1429.0,446.0){\rule[-0.200pt]{2.409pt}{0.400pt}}
\put(170.0,447.0){\rule[-0.200pt]{2.409pt}{0.400pt}}
\put(1429.0,447.0){\rule[-0.200pt]{2.409pt}{0.400pt}}
\put(170.0,447.0){\rule[-0.200pt]{2.409pt}{0.400pt}}
\put(1429.0,447.0){\rule[-0.200pt]{2.409pt}{0.400pt}}
\put(170.0,447.0){\rule[-0.200pt]{2.409pt}{0.400pt}}
\put(1429.0,447.0){\rule[-0.200pt]{2.409pt}{0.400pt}}
\put(170.0,447.0){\rule[-0.200pt]{2.409pt}{0.400pt}}
\put(1429.0,447.0){\rule[-0.200pt]{2.409pt}{0.400pt}}
\put(170.0,447.0){\rule[-0.200pt]{2.409pt}{0.400pt}}
\put(1429.0,447.0){\rule[-0.200pt]{2.409pt}{0.400pt}}
\put(170.0,447.0){\rule[-0.200pt]{2.409pt}{0.400pt}}
\put(1429.0,447.0){\rule[-0.200pt]{2.409pt}{0.400pt}}
\put(170.0,448.0){\rule[-0.200pt]{2.409pt}{0.400pt}}
\put(1429.0,448.0){\rule[-0.200pt]{2.409pt}{0.400pt}}
\put(170.0,448.0){\rule[-0.200pt]{2.409pt}{0.400pt}}
\put(1429.0,448.0){\rule[-0.200pt]{2.409pt}{0.400pt}}
\put(170.0,448.0){\rule[-0.200pt]{2.409pt}{0.400pt}}
\put(1429.0,448.0){\rule[-0.200pt]{2.409pt}{0.400pt}}
\put(170.0,448.0){\rule[-0.200pt]{2.409pt}{0.400pt}}
\put(1429.0,448.0){\rule[-0.200pt]{2.409pt}{0.400pt}}
\put(170.0,448.0){\rule[-0.200pt]{2.409pt}{0.400pt}}
\put(1429.0,448.0){\rule[-0.200pt]{2.409pt}{0.400pt}}
\put(170.0,449.0){\rule[-0.200pt]{2.409pt}{0.400pt}}
\put(1429.0,449.0){\rule[-0.200pt]{2.409pt}{0.400pt}}
\put(170.0,449.0){\rule[-0.200pt]{2.409pt}{0.400pt}}
\put(1429.0,449.0){\rule[-0.200pt]{2.409pt}{0.400pt}}
\put(170.0,449.0){\rule[-0.200pt]{2.409pt}{0.400pt}}
\put(1429.0,449.0){\rule[-0.200pt]{2.409pt}{0.400pt}}
\put(170.0,449.0){\rule[-0.200pt]{2.409pt}{0.400pt}}
\put(1429.0,449.0){\rule[-0.200pt]{2.409pt}{0.400pt}}
\put(170.0,449.0){\rule[-0.200pt]{2.409pt}{0.400pt}}
\put(1429.0,449.0){\rule[-0.200pt]{2.409pt}{0.400pt}}
\put(170.0,449.0){\rule[-0.200pt]{2.409pt}{0.400pt}}
\put(1429.0,449.0){\rule[-0.200pt]{2.409pt}{0.400pt}}
\put(170.0,450.0){\rule[-0.200pt]{2.409pt}{0.400pt}}
\put(1429.0,450.0){\rule[-0.200pt]{2.409pt}{0.400pt}}
\put(170.0,450.0){\rule[-0.200pt]{2.409pt}{0.400pt}}
\put(1429.0,450.0){\rule[-0.200pt]{2.409pt}{0.400pt}}
\put(170.0,450.0){\rule[-0.200pt]{2.409pt}{0.400pt}}
\put(1429.0,450.0){\rule[-0.200pt]{2.409pt}{0.400pt}}
\put(170.0,450.0){\rule[-0.200pt]{2.409pt}{0.400pt}}
\put(1429.0,450.0){\rule[-0.200pt]{2.409pt}{0.400pt}}
\put(170.0,450.0){\rule[-0.200pt]{2.409pt}{0.400pt}}
\put(1429.0,450.0){\rule[-0.200pt]{2.409pt}{0.400pt}}
\put(170.0,450.0){\rule[-0.200pt]{2.409pt}{0.400pt}}
\put(1429.0,450.0){\rule[-0.200pt]{2.409pt}{0.400pt}}
\put(170.0,451.0){\rule[-0.200pt]{2.409pt}{0.400pt}}
\put(1429.0,451.0){\rule[-0.200pt]{2.409pt}{0.400pt}}
\put(170.0,451.0){\rule[-0.200pt]{2.409pt}{0.400pt}}
\put(1429.0,451.0){\rule[-0.200pt]{2.409pt}{0.400pt}}
\put(170.0,451.0){\rule[-0.200pt]{2.409pt}{0.400pt}}
\put(1429.0,451.0){\rule[-0.200pt]{2.409pt}{0.400pt}}
\put(170.0,451.0){\rule[-0.200pt]{2.409pt}{0.400pt}}
\put(1429.0,451.0){\rule[-0.200pt]{2.409pt}{0.400pt}}
\put(170.0,451.0){\rule[-0.200pt]{2.409pt}{0.400pt}}
\put(1429.0,451.0){\rule[-0.200pt]{2.409pt}{0.400pt}}
\put(170.0,451.0){\rule[-0.200pt]{2.409pt}{0.400pt}}
\put(1429.0,451.0){\rule[-0.200pt]{2.409pt}{0.400pt}}
\put(170.0,452.0){\rule[-0.200pt]{2.409pt}{0.400pt}}
\put(1429.0,452.0){\rule[-0.200pt]{2.409pt}{0.400pt}}
\put(170.0,452.0){\rule[-0.200pt]{2.409pt}{0.400pt}}
\put(1429.0,452.0){\rule[-0.200pt]{2.409pt}{0.400pt}}
\put(170.0,452.0){\rule[-0.200pt]{2.409pt}{0.400pt}}
\put(1429.0,452.0){\rule[-0.200pt]{2.409pt}{0.400pt}}
\put(170.0,452.0){\rule[-0.200pt]{2.409pt}{0.400pt}}
\put(1429.0,452.0){\rule[-0.200pt]{2.409pt}{0.400pt}}
\put(170.0,452.0){\rule[-0.200pt]{2.409pt}{0.400pt}}
\put(1429.0,452.0){\rule[-0.200pt]{2.409pt}{0.400pt}}
\put(170.0,452.0){\rule[-0.200pt]{2.409pt}{0.400pt}}
\put(1429.0,452.0){\rule[-0.200pt]{2.409pt}{0.400pt}}
\put(170.0,453.0){\rule[-0.200pt]{2.409pt}{0.400pt}}
\put(1429.0,453.0){\rule[-0.200pt]{2.409pt}{0.400pt}}
\put(170.0,453.0){\rule[-0.200pt]{2.409pt}{0.400pt}}
\put(1429.0,453.0){\rule[-0.200pt]{2.409pt}{0.400pt}}
\put(170.0,453.0){\rule[-0.200pt]{2.409pt}{0.400pt}}
\put(1429.0,453.0){\rule[-0.200pt]{2.409pt}{0.400pt}}
\put(170.0,453.0){\rule[-0.200pt]{2.409pt}{0.400pt}}
\put(1429.0,453.0){\rule[-0.200pt]{2.409pt}{0.400pt}}
\put(170.0,453.0){\rule[-0.200pt]{2.409pt}{0.400pt}}
\put(1429.0,453.0){\rule[-0.200pt]{2.409pt}{0.400pt}}
\put(170.0,453.0){\rule[-0.200pt]{2.409pt}{0.400pt}}
\put(1429.0,453.0){\rule[-0.200pt]{2.409pt}{0.400pt}}
\put(170.0,453.0){\rule[-0.200pt]{2.409pt}{0.400pt}}
\put(1429.0,453.0){\rule[-0.200pt]{2.409pt}{0.400pt}}
\put(170.0,454.0){\rule[-0.200pt]{2.409pt}{0.400pt}}
\put(1429.0,454.0){\rule[-0.200pt]{2.409pt}{0.400pt}}
\put(170.0,454.0){\rule[-0.200pt]{2.409pt}{0.400pt}}
\put(1429.0,454.0){\rule[-0.200pt]{2.409pt}{0.400pt}}
\put(170.0,454.0){\rule[-0.200pt]{2.409pt}{0.400pt}}
\put(1429.0,454.0){\rule[-0.200pt]{2.409pt}{0.400pt}}
\put(170.0,454.0){\rule[-0.200pt]{2.409pt}{0.400pt}}
\put(1429.0,454.0){\rule[-0.200pt]{2.409pt}{0.400pt}}
\put(170.0,454.0){\rule[-0.200pt]{2.409pt}{0.400pt}}
\put(1429.0,454.0){\rule[-0.200pt]{2.409pt}{0.400pt}}
\put(170.0,454.0){\rule[-0.200pt]{2.409pt}{0.400pt}}
\put(1429.0,454.0){\rule[-0.200pt]{2.409pt}{0.400pt}}
\put(170.0,454.0){\rule[-0.200pt]{2.409pt}{0.400pt}}
\put(1429.0,454.0){\rule[-0.200pt]{2.409pt}{0.400pt}}
\put(170.0,454.0){\rule[-0.200pt]{2.409pt}{0.400pt}}
\put(1429.0,454.0){\rule[-0.200pt]{2.409pt}{0.400pt}}
\put(170.0,455.0){\rule[-0.200pt]{2.409pt}{0.400pt}}
\put(1429.0,455.0){\rule[-0.200pt]{2.409pt}{0.400pt}}
\put(170.0,455.0){\rule[-0.200pt]{2.409pt}{0.400pt}}
\put(1429.0,455.0){\rule[-0.200pt]{2.409pt}{0.400pt}}
\put(170.0,455.0){\rule[-0.200pt]{2.409pt}{0.400pt}}
\put(1429.0,455.0){\rule[-0.200pt]{2.409pt}{0.400pt}}
\put(170.0,455.0){\rule[-0.200pt]{2.409pt}{0.400pt}}
\put(1429.0,455.0){\rule[-0.200pt]{2.409pt}{0.400pt}}
\put(170.0,455.0){\rule[-0.200pt]{2.409pt}{0.400pt}}
\put(1429.0,455.0){\rule[-0.200pt]{2.409pt}{0.400pt}}
\put(170.0,455.0){\rule[-0.200pt]{2.409pt}{0.400pt}}
\put(1429.0,455.0){\rule[-0.200pt]{2.409pt}{0.400pt}}
\put(170.0,455.0){\rule[-0.200pt]{2.409pt}{0.400pt}}
\put(1429.0,455.0){\rule[-0.200pt]{2.409pt}{0.400pt}}
\put(170.0,455.0){\rule[-0.200pt]{2.409pt}{0.400pt}}
\put(1429.0,455.0){\rule[-0.200pt]{2.409pt}{0.400pt}}
\put(170.0,456.0){\rule[-0.200pt]{2.409pt}{0.400pt}}
\put(1429.0,456.0){\rule[-0.200pt]{2.409pt}{0.400pt}}
\put(170.0,456.0){\rule[-0.200pt]{2.409pt}{0.400pt}}
\put(1429.0,456.0){\rule[-0.200pt]{2.409pt}{0.400pt}}
\put(170.0,456.0){\rule[-0.200pt]{2.409pt}{0.400pt}}
\put(1429.0,456.0){\rule[-0.200pt]{2.409pt}{0.400pt}}
\put(170.0,456.0){\rule[-0.200pt]{2.409pt}{0.400pt}}
\put(1429.0,456.0){\rule[-0.200pt]{2.409pt}{0.400pt}}
\put(170.0,456.0){\rule[-0.200pt]{2.409pt}{0.400pt}}
\put(1429.0,456.0){\rule[-0.200pt]{2.409pt}{0.400pt}}
\put(170.0,456.0){\rule[-0.200pt]{2.409pt}{0.400pt}}
\put(1429.0,456.0){\rule[-0.200pt]{2.409pt}{0.400pt}}
\put(170.0,456.0){\rule[-0.200pt]{2.409pt}{0.400pt}}
\put(1429.0,456.0){\rule[-0.200pt]{2.409pt}{0.400pt}}
\put(170.0,456.0){\rule[-0.200pt]{2.409pt}{0.400pt}}
\put(1429.0,456.0){\rule[-0.200pt]{2.409pt}{0.400pt}}
\put(170.0,457.0){\rule[-0.200pt]{2.409pt}{0.400pt}}
\put(1429.0,457.0){\rule[-0.200pt]{2.409pt}{0.400pt}}
\put(170.0,457.0){\rule[-0.200pt]{2.409pt}{0.400pt}}
\put(1429.0,457.0){\rule[-0.200pt]{2.409pt}{0.400pt}}
\put(170.0,457.0){\rule[-0.200pt]{2.409pt}{0.400pt}}
\put(1429.0,457.0){\rule[-0.200pt]{2.409pt}{0.400pt}}
\put(170.0,457.0){\rule[-0.200pt]{2.409pt}{0.400pt}}
\put(1429.0,457.0){\rule[-0.200pt]{2.409pt}{0.400pt}}
\put(170.0,457.0){\rule[-0.200pt]{2.409pt}{0.400pt}}
\put(1429.0,457.0){\rule[-0.200pt]{2.409pt}{0.400pt}}
\put(170.0,457.0){\rule[-0.200pt]{2.409pt}{0.400pt}}
\put(1429.0,457.0){\rule[-0.200pt]{2.409pt}{0.400pt}}
\put(170.0,457.0){\rule[-0.200pt]{2.409pt}{0.400pt}}
\put(1429.0,457.0){\rule[-0.200pt]{2.409pt}{0.400pt}}
\put(170.0,457.0){\rule[-0.200pt]{2.409pt}{0.400pt}}
\put(1429.0,457.0){\rule[-0.200pt]{2.409pt}{0.400pt}}
\put(170.0,458.0){\rule[-0.200pt]{2.409pt}{0.400pt}}
\put(1429.0,458.0){\rule[-0.200pt]{2.409pt}{0.400pt}}
\put(170.0,458.0){\rule[-0.200pt]{2.409pt}{0.400pt}}
\put(1429.0,458.0){\rule[-0.200pt]{2.409pt}{0.400pt}}
\put(170.0,458.0){\rule[-0.200pt]{2.409pt}{0.400pt}}
\put(1429.0,458.0){\rule[-0.200pt]{2.409pt}{0.400pt}}
\put(170.0,458.0){\rule[-0.200pt]{2.409pt}{0.400pt}}
\put(1429.0,458.0){\rule[-0.200pt]{2.409pt}{0.400pt}}
\put(170.0,458.0){\rule[-0.200pt]{2.409pt}{0.400pt}}
\put(1429.0,458.0){\rule[-0.200pt]{2.409pt}{0.400pt}}
\put(170.0,458.0){\rule[-0.200pt]{2.409pt}{0.400pt}}
\put(1429.0,458.0){\rule[-0.200pt]{2.409pt}{0.400pt}}
\put(170.0,458.0){\rule[-0.200pt]{2.409pt}{0.400pt}}
\put(1429.0,458.0){\rule[-0.200pt]{2.409pt}{0.400pt}}
\put(170.0,458.0){\rule[-0.200pt]{2.409pt}{0.400pt}}
\put(1429.0,458.0){\rule[-0.200pt]{2.409pt}{0.400pt}}
\put(170.0,458.0){\rule[-0.200pt]{2.409pt}{0.400pt}}
\put(1429.0,458.0){\rule[-0.200pt]{2.409pt}{0.400pt}}
\put(170.0,458.0){\rule[-0.200pt]{2.409pt}{0.400pt}}
\put(1429.0,458.0){\rule[-0.200pt]{2.409pt}{0.400pt}}
\put(170.0,459.0){\rule[-0.200pt]{2.409pt}{0.400pt}}
\put(1429.0,459.0){\rule[-0.200pt]{2.409pt}{0.400pt}}
\put(170.0,459.0){\rule[-0.200pt]{2.409pt}{0.400pt}}
\put(1429.0,459.0){\rule[-0.200pt]{2.409pt}{0.400pt}}
\put(170.0,459.0){\rule[-0.200pt]{2.409pt}{0.400pt}}
\put(1429.0,459.0){\rule[-0.200pt]{2.409pt}{0.400pt}}
\put(170.0,459.0){\rule[-0.200pt]{2.409pt}{0.400pt}}
\put(1429.0,459.0){\rule[-0.200pt]{2.409pt}{0.400pt}}
\put(170.0,459.0){\rule[-0.200pt]{2.409pt}{0.400pt}}
\put(1429.0,459.0){\rule[-0.200pt]{2.409pt}{0.400pt}}
\put(170.0,459.0){\rule[-0.200pt]{2.409pt}{0.400pt}}
\put(1429.0,459.0){\rule[-0.200pt]{2.409pt}{0.400pt}}
\put(170.0,459.0){\rule[-0.200pt]{2.409pt}{0.400pt}}
\put(1429.0,459.0){\rule[-0.200pt]{2.409pt}{0.400pt}}
\put(170.0,459.0){\rule[-0.200pt]{2.409pt}{0.400pt}}
\put(1429.0,459.0){\rule[-0.200pt]{2.409pt}{0.400pt}}
\put(170.0,459.0){\rule[-0.200pt]{2.409pt}{0.400pt}}
\put(1429.0,459.0){\rule[-0.200pt]{2.409pt}{0.400pt}}
\put(170.0,460.0){\rule[-0.200pt]{2.409pt}{0.400pt}}
\put(1429.0,460.0){\rule[-0.200pt]{2.409pt}{0.400pt}}
\put(170.0,460.0){\rule[-0.200pt]{2.409pt}{0.400pt}}
\put(1429.0,460.0){\rule[-0.200pt]{2.409pt}{0.400pt}}
\put(170.0,460.0){\rule[-0.200pt]{2.409pt}{0.400pt}}
\put(1429.0,460.0){\rule[-0.200pt]{2.409pt}{0.400pt}}
\put(170.0,460.0){\rule[-0.200pt]{2.409pt}{0.400pt}}
\put(1429.0,460.0){\rule[-0.200pt]{2.409pt}{0.400pt}}
\put(170.0,460.0){\rule[-0.200pt]{2.409pt}{0.400pt}}
\put(1429.0,460.0){\rule[-0.200pt]{2.409pt}{0.400pt}}
\put(170.0,460.0){\rule[-0.200pt]{2.409pt}{0.400pt}}
\put(1429.0,460.0){\rule[-0.200pt]{2.409pt}{0.400pt}}
\put(170.0,460.0){\rule[-0.200pt]{2.409pt}{0.400pt}}
\put(1429.0,460.0){\rule[-0.200pt]{2.409pt}{0.400pt}}
\put(170.0,460.0){\rule[-0.200pt]{2.409pt}{0.400pt}}
\put(1429.0,460.0){\rule[-0.200pt]{2.409pt}{0.400pt}}
\put(170.0,460.0){\rule[-0.200pt]{2.409pt}{0.400pt}}
\put(1429.0,460.0){\rule[-0.200pt]{2.409pt}{0.400pt}}
\put(170.0,460.0){\rule[-0.200pt]{2.409pt}{0.400pt}}
\put(1429.0,460.0){\rule[-0.200pt]{2.409pt}{0.400pt}}
\put(170.0,461.0){\rule[-0.200pt]{2.409pt}{0.400pt}}
\put(1429.0,461.0){\rule[-0.200pt]{2.409pt}{0.400pt}}
\put(170.0,461.0){\rule[-0.200pt]{2.409pt}{0.400pt}}
\put(1429.0,461.0){\rule[-0.200pt]{2.409pt}{0.400pt}}
\put(170.0,461.0){\rule[-0.200pt]{2.409pt}{0.400pt}}
\put(1429.0,461.0){\rule[-0.200pt]{2.409pt}{0.400pt}}
\put(170.0,461.0){\rule[-0.200pt]{2.409pt}{0.400pt}}
\put(1429.0,461.0){\rule[-0.200pt]{2.409pt}{0.400pt}}
\put(170.0,461.0){\rule[-0.200pt]{2.409pt}{0.400pt}}
\put(1429.0,461.0){\rule[-0.200pt]{2.409pt}{0.400pt}}
\put(170.0,461.0){\rule[-0.200pt]{2.409pt}{0.400pt}}
\put(1429.0,461.0){\rule[-0.200pt]{2.409pt}{0.400pt}}
\put(170.0,461.0){\rule[-0.200pt]{2.409pt}{0.400pt}}
\put(1429.0,461.0){\rule[-0.200pt]{2.409pt}{0.400pt}}
\put(170.0,461.0){\rule[-0.200pt]{2.409pt}{0.400pt}}
\put(1429.0,461.0){\rule[-0.200pt]{2.409pt}{0.400pt}}
\put(170.0,461.0){\rule[-0.200pt]{2.409pt}{0.400pt}}
\put(1429.0,461.0){\rule[-0.200pt]{2.409pt}{0.400pt}}
\put(170.0,461.0){\rule[-0.200pt]{2.409pt}{0.400pt}}
\put(1429.0,461.0){\rule[-0.200pt]{2.409pt}{0.400pt}}
\put(170.0,461.0){\rule[-0.200pt]{2.409pt}{0.400pt}}
\put(1429.0,461.0){\rule[-0.200pt]{2.409pt}{0.400pt}}
\put(170.0,462.0){\rule[-0.200pt]{2.409pt}{0.400pt}}
\put(1429.0,462.0){\rule[-0.200pt]{2.409pt}{0.400pt}}
\put(170.0,462.0){\rule[-0.200pt]{2.409pt}{0.400pt}}
\put(1429.0,462.0){\rule[-0.200pt]{2.409pt}{0.400pt}}
\put(170.0,462.0){\rule[-0.200pt]{2.409pt}{0.400pt}}
\put(1429.0,462.0){\rule[-0.200pt]{2.409pt}{0.400pt}}
\put(170.0,462.0){\rule[-0.200pt]{2.409pt}{0.400pt}}
\put(1429.0,462.0){\rule[-0.200pt]{2.409pt}{0.400pt}}
\put(170.0,462.0){\rule[-0.200pt]{2.409pt}{0.400pt}}
\put(1429.0,462.0){\rule[-0.200pt]{2.409pt}{0.400pt}}
\put(170.0,462.0){\rule[-0.200pt]{2.409pt}{0.400pt}}
\put(1429.0,462.0){\rule[-0.200pt]{2.409pt}{0.400pt}}
\put(170.0,462.0){\rule[-0.200pt]{2.409pt}{0.400pt}}
\put(1429.0,462.0){\rule[-0.200pt]{2.409pt}{0.400pt}}
\put(170.0,462.0){\rule[-0.200pt]{2.409pt}{0.400pt}}
\put(1429.0,462.0){\rule[-0.200pt]{2.409pt}{0.400pt}}
\put(170.0,462.0){\rule[-0.200pt]{2.409pt}{0.400pt}}
\put(1429.0,462.0){\rule[-0.200pt]{2.409pt}{0.400pt}}
\put(170.0,462.0){\rule[-0.200pt]{2.409pt}{0.400pt}}
\put(1429.0,462.0){\rule[-0.200pt]{2.409pt}{0.400pt}}
\put(170.0,462.0){\rule[-0.200pt]{2.409pt}{0.400pt}}
\put(1429.0,462.0){\rule[-0.200pt]{2.409pt}{0.400pt}}
\put(170.0,463.0){\rule[-0.200pt]{2.409pt}{0.400pt}}
\put(1429.0,463.0){\rule[-0.200pt]{2.409pt}{0.400pt}}
\put(170.0,463.0){\rule[-0.200pt]{2.409pt}{0.400pt}}
\put(1429.0,463.0){\rule[-0.200pt]{2.409pt}{0.400pt}}
\put(170.0,463.0){\rule[-0.200pt]{2.409pt}{0.400pt}}
\put(1429.0,463.0){\rule[-0.200pt]{2.409pt}{0.400pt}}
\put(170.0,463.0){\rule[-0.200pt]{2.409pt}{0.400pt}}
\put(1429.0,463.0){\rule[-0.200pt]{2.409pt}{0.400pt}}
\put(170.0,463.0){\rule[-0.200pt]{2.409pt}{0.400pt}}
\put(1429.0,463.0){\rule[-0.200pt]{2.409pt}{0.400pt}}
\put(170.0,463.0){\rule[-0.200pt]{2.409pt}{0.400pt}}
\put(1429.0,463.0){\rule[-0.200pt]{2.409pt}{0.400pt}}
\put(170.0,463.0){\rule[-0.200pt]{2.409pt}{0.400pt}}
\put(1429.0,463.0){\rule[-0.200pt]{2.409pt}{0.400pt}}
\put(170.0,463.0){\rule[-0.200pt]{2.409pt}{0.400pt}}
\put(1429.0,463.0){\rule[-0.200pt]{2.409pt}{0.400pt}}
\put(170.0,463.0){\rule[-0.200pt]{2.409pt}{0.400pt}}
\put(1429.0,463.0){\rule[-0.200pt]{2.409pt}{0.400pt}}
\put(170.0,463.0){\rule[-0.200pt]{2.409pt}{0.400pt}}
\put(1429.0,463.0){\rule[-0.200pt]{2.409pt}{0.400pt}}
\put(170.0,463.0){\rule[-0.200pt]{2.409pt}{0.400pt}}
\put(1429.0,463.0){\rule[-0.200pt]{2.409pt}{0.400pt}}
\put(170.0,463.0){\rule[-0.200pt]{2.409pt}{0.400pt}}
\put(1429.0,463.0){\rule[-0.200pt]{2.409pt}{0.400pt}}
\put(170.0,464.0){\rule[-0.200pt]{2.409pt}{0.400pt}}
\put(1429.0,464.0){\rule[-0.200pt]{2.409pt}{0.400pt}}
\put(170.0,464.0){\rule[-0.200pt]{2.409pt}{0.400pt}}
\put(1429.0,464.0){\rule[-0.200pt]{2.409pt}{0.400pt}}
\put(170.0,464.0){\rule[-0.200pt]{2.409pt}{0.400pt}}
\put(1429.0,464.0){\rule[-0.200pt]{2.409pt}{0.400pt}}
\put(170.0,464.0){\rule[-0.200pt]{2.409pt}{0.400pt}}
\put(1429.0,464.0){\rule[-0.200pt]{2.409pt}{0.400pt}}
\put(170.0,464.0){\rule[-0.200pt]{2.409pt}{0.400pt}}
\put(1429.0,464.0){\rule[-0.200pt]{2.409pt}{0.400pt}}
\put(170.0,464.0){\rule[-0.200pt]{2.409pt}{0.400pt}}
\put(1429.0,464.0){\rule[-0.200pt]{2.409pt}{0.400pt}}
\put(170.0,464.0){\rule[-0.200pt]{2.409pt}{0.400pt}}
\put(1429.0,464.0){\rule[-0.200pt]{2.409pt}{0.400pt}}
\put(170.0,464.0){\rule[-0.200pt]{2.409pt}{0.400pt}}
\put(1429.0,464.0){\rule[-0.200pt]{2.409pt}{0.400pt}}
\put(170.0,464.0){\rule[-0.200pt]{2.409pt}{0.400pt}}
\put(1429.0,464.0){\rule[-0.200pt]{2.409pt}{0.400pt}}
\put(170.0,464.0){\rule[-0.200pt]{2.409pt}{0.400pt}}
\put(1429.0,464.0){\rule[-0.200pt]{2.409pt}{0.400pt}}
\put(170.0,464.0){\rule[-0.200pt]{2.409pt}{0.400pt}}
\put(1429.0,464.0){\rule[-0.200pt]{2.409pt}{0.400pt}}
\put(170.0,464.0){\rule[-0.200pt]{2.409pt}{0.400pt}}
\put(1429.0,464.0){\rule[-0.200pt]{2.409pt}{0.400pt}}
\put(170.0,464.0){\rule[-0.200pt]{2.409pt}{0.400pt}}
\put(1429.0,464.0){\rule[-0.200pt]{2.409pt}{0.400pt}}
\put(170.0,465.0){\rule[-0.200pt]{2.409pt}{0.400pt}}
\put(1429.0,465.0){\rule[-0.200pt]{2.409pt}{0.400pt}}
\put(170.0,465.0){\rule[-0.200pt]{2.409pt}{0.400pt}}
\put(1429.0,465.0){\rule[-0.200pt]{2.409pt}{0.400pt}}
\put(170.0,465.0){\rule[-0.200pt]{2.409pt}{0.400pt}}
\put(1429.0,465.0){\rule[-0.200pt]{2.409pt}{0.400pt}}
\put(170.0,465.0){\rule[-0.200pt]{2.409pt}{0.400pt}}
\put(1429.0,465.0){\rule[-0.200pt]{2.409pt}{0.400pt}}
\put(170.0,465.0){\rule[-0.200pt]{2.409pt}{0.400pt}}
\put(1429.0,465.0){\rule[-0.200pt]{2.409pt}{0.400pt}}
\put(170.0,465.0){\rule[-0.200pt]{2.409pt}{0.400pt}}
\put(1429.0,465.0){\rule[-0.200pt]{2.409pt}{0.400pt}}
\put(170.0,465.0){\rule[-0.200pt]{2.409pt}{0.400pt}}
\put(1429.0,465.0){\rule[-0.200pt]{2.409pt}{0.400pt}}
\put(170.0,465.0){\rule[-0.200pt]{2.409pt}{0.400pt}}
\put(1429.0,465.0){\rule[-0.200pt]{2.409pt}{0.400pt}}
\put(170.0,465.0){\rule[-0.200pt]{2.409pt}{0.400pt}}
\put(1429.0,465.0){\rule[-0.200pt]{2.409pt}{0.400pt}}
\put(170.0,465.0){\rule[-0.200pt]{2.409pt}{0.400pt}}
\put(1429.0,465.0){\rule[-0.200pt]{2.409pt}{0.400pt}}
\put(170.0,465.0){\rule[-0.200pt]{2.409pt}{0.400pt}}
\put(1429.0,465.0){\rule[-0.200pt]{2.409pt}{0.400pt}}
\put(170.0,465.0){\rule[-0.200pt]{2.409pt}{0.400pt}}
\put(1429.0,465.0){\rule[-0.200pt]{2.409pt}{0.400pt}}
\put(170.0,465.0){\rule[-0.200pt]{2.409pt}{0.400pt}}
\put(1429.0,465.0){\rule[-0.200pt]{2.409pt}{0.400pt}}
\put(170.0,466.0){\rule[-0.200pt]{2.409pt}{0.400pt}}
\put(1429.0,466.0){\rule[-0.200pt]{2.409pt}{0.400pt}}
\put(170.0,466.0){\rule[-0.200pt]{2.409pt}{0.400pt}}
\put(1429.0,466.0){\rule[-0.200pt]{2.409pt}{0.400pt}}
\put(170.0,466.0){\rule[-0.200pt]{2.409pt}{0.400pt}}
\put(1429.0,466.0){\rule[-0.200pt]{2.409pt}{0.400pt}}
\put(170.0,466.0){\rule[-0.200pt]{2.409pt}{0.400pt}}
\put(1429.0,466.0){\rule[-0.200pt]{2.409pt}{0.400pt}}
\put(170.0,466.0){\rule[-0.200pt]{2.409pt}{0.400pt}}
\put(1429.0,466.0){\rule[-0.200pt]{2.409pt}{0.400pt}}
\put(170.0,466.0){\rule[-0.200pt]{2.409pt}{0.400pt}}
\put(1429.0,466.0){\rule[-0.200pt]{2.409pt}{0.400pt}}
\put(170.0,466.0){\rule[-0.200pt]{2.409pt}{0.400pt}}
\put(1429.0,466.0){\rule[-0.200pt]{2.409pt}{0.400pt}}
\put(170.0,466.0){\rule[-0.200pt]{2.409pt}{0.400pt}}
\put(1429.0,466.0){\rule[-0.200pt]{2.409pt}{0.400pt}}
\put(170.0,466.0){\rule[-0.200pt]{2.409pt}{0.400pt}}
\put(1429.0,466.0){\rule[-0.200pt]{2.409pt}{0.400pt}}
\put(170.0,466.0){\rule[-0.200pt]{2.409pt}{0.400pt}}
\put(1429.0,466.0){\rule[-0.200pt]{2.409pt}{0.400pt}}
\put(170.0,466.0){\rule[-0.200pt]{2.409pt}{0.400pt}}
\put(1429.0,466.0){\rule[-0.200pt]{2.409pt}{0.400pt}}
\put(170.0,466.0){\rule[-0.200pt]{2.409pt}{0.400pt}}
\put(1429.0,466.0){\rule[-0.200pt]{2.409pt}{0.400pt}}
\put(170.0,466.0){\rule[-0.200pt]{2.409pt}{0.400pt}}
\put(1429.0,466.0){\rule[-0.200pt]{2.409pt}{0.400pt}}
\put(170.0,466.0){\rule[-0.200pt]{2.409pt}{0.400pt}}
\put(1429.0,466.0){\rule[-0.200pt]{2.409pt}{0.400pt}}
\put(170.0,467.0){\rule[-0.200pt]{2.409pt}{0.400pt}}
\put(1429.0,467.0){\rule[-0.200pt]{2.409pt}{0.400pt}}
\put(170.0,467.0){\rule[-0.200pt]{2.409pt}{0.400pt}}
\put(1429.0,467.0){\rule[-0.200pt]{2.409pt}{0.400pt}}
\put(170.0,467.0){\rule[-0.200pt]{2.409pt}{0.400pt}}
\put(1429.0,467.0){\rule[-0.200pt]{2.409pt}{0.400pt}}
\put(170.0,467.0){\rule[-0.200pt]{2.409pt}{0.400pt}}
\put(1429.0,467.0){\rule[-0.200pt]{2.409pt}{0.400pt}}
\put(170.0,467.0){\rule[-0.200pt]{2.409pt}{0.400pt}}
\put(1429.0,467.0){\rule[-0.200pt]{2.409pt}{0.400pt}}
\put(170.0,467.0){\rule[-0.200pt]{2.409pt}{0.400pt}}
\put(1429.0,467.0){\rule[-0.200pt]{2.409pt}{0.400pt}}
\put(170.0,467.0){\rule[-0.200pt]{2.409pt}{0.400pt}}
\put(1429.0,467.0){\rule[-0.200pt]{2.409pt}{0.400pt}}
\put(170.0,467.0){\rule[-0.200pt]{2.409pt}{0.400pt}}
\put(1429.0,467.0){\rule[-0.200pt]{2.409pt}{0.400pt}}
\put(170.0,467.0){\rule[-0.200pt]{2.409pt}{0.400pt}}
\put(1429.0,467.0){\rule[-0.200pt]{2.409pt}{0.400pt}}
\put(170.0,467.0){\rule[-0.200pt]{2.409pt}{0.400pt}}
\put(1429.0,467.0){\rule[-0.200pt]{2.409pt}{0.400pt}}
\put(170.0,467.0){\rule[-0.200pt]{2.409pt}{0.400pt}}
\put(1429.0,467.0){\rule[-0.200pt]{2.409pt}{0.400pt}}
\put(170.0,467.0){\rule[-0.200pt]{2.409pt}{0.400pt}}
\put(1429.0,467.0){\rule[-0.200pt]{2.409pt}{0.400pt}}
\put(170.0,467.0){\rule[-0.200pt]{2.409pt}{0.400pt}}
\put(1429.0,467.0){\rule[-0.200pt]{2.409pt}{0.400pt}}
\put(170.0,467.0){\rule[-0.200pt]{2.409pt}{0.400pt}}
\put(1429.0,467.0){\rule[-0.200pt]{2.409pt}{0.400pt}}
\put(170.0,467.0){\rule[-0.200pt]{2.409pt}{0.400pt}}
\put(1429.0,467.0){\rule[-0.200pt]{2.409pt}{0.400pt}}
\put(170.0,468.0){\rule[-0.200pt]{2.409pt}{0.400pt}}
\put(1429.0,468.0){\rule[-0.200pt]{2.409pt}{0.400pt}}
\put(170.0,468.0){\rule[-0.200pt]{2.409pt}{0.400pt}}
\put(1429.0,468.0){\rule[-0.200pt]{2.409pt}{0.400pt}}
\put(170.0,468.0){\rule[-0.200pt]{2.409pt}{0.400pt}}
\put(1429.0,468.0){\rule[-0.200pt]{2.409pt}{0.400pt}}
\put(170.0,468.0){\rule[-0.200pt]{2.409pt}{0.400pt}}
\put(1429.0,468.0){\rule[-0.200pt]{2.409pt}{0.400pt}}
\put(170.0,468.0){\rule[-0.200pt]{2.409pt}{0.400pt}}
\put(1429.0,468.0){\rule[-0.200pt]{2.409pt}{0.400pt}}
\put(170.0,468.0){\rule[-0.200pt]{2.409pt}{0.400pt}}
\put(1429.0,468.0){\rule[-0.200pt]{2.409pt}{0.400pt}}
\put(170.0,468.0){\rule[-0.200pt]{2.409pt}{0.400pt}}
\put(1429.0,468.0){\rule[-0.200pt]{2.409pt}{0.400pt}}
\put(170.0,468.0){\rule[-0.200pt]{2.409pt}{0.400pt}}
\put(1429.0,468.0){\rule[-0.200pt]{2.409pt}{0.400pt}}
\put(170.0,468.0){\rule[-0.200pt]{2.409pt}{0.400pt}}
\put(1429.0,468.0){\rule[-0.200pt]{2.409pt}{0.400pt}}
\put(170.0,468.0){\rule[-0.200pt]{2.409pt}{0.400pt}}
\put(1429.0,468.0){\rule[-0.200pt]{2.409pt}{0.400pt}}
\put(170.0,468.0){\rule[-0.200pt]{2.409pt}{0.400pt}}
\put(1429.0,468.0){\rule[-0.200pt]{2.409pt}{0.400pt}}
\put(170.0,468.0){\rule[-0.200pt]{2.409pt}{0.400pt}}
\put(1429.0,468.0){\rule[-0.200pt]{2.409pt}{0.400pt}}
\put(170.0,468.0){\rule[-0.200pt]{2.409pt}{0.400pt}}
\put(1429.0,468.0){\rule[-0.200pt]{2.409pt}{0.400pt}}
\put(170.0,468.0){\rule[-0.200pt]{2.409pt}{0.400pt}}
\put(1429.0,468.0){\rule[-0.200pt]{2.409pt}{0.400pt}}
\put(170.0,468.0){\rule[-0.200pt]{2.409pt}{0.400pt}}
\put(1429.0,468.0){\rule[-0.200pt]{2.409pt}{0.400pt}}
\put(170.0,469.0){\rule[-0.200pt]{2.409pt}{0.400pt}}
\put(1429.0,469.0){\rule[-0.200pt]{2.409pt}{0.400pt}}
\put(170.0,469.0){\rule[-0.200pt]{2.409pt}{0.400pt}}
\put(1429.0,469.0){\rule[-0.200pt]{2.409pt}{0.400pt}}
\put(170.0,469.0){\rule[-0.200pt]{2.409pt}{0.400pt}}
\put(1429.0,469.0){\rule[-0.200pt]{2.409pt}{0.400pt}}
\put(170.0,469.0){\rule[-0.200pt]{2.409pt}{0.400pt}}
\put(1429.0,469.0){\rule[-0.200pt]{2.409pt}{0.400pt}}
\put(170.0,469.0){\rule[-0.200pt]{2.409pt}{0.400pt}}
\put(1429.0,469.0){\rule[-0.200pt]{2.409pt}{0.400pt}}
\put(170.0,469.0){\rule[-0.200pt]{2.409pt}{0.400pt}}
\put(1429.0,469.0){\rule[-0.200pt]{2.409pt}{0.400pt}}
\put(170.0,469.0){\rule[-0.200pt]{2.409pt}{0.400pt}}
\put(1429.0,469.0){\rule[-0.200pt]{2.409pt}{0.400pt}}
\put(170.0,469.0){\rule[-0.200pt]{2.409pt}{0.400pt}}
\put(1429.0,469.0){\rule[-0.200pt]{2.409pt}{0.400pt}}
\put(170.0,469.0){\rule[-0.200pt]{2.409pt}{0.400pt}}
\put(1429.0,469.0){\rule[-0.200pt]{2.409pt}{0.400pt}}
\put(170.0,469.0){\rule[-0.200pt]{2.409pt}{0.400pt}}
\put(1429.0,469.0){\rule[-0.200pt]{2.409pt}{0.400pt}}
\put(170.0,469.0){\rule[-0.200pt]{2.409pt}{0.400pt}}
\put(1429.0,469.0){\rule[-0.200pt]{2.409pt}{0.400pt}}
\put(170.0,469.0){\rule[-0.200pt]{2.409pt}{0.400pt}}
\put(1429.0,469.0){\rule[-0.200pt]{2.409pt}{0.400pt}}
\put(170.0,469.0){\rule[-0.200pt]{2.409pt}{0.400pt}}
\put(1429.0,469.0){\rule[-0.200pt]{2.409pt}{0.400pt}}
\put(170.0,469.0){\rule[-0.200pt]{2.409pt}{0.400pt}}
\put(1429.0,469.0){\rule[-0.200pt]{2.409pt}{0.400pt}}
\put(170.0,469.0){\rule[-0.200pt]{2.409pt}{0.400pt}}
\put(1429.0,469.0){\rule[-0.200pt]{2.409pt}{0.400pt}}
\put(170.0,469.0){\rule[-0.200pt]{2.409pt}{0.400pt}}
\put(1429.0,469.0){\rule[-0.200pt]{2.409pt}{0.400pt}}
\put(170.0,469.0){\rule[-0.200pt]{2.409pt}{0.400pt}}
\put(1429.0,469.0){\rule[-0.200pt]{2.409pt}{0.400pt}}
\put(170.0,470.0){\rule[-0.200pt]{2.409pt}{0.400pt}}
\put(1429.0,470.0){\rule[-0.200pt]{2.409pt}{0.400pt}}
\put(170.0,470.0){\rule[-0.200pt]{2.409pt}{0.400pt}}
\put(1429.0,470.0){\rule[-0.200pt]{2.409pt}{0.400pt}}
\put(170.0,470.0){\rule[-0.200pt]{2.409pt}{0.400pt}}
\put(1429.0,470.0){\rule[-0.200pt]{2.409pt}{0.400pt}}
\put(170.0,470.0){\rule[-0.200pt]{2.409pt}{0.400pt}}
\put(1429.0,470.0){\rule[-0.200pt]{2.409pt}{0.400pt}}
\put(170.0,470.0){\rule[-0.200pt]{2.409pt}{0.400pt}}
\put(1429.0,470.0){\rule[-0.200pt]{2.409pt}{0.400pt}}
\put(170.0,470.0){\rule[-0.200pt]{2.409pt}{0.400pt}}
\put(1429.0,470.0){\rule[-0.200pt]{2.409pt}{0.400pt}}
\put(170.0,470.0){\rule[-0.200pt]{2.409pt}{0.400pt}}
\put(1429.0,470.0){\rule[-0.200pt]{2.409pt}{0.400pt}}
\put(170.0,470.0){\rule[-0.200pt]{2.409pt}{0.400pt}}
\put(1429.0,470.0){\rule[-0.200pt]{2.409pt}{0.400pt}}
\put(170.0,470.0){\rule[-0.200pt]{2.409pt}{0.400pt}}
\put(1429.0,470.0){\rule[-0.200pt]{2.409pt}{0.400pt}}
\put(170.0,470.0){\rule[-0.200pt]{2.409pt}{0.400pt}}
\put(1429.0,470.0){\rule[-0.200pt]{2.409pt}{0.400pt}}
\put(170.0,470.0){\rule[-0.200pt]{2.409pt}{0.400pt}}
\put(1429.0,470.0){\rule[-0.200pt]{2.409pt}{0.400pt}}
\put(170.0,470.0){\rule[-0.200pt]{2.409pt}{0.400pt}}
\put(1429.0,470.0){\rule[-0.200pt]{2.409pt}{0.400pt}}
\put(170.0,470.0){\rule[-0.200pt]{2.409pt}{0.400pt}}
\put(1429.0,470.0){\rule[-0.200pt]{2.409pt}{0.400pt}}
\put(170.0,470.0){\rule[-0.200pt]{2.409pt}{0.400pt}}
\put(1429.0,470.0){\rule[-0.200pt]{2.409pt}{0.400pt}}
\put(170.0,470.0){\rule[-0.200pt]{2.409pt}{0.400pt}}
\put(1429.0,470.0){\rule[-0.200pt]{2.409pt}{0.400pt}}
\put(170.0,470.0){\rule[-0.200pt]{2.409pt}{0.400pt}}
\put(1429.0,470.0){\rule[-0.200pt]{2.409pt}{0.400pt}}
\put(170.0,470.0){\rule[-0.200pt]{2.409pt}{0.400pt}}
\put(1429.0,470.0){\rule[-0.200pt]{2.409pt}{0.400pt}}
\put(170.0,471.0){\rule[-0.200pt]{4.818pt}{0.400pt}}
\put(150,471){\makebox(0,0)[r]{ 1e-06}}
\put(1419.0,471.0){\rule[-0.200pt]{4.818pt}{0.400pt}}
\put(170.0,483.0){\rule[-0.200pt]{2.409pt}{0.400pt}}
\put(1429.0,483.0){\rule[-0.200pt]{2.409pt}{0.400pt}}
\put(170.0,501.0){\rule[-0.200pt]{2.409pt}{0.400pt}}
\put(1429.0,501.0){\rule[-0.200pt]{2.409pt}{0.400pt}}
\put(170.0,509.0){\rule[-0.200pt]{2.409pt}{0.400pt}}
\put(1429.0,509.0){\rule[-0.200pt]{2.409pt}{0.400pt}}
\put(170.0,515.0){\rule[-0.200pt]{2.409pt}{0.400pt}}
\put(1429.0,515.0){\rule[-0.200pt]{2.409pt}{0.400pt}}
\put(170.0,520.0){\rule[-0.200pt]{2.409pt}{0.400pt}}
\put(1429.0,520.0){\rule[-0.200pt]{2.409pt}{0.400pt}}
\put(170.0,524.0){\rule[-0.200pt]{2.409pt}{0.400pt}}
\put(1429.0,524.0){\rule[-0.200pt]{2.409pt}{0.400pt}}
\put(170.0,527.0){\rule[-0.200pt]{2.409pt}{0.400pt}}
\put(1429.0,527.0){\rule[-0.200pt]{2.409pt}{0.400pt}}
\put(170.0,529.0){\rule[-0.200pt]{2.409pt}{0.400pt}}
\put(1429.0,529.0){\rule[-0.200pt]{2.409pt}{0.400pt}}
\put(170.0,532.0){\rule[-0.200pt]{2.409pt}{0.400pt}}
\put(1429.0,532.0){\rule[-0.200pt]{2.409pt}{0.400pt}}
\put(170.0,534.0){\rule[-0.200pt]{2.409pt}{0.400pt}}
\put(1429.0,534.0){\rule[-0.200pt]{2.409pt}{0.400pt}}
\put(170.0,535.0){\rule[-0.200pt]{2.409pt}{0.400pt}}
\put(1429.0,535.0){\rule[-0.200pt]{2.409pt}{0.400pt}}
\put(170.0,537.0){\rule[-0.200pt]{2.409pt}{0.400pt}}
\put(1429.0,537.0){\rule[-0.200pt]{2.409pt}{0.400pt}}
\put(170.0,539.0){\rule[-0.200pt]{2.409pt}{0.400pt}}
\put(1429.0,539.0){\rule[-0.200pt]{2.409pt}{0.400pt}}
\put(170.0,540.0){\rule[-0.200pt]{2.409pt}{0.400pt}}
\put(1429.0,540.0){\rule[-0.200pt]{2.409pt}{0.400pt}}
\put(170.0,541.0){\rule[-0.200pt]{2.409pt}{0.400pt}}
\put(1429.0,541.0){\rule[-0.200pt]{2.409pt}{0.400pt}}
\put(170.0,543.0){\rule[-0.200pt]{2.409pt}{0.400pt}}
\put(1429.0,543.0){\rule[-0.200pt]{2.409pt}{0.400pt}}
\put(170.0,544.0){\rule[-0.200pt]{2.409pt}{0.400pt}}
\put(1429.0,544.0){\rule[-0.200pt]{2.409pt}{0.400pt}}
\put(170.0,545.0){\rule[-0.200pt]{2.409pt}{0.400pt}}
\put(1429.0,545.0){\rule[-0.200pt]{2.409pt}{0.400pt}}
\put(170.0,546.0){\rule[-0.200pt]{2.409pt}{0.400pt}}
\put(1429.0,546.0){\rule[-0.200pt]{2.409pt}{0.400pt}}
\put(170.0,547.0){\rule[-0.200pt]{2.409pt}{0.400pt}}
\put(1429.0,547.0){\rule[-0.200pt]{2.409pt}{0.400pt}}
\put(170.0,548.0){\rule[-0.200pt]{2.409pt}{0.400pt}}
\put(1429.0,548.0){\rule[-0.200pt]{2.409pt}{0.400pt}}
\put(170.0,549.0){\rule[-0.200pt]{2.409pt}{0.400pt}}
\put(1429.0,549.0){\rule[-0.200pt]{2.409pt}{0.400pt}}
\put(170.0,550.0){\rule[-0.200pt]{2.409pt}{0.400pt}}
\put(1429.0,550.0){\rule[-0.200pt]{2.409pt}{0.400pt}}
\put(170.0,550.0){\rule[-0.200pt]{2.409pt}{0.400pt}}
\put(1429.0,550.0){\rule[-0.200pt]{2.409pt}{0.400pt}}
\put(170.0,551.0){\rule[-0.200pt]{2.409pt}{0.400pt}}
\put(1429.0,551.0){\rule[-0.200pt]{2.409pt}{0.400pt}}
\put(170.0,552.0){\rule[-0.200pt]{2.409pt}{0.400pt}}
\put(1429.0,552.0){\rule[-0.200pt]{2.409pt}{0.400pt}}
\put(170.0,553.0){\rule[-0.200pt]{2.409pt}{0.400pt}}
\put(1429.0,553.0){\rule[-0.200pt]{2.409pt}{0.400pt}}
\put(170.0,553.0){\rule[-0.200pt]{2.409pt}{0.400pt}}
\put(1429.0,553.0){\rule[-0.200pt]{2.409pt}{0.400pt}}
\put(170.0,554.0){\rule[-0.200pt]{2.409pt}{0.400pt}}
\put(1429.0,554.0){\rule[-0.200pt]{2.409pt}{0.400pt}}
\put(170.0,555.0){\rule[-0.200pt]{2.409pt}{0.400pt}}
\put(1429.0,555.0){\rule[-0.200pt]{2.409pt}{0.400pt}}
\put(170.0,555.0){\rule[-0.200pt]{2.409pt}{0.400pt}}
\put(1429.0,555.0){\rule[-0.200pt]{2.409pt}{0.400pt}}
\put(170.0,556.0){\rule[-0.200pt]{2.409pt}{0.400pt}}
\put(1429.0,556.0){\rule[-0.200pt]{2.409pt}{0.400pt}}
\put(170.0,556.0){\rule[-0.200pt]{2.409pt}{0.400pt}}
\put(1429.0,556.0){\rule[-0.200pt]{2.409pt}{0.400pt}}
\put(170.0,557.0){\rule[-0.200pt]{2.409pt}{0.400pt}}
\put(1429.0,557.0){\rule[-0.200pt]{2.409pt}{0.400pt}}
\put(170.0,558.0){\rule[-0.200pt]{2.409pt}{0.400pt}}
\put(1429.0,558.0){\rule[-0.200pt]{2.409pt}{0.400pt}}
\put(170.0,558.0){\rule[-0.200pt]{2.409pt}{0.400pt}}
\put(1429.0,558.0){\rule[-0.200pt]{2.409pt}{0.400pt}}
\put(170.0,559.0){\rule[-0.200pt]{2.409pt}{0.400pt}}
\put(1429.0,559.0){\rule[-0.200pt]{2.409pt}{0.400pt}}
\put(170.0,559.0){\rule[-0.200pt]{2.409pt}{0.400pt}}
\put(1429.0,559.0){\rule[-0.200pt]{2.409pt}{0.400pt}}
\put(170.0,560.0){\rule[-0.200pt]{2.409pt}{0.400pt}}
\put(1429.0,560.0){\rule[-0.200pt]{2.409pt}{0.400pt}}
\put(170.0,560.0){\rule[-0.200pt]{2.409pt}{0.400pt}}
\put(1429.0,560.0){\rule[-0.200pt]{2.409pt}{0.400pt}}
\put(170.0,561.0){\rule[-0.200pt]{2.409pt}{0.400pt}}
\put(1429.0,561.0){\rule[-0.200pt]{2.409pt}{0.400pt}}
\put(170.0,561.0){\rule[-0.200pt]{2.409pt}{0.400pt}}
\put(1429.0,561.0){\rule[-0.200pt]{2.409pt}{0.400pt}}
\put(170.0,561.0){\rule[-0.200pt]{2.409pt}{0.400pt}}
\put(1429.0,561.0){\rule[-0.200pt]{2.409pt}{0.400pt}}
\put(170.0,562.0){\rule[-0.200pt]{2.409pt}{0.400pt}}
\put(1429.0,562.0){\rule[-0.200pt]{2.409pt}{0.400pt}}
\put(170.0,562.0){\rule[-0.200pt]{2.409pt}{0.400pt}}
\put(1429.0,562.0){\rule[-0.200pt]{2.409pt}{0.400pt}}
\put(170.0,563.0){\rule[-0.200pt]{2.409pt}{0.400pt}}
\put(1429.0,563.0){\rule[-0.200pt]{2.409pt}{0.400pt}}
\put(170.0,563.0){\rule[-0.200pt]{2.409pt}{0.400pt}}
\put(1429.0,563.0){\rule[-0.200pt]{2.409pt}{0.400pt}}
\put(170.0,564.0){\rule[-0.200pt]{2.409pt}{0.400pt}}
\put(1429.0,564.0){\rule[-0.200pt]{2.409pt}{0.400pt}}
\put(170.0,564.0){\rule[-0.200pt]{2.409pt}{0.400pt}}
\put(1429.0,564.0){\rule[-0.200pt]{2.409pt}{0.400pt}}
\put(170.0,564.0){\rule[-0.200pt]{2.409pt}{0.400pt}}
\put(1429.0,564.0){\rule[-0.200pt]{2.409pt}{0.400pt}}
\put(170.0,565.0){\rule[-0.200pt]{2.409pt}{0.400pt}}
\put(1429.0,565.0){\rule[-0.200pt]{2.409pt}{0.400pt}}
\put(170.0,565.0){\rule[-0.200pt]{2.409pt}{0.400pt}}
\put(1429.0,565.0){\rule[-0.200pt]{2.409pt}{0.400pt}}
\put(170.0,565.0){\rule[-0.200pt]{2.409pt}{0.400pt}}
\put(1429.0,565.0){\rule[-0.200pt]{2.409pt}{0.400pt}}
\put(170.0,566.0){\rule[-0.200pt]{2.409pt}{0.400pt}}
\put(1429.0,566.0){\rule[-0.200pt]{2.409pt}{0.400pt}}
\put(170.0,566.0){\rule[-0.200pt]{2.409pt}{0.400pt}}
\put(1429.0,566.0){\rule[-0.200pt]{2.409pt}{0.400pt}}
\put(170.0,566.0){\rule[-0.200pt]{2.409pt}{0.400pt}}
\put(1429.0,566.0){\rule[-0.200pt]{2.409pt}{0.400pt}}
\put(170.0,567.0){\rule[-0.200pt]{2.409pt}{0.400pt}}
\put(1429.0,567.0){\rule[-0.200pt]{2.409pt}{0.400pt}}
\put(170.0,567.0){\rule[-0.200pt]{2.409pt}{0.400pt}}
\put(1429.0,567.0){\rule[-0.200pt]{2.409pt}{0.400pt}}
\put(170.0,567.0){\rule[-0.200pt]{2.409pt}{0.400pt}}
\put(1429.0,567.0){\rule[-0.200pt]{2.409pt}{0.400pt}}
\put(170.0,568.0){\rule[-0.200pt]{2.409pt}{0.400pt}}
\put(1429.0,568.0){\rule[-0.200pt]{2.409pt}{0.400pt}}
\put(170.0,568.0){\rule[-0.200pt]{2.409pt}{0.400pt}}
\put(1429.0,568.0){\rule[-0.200pt]{2.409pt}{0.400pt}}
\put(170.0,568.0){\rule[-0.200pt]{2.409pt}{0.400pt}}
\put(1429.0,568.0){\rule[-0.200pt]{2.409pt}{0.400pt}}
\put(170.0,569.0){\rule[-0.200pt]{2.409pt}{0.400pt}}
\put(1429.0,569.0){\rule[-0.200pt]{2.409pt}{0.400pt}}
\put(170.0,569.0){\rule[-0.200pt]{2.409pt}{0.400pt}}
\put(1429.0,569.0){\rule[-0.200pt]{2.409pt}{0.400pt}}
\put(170.0,569.0){\rule[-0.200pt]{2.409pt}{0.400pt}}
\put(1429.0,569.0){\rule[-0.200pt]{2.409pt}{0.400pt}}
\put(170.0,570.0){\rule[-0.200pt]{2.409pt}{0.400pt}}
\put(1429.0,570.0){\rule[-0.200pt]{2.409pt}{0.400pt}}
\put(170.0,570.0){\rule[-0.200pt]{2.409pt}{0.400pt}}
\put(1429.0,570.0){\rule[-0.200pt]{2.409pt}{0.400pt}}
\put(170.0,570.0){\rule[-0.200pt]{2.409pt}{0.400pt}}
\put(1429.0,570.0){\rule[-0.200pt]{2.409pt}{0.400pt}}
\put(170.0,570.0){\rule[-0.200pt]{2.409pt}{0.400pt}}
\put(1429.0,570.0){\rule[-0.200pt]{2.409pt}{0.400pt}}
\put(170.0,571.0){\rule[-0.200pt]{2.409pt}{0.400pt}}
\put(1429.0,571.0){\rule[-0.200pt]{2.409pt}{0.400pt}}
\put(170.0,571.0){\rule[-0.200pt]{2.409pt}{0.400pt}}
\put(1429.0,571.0){\rule[-0.200pt]{2.409pt}{0.400pt}}
\put(170.0,571.0){\rule[-0.200pt]{2.409pt}{0.400pt}}
\put(1429.0,571.0){\rule[-0.200pt]{2.409pt}{0.400pt}}
\put(170.0,571.0){\rule[-0.200pt]{2.409pt}{0.400pt}}
\put(1429.0,571.0){\rule[-0.200pt]{2.409pt}{0.400pt}}
\put(170.0,572.0){\rule[-0.200pt]{2.409pt}{0.400pt}}
\put(1429.0,572.0){\rule[-0.200pt]{2.409pt}{0.400pt}}
\put(170.0,572.0){\rule[-0.200pt]{2.409pt}{0.400pt}}
\put(1429.0,572.0){\rule[-0.200pt]{2.409pt}{0.400pt}}
\put(170.0,572.0){\rule[-0.200pt]{2.409pt}{0.400pt}}
\put(1429.0,572.0){\rule[-0.200pt]{2.409pt}{0.400pt}}
\put(170.0,572.0){\rule[-0.200pt]{2.409pt}{0.400pt}}
\put(1429.0,572.0){\rule[-0.200pt]{2.409pt}{0.400pt}}
\put(170.0,573.0){\rule[-0.200pt]{2.409pt}{0.400pt}}
\put(1429.0,573.0){\rule[-0.200pt]{2.409pt}{0.400pt}}
\put(170.0,573.0){\rule[-0.200pt]{2.409pt}{0.400pt}}
\put(1429.0,573.0){\rule[-0.200pt]{2.409pt}{0.400pt}}
\put(170.0,573.0){\rule[-0.200pt]{2.409pt}{0.400pt}}
\put(1429.0,573.0){\rule[-0.200pt]{2.409pt}{0.400pt}}
\put(170.0,573.0){\rule[-0.200pt]{2.409pt}{0.400pt}}
\put(1429.0,573.0){\rule[-0.200pt]{2.409pt}{0.400pt}}
\put(170.0,574.0){\rule[-0.200pt]{2.409pt}{0.400pt}}
\put(1429.0,574.0){\rule[-0.200pt]{2.409pt}{0.400pt}}
\put(170.0,574.0){\rule[-0.200pt]{2.409pt}{0.400pt}}
\put(1429.0,574.0){\rule[-0.200pt]{2.409pt}{0.400pt}}
\put(170.0,574.0){\rule[-0.200pt]{2.409pt}{0.400pt}}
\put(1429.0,574.0){\rule[-0.200pt]{2.409pt}{0.400pt}}
\put(170.0,574.0){\rule[-0.200pt]{2.409pt}{0.400pt}}
\put(1429.0,574.0){\rule[-0.200pt]{2.409pt}{0.400pt}}
\put(170.0,575.0){\rule[-0.200pt]{2.409pt}{0.400pt}}
\put(1429.0,575.0){\rule[-0.200pt]{2.409pt}{0.400pt}}
\put(170.0,575.0){\rule[-0.200pt]{2.409pt}{0.400pt}}
\put(1429.0,575.0){\rule[-0.200pt]{2.409pt}{0.400pt}}
\put(170.0,575.0){\rule[-0.200pt]{2.409pt}{0.400pt}}
\put(1429.0,575.0){\rule[-0.200pt]{2.409pt}{0.400pt}}
\put(170.0,575.0){\rule[-0.200pt]{2.409pt}{0.400pt}}
\put(1429.0,575.0){\rule[-0.200pt]{2.409pt}{0.400pt}}
\put(170.0,575.0){\rule[-0.200pt]{2.409pt}{0.400pt}}
\put(1429.0,575.0){\rule[-0.200pt]{2.409pt}{0.400pt}}
\put(170.0,576.0){\rule[-0.200pt]{2.409pt}{0.400pt}}
\put(1429.0,576.0){\rule[-0.200pt]{2.409pt}{0.400pt}}
\put(170.0,576.0){\rule[-0.200pt]{2.409pt}{0.400pt}}
\put(1429.0,576.0){\rule[-0.200pt]{2.409pt}{0.400pt}}
\put(170.0,576.0){\rule[-0.200pt]{2.409pt}{0.400pt}}
\put(1429.0,576.0){\rule[-0.200pt]{2.409pt}{0.400pt}}
\put(170.0,576.0){\rule[-0.200pt]{2.409pt}{0.400pt}}
\put(1429.0,576.0){\rule[-0.200pt]{2.409pt}{0.400pt}}
\put(170.0,576.0){\rule[-0.200pt]{2.409pt}{0.400pt}}
\put(1429.0,576.0){\rule[-0.200pt]{2.409pt}{0.400pt}}
\put(170.0,577.0){\rule[-0.200pt]{2.409pt}{0.400pt}}
\put(1429.0,577.0){\rule[-0.200pt]{2.409pt}{0.400pt}}
\put(170.0,577.0){\rule[-0.200pt]{2.409pt}{0.400pt}}
\put(1429.0,577.0){\rule[-0.200pt]{2.409pt}{0.400pt}}
\put(170.0,577.0){\rule[-0.200pt]{2.409pt}{0.400pt}}
\put(1429.0,577.0){\rule[-0.200pt]{2.409pt}{0.400pt}}
\put(170.0,577.0){\rule[-0.200pt]{2.409pt}{0.400pt}}
\put(1429.0,577.0){\rule[-0.200pt]{2.409pt}{0.400pt}}
\put(170.0,577.0){\rule[-0.200pt]{2.409pt}{0.400pt}}
\put(1429.0,577.0){\rule[-0.200pt]{2.409pt}{0.400pt}}
\put(170.0,578.0){\rule[-0.200pt]{2.409pt}{0.400pt}}
\put(1429.0,578.0){\rule[-0.200pt]{2.409pt}{0.400pt}}
\put(170.0,578.0){\rule[-0.200pt]{2.409pt}{0.400pt}}
\put(1429.0,578.0){\rule[-0.200pt]{2.409pt}{0.400pt}}
\put(170.0,578.0){\rule[-0.200pt]{2.409pt}{0.400pt}}
\put(1429.0,578.0){\rule[-0.200pt]{2.409pt}{0.400pt}}
\put(170.0,578.0){\rule[-0.200pt]{2.409pt}{0.400pt}}
\put(1429.0,578.0){\rule[-0.200pt]{2.409pt}{0.400pt}}
\put(170.0,578.0){\rule[-0.200pt]{2.409pt}{0.400pt}}
\put(1429.0,578.0){\rule[-0.200pt]{2.409pt}{0.400pt}}
\put(170.0,578.0){\rule[-0.200pt]{2.409pt}{0.400pt}}
\put(1429.0,578.0){\rule[-0.200pt]{2.409pt}{0.400pt}}
\put(170.0,579.0){\rule[-0.200pt]{2.409pt}{0.400pt}}
\put(1429.0,579.0){\rule[-0.200pt]{2.409pt}{0.400pt}}
\put(170.0,579.0){\rule[-0.200pt]{2.409pt}{0.400pt}}
\put(1429.0,579.0){\rule[-0.200pt]{2.409pt}{0.400pt}}
\put(170.0,579.0){\rule[-0.200pt]{2.409pt}{0.400pt}}
\put(1429.0,579.0){\rule[-0.200pt]{2.409pt}{0.400pt}}
\put(170.0,579.0){\rule[-0.200pt]{2.409pt}{0.400pt}}
\put(1429.0,579.0){\rule[-0.200pt]{2.409pt}{0.400pt}}
\put(170.0,579.0){\rule[-0.200pt]{2.409pt}{0.400pt}}
\put(1429.0,579.0){\rule[-0.200pt]{2.409pt}{0.400pt}}
\put(170.0,579.0){\rule[-0.200pt]{2.409pt}{0.400pt}}
\put(1429.0,579.0){\rule[-0.200pt]{2.409pt}{0.400pt}}
\put(170.0,580.0){\rule[-0.200pt]{2.409pt}{0.400pt}}
\put(1429.0,580.0){\rule[-0.200pt]{2.409pt}{0.400pt}}
\put(170.0,580.0){\rule[-0.200pt]{2.409pt}{0.400pt}}
\put(1429.0,580.0){\rule[-0.200pt]{2.409pt}{0.400pt}}
\put(170.0,580.0){\rule[-0.200pt]{2.409pt}{0.400pt}}
\put(1429.0,580.0){\rule[-0.200pt]{2.409pt}{0.400pt}}
\put(170.0,580.0){\rule[-0.200pt]{2.409pt}{0.400pt}}
\put(1429.0,580.0){\rule[-0.200pt]{2.409pt}{0.400pt}}
\put(170.0,580.0){\rule[-0.200pt]{2.409pt}{0.400pt}}
\put(1429.0,580.0){\rule[-0.200pt]{2.409pt}{0.400pt}}
\put(170.0,580.0){\rule[-0.200pt]{2.409pt}{0.400pt}}
\put(1429.0,580.0){\rule[-0.200pt]{2.409pt}{0.400pt}}
\put(170.0,581.0){\rule[-0.200pt]{2.409pt}{0.400pt}}
\put(1429.0,581.0){\rule[-0.200pt]{2.409pt}{0.400pt}}
\put(170.0,581.0){\rule[-0.200pt]{2.409pt}{0.400pt}}
\put(1429.0,581.0){\rule[-0.200pt]{2.409pt}{0.400pt}}
\put(170.0,581.0){\rule[-0.200pt]{2.409pt}{0.400pt}}
\put(1429.0,581.0){\rule[-0.200pt]{2.409pt}{0.400pt}}
\put(170.0,581.0){\rule[-0.200pt]{2.409pt}{0.400pt}}
\put(1429.0,581.0){\rule[-0.200pt]{2.409pt}{0.400pt}}
\put(170.0,581.0){\rule[-0.200pt]{2.409pt}{0.400pt}}
\put(1429.0,581.0){\rule[-0.200pt]{2.409pt}{0.400pt}}
\put(170.0,581.0){\rule[-0.200pt]{2.409pt}{0.400pt}}
\put(1429.0,581.0){\rule[-0.200pt]{2.409pt}{0.400pt}}
\put(170.0,582.0){\rule[-0.200pt]{2.409pt}{0.400pt}}
\put(1429.0,582.0){\rule[-0.200pt]{2.409pt}{0.400pt}}
\put(170.0,582.0){\rule[-0.200pt]{2.409pt}{0.400pt}}
\put(1429.0,582.0){\rule[-0.200pt]{2.409pt}{0.400pt}}
\put(170.0,582.0){\rule[-0.200pt]{2.409pt}{0.400pt}}
\put(1429.0,582.0){\rule[-0.200pt]{2.409pt}{0.400pt}}
\put(170.0,582.0){\rule[-0.200pt]{2.409pt}{0.400pt}}
\put(1429.0,582.0){\rule[-0.200pt]{2.409pt}{0.400pt}}
\put(170.0,582.0){\rule[-0.200pt]{2.409pt}{0.400pt}}
\put(1429.0,582.0){\rule[-0.200pt]{2.409pt}{0.400pt}}
\put(170.0,582.0){\rule[-0.200pt]{2.409pt}{0.400pt}}
\put(1429.0,582.0){\rule[-0.200pt]{2.409pt}{0.400pt}}
\put(170.0,582.0){\rule[-0.200pt]{2.409pt}{0.400pt}}
\put(1429.0,582.0){\rule[-0.200pt]{2.409pt}{0.400pt}}
\put(170.0,583.0){\rule[-0.200pt]{2.409pt}{0.400pt}}
\put(1429.0,583.0){\rule[-0.200pt]{2.409pt}{0.400pt}}
\put(170.0,583.0){\rule[-0.200pt]{2.409pt}{0.400pt}}
\put(1429.0,583.0){\rule[-0.200pt]{2.409pt}{0.400pt}}
\put(170.0,583.0){\rule[-0.200pt]{2.409pt}{0.400pt}}
\put(1429.0,583.0){\rule[-0.200pt]{2.409pt}{0.400pt}}
\put(170.0,583.0){\rule[-0.200pt]{2.409pt}{0.400pt}}
\put(1429.0,583.0){\rule[-0.200pt]{2.409pt}{0.400pt}}
\put(170.0,583.0){\rule[-0.200pt]{2.409pt}{0.400pt}}
\put(1429.0,583.0){\rule[-0.200pt]{2.409pt}{0.400pt}}
\put(170.0,583.0){\rule[-0.200pt]{2.409pt}{0.400pt}}
\put(1429.0,583.0){\rule[-0.200pt]{2.409pt}{0.400pt}}
\put(170.0,583.0){\rule[-0.200pt]{2.409pt}{0.400pt}}
\put(1429.0,583.0){\rule[-0.200pt]{2.409pt}{0.400pt}}
\put(170.0,584.0){\rule[-0.200pt]{2.409pt}{0.400pt}}
\put(1429.0,584.0){\rule[-0.200pt]{2.409pt}{0.400pt}}
\put(170.0,584.0){\rule[-0.200pt]{2.409pt}{0.400pt}}
\put(1429.0,584.0){\rule[-0.200pt]{2.409pt}{0.400pt}}
\put(170.0,584.0){\rule[-0.200pt]{2.409pt}{0.400pt}}
\put(1429.0,584.0){\rule[-0.200pt]{2.409pt}{0.400pt}}
\put(170.0,584.0){\rule[-0.200pt]{2.409pt}{0.400pt}}
\put(1429.0,584.0){\rule[-0.200pt]{2.409pt}{0.400pt}}
\put(170.0,584.0){\rule[-0.200pt]{2.409pt}{0.400pt}}
\put(1429.0,584.0){\rule[-0.200pt]{2.409pt}{0.400pt}}
\put(170.0,584.0){\rule[-0.200pt]{2.409pt}{0.400pt}}
\put(1429.0,584.0){\rule[-0.200pt]{2.409pt}{0.400pt}}
\put(170.0,584.0){\rule[-0.200pt]{2.409pt}{0.400pt}}
\put(1429.0,584.0){\rule[-0.200pt]{2.409pt}{0.400pt}}
\put(170.0,584.0){\rule[-0.200pt]{2.409pt}{0.400pt}}
\put(1429.0,584.0){\rule[-0.200pt]{2.409pt}{0.400pt}}
\put(170.0,585.0){\rule[-0.200pt]{2.409pt}{0.400pt}}
\put(1429.0,585.0){\rule[-0.200pt]{2.409pt}{0.400pt}}
\put(170.0,585.0){\rule[-0.200pt]{2.409pt}{0.400pt}}
\put(1429.0,585.0){\rule[-0.200pt]{2.409pt}{0.400pt}}
\put(170.0,585.0){\rule[-0.200pt]{2.409pt}{0.400pt}}
\put(1429.0,585.0){\rule[-0.200pt]{2.409pt}{0.400pt}}
\put(170.0,585.0){\rule[-0.200pt]{2.409pt}{0.400pt}}
\put(1429.0,585.0){\rule[-0.200pt]{2.409pt}{0.400pt}}
\put(170.0,585.0){\rule[-0.200pt]{2.409pt}{0.400pt}}
\put(1429.0,585.0){\rule[-0.200pt]{2.409pt}{0.400pt}}
\put(170.0,585.0){\rule[-0.200pt]{2.409pt}{0.400pt}}
\put(1429.0,585.0){\rule[-0.200pt]{2.409pt}{0.400pt}}
\put(170.0,585.0){\rule[-0.200pt]{2.409pt}{0.400pt}}
\put(1429.0,585.0){\rule[-0.200pt]{2.409pt}{0.400pt}}
\put(170.0,585.0){\rule[-0.200pt]{2.409pt}{0.400pt}}
\put(1429.0,585.0){\rule[-0.200pt]{2.409pt}{0.400pt}}
\put(170.0,586.0){\rule[-0.200pt]{2.409pt}{0.400pt}}
\put(1429.0,586.0){\rule[-0.200pt]{2.409pt}{0.400pt}}
\put(170.0,586.0){\rule[-0.200pt]{2.409pt}{0.400pt}}
\put(1429.0,586.0){\rule[-0.200pt]{2.409pt}{0.400pt}}
\put(170.0,586.0){\rule[-0.200pt]{2.409pt}{0.400pt}}
\put(1429.0,586.0){\rule[-0.200pt]{2.409pt}{0.400pt}}
\put(170.0,586.0){\rule[-0.200pt]{2.409pt}{0.400pt}}
\put(1429.0,586.0){\rule[-0.200pt]{2.409pt}{0.400pt}}
\put(170.0,586.0){\rule[-0.200pt]{2.409pt}{0.400pt}}
\put(1429.0,586.0){\rule[-0.200pt]{2.409pt}{0.400pt}}
\put(170.0,586.0){\rule[-0.200pt]{2.409pt}{0.400pt}}
\put(1429.0,586.0){\rule[-0.200pt]{2.409pt}{0.400pt}}
\put(170.0,586.0){\rule[-0.200pt]{2.409pt}{0.400pt}}
\put(1429.0,586.0){\rule[-0.200pt]{2.409pt}{0.400pt}}
\put(170.0,586.0){\rule[-0.200pt]{2.409pt}{0.400pt}}
\put(1429.0,586.0){\rule[-0.200pt]{2.409pt}{0.400pt}}
\put(170.0,587.0){\rule[-0.200pt]{2.409pt}{0.400pt}}
\put(1429.0,587.0){\rule[-0.200pt]{2.409pt}{0.400pt}}
\put(170.0,587.0){\rule[-0.200pt]{2.409pt}{0.400pt}}
\put(1429.0,587.0){\rule[-0.200pt]{2.409pt}{0.400pt}}
\put(170.0,587.0){\rule[-0.200pt]{2.409pt}{0.400pt}}
\put(1429.0,587.0){\rule[-0.200pt]{2.409pt}{0.400pt}}
\put(170.0,587.0){\rule[-0.200pt]{2.409pt}{0.400pt}}
\put(1429.0,587.0){\rule[-0.200pt]{2.409pt}{0.400pt}}
\put(170.0,587.0){\rule[-0.200pt]{2.409pt}{0.400pt}}
\put(1429.0,587.0){\rule[-0.200pt]{2.409pt}{0.400pt}}
\put(170.0,587.0){\rule[-0.200pt]{2.409pt}{0.400pt}}
\put(1429.0,587.0){\rule[-0.200pt]{2.409pt}{0.400pt}}
\put(170.0,587.0){\rule[-0.200pt]{2.409pt}{0.400pt}}
\put(1429.0,587.0){\rule[-0.200pt]{2.409pt}{0.400pt}}
\put(170.0,587.0){\rule[-0.200pt]{2.409pt}{0.400pt}}
\put(1429.0,587.0){\rule[-0.200pt]{2.409pt}{0.400pt}}
\put(170.0,587.0){\rule[-0.200pt]{2.409pt}{0.400pt}}
\put(1429.0,587.0){\rule[-0.200pt]{2.409pt}{0.400pt}}
\put(170.0,588.0){\rule[-0.200pt]{2.409pt}{0.400pt}}
\put(1429.0,588.0){\rule[-0.200pt]{2.409pt}{0.400pt}}
\put(170.0,588.0){\rule[-0.200pt]{2.409pt}{0.400pt}}
\put(1429.0,588.0){\rule[-0.200pt]{2.409pt}{0.400pt}}
\put(170.0,588.0){\rule[-0.200pt]{2.409pt}{0.400pt}}
\put(1429.0,588.0){\rule[-0.200pt]{2.409pt}{0.400pt}}
\put(170.0,588.0){\rule[-0.200pt]{2.409pt}{0.400pt}}
\put(1429.0,588.0){\rule[-0.200pt]{2.409pt}{0.400pt}}
\put(170.0,588.0){\rule[-0.200pt]{2.409pt}{0.400pt}}
\put(1429.0,588.0){\rule[-0.200pt]{2.409pt}{0.400pt}}
\put(170.0,588.0){\rule[-0.200pt]{2.409pt}{0.400pt}}
\put(1429.0,588.0){\rule[-0.200pt]{2.409pt}{0.400pt}}
\put(170.0,588.0){\rule[-0.200pt]{2.409pt}{0.400pt}}
\put(1429.0,588.0){\rule[-0.200pt]{2.409pt}{0.400pt}}
\put(170.0,588.0){\rule[-0.200pt]{2.409pt}{0.400pt}}
\put(1429.0,588.0){\rule[-0.200pt]{2.409pt}{0.400pt}}
\put(170.0,588.0){\rule[-0.200pt]{2.409pt}{0.400pt}}
\put(1429.0,588.0){\rule[-0.200pt]{2.409pt}{0.400pt}}
\put(170.0,589.0){\rule[-0.200pt]{2.409pt}{0.400pt}}
\put(1429.0,589.0){\rule[-0.200pt]{2.409pt}{0.400pt}}
\put(170.0,589.0){\rule[-0.200pt]{2.409pt}{0.400pt}}
\put(1429.0,589.0){\rule[-0.200pt]{2.409pt}{0.400pt}}
\put(170.0,589.0){\rule[-0.200pt]{2.409pt}{0.400pt}}
\put(1429.0,589.0){\rule[-0.200pt]{2.409pt}{0.400pt}}
\put(170.0,589.0){\rule[-0.200pt]{2.409pt}{0.400pt}}
\put(1429.0,589.0){\rule[-0.200pt]{2.409pt}{0.400pt}}
\put(170.0,589.0){\rule[-0.200pt]{2.409pt}{0.400pt}}
\put(1429.0,589.0){\rule[-0.200pt]{2.409pt}{0.400pt}}
\put(170.0,589.0){\rule[-0.200pt]{2.409pt}{0.400pt}}
\put(1429.0,589.0){\rule[-0.200pt]{2.409pt}{0.400pt}}
\put(170.0,589.0){\rule[-0.200pt]{2.409pt}{0.400pt}}
\put(1429.0,589.0){\rule[-0.200pt]{2.409pt}{0.400pt}}
\put(170.0,589.0){\rule[-0.200pt]{2.409pt}{0.400pt}}
\put(1429.0,589.0){\rule[-0.200pt]{2.409pt}{0.400pt}}
\put(170.0,589.0){\rule[-0.200pt]{2.409pt}{0.400pt}}
\put(1429.0,589.0){\rule[-0.200pt]{2.409pt}{0.400pt}}
\put(170.0,589.0){\rule[-0.200pt]{2.409pt}{0.400pt}}
\put(1429.0,589.0){\rule[-0.200pt]{2.409pt}{0.400pt}}
\put(170.0,590.0){\rule[-0.200pt]{2.409pt}{0.400pt}}
\put(1429.0,590.0){\rule[-0.200pt]{2.409pt}{0.400pt}}
\put(170.0,590.0){\rule[-0.200pt]{2.409pt}{0.400pt}}
\put(1429.0,590.0){\rule[-0.200pt]{2.409pt}{0.400pt}}
\put(170.0,590.0){\rule[-0.200pt]{2.409pt}{0.400pt}}
\put(1429.0,590.0){\rule[-0.200pt]{2.409pt}{0.400pt}}
\put(170.0,590.0){\rule[-0.200pt]{2.409pt}{0.400pt}}
\put(1429.0,590.0){\rule[-0.200pt]{2.409pt}{0.400pt}}
\put(170.0,590.0){\rule[-0.200pt]{2.409pt}{0.400pt}}
\put(1429.0,590.0){\rule[-0.200pt]{2.409pt}{0.400pt}}
\put(170.0,590.0){\rule[-0.200pt]{2.409pt}{0.400pt}}
\put(1429.0,590.0){\rule[-0.200pt]{2.409pt}{0.400pt}}
\put(170.0,590.0){\rule[-0.200pt]{2.409pt}{0.400pt}}
\put(1429.0,590.0){\rule[-0.200pt]{2.409pt}{0.400pt}}
\put(170.0,590.0){\rule[-0.200pt]{2.409pt}{0.400pt}}
\put(1429.0,590.0){\rule[-0.200pt]{2.409pt}{0.400pt}}
\put(170.0,590.0){\rule[-0.200pt]{2.409pt}{0.400pt}}
\put(1429.0,590.0){\rule[-0.200pt]{2.409pt}{0.400pt}}
\put(170.0,590.0){\rule[-0.200pt]{2.409pt}{0.400pt}}
\put(1429.0,590.0){\rule[-0.200pt]{2.409pt}{0.400pt}}
\put(170.0,590.0){\rule[-0.200pt]{2.409pt}{0.400pt}}
\put(1429.0,590.0){\rule[-0.200pt]{2.409pt}{0.400pt}}
\put(170.0,591.0){\rule[-0.200pt]{2.409pt}{0.400pt}}
\put(1429.0,591.0){\rule[-0.200pt]{2.409pt}{0.400pt}}
\put(170.0,591.0){\rule[-0.200pt]{2.409pt}{0.400pt}}
\put(1429.0,591.0){\rule[-0.200pt]{2.409pt}{0.400pt}}
\put(170.0,591.0){\rule[-0.200pt]{2.409pt}{0.400pt}}
\put(1429.0,591.0){\rule[-0.200pt]{2.409pt}{0.400pt}}
\put(170.0,591.0){\rule[-0.200pt]{2.409pt}{0.400pt}}
\put(1429.0,591.0){\rule[-0.200pt]{2.409pt}{0.400pt}}
\put(170.0,591.0){\rule[-0.200pt]{2.409pt}{0.400pt}}
\put(1429.0,591.0){\rule[-0.200pt]{2.409pt}{0.400pt}}
\put(170.0,591.0){\rule[-0.200pt]{2.409pt}{0.400pt}}
\put(1429.0,591.0){\rule[-0.200pt]{2.409pt}{0.400pt}}
\put(170.0,591.0){\rule[-0.200pt]{2.409pt}{0.400pt}}
\put(1429.0,591.0){\rule[-0.200pt]{2.409pt}{0.400pt}}
\put(170.0,591.0){\rule[-0.200pt]{2.409pt}{0.400pt}}
\put(1429.0,591.0){\rule[-0.200pt]{2.409pt}{0.400pt}}
\put(170.0,591.0){\rule[-0.200pt]{2.409pt}{0.400pt}}
\put(1429.0,591.0){\rule[-0.200pt]{2.409pt}{0.400pt}}
\put(170.0,591.0){\rule[-0.200pt]{2.409pt}{0.400pt}}
\put(1429.0,591.0){\rule[-0.200pt]{2.409pt}{0.400pt}}
\put(170.0,591.0){\rule[-0.200pt]{2.409pt}{0.400pt}}
\put(1429.0,591.0){\rule[-0.200pt]{2.409pt}{0.400pt}}
\put(170.0,592.0){\rule[-0.200pt]{2.409pt}{0.400pt}}
\put(1429.0,592.0){\rule[-0.200pt]{2.409pt}{0.400pt}}
\put(170.0,592.0){\rule[-0.200pt]{2.409pt}{0.400pt}}
\put(1429.0,592.0){\rule[-0.200pt]{2.409pt}{0.400pt}}
\put(170.0,592.0){\rule[-0.200pt]{2.409pt}{0.400pt}}
\put(1429.0,592.0){\rule[-0.200pt]{2.409pt}{0.400pt}}
\put(170.0,592.0){\rule[-0.200pt]{2.409pt}{0.400pt}}
\put(1429.0,592.0){\rule[-0.200pt]{2.409pt}{0.400pt}}
\put(170.0,592.0){\rule[-0.200pt]{2.409pt}{0.400pt}}
\put(1429.0,592.0){\rule[-0.200pt]{2.409pt}{0.400pt}}
\put(170.0,592.0){\rule[-0.200pt]{2.409pt}{0.400pt}}
\put(1429.0,592.0){\rule[-0.200pt]{2.409pt}{0.400pt}}
\put(170.0,592.0){\rule[-0.200pt]{2.409pt}{0.400pt}}
\put(1429.0,592.0){\rule[-0.200pt]{2.409pt}{0.400pt}}
\put(170.0,592.0){\rule[-0.200pt]{2.409pt}{0.400pt}}
\put(1429.0,592.0){\rule[-0.200pt]{2.409pt}{0.400pt}}
\put(170.0,592.0){\rule[-0.200pt]{2.409pt}{0.400pt}}
\put(1429.0,592.0){\rule[-0.200pt]{2.409pt}{0.400pt}}
\put(170.0,592.0){\rule[-0.200pt]{2.409pt}{0.400pt}}
\put(1429.0,592.0){\rule[-0.200pt]{2.409pt}{0.400pt}}
\put(170.0,592.0){\rule[-0.200pt]{2.409pt}{0.400pt}}
\put(1429.0,592.0){\rule[-0.200pt]{2.409pt}{0.400pt}}
\put(170.0,593.0){\rule[-0.200pt]{2.409pt}{0.400pt}}
\put(1429.0,593.0){\rule[-0.200pt]{2.409pt}{0.400pt}}
\put(170.0,593.0){\rule[-0.200pt]{2.409pt}{0.400pt}}
\put(1429.0,593.0){\rule[-0.200pt]{2.409pt}{0.400pt}}
\put(170.0,593.0){\rule[-0.200pt]{2.409pt}{0.400pt}}
\put(1429.0,593.0){\rule[-0.200pt]{2.409pt}{0.400pt}}
\put(170.0,593.0){\rule[-0.200pt]{2.409pt}{0.400pt}}
\put(1429.0,593.0){\rule[-0.200pt]{2.409pt}{0.400pt}}
\put(170.0,593.0){\rule[-0.200pt]{2.409pt}{0.400pt}}
\put(1429.0,593.0){\rule[-0.200pt]{2.409pt}{0.400pt}}
\put(170.0,593.0){\rule[-0.200pt]{2.409pt}{0.400pt}}
\put(1429.0,593.0){\rule[-0.200pt]{2.409pt}{0.400pt}}
\put(170.0,593.0){\rule[-0.200pt]{2.409pt}{0.400pt}}
\put(1429.0,593.0){\rule[-0.200pt]{2.409pt}{0.400pt}}
\put(170.0,593.0){\rule[-0.200pt]{2.409pt}{0.400pt}}
\put(1429.0,593.0){\rule[-0.200pt]{2.409pt}{0.400pt}}
\put(170.0,593.0){\rule[-0.200pt]{2.409pt}{0.400pt}}
\put(1429.0,593.0){\rule[-0.200pt]{2.409pt}{0.400pt}}
\put(170.0,593.0){\rule[-0.200pt]{2.409pt}{0.400pt}}
\put(1429.0,593.0){\rule[-0.200pt]{2.409pt}{0.400pt}}
\put(170.0,593.0){\rule[-0.200pt]{2.409pt}{0.400pt}}
\put(1429.0,593.0){\rule[-0.200pt]{2.409pt}{0.400pt}}
\put(170.0,593.0){\rule[-0.200pt]{2.409pt}{0.400pt}}
\put(1429.0,593.0){\rule[-0.200pt]{2.409pt}{0.400pt}}
\put(170.0,593.0){\rule[-0.200pt]{2.409pt}{0.400pt}}
\put(1429.0,593.0){\rule[-0.200pt]{2.409pt}{0.400pt}}
\put(170.0,594.0){\rule[-0.200pt]{2.409pt}{0.400pt}}
\put(1429.0,594.0){\rule[-0.200pt]{2.409pt}{0.400pt}}
\put(170.0,594.0){\rule[-0.200pt]{2.409pt}{0.400pt}}
\put(1429.0,594.0){\rule[-0.200pt]{2.409pt}{0.400pt}}
\put(170.0,594.0){\rule[-0.200pt]{2.409pt}{0.400pt}}
\put(1429.0,594.0){\rule[-0.200pt]{2.409pt}{0.400pt}}
\put(170.0,594.0){\rule[-0.200pt]{2.409pt}{0.400pt}}
\put(1429.0,594.0){\rule[-0.200pt]{2.409pt}{0.400pt}}
\put(170.0,594.0){\rule[-0.200pt]{2.409pt}{0.400pt}}
\put(1429.0,594.0){\rule[-0.200pt]{2.409pt}{0.400pt}}
\put(170.0,594.0){\rule[-0.200pt]{2.409pt}{0.400pt}}
\put(1429.0,594.0){\rule[-0.200pt]{2.409pt}{0.400pt}}
\put(170.0,594.0){\rule[-0.200pt]{2.409pt}{0.400pt}}
\put(1429.0,594.0){\rule[-0.200pt]{2.409pt}{0.400pt}}
\put(170.0,594.0){\rule[-0.200pt]{2.409pt}{0.400pt}}
\put(1429.0,594.0){\rule[-0.200pt]{2.409pt}{0.400pt}}
\put(170.0,594.0){\rule[-0.200pt]{2.409pt}{0.400pt}}
\put(1429.0,594.0){\rule[-0.200pt]{2.409pt}{0.400pt}}
\put(170.0,594.0){\rule[-0.200pt]{2.409pt}{0.400pt}}
\put(1429.0,594.0){\rule[-0.200pt]{2.409pt}{0.400pt}}
\put(170.0,594.0){\rule[-0.200pt]{2.409pt}{0.400pt}}
\put(1429.0,594.0){\rule[-0.200pt]{2.409pt}{0.400pt}}
\put(170.0,594.0){\rule[-0.200pt]{2.409pt}{0.400pt}}
\put(1429.0,594.0){\rule[-0.200pt]{2.409pt}{0.400pt}}
\put(170.0,595.0){\rule[-0.200pt]{2.409pt}{0.400pt}}
\put(1429.0,595.0){\rule[-0.200pt]{2.409pt}{0.400pt}}
\put(170.0,595.0){\rule[-0.200pt]{2.409pt}{0.400pt}}
\put(1429.0,595.0){\rule[-0.200pt]{2.409pt}{0.400pt}}
\put(170.0,595.0){\rule[-0.200pt]{2.409pt}{0.400pt}}
\put(1429.0,595.0){\rule[-0.200pt]{2.409pt}{0.400pt}}
\put(170.0,595.0){\rule[-0.200pt]{2.409pt}{0.400pt}}
\put(1429.0,595.0){\rule[-0.200pt]{2.409pt}{0.400pt}}
\put(170.0,595.0){\rule[-0.200pt]{2.409pt}{0.400pt}}
\put(1429.0,595.0){\rule[-0.200pt]{2.409pt}{0.400pt}}
\put(170.0,595.0){\rule[-0.200pt]{2.409pt}{0.400pt}}
\put(1429.0,595.0){\rule[-0.200pt]{2.409pt}{0.400pt}}
\put(170.0,595.0){\rule[-0.200pt]{2.409pt}{0.400pt}}
\put(1429.0,595.0){\rule[-0.200pt]{2.409pt}{0.400pt}}
\put(170.0,595.0){\rule[-0.200pt]{2.409pt}{0.400pt}}
\put(1429.0,595.0){\rule[-0.200pt]{2.409pt}{0.400pt}}
\put(170.0,595.0){\rule[-0.200pt]{2.409pt}{0.400pt}}
\put(1429.0,595.0){\rule[-0.200pt]{2.409pt}{0.400pt}}
\put(170.0,595.0){\rule[-0.200pt]{2.409pt}{0.400pt}}
\put(1429.0,595.0){\rule[-0.200pt]{2.409pt}{0.400pt}}
\put(170.0,595.0){\rule[-0.200pt]{2.409pt}{0.400pt}}
\put(1429.0,595.0){\rule[-0.200pt]{2.409pt}{0.400pt}}
\put(170.0,595.0){\rule[-0.200pt]{2.409pt}{0.400pt}}
\put(1429.0,595.0){\rule[-0.200pt]{2.409pt}{0.400pt}}
\put(170.0,595.0){\rule[-0.200pt]{2.409pt}{0.400pt}}
\put(1429.0,595.0){\rule[-0.200pt]{2.409pt}{0.400pt}}
\put(170.0,595.0){\rule[-0.200pt]{2.409pt}{0.400pt}}
\put(1429.0,595.0){\rule[-0.200pt]{2.409pt}{0.400pt}}
\put(170.0,596.0){\rule[-0.200pt]{2.409pt}{0.400pt}}
\put(1429.0,596.0){\rule[-0.200pt]{2.409pt}{0.400pt}}
\put(170.0,596.0){\rule[-0.200pt]{2.409pt}{0.400pt}}
\put(1429.0,596.0){\rule[-0.200pt]{2.409pt}{0.400pt}}
\put(170.0,596.0){\rule[-0.200pt]{2.409pt}{0.400pt}}
\put(1429.0,596.0){\rule[-0.200pt]{2.409pt}{0.400pt}}
\put(170.0,596.0){\rule[-0.200pt]{2.409pt}{0.400pt}}
\put(1429.0,596.0){\rule[-0.200pt]{2.409pt}{0.400pt}}
\put(170.0,596.0){\rule[-0.200pt]{2.409pt}{0.400pt}}
\put(1429.0,596.0){\rule[-0.200pt]{2.409pt}{0.400pt}}
\put(170.0,596.0){\rule[-0.200pt]{2.409pt}{0.400pt}}
\put(1429.0,596.0){\rule[-0.200pt]{2.409pt}{0.400pt}}
\put(170.0,596.0){\rule[-0.200pt]{2.409pt}{0.400pt}}
\put(1429.0,596.0){\rule[-0.200pt]{2.409pt}{0.400pt}}
\put(170.0,596.0){\rule[-0.200pt]{2.409pt}{0.400pt}}
\put(1429.0,596.0){\rule[-0.200pt]{2.409pt}{0.400pt}}
\put(170.0,596.0){\rule[-0.200pt]{2.409pt}{0.400pt}}
\put(1429.0,596.0){\rule[-0.200pt]{2.409pt}{0.400pt}}
\put(170.0,596.0){\rule[-0.200pt]{2.409pt}{0.400pt}}
\put(1429.0,596.0){\rule[-0.200pt]{2.409pt}{0.400pt}}
\put(170.0,596.0){\rule[-0.200pt]{2.409pt}{0.400pt}}
\put(1429.0,596.0){\rule[-0.200pt]{2.409pt}{0.400pt}}
\put(170.0,596.0){\rule[-0.200pt]{2.409pt}{0.400pt}}
\put(1429.0,596.0){\rule[-0.200pt]{2.409pt}{0.400pt}}
\put(170.0,596.0){\rule[-0.200pt]{2.409pt}{0.400pt}}
\put(1429.0,596.0){\rule[-0.200pt]{2.409pt}{0.400pt}}
\put(170.0,596.0){\rule[-0.200pt]{2.409pt}{0.400pt}}
\put(1429.0,596.0){\rule[-0.200pt]{2.409pt}{0.400pt}}
\put(170.0,597.0){\rule[-0.200pt]{2.409pt}{0.400pt}}
\put(1429.0,597.0){\rule[-0.200pt]{2.409pt}{0.400pt}}
\put(170.0,597.0){\rule[-0.200pt]{2.409pt}{0.400pt}}
\put(1429.0,597.0){\rule[-0.200pt]{2.409pt}{0.400pt}}
\put(170.0,597.0){\rule[-0.200pt]{2.409pt}{0.400pt}}
\put(1429.0,597.0){\rule[-0.200pt]{2.409pt}{0.400pt}}
\put(170.0,597.0){\rule[-0.200pt]{2.409pt}{0.400pt}}
\put(1429.0,597.0){\rule[-0.200pt]{2.409pt}{0.400pt}}
\put(170.0,597.0){\rule[-0.200pt]{2.409pt}{0.400pt}}
\put(1429.0,597.0){\rule[-0.200pt]{2.409pt}{0.400pt}}
\put(170.0,597.0){\rule[-0.200pt]{2.409pt}{0.400pt}}
\put(1429.0,597.0){\rule[-0.200pt]{2.409pt}{0.400pt}}
\put(170.0,597.0){\rule[-0.200pt]{2.409pt}{0.400pt}}
\put(1429.0,597.0){\rule[-0.200pt]{2.409pt}{0.400pt}}
\put(170.0,597.0){\rule[-0.200pt]{2.409pt}{0.400pt}}
\put(1429.0,597.0){\rule[-0.200pt]{2.409pt}{0.400pt}}
\put(170.0,597.0){\rule[-0.200pt]{2.409pt}{0.400pt}}
\put(1429.0,597.0){\rule[-0.200pt]{2.409pt}{0.400pt}}
\put(170.0,597.0){\rule[-0.200pt]{2.409pt}{0.400pt}}
\put(1429.0,597.0){\rule[-0.200pt]{2.409pt}{0.400pt}}
\put(170.0,597.0){\rule[-0.200pt]{2.409pt}{0.400pt}}
\put(1429.0,597.0){\rule[-0.200pt]{2.409pt}{0.400pt}}
\put(170.0,597.0){\rule[-0.200pt]{2.409pt}{0.400pt}}
\put(1429.0,597.0){\rule[-0.200pt]{2.409pt}{0.400pt}}
\put(170.0,597.0){\rule[-0.200pt]{2.409pt}{0.400pt}}
\put(1429.0,597.0){\rule[-0.200pt]{2.409pt}{0.400pt}}
\put(170.0,597.0){\rule[-0.200pt]{2.409pt}{0.400pt}}
\put(1429.0,597.0){\rule[-0.200pt]{2.409pt}{0.400pt}}
\put(170.0,597.0){\rule[-0.200pt]{2.409pt}{0.400pt}}
\put(1429.0,597.0){\rule[-0.200pt]{2.409pt}{0.400pt}}
\put(170.0,597.0){\rule[-0.200pt]{2.409pt}{0.400pt}}
\put(1429.0,597.0){\rule[-0.200pt]{2.409pt}{0.400pt}}
\put(170.0,598.0){\rule[-0.200pt]{2.409pt}{0.400pt}}
\put(1429.0,598.0){\rule[-0.200pt]{2.409pt}{0.400pt}}
\put(170.0,598.0){\rule[-0.200pt]{2.409pt}{0.400pt}}
\put(1429.0,598.0){\rule[-0.200pt]{2.409pt}{0.400pt}}
\put(170.0,598.0){\rule[-0.200pt]{2.409pt}{0.400pt}}
\put(1429.0,598.0){\rule[-0.200pt]{2.409pt}{0.400pt}}
\put(170.0,598.0){\rule[-0.200pt]{2.409pt}{0.400pt}}
\put(1429.0,598.0){\rule[-0.200pt]{2.409pt}{0.400pt}}
\put(170.0,598.0){\rule[-0.200pt]{2.409pt}{0.400pt}}
\put(1429.0,598.0){\rule[-0.200pt]{2.409pt}{0.400pt}}
\put(170.0,598.0){\rule[-0.200pt]{2.409pt}{0.400pt}}
\put(1429.0,598.0){\rule[-0.200pt]{2.409pt}{0.400pt}}
\put(170.0,598.0){\rule[-0.200pt]{2.409pt}{0.400pt}}
\put(1429.0,598.0){\rule[-0.200pt]{2.409pt}{0.400pt}}
\put(170.0,598.0){\rule[-0.200pt]{2.409pt}{0.400pt}}
\put(1429.0,598.0){\rule[-0.200pt]{2.409pt}{0.400pt}}
\put(170.0,598.0){\rule[-0.200pt]{2.409pt}{0.400pt}}
\put(1429.0,598.0){\rule[-0.200pt]{2.409pt}{0.400pt}}
\put(170.0,598.0){\rule[-0.200pt]{2.409pt}{0.400pt}}
\put(1429.0,598.0){\rule[-0.200pt]{2.409pt}{0.400pt}}
\put(170.0,598.0){\rule[-0.200pt]{2.409pt}{0.400pt}}
\put(1429.0,598.0){\rule[-0.200pt]{2.409pt}{0.400pt}}
\put(170.0,598.0){\rule[-0.200pt]{2.409pt}{0.400pt}}
\put(1429.0,598.0){\rule[-0.200pt]{2.409pt}{0.400pt}}
\put(170.0,598.0){\rule[-0.200pt]{2.409pt}{0.400pt}}
\put(1429.0,598.0){\rule[-0.200pt]{2.409pt}{0.400pt}}
\put(170.0,598.0){\rule[-0.200pt]{2.409pt}{0.400pt}}
\put(1429.0,598.0){\rule[-0.200pt]{2.409pt}{0.400pt}}
\put(170.0,598.0){\rule[-0.200pt]{2.409pt}{0.400pt}}
\put(1429.0,598.0){\rule[-0.200pt]{2.409pt}{0.400pt}}
\put(170.0,598.0){\rule[-0.200pt]{2.409pt}{0.400pt}}
\put(1429.0,598.0){\rule[-0.200pt]{2.409pt}{0.400pt}}
\put(170.0,599.0){\rule[-0.200pt]{2.409pt}{0.400pt}}
\put(1429.0,599.0){\rule[-0.200pt]{2.409pt}{0.400pt}}
\put(170.0,599.0){\rule[-0.200pt]{2.409pt}{0.400pt}}
\put(1429.0,599.0){\rule[-0.200pt]{2.409pt}{0.400pt}}
\put(170.0,599.0){\rule[-0.200pt]{2.409pt}{0.400pt}}
\put(1429.0,599.0){\rule[-0.200pt]{2.409pt}{0.400pt}}
\put(170.0,599.0){\rule[-0.200pt]{2.409pt}{0.400pt}}
\put(1429.0,599.0){\rule[-0.200pt]{2.409pt}{0.400pt}}
\put(170.0,599.0){\rule[-0.200pt]{2.409pt}{0.400pt}}
\put(1429.0,599.0){\rule[-0.200pt]{2.409pt}{0.400pt}}
\put(170.0,599.0){\rule[-0.200pt]{2.409pt}{0.400pt}}
\put(1429.0,599.0){\rule[-0.200pt]{2.409pt}{0.400pt}}
\put(170.0,599.0){\rule[-0.200pt]{2.409pt}{0.400pt}}
\put(1429.0,599.0){\rule[-0.200pt]{2.409pt}{0.400pt}}
\put(170.0,599.0){\rule[-0.200pt]{2.409pt}{0.400pt}}
\put(1429.0,599.0){\rule[-0.200pt]{2.409pt}{0.400pt}}
\put(170.0,599.0){\rule[-0.200pt]{2.409pt}{0.400pt}}
\put(1429.0,599.0){\rule[-0.200pt]{2.409pt}{0.400pt}}
\put(170.0,599.0){\rule[-0.200pt]{2.409pt}{0.400pt}}
\put(1429.0,599.0){\rule[-0.200pt]{2.409pt}{0.400pt}}
\put(170.0,599.0){\rule[-0.200pt]{2.409pt}{0.400pt}}
\put(1429.0,599.0){\rule[-0.200pt]{2.409pt}{0.400pt}}
\put(170.0,599.0){\rule[-0.200pt]{2.409pt}{0.400pt}}
\put(1429.0,599.0){\rule[-0.200pt]{2.409pt}{0.400pt}}
\put(170.0,599.0){\rule[-0.200pt]{2.409pt}{0.400pt}}
\put(1429.0,599.0){\rule[-0.200pt]{2.409pt}{0.400pt}}
\put(170.0,599.0){\rule[-0.200pt]{2.409pt}{0.400pt}}
\put(1429.0,599.0){\rule[-0.200pt]{2.409pt}{0.400pt}}
\put(170.0,599.0){\rule[-0.200pt]{2.409pt}{0.400pt}}
\put(1429.0,599.0){\rule[-0.200pt]{2.409pt}{0.400pt}}
\put(170.0,599.0){\rule[-0.200pt]{2.409pt}{0.400pt}}
\put(1429.0,599.0){\rule[-0.200pt]{2.409pt}{0.400pt}}
\put(170.0,600.0){\rule[-0.200pt]{2.409pt}{0.400pt}}
\put(1429.0,600.0){\rule[-0.200pt]{2.409pt}{0.400pt}}
\put(170.0,600.0){\rule[-0.200pt]{2.409pt}{0.400pt}}
\put(1429.0,600.0){\rule[-0.200pt]{2.409pt}{0.400pt}}
\put(170.0,600.0){\rule[-0.200pt]{2.409pt}{0.400pt}}
\put(1429.0,600.0){\rule[-0.200pt]{2.409pt}{0.400pt}}
\put(170.0,600.0){\rule[-0.200pt]{2.409pt}{0.400pt}}
\put(1429.0,600.0){\rule[-0.200pt]{2.409pt}{0.400pt}}
\put(170.0,600.0){\rule[-0.200pt]{2.409pt}{0.400pt}}
\put(1429.0,600.0){\rule[-0.200pt]{2.409pt}{0.400pt}}
\put(170.0,600.0){\rule[-0.200pt]{2.409pt}{0.400pt}}
\put(1429.0,600.0){\rule[-0.200pt]{2.409pt}{0.400pt}}
\put(170.0,600.0){\rule[-0.200pt]{2.409pt}{0.400pt}}
\put(1429.0,600.0){\rule[-0.200pt]{2.409pt}{0.400pt}}
\put(170.0,600.0){\rule[-0.200pt]{2.409pt}{0.400pt}}
\put(1429.0,600.0){\rule[-0.200pt]{2.409pt}{0.400pt}}
\put(170.0,600.0){\rule[-0.200pt]{2.409pt}{0.400pt}}
\put(1429.0,600.0){\rule[-0.200pt]{2.409pt}{0.400pt}}
\put(170.0,600.0){\rule[-0.200pt]{4.818pt}{0.400pt}}
\put(150,600){\makebox(0,0)[r]{ 0.001}}
\put(1419.0,600.0){\rule[-0.200pt]{4.818pt}{0.400pt}}
\put(170.0,613.0){\rule[-0.200pt]{2.409pt}{0.400pt}}
\put(1429.0,613.0){\rule[-0.200pt]{2.409pt}{0.400pt}}
\put(170.0,630.0){\rule[-0.200pt]{2.409pt}{0.400pt}}
\put(1429.0,630.0){\rule[-0.200pt]{2.409pt}{0.400pt}}
\put(170.0,639.0){\rule[-0.200pt]{2.409pt}{0.400pt}}
\put(1429.0,639.0){\rule[-0.200pt]{2.409pt}{0.400pt}}
\put(170.0,645.0){\rule[-0.200pt]{2.409pt}{0.400pt}}
\put(1429.0,645.0){\rule[-0.200pt]{2.409pt}{0.400pt}}
\put(170.0,649.0){\rule[-0.200pt]{2.409pt}{0.400pt}}
\put(1429.0,649.0){\rule[-0.200pt]{2.409pt}{0.400pt}}
\put(170.0,653.0){\rule[-0.200pt]{2.409pt}{0.400pt}}
\put(1429.0,653.0){\rule[-0.200pt]{2.409pt}{0.400pt}}
\put(170.0,656.0){\rule[-0.200pt]{2.409pt}{0.400pt}}
\put(1429.0,656.0){\rule[-0.200pt]{2.409pt}{0.400pt}}
\put(170.0,659.0){\rule[-0.200pt]{2.409pt}{0.400pt}}
\put(1429.0,659.0){\rule[-0.200pt]{2.409pt}{0.400pt}}
\put(170.0,661.0){\rule[-0.200pt]{2.409pt}{0.400pt}}
\put(1429.0,661.0){\rule[-0.200pt]{2.409pt}{0.400pt}}
\put(170.0,663.0){\rule[-0.200pt]{2.409pt}{0.400pt}}
\put(1429.0,663.0){\rule[-0.200pt]{2.409pt}{0.400pt}}
\put(170.0,665.0){\rule[-0.200pt]{2.409pt}{0.400pt}}
\put(1429.0,665.0){\rule[-0.200pt]{2.409pt}{0.400pt}}
\put(170.0,667.0){\rule[-0.200pt]{2.409pt}{0.400pt}}
\put(1429.0,667.0){\rule[-0.200pt]{2.409pt}{0.400pt}}
\put(170.0,668.0){\rule[-0.200pt]{2.409pt}{0.400pt}}
\put(1429.0,668.0){\rule[-0.200pt]{2.409pt}{0.400pt}}
\put(170.0,670.0){\rule[-0.200pt]{2.409pt}{0.400pt}}
\put(1429.0,670.0){\rule[-0.200pt]{2.409pt}{0.400pt}}
\put(170.0,671.0){\rule[-0.200pt]{2.409pt}{0.400pt}}
\put(1429.0,671.0){\rule[-0.200pt]{2.409pt}{0.400pt}}
\put(170.0,672.0){\rule[-0.200pt]{2.409pt}{0.400pt}}
\put(1429.0,672.0){\rule[-0.200pt]{2.409pt}{0.400pt}}
\put(170.0,673.0){\rule[-0.200pt]{2.409pt}{0.400pt}}
\put(1429.0,673.0){\rule[-0.200pt]{2.409pt}{0.400pt}}
\put(170.0,674.0){\rule[-0.200pt]{2.409pt}{0.400pt}}
\put(1429.0,674.0){\rule[-0.200pt]{2.409pt}{0.400pt}}
\put(170.0,675.0){\rule[-0.200pt]{2.409pt}{0.400pt}}
\put(1429.0,675.0){\rule[-0.200pt]{2.409pt}{0.400pt}}
\put(170.0,676.0){\rule[-0.200pt]{2.409pt}{0.400pt}}
\put(1429.0,676.0){\rule[-0.200pt]{2.409pt}{0.400pt}}
\put(170.0,677.0){\rule[-0.200pt]{2.409pt}{0.400pt}}
\put(1429.0,677.0){\rule[-0.200pt]{2.409pt}{0.400pt}}
\put(170.0,678.0){\rule[-0.200pt]{2.409pt}{0.400pt}}
\put(1429.0,678.0){\rule[-0.200pt]{2.409pt}{0.400pt}}
\put(170.0,679.0){\rule[-0.200pt]{2.409pt}{0.400pt}}
\put(1429.0,679.0){\rule[-0.200pt]{2.409pt}{0.400pt}}
\put(170.0,680.0){\rule[-0.200pt]{2.409pt}{0.400pt}}
\put(1429.0,680.0){\rule[-0.200pt]{2.409pt}{0.400pt}}
\put(170.0,681.0){\rule[-0.200pt]{2.409pt}{0.400pt}}
\put(1429.0,681.0){\rule[-0.200pt]{2.409pt}{0.400pt}}
\put(170.0,681.0){\rule[-0.200pt]{2.409pt}{0.400pt}}
\put(1429.0,681.0){\rule[-0.200pt]{2.409pt}{0.400pt}}
\put(170.0,682.0){\rule[-0.200pt]{2.409pt}{0.400pt}}
\put(1429.0,682.0){\rule[-0.200pt]{2.409pt}{0.400pt}}
\put(170.0,683.0){\rule[-0.200pt]{2.409pt}{0.400pt}}
\put(1429.0,683.0){\rule[-0.200pt]{2.409pt}{0.400pt}}
\put(170.0,684.0){\rule[-0.200pt]{2.409pt}{0.400pt}}
\put(1429.0,684.0){\rule[-0.200pt]{2.409pt}{0.400pt}}
\put(170.0,684.0){\rule[-0.200pt]{2.409pt}{0.400pt}}
\put(1429.0,684.0){\rule[-0.200pt]{2.409pt}{0.400pt}}
\put(170.0,685.0){\rule[-0.200pt]{2.409pt}{0.400pt}}
\put(1429.0,685.0){\rule[-0.200pt]{2.409pt}{0.400pt}}
\put(170.0,685.0){\rule[-0.200pt]{2.409pt}{0.400pt}}
\put(1429.0,685.0){\rule[-0.200pt]{2.409pt}{0.400pt}}
\put(170.0,686.0){\rule[-0.200pt]{2.409pt}{0.400pt}}
\put(1429.0,686.0){\rule[-0.200pt]{2.409pt}{0.400pt}}
\put(170.0,687.0){\rule[-0.200pt]{2.409pt}{0.400pt}}
\put(1429.0,687.0){\rule[-0.200pt]{2.409pt}{0.400pt}}
\put(170.0,687.0){\rule[-0.200pt]{2.409pt}{0.400pt}}
\put(1429.0,687.0){\rule[-0.200pt]{2.409pt}{0.400pt}}
\put(170.0,688.0){\rule[-0.200pt]{2.409pt}{0.400pt}}
\put(1429.0,688.0){\rule[-0.200pt]{2.409pt}{0.400pt}}
\put(170.0,688.0){\rule[-0.200pt]{2.409pt}{0.400pt}}
\put(1429.0,688.0){\rule[-0.200pt]{2.409pt}{0.400pt}}
\put(170.0,689.0){\rule[-0.200pt]{2.409pt}{0.400pt}}
\put(1429.0,689.0){\rule[-0.200pt]{2.409pt}{0.400pt}}
\put(170.0,689.0){\rule[-0.200pt]{2.409pt}{0.400pt}}
\put(1429.0,689.0){\rule[-0.200pt]{2.409pt}{0.400pt}}
\put(170.0,690.0){\rule[-0.200pt]{2.409pt}{0.400pt}}
\put(1429.0,690.0){\rule[-0.200pt]{2.409pt}{0.400pt}}
\put(170.0,690.0){\rule[-0.200pt]{2.409pt}{0.400pt}}
\put(1429.0,690.0){\rule[-0.200pt]{2.409pt}{0.400pt}}
\put(170.0,691.0){\rule[-0.200pt]{2.409pt}{0.400pt}}
\put(1429.0,691.0){\rule[-0.200pt]{2.409pt}{0.400pt}}
\put(170.0,691.0){\rule[-0.200pt]{2.409pt}{0.400pt}}
\put(1429.0,691.0){\rule[-0.200pt]{2.409pt}{0.400pt}}
\put(170.0,691.0){\rule[-0.200pt]{2.409pt}{0.400pt}}
\put(1429.0,691.0){\rule[-0.200pt]{2.409pt}{0.400pt}}
\put(170.0,692.0){\rule[-0.200pt]{2.409pt}{0.400pt}}
\put(1429.0,692.0){\rule[-0.200pt]{2.409pt}{0.400pt}}
\put(170.0,692.0){\rule[-0.200pt]{2.409pt}{0.400pt}}
\put(1429.0,692.0){\rule[-0.200pt]{2.409pt}{0.400pt}}
\put(170.0,693.0){\rule[-0.200pt]{2.409pt}{0.400pt}}
\put(1429.0,693.0){\rule[-0.200pt]{2.409pt}{0.400pt}}
\put(170.0,693.0){\rule[-0.200pt]{2.409pt}{0.400pt}}
\put(1429.0,693.0){\rule[-0.200pt]{2.409pt}{0.400pt}}
\put(170.0,693.0){\rule[-0.200pt]{2.409pt}{0.400pt}}
\put(1429.0,693.0){\rule[-0.200pt]{2.409pt}{0.400pt}}
\put(170.0,694.0){\rule[-0.200pt]{2.409pt}{0.400pt}}
\put(1429.0,694.0){\rule[-0.200pt]{2.409pt}{0.400pt}}
\put(170.0,694.0){\rule[-0.200pt]{2.409pt}{0.400pt}}
\put(1429.0,694.0){\rule[-0.200pt]{2.409pt}{0.400pt}}
\put(170.0,695.0){\rule[-0.200pt]{2.409pt}{0.400pt}}
\put(1429.0,695.0){\rule[-0.200pt]{2.409pt}{0.400pt}}
\put(170.0,695.0){\rule[-0.200pt]{2.409pt}{0.400pt}}
\put(1429.0,695.0){\rule[-0.200pt]{2.409pt}{0.400pt}}
\put(170.0,695.0){\rule[-0.200pt]{2.409pt}{0.400pt}}
\put(1429.0,695.0){\rule[-0.200pt]{2.409pt}{0.400pt}}
\put(170.0,696.0){\rule[-0.200pt]{2.409pt}{0.400pt}}
\put(1429.0,696.0){\rule[-0.200pt]{2.409pt}{0.400pt}}
\put(170.0,696.0){\rule[-0.200pt]{2.409pt}{0.400pt}}
\put(1429.0,696.0){\rule[-0.200pt]{2.409pt}{0.400pt}}
\put(170.0,696.0){\rule[-0.200pt]{2.409pt}{0.400pt}}
\put(1429.0,696.0){\rule[-0.200pt]{2.409pt}{0.400pt}}
\put(170.0,697.0){\rule[-0.200pt]{2.409pt}{0.400pt}}
\put(1429.0,697.0){\rule[-0.200pt]{2.409pt}{0.400pt}}
\put(170.0,697.0){\rule[-0.200pt]{2.409pt}{0.400pt}}
\put(1429.0,697.0){\rule[-0.200pt]{2.409pt}{0.400pt}}
\put(170.0,697.0){\rule[-0.200pt]{2.409pt}{0.400pt}}
\put(1429.0,697.0){\rule[-0.200pt]{2.409pt}{0.400pt}}
\put(170.0,698.0){\rule[-0.200pt]{2.409pt}{0.400pt}}
\put(1429.0,698.0){\rule[-0.200pt]{2.409pt}{0.400pt}}
\put(170.0,698.0){\rule[-0.200pt]{2.409pt}{0.400pt}}
\put(1429.0,698.0){\rule[-0.200pt]{2.409pt}{0.400pt}}
\put(170.0,698.0){\rule[-0.200pt]{2.409pt}{0.400pt}}
\put(1429.0,698.0){\rule[-0.200pt]{2.409pt}{0.400pt}}
\put(170.0,698.0){\rule[-0.200pt]{2.409pt}{0.400pt}}
\put(1429.0,698.0){\rule[-0.200pt]{2.409pt}{0.400pt}}
\put(170.0,699.0){\rule[-0.200pt]{2.409pt}{0.400pt}}
\put(1429.0,699.0){\rule[-0.200pt]{2.409pt}{0.400pt}}
\put(170.0,699.0){\rule[-0.200pt]{2.409pt}{0.400pt}}
\put(1429.0,699.0){\rule[-0.200pt]{2.409pt}{0.400pt}}
\put(170.0,699.0){\rule[-0.200pt]{2.409pt}{0.400pt}}
\put(1429.0,699.0){\rule[-0.200pt]{2.409pt}{0.400pt}}
\put(170.0,700.0){\rule[-0.200pt]{2.409pt}{0.400pt}}
\put(1429.0,700.0){\rule[-0.200pt]{2.409pt}{0.400pt}}
\put(170.0,700.0){\rule[-0.200pt]{2.409pt}{0.400pt}}
\put(1429.0,700.0){\rule[-0.200pt]{2.409pt}{0.400pt}}
\put(170.0,700.0){\rule[-0.200pt]{2.409pt}{0.400pt}}
\put(1429.0,700.0){\rule[-0.200pt]{2.409pt}{0.400pt}}
\put(170.0,700.0){\rule[-0.200pt]{2.409pt}{0.400pt}}
\put(1429.0,700.0){\rule[-0.200pt]{2.409pt}{0.400pt}}
\put(170.0,701.0){\rule[-0.200pt]{2.409pt}{0.400pt}}
\put(1429.0,701.0){\rule[-0.200pt]{2.409pt}{0.400pt}}
\put(170.0,701.0){\rule[-0.200pt]{2.409pt}{0.400pt}}
\put(1429.0,701.0){\rule[-0.200pt]{2.409pt}{0.400pt}}
\put(170.0,701.0){\rule[-0.200pt]{2.409pt}{0.400pt}}
\put(1429.0,701.0){\rule[-0.200pt]{2.409pt}{0.400pt}}
\put(170.0,701.0){\rule[-0.200pt]{2.409pt}{0.400pt}}
\put(1429.0,701.0){\rule[-0.200pt]{2.409pt}{0.400pt}}
\put(170.0,702.0){\rule[-0.200pt]{2.409pt}{0.400pt}}
\put(1429.0,702.0){\rule[-0.200pt]{2.409pt}{0.400pt}}
\put(170.0,702.0){\rule[-0.200pt]{2.409pt}{0.400pt}}
\put(1429.0,702.0){\rule[-0.200pt]{2.409pt}{0.400pt}}
\put(170.0,702.0){\rule[-0.200pt]{2.409pt}{0.400pt}}
\put(1429.0,702.0){\rule[-0.200pt]{2.409pt}{0.400pt}}
\put(170.0,702.0){\rule[-0.200pt]{2.409pt}{0.400pt}}
\put(1429.0,702.0){\rule[-0.200pt]{2.409pt}{0.400pt}}
\put(170.0,703.0){\rule[-0.200pt]{2.409pt}{0.400pt}}
\put(1429.0,703.0){\rule[-0.200pt]{2.409pt}{0.400pt}}
\put(170.0,703.0){\rule[-0.200pt]{2.409pt}{0.400pt}}
\put(1429.0,703.0){\rule[-0.200pt]{2.409pt}{0.400pt}}
\put(170.0,703.0){\rule[-0.200pt]{2.409pt}{0.400pt}}
\put(1429.0,703.0){\rule[-0.200pt]{2.409pt}{0.400pt}}
\put(170.0,703.0){\rule[-0.200pt]{2.409pt}{0.400pt}}
\put(1429.0,703.0){\rule[-0.200pt]{2.409pt}{0.400pt}}
\put(170.0,704.0){\rule[-0.200pt]{2.409pt}{0.400pt}}
\put(1429.0,704.0){\rule[-0.200pt]{2.409pt}{0.400pt}}
\put(170.0,704.0){\rule[-0.200pt]{2.409pt}{0.400pt}}
\put(1429.0,704.0){\rule[-0.200pt]{2.409pt}{0.400pt}}
\put(170.0,704.0){\rule[-0.200pt]{2.409pt}{0.400pt}}
\put(1429.0,704.0){\rule[-0.200pt]{2.409pt}{0.400pt}}
\put(170.0,704.0){\rule[-0.200pt]{2.409pt}{0.400pt}}
\put(1429.0,704.0){\rule[-0.200pt]{2.409pt}{0.400pt}}
\put(170.0,704.0){\rule[-0.200pt]{2.409pt}{0.400pt}}
\put(1429.0,704.0){\rule[-0.200pt]{2.409pt}{0.400pt}}
\put(170.0,705.0){\rule[-0.200pt]{2.409pt}{0.400pt}}
\put(1429.0,705.0){\rule[-0.200pt]{2.409pt}{0.400pt}}
\put(170.0,705.0){\rule[-0.200pt]{2.409pt}{0.400pt}}
\put(1429.0,705.0){\rule[-0.200pt]{2.409pt}{0.400pt}}
\put(170.0,705.0){\rule[-0.200pt]{2.409pt}{0.400pt}}
\put(1429.0,705.0){\rule[-0.200pt]{2.409pt}{0.400pt}}
\put(170.0,705.0){\rule[-0.200pt]{2.409pt}{0.400pt}}
\put(1429.0,705.0){\rule[-0.200pt]{2.409pt}{0.400pt}}
\put(170.0,706.0){\rule[-0.200pt]{2.409pt}{0.400pt}}
\put(1429.0,706.0){\rule[-0.200pt]{2.409pt}{0.400pt}}
\put(170.0,706.0){\rule[-0.200pt]{2.409pt}{0.400pt}}
\put(1429.0,706.0){\rule[-0.200pt]{2.409pt}{0.400pt}}
\put(170.0,706.0){\rule[-0.200pt]{2.409pt}{0.400pt}}
\put(1429.0,706.0){\rule[-0.200pt]{2.409pt}{0.400pt}}
\put(170.0,706.0){\rule[-0.200pt]{2.409pt}{0.400pt}}
\put(1429.0,706.0){\rule[-0.200pt]{2.409pt}{0.400pt}}
\put(170.0,706.0){\rule[-0.200pt]{2.409pt}{0.400pt}}
\put(1429.0,706.0){\rule[-0.200pt]{2.409pt}{0.400pt}}
\put(170.0,706.0){\rule[-0.200pt]{2.409pt}{0.400pt}}
\put(1429.0,706.0){\rule[-0.200pt]{2.409pt}{0.400pt}}
\put(170.0,707.0){\rule[-0.200pt]{2.409pt}{0.400pt}}
\put(1429.0,707.0){\rule[-0.200pt]{2.409pt}{0.400pt}}
\put(170.0,707.0){\rule[-0.200pt]{2.409pt}{0.400pt}}
\put(1429.0,707.0){\rule[-0.200pt]{2.409pt}{0.400pt}}
\put(170.0,707.0){\rule[-0.200pt]{2.409pt}{0.400pt}}
\put(1429.0,707.0){\rule[-0.200pt]{2.409pt}{0.400pt}}
\put(170.0,707.0){\rule[-0.200pt]{2.409pt}{0.400pt}}
\put(1429.0,707.0){\rule[-0.200pt]{2.409pt}{0.400pt}}
\put(170.0,707.0){\rule[-0.200pt]{2.409pt}{0.400pt}}
\put(1429.0,707.0){\rule[-0.200pt]{2.409pt}{0.400pt}}
\put(170.0,708.0){\rule[-0.200pt]{2.409pt}{0.400pt}}
\put(1429.0,708.0){\rule[-0.200pt]{2.409pt}{0.400pt}}
\put(170.0,708.0){\rule[-0.200pt]{2.409pt}{0.400pt}}
\put(1429.0,708.0){\rule[-0.200pt]{2.409pt}{0.400pt}}
\put(170.0,708.0){\rule[-0.200pt]{2.409pt}{0.400pt}}
\put(1429.0,708.0){\rule[-0.200pt]{2.409pt}{0.400pt}}
\put(170.0,708.0){\rule[-0.200pt]{2.409pt}{0.400pt}}
\put(1429.0,708.0){\rule[-0.200pt]{2.409pt}{0.400pt}}
\put(170.0,708.0){\rule[-0.200pt]{2.409pt}{0.400pt}}
\put(1429.0,708.0){\rule[-0.200pt]{2.409pt}{0.400pt}}
\put(170.0,708.0){\rule[-0.200pt]{2.409pt}{0.400pt}}
\put(1429.0,708.0){\rule[-0.200pt]{2.409pt}{0.400pt}}
\put(170.0,709.0){\rule[-0.200pt]{2.409pt}{0.400pt}}
\put(1429.0,709.0){\rule[-0.200pt]{2.409pt}{0.400pt}}
\put(170.0,709.0){\rule[-0.200pt]{2.409pt}{0.400pt}}
\put(1429.0,709.0){\rule[-0.200pt]{2.409pt}{0.400pt}}
\put(170.0,709.0){\rule[-0.200pt]{2.409pt}{0.400pt}}
\put(1429.0,709.0){\rule[-0.200pt]{2.409pt}{0.400pt}}
\put(170.0,709.0){\rule[-0.200pt]{2.409pt}{0.400pt}}
\put(1429.0,709.0){\rule[-0.200pt]{2.409pt}{0.400pt}}
\put(170.0,709.0){\rule[-0.200pt]{2.409pt}{0.400pt}}
\put(1429.0,709.0){\rule[-0.200pt]{2.409pt}{0.400pt}}
\put(170.0,709.0){\rule[-0.200pt]{2.409pt}{0.400pt}}
\put(1429.0,709.0){\rule[-0.200pt]{2.409pt}{0.400pt}}
\put(170.0,710.0){\rule[-0.200pt]{2.409pt}{0.400pt}}
\put(1429.0,710.0){\rule[-0.200pt]{2.409pt}{0.400pt}}
\put(170.0,710.0){\rule[-0.200pt]{2.409pt}{0.400pt}}
\put(1429.0,710.0){\rule[-0.200pt]{2.409pt}{0.400pt}}
\put(170.0,710.0){\rule[-0.200pt]{2.409pt}{0.400pt}}
\put(1429.0,710.0){\rule[-0.200pt]{2.409pt}{0.400pt}}
\put(170.0,710.0){\rule[-0.200pt]{2.409pt}{0.400pt}}
\put(1429.0,710.0){\rule[-0.200pt]{2.409pt}{0.400pt}}
\put(170.0,710.0){\rule[-0.200pt]{2.409pt}{0.400pt}}
\put(1429.0,710.0){\rule[-0.200pt]{2.409pt}{0.400pt}}
\put(170.0,710.0){\rule[-0.200pt]{2.409pt}{0.400pt}}
\put(1429.0,710.0){\rule[-0.200pt]{2.409pt}{0.400pt}}
\put(170.0,711.0){\rule[-0.200pt]{2.409pt}{0.400pt}}
\put(1429.0,711.0){\rule[-0.200pt]{2.409pt}{0.400pt}}
\put(170.0,711.0){\rule[-0.200pt]{2.409pt}{0.400pt}}
\put(1429.0,711.0){\rule[-0.200pt]{2.409pt}{0.400pt}}
\put(170.0,711.0){\rule[-0.200pt]{2.409pt}{0.400pt}}
\put(1429.0,711.0){\rule[-0.200pt]{2.409pt}{0.400pt}}
\put(170.0,711.0){\rule[-0.200pt]{2.409pt}{0.400pt}}
\put(1429.0,711.0){\rule[-0.200pt]{2.409pt}{0.400pt}}
\put(170.0,711.0){\rule[-0.200pt]{2.409pt}{0.400pt}}
\put(1429.0,711.0){\rule[-0.200pt]{2.409pt}{0.400pt}}
\put(170.0,711.0){\rule[-0.200pt]{2.409pt}{0.400pt}}
\put(1429.0,711.0){\rule[-0.200pt]{2.409pt}{0.400pt}}
\put(170.0,712.0){\rule[-0.200pt]{2.409pt}{0.400pt}}
\put(1429.0,712.0){\rule[-0.200pt]{2.409pt}{0.400pt}}
\put(170.0,712.0){\rule[-0.200pt]{2.409pt}{0.400pt}}
\put(1429.0,712.0){\rule[-0.200pt]{2.409pt}{0.400pt}}
\put(170.0,712.0){\rule[-0.200pt]{2.409pt}{0.400pt}}
\put(1429.0,712.0){\rule[-0.200pt]{2.409pt}{0.400pt}}
\put(170.0,712.0){\rule[-0.200pt]{2.409pt}{0.400pt}}
\put(1429.0,712.0){\rule[-0.200pt]{2.409pt}{0.400pt}}
\put(170.0,712.0){\rule[-0.200pt]{2.409pt}{0.400pt}}
\put(1429.0,712.0){\rule[-0.200pt]{2.409pt}{0.400pt}}
\put(170.0,712.0){\rule[-0.200pt]{2.409pt}{0.400pt}}
\put(1429.0,712.0){\rule[-0.200pt]{2.409pt}{0.400pt}}
\put(170.0,712.0){\rule[-0.200pt]{2.409pt}{0.400pt}}
\put(1429.0,712.0){\rule[-0.200pt]{2.409pt}{0.400pt}}
\put(170.0,713.0){\rule[-0.200pt]{2.409pt}{0.400pt}}
\put(1429.0,713.0){\rule[-0.200pt]{2.409pt}{0.400pt}}
\put(170.0,713.0){\rule[-0.200pt]{2.409pt}{0.400pt}}
\put(1429.0,713.0){\rule[-0.200pt]{2.409pt}{0.400pt}}
\put(170.0,713.0){\rule[-0.200pt]{2.409pt}{0.400pt}}
\put(1429.0,713.0){\rule[-0.200pt]{2.409pt}{0.400pt}}
\put(170.0,713.0){\rule[-0.200pt]{2.409pt}{0.400pt}}
\put(1429.0,713.0){\rule[-0.200pt]{2.409pt}{0.400pt}}
\put(170.0,713.0){\rule[-0.200pt]{2.409pt}{0.400pt}}
\put(1429.0,713.0){\rule[-0.200pt]{2.409pt}{0.400pt}}
\put(170.0,713.0){\rule[-0.200pt]{2.409pt}{0.400pt}}
\put(1429.0,713.0){\rule[-0.200pt]{2.409pt}{0.400pt}}
\put(170.0,713.0){\rule[-0.200pt]{2.409pt}{0.400pt}}
\put(1429.0,713.0){\rule[-0.200pt]{2.409pt}{0.400pt}}
\put(170.0,713.0){\rule[-0.200pt]{2.409pt}{0.400pt}}
\put(1429.0,713.0){\rule[-0.200pt]{2.409pt}{0.400pt}}
\put(170.0,714.0){\rule[-0.200pt]{2.409pt}{0.400pt}}
\put(1429.0,714.0){\rule[-0.200pt]{2.409pt}{0.400pt}}
\put(170.0,714.0){\rule[-0.200pt]{2.409pt}{0.400pt}}
\put(1429.0,714.0){\rule[-0.200pt]{2.409pt}{0.400pt}}
\put(170.0,714.0){\rule[-0.200pt]{2.409pt}{0.400pt}}
\put(1429.0,714.0){\rule[-0.200pt]{2.409pt}{0.400pt}}
\put(170.0,714.0){\rule[-0.200pt]{2.409pt}{0.400pt}}
\put(1429.0,714.0){\rule[-0.200pt]{2.409pt}{0.400pt}}
\put(170.0,714.0){\rule[-0.200pt]{2.409pt}{0.400pt}}
\put(1429.0,714.0){\rule[-0.200pt]{2.409pt}{0.400pt}}
\put(170.0,714.0){\rule[-0.200pt]{2.409pt}{0.400pt}}
\put(1429.0,714.0){\rule[-0.200pt]{2.409pt}{0.400pt}}
\put(170.0,714.0){\rule[-0.200pt]{2.409pt}{0.400pt}}
\put(1429.0,714.0){\rule[-0.200pt]{2.409pt}{0.400pt}}
\put(170.0,714.0){\rule[-0.200pt]{2.409pt}{0.400pt}}
\put(1429.0,714.0){\rule[-0.200pt]{2.409pt}{0.400pt}}
\put(170.0,715.0){\rule[-0.200pt]{2.409pt}{0.400pt}}
\put(1429.0,715.0){\rule[-0.200pt]{2.409pt}{0.400pt}}
\put(170.0,715.0){\rule[-0.200pt]{2.409pt}{0.400pt}}
\put(1429.0,715.0){\rule[-0.200pt]{2.409pt}{0.400pt}}
\put(170.0,715.0){\rule[-0.200pt]{2.409pt}{0.400pt}}
\put(1429.0,715.0){\rule[-0.200pt]{2.409pt}{0.400pt}}
\put(170.0,715.0){\rule[-0.200pt]{2.409pt}{0.400pt}}
\put(1429.0,715.0){\rule[-0.200pt]{2.409pt}{0.400pt}}
\put(170.0,715.0){\rule[-0.200pt]{2.409pt}{0.400pt}}
\put(1429.0,715.0){\rule[-0.200pt]{2.409pt}{0.400pt}}
\put(170.0,715.0){\rule[-0.200pt]{2.409pt}{0.400pt}}
\put(1429.0,715.0){\rule[-0.200pt]{2.409pt}{0.400pt}}
\put(170.0,715.0){\rule[-0.200pt]{2.409pt}{0.400pt}}
\put(1429.0,715.0){\rule[-0.200pt]{2.409pt}{0.400pt}}
\put(170.0,715.0){\rule[-0.200pt]{2.409pt}{0.400pt}}
\put(1429.0,715.0){\rule[-0.200pt]{2.409pt}{0.400pt}}
\put(170.0,716.0){\rule[-0.200pt]{2.409pt}{0.400pt}}
\put(1429.0,716.0){\rule[-0.200pt]{2.409pt}{0.400pt}}
\put(170.0,716.0){\rule[-0.200pt]{2.409pt}{0.400pt}}
\put(1429.0,716.0){\rule[-0.200pt]{2.409pt}{0.400pt}}
\put(170.0,716.0){\rule[-0.200pt]{2.409pt}{0.400pt}}
\put(1429.0,716.0){\rule[-0.200pt]{2.409pt}{0.400pt}}
\put(170.0,716.0){\rule[-0.200pt]{2.409pt}{0.400pt}}
\put(1429.0,716.0){\rule[-0.200pt]{2.409pt}{0.400pt}}
\put(170.0,716.0){\rule[-0.200pt]{2.409pt}{0.400pt}}
\put(1429.0,716.0){\rule[-0.200pt]{2.409pt}{0.400pt}}
\put(170.0,716.0){\rule[-0.200pt]{2.409pt}{0.400pt}}
\put(1429.0,716.0){\rule[-0.200pt]{2.409pt}{0.400pt}}
\put(170.0,716.0){\rule[-0.200pt]{2.409pt}{0.400pt}}
\put(1429.0,716.0){\rule[-0.200pt]{2.409pt}{0.400pt}}
\put(170.0,716.0){\rule[-0.200pt]{2.409pt}{0.400pt}}
\put(1429.0,716.0){\rule[-0.200pt]{2.409pt}{0.400pt}}
\put(170.0,717.0){\rule[-0.200pt]{2.409pt}{0.400pt}}
\put(1429.0,717.0){\rule[-0.200pt]{2.409pt}{0.400pt}}
\put(170.0,717.0){\rule[-0.200pt]{2.409pt}{0.400pt}}
\put(1429.0,717.0){\rule[-0.200pt]{2.409pt}{0.400pt}}
\put(170.0,717.0){\rule[-0.200pt]{2.409pt}{0.400pt}}
\put(1429.0,717.0){\rule[-0.200pt]{2.409pt}{0.400pt}}
\put(170.0,717.0){\rule[-0.200pt]{2.409pt}{0.400pt}}
\put(1429.0,717.0){\rule[-0.200pt]{2.409pt}{0.400pt}}
\put(170.0,717.0){\rule[-0.200pt]{2.409pt}{0.400pt}}
\put(1429.0,717.0){\rule[-0.200pt]{2.409pt}{0.400pt}}
\put(170.0,717.0){\rule[-0.200pt]{2.409pt}{0.400pt}}
\put(1429.0,717.0){\rule[-0.200pt]{2.409pt}{0.400pt}}
\put(170.0,717.0){\rule[-0.200pt]{2.409pt}{0.400pt}}
\put(1429.0,717.0){\rule[-0.200pt]{2.409pt}{0.400pt}}
\put(170.0,717.0){\rule[-0.200pt]{2.409pt}{0.400pt}}
\put(1429.0,717.0){\rule[-0.200pt]{2.409pt}{0.400pt}}
\put(170.0,717.0){\rule[-0.200pt]{2.409pt}{0.400pt}}
\put(1429.0,717.0){\rule[-0.200pt]{2.409pt}{0.400pt}}
\put(170.0,717.0){\rule[-0.200pt]{2.409pt}{0.400pt}}
\put(1429.0,717.0){\rule[-0.200pt]{2.409pt}{0.400pt}}
\put(170.0,718.0){\rule[-0.200pt]{2.409pt}{0.400pt}}
\put(1429.0,718.0){\rule[-0.200pt]{2.409pt}{0.400pt}}
\put(170.0,718.0){\rule[-0.200pt]{2.409pt}{0.400pt}}
\put(1429.0,718.0){\rule[-0.200pt]{2.409pt}{0.400pt}}
\put(170.0,718.0){\rule[-0.200pt]{2.409pt}{0.400pt}}
\put(1429.0,718.0){\rule[-0.200pt]{2.409pt}{0.400pt}}
\put(170.0,718.0){\rule[-0.200pt]{2.409pt}{0.400pt}}
\put(1429.0,718.0){\rule[-0.200pt]{2.409pt}{0.400pt}}
\put(170.0,718.0){\rule[-0.200pt]{2.409pt}{0.400pt}}
\put(1429.0,718.0){\rule[-0.200pt]{2.409pt}{0.400pt}}
\put(170.0,718.0){\rule[-0.200pt]{2.409pt}{0.400pt}}
\put(1429.0,718.0){\rule[-0.200pt]{2.409pt}{0.400pt}}
\put(170.0,718.0){\rule[-0.200pt]{2.409pt}{0.400pt}}
\put(1429.0,718.0){\rule[-0.200pt]{2.409pt}{0.400pt}}
\put(170.0,718.0){\rule[-0.200pt]{2.409pt}{0.400pt}}
\put(1429.0,718.0){\rule[-0.200pt]{2.409pt}{0.400pt}}
\put(170.0,718.0){\rule[-0.200pt]{2.409pt}{0.400pt}}
\put(1429.0,718.0){\rule[-0.200pt]{2.409pt}{0.400pt}}
\put(170.0,719.0){\rule[-0.200pt]{2.409pt}{0.400pt}}
\put(1429.0,719.0){\rule[-0.200pt]{2.409pt}{0.400pt}}
\put(170.0,719.0){\rule[-0.200pt]{2.409pt}{0.400pt}}
\put(1429.0,719.0){\rule[-0.200pt]{2.409pt}{0.400pt}}
\put(170.0,719.0){\rule[-0.200pt]{2.409pt}{0.400pt}}
\put(1429.0,719.0){\rule[-0.200pt]{2.409pt}{0.400pt}}
\put(170.0,719.0){\rule[-0.200pt]{2.409pt}{0.400pt}}
\put(1429.0,719.0){\rule[-0.200pt]{2.409pt}{0.400pt}}
\put(170.0,719.0){\rule[-0.200pt]{2.409pt}{0.400pt}}
\put(1429.0,719.0){\rule[-0.200pt]{2.409pt}{0.400pt}}
\put(170.0,719.0){\rule[-0.200pt]{2.409pt}{0.400pt}}
\put(1429.0,719.0){\rule[-0.200pt]{2.409pt}{0.400pt}}
\put(170.0,719.0){\rule[-0.200pt]{2.409pt}{0.400pt}}
\put(1429.0,719.0){\rule[-0.200pt]{2.409pt}{0.400pt}}
\put(170.0,719.0){\rule[-0.200pt]{2.409pt}{0.400pt}}
\put(1429.0,719.0){\rule[-0.200pt]{2.409pt}{0.400pt}}
\put(170.0,719.0){\rule[-0.200pt]{2.409pt}{0.400pt}}
\put(1429.0,719.0){\rule[-0.200pt]{2.409pt}{0.400pt}}
\put(170.0,719.0){\rule[-0.200pt]{2.409pt}{0.400pt}}
\put(1429.0,719.0){\rule[-0.200pt]{2.409pt}{0.400pt}}
\put(170.0,720.0){\rule[-0.200pt]{2.409pt}{0.400pt}}
\put(1429.0,720.0){\rule[-0.200pt]{2.409pt}{0.400pt}}
\put(170.0,720.0){\rule[-0.200pt]{2.409pt}{0.400pt}}
\put(1429.0,720.0){\rule[-0.200pt]{2.409pt}{0.400pt}}
\put(170.0,720.0){\rule[-0.200pt]{2.409pt}{0.400pt}}
\put(1429.0,720.0){\rule[-0.200pt]{2.409pt}{0.400pt}}
\put(170.0,720.0){\rule[-0.200pt]{2.409pt}{0.400pt}}
\put(1429.0,720.0){\rule[-0.200pt]{2.409pt}{0.400pt}}
\put(170.0,720.0){\rule[-0.200pt]{2.409pt}{0.400pt}}
\put(1429.0,720.0){\rule[-0.200pt]{2.409pt}{0.400pt}}
\put(170.0,720.0){\rule[-0.200pt]{2.409pt}{0.400pt}}
\put(1429.0,720.0){\rule[-0.200pt]{2.409pt}{0.400pt}}
\put(170.0,720.0){\rule[-0.200pt]{2.409pt}{0.400pt}}
\put(1429.0,720.0){\rule[-0.200pt]{2.409pt}{0.400pt}}
\put(170.0,720.0){\rule[-0.200pt]{2.409pt}{0.400pt}}
\put(1429.0,720.0){\rule[-0.200pt]{2.409pt}{0.400pt}}
\put(170.0,720.0){\rule[-0.200pt]{2.409pt}{0.400pt}}
\put(1429.0,720.0){\rule[-0.200pt]{2.409pt}{0.400pt}}
\put(170.0,720.0){\rule[-0.200pt]{2.409pt}{0.400pt}}
\put(1429.0,720.0){\rule[-0.200pt]{2.409pt}{0.400pt}}
\put(170.0,720.0){\rule[-0.200pt]{2.409pt}{0.400pt}}
\put(1429.0,720.0){\rule[-0.200pt]{2.409pt}{0.400pt}}
\put(170.0,721.0){\rule[-0.200pt]{2.409pt}{0.400pt}}
\put(1429.0,721.0){\rule[-0.200pt]{2.409pt}{0.400pt}}
\put(170.0,721.0){\rule[-0.200pt]{2.409pt}{0.400pt}}
\put(1429.0,721.0){\rule[-0.200pt]{2.409pt}{0.400pt}}
\put(170.0,721.0){\rule[-0.200pt]{2.409pt}{0.400pt}}
\put(1429.0,721.0){\rule[-0.200pt]{2.409pt}{0.400pt}}
\put(170.0,721.0){\rule[-0.200pt]{2.409pt}{0.400pt}}
\put(1429.0,721.0){\rule[-0.200pt]{2.409pt}{0.400pt}}
\put(170.0,721.0){\rule[-0.200pt]{2.409pt}{0.400pt}}
\put(1429.0,721.0){\rule[-0.200pt]{2.409pt}{0.400pt}}
\put(170.0,721.0){\rule[-0.200pt]{2.409pt}{0.400pt}}
\put(1429.0,721.0){\rule[-0.200pt]{2.409pt}{0.400pt}}
\put(170.0,721.0){\rule[-0.200pt]{2.409pt}{0.400pt}}
\put(1429.0,721.0){\rule[-0.200pt]{2.409pt}{0.400pt}}
\put(170.0,721.0){\rule[-0.200pt]{2.409pt}{0.400pt}}
\put(1429.0,721.0){\rule[-0.200pt]{2.409pt}{0.400pt}}
\put(170.0,721.0){\rule[-0.200pt]{2.409pt}{0.400pt}}
\put(1429.0,721.0){\rule[-0.200pt]{2.409pt}{0.400pt}}
\put(170.0,721.0){\rule[-0.200pt]{2.409pt}{0.400pt}}
\put(1429.0,721.0){\rule[-0.200pt]{2.409pt}{0.400pt}}
\put(170.0,721.0){\rule[-0.200pt]{2.409pt}{0.400pt}}
\put(1429.0,721.0){\rule[-0.200pt]{2.409pt}{0.400pt}}
\put(170.0,722.0){\rule[-0.200pt]{2.409pt}{0.400pt}}
\put(1429.0,722.0){\rule[-0.200pt]{2.409pt}{0.400pt}}
\put(170.0,722.0){\rule[-0.200pt]{2.409pt}{0.400pt}}
\put(1429.0,722.0){\rule[-0.200pt]{2.409pt}{0.400pt}}
\put(170.0,722.0){\rule[-0.200pt]{2.409pt}{0.400pt}}
\put(1429.0,722.0){\rule[-0.200pt]{2.409pt}{0.400pt}}
\put(170.0,722.0){\rule[-0.200pt]{2.409pt}{0.400pt}}
\put(1429.0,722.0){\rule[-0.200pt]{2.409pt}{0.400pt}}
\put(170.0,722.0){\rule[-0.200pt]{2.409pt}{0.400pt}}
\put(1429.0,722.0){\rule[-0.200pt]{2.409pt}{0.400pt}}
\put(170.0,722.0){\rule[-0.200pt]{2.409pt}{0.400pt}}
\put(1429.0,722.0){\rule[-0.200pt]{2.409pt}{0.400pt}}
\put(170.0,722.0){\rule[-0.200pt]{2.409pt}{0.400pt}}
\put(1429.0,722.0){\rule[-0.200pt]{2.409pt}{0.400pt}}
\put(170.0,722.0){\rule[-0.200pt]{2.409pt}{0.400pt}}
\put(1429.0,722.0){\rule[-0.200pt]{2.409pt}{0.400pt}}
\put(170.0,722.0){\rule[-0.200pt]{2.409pt}{0.400pt}}
\put(1429.0,722.0){\rule[-0.200pt]{2.409pt}{0.400pt}}
\put(170.0,722.0){\rule[-0.200pt]{2.409pt}{0.400pt}}
\put(1429.0,722.0){\rule[-0.200pt]{2.409pt}{0.400pt}}
\put(170.0,722.0){\rule[-0.200pt]{2.409pt}{0.400pt}}
\put(1429.0,722.0){\rule[-0.200pt]{2.409pt}{0.400pt}}
\put(170.0,722.0){\rule[-0.200pt]{2.409pt}{0.400pt}}
\put(1429.0,722.0){\rule[-0.200pt]{2.409pt}{0.400pt}}
\put(170.0,723.0){\rule[-0.200pt]{2.409pt}{0.400pt}}
\put(1429.0,723.0){\rule[-0.200pt]{2.409pt}{0.400pt}}
\put(170.0,723.0){\rule[-0.200pt]{2.409pt}{0.400pt}}
\put(1429.0,723.0){\rule[-0.200pt]{2.409pt}{0.400pt}}
\put(170.0,723.0){\rule[-0.200pt]{2.409pt}{0.400pt}}
\put(1429.0,723.0){\rule[-0.200pt]{2.409pt}{0.400pt}}
\put(170.0,723.0){\rule[-0.200pt]{2.409pt}{0.400pt}}
\put(1429.0,723.0){\rule[-0.200pt]{2.409pt}{0.400pt}}
\put(170.0,723.0){\rule[-0.200pt]{2.409pt}{0.400pt}}
\put(1429.0,723.0){\rule[-0.200pt]{2.409pt}{0.400pt}}
\put(170.0,723.0){\rule[-0.200pt]{2.409pt}{0.400pt}}
\put(1429.0,723.0){\rule[-0.200pt]{2.409pt}{0.400pt}}
\put(170.0,723.0){\rule[-0.200pt]{2.409pt}{0.400pt}}
\put(1429.0,723.0){\rule[-0.200pt]{2.409pt}{0.400pt}}
\put(170.0,723.0){\rule[-0.200pt]{2.409pt}{0.400pt}}
\put(1429.0,723.0){\rule[-0.200pt]{2.409pt}{0.400pt}}
\put(170.0,723.0){\rule[-0.200pt]{2.409pt}{0.400pt}}
\put(1429.0,723.0){\rule[-0.200pt]{2.409pt}{0.400pt}}
\put(170.0,723.0){\rule[-0.200pt]{2.409pt}{0.400pt}}
\put(1429.0,723.0){\rule[-0.200pt]{2.409pt}{0.400pt}}
\put(170.0,723.0){\rule[-0.200pt]{2.409pt}{0.400pt}}
\put(1429.0,723.0){\rule[-0.200pt]{2.409pt}{0.400pt}}
\put(170.0,723.0){\rule[-0.200pt]{2.409pt}{0.400pt}}
\put(1429.0,723.0){\rule[-0.200pt]{2.409pt}{0.400pt}}
\put(170.0,723.0){\rule[-0.200pt]{2.409pt}{0.400pt}}
\put(1429.0,723.0){\rule[-0.200pt]{2.409pt}{0.400pt}}
\put(170.0,724.0){\rule[-0.200pt]{2.409pt}{0.400pt}}
\put(1429.0,724.0){\rule[-0.200pt]{2.409pt}{0.400pt}}
\put(170.0,724.0){\rule[-0.200pt]{2.409pt}{0.400pt}}
\put(1429.0,724.0){\rule[-0.200pt]{2.409pt}{0.400pt}}
\put(170.0,724.0){\rule[-0.200pt]{2.409pt}{0.400pt}}
\put(1429.0,724.0){\rule[-0.200pt]{2.409pt}{0.400pt}}
\put(170.0,724.0){\rule[-0.200pt]{2.409pt}{0.400pt}}
\put(1429.0,724.0){\rule[-0.200pt]{2.409pt}{0.400pt}}
\put(170.0,724.0){\rule[-0.200pt]{2.409pt}{0.400pt}}
\put(1429.0,724.0){\rule[-0.200pt]{2.409pt}{0.400pt}}
\put(170.0,724.0){\rule[-0.200pt]{2.409pt}{0.400pt}}
\put(1429.0,724.0){\rule[-0.200pt]{2.409pt}{0.400pt}}
\put(170.0,724.0){\rule[-0.200pt]{2.409pt}{0.400pt}}
\put(1429.0,724.0){\rule[-0.200pt]{2.409pt}{0.400pt}}
\put(170.0,724.0){\rule[-0.200pt]{2.409pt}{0.400pt}}
\put(1429.0,724.0){\rule[-0.200pt]{2.409pt}{0.400pt}}
\put(170.0,724.0){\rule[-0.200pt]{2.409pt}{0.400pt}}
\put(1429.0,724.0){\rule[-0.200pt]{2.409pt}{0.400pt}}
\put(170.0,724.0){\rule[-0.200pt]{2.409pt}{0.400pt}}
\put(1429.0,724.0){\rule[-0.200pt]{2.409pt}{0.400pt}}
\put(170.0,724.0){\rule[-0.200pt]{2.409pt}{0.400pt}}
\put(1429.0,724.0){\rule[-0.200pt]{2.409pt}{0.400pt}}
\put(170.0,724.0){\rule[-0.200pt]{2.409pt}{0.400pt}}
\put(1429.0,724.0){\rule[-0.200pt]{2.409pt}{0.400pt}}
\put(170.0,724.0){\rule[-0.200pt]{2.409pt}{0.400pt}}
\put(1429.0,724.0){\rule[-0.200pt]{2.409pt}{0.400pt}}
\put(170.0,725.0){\rule[-0.200pt]{2.409pt}{0.400pt}}
\put(1429.0,725.0){\rule[-0.200pt]{2.409pt}{0.400pt}}
\put(170.0,725.0){\rule[-0.200pt]{2.409pt}{0.400pt}}
\put(1429.0,725.0){\rule[-0.200pt]{2.409pt}{0.400pt}}
\put(170.0,725.0){\rule[-0.200pt]{2.409pt}{0.400pt}}
\put(1429.0,725.0){\rule[-0.200pt]{2.409pt}{0.400pt}}
\put(170.0,725.0){\rule[-0.200pt]{2.409pt}{0.400pt}}
\put(1429.0,725.0){\rule[-0.200pt]{2.409pt}{0.400pt}}
\put(170.0,725.0){\rule[-0.200pt]{2.409pt}{0.400pt}}
\put(1429.0,725.0){\rule[-0.200pt]{2.409pt}{0.400pt}}
\put(170.0,725.0){\rule[-0.200pt]{2.409pt}{0.400pt}}
\put(1429.0,725.0){\rule[-0.200pt]{2.409pt}{0.400pt}}
\put(170.0,725.0){\rule[-0.200pt]{2.409pt}{0.400pt}}
\put(1429.0,725.0){\rule[-0.200pt]{2.409pt}{0.400pt}}
\put(170.0,725.0){\rule[-0.200pt]{2.409pt}{0.400pt}}
\put(1429.0,725.0){\rule[-0.200pt]{2.409pt}{0.400pt}}
\put(170.0,725.0){\rule[-0.200pt]{2.409pt}{0.400pt}}
\put(1429.0,725.0){\rule[-0.200pt]{2.409pt}{0.400pt}}
\put(170.0,725.0){\rule[-0.200pt]{2.409pt}{0.400pt}}
\put(1429.0,725.0){\rule[-0.200pt]{2.409pt}{0.400pt}}
\put(170.0,725.0){\rule[-0.200pt]{2.409pt}{0.400pt}}
\put(1429.0,725.0){\rule[-0.200pt]{2.409pt}{0.400pt}}
\put(170.0,725.0){\rule[-0.200pt]{2.409pt}{0.400pt}}
\put(1429.0,725.0){\rule[-0.200pt]{2.409pt}{0.400pt}}
\put(170.0,725.0){\rule[-0.200pt]{2.409pt}{0.400pt}}
\put(1429.0,725.0){\rule[-0.200pt]{2.409pt}{0.400pt}}
\put(170.0,725.0){\rule[-0.200pt]{2.409pt}{0.400pt}}
\put(1429.0,725.0){\rule[-0.200pt]{2.409pt}{0.400pt}}
\put(170.0,726.0){\rule[-0.200pt]{2.409pt}{0.400pt}}
\put(1429.0,726.0){\rule[-0.200pt]{2.409pt}{0.400pt}}
\put(170.0,726.0){\rule[-0.200pt]{2.409pt}{0.400pt}}
\put(1429.0,726.0){\rule[-0.200pt]{2.409pt}{0.400pt}}
\put(170.0,726.0){\rule[-0.200pt]{2.409pt}{0.400pt}}
\put(1429.0,726.0){\rule[-0.200pt]{2.409pt}{0.400pt}}
\put(170.0,726.0){\rule[-0.200pt]{2.409pt}{0.400pt}}
\put(1429.0,726.0){\rule[-0.200pt]{2.409pt}{0.400pt}}
\put(170.0,726.0){\rule[-0.200pt]{2.409pt}{0.400pt}}
\put(1429.0,726.0){\rule[-0.200pt]{2.409pt}{0.400pt}}
\put(170.0,726.0){\rule[-0.200pt]{2.409pt}{0.400pt}}
\put(1429.0,726.0){\rule[-0.200pt]{2.409pt}{0.400pt}}
\put(170.0,726.0){\rule[-0.200pt]{2.409pt}{0.400pt}}
\put(1429.0,726.0){\rule[-0.200pt]{2.409pt}{0.400pt}}
\put(170.0,726.0){\rule[-0.200pt]{2.409pt}{0.400pt}}
\put(1429.0,726.0){\rule[-0.200pt]{2.409pt}{0.400pt}}
\put(170.0,726.0){\rule[-0.200pt]{2.409pt}{0.400pt}}
\put(1429.0,726.0){\rule[-0.200pt]{2.409pt}{0.400pt}}
\put(170.0,726.0){\rule[-0.200pt]{2.409pt}{0.400pt}}
\put(1429.0,726.0){\rule[-0.200pt]{2.409pt}{0.400pt}}
\put(170.0,726.0){\rule[-0.200pt]{2.409pt}{0.400pt}}
\put(1429.0,726.0){\rule[-0.200pt]{2.409pt}{0.400pt}}
\put(170.0,726.0){\rule[-0.200pt]{2.409pt}{0.400pt}}
\put(1429.0,726.0){\rule[-0.200pt]{2.409pt}{0.400pt}}
\put(170.0,726.0){\rule[-0.200pt]{2.409pt}{0.400pt}}
\put(1429.0,726.0){\rule[-0.200pt]{2.409pt}{0.400pt}}
\put(170.0,726.0){\rule[-0.200pt]{2.409pt}{0.400pt}}
\put(1429.0,726.0){\rule[-0.200pt]{2.409pt}{0.400pt}}
\put(170.0,726.0){\rule[-0.200pt]{2.409pt}{0.400pt}}
\put(1429.0,726.0){\rule[-0.200pt]{2.409pt}{0.400pt}}
\put(170.0,727.0){\rule[-0.200pt]{2.409pt}{0.400pt}}
\put(1429.0,727.0){\rule[-0.200pt]{2.409pt}{0.400pt}}
\put(170.0,727.0){\rule[-0.200pt]{2.409pt}{0.400pt}}
\put(1429.0,727.0){\rule[-0.200pt]{2.409pt}{0.400pt}}
\put(170.0,727.0){\rule[-0.200pt]{2.409pt}{0.400pt}}
\put(1429.0,727.0){\rule[-0.200pt]{2.409pt}{0.400pt}}
\put(170.0,727.0){\rule[-0.200pt]{2.409pt}{0.400pt}}
\put(1429.0,727.0){\rule[-0.200pt]{2.409pt}{0.400pt}}
\put(170.0,727.0){\rule[-0.200pt]{2.409pt}{0.400pt}}
\put(1429.0,727.0){\rule[-0.200pt]{2.409pt}{0.400pt}}
\put(170.0,727.0){\rule[-0.200pt]{2.409pt}{0.400pt}}
\put(1429.0,727.0){\rule[-0.200pt]{2.409pt}{0.400pt}}
\put(170.0,727.0){\rule[-0.200pt]{2.409pt}{0.400pt}}
\put(1429.0,727.0){\rule[-0.200pt]{2.409pt}{0.400pt}}
\put(170.0,727.0){\rule[-0.200pt]{2.409pt}{0.400pt}}
\put(1429.0,727.0){\rule[-0.200pt]{2.409pt}{0.400pt}}
\put(170.0,727.0){\rule[-0.200pt]{2.409pt}{0.400pt}}
\put(1429.0,727.0){\rule[-0.200pt]{2.409pt}{0.400pt}}
\put(170.0,727.0){\rule[-0.200pt]{2.409pt}{0.400pt}}
\put(1429.0,727.0){\rule[-0.200pt]{2.409pt}{0.400pt}}
\put(170.0,727.0){\rule[-0.200pt]{2.409pt}{0.400pt}}
\put(1429.0,727.0){\rule[-0.200pt]{2.409pt}{0.400pt}}
\put(170.0,727.0){\rule[-0.200pt]{2.409pt}{0.400pt}}
\put(1429.0,727.0){\rule[-0.200pt]{2.409pt}{0.400pt}}
\put(170.0,727.0){\rule[-0.200pt]{2.409pt}{0.400pt}}
\put(1429.0,727.0){\rule[-0.200pt]{2.409pt}{0.400pt}}
\put(170.0,727.0){\rule[-0.200pt]{2.409pt}{0.400pt}}
\put(1429.0,727.0){\rule[-0.200pt]{2.409pt}{0.400pt}}
\put(170.0,727.0){\rule[-0.200pt]{2.409pt}{0.400pt}}
\put(1429.0,727.0){\rule[-0.200pt]{2.409pt}{0.400pt}}
\put(170.0,728.0){\rule[-0.200pt]{2.409pt}{0.400pt}}
\put(1429.0,728.0){\rule[-0.200pt]{2.409pt}{0.400pt}}
\put(170.0,728.0){\rule[-0.200pt]{2.409pt}{0.400pt}}
\put(1429.0,728.0){\rule[-0.200pt]{2.409pt}{0.400pt}}
\put(170.0,728.0){\rule[-0.200pt]{2.409pt}{0.400pt}}
\put(1429.0,728.0){\rule[-0.200pt]{2.409pt}{0.400pt}}
\put(170.0,728.0){\rule[-0.200pt]{2.409pt}{0.400pt}}
\put(1429.0,728.0){\rule[-0.200pt]{2.409pt}{0.400pt}}
\put(170.0,728.0){\rule[-0.200pt]{2.409pt}{0.400pt}}
\put(1429.0,728.0){\rule[-0.200pt]{2.409pt}{0.400pt}}
\put(170.0,728.0){\rule[-0.200pt]{2.409pt}{0.400pt}}
\put(1429.0,728.0){\rule[-0.200pt]{2.409pt}{0.400pt}}
\put(170.0,728.0){\rule[-0.200pt]{2.409pt}{0.400pt}}
\put(1429.0,728.0){\rule[-0.200pt]{2.409pt}{0.400pt}}
\put(170.0,728.0){\rule[-0.200pt]{2.409pt}{0.400pt}}
\put(1429.0,728.0){\rule[-0.200pt]{2.409pt}{0.400pt}}
\put(170.0,728.0){\rule[-0.200pt]{2.409pt}{0.400pt}}
\put(1429.0,728.0){\rule[-0.200pt]{2.409pt}{0.400pt}}
\put(170.0,728.0){\rule[-0.200pt]{2.409pt}{0.400pt}}
\put(1429.0,728.0){\rule[-0.200pt]{2.409pt}{0.400pt}}
\put(170.0,728.0){\rule[-0.200pt]{2.409pt}{0.400pt}}
\put(1429.0,728.0){\rule[-0.200pt]{2.409pt}{0.400pt}}
\put(170.0,728.0){\rule[-0.200pt]{2.409pt}{0.400pt}}
\put(1429.0,728.0){\rule[-0.200pt]{2.409pt}{0.400pt}}
\put(170.0,728.0){\rule[-0.200pt]{2.409pt}{0.400pt}}
\put(1429.0,728.0){\rule[-0.200pt]{2.409pt}{0.400pt}}
\put(170.0,728.0){\rule[-0.200pt]{2.409pt}{0.400pt}}
\put(1429.0,728.0){\rule[-0.200pt]{2.409pt}{0.400pt}}
\put(170.0,728.0){\rule[-0.200pt]{2.409pt}{0.400pt}}
\put(1429.0,728.0){\rule[-0.200pt]{2.409pt}{0.400pt}}
\put(170.0,728.0){\rule[-0.200pt]{2.409pt}{0.400pt}}
\put(1429.0,728.0){\rule[-0.200pt]{2.409pt}{0.400pt}}
\put(170.0,728.0){\rule[-0.200pt]{2.409pt}{0.400pt}}
\put(1429.0,728.0){\rule[-0.200pt]{2.409pt}{0.400pt}}
\put(170.0,729.0){\rule[-0.200pt]{2.409pt}{0.400pt}}
\put(1429.0,729.0){\rule[-0.200pt]{2.409pt}{0.400pt}}
\put(170.0,729.0){\rule[-0.200pt]{2.409pt}{0.400pt}}
\put(1429.0,729.0){\rule[-0.200pt]{2.409pt}{0.400pt}}
\put(170.0,729.0){\rule[-0.200pt]{2.409pt}{0.400pt}}
\put(1429.0,729.0){\rule[-0.200pt]{2.409pt}{0.400pt}}
\put(170.0,729.0){\rule[-0.200pt]{2.409pt}{0.400pt}}
\put(1429.0,729.0){\rule[-0.200pt]{2.409pt}{0.400pt}}
\put(170.0,729.0){\rule[-0.200pt]{2.409pt}{0.400pt}}
\put(1429.0,729.0){\rule[-0.200pt]{2.409pt}{0.400pt}}
\put(170.0,729.0){\rule[-0.200pt]{2.409pt}{0.400pt}}
\put(1429.0,729.0){\rule[-0.200pt]{2.409pt}{0.400pt}}
\put(170.0,729.0){\rule[-0.200pt]{2.409pt}{0.400pt}}
\put(1429.0,729.0){\rule[-0.200pt]{2.409pt}{0.400pt}}
\put(170.0,729.0){\rule[-0.200pt]{2.409pt}{0.400pt}}
\put(1429.0,729.0){\rule[-0.200pt]{2.409pt}{0.400pt}}
\put(170.0,729.0){\rule[-0.200pt]{2.409pt}{0.400pt}}
\put(1429.0,729.0){\rule[-0.200pt]{2.409pt}{0.400pt}}
\put(170.0,729.0){\rule[-0.200pt]{2.409pt}{0.400pt}}
\put(1429.0,729.0){\rule[-0.200pt]{2.409pt}{0.400pt}}
\put(170.0,729.0){\rule[-0.200pt]{2.409pt}{0.400pt}}
\put(1429.0,729.0){\rule[-0.200pt]{2.409pt}{0.400pt}}
\put(170.0,729.0){\rule[-0.200pt]{2.409pt}{0.400pt}}
\put(1429.0,729.0){\rule[-0.200pt]{2.409pt}{0.400pt}}
\put(170.0,729.0){\rule[-0.200pt]{2.409pt}{0.400pt}}
\put(1429.0,729.0){\rule[-0.200pt]{2.409pt}{0.400pt}}
\put(170.0,729.0){\rule[-0.200pt]{2.409pt}{0.400pt}}
\put(1429.0,729.0){\rule[-0.200pt]{2.409pt}{0.400pt}}
\put(170.0,729.0){\rule[-0.200pt]{2.409pt}{0.400pt}}
\put(1429.0,729.0){\rule[-0.200pt]{2.409pt}{0.400pt}}
\put(170.0,729.0){\rule[-0.200pt]{2.409pt}{0.400pt}}
\put(1429.0,729.0){\rule[-0.200pt]{2.409pt}{0.400pt}}
\put(170.0,729.0){\rule[-0.200pt]{2.409pt}{0.400pt}}
\put(1429.0,729.0){\rule[-0.200pt]{2.409pt}{0.400pt}}
\put(170.0,730.0){\rule[-0.200pt]{4.818pt}{0.400pt}}
\put(150,730){\makebox(0,0)[r]{ 1}}
\put(1419.0,730.0){\rule[-0.200pt]{4.818pt}{0.400pt}}
\put(170.0,742.0){\rule[-0.200pt]{2.409pt}{0.400pt}}
\put(1429.0,742.0){\rule[-0.200pt]{2.409pt}{0.400pt}}
\put(170.0,760.0){\rule[-0.200pt]{2.409pt}{0.400pt}}
\put(1429.0,760.0){\rule[-0.200pt]{2.409pt}{0.400pt}}
\put(170.0,768.0){\rule[-0.200pt]{2.409pt}{0.400pt}}
\put(1429.0,768.0){\rule[-0.200pt]{2.409pt}{0.400pt}}
\put(170.0,774.0){\rule[-0.200pt]{2.409pt}{0.400pt}}
\put(1429.0,774.0){\rule[-0.200pt]{2.409pt}{0.400pt}}
\put(170.0,779.0){\rule[-0.200pt]{2.409pt}{0.400pt}}
\put(1429.0,779.0){\rule[-0.200pt]{2.409pt}{0.400pt}}
\put(170.0,783.0){\rule[-0.200pt]{2.409pt}{0.400pt}}
\put(1429.0,783.0){\rule[-0.200pt]{2.409pt}{0.400pt}}
\put(170.0,786.0){\rule[-0.200pt]{2.409pt}{0.400pt}}
\put(1429.0,786.0){\rule[-0.200pt]{2.409pt}{0.400pt}}
\put(170.0,788.0){\rule[-0.200pt]{2.409pt}{0.400pt}}
\put(1429.0,788.0){\rule[-0.200pt]{2.409pt}{0.400pt}}
\put(170.0,791.0){\rule[-0.200pt]{2.409pt}{0.400pt}}
\put(1429.0,791.0){\rule[-0.200pt]{2.409pt}{0.400pt}}
\put(170.0,793.0){\rule[-0.200pt]{2.409pt}{0.400pt}}
\put(1429.0,793.0){\rule[-0.200pt]{2.409pt}{0.400pt}}
\put(170.0,794.0){\rule[-0.200pt]{2.409pt}{0.400pt}}
\put(1429.0,794.0){\rule[-0.200pt]{2.409pt}{0.400pt}}
\put(170.0,796.0){\rule[-0.200pt]{2.409pt}{0.400pt}}
\put(1429.0,796.0){\rule[-0.200pt]{2.409pt}{0.400pt}}
\put(170.0,798.0){\rule[-0.200pt]{2.409pt}{0.400pt}}
\put(1429.0,798.0){\rule[-0.200pt]{2.409pt}{0.400pt}}
\put(170.0,799.0){\rule[-0.200pt]{2.409pt}{0.400pt}}
\put(1429.0,799.0){\rule[-0.200pt]{2.409pt}{0.400pt}}
\put(170.0,800.0){\rule[-0.200pt]{2.409pt}{0.400pt}}
\put(1429.0,800.0){\rule[-0.200pt]{2.409pt}{0.400pt}}
\put(170.0,802.0){\rule[-0.200pt]{2.409pt}{0.400pt}}
\put(1429.0,802.0){\rule[-0.200pt]{2.409pt}{0.400pt}}
\put(170.0,803.0){\rule[-0.200pt]{2.409pt}{0.400pt}}
\put(1429.0,803.0){\rule[-0.200pt]{2.409pt}{0.400pt}}
\put(170.0,804.0){\rule[-0.200pt]{2.409pt}{0.400pt}}
\put(1429.0,804.0){\rule[-0.200pt]{2.409pt}{0.400pt}}
\put(170.0,805.0){\rule[-0.200pt]{2.409pt}{0.400pt}}
\put(1429.0,805.0){\rule[-0.200pt]{2.409pt}{0.400pt}}
\put(170.0,806.0){\rule[-0.200pt]{2.409pt}{0.400pt}}
\put(1429.0,806.0){\rule[-0.200pt]{2.409pt}{0.400pt}}
\put(170.0,807.0){\rule[-0.200pt]{2.409pt}{0.400pt}}
\put(1429.0,807.0){\rule[-0.200pt]{2.409pt}{0.400pt}}
\put(170.0,808.0){\rule[-0.200pt]{2.409pt}{0.400pt}}
\put(1429.0,808.0){\rule[-0.200pt]{2.409pt}{0.400pt}}
\put(170.0,809.0){\rule[-0.200pt]{2.409pt}{0.400pt}}
\put(1429.0,809.0){\rule[-0.200pt]{2.409pt}{0.400pt}}
\put(170.0,809.0){\rule[-0.200pt]{2.409pt}{0.400pt}}
\put(1429.0,809.0){\rule[-0.200pt]{2.409pt}{0.400pt}}
\put(170.0,810.0){\rule[-0.200pt]{2.409pt}{0.400pt}}
\put(1429.0,810.0){\rule[-0.200pt]{2.409pt}{0.400pt}}
\put(170.0,811.0){\rule[-0.200pt]{2.409pt}{0.400pt}}
\put(1429.0,811.0){\rule[-0.200pt]{2.409pt}{0.400pt}}
\put(170.0,812.0){\rule[-0.200pt]{2.409pt}{0.400pt}}
\put(1429.0,812.0){\rule[-0.200pt]{2.409pt}{0.400pt}}
\put(170.0,812.0){\rule[-0.200pt]{2.409pt}{0.400pt}}
\put(1429.0,812.0){\rule[-0.200pt]{2.409pt}{0.400pt}}
\put(170.0,813.0){\rule[-0.200pt]{2.409pt}{0.400pt}}
\put(1429.0,813.0){\rule[-0.200pt]{2.409pt}{0.400pt}}
\put(170.0,814.0){\rule[-0.200pt]{2.409pt}{0.400pt}}
\put(1429.0,814.0){\rule[-0.200pt]{2.409pt}{0.400pt}}
\put(170.0,814.0){\rule[-0.200pt]{2.409pt}{0.400pt}}
\put(1429.0,814.0){\rule[-0.200pt]{2.409pt}{0.400pt}}
\put(170.0,815.0){\rule[-0.200pt]{2.409pt}{0.400pt}}
\put(1429.0,815.0){\rule[-0.200pt]{2.409pt}{0.400pt}}
\put(170.0,815.0){\rule[-0.200pt]{2.409pt}{0.400pt}}
\put(1429.0,815.0){\rule[-0.200pt]{2.409pt}{0.400pt}}
\put(170.0,816.0){\rule[-0.200pt]{2.409pt}{0.400pt}}
\put(1429.0,816.0){\rule[-0.200pt]{2.409pt}{0.400pt}}
\put(170.0,817.0){\rule[-0.200pt]{2.409pt}{0.400pt}}
\put(1429.0,817.0){\rule[-0.200pt]{2.409pt}{0.400pt}}
\put(170.0,817.0){\rule[-0.200pt]{2.409pt}{0.400pt}}
\put(1429.0,817.0){\rule[-0.200pt]{2.409pt}{0.400pt}}
\put(170.0,818.0){\rule[-0.200pt]{2.409pt}{0.400pt}}
\put(1429.0,818.0){\rule[-0.200pt]{2.409pt}{0.400pt}}
\put(170.0,818.0){\rule[-0.200pt]{2.409pt}{0.400pt}}
\put(1429.0,818.0){\rule[-0.200pt]{2.409pt}{0.400pt}}
\put(170.0,819.0){\rule[-0.200pt]{2.409pt}{0.400pt}}
\put(1429.0,819.0){\rule[-0.200pt]{2.409pt}{0.400pt}}
\put(170.0,819.0){\rule[-0.200pt]{2.409pt}{0.400pt}}
\put(1429.0,819.0){\rule[-0.200pt]{2.409pt}{0.400pt}}
\put(170.0,820.0){\rule[-0.200pt]{2.409pt}{0.400pt}}
\put(1429.0,820.0){\rule[-0.200pt]{2.409pt}{0.400pt}}
\put(170.0,820.0){\rule[-0.200pt]{2.409pt}{0.400pt}}
\put(1429.0,820.0){\rule[-0.200pt]{2.409pt}{0.400pt}}
\put(170.0,820.0){\rule[-0.200pt]{2.409pt}{0.400pt}}
\put(1429.0,820.0){\rule[-0.200pt]{2.409pt}{0.400pt}}
\put(170.0,821.0){\rule[-0.200pt]{2.409pt}{0.400pt}}
\put(1429.0,821.0){\rule[-0.200pt]{2.409pt}{0.400pt}}
\put(170.0,821.0){\rule[-0.200pt]{2.409pt}{0.400pt}}
\put(1429.0,821.0){\rule[-0.200pt]{2.409pt}{0.400pt}}
\put(170.0,822.0){\rule[-0.200pt]{2.409pt}{0.400pt}}
\put(1429.0,822.0){\rule[-0.200pt]{2.409pt}{0.400pt}}
\put(170.0,822.0){\rule[-0.200pt]{2.409pt}{0.400pt}}
\put(1429.0,822.0){\rule[-0.200pt]{2.409pt}{0.400pt}}
\put(170.0,823.0){\rule[-0.200pt]{2.409pt}{0.400pt}}
\put(1429.0,823.0){\rule[-0.200pt]{2.409pt}{0.400pt}}
\put(170.0,823.0){\rule[-0.200pt]{2.409pt}{0.400pt}}
\put(1429.0,823.0){\rule[-0.200pt]{2.409pt}{0.400pt}}
\put(170.0,823.0){\rule[-0.200pt]{2.409pt}{0.400pt}}
\put(1429.0,823.0){\rule[-0.200pt]{2.409pt}{0.400pt}}
\put(170.0,824.0){\rule[-0.200pt]{2.409pt}{0.400pt}}
\put(1429.0,824.0){\rule[-0.200pt]{2.409pt}{0.400pt}}
\put(170.0,824.0){\rule[-0.200pt]{2.409pt}{0.400pt}}
\put(1429.0,824.0){\rule[-0.200pt]{2.409pt}{0.400pt}}
\put(170.0,824.0){\rule[-0.200pt]{2.409pt}{0.400pt}}
\put(1429.0,824.0){\rule[-0.200pt]{2.409pt}{0.400pt}}
\put(170.0,825.0){\rule[-0.200pt]{2.409pt}{0.400pt}}
\put(1429.0,825.0){\rule[-0.200pt]{2.409pt}{0.400pt}}
\put(170.0,825.0){\rule[-0.200pt]{2.409pt}{0.400pt}}
\put(1429.0,825.0){\rule[-0.200pt]{2.409pt}{0.400pt}}
\put(170.0,825.0){\rule[-0.200pt]{2.409pt}{0.400pt}}
\put(1429.0,825.0){\rule[-0.200pt]{2.409pt}{0.400pt}}
\put(170.0,826.0){\rule[-0.200pt]{2.409pt}{0.400pt}}
\put(1429.0,826.0){\rule[-0.200pt]{2.409pt}{0.400pt}}
\put(170.0,826.0){\rule[-0.200pt]{2.409pt}{0.400pt}}
\put(1429.0,826.0){\rule[-0.200pt]{2.409pt}{0.400pt}}
\put(170.0,826.0){\rule[-0.200pt]{2.409pt}{0.400pt}}
\put(1429.0,826.0){\rule[-0.200pt]{2.409pt}{0.400pt}}
\put(170.0,827.0){\rule[-0.200pt]{2.409pt}{0.400pt}}
\put(1429.0,827.0){\rule[-0.200pt]{2.409pt}{0.400pt}}
\put(170.0,827.0){\rule[-0.200pt]{2.409pt}{0.400pt}}
\put(1429.0,827.0){\rule[-0.200pt]{2.409pt}{0.400pt}}
\put(170.0,827.0){\rule[-0.200pt]{2.409pt}{0.400pt}}
\put(1429.0,827.0){\rule[-0.200pt]{2.409pt}{0.400pt}}
\put(170.0,828.0){\rule[-0.200pt]{2.409pt}{0.400pt}}
\put(1429.0,828.0){\rule[-0.200pt]{2.409pt}{0.400pt}}
\put(170.0,828.0){\rule[-0.200pt]{2.409pt}{0.400pt}}
\put(1429.0,828.0){\rule[-0.200pt]{2.409pt}{0.400pt}}
\put(170.0,828.0){\rule[-0.200pt]{2.409pt}{0.400pt}}
\put(1429.0,828.0){\rule[-0.200pt]{2.409pt}{0.400pt}}
\put(170.0,829.0){\rule[-0.200pt]{2.409pt}{0.400pt}}
\put(1429.0,829.0){\rule[-0.200pt]{2.409pt}{0.400pt}}
\put(170.0,829.0){\rule[-0.200pt]{2.409pt}{0.400pt}}
\put(1429.0,829.0){\rule[-0.200pt]{2.409pt}{0.400pt}}
\put(170.0,829.0){\rule[-0.200pt]{2.409pt}{0.400pt}}
\put(1429.0,829.0){\rule[-0.200pt]{2.409pt}{0.400pt}}
\put(170.0,829.0){\rule[-0.200pt]{2.409pt}{0.400pt}}
\put(1429.0,829.0){\rule[-0.200pt]{2.409pt}{0.400pt}}
\put(170.0,830.0){\rule[-0.200pt]{2.409pt}{0.400pt}}
\put(1429.0,830.0){\rule[-0.200pt]{2.409pt}{0.400pt}}
\put(170.0,830.0){\rule[-0.200pt]{2.409pt}{0.400pt}}
\put(1429.0,830.0){\rule[-0.200pt]{2.409pt}{0.400pt}}
\put(170.0,830.0){\rule[-0.200pt]{2.409pt}{0.400pt}}
\put(1429.0,830.0){\rule[-0.200pt]{2.409pt}{0.400pt}}
\put(170.0,830.0){\rule[-0.200pt]{2.409pt}{0.400pt}}
\put(1429.0,830.0){\rule[-0.200pt]{2.409pt}{0.400pt}}
\put(170.0,831.0){\rule[-0.200pt]{2.409pt}{0.400pt}}
\put(1429.0,831.0){\rule[-0.200pt]{2.409pt}{0.400pt}}
\put(170.0,831.0){\rule[-0.200pt]{2.409pt}{0.400pt}}
\put(1429.0,831.0){\rule[-0.200pt]{2.409pt}{0.400pt}}
\put(170.0,831.0){\rule[-0.200pt]{2.409pt}{0.400pt}}
\put(1429.0,831.0){\rule[-0.200pt]{2.409pt}{0.400pt}}
\put(170.0,831.0){\rule[-0.200pt]{2.409pt}{0.400pt}}
\put(1429.0,831.0){\rule[-0.200pt]{2.409pt}{0.400pt}}
\put(170.0,832.0){\rule[-0.200pt]{2.409pt}{0.400pt}}
\put(1429.0,832.0){\rule[-0.200pt]{2.409pt}{0.400pt}}
\put(170.0,832.0){\rule[-0.200pt]{2.409pt}{0.400pt}}
\put(1429.0,832.0){\rule[-0.200pt]{2.409pt}{0.400pt}}
\put(170.0,832.0){\rule[-0.200pt]{2.409pt}{0.400pt}}
\put(1429.0,832.0){\rule[-0.200pt]{2.409pt}{0.400pt}}
\put(170.0,832.0){\rule[-0.200pt]{2.409pt}{0.400pt}}
\put(1429.0,832.0){\rule[-0.200pt]{2.409pt}{0.400pt}}
\put(170.0,833.0){\rule[-0.200pt]{2.409pt}{0.400pt}}
\put(1429.0,833.0){\rule[-0.200pt]{2.409pt}{0.400pt}}
\put(170.0,833.0){\rule[-0.200pt]{2.409pt}{0.400pt}}
\put(1429.0,833.0){\rule[-0.200pt]{2.409pt}{0.400pt}}
\put(170.0,833.0){\rule[-0.200pt]{2.409pt}{0.400pt}}
\put(1429.0,833.0){\rule[-0.200pt]{2.409pt}{0.400pt}}
\put(170.0,833.0){\rule[-0.200pt]{2.409pt}{0.400pt}}
\put(1429.0,833.0){\rule[-0.200pt]{2.409pt}{0.400pt}}
\put(170.0,834.0){\rule[-0.200pt]{2.409pt}{0.400pt}}
\put(1429.0,834.0){\rule[-0.200pt]{2.409pt}{0.400pt}}
\put(170.0,834.0){\rule[-0.200pt]{2.409pt}{0.400pt}}
\put(1429.0,834.0){\rule[-0.200pt]{2.409pt}{0.400pt}}
\put(170.0,834.0){\rule[-0.200pt]{2.409pt}{0.400pt}}
\put(1429.0,834.0){\rule[-0.200pt]{2.409pt}{0.400pt}}
\put(170.0,834.0){\rule[-0.200pt]{2.409pt}{0.400pt}}
\put(1429.0,834.0){\rule[-0.200pt]{2.409pt}{0.400pt}}
\put(170.0,834.0){\rule[-0.200pt]{2.409pt}{0.400pt}}
\put(1429.0,834.0){\rule[-0.200pt]{2.409pt}{0.400pt}}
\put(170.0,835.0){\rule[-0.200pt]{2.409pt}{0.400pt}}
\put(1429.0,835.0){\rule[-0.200pt]{2.409pt}{0.400pt}}
\put(170.0,835.0){\rule[-0.200pt]{2.409pt}{0.400pt}}
\put(1429.0,835.0){\rule[-0.200pt]{2.409pt}{0.400pt}}
\put(170.0,835.0){\rule[-0.200pt]{2.409pt}{0.400pt}}
\put(1429.0,835.0){\rule[-0.200pt]{2.409pt}{0.400pt}}
\put(170.0,835.0){\rule[-0.200pt]{2.409pt}{0.400pt}}
\put(1429.0,835.0){\rule[-0.200pt]{2.409pt}{0.400pt}}
\put(170.0,835.0){\rule[-0.200pt]{2.409pt}{0.400pt}}
\put(1429.0,835.0){\rule[-0.200pt]{2.409pt}{0.400pt}}
\put(170.0,836.0){\rule[-0.200pt]{2.409pt}{0.400pt}}
\put(1429.0,836.0){\rule[-0.200pt]{2.409pt}{0.400pt}}
\put(170.0,836.0){\rule[-0.200pt]{2.409pt}{0.400pt}}
\put(1429.0,836.0){\rule[-0.200pt]{2.409pt}{0.400pt}}
\put(170.0,836.0){\rule[-0.200pt]{2.409pt}{0.400pt}}
\put(1429.0,836.0){\rule[-0.200pt]{2.409pt}{0.400pt}}
\put(170.0,836.0){\rule[-0.200pt]{2.409pt}{0.400pt}}
\put(1429.0,836.0){\rule[-0.200pt]{2.409pt}{0.400pt}}
\put(170.0,836.0){\rule[-0.200pt]{2.409pt}{0.400pt}}
\put(1429.0,836.0){\rule[-0.200pt]{2.409pt}{0.400pt}}
\put(170.0,837.0){\rule[-0.200pt]{2.409pt}{0.400pt}}
\put(1429.0,837.0){\rule[-0.200pt]{2.409pt}{0.400pt}}
\put(170.0,837.0){\rule[-0.200pt]{2.409pt}{0.400pt}}
\put(1429.0,837.0){\rule[-0.200pt]{2.409pt}{0.400pt}}
\put(170.0,837.0){\rule[-0.200pt]{2.409pt}{0.400pt}}
\put(1429.0,837.0){\rule[-0.200pt]{2.409pt}{0.400pt}}
\put(170.0,837.0){\rule[-0.200pt]{2.409pt}{0.400pt}}
\put(1429.0,837.0){\rule[-0.200pt]{2.409pt}{0.400pt}}
\put(170.0,837.0){\rule[-0.200pt]{2.409pt}{0.400pt}}
\put(1429.0,837.0){\rule[-0.200pt]{2.409pt}{0.400pt}}
\put(170.0,837.0){\rule[-0.200pt]{2.409pt}{0.400pt}}
\put(1429.0,837.0){\rule[-0.200pt]{2.409pt}{0.400pt}}
\put(170.0,838.0){\rule[-0.200pt]{2.409pt}{0.400pt}}
\put(1429.0,838.0){\rule[-0.200pt]{2.409pt}{0.400pt}}
\put(170.0,838.0){\rule[-0.200pt]{2.409pt}{0.400pt}}
\put(1429.0,838.0){\rule[-0.200pt]{2.409pt}{0.400pt}}
\put(170.0,838.0){\rule[-0.200pt]{2.409pt}{0.400pt}}
\put(1429.0,838.0){\rule[-0.200pt]{2.409pt}{0.400pt}}
\put(170.0,838.0){\rule[-0.200pt]{2.409pt}{0.400pt}}
\put(1429.0,838.0){\rule[-0.200pt]{2.409pt}{0.400pt}}
\put(170.0,838.0){\rule[-0.200pt]{2.409pt}{0.400pt}}
\put(1429.0,838.0){\rule[-0.200pt]{2.409pt}{0.400pt}}
\put(170.0,838.0){\rule[-0.200pt]{2.409pt}{0.400pt}}
\put(1429.0,838.0){\rule[-0.200pt]{2.409pt}{0.400pt}}
\put(170.0,839.0){\rule[-0.200pt]{2.409pt}{0.400pt}}
\put(1429.0,839.0){\rule[-0.200pt]{2.409pt}{0.400pt}}
\put(170.0,839.0){\rule[-0.200pt]{2.409pt}{0.400pt}}
\put(1429.0,839.0){\rule[-0.200pt]{2.409pt}{0.400pt}}
\put(170.0,839.0){\rule[-0.200pt]{2.409pt}{0.400pt}}
\put(1429.0,839.0){\rule[-0.200pt]{2.409pt}{0.400pt}}
\put(170.0,839.0){\rule[-0.200pt]{2.409pt}{0.400pt}}
\put(1429.0,839.0){\rule[-0.200pt]{2.409pt}{0.400pt}}
\put(170.0,839.0){\rule[-0.200pt]{2.409pt}{0.400pt}}
\put(1429.0,839.0){\rule[-0.200pt]{2.409pt}{0.400pt}}
\put(170.0,839.0){\rule[-0.200pt]{2.409pt}{0.400pt}}
\put(1429.0,839.0){\rule[-0.200pt]{2.409pt}{0.400pt}}
\put(170.0,840.0){\rule[-0.200pt]{2.409pt}{0.400pt}}
\put(1429.0,840.0){\rule[-0.200pt]{2.409pt}{0.400pt}}
\put(170.0,840.0){\rule[-0.200pt]{2.409pt}{0.400pt}}
\put(1429.0,840.0){\rule[-0.200pt]{2.409pt}{0.400pt}}
\put(170.0,840.0){\rule[-0.200pt]{2.409pt}{0.400pt}}
\put(1429.0,840.0){\rule[-0.200pt]{2.409pt}{0.400pt}}
\put(170.0,840.0){\rule[-0.200pt]{2.409pt}{0.400pt}}
\put(1429.0,840.0){\rule[-0.200pt]{2.409pt}{0.400pt}}
\put(170.0,840.0){\rule[-0.200pt]{2.409pt}{0.400pt}}
\put(1429.0,840.0){\rule[-0.200pt]{2.409pt}{0.400pt}}
\put(170.0,840.0){\rule[-0.200pt]{2.409pt}{0.400pt}}
\put(1429.0,840.0){\rule[-0.200pt]{2.409pt}{0.400pt}}
\put(170.0,841.0){\rule[-0.200pt]{2.409pt}{0.400pt}}
\put(1429.0,841.0){\rule[-0.200pt]{2.409pt}{0.400pt}}
\put(170.0,841.0){\rule[-0.200pt]{2.409pt}{0.400pt}}
\put(1429.0,841.0){\rule[-0.200pt]{2.409pt}{0.400pt}}
\put(170.0,841.0){\rule[-0.200pt]{2.409pt}{0.400pt}}
\put(1429.0,841.0){\rule[-0.200pt]{2.409pt}{0.400pt}}
\put(170.0,841.0){\rule[-0.200pt]{2.409pt}{0.400pt}}
\put(1429.0,841.0){\rule[-0.200pt]{2.409pt}{0.400pt}}
\put(170.0,841.0){\rule[-0.200pt]{2.409pt}{0.400pt}}
\put(1429.0,841.0){\rule[-0.200pt]{2.409pt}{0.400pt}}
\put(170.0,841.0){\rule[-0.200pt]{2.409pt}{0.400pt}}
\put(1429.0,841.0){\rule[-0.200pt]{2.409pt}{0.400pt}}
\put(170.0,841.0){\rule[-0.200pt]{2.409pt}{0.400pt}}
\put(1429.0,841.0){\rule[-0.200pt]{2.409pt}{0.400pt}}
\put(170.0,842.0){\rule[-0.200pt]{2.409pt}{0.400pt}}
\put(1429.0,842.0){\rule[-0.200pt]{2.409pt}{0.400pt}}
\put(170.0,842.0){\rule[-0.200pt]{2.409pt}{0.400pt}}
\put(1429.0,842.0){\rule[-0.200pt]{2.409pt}{0.400pt}}
\put(170.0,842.0){\rule[-0.200pt]{2.409pt}{0.400pt}}
\put(1429.0,842.0){\rule[-0.200pt]{2.409pt}{0.400pt}}
\put(170.0,842.0){\rule[-0.200pt]{2.409pt}{0.400pt}}
\put(1429.0,842.0){\rule[-0.200pt]{2.409pt}{0.400pt}}
\put(170.0,842.0){\rule[-0.200pt]{2.409pt}{0.400pt}}
\put(1429.0,842.0){\rule[-0.200pt]{2.409pt}{0.400pt}}
\put(170.0,842.0){\rule[-0.200pt]{2.409pt}{0.400pt}}
\put(1429.0,842.0){\rule[-0.200pt]{2.409pt}{0.400pt}}
\put(170.0,842.0){\rule[-0.200pt]{2.409pt}{0.400pt}}
\put(1429.0,842.0){\rule[-0.200pt]{2.409pt}{0.400pt}}
\put(170.0,843.0){\rule[-0.200pt]{2.409pt}{0.400pt}}
\put(1429.0,843.0){\rule[-0.200pt]{2.409pt}{0.400pt}}
\put(170.0,843.0){\rule[-0.200pt]{2.409pt}{0.400pt}}
\put(1429.0,843.0){\rule[-0.200pt]{2.409pt}{0.400pt}}
\put(170.0,843.0){\rule[-0.200pt]{2.409pt}{0.400pt}}
\put(1429.0,843.0){\rule[-0.200pt]{2.409pt}{0.400pt}}
\put(170.0,843.0){\rule[-0.200pt]{2.409pt}{0.400pt}}
\put(1429.0,843.0){\rule[-0.200pt]{2.409pt}{0.400pt}}
\put(170.0,843.0){\rule[-0.200pt]{2.409pt}{0.400pt}}
\put(1429.0,843.0){\rule[-0.200pt]{2.409pt}{0.400pt}}
\put(170.0,843.0){\rule[-0.200pt]{2.409pt}{0.400pt}}
\put(1429.0,843.0){\rule[-0.200pt]{2.409pt}{0.400pt}}
\put(170.0,843.0){\rule[-0.200pt]{2.409pt}{0.400pt}}
\put(1429.0,843.0){\rule[-0.200pt]{2.409pt}{0.400pt}}
\put(170.0,843.0){\rule[-0.200pt]{2.409pt}{0.400pt}}
\put(1429.0,843.0){\rule[-0.200pt]{2.409pt}{0.400pt}}
\put(170.0,844.0){\rule[-0.200pt]{2.409pt}{0.400pt}}
\put(1429.0,844.0){\rule[-0.200pt]{2.409pt}{0.400pt}}
\put(170.0,844.0){\rule[-0.200pt]{2.409pt}{0.400pt}}
\put(1429.0,844.0){\rule[-0.200pt]{2.409pt}{0.400pt}}
\put(170.0,844.0){\rule[-0.200pt]{2.409pt}{0.400pt}}
\put(1429.0,844.0){\rule[-0.200pt]{2.409pt}{0.400pt}}
\put(170.0,844.0){\rule[-0.200pt]{2.409pt}{0.400pt}}
\put(1429.0,844.0){\rule[-0.200pt]{2.409pt}{0.400pt}}
\put(170.0,844.0){\rule[-0.200pt]{2.409pt}{0.400pt}}
\put(1429.0,844.0){\rule[-0.200pt]{2.409pt}{0.400pt}}
\put(170.0,844.0){\rule[-0.200pt]{2.409pt}{0.400pt}}
\put(1429.0,844.0){\rule[-0.200pt]{2.409pt}{0.400pt}}
\put(170.0,844.0){\rule[-0.200pt]{2.409pt}{0.400pt}}
\put(1429.0,844.0){\rule[-0.200pt]{2.409pt}{0.400pt}}
\put(170.0,844.0){\rule[-0.200pt]{2.409pt}{0.400pt}}
\put(1429.0,844.0){\rule[-0.200pt]{2.409pt}{0.400pt}}
\put(170.0,845.0){\rule[-0.200pt]{2.409pt}{0.400pt}}
\put(1429.0,845.0){\rule[-0.200pt]{2.409pt}{0.400pt}}
\put(170.0,845.0){\rule[-0.200pt]{2.409pt}{0.400pt}}
\put(1429.0,845.0){\rule[-0.200pt]{2.409pt}{0.400pt}}
\put(170.0,845.0){\rule[-0.200pt]{2.409pt}{0.400pt}}
\put(1429.0,845.0){\rule[-0.200pt]{2.409pt}{0.400pt}}
\put(170.0,845.0){\rule[-0.200pt]{2.409pt}{0.400pt}}
\put(1429.0,845.0){\rule[-0.200pt]{2.409pt}{0.400pt}}
\put(170.0,845.0){\rule[-0.200pt]{2.409pt}{0.400pt}}
\put(1429.0,845.0){\rule[-0.200pt]{2.409pt}{0.400pt}}
\put(170.0,845.0){\rule[-0.200pt]{2.409pt}{0.400pt}}
\put(1429.0,845.0){\rule[-0.200pt]{2.409pt}{0.400pt}}
\put(170.0,845.0){\rule[-0.200pt]{2.409pt}{0.400pt}}
\put(1429.0,845.0){\rule[-0.200pt]{2.409pt}{0.400pt}}
\put(170.0,845.0){\rule[-0.200pt]{2.409pt}{0.400pt}}
\put(1429.0,845.0){\rule[-0.200pt]{2.409pt}{0.400pt}}
\put(170.0,846.0){\rule[-0.200pt]{2.409pt}{0.400pt}}
\put(1429.0,846.0){\rule[-0.200pt]{2.409pt}{0.400pt}}
\put(170.0,846.0){\rule[-0.200pt]{2.409pt}{0.400pt}}
\put(1429.0,846.0){\rule[-0.200pt]{2.409pt}{0.400pt}}
\put(170.0,846.0){\rule[-0.200pt]{2.409pt}{0.400pt}}
\put(1429.0,846.0){\rule[-0.200pt]{2.409pt}{0.400pt}}
\put(170.0,846.0){\rule[-0.200pt]{2.409pt}{0.400pt}}
\put(1429.0,846.0){\rule[-0.200pt]{2.409pt}{0.400pt}}
\put(170.0,846.0){\rule[-0.200pt]{2.409pt}{0.400pt}}
\put(1429.0,846.0){\rule[-0.200pt]{2.409pt}{0.400pt}}
\put(170.0,846.0){\rule[-0.200pt]{2.409pt}{0.400pt}}
\put(1429.0,846.0){\rule[-0.200pt]{2.409pt}{0.400pt}}
\put(170.0,846.0){\rule[-0.200pt]{2.409pt}{0.400pt}}
\put(1429.0,846.0){\rule[-0.200pt]{2.409pt}{0.400pt}}
\put(170.0,846.0){\rule[-0.200pt]{2.409pt}{0.400pt}}
\put(1429.0,846.0){\rule[-0.200pt]{2.409pt}{0.400pt}}
\put(170.0,846.0){\rule[-0.200pt]{2.409pt}{0.400pt}}
\put(1429.0,846.0){\rule[-0.200pt]{2.409pt}{0.400pt}}
\put(170.0,847.0){\rule[-0.200pt]{2.409pt}{0.400pt}}
\put(1429.0,847.0){\rule[-0.200pt]{2.409pt}{0.400pt}}
\put(170.0,847.0){\rule[-0.200pt]{2.409pt}{0.400pt}}
\put(1429.0,847.0){\rule[-0.200pt]{2.409pt}{0.400pt}}
\put(170.0,847.0){\rule[-0.200pt]{2.409pt}{0.400pt}}
\put(1429.0,847.0){\rule[-0.200pt]{2.409pt}{0.400pt}}
\put(170.0,847.0){\rule[-0.200pt]{2.409pt}{0.400pt}}
\put(1429.0,847.0){\rule[-0.200pt]{2.409pt}{0.400pt}}
\put(170.0,847.0){\rule[-0.200pt]{2.409pt}{0.400pt}}
\put(1429.0,847.0){\rule[-0.200pt]{2.409pt}{0.400pt}}
\put(170.0,847.0){\rule[-0.200pt]{2.409pt}{0.400pt}}
\put(1429.0,847.0){\rule[-0.200pt]{2.409pt}{0.400pt}}
\put(170.0,847.0){\rule[-0.200pt]{2.409pt}{0.400pt}}
\put(1429.0,847.0){\rule[-0.200pt]{2.409pt}{0.400pt}}
\put(170.0,847.0){\rule[-0.200pt]{2.409pt}{0.400pt}}
\put(1429.0,847.0){\rule[-0.200pt]{2.409pt}{0.400pt}}
\put(170.0,847.0){\rule[-0.200pt]{2.409pt}{0.400pt}}
\put(1429.0,847.0){\rule[-0.200pt]{2.409pt}{0.400pt}}
\put(170.0,848.0){\rule[-0.200pt]{2.409pt}{0.400pt}}
\put(1429.0,848.0){\rule[-0.200pt]{2.409pt}{0.400pt}}
\put(170.0,848.0){\rule[-0.200pt]{2.409pt}{0.400pt}}
\put(1429.0,848.0){\rule[-0.200pt]{2.409pt}{0.400pt}}
\put(170.0,848.0){\rule[-0.200pt]{2.409pt}{0.400pt}}
\put(1429.0,848.0){\rule[-0.200pt]{2.409pt}{0.400pt}}
\put(170.0,848.0){\rule[-0.200pt]{2.409pt}{0.400pt}}
\put(1429.0,848.0){\rule[-0.200pt]{2.409pt}{0.400pt}}
\put(170.0,848.0){\rule[-0.200pt]{2.409pt}{0.400pt}}
\put(1429.0,848.0){\rule[-0.200pt]{2.409pt}{0.400pt}}
\put(170.0,848.0){\rule[-0.200pt]{2.409pt}{0.400pt}}
\put(1429.0,848.0){\rule[-0.200pt]{2.409pt}{0.400pt}}
\put(170.0,848.0){\rule[-0.200pt]{2.409pt}{0.400pt}}
\put(1429.0,848.0){\rule[-0.200pt]{2.409pt}{0.400pt}}
\put(170.0,848.0){\rule[-0.200pt]{2.409pt}{0.400pt}}
\put(1429.0,848.0){\rule[-0.200pt]{2.409pt}{0.400pt}}
\put(170.0,848.0){\rule[-0.200pt]{2.409pt}{0.400pt}}
\put(1429.0,848.0){\rule[-0.200pt]{2.409pt}{0.400pt}}
\put(170.0,848.0){\rule[-0.200pt]{2.409pt}{0.400pt}}
\put(1429.0,848.0){\rule[-0.200pt]{2.409pt}{0.400pt}}
\put(170.0,849.0){\rule[-0.200pt]{2.409pt}{0.400pt}}
\put(1429.0,849.0){\rule[-0.200pt]{2.409pt}{0.400pt}}
\put(170.0,849.0){\rule[-0.200pt]{2.409pt}{0.400pt}}
\put(1429.0,849.0){\rule[-0.200pt]{2.409pt}{0.400pt}}
\put(170.0,849.0){\rule[-0.200pt]{2.409pt}{0.400pt}}
\put(1429.0,849.0){\rule[-0.200pt]{2.409pt}{0.400pt}}
\put(170.0,849.0){\rule[-0.200pt]{2.409pt}{0.400pt}}
\put(1429.0,849.0){\rule[-0.200pt]{2.409pt}{0.400pt}}
\put(170.0,849.0){\rule[-0.200pt]{2.409pt}{0.400pt}}
\put(1429.0,849.0){\rule[-0.200pt]{2.409pt}{0.400pt}}
\put(170.0,849.0){\rule[-0.200pt]{2.409pt}{0.400pt}}
\put(1429.0,849.0){\rule[-0.200pt]{2.409pt}{0.400pt}}
\put(170.0,849.0){\rule[-0.200pt]{2.409pt}{0.400pt}}
\put(1429.0,849.0){\rule[-0.200pt]{2.409pt}{0.400pt}}
\put(170.0,849.0){\rule[-0.200pt]{2.409pt}{0.400pt}}
\put(1429.0,849.0){\rule[-0.200pt]{2.409pt}{0.400pt}}
\put(170.0,849.0){\rule[-0.200pt]{2.409pt}{0.400pt}}
\put(1429.0,849.0){\rule[-0.200pt]{2.409pt}{0.400pt}}
\put(170.0,849.0){\rule[-0.200pt]{2.409pt}{0.400pt}}
\put(1429.0,849.0){\rule[-0.200pt]{2.409pt}{0.400pt}}
\put(170.0,849.0){\rule[-0.200pt]{2.409pt}{0.400pt}}
\put(1429.0,849.0){\rule[-0.200pt]{2.409pt}{0.400pt}}
\put(170.0,850.0){\rule[-0.200pt]{2.409pt}{0.400pt}}
\put(1429.0,850.0){\rule[-0.200pt]{2.409pt}{0.400pt}}
\put(170.0,850.0){\rule[-0.200pt]{2.409pt}{0.400pt}}
\put(1429.0,850.0){\rule[-0.200pt]{2.409pt}{0.400pt}}
\put(170.0,850.0){\rule[-0.200pt]{2.409pt}{0.400pt}}
\put(1429.0,850.0){\rule[-0.200pt]{2.409pt}{0.400pt}}
\put(170.0,850.0){\rule[-0.200pt]{2.409pt}{0.400pt}}
\put(1429.0,850.0){\rule[-0.200pt]{2.409pt}{0.400pt}}
\put(170.0,850.0){\rule[-0.200pt]{2.409pt}{0.400pt}}
\put(1429.0,850.0){\rule[-0.200pt]{2.409pt}{0.400pt}}
\put(170.0,850.0){\rule[-0.200pt]{2.409pt}{0.400pt}}
\put(1429.0,850.0){\rule[-0.200pt]{2.409pt}{0.400pt}}
\put(170.0,850.0){\rule[-0.200pt]{2.409pt}{0.400pt}}
\put(1429.0,850.0){\rule[-0.200pt]{2.409pt}{0.400pt}}
\put(170.0,850.0){\rule[-0.200pt]{2.409pt}{0.400pt}}
\put(1429.0,850.0){\rule[-0.200pt]{2.409pt}{0.400pt}}
\put(170.0,850.0){\rule[-0.200pt]{2.409pt}{0.400pt}}
\put(1429.0,850.0){\rule[-0.200pt]{2.409pt}{0.400pt}}
\put(170.0,850.0){\rule[-0.200pt]{2.409pt}{0.400pt}}
\put(1429.0,850.0){\rule[-0.200pt]{2.409pt}{0.400pt}}
\put(170.0,850.0){\rule[-0.200pt]{2.409pt}{0.400pt}}
\put(1429.0,850.0){\rule[-0.200pt]{2.409pt}{0.400pt}}
\put(170.0,851.0){\rule[-0.200pt]{2.409pt}{0.400pt}}
\put(1429.0,851.0){\rule[-0.200pt]{2.409pt}{0.400pt}}
\put(170.0,851.0){\rule[-0.200pt]{2.409pt}{0.400pt}}
\put(1429.0,851.0){\rule[-0.200pt]{2.409pt}{0.400pt}}
\put(170.0,851.0){\rule[-0.200pt]{2.409pt}{0.400pt}}
\put(1429.0,851.0){\rule[-0.200pt]{2.409pt}{0.400pt}}
\put(170.0,851.0){\rule[-0.200pt]{2.409pt}{0.400pt}}
\put(1429.0,851.0){\rule[-0.200pt]{2.409pt}{0.400pt}}
\put(170.0,851.0){\rule[-0.200pt]{2.409pt}{0.400pt}}
\put(1429.0,851.0){\rule[-0.200pt]{2.409pt}{0.400pt}}
\put(170.0,851.0){\rule[-0.200pt]{2.409pt}{0.400pt}}
\put(1429.0,851.0){\rule[-0.200pt]{2.409pt}{0.400pt}}
\put(170.0,851.0){\rule[-0.200pt]{2.409pt}{0.400pt}}
\put(1429.0,851.0){\rule[-0.200pt]{2.409pt}{0.400pt}}
\put(170.0,851.0){\rule[-0.200pt]{2.409pt}{0.400pt}}
\put(1429.0,851.0){\rule[-0.200pt]{2.409pt}{0.400pt}}
\put(170.0,851.0){\rule[-0.200pt]{2.409pt}{0.400pt}}
\put(1429.0,851.0){\rule[-0.200pt]{2.409pt}{0.400pt}}
\put(170.0,851.0){\rule[-0.200pt]{2.409pt}{0.400pt}}
\put(1429.0,851.0){\rule[-0.200pt]{2.409pt}{0.400pt}}
\put(170.0,851.0){\rule[-0.200pt]{2.409pt}{0.400pt}}
\put(1429.0,851.0){\rule[-0.200pt]{2.409pt}{0.400pt}}
\put(170.0,852.0){\rule[-0.200pt]{2.409pt}{0.400pt}}
\put(1429.0,852.0){\rule[-0.200pt]{2.409pt}{0.400pt}}
\put(170.0,852.0){\rule[-0.200pt]{2.409pt}{0.400pt}}
\put(1429.0,852.0){\rule[-0.200pt]{2.409pt}{0.400pt}}
\put(170.0,852.0){\rule[-0.200pt]{2.409pt}{0.400pt}}
\put(1429.0,852.0){\rule[-0.200pt]{2.409pt}{0.400pt}}
\put(170.0,852.0){\rule[-0.200pt]{2.409pt}{0.400pt}}
\put(1429.0,852.0){\rule[-0.200pt]{2.409pt}{0.400pt}}
\put(170.0,852.0){\rule[-0.200pt]{2.409pt}{0.400pt}}
\put(1429.0,852.0){\rule[-0.200pt]{2.409pt}{0.400pt}}
\put(170.0,852.0){\rule[-0.200pt]{2.409pt}{0.400pt}}
\put(1429.0,852.0){\rule[-0.200pt]{2.409pt}{0.400pt}}
\put(170.0,852.0){\rule[-0.200pt]{2.409pt}{0.400pt}}
\put(1429.0,852.0){\rule[-0.200pt]{2.409pt}{0.400pt}}
\put(170.0,852.0){\rule[-0.200pt]{2.409pt}{0.400pt}}
\put(1429.0,852.0){\rule[-0.200pt]{2.409pt}{0.400pt}}
\put(170.0,852.0){\rule[-0.200pt]{2.409pt}{0.400pt}}
\put(1429.0,852.0){\rule[-0.200pt]{2.409pt}{0.400pt}}
\put(170.0,852.0){\rule[-0.200pt]{2.409pt}{0.400pt}}
\put(1429.0,852.0){\rule[-0.200pt]{2.409pt}{0.400pt}}
\put(170.0,852.0){\rule[-0.200pt]{2.409pt}{0.400pt}}
\put(1429.0,852.0){\rule[-0.200pt]{2.409pt}{0.400pt}}
\put(170.0,852.0){\rule[-0.200pt]{2.409pt}{0.400pt}}
\put(1429.0,852.0){\rule[-0.200pt]{2.409pt}{0.400pt}}
\put(170.0,852.0){\rule[-0.200pt]{2.409pt}{0.400pt}}
\put(1429.0,852.0){\rule[-0.200pt]{2.409pt}{0.400pt}}
\put(170.0,853.0){\rule[-0.200pt]{2.409pt}{0.400pt}}
\put(1429.0,853.0){\rule[-0.200pt]{2.409pt}{0.400pt}}
\put(170.0,853.0){\rule[-0.200pt]{2.409pt}{0.400pt}}
\put(1429.0,853.0){\rule[-0.200pt]{2.409pt}{0.400pt}}
\put(170.0,853.0){\rule[-0.200pt]{2.409pt}{0.400pt}}
\put(1429.0,853.0){\rule[-0.200pt]{2.409pt}{0.400pt}}
\put(170.0,853.0){\rule[-0.200pt]{2.409pt}{0.400pt}}
\put(1429.0,853.0){\rule[-0.200pt]{2.409pt}{0.400pt}}
\put(170.0,853.0){\rule[-0.200pt]{2.409pt}{0.400pt}}
\put(1429.0,853.0){\rule[-0.200pt]{2.409pt}{0.400pt}}
\put(170.0,853.0){\rule[-0.200pt]{2.409pt}{0.400pt}}
\put(1429.0,853.0){\rule[-0.200pt]{2.409pt}{0.400pt}}
\put(170.0,853.0){\rule[-0.200pt]{2.409pt}{0.400pt}}
\put(1429.0,853.0){\rule[-0.200pt]{2.409pt}{0.400pt}}
\put(170.0,853.0){\rule[-0.200pt]{2.409pt}{0.400pt}}
\put(1429.0,853.0){\rule[-0.200pt]{2.409pt}{0.400pt}}
\put(170.0,853.0){\rule[-0.200pt]{2.409pt}{0.400pt}}
\put(1429.0,853.0){\rule[-0.200pt]{2.409pt}{0.400pt}}
\put(170.0,853.0){\rule[-0.200pt]{2.409pt}{0.400pt}}
\put(1429.0,853.0){\rule[-0.200pt]{2.409pt}{0.400pt}}
\put(170.0,853.0){\rule[-0.200pt]{2.409pt}{0.400pt}}
\put(1429.0,853.0){\rule[-0.200pt]{2.409pt}{0.400pt}}
\put(170.0,853.0){\rule[-0.200pt]{2.409pt}{0.400pt}}
\put(1429.0,853.0){\rule[-0.200pt]{2.409pt}{0.400pt}}
\put(170.0,854.0){\rule[-0.200pt]{2.409pt}{0.400pt}}
\put(1429.0,854.0){\rule[-0.200pt]{2.409pt}{0.400pt}}
\put(170.0,854.0){\rule[-0.200pt]{2.409pt}{0.400pt}}
\put(1429.0,854.0){\rule[-0.200pt]{2.409pt}{0.400pt}}
\put(170.0,854.0){\rule[-0.200pt]{2.409pt}{0.400pt}}
\put(1429.0,854.0){\rule[-0.200pt]{2.409pt}{0.400pt}}
\put(170.0,854.0){\rule[-0.200pt]{2.409pt}{0.400pt}}
\put(1429.0,854.0){\rule[-0.200pt]{2.409pt}{0.400pt}}
\put(170.0,854.0){\rule[-0.200pt]{2.409pt}{0.400pt}}
\put(1429.0,854.0){\rule[-0.200pt]{2.409pt}{0.400pt}}
\put(170.0,854.0){\rule[-0.200pt]{2.409pt}{0.400pt}}
\put(1429.0,854.0){\rule[-0.200pt]{2.409pt}{0.400pt}}
\put(170.0,854.0){\rule[-0.200pt]{2.409pt}{0.400pt}}
\put(1429.0,854.0){\rule[-0.200pt]{2.409pt}{0.400pt}}
\put(170.0,854.0){\rule[-0.200pt]{2.409pt}{0.400pt}}
\put(1429.0,854.0){\rule[-0.200pt]{2.409pt}{0.400pt}}
\put(170.0,854.0){\rule[-0.200pt]{2.409pt}{0.400pt}}
\put(1429.0,854.0){\rule[-0.200pt]{2.409pt}{0.400pt}}
\put(170.0,854.0){\rule[-0.200pt]{2.409pt}{0.400pt}}
\put(1429.0,854.0){\rule[-0.200pt]{2.409pt}{0.400pt}}
\put(170.0,854.0){\rule[-0.200pt]{2.409pt}{0.400pt}}
\put(1429.0,854.0){\rule[-0.200pt]{2.409pt}{0.400pt}}
\put(170.0,854.0){\rule[-0.200pt]{2.409pt}{0.400pt}}
\put(1429.0,854.0){\rule[-0.200pt]{2.409pt}{0.400pt}}
\put(170.0,854.0){\rule[-0.200pt]{2.409pt}{0.400pt}}
\put(1429.0,854.0){\rule[-0.200pt]{2.409pt}{0.400pt}}
\put(170.0,854.0){\rule[-0.200pt]{2.409pt}{0.400pt}}
\put(1429.0,854.0){\rule[-0.200pt]{2.409pt}{0.400pt}}
\put(170.0,855.0){\rule[-0.200pt]{2.409pt}{0.400pt}}
\put(1429.0,855.0){\rule[-0.200pt]{2.409pt}{0.400pt}}
\put(170.0,855.0){\rule[-0.200pt]{2.409pt}{0.400pt}}
\put(1429.0,855.0){\rule[-0.200pt]{2.409pt}{0.400pt}}
\put(170.0,855.0){\rule[-0.200pt]{2.409pt}{0.400pt}}
\put(1429.0,855.0){\rule[-0.200pt]{2.409pt}{0.400pt}}
\put(170.0,855.0){\rule[-0.200pt]{2.409pt}{0.400pt}}
\put(1429.0,855.0){\rule[-0.200pt]{2.409pt}{0.400pt}}
\put(170.0,855.0){\rule[-0.200pt]{2.409pt}{0.400pt}}
\put(1429.0,855.0){\rule[-0.200pt]{2.409pt}{0.400pt}}
\put(170.0,855.0){\rule[-0.200pt]{2.409pt}{0.400pt}}
\put(1429.0,855.0){\rule[-0.200pt]{2.409pt}{0.400pt}}
\put(170.0,855.0){\rule[-0.200pt]{2.409pt}{0.400pt}}
\put(1429.0,855.0){\rule[-0.200pt]{2.409pt}{0.400pt}}
\put(170.0,855.0){\rule[-0.200pt]{2.409pt}{0.400pt}}
\put(1429.0,855.0){\rule[-0.200pt]{2.409pt}{0.400pt}}
\put(170.0,855.0){\rule[-0.200pt]{2.409pt}{0.400pt}}
\put(1429.0,855.0){\rule[-0.200pt]{2.409pt}{0.400pt}}
\put(170.0,855.0){\rule[-0.200pt]{2.409pt}{0.400pt}}
\put(1429.0,855.0){\rule[-0.200pt]{2.409pt}{0.400pt}}
\put(170.0,855.0){\rule[-0.200pt]{2.409pt}{0.400pt}}
\put(1429.0,855.0){\rule[-0.200pt]{2.409pt}{0.400pt}}
\put(170.0,855.0){\rule[-0.200pt]{2.409pt}{0.400pt}}
\put(1429.0,855.0){\rule[-0.200pt]{2.409pt}{0.400pt}}
\put(170.0,855.0){\rule[-0.200pt]{2.409pt}{0.400pt}}
\put(1429.0,855.0){\rule[-0.200pt]{2.409pt}{0.400pt}}
\put(170.0,855.0){\rule[-0.200pt]{2.409pt}{0.400pt}}
\put(1429.0,855.0){\rule[-0.200pt]{2.409pt}{0.400pt}}
\put(170.0,856.0){\rule[-0.200pt]{2.409pt}{0.400pt}}
\put(1429.0,856.0){\rule[-0.200pt]{2.409pt}{0.400pt}}
\put(170.0,856.0){\rule[-0.200pt]{2.409pt}{0.400pt}}
\put(1429.0,856.0){\rule[-0.200pt]{2.409pt}{0.400pt}}
\put(170.0,856.0){\rule[-0.200pt]{2.409pt}{0.400pt}}
\put(1429.0,856.0){\rule[-0.200pt]{2.409pt}{0.400pt}}
\put(170.0,856.0){\rule[-0.200pt]{2.409pt}{0.400pt}}
\put(1429.0,856.0){\rule[-0.200pt]{2.409pt}{0.400pt}}
\put(170.0,856.0){\rule[-0.200pt]{2.409pt}{0.400pt}}
\put(1429.0,856.0){\rule[-0.200pt]{2.409pt}{0.400pt}}
\put(170.0,856.0){\rule[-0.200pt]{2.409pt}{0.400pt}}
\put(1429.0,856.0){\rule[-0.200pt]{2.409pt}{0.400pt}}
\put(170.0,856.0){\rule[-0.200pt]{2.409pt}{0.400pt}}
\put(1429.0,856.0){\rule[-0.200pt]{2.409pt}{0.400pt}}
\put(170.0,856.0){\rule[-0.200pt]{2.409pt}{0.400pt}}
\put(1429.0,856.0){\rule[-0.200pt]{2.409pt}{0.400pt}}
\put(170.0,856.0){\rule[-0.200pt]{2.409pt}{0.400pt}}
\put(1429.0,856.0){\rule[-0.200pt]{2.409pt}{0.400pt}}
\put(170.0,856.0){\rule[-0.200pt]{2.409pt}{0.400pt}}
\put(1429.0,856.0){\rule[-0.200pt]{2.409pt}{0.400pt}}
\put(170.0,856.0){\rule[-0.200pt]{2.409pt}{0.400pt}}
\put(1429.0,856.0){\rule[-0.200pt]{2.409pt}{0.400pt}}
\put(170.0,856.0){\rule[-0.200pt]{2.409pt}{0.400pt}}
\put(1429.0,856.0){\rule[-0.200pt]{2.409pt}{0.400pt}}
\put(170.0,856.0){\rule[-0.200pt]{2.409pt}{0.400pt}}
\put(1429.0,856.0){\rule[-0.200pt]{2.409pt}{0.400pt}}
\put(170.0,856.0){\rule[-0.200pt]{2.409pt}{0.400pt}}
\put(1429.0,856.0){\rule[-0.200pt]{2.409pt}{0.400pt}}
\put(170.0,856.0){\rule[-0.200pt]{2.409pt}{0.400pt}}
\put(1429.0,856.0){\rule[-0.200pt]{2.409pt}{0.400pt}}
\put(170.0,856.0){\rule[-0.200pt]{2.409pt}{0.400pt}}
\put(1429.0,856.0){\rule[-0.200pt]{2.409pt}{0.400pt}}
\put(170.0,857.0){\rule[-0.200pt]{2.409pt}{0.400pt}}
\put(1429.0,857.0){\rule[-0.200pt]{2.409pt}{0.400pt}}
\put(170.0,857.0){\rule[-0.200pt]{2.409pt}{0.400pt}}
\put(1429.0,857.0){\rule[-0.200pt]{2.409pt}{0.400pt}}
\put(170.0,857.0){\rule[-0.200pt]{2.409pt}{0.400pt}}
\put(1429.0,857.0){\rule[-0.200pt]{2.409pt}{0.400pt}}
\put(170.0,857.0){\rule[-0.200pt]{2.409pt}{0.400pt}}
\put(1429.0,857.0){\rule[-0.200pt]{2.409pt}{0.400pt}}
\put(170.0,857.0){\rule[-0.200pt]{2.409pt}{0.400pt}}
\put(1429.0,857.0){\rule[-0.200pt]{2.409pt}{0.400pt}}
\put(170.0,857.0){\rule[-0.200pt]{2.409pt}{0.400pt}}
\put(1429.0,857.0){\rule[-0.200pt]{2.409pt}{0.400pt}}
\put(170.0,857.0){\rule[-0.200pt]{2.409pt}{0.400pt}}
\put(1429.0,857.0){\rule[-0.200pt]{2.409pt}{0.400pt}}
\put(170.0,857.0){\rule[-0.200pt]{2.409pt}{0.400pt}}
\put(1429.0,857.0){\rule[-0.200pt]{2.409pt}{0.400pt}}
\put(170.0,857.0){\rule[-0.200pt]{2.409pt}{0.400pt}}
\put(1429.0,857.0){\rule[-0.200pt]{2.409pt}{0.400pt}}
\put(170.0,857.0){\rule[-0.200pt]{2.409pt}{0.400pt}}
\put(1429.0,857.0){\rule[-0.200pt]{2.409pt}{0.400pt}}
\put(170.0,857.0){\rule[-0.200pt]{2.409pt}{0.400pt}}
\put(1429.0,857.0){\rule[-0.200pt]{2.409pt}{0.400pt}}
\put(170.0,857.0){\rule[-0.200pt]{2.409pt}{0.400pt}}
\put(1429.0,857.0){\rule[-0.200pt]{2.409pt}{0.400pt}}
\put(170.0,857.0){\rule[-0.200pt]{2.409pt}{0.400pt}}
\put(1429.0,857.0){\rule[-0.200pt]{2.409pt}{0.400pt}}
\put(170.0,857.0){\rule[-0.200pt]{2.409pt}{0.400pt}}
\put(1429.0,857.0){\rule[-0.200pt]{2.409pt}{0.400pt}}
\put(170.0,857.0){\rule[-0.200pt]{2.409pt}{0.400pt}}
\put(1429.0,857.0){\rule[-0.200pt]{2.409pt}{0.400pt}}
\put(170.0,857.0){\rule[-0.200pt]{2.409pt}{0.400pt}}
\put(1429.0,857.0){\rule[-0.200pt]{2.409pt}{0.400pt}}
\put(170.0,858.0){\rule[-0.200pt]{2.409pt}{0.400pt}}
\put(1429.0,858.0){\rule[-0.200pt]{2.409pt}{0.400pt}}
\put(170.0,858.0){\rule[-0.200pt]{2.409pt}{0.400pt}}
\put(1429.0,858.0){\rule[-0.200pt]{2.409pt}{0.400pt}}
\put(170.0,858.0){\rule[-0.200pt]{2.409pt}{0.400pt}}
\put(1429.0,858.0){\rule[-0.200pt]{2.409pt}{0.400pt}}
\put(170.0,858.0){\rule[-0.200pt]{2.409pt}{0.400pt}}
\put(1429.0,858.0){\rule[-0.200pt]{2.409pt}{0.400pt}}
\put(170.0,858.0){\rule[-0.200pt]{2.409pt}{0.400pt}}
\put(1429.0,858.0){\rule[-0.200pt]{2.409pt}{0.400pt}}
\put(170.0,858.0){\rule[-0.200pt]{2.409pt}{0.400pt}}
\put(1429.0,858.0){\rule[-0.200pt]{2.409pt}{0.400pt}}
\put(170.0,858.0){\rule[-0.200pt]{2.409pt}{0.400pt}}
\put(1429.0,858.0){\rule[-0.200pt]{2.409pt}{0.400pt}}
\put(170.0,858.0){\rule[-0.200pt]{2.409pt}{0.400pt}}
\put(1429.0,858.0){\rule[-0.200pt]{2.409pt}{0.400pt}}
\put(170.0,858.0){\rule[-0.200pt]{2.409pt}{0.400pt}}
\put(1429.0,858.0){\rule[-0.200pt]{2.409pt}{0.400pt}}
\put(170.0,858.0){\rule[-0.200pt]{2.409pt}{0.400pt}}
\put(1429.0,858.0){\rule[-0.200pt]{2.409pt}{0.400pt}}
\put(170.0,858.0){\rule[-0.200pt]{2.409pt}{0.400pt}}
\put(1429.0,858.0){\rule[-0.200pt]{2.409pt}{0.400pt}}
\put(170.0,858.0){\rule[-0.200pt]{2.409pt}{0.400pt}}
\put(1429.0,858.0){\rule[-0.200pt]{2.409pt}{0.400pt}}
\put(170.0,858.0){\rule[-0.200pt]{2.409pt}{0.400pt}}
\put(1429.0,858.0){\rule[-0.200pt]{2.409pt}{0.400pt}}
\put(170.0,858.0){\rule[-0.200pt]{2.409pt}{0.400pt}}
\put(1429.0,858.0){\rule[-0.200pt]{2.409pt}{0.400pt}}
\put(170.0,858.0){\rule[-0.200pt]{2.409pt}{0.400pt}}
\put(1429.0,858.0){\rule[-0.200pt]{2.409pt}{0.400pt}}
\put(170.0,858.0){\rule[-0.200pt]{2.409pt}{0.400pt}}
\put(1429.0,858.0){\rule[-0.200pt]{2.409pt}{0.400pt}}
\put(170.0,859.0){\rule[-0.200pt]{2.409pt}{0.400pt}}
\put(1429.0,859.0){\rule[-0.200pt]{2.409pt}{0.400pt}}
\put(170.0,859.0){\rule[-0.200pt]{2.409pt}{0.400pt}}
\put(1429.0,859.0){\rule[-0.200pt]{2.409pt}{0.400pt}}
\put(170.0,859.0){\rule[-0.200pt]{2.409pt}{0.400pt}}
\put(1429.0,859.0){\rule[-0.200pt]{2.409pt}{0.400pt}}
\put(170.0,859.0){\rule[-0.200pt]{2.409pt}{0.400pt}}
\put(1429.0,859.0){\rule[-0.200pt]{2.409pt}{0.400pt}}
\put(170.0,859.0){\rule[-0.200pt]{2.409pt}{0.400pt}}
\put(1429.0,859.0){\rule[-0.200pt]{2.409pt}{0.400pt}}
\put(170.0,859.0){\rule[-0.200pt]{2.409pt}{0.400pt}}
\put(1429.0,859.0){\rule[-0.200pt]{2.409pt}{0.400pt}}
\put(170.0,859.0){\rule[-0.200pt]{2.409pt}{0.400pt}}
\put(1429.0,859.0){\rule[-0.200pt]{2.409pt}{0.400pt}}
\put(170.0,859.0){\rule[-0.200pt]{2.409pt}{0.400pt}}
\put(1429.0,859.0){\rule[-0.200pt]{2.409pt}{0.400pt}}
\put(170.0,859.0){\rule[-0.200pt]{2.409pt}{0.400pt}}
\put(1429.0,859.0){\rule[-0.200pt]{2.409pt}{0.400pt}}
\put(170.0,859.0){\rule[-0.200pt]{4.818pt}{0.400pt}}
\put(150,859){\makebox(0,0)[r]{ 1000}}
\put(1419.0,859.0){\rule[-0.200pt]{4.818pt}{0.400pt}}
\put(170.0,82.0){\rule[-0.200pt]{0.400pt}{4.818pt}}
\put(170,41){\makebox(0,0){ 0}}
\put(170.0,839.0){\rule[-0.200pt]{0.400pt}{4.818pt}}
\put(424.0,82.0){\rule[-0.200pt]{0.400pt}{4.818pt}}
\put(424,41){\makebox(0,0){ 100}}
\put(424.0,839.0){\rule[-0.200pt]{0.400pt}{4.818pt}}
\put(678.0,82.0){\rule[-0.200pt]{0.400pt}{4.818pt}}
\put(678,41){\makebox(0,0){ 200}}
\put(678.0,839.0){\rule[-0.200pt]{0.400pt}{4.818pt}}
\put(931.0,82.0){\rule[-0.200pt]{0.400pt}{4.818pt}}
\put(931,41){\makebox(0,0){ 300}}
\put(931.0,839.0){\rule[-0.200pt]{0.400pt}{4.818pt}}
\put(1185.0,82.0){\rule[-0.200pt]{0.400pt}{4.818pt}}
\put(1185,41){\makebox(0,0){ 400}}
\put(1185.0,839.0){\rule[-0.200pt]{0.400pt}{4.818pt}}
\put(1439.0,82.0){\rule[-0.200pt]{0.400pt}{4.818pt}}
\put(1439,41){\makebox(0,0){ 500}}
\put(1439.0,839.0){\rule[-0.200pt]{0.400pt}{4.818pt}}
\put(170.0,82.0){\rule[-0.200pt]{0.400pt}{187.179pt}}
\put(170.0,82.0){\rule[-0.200pt]{305.702pt}{0.400pt}}
\put(1439.0,82.0){\rule[-0.200pt]{0.400pt}{187.179pt}}
\put(170.0,859.0){\rule[-0.200pt]{305.702pt}{0.400pt}}
\put(1279,819){\makebox(0,0)[r]{algorytm naturalny}}
\put(1299.0,819.0){\rule[-0.200pt]{24.090pt}{0.400pt}}
\put(175,93){\usebox{\plotpoint}}
\multiput(175.58,93.00)(0.493,2.955){23}{\rule{0.119pt}{2.408pt}}
\multiput(174.17,93.00)(13.000,70.003){2}{\rule{0.400pt}{1.204pt}}
\multiput(188.58,168.00)(0.493,1.369){23}{\rule{0.119pt}{1.177pt}}
\multiput(187.17,168.00)(13.000,32.557){2}{\rule{0.400pt}{0.588pt}}
\multiput(201.58,203.00)(0.493,0.734){23}{\rule{0.119pt}{0.685pt}}
\multiput(200.17,203.00)(13.000,17.579){2}{\rule{0.400pt}{0.342pt}}
\multiput(214.00,222.58)(0.497,0.493){23}{\rule{0.500pt}{0.119pt}}
\multiput(214.00,221.17)(11.962,13.000){2}{\rule{0.250pt}{0.400pt}}
\multiput(227.00,235.59)(0.824,0.488){13}{\rule{0.750pt}{0.117pt}}
\multiput(227.00,234.17)(11.443,8.000){2}{\rule{0.375pt}{0.400pt}}
\multiput(240.00,243.59)(1.123,0.482){9}{\rule{0.967pt}{0.116pt}}
\multiput(240.00,242.17)(10.994,6.000){2}{\rule{0.483pt}{0.400pt}}
\multiput(253.00,249.59)(1.378,0.477){7}{\rule{1.140pt}{0.115pt}}
\multiput(253.00,248.17)(10.634,5.000){2}{\rule{0.570pt}{0.400pt}}
\multiput(266.00,254.59)(1.123,0.482){9}{\rule{0.967pt}{0.116pt}}
\multiput(266.00,253.17)(10.994,6.000){2}{\rule{0.483pt}{0.400pt}}
\multiput(279.00,260.59)(1.378,0.477){7}{\rule{1.140pt}{0.115pt}}
\multiput(279.00,259.17)(10.634,5.000){2}{\rule{0.570pt}{0.400pt}}
\multiput(292.00,265.60)(1.797,0.468){5}{\rule{1.400pt}{0.113pt}}
\multiput(292.00,264.17)(10.094,4.000){2}{\rule{0.700pt}{0.400pt}}
\put(305,269.17){\rule{2.700pt}{0.400pt}}
\multiput(305.00,268.17)(7.396,2.000){2}{\rule{1.350pt}{0.400pt}}
\put(331,269.67){\rule{3.132pt}{0.400pt}}
\multiput(331.00,270.17)(6.500,-1.000){2}{\rule{1.566pt}{0.400pt}}
\put(344,268.67){\rule{3.132pt}{0.400pt}}
\multiput(344.00,269.17)(6.500,-1.000){2}{\rule{1.566pt}{0.400pt}}
\put(318.0,271.0){\rule[-0.200pt]{3.132pt}{0.400pt}}
\put(383,269.17){\rule{2.700pt}{0.400pt}}
\multiput(383.00,268.17)(7.396,2.000){2}{\rule{1.350pt}{0.400pt}}
\put(396,271.17){\rule{2.700pt}{0.400pt}}
\multiput(396.00,270.17)(7.396,2.000){2}{\rule{1.350pt}{0.400pt}}
\put(409,273.17){\rule{2.900pt}{0.400pt}}
\multiput(409.00,272.17)(7.981,2.000){2}{\rule{1.450pt}{0.400pt}}
\multiput(423.00,275.61)(2.695,0.447){3}{\rule{1.833pt}{0.108pt}}
\multiput(423.00,274.17)(9.195,3.000){2}{\rule{0.917pt}{0.400pt}}
\multiput(436.00,278.61)(2.695,0.447){3}{\rule{1.833pt}{0.108pt}}
\multiput(436.00,277.17)(9.195,3.000){2}{\rule{0.917pt}{0.400pt}}
\put(449,281.17){\rule{2.700pt}{0.400pt}}
\multiput(449.00,280.17)(7.396,2.000){2}{\rule{1.350pt}{0.400pt}}
\multiput(462.00,283.61)(2.695,0.447){3}{\rule{1.833pt}{0.108pt}}
\multiput(462.00,282.17)(9.195,3.000){2}{\rule{0.917pt}{0.400pt}}
\put(475,286.17){\rule{2.700pt}{0.400pt}}
\multiput(475.00,285.17)(7.396,2.000){2}{\rule{1.350pt}{0.400pt}}
\put(488,288.17){\rule{2.700pt}{0.400pt}}
\multiput(488.00,287.17)(7.396,2.000){2}{\rule{1.350pt}{0.400pt}}
\put(501,289.67){\rule{3.132pt}{0.400pt}}
\multiput(501.00,289.17)(6.500,1.000){2}{\rule{1.566pt}{0.400pt}}
\put(514,290.67){\rule{3.132pt}{0.400pt}}
\multiput(514.00,290.17)(6.500,1.000){2}{\rule{1.566pt}{0.400pt}}
\put(527,291.67){\rule{3.132pt}{0.400pt}}
\multiput(527.00,291.17)(6.500,1.000){2}{\rule{1.566pt}{0.400pt}}
\put(357.0,269.0){\rule[-0.200pt]{6.263pt}{0.400pt}}
\put(579,292.67){\rule{3.132pt}{0.400pt}}
\multiput(579.00,292.17)(6.500,1.000){2}{\rule{1.566pt}{0.400pt}}
\put(540.0,293.0){\rule[-0.200pt]{9.395pt}{0.400pt}}
\put(605,293.67){\rule{3.132pt}{0.400pt}}
\multiput(605.00,293.17)(6.500,1.000){2}{\rule{1.566pt}{0.400pt}}
\put(618,294.67){\rule{3.132pt}{0.400pt}}
\multiput(618.00,294.17)(6.500,1.000){2}{\rule{1.566pt}{0.400pt}}
\put(631,295.67){\rule{3.132pt}{0.400pt}}
\multiput(631.00,295.17)(6.500,1.000){2}{\rule{1.566pt}{0.400pt}}
\put(644,296.67){\rule{3.132pt}{0.400pt}}
\multiput(644.00,296.17)(6.500,1.000){2}{\rule{1.566pt}{0.400pt}}
\put(657,298.17){\rule{2.700pt}{0.400pt}}
\multiput(657.00,297.17)(7.396,2.000){2}{\rule{1.350pt}{0.400pt}}
\put(670,300.17){\rule{2.700pt}{0.400pt}}
\multiput(670.00,299.17)(7.396,2.000){2}{\rule{1.350pt}{0.400pt}}
\put(683,302.17){\rule{2.700pt}{0.400pt}}
\multiput(683.00,301.17)(7.396,2.000){2}{\rule{1.350pt}{0.400pt}}
\put(696,304.17){\rule{2.700pt}{0.400pt}}
\multiput(696.00,303.17)(7.396,2.000){2}{\rule{1.350pt}{0.400pt}}
\put(709,306.17){\rule{2.700pt}{0.400pt}}
\multiput(709.00,305.17)(7.396,2.000){2}{\rule{1.350pt}{0.400pt}}
\put(722,308.17){\rule{2.700pt}{0.400pt}}
\multiput(722.00,307.17)(7.396,2.000){2}{\rule{1.350pt}{0.400pt}}
\put(735,309.67){\rule{3.132pt}{0.400pt}}
\multiput(735.00,309.17)(6.500,1.000){2}{\rule{1.566pt}{0.400pt}}
\put(748,311.17){\rule{2.700pt}{0.400pt}}
\multiput(748.00,310.17)(7.396,2.000){2}{\rule{1.350pt}{0.400pt}}
\put(592.0,294.0){\rule[-0.200pt]{3.132pt}{0.400pt}}
\put(774,312.67){\rule{3.132pt}{0.400pt}}
\multiput(774.00,312.17)(6.500,1.000){2}{\rule{1.566pt}{0.400pt}}
\put(787,312.67){\rule{3.132pt}{0.400pt}}
\multiput(787.00,313.17)(6.500,-1.000){2}{\rule{1.566pt}{0.400pt}}
\put(761.0,313.0){\rule[-0.200pt]{3.132pt}{0.400pt}}
\put(826,311.67){\rule{3.132pt}{0.400pt}}
\multiput(826.00,312.17)(6.500,-1.000){2}{\rule{1.566pt}{0.400pt}}
\put(800.0,313.0){\rule[-0.200pt]{6.263pt}{0.400pt}}
\put(878,311.67){\rule{3.132pt}{0.400pt}}
\multiput(878.00,311.17)(6.500,1.000){2}{\rule{1.566pt}{0.400pt}}
\put(891,313.17){\rule{2.700pt}{0.400pt}}
\multiput(891.00,312.17)(7.396,2.000){2}{\rule{1.350pt}{0.400pt}}
\put(904,315.17){\rule{2.700pt}{0.400pt}}
\multiput(904.00,314.17)(7.396,2.000){2}{\rule{1.350pt}{0.400pt}}
\put(917,317.17){\rule{2.700pt}{0.400pt}}
\multiput(917.00,316.17)(7.396,2.000){2}{\rule{1.350pt}{0.400pt}}
\multiput(930.00,319.61)(2.695,0.447){3}{\rule{1.833pt}{0.108pt}}
\multiput(930.00,318.17)(9.195,3.000){2}{\rule{0.917pt}{0.400pt}}
\put(943,322.17){\rule{2.700pt}{0.400pt}}
\multiput(943.00,321.17)(7.396,2.000){2}{\rule{1.350pt}{0.400pt}}
\put(956,324.17){\rule{2.700pt}{0.400pt}}
\multiput(956.00,323.17)(7.396,2.000){2}{\rule{1.350pt}{0.400pt}}
\put(969,326.17){\rule{2.900pt}{0.400pt}}
\multiput(969.00,325.17)(7.981,2.000){2}{\rule{1.450pt}{0.400pt}}
\put(983,327.67){\rule{3.132pt}{0.400pt}}
\multiput(983.00,327.17)(6.500,1.000){2}{\rule{1.566pt}{0.400pt}}
\put(996,328.67){\rule{3.132pt}{0.400pt}}
\multiput(996.00,328.17)(6.500,1.000){2}{\rule{1.566pt}{0.400pt}}
\put(1009,328.67){\rule{3.132pt}{0.400pt}}
\multiput(1009.00,329.17)(6.500,-1.000){2}{\rule{1.566pt}{0.400pt}}
\put(839.0,312.0){\rule[-0.200pt]{9.395pt}{0.400pt}}
\put(1035,327.67){\rule{3.132pt}{0.400pt}}
\multiput(1035.00,328.17)(6.500,-1.000){2}{\rule{1.566pt}{0.400pt}}
\put(1048,326.67){\rule{3.132pt}{0.400pt}}
\multiput(1048.00,327.17)(6.500,-1.000){2}{\rule{1.566pt}{0.400pt}}
\put(1022.0,329.0){\rule[-0.200pt]{3.132pt}{0.400pt}}
\put(1087,326.67){\rule{3.132pt}{0.400pt}}
\multiput(1087.00,326.17)(6.500,1.000){2}{\rule{1.566pt}{0.400pt}}
\put(1100,327.67){\rule{3.132pt}{0.400pt}}
\multiput(1100.00,327.17)(6.500,1.000){2}{\rule{1.566pt}{0.400pt}}
\put(1113,329.17){\rule{2.700pt}{0.400pt}}
\multiput(1113.00,328.17)(7.396,2.000){2}{\rule{1.350pt}{0.400pt}}
\put(1126,331.17){\rule{2.700pt}{0.400pt}}
\multiput(1126.00,330.17)(7.396,2.000){2}{\rule{1.350pt}{0.400pt}}
\put(1139,333.17){\rule{2.700pt}{0.400pt}}
\multiput(1139.00,332.17)(7.396,2.000){2}{\rule{1.350pt}{0.400pt}}
\put(1152,334.67){\rule{3.132pt}{0.400pt}}
\multiput(1152.00,334.17)(6.500,1.000){2}{\rule{1.566pt}{0.400pt}}
\put(1061.0,327.0){\rule[-0.200pt]{6.263pt}{0.400pt}}
\put(1191,334.67){\rule{3.132pt}{0.400pt}}
\multiput(1191.00,335.17)(6.500,-1.000){2}{\rule{1.566pt}{0.400pt}}
\put(1204,333.67){\rule{3.132pt}{0.400pt}}
\multiput(1204.00,334.17)(6.500,-1.000){2}{\rule{1.566pt}{0.400pt}}
\put(1217,332.67){\rule{3.132pt}{0.400pt}}
\multiput(1217.00,333.17)(6.500,-1.000){2}{\rule{1.566pt}{0.400pt}}
\put(1230,331.67){\rule{3.132pt}{0.400pt}}
\multiput(1230.00,332.17)(6.500,-1.000){2}{\rule{1.566pt}{0.400pt}}
\put(1243,330.67){\rule{3.132pt}{0.400pt}}
\multiput(1243.00,331.17)(6.500,-1.000){2}{\rule{1.566pt}{0.400pt}}
\put(1256,329.67){\rule{3.132pt}{0.400pt}}
\multiput(1256.00,330.17)(6.500,-1.000){2}{\rule{1.566pt}{0.400pt}}
\put(1269,328.67){\rule{3.132pt}{0.400pt}}
\multiput(1269.00,329.17)(6.500,-1.000){2}{\rule{1.566pt}{0.400pt}}
\put(1165.0,336.0){\rule[-0.200pt]{6.263pt}{0.400pt}}
\put(1295,327.67){\rule{3.132pt}{0.400pt}}
\multiput(1295.00,328.17)(6.500,-1.000){2}{\rule{1.566pt}{0.400pt}}
\put(1308,327.67){\rule{3.132pt}{0.400pt}}
\multiput(1308.00,327.17)(6.500,1.000){2}{\rule{1.566pt}{0.400pt}}
\put(1321,329.17){\rule{2.700pt}{0.400pt}}
\multiput(1321.00,328.17)(7.396,2.000){2}{\rule{1.350pt}{0.400pt}}
\multiput(1334.00,331.61)(2.695,0.447){3}{\rule{1.833pt}{0.108pt}}
\multiput(1334.00,330.17)(9.195,3.000){2}{\rule{0.917pt}{0.400pt}}
\multiput(1347.00,334.60)(1.797,0.468){5}{\rule{1.400pt}{0.113pt}}
\multiput(1347.00,333.17)(10.094,4.000){2}{\rule{0.700pt}{0.400pt}}
\multiput(1360.00,338.61)(2.695,0.447){3}{\rule{1.833pt}{0.108pt}}
\multiput(1360.00,337.17)(9.195,3.000){2}{\rule{0.917pt}{0.400pt}}
\put(1373,340.67){\rule{3.132pt}{0.400pt}}
\multiput(1373.00,340.17)(6.500,1.000){2}{\rule{1.566pt}{0.400pt}}
\put(1386,340.67){\rule{3.132pt}{0.400pt}}
\multiput(1386.00,341.17)(6.500,-1.000){2}{\rule{1.566pt}{0.400pt}}
\multiput(1399.00,339.94)(1.797,-0.468){5}{\rule{1.400pt}{0.113pt}}
\multiput(1399.00,340.17)(10.094,-4.000){2}{\rule{0.700pt}{0.400pt}}
\multiput(1412.00,335.93)(1.123,-0.482){9}{\rule{0.967pt}{0.116pt}}
\multiput(1412.00,336.17)(10.994,-6.000){2}{\rule{0.483pt}{0.400pt}}
\put(1425,329.17){\rule{2.700pt}{0.400pt}}
\multiput(1425.00,330.17)(7.396,-2.000){2}{\rule{1.350pt}{0.400pt}}
\put(1282.0,329.0){\rule[-0.200pt]{3.132pt}{0.400pt}}
\put(1438.0,329.0){\usebox{\plotpoint}}
\sbox{\plotpoint}{\rule[-0.500pt]{1.000pt}{1.000pt}}%
\sbox{\plotpoint}{\rule[-0.200pt]{0.400pt}{0.400pt}}%
\put(1279,778){\makebox(0,0)[r]{algorytm Strassena}}
\sbox{\plotpoint}{\rule[-0.500pt]{1.000pt}{1.000pt}}%
\multiput(1299,778)(20.756,0.000){5}{\usebox{\plotpoint}}
\put(1399,778){\usebox{\plotpoint}}
\put(175,146){\usebox{\plotpoint}}
\multiput(175,146)(2.574,20.595){6}{\usebox{\plotpoint}}
\multiput(188,250)(4.011,20.364){3}{\usebox{\plotpoint}}
\multiput(201,316)(6.006,19.867){2}{\usebox{\plotpoint}}
\multiput(214,359)(7.812,19.229){2}{\usebox{\plotpoint}}
\put(235.43,407.22){\usebox{\plotpoint}}
\put(246.13,424.96){\usebox{\plotpoint}}
\put(258.42,441.67){\usebox{\plotpoint}}
\put(271.72,457.60){\usebox{\plotpoint}}
\put(285.82,472.82){\usebox{\plotpoint}}
\put(301.87,485.84){\usebox{\plotpoint}}
\put(319.58,496.61){\usebox{\plotpoint}}
\put(339.14,503.50){\usebox{\plotpoint}}
\put(358.67,510.51){\usebox{\plotpoint}}
\put(378.08,517.73){\usebox{\plotpoint}}
\put(396.92,526.43){\usebox{\plotpoint}}
\put(415.85,534.94){\usebox{\plotpoint}}
\put(434.43,544.15){\usebox{\plotpoint}}
\put(453.22,552.95){\usebox{\plotpoint}}
\put(472.35,560.98){\usebox{\plotpoint}}
\put(491.72,568.43){\usebox{\plotpoint}}
\put(511.34,575.18){\usebox{\plotpoint}}
\put(531.32,580.66){\usebox{\plotpoint}}
\put(551.66,584.69){\usebox{\plotpoint}}
\put(572.16,587.95){\usebox{\plotpoint}}
\put(592.67,591.10){\usebox{\plotpoint}}
\put(613.19,594.26){\usebox{\plotpoint}}
\put(633.66,597.61){\usebox{\plotpoint}}
\put(653.89,602.28){\usebox{\plotpoint}}
\put(674.11,606.95){\usebox{\plotpoint}}
\put(694.33,611.62){\usebox{\plotpoint}}
\put(714.31,617.22){\usebox{\plotpoint}}
\put(734.53,621.89){\usebox{\plotpoint}}
\put(754.75,626.56){\usebox{\plotpoint}}
\put(775.18,630.18){\usebox{\plotpoint}}
\put(795.77,632.67){\usebox{\plotpoint}}
\put(816.35,635.26){\usebox{\plotpoint}}
\put(837.04,636.85){\usebox{\plotpoint}}
\put(857.62,639.43){\usebox{\plotpoint}}
\put(878.20,642.03){\usebox{\plotpoint}}
\put(898.72,645.19){\usebox{\plotpoint}}
\put(919.01,649.46){\usebox{\plotpoint}}
\put(939.37,653.44){\usebox{\plotpoint}}
\put(959.70,657.57){\usebox{\plotpoint}}
\put(980.23,660.60){\usebox{\plotpoint}}
\put(1000.90,662.38){\usebox{\plotpoint}}
\put(1021.60,663.97){\usebox{\plotpoint}}
\put(1042.33,664.56){\usebox{\plotpoint}}
\put(1063.03,666.16){\usebox{\plotpoint}}
\put(1083.64,668.48){\usebox{\plotpoint}}
\put(1103.91,672.90){\usebox{\plotpoint}}
\put(1124.13,677.57){\usebox{\plotpoint}}
\put(1144.18,682.80){\usebox{\plotpoint}}
\put(1164.51,686.89){\usebox{\plotpoint}}
\put(1185.21,688.00){\usebox{\plotpoint}}
\put(1205.96,687.85){\usebox{\plotpoint}}
\put(1226.66,686.26){\usebox{\plotpoint}}
\put(1247.35,684.67){\usebox{\plotpoint}}
\put(1267.94,682.16){\usebox{\plotpoint}}
\put(1288.57,679.99){\usebox{\plotpoint}}
\put(1309.23,679.19){\usebox{\plotpoint}}
\put(1329.63,682.99){\usebox{\plotpoint}}
\put(1349.12,689.98){\usebox{\plotpoint}}
\put(1367.97,698.68){\usebox{\plotpoint}}
\put(1387.61,705.12){\usebox{\plotpoint}}
\put(1407.92,703.26){\usebox{\plotpoint}}
\put(1427.52,696.61){\usebox{\plotpoint}}
\put(1439,696){\usebox{\plotpoint}}
\sbox{\plotpoint}{\rule[-0.600pt]{1.200pt}{1.200pt}}%
\sbox{\plotpoint}{\rule[-0.200pt]{0.400pt}{0.400pt}}%
\put(1279,737){\makebox(0,0)[r]{algorytm z progiem}}
\sbox{\plotpoint}{\rule[-0.600pt]{1.200pt}{1.200pt}}%
\put(1299.0,737.0){\rule[-0.600pt]{24.090pt}{1.200pt}}
\put(175,93){\usebox{\plotpoint}}
\multiput(177.24,93.00)(0.501,3.001){16}{\rule{0.121pt}{7.223pt}}
\multiput(172.51,93.00)(13.000,60.008){2}{\rule{1.200pt}{3.612pt}}
\multiput(190.24,168.00)(0.501,1.356){16}{\rule{0.121pt}{3.531pt}}
\multiput(185.51,168.00)(13.000,27.672){2}{\rule{1.200pt}{1.765pt}}
\multiput(203.24,203.00)(0.501,0.698){16}{\rule{0.121pt}{2.054pt}}
\multiput(198.51,203.00)(13.000,14.737){2}{\rule{1.200pt}{1.027pt}}
\multiput(214.00,224.24)(0.452,0.501){16}{\rule{1.500pt}{0.121pt}}
\multiput(214.00,219.51)(9.887,13.000){2}{\rule{0.750pt}{1.200pt}}
\multiput(227.00,237.24)(0.732,0.503){6}{\rule{2.250pt}{0.121pt}}
\multiput(227.00,232.51)(8.330,8.000){2}{\rule{1.125pt}{1.200pt}}
\multiput(240.00,245.24)(0.962,0.509){2}{\rule{2.900pt}{0.123pt}}
\multiput(240.00,240.51)(6.981,6.000){2}{\rule{1.450pt}{1.200pt}}
\put(253,249.01){\rule{3.132pt}{1.200pt}}
\multiput(253.00,246.51)(6.500,5.000){2}{\rule{1.566pt}{1.200pt}}
\multiput(266.00,256.24)(0.962,0.509){2}{\rule{2.900pt}{0.123pt}}
\multiput(266.00,251.51)(6.981,6.000){2}{\rule{1.450pt}{1.200pt}}
\put(279,260.01){\rule{3.132pt}{1.200pt}}
\multiput(279.00,257.51)(6.500,5.000){2}{\rule{1.566pt}{1.200pt}}
\put(292,264.51){\rule{3.132pt}{1.200pt}}
\multiput(292.00,262.51)(6.500,4.000){2}{\rule{1.566pt}{1.200pt}}
\put(305,267.51){\rule{3.132pt}{1.200pt}}
\multiput(305.00,266.51)(6.500,2.000){2}{\rule{1.566pt}{1.200pt}}
\put(331,268.01){\rule{3.132pt}{1.200pt}}
\multiput(331.00,268.51)(6.500,-1.000){2}{\rule{1.566pt}{1.200pt}}
\put(344,267.01){\rule{3.132pt}{1.200pt}}
\multiput(344.00,267.51)(6.500,-1.000){2}{\rule{1.566pt}{1.200pt}}
\put(318.0,271.0){\rule[-0.600pt]{3.132pt}{1.200pt}}
\put(383,267.51){\rule{3.132pt}{1.200pt}}
\multiput(383.00,266.51)(6.500,2.000){2}{\rule{1.566pt}{1.200pt}}
\put(396,269.51){\rule{3.132pt}{1.200pt}}
\multiput(396.00,268.51)(6.500,2.000){2}{\rule{1.566pt}{1.200pt}}
\put(409,272.01){\rule{3.373pt}{1.200pt}}
\multiput(409.00,270.51)(7.000,3.000){2}{\rule{1.686pt}{1.200pt}}
\put(423,275.01){\rule{3.132pt}{1.200pt}}
\multiput(423.00,273.51)(6.500,3.000){2}{\rule{1.566pt}{1.200pt}}
\put(436,278.01){\rule{3.132pt}{1.200pt}}
\multiput(436.00,276.51)(6.500,3.000){2}{\rule{1.566pt}{1.200pt}}
\put(449,281.01){\rule{3.132pt}{1.200pt}}
\multiput(449.00,279.51)(6.500,3.000){2}{\rule{1.566pt}{1.200pt}}
\put(462,284.51){\rule{3.132pt}{1.200pt}}
\multiput(462.00,282.51)(6.500,4.000){2}{\rule{1.566pt}{1.200pt}}
\put(475,288.01){\rule{3.132pt}{1.200pt}}
\multiput(475.00,286.51)(6.500,3.000){2}{\rule{1.566pt}{1.200pt}}
\put(488,291.01){\rule{3.132pt}{1.200pt}}
\multiput(488.00,289.51)(6.500,3.000){2}{\rule{1.566pt}{1.200pt}}
\put(501,294.01){\rule{3.132pt}{1.200pt}}
\multiput(501.00,292.51)(6.500,3.000){2}{\rule{1.566pt}{1.200pt}}
\put(514,297.01){\rule{3.132pt}{1.200pt}}
\multiput(514.00,295.51)(6.500,3.000){2}{\rule{1.566pt}{1.200pt}}
\put(527,299.51){\rule{3.132pt}{1.200pt}}
\multiput(527.00,298.51)(6.500,2.000){2}{\rule{1.566pt}{1.200pt}}
\put(540,301.51){\rule{3.132pt}{1.200pt}}
\multiput(540.00,300.51)(6.500,2.000){2}{\rule{1.566pt}{1.200pt}}
\put(553,303.51){\rule{3.132pt}{1.200pt}}
\multiput(553.00,302.51)(6.500,2.000){2}{\rule{1.566pt}{1.200pt}}
\put(566,305.01){\rule{3.132pt}{1.200pt}}
\multiput(566.00,304.51)(6.500,1.000){2}{\rule{1.566pt}{1.200pt}}
\put(579,306.01){\rule{3.132pt}{1.200pt}}
\multiput(579.00,305.51)(6.500,1.000){2}{\rule{1.566pt}{1.200pt}}
\put(592,307.51){\rule{3.132pt}{1.200pt}}
\multiput(592.00,306.51)(6.500,2.000){2}{\rule{1.566pt}{1.200pt}}
\put(605,309.01){\rule{3.132pt}{1.200pt}}
\multiput(605.00,308.51)(6.500,1.000){2}{\rule{1.566pt}{1.200pt}}
\put(618,310.51){\rule{3.132pt}{1.200pt}}
\multiput(618.00,309.51)(6.500,2.000){2}{\rule{1.566pt}{1.200pt}}
\put(631,312.51){\rule{3.132pt}{1.200pt}}
\multiput(631.00,311.51)(6.500,2.000){2}{\rule{1.566pt}{1.200pt}}
\put(644,315.01){\rule{3.132pt}{1.200pt}}
\multiput(644.00,313.51)(6.500,3.000){2}{\rule{1.566pt}{1.200pt}}
\put(657,318.01){\rule{3.132pt}{1.200pt}}
\multiput(657.00,316.51)(6.500,3.000){2}{\rule{1.566pt}{1.200pt}}
\put(670,321.01){\rule{3.132pt}{1.200pt}}
\multiput(670.00,319.51)(6.500,3.000){2}{\rule{1.566pt}{1.200pt}}
\put(683,324.01){\rule{3.132pt}{1.200pt}}
\multiput(683.00,322.51)(6.500,3.000){2}{\rule{1.566pt}{1.200pt}}
\put(696,327.01){\rule{3.132pt}{1.200pt}}
\multiput(696.00,325.51)(6.500,3.000){2}{\rule{1.566pt}{1.200pt}}
\put(709,330.51){\rule{3.132pt}{1.200pt}}
\multiput(709.00,328.51)(6.500,4.000){2}{\rule{1.566pt}{1.200pt}}
\put(722,334.01){\rule{3.132pt}{1.200pt}}
\multiput(722.00,332.51)(6.500,3.000){2}{\rule{1.566pt}{1.200pt}}
\put(735,336.51){\rule{3.132pt}{1.200pt}}
\multiput(735.00,335.51)(6.500,2.000){2}{\rule{1.566pt}{1.200pt}}
\put(748,338.51){\rule{3.132pt}{1.200pt}}
\multiput(748.00,337.51)(6.500,2.000){2}{\rule{1.566pt}{1.200pt}}
\put(761,340.51){\rule{3.132pt}{1.200pt}}
\multiput(761.00,339.51)(6.500,2.000){2}{\rule{1.566pt}{1.200pt}}
\put(774,342.01){\rule{3.132pt}{1.200pt}}
\multiput(774.00,341.51)(6.500,1.000){2}{\rule{1.566pt}{1.200pt}}
\put(787,343.01){\rule{3.132pt}{1.200pt}}
\multiput(787.00,342.51)(6.500,1.000){2}{\rule{1.566pt}{1.200pt}}
\put(800,344.01){\rule{3.132pt}{1.200pt}}
\multiput(800.00,343.51)(6.500,1.000){2}{\rule{1.566pt}{1.200pt}}
\put(813,345.01){\rule{3.132pt}{1.200pt}}
\multiput(813.00,344.51)(6.500,1.000){2}{\rule{1.566pt}{1.200pt}}
\put(826,346.01){\rule{3.132pt}{1.200pt}}
\multiput(826.00,345.51)(6.500,1.000){2}{\rule{1.566pt}{1.200pt}}
\put(839,347.01){\rule{3.132pt}{1.200pt}}
\multiput(839.00,346.51)(6.500,1.000){2}{\rule{1.566pt}{1.200pt}}
\put(852,348.51){\rule{3.132pt}{1.200pt}}
\multiput(852.00,347.51)(6.500,2.000){2}{\rule{1.566pt}{1.200pt}}
\put(865,350.51){\rule{3.132pt}{1.200pt}}
\multiput(865.00,349.51)(6.500,2.000){2}{\rule{1.566pt}{1.200pt}}
\put(878,352.51){\rule{3.132pt}{1.200pt}}
\multiput(878.00,351.51)(6.500,2.000){2}{\rule{1.566pt}{1.200pt}}
\put(891,355.01){\rule{3.132pt}{1.200pt}}
\multiput(891.00,353.51)(6.500,3.000){2}{\rule{1.566pt}{1.200pt}}
\put(904,358.01){\rule{3.132pt}{1.200pt}}
\multiput(904.00,356.51)(6.500,3.000){2}{\rule{1.566pt}{1.200pt}}
\put(917,361.51){\rule{3.132pt}{1.200pt}}
\multiput(917.00,359.51)(6.500,4.000){2}{\rule{1.566pt}{1.200pt}}
\put(930,365.01){\rule{3.132pt}{1.200pt}}
\multiput(930.00,363.51)(6.500,3.000){2}{\rule{1.566pt}{1.200pt}}
\put(943,368.01){\rule{3.132pt}{1.200pt}}
\multiput(943.00,366.51)(6.500,3.000){2}{\rule{1.566pt}{1.200pt}}
\put(956,371.01){\rule{3.132pt}{1.200pt}}
\multiput(956.00,369.51)(6.500,3.000){2}{\rule{1.566pt}{1.200pt}}
\put(969,373.51){\rule{3.373pt}{1.200pt}}
\multiput(969.00,372.51)(7.000,2.000){2}{\rule{1.686pt}{1.200pt}}
\put(983,375.51){\rule{3.132pt}{1.200pt}}
\multiput(983.00,374.51)(6.500,2.000){2}{\rule{1.566pt}{1.200pt}}
\put(996,377.01){\rule{3.132pt}{1.200pt}}
\multiput(996.00,376.51)(6.500,1.000){2}{\rule{1.566pt}{1.200pt}}
\put(1009,378.01){\rule{3.132pt}{1.200pt}}
\multiput(1009.00,377.51)(6.500,1.000){2}{\rule{1.566pt}{1.200pt}}
\put(357.0,269.0){\rule[-0.600pt]{6.263pt}{1.200pt}}
\put(1035,379.01){\rule{3.132pt}{1.200pt}}
\multiput(1035.00,378.51)(6.500,1.000){2}{\rule{1.566pt}{1.200pt}}
\put(1048,380.01){\rule{3.132pt}{1.200pt}}
\multiput(1048.00,379.51)(6.500,1.000){2}{\rule{1.566pt}{1.200pt}}
\put(1061,381.01){\rule{3.132pt}{1.200pt}}
\multiput(1061.00,380.51)(6.500,1.000){2}{\rule{1.566pt}{1.200pt}}
\put(1074,382.51){\rule{3.132pt}{1.200pt}}
\multiput(1074.00,381.51)(6.500,2.000){2}{\rule{1.566pt}{1.200pt}}
\put(1087,384.51){\rule{3.132pt}{1.200pt}}
\multiput(1087.00,383.51)(6.500,2.000){2}{\rule{1.566pt}{1.200pt}}
\put(1100,387.01){\rule{3.132pt}{1.200pt}}
\multiput(1100.00,385.51)(6.500,3.000){2}{\rule{1.566pt}{1.200pt}}
\put(1113,390.01){\rule{3.132pt}{1.200pt}}
\multiput(1113.00,388.51)(6.500,3.000){2}{\rule{1.566pt}{1.200pt}}
\put(1126,393.01){\rule{3.132pt}{1.200pt}}
\multiput(1126.00,391.51)(6.500,3.000){2}{\rule{1.566pt}{1.200pt}}
\put(1139,395.51){\rule{3.132pt}{1.200pt}}
\multiput(1139.00,394.51)(6.500,2.000){2}{\rule{1.566pt}{1.200pt}}
\put(1152,397.51){\rule{3.132pt}{1.200pt}}
\multiput(1152.00,396.51)(6.500,2.000){2}{\rule{1.566pt}{1.200pt}}
\put(1165,399.01){\rule{3.132pt}{1.200pt}}
\multiput(1165.00,398.51)(6.500,1.000){2}{\rule{1.566pt}{1.200pt}}
\put(1178,399.01){\rule{3.132pt}{1.200pt}}
\multiput(1178.00,399.51)(6.500,-1.000){2}{\rule{1.566pt}{1.200pt}}
\put(1022.0,381.0){\rule[-0.600pt]{3.132pt}{1.200pt}}
\put(1204,398.01){\rule{3.132pt}{1.200pt}}
\multiput(1204.00,398.51)(6.500,-1.000){2}{\rule{1.566pt}{1.200pt}}
\put(1217,396.51){\rule{3.132pt}{1.200pt}}
\multiput(1217.00,397.51)(6.500,-2.000){2}{\rule{1.566pt}{1.200pt}}
\put(1230,395.01){\rule{3.132pt}{1.200pt}}
\multiput(1230.00,395.51)(6.500,-1.000){2}{\rule{1.566pt}{1.200pt}}
\put(1243,393.51){\rule{3.132pt}{1.200pt}}
\multiput(1243.00,394.51)(6.500,-2.000){2}{\rule{1.566pt}{1.200pt}}
\put(1256,392.01){\rule{3.132pt}{1.200pt}}
\multiput(1256.00,392.51)(6.500,-1.000){2}{\rule{1.566pt}{1.200pt}}
\put(1269,390.51){\rule{3.132pt}{1.200pt}}
\multiput(1269.00,391.51)(6.500,-2.000){2}{\rule{1.566pt}{1.200pt}}
\put(1282,388.51){\rule{3.132pt}{1.200pt}}
\multiput(1282.00,389.51)(6.500,-2.000){2}{\rule{1.566pt}{1.200pt}}
\put(1295,387.01){\rule{3.132pt}{1.200pt}}
\multiput(1295.00,387.51)(6.500,-1.000){2}{\rule{1.566pt}{1.200pt}}
\put(1191.0,401.0){\rule[-0.600pt]{3.132pt}{1.200pt}}
\put(1321,387.51){\rule{3.132pt}{1.200pt}}
\multiput(1321.00,386.51)(6.500,2.000){2}{\rule{1.566pt}{1.200pt}}
\put(1334,390.51){\rule{3.132pt}{1.200pt}}
\multiput(1334.00,388.51)(6.500,4.000){2}{\rule{1.566pt}{1.200pt}}
\put(1347,395.01){\rule{3.132pt}{1.200pt}}
\multiput(1347.00,392.51)(6.500,5.000){2}{\rule{1.566pt}{1.200pt}}
\put(1360,400.01){\rule{3.132pt}{1.200pt}}
\multiput(1360.00,397.51)(6.500,5.000){2}{\rule{1.566pt}{1.200pt}}
\put(1373,404.01){\rule{3.132pt}{1.200pt}}
\multiput(1373.00,402.51)(6.500,3.000){2}{\rule{1.566pt}{1.200pt}}
\put(1386,406.01){\rule{3.132pt}{1.200pt}}
\multiput(1386.00,405.51)(6.500,1.000){2}{\rule{1.566pt}{1.200pt}}
\put(1399,404.51){\rule{3.132pt}{1.200pt}}
\multiput(1399.00,406.51)(6.500,-4.000){2}{\rule{1.566pt}{1.200pt}}
\multiput(1412.00,402.25)(0.962,-0.509){2}{\rule{2.900pt}{0.123pt}}
\multiput(1412.00,402.51)(6.981,-6.000){2}{\rule{1.450pt}{1.200pt}}
\put(1425,394.51){\rule{3.132pt}{1.200pt}}
\multiput(1425.00,396.51)(6.500,-4.000){2}{\rule{1.566pt}{1.200pt}}
\put(1438,393.01){\rule{0.241pt}{1.200pt}}
\multiput(1438.00,392.51)(0.500,1.000){2}{\rule{0.120pt}{1.200pt}}
\put(1308.0,389.0){\rule[-0.600pt]{3.132pt}{1.200pt}}
\sbox{\plotpoint}{\rule[-0.200pt]{0.400pt}{0.400pt}}%
\put(170.0,82.0){\rule[-0.200pt]{0.400pt}{187.179pt}}
\put(170.0,82.0){\rule[-0.200pt]{305.702pt}{0.400pt}}
\put(1439.0,82.0){\rule[-0.200pt]{0.400pt}{187.179pt}}
\put(170.0,859.0){\rule[-0.200pt]{305.702pt}{0.400pt}}
\end{picture}

\caption{Wykres zależności współczynnika $\Delta(X^{-1}X-I)$ od wielkości macierzy w arytmetyce single}
\end{center}
\end{figure}
%***********************************************************************

Ostatni przebadany przez nas współczynnik dokładności mnożenia macierzy zadany jest wzorem:
$$\Delta((XY)V-X(YV)),$$
gdzie $X, Y, V$ są macierzami kwadratowymi. Obliczyliśmy jego wartości dla losowo wygenerowanych macierzy o wielkości z przedziału $n \in [2,500]$
odpowiednio wykorzystując algorytm naturalny, algorytm Strassena i stworzony przez nas algorytm łączący obie metody.
Wykres {\bf 3.7} przedstawia wyniki uzyskane przez nas w przypadku użycia arytmetyki double. Tak jak dla poprzednich współczynników, algorytm
Strassena daje najmniej dokładne wyniki, wartość wyrażenia $\Delta((XY)V-X(YV))$ dla badanych przez nas wielkości macierzy sięga nawet $10^{-16}$,
co stanowi około $10^{3}$ krotnie gorszy wynik, niż pozostałe metody. Patrząc na wykres,
można zauważyć, że wielkości współczynników dla pozostałych metod są sobie bliskie.
\begin{figure}[h!tb]
\begin{center}
% GNUPLOT: LaTeX picture
\setlength{\unitlength}{0.240900pt}
\ifx\plotpoint\undefined\newsavebox{\plotpoint}\fi
\sbox{\plotpoint}{\rule[-0.200pt]{0.400pt}{0.400pt}}%
\begin{picture}(1500,900)(0,0)
\sbox{\plotpoint}{\rule[-0.200pt]{0.400pt}{0.400pt}}%
\put(371.0,82.0){\rule[-0.200pt]{4.818pt}{0.400pt}}
\put(351,82){\makebox(0,0)[r]{ 0.001}}
\put(1419.0,82.0){\rule[-0.200pt]{4.818pt}{0.400pt}}
\put(371.0,95.0){\rule[-0.200pt]{2.409pt}{0.400pt}}
\put(1429.0,95.0){\rule[-0.200pt]{2.409pt}{0.400pt}}
\put(371.0,112.0){\rule[-0.200pt]{2.409pt}{0.400pt}}
\put(1429.0,112.0){\rule[-0.200pt]{2.409pt}{0.400pt}}
\put(371.0,121.0){\rule[-0.200pt]{2.409pt}{0.400pt}}
\put(1429.0,121.0){\rule[-0.200pt]{2.409pt}{0.400pt}}
\put(371.0,127.0){\rule[-0.200pt]{2.409pt}{0.400pt}}
\put(1429.0,127.0){\rule[-0.200pt]{2.409pt}{0.400pt}}
\put(371.0,131.0){\rule[-0.200pt]{2.409pt}{0.400pt}}
\put(1429.0,131.0){\rule[-0.200pt]{2.409pt}{0.400pt}}
\put(371.0,135.0){\rule[-0.200pt]{2.409pt}{0.400pt}}
\put(1429.0,135.0){\rule[-0.200pt]{2.409pt}{0.400pt}}
\put(371.0,138.0){\rule[-0.200pt]{2.409pt}{0.400pt}}
\put(1429.0,138.0){\rule[-0.200pt]{2.409pt}{0.400pt}}
\put(371.0,141.0){\rule[-0.200pt]{2.409pt}{0.400pt}}
\put(1429.0,141.0){\rule[-0.200pt]{2.409pt}{0.400pt}}
\put(371.0,143.0){\rule[-0.200pt]{2.409pt}{0.400pt}}
\put(1429.0,143.0){\rule[-0.200pt]{2.409pt}{0.400pt}}
\put(371.0,145.0){\rule[-0.200pt]{2.409pt}{0.400pt}}
\put(1429.0,145.0){\rule[-0.200pt]{2.409pt}{0.400pt}}
\put(371.0,147.0){\rule[-0.200pt]{2.409pt}{0.400pt}}
\put(1429.0,147.0){\rule[-0.200pt]{2.409pt}{0.400pt}}
\put(371.0,149.0){\rule[-0.200pt]{2.409pt}{0.400pt}}
\put(1429.0,149.0){\rule[-0.200pt]{2.409pt}{0.400pt}}
\put(371.0,150.0){\rule[-0.200pt]{2.409pt}{0.400pt}}
\put(1429.0,150.0){\rule[-0.200pt]{2.409pt}{0.400pt}}
\put(371.0,152.0){\rule[-0.200pt]{2.409pt}{0.400pt}}
\put(1429.0,152.0){\rule[-0.200pt]{2.409pt}{0.400pt}}
\put(371.0,153.0){\rule[-0.200pt]{2.409pt}{0.400pt}}
\put(1429.0,153.0){\rule[-0.200pt]{2.409pt}{0.400pt}}
\put(371.0,154.0){\rule[-0.200pt]{2.409pt}{0.400pt}}
\put(1429.0,154.0){\rule[-0.200pt]{2.409pt}{0.400pt}}
\put(371.0,155.0){\rule[-0.200pt]{2.409pt}{0.400pt}}
\put(1429.0,155.0){\rule[-0.200pt]{2.409pt}{0.400pt}}
\put(371.0,156.0){\rule[-0.200pt]{2.409pt}{0.400pt}}
\put(1429.0,156.0){\rule[-0.200pt]{2.409pt}{0.400pt}}
\put(371.0,157.0){\rule[-0.200pt]{2.409pt}{0.400pt}}
\put(1429.0,157.0){\rule[-0.200pt]{2.409pt}{0.400pt}}
\put(371.0,158.0){\rule[-0.200pt]{2.409pt}{0.400pt}}
\put(1429.0,158.0){\rule[-0.200pt]{2.409pt}{0.400pt}}
\put(371.0,159.0){\rule[-0.200pt]{2.409pt}{0.400pt}}
\put(1429.0,159.0){\rule[-0.200pt]{2.409pt}{0.400pt}}
\put(371.0,160.0){\rule[-0.200pt]{2.409pt}{0.400pt}}
\put(1429.0,160.0){\rule[-0.200pt]{2.409pt}{0.400pt}}
\put(371.0,161.0){\rule[-0.200pt]{2.409pt}{0.400pt}}
\put(1429.0,161.0){\rule[-0.200pt]{2.409pt}{0.400pt}}
\put(371.0,162.0){\rule[-0.200pt]{2.409pt}{0.400pt}}
\put(1429.0,162.0){\rule[-0.200pt]{2.409pt}{0.400pt}}
\put(371.0,163.0){\rule[-0.200pt]{2.409pt}{0.400pt}}
\put(1429.0,163.0){\rule[-0.200pt]{2.409pt}{0.400pt}}
\put(371.0,163.0){\rule[-0.200pt]{2.409pt}{0.400pt}}
\put(1429.0,163.0){\rule[-0.200pt]{2.409pt}{0.400pt}}
\put(371.0,164.0){\rule[-0.200pt]{2.409pt}{0.400pt}}
\put(1429.0,164.0){\rule[-0.200pt]{2.409pt}{0.400pt}}
\put(371.0,165.0){\rule[-0.200pt]{2.409pt}{0.400pt}}
\put(1429.0,165.0){\rule[-0.200pt]{2.409pt}{0.400pt}}
\put(371.0,166.0){\rule[-0.200pt]{2.409pt}{0.400pt}}
\put(1429.0,166.0){\rule[-0.200pt]{2.409pt}{0.400pt}}
\put(371.0,166.0){\rule[-0.200pt]{2.409pt}{0.400pt}}
\put(1429.0,166.0){\rule[-0.200pt]{2.409pt}{0.400pt}}
\put(371.0,167.0){\rule[-0.200pt]{2.409pt}{0.400pt}}
\put(1429.0,167.0){\rule[-0.200pt]{2.409pt}{0.400pt}}
\put(371.0,167.0){\rule[-0.200pt]{2.409pt}{0.400pt}}
\put(1429.0,167.0){\rule[-0.200pt]{2.409pt}{0.400pt}}
\put(371.0,168.0){\rule[-0.200pt]{2.409pt}{0.400pt}}
\put(1429.0,168.0){\rule[-0.200pt]{2.409pt}{0.400pt}}
\put(371.0,169.0){\rule[-0.200pt]{2.409pt}{0.400pt}}
\put(1429.0,169.0){\rule[-0.200pt]{2.409pt}{0.400pt}}
\put(371.0,169.0){\rule[-0.200pt]{2.409pt}{0.400pt}}
\put(1429.0,169.0){\rule[-0.200pt]{2.409pt}{0.400pt}}
\put(371.0,170.0){\rule[-0.200pt]{2.409pt}{0.400pt}}
\put(1429.0,170.0){\rule[-0.200pt]{2.409pt}{0.400pt}}
\put(371.0,170.0){\rule[-0.200pt]{2.409pt}{0.400pt}}
\put(1429.0,170.0){\rule[-0.200pt]{2.409pt}{0.400pt}}
\put(371.0,171.0){\rule[-0.200pt]{2.409pt}{0.400pt}}
\put(1429.0,171.0){\rule[-0.200pt]{2.409pt}{0.400pt}}
\put(371.0,171.0){\rule[-0.200pt]{2.409pt}{0.400pt}}
\put(1429.0,171.0){\rule[-0.200pt]{2.409pt}{0.400pt}}
\put(371.0,172.0){\rule[-0.200pt]{2.409pt}{0.400pt}}
\put(1429.0,172.0){\rule[-0.200pt]{2.409pt}{0.400pt}}
\put(371.0,172.0){\rule[-0.200pt]{2.409pt}{0.400pt}}
\put(1429.0,172.0){\rule[-0.200pt]{2.409pt}{0.400pt}}
\put(371.0,173.0){\rule[-0.200pt]{2.409pt}{0.400pt}}
\put(1429.0,173.0){\rule[-0.200pt]{2.409pt}{0.400pt}}
\put(371.0,173.0){\rule[-0.200pt]{2.409pt}{0.400pt}}
\put(1429.0,173.0){\rule[-0.200pt]{2.409pt}{0.400pt}}
\put(371.0,173.0){\rule[-0.200pt]{2.409pt}{0.400pt}}
\put(1429.0,173.0){\rule[-0.200pt]{2.409pt}{0.400pt}}
\put(371.0,174.0){\rule[-0.200pt]{2.409pt}{0.400pt}}
\put(1429.0,174.0){\rule[-0.200pt]{2.409pt}{0.400pt}}
\put(371.0,174.0){\rule[-0.200pt]{2.409pt}{0.400pt}}
\put(1429.0,174.0){\rule[-0.200pt]{2.409pt}{0.400pt}}
\put(371.0,175.0){\rule[-0.200pt]{2.409pt}{0.400pt}}
\put(1429.0,175.0){\rule[-0.200pt]{2.409pt}{0.400pt}}
\put(371.0,175.0){\rule[-0.200pt]{2.409pt}{0.400pt}}
\put(1429.0,175.0){\rule[-0.200pt]{2.409pt}{0.400pt}}
\put(371.0,175.0){\rule[-0.200pt]{2.409pt}{0.400pt}}
\put(1429.0,175.0){\rule[-0.200pt]{2.409pt}{0.400pt}}
\put(371.0,176.0){\rule[-0.200pt]{2.409pt}{0.400pt}}
\put(1429.0,176.0){\rule[-0.200pt]{2.409pt}{0.400pt}}
\put(371.0,176.0){\rule[-0.200pt]{2.409pt}{0.400pt}}
\put(1429.0,176.0){\rule[-0.200pt]{2.409pt}{0.400pt}}
\put(371.0,177.0){\rule[-0.200pt]{2.409pt}{0.400pt}}
\put(1429.0,177.0){\rule[-0.200pt]{2.409pt}{0.400pt}}
\put(371.0,177.0){\rule[-0.200pt]{2.409pt}{0.400pt}}
\put(1429.0,177.0){\rule[-0.200pt]{2.409pt}{0.400pt}}
\put(371.0,177.0){\rule[-0.200pt]{2.409pt}{0.400pt}}
\put(1429.0,177.0){\rule[-0.200pt]{2.409pt}{0.400pt}}
\put(371.0,178.0){\rule[-0.200pt]{2.409pt}{0.400pt}}
\put(1429.0,178.0){\rule[-0.200pt]{2.409pt}{0.400pt}}
\put(371.0,178.0){\rule[-0.200pt]{2.409pt}{0.400pt}}
\put(1429.0,178.0){\rule[-0.200pt]{2.409pt}{0.400pt}}
\put(371.0,178.0){\rule[-0.200pt]{2.409pt}{0.400pt}}
\put(1429.0,178.0){\rule[-0.200pt]{2.409pt}{0.400pt}}
\put(371.0,179.0){\rule[-0.200pt]{2.409pt}{0.400pt}}
\put(1429.0,179.0){\rule[-0.200pt]{2.409pt}{0.400pt}}
\put(371.0,179.0){\rule[-0.200pt]{2.409pt}{0.400pt}}
\put(1429.0,179.0){\rule[-0.200pt]{2.409pt}{0.400pt}}
\put(371.0,179.0){\rule[-0.200pt]{2.409pt}{0.400pt}}
\put(1429.0,179.0){\rule[-0.200pt]{2.409pt}{0.400pt}}
\put(371.0,180.0){\rule[-0.200pt]{2.409pt}{0.400pt}}
\put(1429.0,180.0){\rule[-0.200pt]{2.409pt}{0.400pt}}
\put(371.0,180.0){\rule[-0.200pt]{2.409pt}{0.400pt}}
\put(1429.0,180.0){\rule[-0.200pt]{2.409pt}{0.400pt}}
\put(371.0,180.0){\rule[-0.200pt]{2.409pt}{0.400pt}}
\put(1429.0,180.0){\rule[-0.200pt]{2.409pt}{0.400pt}}
\put(371.0,180.0){\rule[-0.200pt]{2.409pt}{0.400pt}}
\put(1429.0,180.0){\rule[-0.200pt]{2.409pt}{0.400pt}}
\put(371.0,181.0){\rule[-0.200pt]{2.409pt}{0.400pt}}
\put(1429.0,181.0){\rule[-0.200pt]{2.409pt}{0.400pt}}
\put(371.0,181.0){\rule[-0.200pt]{2.409pt}{0.400pt}}
\put(1429.0,181.0){\rule[-0.200pt]{2.409pt}{0.400pt}}
\put(371.0,181.0){\rule[-0.200pt]{2.409pt}{0.400pt}}
\put(1429.0,181.0){\rule[-0.200pt]{2.409pt}{0.400pt}}
\put(371.0,182.0){\rule[-0.200pt]{2.409pt}{0.400pt}}
\put(1429.0,182.0){\rule[-0.200pt]{2.409pt}{0.400pt}}
\put(371.0,182.0){\rule[-0.200pt]{2.409pt}{0.400pt}}
\put(1429.0,182.0){\rule[-0.200pt]{2.409pt}{0.400pt}}
\put(371.0,182.0){\rule[-0.200pt]{2.409pt}{0.400pt}}
\put(1429.0,182.0){\rule[-0.200pt]{2.409pt}{0.400pt}}
\put(371.0,182.0){\rule[-0.200pt]{2.409pt}{0.400pt}}
\put(1429.0,182.0){\rule[-0.200pt]{2.409pt}{0.400pt}}
\put(371.0,183.0){\rule[-0.200pt]{2.409pt}{0.400pt}}
\put(1429.0,183.0){\rule[-0.200pt]{2.409pt}{0.400pt}}
\put(371.0,183.0){\rule[-0.200pt]{2.409pt}{0.400pt}}
\put(1429.0,183.0){\rule[-0.200pt]{2.409pt}{0.400pt}}
\put(371.0,183.0){\rule[-0.200pt]{2.409pt}{0.400pt}}
\put(1429.0,183.0){\rule[-0.200pt]{2.409pt}{0.400pt}}
\put(371.0,183.0){\rule[-0.200pt]{2.409pt}{0.400pt}}
\put(1429.0,183.0){\rule[-0.200pt]{2.409pt}{0.400pt}}
\put(371.0,184.0){\rule[-0.200pt]{2.409pt}{0.400pt}}
\put(1429.0,184.0){\rule[-0.200pt]{2.409pt}{0.400pt}}
\put(371.0,184.0){\rule[-0.200pt]{2.409pt}{0.400pt}}
\put(1429.0,184.0){\rule[-0.200pt]{2.409pt}{0.400pt}}
\put(371.0,184.0){\rule[-0.200pt]{2.409pt}{0.400pt}}
\put(1429.0,184.0){\rule[-0.200pt]{2.409pt}{0.400pt}}
\put(371.0,184.0){\rule[-0.200pt]{2.409pt}{0.400pt}}
\put(1429.0,184.0){\rule[-0.200pt]{2.409pt}{0.400pt}}
\put(371.0,185.0){\rule[-0.200pt]{2.409pt}{0.400pt}}
\put(1429.0,185.0){\rule[-0.200pt]{2.409pt}{0.400pt}}
\put(371.0,185.0){\rule[-0.200pt]{2.409pt}{0.400pt}}
\put(1429.0,185.0){\rule[-0.200pt]{2.409pt}{0.400pt}}
\put(371.0,185.0){\rule[-0.200pt]{2.409pt}{0.400pt}}
\put(1429.0,185.0){\rule[-0.200pt]{2.409pt}{0.400pt}}
\put(371.0,185.0){\rule[-0.200pt]{2.409pt}{0.400pt}}
\put(1429.0,185.0){\rule[-0.200pt]{2.409pt}{0.400pt}}
\put(371.0,186.0){\rule[-0.200pt]{2.409pt}{0.400pt}}
\put(1429.0,186.0){\rule[-0.200pt]{2.409pt}{0.400pt}}
\put(371.0,186.0){\rule[-0.200pt]{2.409pt}{0.400pt}}
\put(1429.0,186.0){\rule[-0.200pt]{2.409pt}{0.400pt}}
\put(371.0,186.0){\rule[-0.200pt]{2.409pt}{0.400pt}}
\put(1429.0,186.0){\rule[-0.200pt]{2.409pt}{0.400pt}}
\put(371.0,186.0){\rule[-0.200pt]{2.409pt}{0.400pt}}
\put(1429.0,186.0){\rule[-0.200pt]{2.409pt}{0.400pt}}
\put(371.0,186.0){\rule[-0.200pt]{2.409pt}{0.400pt}}
\put(1429.0,186.0){\rule[-0.200pt]{2.409pt}{0.400pt}}
\put(371.0,187.0){\rule[-0.200pt]{2.409pt}{0.400pt}}
\put(1429.0,187.0){\rule[-0.200pt]{2.409pt}{0.400pt}}
\put(371.0,187.0){\rule[-0.200pt]{2.409pt}{0.400pt}}
\put(1429.0,187.0){\rule[-0.200pt]{2.409pt}{0.400pt}}
\put(371.0,187.0){\rule[-0.200pt]{2.409pt}{0.400pt}}
\put(1429.0,187.0){\rule[-0.200pt]{2.409pt}{0.400pt}}
\put(371.0,187.0){\rule[-0.200pt]{2.409pt}{0.400pt}}
\put(1429.0,187.0){\rule[-0.200pt]{2.409pt}{0.400pt}}
\put(371.0,188.0){\rule[-0.200pt]{2.409pt}{0.400pt}}
\put(1429.0,188.0){\rule[-0.200pt]{2.409pt}{0.400pt}}
\put(371.0,188.0){\rule[-0.200pt]{2.409pt}{0.400pt}}
\put(1429.0,188.0){\rule[-0.200pt]{2.409pt}{0.400pt}}
\put(371.0,188.0){\rule[-0.200pt]{2.409pt}{0.400pt}}
\put(1429.0,188.0){\rule[-0.200pt]{2.409pt}{0.400pt}}
\put(371.0,188.0){\rule[-0.200pt]{2.409pt}{0.400pt}}
\put(1429.0,188.0){\rule[-0.200pt]{2.409pt}{0.400pt}}
\put(371.0,188.0){\rule[-0.200pt]{2.409pt}{0.400pt}}
\put(1429.0,188.0){\rule[-0.200pt]{2.409pt}{0.400pt}}
\put(371.0,188.0){\rule[-0.200pt]{2.409pt}{0.400pt}}
\put(1429.0,188.0){\rule[-0.200pt]{2.409pt}{0.400pt}}
\put(371.0,189.0){\rule[-0.200pt]{2.409pt}{0.400pt}}
\put(1429.0,189.0){\rule[-0.200pt]{2.409pt}{0.400pt}}
\put(371.0,189.0){\rule[-0.200pt]{2.409pt}{0.400pt}}
\put(1429.0,189.0){\rule[-0.200pt]{2.409pt}{0.400pt}}
\put(371.0,189.0){\rule[-0.200pt]{2.409pt}{0.400pt}}
\put(1429.0,189.0){\rule[-0.200pt]{2.409pt}{0.400pt}}
\put(371.0,189.0){\rule[-0.200pt]{2.409pt}{0.400pt}}
\put(1429.0,189.0){\rule[-0.200pt]{2.409pt}{0.400pt}}
\put(371.0,189.0){\rule[-0.200pt]{2.409pt}{0.400pt}}
\put(1429.0,189.0){\rule[-0.200pt]{2.409pt}{0.400pt}}
\put(371.0,190.0){\rule[-0.200pt]{2.409pt}{0.400pt}}
\put(1429.0,190.0){\rule[-0.200pt]{2.409pt}{0.400pt}}
\put(371.0,190.0){\rule[-0.200pt]{2.409pt}{0.400pt}}
\put(1429.0,190.0){\rule[-0.200pt]{2.409pt}{0.400pt}}
\put(371.0,190.0){\rule[-0.200pt]{2.409pt}{0.400pt}}
\put(1429.0,190.0){\rule[-0.200pt]{2.409pt}{0.400pt}}
\put(371.0,190.0){\rule[-0.200pt]{2.409pt}{0.400pt}}
\put(1429.0,190.0){\rule[-0.200pt]{2.409pt}{0.400pt}}
\put(371.0,190.0){\rule[-0.200pt]{2.409pt}{0.400pt}}
\put(1429.0,190.0){\rule[-0.200pt]{2.409pt}{0.400pt}}
\put(371.0,190.0){\rule[-0.200pt]{2.409pt}{0.400pt}}
\put(1429.0,190.0){\rule[-0.200pt]{2.409pt}{0.400pt}}
\put(371.0,191.0){\rule[-0.200pt]{2.409pt}{0.400pt}}
\put(1429.0,191.0){\rule[-0.200pt]{2.409pt}{0.400pt}}
\put(371.0,191.0){\rule[-0.200pt]{2.409pt}{0.400pt}}
\put(1429.0,191.0){\rule[-0.200pt]{2.409pt}{0.400pt}}
\put(371.0,191.0){\rule[-0.200pt]{2.409pt}{0.400pt}}
\put(1429.0,191.0){\rule[-0.200pt]{2.409pt}{0.400pt}}
\put(371.0,191.0){\rule[-0.200pt]{2.409pt}{0.400pt}}
\put(1429.0,191.0){\rule[-0.200pt]{2.409pt}{0.400pt}}
\put(371.0,191.0){\rule[-0.200pt]{2.409pt}{0.400pt}}
\put(1429.0,191.0){\rule[-0.200pt]{2.409pt}{0.400pt}}
\put(371.0,191.0){\rule[-0.200pt]{2.409pt}{0.400pt}}
\put(1429.0,191.0){\rule[-0.200pt]{2.409pt}{0.400pt}}
\put(371.0,192.0){\rule[-0.200pt]{2.409pt}{0.400pt}}
\put(1429.0,192.0){\rule[-0.200pt]{2.409pt}{0.400pt}}
\put(371.0,192.0){\rule[-0.200pt]{2.409pt}{0.400pt}}
\put(1429.0,192.0){\rule[-0.200pt]{2.409pt}{0.400pt}}
\put(371.0,192.0){\rule[-0.200pt]{2.409pt}{0.400pt}}
\put(1429.0,192.0){\rule[-0.200pt]{2.409pt}{0.400pt}}
\put(371.0,192.0){\rule[-0.200pt]{2.409pt}{0.400pt}}
\put(1429.0,192.0){\rule[-0.200pt]{2.409pt}{0.400pt}}
\put(371.0,192.0){\rule[-0.200pt]{2.409pt}{0.400pt}}
\put(1429.0,192.0){\rule[-0.200pt]{2.409pt}{0.400pt}}
\put(371.0,192.0){\rule[-0.200pt]{2.409pt}{0.400pt}}
\put(1429.0,192.0){\rule[-0.200pt]{2.409pt}{0.400pt}}
\put(371.0,193.0){\rule[-0.200pt]{2.409pt}{0.400pt}}
\put(1429.0,193.0){\rule[-0.200pt]{2.409pt}{0.400pt}}
\put(371.0,193.0){\rule[-0.200pt]{2.409pt}{0.400pt}}
\put(1429.0,193.0){\rule[-0.200pt]{2.409pt}{0.400pt}}
\put(371.0,193.0){\rule[-0.200pt]{2.409pt}{0.400pt}}
\put(1429.0,193.0){\rule[-0.200pt]{2.409pt}{0.400pt}}
\put(371.0,193.0){\rule[-0.200pt]{2.409pt}{0.400pt}}
\put(1429.0,193.0){\rule[-0.200pt]{2.409pt}{0.400pt}}
\put(371.0,193.0){\rule[-0.200pt]{2.409pt}{0.400pt}}
\put(1429.0,193.0){\rule[-0.200pt]{2.409pt}{0.400pt}}
\put(371.0,193.0){\rule[-0.200pt]{2.409pt}{0.400pt}}
\put(1429.0,193.0){\rule[-0.200pt]{2.409pt}{0.400pt}}
\put(371.0,194.0){\rule[-0.200pt]{2.409pt}{0.400pt}}
\put(1429.0,194.0){\rule[-0.200pt]{2.409pt}{0.400pt}}
\put(371.0,194.0){\rule[-0.200pt]{2.409pt}{0.400pt}}
\put(1429.0,194.0){\rule[-0.200pt]{2.409pt}{0.400pt}}
\put(371.0,194.0){\rule[-0.200pt]{2.409pt}{0.400pt}}
\put(1429.0,194.0){\rule[-0.200pt]{2.409pt}{0.400pt}}
\put(371.0,194.0){\rule[-0.200pt]{2.409pt}{0.400pt}}
\put(1429.0,194.0){\rule[-0.200pt]{2.409pt}{0.400pt}}
\put(371.0,194.0){\rule[-0.200pt]{2.409pt}{0.400pt}}
\put(1429.0,194.0){\rule[-0.200pt]{2.409pt}{0.400pt}}
\put(371.0,194.0){\rule[-0.200pt]{2.409pt}{0.400pt}}
\put(1429.0,194.0){\rule[-0.200pt]{2.409pt}{0.400pt}}
\put(371.0,194.0){\rule[-0.200pt]{2.409pt}{0.400pt}}
\put(1429.0,194.0){\rule[-0.200pt]{2.409pt}{0.400pt}}
\put(371.0,195.0){\rule[-0.200pt]{2.409pt}{0.400pt}}
\put(1429.0,195.0){\rule[-0.200pt]{2.409pt}{0.400pt}}
\put(371.0,195.0){\rule[-0.200pt]{2.409pt}{0.400pt}}
\put(1429.0,195.0){\rule[-0.200pt]{2.409pt}{0.400pt}}
\put(371.0,195.0){\rule[-0.200pt]{2.409pt}{0.400pt}}
\put(1429.0,195.0){\rule[-0.200pt]{2.409pt}{0.400pt}}
\put(371.0,195.0){\rule[-0.200pt]{2.409pt}{0.400pt}}
\put(1429.0,195.0){\rule[-0.200pt]{2.409pt}{0.400pt}}
\put(371.0,195.0){\rule[-0.200pt]{2.409pt}{0.400pt}}
\put(1429.0,195.0){\rule[-0.200pt]{2.409pt}{0.400pt}}
\put(371.0,195.0){\rule[-0.200pt]{2.409pt}{0.400pt}}
\put(1429.0,195.0){\rule[-0.200pt]{2.409pt}{0.400pt}}
\put(371.0,195.0){\rule[-0.200pt]{2.409pt}{0.400pt}}
\put(1429.0,195.0){\rule[-0.200pt]{2.409pt}{0.400pt}}
\put(371.0,195.0){\rule[-0.200pt]{2.409pt}{0.400pt}}
\put(1429.0,195.0){\rule[-0.200pt]{2.409pt}{0.400pt}}
\put(371.0,196.0){\rule[-0.200pt]{2.409pt}{0.400pt}}
\put(1429.0,196.0){\rule[-0.200pt]{2.409pt}{0.400pt}}
\put(371.0,196.0){\rule[-0.200pt]{2.409pt}{0.400pt}}
\put(1429.0,196.0){\rule[-0.200pt]{2.409pt}{0.400pt}}
\put(371.0,196.0){\rule[-0.200pt]{2.409pt}{0.400pt}}
\put(1429.0,196.0){\rule[-0.200pt]{2.409pt}{0.400pt}}
\put(371.0,196.0){\rule[-0.200pt]{2.409pt}{0.400pt}}
\put(1429.0,196.0){\rule[-0.200pt]{2.409pt}{0.400pt}}
\put(371.0,196.0){\rule[-0.200pt]{2.409pt}{0.400pt}}
\put(1429.0,196.0){\rule[-0.200pt]{2.409pt}{0.400pt}}
\put(371.0,196.0){\rule[-0.200pt]{2.409pt}{0.400pt}}
\put(1429.0,196.0){\rule[-0.200pt]{2.409pt}{0.400pt}}
\put(371.0,196.0){\rule[-0.200pt]{2.409pt}{0.400pt}}
\put(1429.0,196.0){\rule[-0.200pt]{2.409pt}{0.400pt}}
\put(371.0,196.0){\rule[-0.200pt]{2.409pt}{0.400pt}}
\put(1429.0,196.0){\rule[-0.200pt]{2.409pt}{0.400pt}}
\put(371.0,197.0){\rule[-0.200pt]{2.409pt}{0.400pt}}
\put(1429.0,197.0){\rule[-0.200pt]{2.409pt}{0.400pt}}
\put(371.0,197.0){\rule[-0.200pt]{2.409pt}{0.400pt}}
\put(1429.0,197.0){\rule[-0.200pt]{2.409pt}{0.400pt}}
\put(371.0,197.0){\rule[-0.200pt]{2.409pt}{0.400pt}}
\put(1429.0,197.0){\rule[-0.200pt]{2.409pt}{0.400pt}}
\put(371.0,197.0){\rule[-0.200pt]{2.409pt}{0.400pt}}
\put(1429.0,197.0){\rule[-0.200pt]{2.409pt}{0.400pt}}
\put(371.0,197.0){\rule[-0.200pt]{2.409pt}{0.400pt}}
\put(1429.0,197.0){\rule[-0.200pt]{2.409pt}{0.400pt}}
\put(371.0,197.0){\rule[-0.200pt]{2.409pt}{0.400pt}}
\put(1429.0,197.0){\rule[-0.200pt]{2.409pt}{0.400pt}}
\put(371.0,197.0){\rule[-0.200pt]{2.409pt}{0.400pt}}
\put(1429.0,197.0){\rule[-0.200pt]{2.409pt}{0.400pt}}
\put(371.0,197.0){\rule[-0.200pt]{2.409pt}{0.400pt}}
\put(1429.0,197.0){\rule[-0.200pt]{2.409pt}{0.400pt}}
\put(371.0,198.0){\rule[-0.200pt]{2.409pt}{0.400pt}}
\put(1429.0,198.0){\rule[-0.200pt]{2.409pt}{0.400pt}}
\put(371.0,198.0){\rule[-0.200pt]{2.409pt}{0.400pt}}
\put(1429.0,198.0){\rule[-0.200pt]{2.409pt}{0.400pt}}
\put(371.0,198.0){\rule[-0.200pt]{2.409pt}{0.400pt}}
\put(1429.0,198.0){\rule[-0.200pt]{2.409pt}{0.400pt}}
\put(371.0,198.0){\rule[-0.200pt]{2.409pt}{0.400pt}}
\put(1429.0,198.0){\rule[-0.200pt]{2.409pt}{0.400pt}}
\put(371.0,198.0){\rule[-0.200pt]{2.409pt}{0.400pt}}
\put(1429.0,198.0){\rule[-0.200pt]{2.409pt}{0.400pt}}
\put(371.0,198.0){\rule[-0.200pt]{2.409pt}{0.400pt}}
\put(1429.0,198.0){\rule[-0.200pt]{2.409pt}{0.400pt}}
\put(371.0,198.0){\rule[-0.200pt]{2.409pt}{0.400pt}}
\put(1429.0,198.0){\rule[-0.200pt]{2.409pt}{0.400pt}}
\put(371.0,198.0){\rule[-0.200pt]{2.409pt}{0.400pt}}
\put(1429.0,198.0){\rule[-0.200pt]{2.409pt}{0.400pt}}
\put(371.0,199.0){\rule[-0.200pt]{2.409pt}{0.400pt}}
\put(1429.0,199.0){\rule[-0.200pt]{2.409pt}{0.400pt}}
\put(371.0,199.0){\rule[-0.200pt]{2.409pt}{0.400pt}}
\put(1429.0,199.0){\rule[-0.200pt]{2.409pt}{0.400pt}}
\put(371.0,199.0){\rule[-0.200pt]{2.409pt}{0.400pt}}
\put(1429.0,199.0){\rule[-0.200pt]{2.409pt}{0.400pt}}
\put(371.0,199.0){\rule[-0.200pt]{2.409pt}{0.400pt}}
\put(1429.0,199.0){\rule[-0.200pt]{2.409pt}{0.400pt}}
\put(371.0,199.0){\rule[-0.200pt]{2.409pt}{0.400pt}}
\put(1429.0,199.0){\rule[-0.200pt]{2.409pt}{0.400pt}}
\put(371.0,199.0){\rule[-0.200pt]{2.409pt}{0.400pt}}
\put(1429.0,199.0){\rule[-0.200pt]{2.409pt}{0.400pt}}
\put(371.0,199.0){\rule[-0.200pt]{2.409pt}{0.400pt}}
\put(1429.0,199.0){\rule[-0.200pt]{2.409pt}{0.400pt}}
\put(371.0,199.0){\rule[-0.200pt]{2.409pt}{0.400pt}}
\put(1429.0,199.0){\rule[-0.200pt]{2.409pt}{0.400pt}}
\put(371.0,199.0){\rule[-0.200pt]{2.409pt}{0.400pt}}
\put(1429.0,199.0){\rule[-0.200pt]{2.409pt}{0.400pt}}
\put(371.0,199.0){\rule[-0.200pt]{2.409pt}{0.400pt}}
\put(1429.0,199.0){\rule[-0.200pt]{2.409pt}{0.400pt}}
\put(371.0,200.0){\rule[-0.200pt]{2.409pt}{0.400pt}}
\put(1429.0,200.0){\rule[-0.200pt]{2.409pt}{0.400pt}}
\put(371.0,200.0){\rule[-0.200pt]{2.409pt}{0.400pt}}
\put(1429.0,200.0){\rule[-0.200pt]{2.409pt}{0.400pt}}
\put(371.0,200.0){\rule[-0.200pt]{2.409pt}{0.400pt}}
\put(1429.0,200.0){\rule[-0.200pt]{2.409pt}{0.400pt}}
\put(371.0,200.0){\rule[-0.200pt]{2.409pt}{0.400pt}}
\put(1429.0,200.0){\rule[-0.200pt]{2.409pt}{0.400pt}}
\put(371.0,200.0){\rule[-0.200pt]{2.409pt}{0.400pt}}
\put(1429.0,200.0){\rule[-0.200pt]{2.409pt}{0.400pt}}
\put(371.0,200.0){\rule[-0.200pt]{2.409pt}{0.400pt}}
\put(1429.0,200.0){\rule[-0.200pt]{2.409pt}{0.400pt}}
\put(371.0,200.0){\rule[-0.200pt]{2.409pt}{0.400pt}}
\put(1429.0,200.0){\rule[-0.200pt]{2.409pt}{0.400pt}}
\put(371.0,200.0){\rule[-0.200pt]{2.409pt}{0.400pt}}
\put(1429.0,200.0){\rule[-0.200pt]{2.409pt}{0.400pt}}
\put(371.0,200.0){\rule[-0.200pt]{2.409pt}{0.400pt}}
\put(1429.0,200.0){\rule[-0.200pt]{2.409pt}{0.400pt}}
\put(371.0,201.0){\rule[-0.200pt]{2.409pt}{0.400pt}}
\put(1429.0,201.0){\rule[-0.200pt]{2.409pt}{0.400pt}}
\put(371.0,201.0){\rule[-0.200pt]{2.409pt}{0.400pt}}
\put(1429.0,201.0){\rule[-0.200pt]{2.409pt}{0.400pt}}
\put(371.0,201.0){\rule[-0.200pt]{2.409pt}{0.400pt}}
\put(1429.0,201.0){\rule[-0.200pt]{2.409pt}{0.400pt}}
\put(371.0,201.0){\rule[-0.200pt]{2.409pt}{0.400pt}}
\put(1429.0,201.0){\rule[-0.200pt]{2.409pt}{0.400pt}}
\put(371.0,201.0){\rule[-0.200pt]{2.409pt}{0.400pt}}
\put(1429.0,201.0){\rule[-0.200pt]{2.409pt}{0.400pt}}
\put(371.0,201.0){\rule[-0.200pt]{2.409pt}{0.400pt}}
\put(1429.0,201.0){\rule[-0.200pt]{2.409pt}{0.400pt}}
\put(371.0,201.0){\rule[-0.200pt]{2.409pt}{0.400pt}}
\put(1429.0,201.0){\rule[-0.200pt]{2.409pt}{0.400pt}}
\put(371.0,201.0){\rule[-0.200pt]{2.409pt}{0.400pt}}
\put(1429.0,201.0){\rule[-0.200pt]{2.409pt}{0.400pt}}
\put(371.0,201.0){\rule[-0.200pt]{2.409pt}{0.400pt}}
\put(1429.0,201.0){\rule[-0.200pt]{2.409pt}{0.400pt}}
\put(371.0,201.0){\rule[-0.200pt]{2.409pt}{0.400pt}}
\put(1429.0,201.0){\rule[-0.200pt]{2.409pt}{0.400pt}}
\put(371.0,202.0){\rule[-0.200pt]{2.409pt}{0.400pt}}
\put(1429.0,202.0){\rule[-0.200pt]{2.409pt}{0.400pt}}
\put(371.0,202.0){\rule[-0.200pt]{2.409pt}{0.400pt}}
\put(1429.0,202.0){\rule[-0.200pt]{2.409pt}{0.400pt}}
\put(371.0,202.0){\rule[-0.200pt]{2.409pt}{0.400pt}}
\put(1429.0,202.0){\rule[-0.200pt]{2.409pt}{0.400pt}}
\put(371.0,202.0){\rule[-0.200pt]{2.409pt}{0.400pt}}
\put(1429.0,202.0){\rule[-0.200pt]{2.409pt}{0.400pt}}
\put(371.0,202.0){\rule[-0.200pt]{2.409pt}{0.400pt}}
\put(1429.0,202.0){\rule[-0.200pt]{2.409pt}{0.400pt}}
\put(371.0,202.0){\rule[-0.200pt]{2.409pt}{0.400pt}}
\put(1429.0,202.0){\rule[-0.200pt]{2.409pt}{0.400pt}}
\put(371.0,202.0){\rule[-0.200pt]{2.409pt}{0.400pt}}
\put(1429.0,202.0){\rule[-0.200pt]{2.409pt}{0.400pt}}
\put(371.0,202.0){\rule[-0.200pt]{2.409pt}{0.400pt}}
\put(1429.0,202.0){\rule[-0.200pt]{2.409pt}{0.400pt}}
\put(371.0,202.0){\rule[-0.200pt]{2.409pt}{0.400pt}}
\put(1429.0,202.0){\rule[-0.200pt]{2.409pt}{0.400pt}}
\put(371.0,202.0){\rule[-0.200pt]{2.409pt}{0.400pt}}
\put(1429.0,202.0){\rule[-0.200pt]{2.409pt}{0.400pt}}
\put(371.0,202.0){\rule[-0.200pt]{2.409pt}{0.400pt}}
\put(1429.0,202.0){\rule[-0.200pt]{2.409pt}{0.400pt}}
\put(371.0,203.0){\rule[-0.200pt]{2.409pt}{0.400pt}}
\put(1429.0,203.0){\rule[-0.200pt]{2.409pt}{0.400pt}}
\put(371.0,203.0){\rule[-0.200pt]{2.409pt}{0.400pt}}
\put(1429.0,203.0){\rule[-0.200pt]{2.409pt}{0.400pt}}
\put(371.0,203.0){\rule[-0.200pt]{2.409pt}{0.400pt}}
\put(1429.0,203.0){\rule[-0.200pt]{2.409pt}{0.400pt}}
\put(371.0,203.0){\rule[-0.200pt]{2.409pt}{0.400pt}}
\put(1429.0,203.0){\rule[-0.200pt]{2.409pt}{0.400pt}}
\put(371.0,203.0){\rule[-0.200pt]{2.409pt}{0.400pt}}
\put(1429.0,203.0){\rule[-0.200pt]{2.409pt}{0.400pt}}
\put(371.0,203.0){\rule[-0.200pt]{2.409pt}{0.400pt}}
\put(1429.0,203.0){\rule[-0.200pt]{2.409pt}{0.400pt}}
\put(371.0,203.0){\rule[-0.200pt]{2.409pt}{0.400pt}}
\put(1429.0,203.0){\rule[-0.200pt]{2.409pt}{0.400pt}}
\put(371.0,203.0){\rule[-0.200pt]{2.409pt}{0.400pt}}
\put(1429.0,203.0){\rule[-0.200pt]{2.409pt}{0.400pt}}
\put(371.0,203.0){\rule[-0.200pt]{2.409pt}{0.400pt}}
\put(1429.0,203.0){\rule[-0.200pt]{2.409pt}{0.400pt}}
\put(371.0,203.0){\rule[-0.200pt]{2.409pt}{0.400pt}}
\put(1429.0,203.0){\rule[-0.200pt]{2.409pt}{0.400pt}}
\put(371.0,203.0){\rule[-0.200pt]{2.409pt}{0.400pt}}
\put(1429.0,203.0){\rule[-0.200pt]{2.409pt}{0.400pt}}
\put(371.0,204.0){\rule[-0.200pt]{2.409pt}{0.400pt}}
\put(1429.0,204.0){\rule[-0.200pt]{2.409pt}{0.400pt}}
\put(371.0,204.0){\rule[-0.200pt]{2.409pt}{0.400pt}}
\put(1429.0,204.0){\rule[-0.200pt]{2.409pt}{0.400pt}}
\put(371.0,204.0){\rule[-0.200pt]{2.409pt}{0.400pt}}
\put(1429.0,204.0){\rule[-0.200pt]{2.409pt}{0.400pt}}
\put(371.0,204.0){\rule[-0.200pt]{2.409pt}{0.400pt}}
\put(1429.0,204.0){\rule[-0.200pt]{2.409pt}{0.400pt}}
\put(371.0,204.0){\rule[-0.200pt]{2.409pt}{0.400pt}}
\put(1429.0,204.0){\rule[-0.200pt]{2.409pt}{0.400pt}}
\put(371.0,204.0){\rule[-0.200pt]{2.409pt}{0.400pt}}
\put(1429.0,204.0){\rule[-0.200pt]{2.409pt}{0.400pt}}
\put(371.0,204.0){\rule[-0.200pt]{2.409pt}{0.400pt}}
\put(1429.0,204.0){\rule[-0.200pt]{2.409pt}{0.400pt}}
\put(371.0,204.0){\rule[-0.200pt]{2.409pt}{0.400pt}}
\put(1429.0,204.0){\rule[-0.200pt]{2.409pt}{0.400pt}}
\put(371.0,204.0){\rule[-0.200pt]{2.409pt}{0.400pt}}
\put(1429.0,204.0){\rule[-0.200pt]{2.409pt}{0.400pt}}
\put(371.0,204.0){\rule[-0.200pt]{2.409pt}{0.400pt}}
\put(1429.0,204.0){\rule[-0.200pt]{2.409pt}{0.400pt}}
\put(371.0,204.0){\rule[-0.200pt]{2.409pt}{0.400pt}}
\put(1429.0,204.0){\rule[-0.200pt]{2.409pt}{0.400pt}}
\put(371.0,204.0){\rule[-0.200pt]{2.409pt}{0.400pt}}
\put(1429.0,204.0){\rule[-0.200pt]{2.409pt}{0.400pt}}
\put(371.0,205.0){\rule[-0.200pt]{2.409pt}{0.400pt}}
\put(1429.0,205.0){\rule[-0.200pt]{2.409pt}{0.400pt}}
\put(371.0,205.0){\rule[-0.200pt]{2.409pt}{0.400pt}}
\put(1429.0,205.0){\rule[-0.200pt]{2.409pt}{0.400pt}}
\put(371.0,205.0){\rule[-0.200pt]{2.409pt}{0.400pt}}
\put(1429.0,205.0){\rule[-0.200pt]{2.409pt}{0.400pt}}
\put(371.0,205.0){\rule[-0.200pt]{2.409pt}{0.400pt}}
\put(1429.0,205.0){\rule[-0.200pt]{2.409pt}{0.400pt}}
\put(371.0,205.0){\rule[-0.200pt]{2.409pt}{0.400pt}}
\put(1429.0,205.0){\rule[-0.200pt]{2.409pt}{0.400pt}}
\put(371.0,205.0){\rule[-0.200pt]{2.409pt}{0.400pt}}
\put(1429.0,205.0){\rule[-0.200pt]{2.409pt}{0.400pt}}
\put(371.0,205.0){\rule[-0.200pt]{2.409pt}{0.400pt}}
\put(1429.0,205.0){\rule[-0.200pt]{2.409pt}{0.400pt}}
\put(371.0,205.0){\rule[-0.200pt]{2.409pt}{0.400pt}}
\put(1429.0,205.0){\rule[-0.200pt]{2.409pt}{0.400pt}}
\put(371.0,205.0){\rule[-0.200pt]{2.409pt}{0.400pt}}
\put(1429.0,205.0){\rule[-0.200pt]{2.409pt}{0.400pt}}
\put(371.0,205.0){\rule[-0.200pt]{2.409pt}{0.400pt}}
\put(1429.0,205.0){\rule[-0.200pt]{2.409pt}{0.400pt}}
\put(371.0,205.0){\rule[-0.200pt]{2.409pt}{0.400pt}}
\put(1429.0,205.0){\rule[-0.200pt]{2.409pt}{0.400pt}}
\put(371.0,205.0){\rule[-0.200pt]{2.409pt}{0.400pt}}
\put(1429.0,205.0){\rule[-0.200pt]{2.409pt}{0.400pt}}
\put(371.0,205.0){\rule[-0.200pt]{2.409pt}{0.400pt}}
\put(1429.0,205.0){\rule[-0.200pt]{2.409pt}{0.400pt}}
\put(371.0,206.0){\rule[-0.200pt]{2.409pt}{0.400pt}}
\put(1429.0,206.0){\rule[-0.200pt]{2.409pt}{0.400pt}}
\put(371.0,206.0){\rule[-0.200pt]{2.409pt}{0.400pt}}
\put(1429.0,206.0){\rule[-0.200pt]{2.409pt}{0.400pt}}
\put(371.0,206.0){\rule[-0.200pt]{2.409pt}{0.400pt}}
\put(1429.0,206.0){\rule[-0.200pt]{2.409pt}{0.400pt}}
\put(371.0,206.0){\rule[-0.200pt]{2.409pt}{0.400pt}}
\put(1429.0,206.0){\rule[-0.200pt]{2.409pt}{0.400pt}}
\put(371.0,206.0){\rule[-0.200pt]{2.409pt}{0.400pt}}
\put(1429.0,206.0){\rule[-0.200pt]{2.409pt}{0.400pt}}
\put(371.0,206.0){\rule[-0.200pt]{2.409pt}{0.400pt}}
\put(1429.0,206.0){\rule[-0.200pt]{2.409pt}{0.400pt}}
\put(371.0,206.0){\rule[-0.200pt]{2.409pt}{0.400pt}}
\put(1429.0,206.0){\rule[-0.200pt]{2.409pt}{0.400pt}}
\put(371.0,206.0){\rule[-0.200pt]{2.409pt}{0.400pt}}
\put(1429.0,206.0){\rule[-0.200pt]{2.409pt}{0.400pt}}
\put(371.0,206.0){\rule[-0.200pt]{2.409pt}{0.400pt}}
\put(1429.0,206.0){\rule[-0.200pt]{2.409pt}{0.400pt}}
\put(371.0,206.0){\rule[-0.200pt]{2.409pt}{0.400pt}}
\put(1429.0,206.0){\rule[-0.200pt]{2.409pt}{0.400pt}}
\put(371.0,206.0){\rule[-0.200pt]{2.409pt}{0.400pt}}
\put(1429.0,206.0){\rule[-0.200pt]{2.409pt}{0.400pt}}
\put(371.0,206.0){\rule[-0.200pt]{2.409pt}{0.400pt}}
\put(1429.0,206.0){\rule[-0.200pt]{2.409pt}{0.400pt}}
\put(371.0,206.0){\rule[-0.200pt]{2.409pt}{0.400pt}}
\put(1429.0,206.0){\rule[-0.200pt]{2.409pt}{0.400pt}}
\put(371.0,207.0){\rule[-0.200pt]{2.409pt}{0.400pt}}
\put(1429.0,207.0){\rule[-0.200pt]{2.409pt}{0.400pt}}
\put(371.0,207.0){\rule[-0.200pt]{2.409pt}{0.400pt}}
\put(1429.0,207.0){\rule[-0.200pt]{2.409pt}{0.400pt}}
\put(371.0,207.0){\rule[-0.200pt]{2.409pt}{0.400pt}}
\put(1429.0,207.0){\rule[-0.200pt]{2.409pt}{0.400pt}}
\put(371.0,207.0){\rule[-0.200pt]{2.409pt}{0.400pt}}
\put(1429.0,207.0){\rule[-0.200pt]{2.409pt}{0.400pt}}
\put(371.0,207.0){\rule[-0.200pt]{2.409pt}{0.400pt}}
\put(1429.0,207.0){\rule[-0.200pt]{2.409pt}{0.400pt}}
\put(371.0,207.0){\rule[-0.200pt]{2.409pt}{0.400pt}}
\put(1429.0,207.0){\rule[-0.200pt]{2.409pt}{0.400pt}}
\put(371.0,207.0){\rule[-0.200pt]{2.409pt}{0.400pt}}
\put(1429.0,207.0){\rule[-0.200pt]{2.409pt}{0.400pt}}
\put(371.0,207.0){\rule[-0.200pt]{2.409pt}{0.400pt}}
\put(1429.0,207.0){\rule[-0.200pt]{2.409pt}{0.400pt}}
\put(371.0,207.0){\rule[-0.200pt]{2.409pt}{0.400pt}}
\put(1429.0,207.0){\rule[-0.200pt]{2.409pt}{0.400pt}}
\put(371.0,207.0){\rule[-0.200pt]{2.409pt}{0.400pt}}
\put(1429.0,207.0){\rule[-0.200pt]{2.409pt}{0.400pt}}
\put(371.0,207.0){\rule[-0.200pt]{2.409pt}{0.400pt}}
\put(1429.0,207.0){\rule[-0.200pt]{2.409pt}{0.400pt}}
\put(371.0,207.0){\rule[-0.200pt]{2.409pt}{0.400pt}}
\put(1429.0,207.0){\rule[-0.200pt]{2.409pt}{0.400pt}}
\put(371.0,207.0){\rule[-0.200pt]{2.409pt}{0.400pt}}
\put(1429.0,207.0){\rule[-0.200pt]{2.409pt}{0.400pt}}
\put(371.0,207.0){\rule[-0.200pt]{2.409pt}{0.400pt}}
\put(1429.0,207.0){\rule[-0.200pt]{2.409pt}{0.400pt}}
\put(371.0,208.0){\rule[-0.200pt]{2.409pt}{0.400pt}}
\put(1429.0,208.0){\rule[-0.200pt]{2.409pt}{0.400pt}}
\put(371.0,208.0){\rule[-0.200pt]{2.409pt}{0.400pt}}
\put(1429.0,208.0){\rule[-0.200pt]{2.409pt}{0.400pt}}
\put(371.0,208.0){\rule[-0.200pt]{2.409pt}{0.400pt}}
\put(1429.0,208.0){\rule[-0.200pt]{2.409pt}{0.400pt}}
\put(371.0,208.0){\rule[-0.200pt]{2.409pt}{0.400pt}}
\put(1429.0,208.0){\rule[-0.200pt]{2.409pt}{0.400pt}}
\put(371.0,208.0){\rule[-0.200pt]{2.409pt}{0.400pt}}
\put(1429.0,208.0){\rule[-0.200pt]{2.409pt}{0.400pt}}
\put(371.0,208.0){\rule[-0.200pt]{2.409pt}{0.400pt}}
\put(1429.0,208.0){\rule[-0.200pt]{2.409pt}{0.400pt}}
\put(371.0,208.0){\rule[-0.200pt]{2.409pt}{0.400pt}}
\put(1429.0,208.0){\rule[-0.200pt]{2.409pt}{0.400pt}}
\put(371.0,208.0){\rule[-0.200pt]{2.409pt}{0.400pt}}
\put(1429.0,208.0){\rule[-0.200pt]{2.409pt}{0.400pt}}
\put(371.0,208.0){\rule[-0.200pt]{2.409pt}{0.400pt}}
\put(1429.0,208.0){\rule[-0.200pt]{2.409pt}{0.400pt}}
\put(371.0,208.0){\rule[-0.200pt]{2.409pt}{0.400pt}}
\put(1429.0,208.0){\rule[-0.200pt]{2.409pt}{0.400pt}}
\put(371.0,208.0){\rule[-0.200pt]{2.409pt}{0.400pt}}
\put(1429.0,208.0){\rule[-0.200pt]{2.409pt}{0.400pt}}
\put(371.0,208.0){\rule[-0.200pt]{2.409pt}{0.400pt}}
\put(1429.0,208.0){\rule[-0.200pt]{2.409pt}{0.400pt}}
\put(371.0,208.0){\rule[-0.200pt]{2.409pt}{0.400pt}}
\put(1429.0,208.0){\rule[-0.200pt]{2.409pt}{0.400pt}}
\put(371.0,208.0){\rule[-0.200pt]{2.409pt}{0.400pt}}
\put(1429.0,208.0){\rule[-0.200pt]{2.409pt}{0.400pt}}
\put(371.0,208.0){\rule[-0.200pt]{2.409pt}{0.400pt}}
\put(1429.0,208.0){\rule[-0.200pt]{2.409pt}{0.400pt}}
\put(371.0,209.0){\rule[-0.200pt]{2.409pt}{0.400pt}}
\put(1429.0,209.0){\rule[-0.200pt]{2.409pt}{0.400pt}}
\put(371.0,209.0){\rule[-0.200pt]{2.409pt}{0.400pt}}
\put(1429.0,209.0){\rule[-0.200pt]{2.409pt}{0.400pt}}
\put(371.0,209.0){\rule[-0.200pt]{2.409pt}{0.400pt}}
\put(1429.0,209.0){\rule[-0.200pt]{2.409pt}{0.400pt}}
\put(371.0,209.0){\rule[-0.200pt]{2.409pt}{0.400pt}}
\put(1429.0,209.0){\rule[-0.200pt]{2.409pt}{0.400pt}}
\put(371.0,209.0){\rule[-0.200pt]{2.409pt}{0.400pt}}
\put(1429.0,209.0){\rule[-0.200pt]{2.409pt}{0.400pt}}
\put(371.0,209.0){\rule[-0.200pt]{2.409pt}{0.400pt}}
\put(1429.0,209.0){\rule[-0.200pt]{2.409pt}{0.400pt}}
\put(371.0,209.0){\rule[-0.200pt]{2.409pt}{0.400pt}}
\put(1429.0,209.0){\rule[-0.200pt]{2.409pt}{0.400pt}}
\put(371.0,209.0){\rule[-0.200pt]{2.409pt}{0.400pt}}
\put(1429.0,209.0){\rule[-0.200pt]{2.409pt}{0.400pt}}
\put(371.0,209.0){\rule[-0.200pt]{2.409pt}{0.400pt}}
\put(1429.0,209.0){\rule[-0.200pt]{2.409pt}{0.400pt}}
\put(371.0,209.0){\rule[-0.200pt]{2.409pt}{0.400pt}}
\put(1429.0,209.0){\rule[-0.200pt]{2.409pt}{0.400pt}}
\put(371.0,209.0){\rule[-0.200pt]{2.409pt}{0.400pt}}
\put(1429.0,209.0){\rule[-0.200pt]{2.409pt}{0.400pt}}
\put(371.0,209.0){\rule[-0.200pt]{2.409pt}{0.400pt}}
\put(1429.0,209.0){\rule[-0.200pt]{2.409pt}{0.400pt}}
\put(371.0,209.0){\rule[-0.200pt]{2.409pt}{0.400pt}}
\put(1429.0,209.0){\rule[-0.200pt]{2.409pt}{0.400pt}}
\put(371.0,209.0){\rule[-0.200pt]{2.409pt}{0.400pt}}
\put(1429.0,209.0){\rule[-0.200pt]{2.409pt}{0.400pt}}
\put(371.0,209.0){\rule[-0.200pt]{2.409pt}{0.400pt}}
\put(1429.0,209.0){\rule[-0.200pt]{2.409pt}{0.400pt}}
\put(371.0,210.0){\rule[-0.200pt]{2.409pt}{0.400pt}}
\put(1429.0,210.0){\rule[-0.200pt]{2.409pt}{0.400pt}}
\put(371.0,210.0){\rule[-0.200pt]{2.409pt}{0.400pt}}
\put(1429.0,210.0){\rule[-0.200pt]{2.409pt}{0.400pt}}
\put(371.0,210.0){\rule[-0.200pt]{2.409pt}{0.400pt}}
\put(1429.0,210.0){\rule[-0.200pt]{2.409pt}{0.400pt}}
\put(371.0,210.0){\rule[-0.200pt]{2.409pt}{0.400pt}}
\put(1429.0,210.0){\rule[-0.200pt]{2.409pt}{0.400pt}}
\put(371.0,210.0){\rule[-0.200pt]{2.409pt}{0.400pt}}
\put(1429.0,210.0){\rule[-0.200pt]{2.409pt}{0.400pt}}
\put(371.0,210.0){\rule[-0.200pt]{2.409pt}{0.400pt}}
\put(1429.0,210.0){\rule[-0.200pt]{2.409pt}{0.400pt}}
\put(371.0,210.0){\rule[-0.200pt]{2.409pt}{0.400pt}}
\put(1429.0,210.0){\rule[-0.200pt]{2.409pt}{0.400pt}}
\put(371.0,210.0){\rule[-0.200pt]{2.409pt}{0.400pt}}
\put(1429.0,210.0){\rule[-0.200pt]{2.409pt}{0.400pt}}
\put(371.0,210.0){\rule[-0.200pt]{2.409pt}{0.400pt}}
\put(1429.0,210.0){\rule[-0.200pt]{2.409pt}{0.400pt}}
\put(371.0,210.0){\rule[-0.200pt]{2.409pt}{0.400pt}}
\put(1429.0,210.0){\rule[-0.200pt]{2.409pt}{0.400pt}}
\put(371.0,210.0){\rule[-0.200pt]{2.409pt}{0.400pt}}
\put(1429.0,210.0){\rule[-0.200pt]{2.409pt}{0.400pt}}
\put(371.0,210.0){\rule[-0.200pt]{2.409pt}{0.400pt}}
\put(1429.0,210.0){\rule[-0.200pt]{2.409pt}{0.400pt}}
\put(371.0,210.0){\rule[-0.200pt]{2.409pt}{0.400pt}}
\put(1429.0,210.0){\rule[-0.200pt]{2.409pt}{0.400pt}}
\put(371.0,210.0){\rule[-0.200pt]{2.409pt}{0.400pt}}
\put(1429.0,210.0){\rule[-0.200pt]{2.409pt}{0.400pt}}
\put(371.0,210.0){\rule[-0.200pt]{2.409pt}{0.400pt}}
\put(1429.0,210.0){\rule[-0.200pt]{2.409pt}{0.400pt}}
\put(371.0,210.0){\rule[-0.200pt]{2.409pt}{0.400pt}}
\put(1429.0,210.0){\rule[-0.200pt]{2.409pt}{0.400pt}}
\put(371.0,210.0){\rule[-0.200pt]{2.409pt}{0.400pt}}
\put(1429.0,210.0){\rule[-0.200pt]{2.409pt}{0.400pt}}
\put(371.0,211.0){\rule[-0.200pt]{2.409pt}{0.400pt}}
\put(1429.0,211.0){\rule[-0.200pt]{2.409pt}{0.400pt}}
\put(371.0,211.0){\rule[-0.200pt]{2.409pt}{0.400pt}}
\put(1429.0,211.0){\rule[-0.200pt]{2.409pt}{0.400pt}}
\put(371.0,211.0){\rule[-0.200pt]{2.409pt}{0.400pt}}
\put(1429.0,211.0){\rule[-0.200pt]{2.409pt}{0.400pt}}
\put(371.0,211.0){\rule[-0.200pt]{2.409pt}{0.400pt}}
\put(1429.0,211.0){\rule[-0.200pt]{2.409pt}{0.400pt}}
\put(371.0,211.0){\rule[-0.200pt]{2.409pt}{0.400pt}}
\put(1429.0,211.0){\rule[-0.200pt]{2.409pt}{0.400pt}}
\put(371.0,211.0){\rule[-0.200pt]{2.409pt}{0.400pt}}
\put(1429.0,211.0){\rule[-0.200pt]{2.409pt}{0.400pt}}
\put(371.0,211.0){\rule[-0.200pt]{2.409pt}{0.400pt}}
\put(1429.0,211.0){\rule[-0.200pt]{2.409pt}{0.400pt}}
\put(371.0,211.0){\rule[-0.200pt]{2.409pt}{0.400pt}}
\put(1429.0,211.0){\rule[-0.200pt]{2.409pt}{0.400pt}}
\put(371.0,211.0){\rule[-0.200pt]{2.409pt}{0.400pt}}
\put(1429.0,211.0){\rule[-0.200pt]{2.409pt}{0.400pt}}
\put(371.0,211.0){\rule[-0.200pt]{2.409pt}{0.400pt}}
\put(1429.0,211.0){\rule[-0.200pt]{2.409pt}{0.400pt}}
\put(371.0,211.0){\rule[-0.200pt]{2.409pt}{0.400pt}}
\put(1429.0,211.0){\rule[-0.200pt]{2.409pt}{0.400pt}}
\put(371.0,211.0){\rule[-0.200pt]{2.409pt}{0.400pt}}
\put(1429.0,211.0){\rule[-0.200pt]{2.409pt}{0.400pt}}
\put(371.0,211.0){\rule[-0.200pt]{2.409pt}{0.400pt}}
\put(1429.0,211.0){\rule[-0.200pt]{2.409pt}{0.400pt}}
\put(371.0,211.0){\rule[-0.200pt]{2.409pt}{0.400pt}}
\put(1429.0,211.0){\rule[-0.200pt]{2.409pt}{0.400pt}}
\put(371.0,211.0){\rule[-0.200pt]{2.409pt}{0.400pt}}
\put(1429.0,211.0){\rule[-0.200pt]{2.409pt}{0.400pt}}
\put(371.0,211.0){\rule[-0.200pt]{2.409pt}{0.400pt}}
\put(1429.0,211.0){\rule[-0.200pt]{2.409pt}{0.400pt}}
\put(371.0,211.0){\rule[-0.200pt]{2.409pt}{0.400pt}}
\put(1429.0,211.0){\rule[-0.200pt]{2.409pt}{0.400pt}}
\put(371.0,212.0){\rule[-0.200pt]{4.818pt}{0.400pt}}
\put(351,212){\makebox(0,0)[r]{ 1}}
\put(1419.0,212.0){\rule[-0.200pt]{4.818pt}{0.400pt}}
\put(371.0,224.0){\rule[-0.200pt]{2.409pt}{0.400pt}}
\put(1429.0,224.0){\rule[-0.200pt]{2.409pt}{0.400pt}}
\put(371.0,242.0){\rule[-0.200pt]{2.409pt}{0.400pt}}
\put(1429.0,242.0){\rule[-0.200pt]{2.409pt}{0.400pt}}
\put(371.0,250.0){\rule[-0.200pt]{2.409pt}{0.400pt}}
\put(1429.0,250.0){\rule[-0.200pt]{2.409pt}{0.400pt}}
\put(371.0,256.0){\rule[-0.200pt]{2.409pt}{0.400pt}}
\put(1429.0,256.0){\rule[-0.200pt]{2.409pt}{0.400pt}}
\put(371.0,261.0){\rule[-0.200pt]{2.409pt}{0.400pt}}
\put(1429.0,261.0){\rule[-0.200pt]{2.409pt}{0.400pt}}
\put(371.0,265.0){\rule[-0.200pt]{2.409pt}{0.400pt}}
\put(1429.0,265.0){\rule[-0.200pt]{2.409pt}{0.400pt}}
\put(371.0,268.0){\rule[-0.200pt]{2.409pt}{0.400pt}}
\put(1429.0,268.0){\rule[-0.200pt]{2.409pt}{0.400pt}}
\put(371.0,270.0){\rule[-0.200pt]{2.409pt}{0.400pt}}
\put(1429.0,270.0){\rule[-0.200pt]{2.409pt}{0.400pt}}
\put(371.0,273.0){\rule[-0.200pt]{2.409pt}{0.400pt}}
\put(1429.0,273.0){\rule[-0.200pt]{2.409pt}{0.400pt}}
\put(371.0,275.0){\rule[-0.200pt]{2.409pt}{0.400pt}}
\put(1429.0,275.0){\rule[-0.200pt]{2.409pt}{0.400pt}}
\put(371.0,276.0){\rule[-0.200pt]{2.409pt}{0.400pt}}
\put(1429.0,276.0){\rule[-0.200pt]{2.409pt}{0.400pt}}
\put(371.0,278.0){\rule[-0.200pt]{2.409pt}{0.400pt}}
\put(1429.0,278.0){\rule[-0.200pt]{2.409pt}{0.400pt}}
\put(371.0,280.0){\rule[-0.200pt]{2.409pt}{0.400pt}}
\put(1429.0,280.0){\rule[-0.200pt]{2.409pt}{0.400pt}}
\put(371.0,281.0){\rule[-0.200pt]{2.409pt}{0.400pt}}
\put(1429.0,281.0){\rule[-0.200pt]{2.409pt}{0.400pt}}
\put(371.0,282.0){\rule[-0.200pt]{2.409pt}{0.400pt}}
\put(1429.0,282.0){\rule[-0.200pt]{2.409pt}{0.400pt}}
\put(371.0,284.0){\rule[-0.200pt]{2.409pt}{0.400pt}}
\put(1429.0,284.0){\rule[-0.200pt]{2.409pt}{0.400pt}}
\put(371.0,285.0){\rule[-0.200pt]{2.409pt}{0.400pt}}
\put(1429.0,285.0){\rule[-0.200pt]{2.409pt}{0.400pt}}
\put(371.0,286.0){\rule[-0.200pt]{2.409pt}{0.400pt}}
\put(1429.0,286.0){\rule[-0.200pt]{2.409pt}{0.400pt}}
\put(371.0,287.0){\rule[-0.200pt]{2.409pt}{0.400pt}}
\put(1429.0,287.0){\rule[-0.200pt]{2.409pt}{0.400pt}}
\put(371.0,288.0){\rule[-0.200pt]{2.409pt}{0.400pt}}
\put(1429.0,288.0){\rule[-0.200pt]{2.409pt}{0.400pt}}
\put(371.0,289.0){\rule[-0.200pt]{2.409pt}{0.400pt}}
\put(1429.0,289.0){\rule[-0.200pt]{2.409pt}{0.400pt}}
\put(371.0,290.0){\rule[-0.200pt]{2.409pt}{0.400pt}}
\put(1429.0,290.0){\rule[-0.200pt]{2.409pt}{0.400pt}}
\put(371.0,291.0){\rule[-0.200pt]{2.409pt}{0.400pt}}
\put(1429.0,291.0){\rule[-0.200pt]{2.409pt}{0.400pt}}
\put(371.0,291.0){\rule[-0.200pt]{2.409pt}{0.400pt}}
\put(1429.0,291.0){\rule[-0.200pt]{2.409pt}{0.400pt}}
\put(371.0,292.0){\rule[-0.200pt]{2.409pt}{0.400pt}}
\put(1429.0,292.0){\rule[-0.200pt]{2.409pt}{0.400pt}}
\put(371.0,293.0){\rule[-0.200pt]{2.409pt}{0.400pt}}
\put(1429.0,293.0){\rule[-0.200pt]{2.409pt}{0.400pt}}
\put(371.0,294.0){\rule[-0.200pt]{2.409pt}{0.400pt}}
\put(1429.0,294.0){\rule[-0.200pt]{2.409pt}{0.400pt}}
\put(371.0,294.0){\rule[-0.200pt]{2.409pt}{0.400pt}}
\put(1429.0,294.0){\rule[-0.200pt]{2.409pt}{0.400pt}}
\put(371.0,295.0){\rule[-0.200pt]{2.409pt}{0.400pt}}
\put(1429.0,295.0){\rule[-0.200pt]{2.409pt}{0.400pt}}
\put(371.0,296.0){\rule[-0.200pt]{2.409pt}{0.400pt}}
\put(1429.0,296.0){\rule[-0.200pt]{2.409pt}{0.400pt}}
\put(371.0,296.0){\rule[-0.200pt]{2.409pt}{0.400pt}}
\put(1429.0,296.0){\rule[-0.200pt]{2.409pt}{0.400pt}}
\put(371.0,297.0){\rule[-0.200pt]{2.409pt}{0.400pt}}
\put(1429.0,297.0){\rule[-0.200pt]{2.409pt}{0.400pt}}
\put(371.0,297.0){\rule[-0.200pt]{2.409pt}{0.400pt}}
\put(1429.0,297.0){\rule[-0.200pt]{2.409pt}{0.400pt}}
\put(371.0,298.0){\rule[-0.200pt]{2.409pt}{0.400pt}}
\put(1429.0,298.0){\rule[-0.200pt]{2.409pt}{0.400pt}}
\put(371.0,299.0){\rule[-0.200pt]{2.409pt}{0.400pt}}
\put(1429.0,299.0){\rule[-0.200pt]{2.409pt}{0.400pt}}
\put(371.0,299.0){\rule[-0.200pt]{2.409pt}{0.400pt}}
\put(1429.0,299.0){\rule[-0.200pt]{2.409pt}{0.400pt}}
\put(371.0,300.0){\rule[-0.200pt]{2.409pt}{0.400pt}}
\put(1429.0,300.0){\rule[-0.200pt]{2.409pt}{0.400pt}}
\put(371.0,300.0){\rule[-0.200pt]{2.409pt}{0.400pt}}
\put(1429.0,300.0){\rule[-0.200pt]{2.409pt}{0.400pt}}
\put(371.0,301.0){\rule[-0.200pt]{2.409pt}{0.400pt}}
\put(1429.0,301.0){\rule[-0.200pt]{2.409pt}{0.400pt}}
\put(371.0,301.0){\rule[-0.200pt]{2.409pt}{0.400pt}}
\put(1429.0,301.0){\rule[-0.200pt]{2.409pt}{0.400pt}}
\put(371.0,302.0){\rule[-0.200pt]{2.409pt}{0.400pt}}
\put(1429.0,302.0){\rule[-0.200pt]{2.409pt}{0.400pt}}
\put(371.0,302.0){\rule[-0.200pt]{2.409pt}{0.400pt}}
\put(1429.0,302.0){\rule[-0.200pt]{2.409pt}{0.400pt}}
\put(371.0,302.0){\rule[-0.200pt]{2.409pt}{0.400pt}}
\put(1429.0,302.0){\rule[-0.200pt]{2.409pt}{0.400pt}}
\put(371.0,303.0){\rule[-0.200pt]{2.409pt}{0.400pt}}
\put(1429.0,303.0){\rule[-0.200pt]{2.409pt}{0.400pt}}
\put(371.0,303.0){\rule[-0.200pt]{2.409pt}{0.400pt}}
\put(1429.0,303.0){\rule[-0.200pt]{2.409pt}{0.400pt}}
\put(371.0,304.0){\rule[-0.200pt]{2.409pt}{0.400pt}}
\put(1429.0,304.0){\rule[-0.200pt]{2.409pt}{0.400pt}}
\put(371.0,304.0){\rule[-0.200pt]{2.409pt}{0.400pt}}
\put(1429.0,304.0){\rule[-0.200pt]{2.409pt}{0.400pt}}
\put(371.0,305.0){\rule[-0.200pt]{2.409pt}{0.400pt}}
\put(1429.0,305.0){\rule[-0.200pt]{2.409pt}{0.400pt}}
\put(371.0,305.0){\rule[-0.200pt]{2.409pt}{0.400pt}}
\put(1429.0,305.0){\rule[-0.200pt]{2.409pt}{0.400pt}}
\put(371.0,305.0){\rule[-0.200pt]{2.409pt}{0.400pt}}
\put(1429.0,305.0){\rule[-0.200pt]{2.409pt}{0.400pt}}
\put(371.0,306.0){\rule[-0.200pt]{2.409pt}{0.400pt}}
\put(1429.0,306.0){\rule[-0.200pt]{2.409pt}{0.400pt}}
\put(371.0,306.0){\rule[-0.200pt]{2.409pt}{0.400pt}}
\put(1429.0,306.0){\rule[-0.200pt]{2.409pt}{0.400pt}}
\put(371.0,306.0){\rule[-0.200pt]{2.409pt}{0.400pt}}
\put(1429.0,306.0){\rule[-0.200pt]{2.409pt}{0.400pt}}
\put(371.0,307.0){\rule[-0.200pt]{2.409pt}{0.400pt}}
\put(1429.0,307.0){\rule[-0.200pt]{2.409pt}{0.400pt}}
\put(371.0,307.0){\rule[-0.200pt]{2.409pt}{0.400pt}}
\put(1429.0,307.0){\rule[-0.200pt]{2.409pt}{0.400pt}}
\put(371.0,307.0){\rule[-0.200pt]{2.409pt}{0.400pt}}
\put(1429.0,307.0){\rule[-0.200pt]{2.409pt}{0.400pt}}
\put(371.0,308.0){\rule[-0.200pt]{2.409pt}{0.400pt}}
\put(1429.0,308.0){\rule[-0.200pt]{2.409pt}{0.400pt}}
\put(371.0,308.0){\rule[-0.200pt]{2.409pt}{0.400pt}}
\put(1429.0,308.0){\rule[-0.200pt]{2.409pt}{0.400pt}}
\put(371.0,308.0){\rule[-0.200pt]{2.409pt}{0.400pt}}
\put(1429.0,308.0){\rule[-0.200pt]{2.409pt}{0.400pt}}
\put(371.0,309.0){\rule[-0.200pt]{2.409pt}{0.400pt}}
\put(1429.0,309.0){\rule[-0.200pt]{2.409pt}{0.400pt}}
\put(371.0,309.0){\rule[-0.200pt]{2.409pt}{0.400pt}}
\put(1429.0,309.0){\rule[-0.200pt]{2.409pt}{0.400pt}}
\put(371.0,309.0){\rule[-0.200pt]{2.409pt}{0.400pt}}
\put(1429.0,309.0){\rule[-0.200pt]{2.409pt}{0.400pt}}
\put(371.0,310.0){\rule[-0.200pt]{2.409pt}{0.400pt}}
\put(1429.0,310.0){\rule[-0.200pt]{2.409pt}{0.400pt}}
\put(371.0,310.0){\rule[-0.200pt]{2.409pt}{0.400pt}}
\put(1429.0,310.0){\rule[-0.200pt]{2.409pt}{0.400pt}}
\put(371.0,310.0){\rule[-0.200pt]{2.409pt}{0.400pt}}
\put(1429.0,310.0){\rule[-0.200pt]{2.409pt}{0.400pt}}
\put(371.0,311.0){\rule[-0.200pt]{2.409pt}{0.400pt}}
\put(1429.0,311.0){\rule[-0.200pt]{2.409pt}{0.400pt}}
\put(371.0,311.0){\rule[-0.200pt]{2.409pt}{0.400pt}}
\put(1429.0,311.0){\rule[-0.200pt]{2.409pt}{0.400pt}}
\put(371.0,311.0){\rule[-0.200pt]{2.409pt}{0.400pt}}
\put(1429.0,311.0){\rule[-0.200pt]{2.409pt}{0.400pt}}
\put(371.0,311.0){\rule[-0.200pt]{2.409pt}{0.400pt}}
\put(1429.0,311.0){\rule[-0.200pt]{2.409pt}{0.400pt}}
\put(371.0,312.0){\rule[-0.200pt]{2.409pt}{0.400pt}}
\put(1429.0,312.0){\rule[-0.200pt]{2.409pt}{0.400pt}}
\put(371.0,312.0){\rule[-0.200pt]{2.409pt}{0.400pt}}
\put(1429.0,312.0){\rule[-0.200pt]{2.409pt}{0.400pt}}
\put(371.0,312.0){\rule[-0.200pt]{2.409pt}{0.400pt}}
\put(1429.0,312.0){\rule[-0.200pt]{2.409pt}{0.400pt}}
\put(371.0,312.0){\rule[-0.200pt]{2.409pt}{0.400pt}}
\put(1429.0,312.0){\rule[-0.200pt]{2.409pt}{0.400pt}}
\put(371.0,313.0){\rule[-0.200pt]{2.409pt}{0.400pt}}
\put(1429.0,313.0){\rule[-0.200pt]{2.409pt}{0.400pt}}
\put(371.0,313.0){\rule[-0.200pt]{2.409pt}{0.400pt}}
\put(1429.0,313.0){\rule[-0.200pt]{2.409pt}{0.400pt}}
\put(371.0,313.0){\rule[-0.200pt]{2.409pt}{0.400pt}}
\put(1429.0,313.0){\rule[-0.200pt]{2.409pt}{0.400pt}}
\put(371.0,313.0){\rule[-0.200pt]{2.409pt}{0.400pt}}
\put(1429.0,313.0){\rule[-0.200pt]{2.409pt}{0.400pt}}
\put(371.0,314.0){\rule[-0.200pt]{2.409pt}{0.400pt}}
\put(1429.0,314.0){\rule[-0.200pt]{2.409pt}{0.400pt}}
\put(371.0,314.0){\rule[-0.200pt]{2.409pt}{0.400pt}}
\put(1429.0,314.0){\rule[-0.200pt]{2.409pt}{0.400pt}}
\put(371.0,314.0){\rule[-0.200pt]{2.409pt}{0.400pt}}
\put(1429.0,314.0){\rule[-0.200pt]{2.409pt}{0.400pt}}
\put(371.0,314.0){\rule[-0.200pt]{2.409pt}{0.400pt}}
\put(1429.0,314.0){\rule[-0.200pt]{2.409pt}{0.400pt}}
\put(371.0,315.0){\rule[-0.200pt]{2.409pt}{0.400pt}}
\put(1429.0,315.0){\rule[-0.200pt]{2.409pt}{0.400pt}}
\put(371.0,315.0){\rule[-0.200pt]{2.409pt}{0.400pt}}
\put(1429.0,315.0){\rule[-0.200pt]{2.409pt}{0.400pt}}
\put(371.0,315.0){\rule[-0.200pt]{2.409pt}{0.400pt}}
\put(1429.0,315.0){\rule[-0.200pt]{2.409pt}{0.400pt}}
\put(371.0,315.0){\rule[-0.200pt]{2.409pt}{0.400pt}}
\put(1429.0,315.0){\rule[-0.200pt]{2.409pt}{0.400pt}}
\put(371.0,316.0){\rule[-0.200pt]{2.409pt}{0.400pt}}
\put(1429.0,316.0){\rule[-0.200pt]{2.409pt}{0.400pt}}
\put(371.0,316.0){\rule[-0.200pt]{2.409pt}{0.400pt}}
\put(1429.0,316.0){\rule[-0.200pt]{2.409pt}{0.400pt}}
\put(371.0,316.0){\rule[-0.200pt]{2.409pt}{0.400pt}}
\put(1429.0,316.0){\rule[-0.200pt]{2.409pt}{0.400pt}}
\put(371.0,316.0){\rule[-0.200pt]{2.409pt}{0.400pt}}
\put(1429.0,316.0){\rule[-0.200pt]{2.409pt}{0.400pt}}
\put(371.0,316.0){\rule[-0.200pt]{2.409pt}{0.400pt}}
\put(1429.0,316.0){\rule[-0.200pt]{2.409pt}{0.400pt}}
\put(371.0,317.0){\rule[-0.200pt]{2.409pt}{0.400pt}}
\put(1429.0,317.0){\rule[-0.200pt]{2.409pt}{0.400pt}}
\put(371.0,317.0){\rule[-0.200pt]{2.409pt}{0.400pt}}
\put(1429.0,317.0){\rule[-0.200pt]{2.409pt}{0.400pt}}
\put(371.0,317.0){\rule[-0.200pt]{2.409pt}{0.400pt}}
\put(1429.0,317.0){\rule[-0.200pt]{2.409pt}{0.400pt}}
\put(371.0,317.0){\rule[-0.200pt]{2.409pt}{0.400pt}}
\put(1429.0,317.0){\rule[-0.200pt]{2.409pt}{0.400pt}}
\put(371.0,317.0){\rule[-0.200pt]{2.409pt}{0.400pt}}
\put(1429.0,317.0){\rule[-0.200pt]{2.409pt}{0.400pt}}
\put(371.0,318.0){\rule[-0.200pt]{2.409pt}{0.400pt}}
\put(1429.0,318.0){\rule[-0.200pt]{2.409pt}{0.400pt}}
\put(371.0,318.0){\rule[-0.200pt]{2.409pt}{0.400pt}}
\put(1429.0,318.0){\rule[-0.200pt]{2.409pt}{0.400pt}}
\put(371.0,318.0){\rule[-0.200pt]{2.409pt}{0.400pt}}
\put(1429.0,318.0){\rule[-0.200pt]{2.409pt}{0.400pt}}
\put(371.0,318.0){\rule[-0.200pt]{2.409pt}{0.400pt}}
\put(1429.0,318.0){\rule[-0.200pt]{2.409pt}{0.400pt}}
\put(371.0,318.0){\rule[-0.200pt]{2.409pt}{0.400pt}}
\put(1429.0,318.0){\rule[-0.200pt]{2.409pt}{0.400pt}}
\put(371.0,319.0){\rule[-0.200pt]{2.409pt}{0.400pt}}
\put(1429.0,319.0){\rule[-0.200pt]{2.409pt}{0.400pt}}
\put(371.0,319.0){\rule[-0.200pt]{2.409pt}{0.400pt}}
\put(1429.0,319.0){\rule[-0.200pt]{2.409pt}{0.400pt}}
\put(371.0,319.0){\rule[-0.200pt]{2.409pt}{0.400pt}}
\put(1429.0,319.0){\rule[-0.200pt]{2.409pt}{0.400pt}}
\put(371.0,319.0){\rule[-0.200pt]{2.409pt}{0.400pt}}
\put(1429.0,319.0){\rule[-0.200pt]{2.409pt}{0.400pt}}
\put(371.0,319.0){\rule[-0.200pt]{2.409pt}{0.400pt}}
\put(1429.0,319.0){\rule[-0.200pt]{2.409pt}{0.400pt}}
\put(371.0,319.0){\rule[-0.200pt]{2.409pt}{0.400pt}}
\put(1429.0,319.0){\rule[-0.200pt]{2.409pt}{0.400pt}}
\put(371.0,320.0){\rule[-0.200pt]{2.409pt}{0.400pt}}
\put(1429.0,320.0){\rule[-0.200pt]{2.409pt}{0.400pt}}
\put(371.0,320.0){\rule[-0.200pt]{2.409pt}{0.400pt}}
\put(1429.0,320.0){\rule[-0.200pt]{2.409pt}{0.400pt}}
\put(371.0,320.0){\rule[-0.200pt]{2.409pt}{0.400pt}}
\put(1429.0,320.0){\rule[-0.200pt]{2.409pt}{0.400pt}}
\put(371.0,320.0){\rule[-0.200pt]{2.409pt}{0.400pt}}
\put(1429.0,320.0){\rule[-0.200pt]{2.409pt}{0.400pt}}
\put(371.0,320.0){\rule[-0.200pt]{2.409pt}{0.400pt}}
\put(1429.0,320.0){\rule[-0.200pt]{2.409pt}{0.400pt}}
\put(371.0,320.0){\rule[-0.200pt]{2.409pt}{0.400pt}}
\put(1429.0,320.0){\rule[-0.200pt]{2.409pt}{0.400pt}}
\put(371.0,321.0){\rule[-0.200pt]{2.409pt}{0.400pt}}
\put(1429.0,321.0){\rule[-0.200pt]{2.409pt}{0.400pt}}
\put(371.0,321.0){\rule[-0.200pt]{2.409pt}{0.400pt}}
\put(1429.0,321.0){\rule[-0.200pt]{2.409pt}{0.400pt}}
\put(371.0,321.0){\rule[-0.200pt]{2.409pt}{0.400pt}}
\put(1429.0,321.0){\rule[-0.200pt]{2.409pt}{0.400pt}}
\put(371.0,321.0){\rule[-0.200pt]{2.409pt}{0.400pt}}
\put(1429.0,321.0){\rule[-0.200pt]{2.409pt}{0.400pt}}
\put(371.0,321.0){\rule[-0.200pt]{2.409pt}{0.400pt}}
\put(1429.0,321.0){\rule[-0.200pt]{2.409pt}{0.400pt}}
\put(371.0,321.0){\rule[-0.200pt]{2.409pt}{0.400pt}}
\put(1429.0,321.0){\rule[-0.200pt]{2.409pt}{0.400pt}}
\put(371.0,322.0){\rule[-0.200pt]{2.409pt}{0.400pt}}
\put(1429.0,322.0){\rule[-0.200pt]{2.409pt}{0.400pt}}
\put(371.0,322.0){\rule[-0.200pt]{2.409pt}{0.400pt}}
\put(1429.0,322.0){\rule[-0.200pt]{2.409pt}{0.400pt}}
\put(371.0,322.0){\rule[-0.200pt]{2.409pt}{0.400pt}}
\put(1429.0,322.0){\rule[-0.200pt]{2.409pt}{0.400pt}}
\put(371.0,322.0){\rule[-0.200pt]{2.409pt}{0.400pt}}
\put(1429.0,322.0){\rule[-0.200pt]{2.409pt}{0.400pt}}
\put(371.0,322.0){\rule[-0.200pt]{2.409pt}{0.400pt}}
\put(1429.0,322.0){\rule[-0.200pt]{2.409pt}{0.400pt}}
\put(371.0,322.0){\rule[-0.200pt]{2.409pt}{0.400pt}}
\put(1429.0,322.0){\rule[-0.200pt]{2.409pt}{0.400pt}}
\put(371.0,323.0){\rule[-0.200pt]{2.409pt}{0.400pt}}
\put(1429.0,323.0){\rule[-0.200pt]{2.409pt}{0.400pt}}
\put(371.0,323.0){\rule[-0.200pt]{2.409pt}{0.400pt}}
\put(1429.0,323.0){\rule[-0.200pt]{2.409pt}{0.400pt}}
\put(371.0,323.0){\rule[-0.200pt]{2.409pt}{0.400pt}}
\put(1429.0,323.0){\rule[-0.200pt]{2.409pt}{0.400pt}}
\put(371.0,323.0){\rule[-0.200pt]{2.409pt}{0.400pt}}
\put(1429.0,323.0){\rule[-0.200pt]{2.409pt}{0.400pt}}
\put(371.0,323.0){\rule[-0.200pt]{2.409pt}{0.400pt}}
\put(1429.0,323.0){\rule[-0.200pt]{2.409pt}{0.400pt}}
\put(371.0,323.0){\rule[-0.200pt]{2.409pt}{0.400pt}}
\put(1429.0,323.0){\rule[-0.200pt]{2.409pt}{0.400pt}}
\put(371.0,323.0){\rule[-0.200pt]{2.409pt}{0.400pt}}
\put(1429.0,323.0){\rule[-0.200pt]{2.409pt}{0.400pt}}
\put(371.0,324.0){\rule[-0.200pt]{2.409pt}{0.400pt}}
\put(1429.0,324.0){\rule[-0.200pt]{2.409pt}{0.400pt}}
\put(371.0,324.0){\rule[-0.200pt]{2.409pt}{0.400pt}}
\put(1429.0,324.0){\rule[-0.200pt]{2.409pt}{0.400pt}}
\put(371.0,324.0){\rule[-0.200pt]{2.409pt}{0.400pt}}
\put(1429.0,324.0){\rule[-0.200pt]{2.409pt}{0.400pt}}
\put(371.0,324.0){\rule[-0.200pt]{2.409pt}{0.400pt}}
\put(1429.0,324.0){\rule[-0.200pt]{2.409pt}{0.400pt}}
\put(371.0,324.0){\rule[-0.200pt]{2.409pt}{0.400pt}}
\put(1429.0,324.0){\rule[-0.200pt]{2.409pt}{0.400pt}}
\put(371.0,324.0){\rule[-0.200pt]{2.409pt}{0.400pt}}
\put(1429.0,324.0){\rule[-0.200pt]{2.409pt}{0.400pt}}
\put(371.0,324.0){\rule[-0.200pt]{2.409pt}{0.400pt}}
\put(1429.0,324.0){\rule[-0.200pt]{2.409pt}{0.400pt}}
\put(371.0,325.0){\rule[-0.200pt]{2.409pt}{0.400pt}}
\put(1429.0,325.0){\rule[-0.200pt]{2.409pt}{0.400pt}}
\put(371.0,325.0){\rule[-0.200pt]{2.409pt}{0.400pt}}
\put(1429.0,325.0){\rule[-0.200pt]{2.409pt}{0.400pt}}
\put(371.0,325.0){\rule[-0.200pt]{2.409pt}{0.400pt}}
\put(1429.0,325.0){\rule[-0.200pt]{2.409pt}{0.400pt}}
\put(371.0,325.0){\rule[-0.200pt]{2.409pt}{0.400pt}}
\put(1429.0,325.0){\rule[-0.200pt]{2.409pt}{0.400pt}}
\put(371.0,325.0){\rule[-0.200pt]{2.409pt}{0.400pt}}
\put(1429.0,325.0){\rule[-0.200pt]{2.409pt}{0.400pt}}
\put(371.0,325.0){\rule[-0.200pt]{2.409pt}{0.400pt}}
\put(1429.0,325.0){\rule[-0.200pt]{2.409pt}{0.400pt}}
\put(371.0,325.0){\rule[-0.200pt]{2.409pt}{0.400pt}}
\put(1429.0,325.0){\rule[-0.200pt]{2.409pt}{0.400pt}}
\put(371.0,325.0){\rule[-0.200pt]{2.409pt}{0.400pt}}
\put(1429.0,325.0){\rule[-0.200pt]{2.409pt}{0.400pt}}
\put(371.0,326.0){\rule[-0.200pt]{2.409pt}{0.400pt}}
\put(1429.0,326.0){\rule[-0.200pt]{2.409pt}{0.400pt}}
\put(371.0,326.0){\rule[-0.200pt]{2.409pt}{0.400pt}}
\put(1429.0,326.0){\rule[-0.200pt]{2.409pt}{0.400pt}}
\put(371.0,326.0){\rule[-0.200pt]{2.409pt}{0.400pt}}
\put(1429.0,326.0){\rule[-0.200pt]{2.409pt}{0.400pt}}
\put(371.0,326.0){\rule[-0.200pt]{2.409pt}{0.400pt}}
\put(1429.0,326.0){\rule[-0.200pt]{2.409pt}{0.400pt}}
\put(371.0,326.0){\rule[-0.200pt]{2.409pt}{0.400pt}}
\put(1429.0,326.0){\rule[-0.200pt]{2.409pt}{0.400pt}}
\put(371.0,326.0){\rule[-0.200pt]{2.409pt}{0.400pt}}
\put(1429.0,326.0){\rule[-0.200pt]{2.409pt}{0.400pt}}
\put(371.0,326.0){\rule[-0.200pt]{2.409pt}{0.400pt}}
\put(1429.0,326.0){\rule[-0.200pt]{2.409pt}{0.400pt}}
\put(371.0,326.0){\rule[-0.200pt]{2.409pt}{0.400pt}}
\put(1429.0,326.0){\rule[-0.200pt]{2.409pt}{0.400pt}}
\put(371.0,327.0){\rule[-0.200pt]{2.409pt}{0.400pt}}
\put(1429.0,327.0){\rule[-0.200pt]{2.409pt}{0.400pt}}
\put(371.0,327.0){\rule[-0.200pt]{2.409pt}{0.400pt}}
\put(1429.0,327.0){\rule[-0.200pt]{2.409pt}{0.400pt}}
\put(371.0,327.0){\rule[-0.200pt]{2.409pt}{0.400pt}}
\put(1429.0,327.0){\rule[-0.200pt]{2.409pt}{0.400pt}}
\put(371.0,327.0){\rule[-0.200pt]{2.409pt}{0.400pt}}
\put(1429.0,327.0){\rule[-0.200pt]{2.409pt}{0.400pt}}
\put(371.0,327.0){\rule[-0.200pt]{2.409pt}{0.400pt}}
\put(1429.0,327.0){\rule[-0.200pt]{2.409pt}{0.400pt}}
\put(371.0,327.0){\rule[-0.200pt]{2.409pt}{0.400pt}}
\put(1429.0,327.0){\rule[-0.200pt]{2.409pt}{0.400pt}}
\put(371.0,327.0){\rule[-0.200pt]{2.409pt}{0.400pt}}
\put(1429.0,327.0){\rule[-0.200pt]{2.409pt}{0.400pt}}
\put(371.0,327.0){\rule[-0.200pt]{2.409pt}{0.400pt}}
\put(1429.0,327.0){\rule[-0.200pt]{2.409pt}{0.400pt}}
\put(371.0,328.0){\rule[-0.200pt]{2.409pt}{0.400pt}}
\put(1429.0,328.0){\rule[-0.200pt]{2.409pt}{0.400pt}}
\put(371.0,328.0){\rule[-0.200pt]{2.409pt}{0.400pt}}
\put(1429.0,328.0){\rule[-0.200pt]{2.409pt}{0.400pt}}
\put(371.0,328.0){\rule[-0.200pt]{2.409pt}{0.400pt}}
\put(1429.0,328.0){\rule[-0.200pt]{2.409pt}{0.400pt}}
\put(371.0,328.0){\rule[-0.200pt]{2.409pt}{0.400pt}}
\put(1429.0,328.0){\rule[-0.200pt]{2.409pt}{0.400pt}}
\put(371.0,328.0){\rule[-0.200pt]{2.409pt}{0.400pt}}
\put(1429.0,328.0){\rule[-0.200pt]{2.409pt}{0.400pt}}
\put(371.0,328.0){\rule[-0.200pt]{2.409pt}{0.400pt}}
\put(1429.0,328.0){\rule[-0.200pt]{2.409pt}{0.400pt}}
\put(371.0,328.0){\rule[-0.200pt]{2.409pt}{0.400pt}}
\put(1429.0,328.0){\rule[-0.200pt]{2.409pt}{0.400pt}}
\put(371.0,328.0){\rule[-0.200pt]{2.409pt}{0.400pt}}
\put(1429.0,328.0){\rule[-0.200pt]{2.409pt}{0.400pt}}
\put(371.0,328.0){\rule[-0.200pt]{2.409pt}{0.400pt}}
\put(1429.0,328.0){\rule[-0.200pt]{2.409pt}{0.400pt}}
\put(371.0,329.0){\rule[-0.200pt]{2.409pt}{0.400pt}}
\put(1429.0,329.0){\rule[-0.200pt]{2.409pt}{0.400pt}}
\put(371.0,329.0){\rule[-0.200pt]{2.409pt}{0.400pt}}
\put(1429.0,329.0){\rule[-0.200pt]{2.409pt}{0.400pt}}
\put(371.0,329.0){\rule[-0.200pt]{2.409pt}{0.400pt}}
\put(1429.0,329.0){\rule[-0.200pt]{2.409pt}{0.400pt}}
\put(371.0,329.0){\rule[-0.200pt]{2.409pt}{0.400pt}}
\put(1429.0,329.0){\rule[-0.200pt]{2.409pt}{0.400pt}}
\put(371.0,329.0){\rule[-0.200pt]{2.409pt}{0.400pt}}
\put(1429.0,329.0){\rule[-0.200pt]{2.409pt}{0.400pt}}
\put(371.0,329.0){\rule[-0.200pt]{2.409pt}{0.400pt}}
\put(1429.0,329.0){\rule[-0.200pt]{2.409pt}{0.400pt}}
\put(371.0,329.0){\rule[-0.200pt]{2.409pt}{0.400pt}}
\put(1429.0,329.0){\rule[-0.200pt]{2.409pt}{0.400pt}}
\put(371.0,329.0){\rule[-0.200pt]{2.409pt}{0.400pt}}
\put(1429.0,329.0){\rule[-0.200pt]{2.409pt}{0.400pt}}
\put(371.0,329.0){\rule[-0.200pt]{2.409pt}{0.400pt}}
\put(1429.0,329.0){\rule[-0.200pt]{2.409pt}{0.400pt}}
\put(371.0,330.0){\rule[-0.200pt]{2.409pt}{0.400pt}}
\put(1429.0,330.0){\rule[-0.200pt]{2.409pt}{0.400pt}}
\put(371.0,330.0){\rule[-0.200pt]{2.409pt}{0.400pt}}
\put(1429.0,330.0){\rule[-0.200pt]{2.409pt}{0.400pt}}
\put(371.0,330.0){\rule[-0.200pt]{2.409pt}{0.400pt}}
\put(1429.0,330.0){\rule[-0.200pt]{2.409pt}{0.400pt}}
\put(371.0,330.0){\rule[-0.200pt]{2.409pt}{0.400pt}}
\put(1429.0,330.0){\rule[-0.200pt]{2.409pt}{0.400pt}}
\put(371.0,330.0){\rule[-0.200pt]{2.409pt}{0.400pt}}
\put(1429.0,330.0){\rule[-0.200pt]{2.409pt}{0.400pt}}
\put(371.0,330.0){\rule[-0.200pt]{2.409pt}{0.400pt}}
\put(1429.0,330.0){\rule[-0.200pt]{2.409pt}{0.400pt}}
\put(371.0,330.0){\rule[-0.200pt]{2.409pt}{0.400pt}}
\put(1429.0,330.0){\rule[-0.200pt]{2.409pt}{0.400pt}}
\put(371.0,330.0){\rule[-0.200pt]{2.409pt}{0.400pt}}
\put(1429.0,330.0){\rule[-0.200pt]{2.409pt}{0.400pt}}
\put(371.0,330.0){\rule[-0.200pt]{2.409pt}{0.400pt}}
\put(1429.0,330.0){\rule[-0.200pt]{2.409pt}{0.400pt}}
\put(371.0,330.0){\rule[-0.200pt]{2.409pt}{0.400pt}}
\put(1429.0,330.0){\rule[-0.200pt]{2.409pt}{0.400pt}}
\put(371.0,331.0){\rule[-0.200pt]{2.409pt}{0.400pt}}
\put(1429.0,331.0){\rule[-0.200pt]{2.409pt}{0.400pt}}
\put(371.0,331.0){\rule[-0.200pt]{2.409pt}{0.400pt}}
\put(1429.0,331.0){\rule[-0.200pt]{2.409pt}{0.400pt}}
\put(371.0,331.0){\rule[-0.200pt]{2.409pt}{0.400pt}}
\put(1429.0,331.0){\rule[-0.200pt]{2.409pt}{0.400pt}}
\put(371.0,331.0){\rule[-0.200pt]{2.409pt}{0.400pt}}
\put(1429.0,331.0){\rule[-0.200pt]{2.409pt}{0.400pt}}
\put(371.0,331.0){\rule[-0.200pt]{2.409pt}{0.400pt}}
\put(1429.0,331.0){\rule[-0.200pt]{2.409pt}{0.400pt}}
\put(371.0,331.0){\rule[-0.200pt]{2.409pt}{0.400pt}}
\put(1429.0,331.0){\rule[-0.200pt]{2.409pt}{0.400pt}}
\put(371.0,331.0){\rule[-0.200pt]{2.409pt}{0.400pt}}
\put(1429.0,331.0){\rule[-0.200pt]{2.409pt}{0.400pt}}
\put(371.0,331.0){\rule[-0.200pt]{2.409pt}{0.400pt}}
\put(1429.0,331.0){\rule[-0.200pt]{2.409pt}{0.400pt}}
\put(371.0,331.0){\rule[-0.200pt]{2.409pt}{0.400pt}}
\put(1429.0,331.0){\rule[-0.200pt]{2.409pt}{0.400pt}}
\put(371.0,331.0){\rule[-0.200pt]{2.409pt}{0.400pt}}
\put(1429.0,331.0){\rule[-0.200pt]{2.409pt}{0.400pt}}
\put(371.0,331.0){\rule[-0.200pt]{2.409pt}{0.400pt}}
\put(1429.0,331.0){\rule[-0.200pt]{2.409pt}{0.400pt}}
\put(371.0,332.0){\rule[-0.200pt]{2.409pt}{0.400pt}}
\put(1429.0,332.0){\rule[-0.200pt]{2.409pt}{0.400pt}}
\put(371.0,332.0){\rule[-0.200pt]{2.409pt}{0.400pt}}
\put(1429.0,332.0){\rule[-0.200pt]{2.409pt}{0.400pt}}
\put(371.0,332.0){\rule[-0.200pt]{2.409pt}{0.400pt}}
\put(1429.0,332.0){\rule[-0.200pt]{2.409pt}{0.400pt}}
\put(371.0,332.0){\rule[-0.200pt]{2.409pt}{0.400pt}}
\put(1429.0,332.0){\rule[-0.200pt]{2.409pt}{0.400pt}}
\put(371.0,332.0){\rule[-0.200pt]{2.409pt}{0.400pt}}
\put(1429.0,332.0){\rule[-0.200pt]{2.409pt}{0.400pt}}
\put(371.0,332.0){\rule[-0.200pt]{2.409pt}{0.400pt}}
\put(1429.0,332.0){\rule[-0.200pt]{2.409pt}{0.400pt}}
\put(371.0,332.0){\rule[-0.200pt]{2.409pt}{0.400pt}}
\put(1429.0,332.0){\rule[-0.200pt]{2.409pt}{0.400pt}}
\put(371.0,332.0){\rule[-0.200pt]{2.409pt}{0.400pt}}
\put(1429.0,332.0){\rule[-0.200pt]{2.409pt}{0.400pt}}
\put(371.0,332.0){\rule[-0.200pt]{2.409pt}{0.400pt}}
\put(1429.0,332.0){\rule[-0.200pt]{2.409pt}{0.400pt}}
\put(371.0,332.0){\rule[-0.200pt]{2.409pt}{0.400pt}}
\put(1429.0,332.0){\rule[-0.200pt]{2.409pt}{0.400pt}}
\put(371.0,332.0){\rule[-0.200pt]{2.409pt}{0.400pt}}
\put(1429.0,332.0){\rule[-0.200pt]{2.409pt}{0.400pt}}
\put(371.0,333.0){\rule[-0.200pt]{2.409pt}{0.400pt}}
\put(1429.0,333.0){\rule[-0.200pt]{2.409pt}{0.400pt}}
\put(371.0,333.0){\rule[-0.200pt]{2.409pt}{0.400pt}}
\put(1429.0,333.0){\rule[-0.200pt]{2.409pt}{0.400pt}}
\put(371.0,333.0){\rule[-0.200pt]{2.409pt}{0.400pt}}
\put(1429.0,333.0){\rule[-0.200pt]{2.409pt}{0.400pt}}
\put(371.0,333.0){\rule[-0.200pt]{2.409pt}{0.400pt}}
\put(1429.0,333.0){\rule[-0.200pt]{2.409pt}{0.400pt}}
\put(371.0,333.0){\rule[-0.200pt]{2.409pt}{0.400pt}}
\put(1429.0,333.0){\rule[-0.200pt]{2.409pt}{0.400pt}}
\put(371.0,333.0){\rule[-0.200pt]{2.409pt}{0.400pt}}
\put(1429.0,333.0){\rule[-0.200pt]{2.409pt}{0.400pt}}
\put(371.0,333.0){\rule[-0.200pt]{2.409pt}{0.400pt}}
\put(1429.0,333.0){\rule[-0.200pt]{2.409pt}{0.400pt}}
\put(371.0,333.0){\rule[-0.200pt]{2.409pt}{0.400pt}}
\put(1429.0,333.0){\rule[-0.200pt]{2.409pt}{0.400pt}}
\put(371.0,333.0){\rule[-0.200pt]{2.409pt}{0.400pt}}
\put(1429.0,333.0){\rule[-0.200pt]{2.409pt}{0.400pt}}
\put(371.0,333.0){\rule[-0.200pt]{2.409pt}{0.400pt}}
\put(1429.0,333.0){\rule[-0.200pt]{2.409pt}{0.400pt}}
\put(371.0,333.0){\rule[-0.200pt]{2.409pt}{0.400pt}}
\put(1429.0,333.0){\rule[-0.200pt]{2.409pt}{0.400pt}}
\put(371.0,334.0){\rule[-0.200pt]{2.409pt}{0.400pt}}
\put(1429.0,334.0){\rule[-0.200pt]{2.409pt}{0.400pt}}
\put(371.0,334.0){\rule[-0.200pt]{2.409pt}{0.400pt}}
\put(1429.0,334.0){\rule[-0.200pt]{2.409pt}{0.400pt}}
\put(371.0,334.0){\rule[-0.200pt]{2.409pt}{0.400pt}}
\put(1429.0,334.0){\rule[-0.200pt]{2.409pt}{0.400pt}}
\put(371.0,334.0){\rule[-0.200pt]{2.409pt}{0.400pt}}
\put(1429.0,334.0){\rule[-0.200pt]{2.409pt}{0.400pt}}
\put(371.0,334.0){\rule[-0.200pt]{2.409pt}{0.400pt}}
\put(1429.0,334.0){\rule[-0.200pt]{2.409pt}{0.400pt}}
\put(371.0,334.0){\rule[-0.200pt]{2.409pt}{0.400pt}}
\put(1429.0,334.0){\rule[-0.200pt]{2.409pt}{0.400pt}}
\put(371.0,334.0){\rule[-0.200pt]{2.409pt}{0.400pt}}
\put(1429.0,334.0){\rule[-0.200pt]{2.409pt}{0.400pt}}
\put(371.0,334.0){\rule[-0.200pt]{2.409pt}{0.400pt}}
\put(1429.0,334.0){\rule[-0.200pt]{2.409pt}{0.400pt}}
\put(371.0,334.0){\rule[-0.200pt]{2.409pt}{0.400pt}}
\put(1429.0,334.0){\rule[-0.200pt]{2.409pt}{0.400pt}}
\put(371.0,334.0){\rule[-0.200pt]{2.409pt}{0.400pt}}
\put(1429.0,334.0){\rule[-0.200pt]{2.409pt}{0.400pt}}
\put(371.0,334.0){\rule[-0.200pt]{2.409pt}{0.400pt}}
\put(1429.0,334.0){\rule[-0.200pt]{2.409pt}{0.400pt}}
\put(371.0,334.0){\rule[-0.200pt]{2.409pt}{0.400pt}}
\put(1429.0,334.0){\rule[-0.200pt]{2.409pt}{0.400pt}}
\put(371.0,334.0){\rule[-0.200pt]{2.409pt}{0.400pt}}
\put(1429.0,334.0){\rule[-0.200pt]{2.409pt}{0.400pt}}
\put(371.0,335.0){\rule[-0.200pt]{2.409pt}{0.400pt}}
\put(1429.0,335.0){\rule[-0.200pt]{2.409pt}{0.400pt}}
\put(371.0,335.0){\rule[-0.200pt]{2.409pt}{0.400pt}}
\put(1429.0,335.0){\rule[-0.200pt]{2.409pt}{0.400pt}}
\put(371.0,335.0){\rule[-0.200pt]{2.409pt}{0.400pt}}
\put(1429.0,335.0){\rule[-0.200pt]{2.409pt}{0.400pt}}
\put(371.0,335.0){\rule[-0.200pt]{2.409pt}{0.400pt}}
\put(1429.0,335.0){\rule[-0.200pt]{2.409pt}{0.400pt}}
\put(371.0,335.0){\rule[-0.200pt]{2.409pt}{0.400pt}}
\put(1429.0,335.0){\rule[-0.200pt]{2.409pt}{0.400pt}}
\put(371.0,335.0){\rule[-0.200pt]{2.409pt}{0.400pt}}
\put(1429.0,335.0){\rule[-0.200pt]{2.409pt}{0.400pt}}
\put(371.0,335.0){\rule[-0.200pt]{2.409pt}{0.400pt}}
\put(1429.0,335.0){\rule[-0.200pt]{2.409pt}{0.400pt}}
\put(371.0,335.0){\rule[-0.200pt]{2.409pt}{0.400pt}}
\put(1429.0,335.0){\rule[-0.200pt]{2.409pt}{0.400pt}}
\put(371.0,335.0){\rule[-0.200pt]{2.409pt}{0.400pt}}
\put(1429.0,335.0){\rule[-0.200pt]{2.409pt}{0.400pt}}
\put(371.0,335.0){\rule[-0.200pt]{2.409pt}{0.400pt}}
\put(1429.0,335.0){\rule[-0.200pt]{2.409pt}{0.400pt}}
\put(371.0,335.0){\rule[-0.200pt]{2.409pt}{0.400pt}}
\put(1429.0,335.0){\rule[-0.200pt]{2.409pt}{0.400pt}}
\put(371.0,335.0){\rule[-0.200pt]{2.409pt}{0.400pt}}
\put(1429.0,335.0){\rule[-0.200pt]{2.409pt}{0.400pt}}
\put(371.0,336.0){\rule[-0.200pt]{2.409pt}{0.400pt}}
\put(1429.0,336.0){\rule[-0.200pt]{2.409pt}{0.400pt}}
\put(371.0,336.0){\rule[-0.200pt]{2.409pt}{0.400pt}}
\put(1429.0,336.0){\rule[-0.200pt]{2.409pt}{0.400pt}}
\put(371.0,336.0){\rule[-0.200pt]{2.409pt}{0.400pt}}
\put(1429.0,336.0){\rule[-0.200pt]{2.409pt}{0.400pt}}
\put(371.0,336.0){\rule[-0.200pt]{2.409pt}{0.400pt}}
\put(1429.0,336.0){\rule[-0.200pt]{2.409pt}{0.400pt}}
\put(371.0,336.0){\rule[-0.200pt]{2.409pt}{0.400pt}}
\put(1429.0,336.0){\rule[-0.200pt]{2.409pt}{0.400pt}}
\put(371.0,336.0){\rule[-0.200pt]{2.409pt}{0.400pt}}
\put(1429.0,336.0){\rule[-0.200pt]{2.409pt}{0.400pt}}
\put(371.0,336.0){\rule[-0.200pt]{2.409pt}{0.400pt}}
\put(1429.0,336.0){\rule[-0.200pt]{2.409pt}{0.400pt}}
\put(371.0,336.0){\rule[-0.200pt]{2.409pt}{0.400pt}}
\put(1429.0,336.0){\rule[-0.200pt]{2.409pt}{0.400pt}}
\put(371.0,336.0){\rule[-0.200pt]{2.409pt}{0.400pt}}
\put(1429.0,336.0){\rule[-0.200pt]{2.409pt}{0.400pt}}
\put(371.0,336.0){\rule[-0.200pt]{2.409pt}{0.400pt}}
\put(1429.0,336.0){\rule[-0.200pt]{2.409pt}{0.400pt}}
\put(371.0,336.0){\rule[-0.200pt]{2.409pt}{0.400pt}}
\put(1429.0,336.0){\rule[-0.200pt]{2.409pt}{0.400pt}}
\put(371.0,336.0){\rule[-0.200pt]{2.409pt}{0.400pt}}
\put(1429.0,336.0){\rule[-0.200pt]{2.409pt}{0.400pt}}
\put(371.0,336.0){\rule[-0.200pt]{2.409pt}{0.400pt}}
\put(1429.0,336.0){\rule[-0.200pt]{2.409pt}{0.400pt}}
\put(371.0,336.0){\rule[-0.200pt]{2.409pt}{0.400pt}}
\put(1429.0,336.0){\rule[-0.200pt]{2.409pt}{0.400pt}}
\put(371.0,337.0){\rule[-0.200pt]{2.409pt}{0.400pt}}
\put(1429.0,337.0){\rule[-0.200pt]{2.409pt}{0.400pt}}
\put(371.0,337.0){\rule[-0.200pt]{2.409pt}{0.400pt}}
\put(1429.0,337.0){\rule[-0.200pt]{2.409pt}{0.400pt}}
\put(371.0,337.0){\rule[-0.200pt]{2.409pt}{0.400pt}}
\put(1429.0,337.0){\rule[-0.200pt]{2.409pt}{0.400pt}}
\put(371.0,337.0){\rule[-0.200pt]{2.409pt}{0.400pt}}
\put(1429.0,337.0){\rule[-0.200pt]{2.409pt}{0.400pt}}
\put(371.0,337.0){\rule[-0.200pt]{2.409pt}{0.400pt}}
\put(1429.0,337.0){\rule[-0.200pt]{2.409pt}{0.400pt}}
\put(371.0,337.0){\rule[-0.200pt]{2.409pt}{0.400pt}}
\put(1429.0,337.0){\rule[-0.200pt]{2.409pt}{0.400pt}}
\put(371.0,337.0){\rule[-0.200pt]{2.409pt}{0.400pt}}
\put(1429.0,337.0){\rule[-0.200pt]{2.409pt}{0.400pt}}
\put(371.0,337.0){\rule[-0.200pt]{2.409pt}{0.400pt}}
\put(1429.0,337.0){\rule[-0.200pt]{2.409pt}{0.400pt}}
\put(371.0,337.0){\rule[-0.200pt]{2.409pt}{0.400pt}}
\put(1429.0,337.0){\rule[-0.200pt]{2.409pt}{0.400pt}}
\put(371.0,337.0){\rule[-0.200pt]{2.409pt}{0.400pt}}
\put(1429.0,337.0){\rule[-0.200pt]{2.409pt}{0.400pt}}
\put(371.0,337.0){\rule[-0.200pt]{2.409pt}{0.400pt}}
\put(1429.0,337.0){\rule[-0.200pt]{2.409pt}{0.400pt}}
\put(371.0,337.0){\rule[-0.200pt]{2.409pt}{0.400pt}}
\put(1429.0,337.0){\rule[-0.200pt]{2.409pt}{0.400pt}}
\put(371.0,337.0){\rule[-0.200pt]{2.409pt}{0.400pt}}
\put(1429.0,337.0){\rule[-0.200pt]{2.409pt}{0.400pt}}
\put(371.0,337.0){\rule[-0.200pt]{2.409pt}{0.400pt}}
\put(1429.0,337.0){\rule[-0.200pt]{2.409pt}{0.400pt}}
\put(371.0,338.0){\rule[-0.200pt]{2.409pt}{0.400pt}}
\put(1429.0,338.0){\rule[-0.200pt]{2.409pt}{0.400pt}}
\put(371.0,338.0){\rule[-0.200pt]{2.409pt}{0.400pt}}
\put(1429.0,338.0){\rule[-0.200pt]{2.409pt}{0.400pt}}
\put(371.0,338.0){\rule[-0.200pt]{2.409pt}{0.400pt}}
\put(1429.0,338.0){\rule[-0.200pt]{2.409pt}{0.400pt}}
\put(371.0,338.0){\rule[-0.200pt]{2.409pt}{0.400pt}}
\put(1429.0,338.0){\rule[-0.200pt]{2.409pt}{0.400pt}}
\put(371.0,338.0){\rule[-0.200pt]{2.409pt}{0.400pt}}
\put(1429.0,338.0){\rule[-0.200pt]{2.409pt}{0.400pt}}
\put(371.0,338.0){\rule[-0.200pt]{2.409pt}{0.400pt}}
\put(1429.0,338.0){\rule[-0.200pt]{2.409pt}{0.400pt}}
\put(371.0,338.0){\rule[-0.200pt]{2.409pt}{0.400pt}}
\put(1429.0,338.0){\rule[-0.200pt]{2.409pt}{0.400pt}}
\put(371.0,338.0){\rule[-0.200pt]{2.409pt}{0.400pt}}
\put(1429.0,338.0){\rule[-0.200pt]{2.409pt}{0.400pt}}
\put(371.0,338.0){\rule[-0.200pt]{2.409pt}{0.400pt}}
\put(1429.0,338.0){\rule[-0.200pt]{2.409pt}{0.400pt}}
\put(371.0,338.0){\rule[-0.200pt]{2.409pt}{0.400pt}}
\put(1429.0,338.0){\rule[-0.200pt]{2.409pt}{0.400pt}}
\put(371.0,338.0){\rule[-0.200pt]{2.409pt}{0.400pt}}
\put(1429.0,338.0){\rule[-0.200pt]{2.409pt}{0.400pt}}
\put(371.0,338.0){\rule[-0.200pt]{2.409pt}{0.400pt}}
\put(1429.0,338.0){\rule[-0.200pt]{2.409pt}{0.400pt}}
\put(371.0,338.0){\rule[-0.200pt]{2.409pt}{0.400pt}}
\put(1429.0,338.0){\rule[-0.200pt]{2.409pt}{0.400pt}}
\put(371.0,338.0){\rule[-0.200pt]{2.409pt}{0.400pt}}
\put(1429.0,338.0){\rule[-0.200pt]{2.409pt}{0.400pt}}
\put(371.0,338.0){\rule[-0.200pt]{2.409pt}{0.400pt}}
\put(1429.0,338.0){\rule[-0.200pt]{2.409pt}{0.400pt}}
\put(371.0,338.0){\rule[-0.200pt]{2.409pt}{0.400pt}}
\put(1429.0,338.0){\rule[-0.200pt]{2.409pt}{0.400pt}}
\put(371.0,339.0){\rule[-0.200pt]{2.409pt}{0.400pt}}
\put(1429.0,339.0){\rule[-0.200pt]{2.409pt}{0.400pt}}
\put(371.0,339.0){\rule[-0.200pt]{2.409pt}{0.400pt}}
\put(1429.0,339.0){\rule[-0.200pt]{2.409pt}{0.400pt}}
\put(371.0,339.0){\rule[-0.200pt]{2.409pt}{0.400pt}}
\put(1429.0,339.0){\rule[-0.200pt]{2.409pt}{0.400pt}}
\put(371.0,339.0){\rule[-0.200pt]{2.409pt}{0.400pt}}
\put(1429.0,339.0){\rule[-0.200pt]{2.409pt}{0.400pt}}
\put(371.0,339.0){\rule[-0.200pt]{2.409pt}{0.400pt}}
\put(1429.0,339.0){\rule[-0.200pt]{2.409pt}{0.400pt}}
\put(371.0,339.0){\rule[-0.200pt]{2.409pt}{0.400pt}}
\put(1429.0,339.0){\rule[-0.200pt]{2.409pt}{0.400pt}}
\put(371.0,339.0){\rule[-0.200pt]{2.409pt}{0.400pt}}
\put(1429.0,339.0){\rule[-0.200pt]{2.409pt}{0.400pt}}
\put(371.0,339.0){\rule[-0.200pt]{2.409pt}{0.400pt}}
\put(1429.0,339.0){\rule[-0.200pt]{2.409pt}{0.400pt}}
\put(371.0,339.0){\rule[-0.200pt]{2.409pt}{0.400pt}}
\put(1429.0,339.0){\rule[-0.200pt]{2.409pt}{0.400pt}}
\put(371.0,339.0){\rule[-0.200pt]{2.409pt}{0.400pt}}
\put(1429.0,339.0){\rule[-0.200pt]{2.409pt}{0.400pt}}
\put(371.0,339.0){\rule[-0.200pt]{2.409pt}{0.400pt}}
\put(1429.0,339.0){\rule[-0.200pt]{2.409pt}{0.400pt}}
\put(371.0,339.0){\rule[-0.200pt]{2.409pt}{0.400pt}}
\put(1429.0,339.0){\rule[-0.200pt]{2.409pt}{0.400pt}}
\put(371.0,339.0){\rule[-0.200pt]{2.409pt}{0.400pt}}
\put(1429.0,339.0){\rule[-0.200pt]{2.409pt}{0.400pt}}
\put(371.0,339.0){\rule[-0.200pt]{2.409pt}{0.400pt}}
\put(1429.0,339.0){\rule[-0.200pt]{2.409pt}{0.400pt}}
\put(371.0,339.0){\rule[-0.200pt]{2.409pt}{0.400pt}}
\put(1429.0,339.0){\rule[-0.200pt]{2.409pt}{0.400pt}}
\put(371.0,339.0){\rule[-0.200pt]{2.409pt}{0.400pt}}
\put(1429.0,339.0){\rule[-0.200pt]{2.409pt}{0.400pt}}
\put(371.0,340.0){\rule[-0.200pt]{2.409pt}{0.400pt}}
\put(1429.0,340.0){\rule[-0.200pt]{2.409pt}{0.400pt}}
\put(371.0,340.0){\rule[-0.200pt]{2.409pt}{0.400pt}}
\put(1429.0,340.0){\rule[-0.200pt]{2.409pt}{0.400pt}}
\put(371.0,340.0){\rule[-0.200pt]{2.409pt}{0.400pt}}
\put(1429.0,340.0){\rule[-0.200pt]{2.409pt}{0.400pt}}
\put(371.0,340.0){\rule[-0.200pt]{2.409pt}{0.400pt}}
\put(1429.0,340.0){\rule[-0.200pt]{2.409pt}{0.400pt}}
\put(371.0,340.0){\rule[-0.200pt]{2.409pt}{0.400pt}}
\put(1429.0,340.0){\rule[-0.200pt]{2.409pt}{0.400pt}}
\put(371.0,340.0){\rule[-0.200pt]{2.409pt}{0.400pt}}
\put(1429.0,340.0){\rule[-0.200pt]{2.409pt}{0.400pt}}
\put(371.0,340.0){\rule[-0.200pt]{2.409pt}{0.400pt}}
\put(1429.0,340.0){\rule[-0.200pt]{2.409pt}{0.400pt}}
\put(371.0,340.0){\rule[-0.200pt]{2.409pt}{0.400pt}}
\put(1429.0,340.0){\rule[-0.200pt]{2.409pt}{0.400pt}}
\put(371.0,340.0){\rule[-0.200pt]{2.409pt}{0.400pt}}
\put(1429.0,340.0){\rule[-0.200pt]{2.409pt}{0.400pt}}
\put(371.0,340.0){\rule[-0.200pt]{2.409pt}{0.400pt}}
\put(1429.0,340.0){\rule[-0.200pt]{2.409pt}{0.400pt}}
\put(371.0,340.0){\rule[-0.200pt]{2.409pt}{0.400pt}}
\put(1429.0,340.0){\rule[-0.200pt]{2.409pt}{0.400pt}}
\put(371.0,340.0){\rule[-0.200pt]{2.409pt}{0.400pt}}
\put(1429.0,340.0){\rule[-0.200pt]{2.409pt}{0.400pt}}
\put(371.0,340.0){\rule[-0.200pt]{2.409pt}{0.400pt}}
\put(1429.0,340.0){\rule[-0.200pt]{2.409pt}{0.400pt}}
\put(371.0,340.0){\rule[-0.200pt]{2.409pt}{0.400pt}}
\put(1429.0,340.0){\rule[-0.200pt]{2.409pt}{0.400pt}}
\put(371.0,340.0){\rule[-0.200pt]{2.409pt}{0.400pt}}
\put(1429.0,340.0){\rule[-0.200pt]{2.409pt}{0.400pt}}
\put(371.0,340.0){\rule[-0.200pt]{2.409pt}{0.400pt}}
\put(1429.0,340.0){\rule[-0.200pt]{2.409pt}{0.400pt}}
\put(371.0,341.0){\rule[-0.200pt]{2.409pt}{0.400pt}}
\put(1429.0,341.0){\rule[-0.200pt]{2.409pt}{0.400pt}}
\put(371.0,341.0){\rule[-0.200pt]{2.409pt}{0.400pt}}
\put(1429.0,341.0){\rule[-0.200pt]{2.409pt}{0.400pt}}
\put(371.0,341.0){\rule[-0.200pt]{2.409pt}{0.400pt}}
\put(1429.0,341.0){\rule[-0.200pt]{2.409pt}{0.400pt}}
\put(371.0,341.0){\rule[-0.200pt]{2.409pt}{0.400pt}}
\put(1429.0,341.0){\rule[-0.200pt]{2.409pt}{0.400pt}}
\put(371.0,341.0){\rule[-0.200pt]{2.409pt}{0.400pt}}
\put(1429.0,341.0){\rule[-0.200pt]{2.409pt}{0.400pt}}
\put(371.0,341.0){\rule[-0.200pt]{2.409pt}{0.400pt}}
\put(1429.0,341.0){\rule[-0.200pt]{2.409pt}{0.400pt}}
\put(371.0,341.0){\rule[-0.200pt]{2.409pt}{0.400pt}}
\put(1429.0,341.0){\rule[-0.200pt]{2.409pt}{0.400pt}}
\put(371.0,341.0){\rule[-0.200pt]{2.409pt}{0.400pt}}
\put(1429.0,341.0){\rule[-0.200pt]{2.409pt}{0.400pt}}
\put(371.0,341.0){\rule[-0.200pt]{2.409pt}{0.400pt}}
\put(1429.0,341.0){\rule[-0.200pt]{2.409pt}{0.400pt}}
\put(371.0,341.0){\rule[-0.200pt]{4.818pt}{0.400pt}}
\put(351,341){\makebox(0,0)[r]{ 1000}}
\put(1419.0,341.0){\rule[-0.200pt]{4.818pt}{0.400pt}}
\put(371.0,354.0){\rule[-0.200pt]{2.409pt}{0.400pt}}
\put(1429.0,354.0){\rule[-0.200pt]{2.409pt}{0.400pt}}
\put(371.0,371.0){\rule[-0.200pt]{2.409pt}{0.400pt}}
\put(1429.0,371.0){\rule[-0.200pt]{2.409pt}{0.400pt}}
\put(371.0,380.0){\rule[-0.200pt]{2.409pt}{0.400pt}}
\put(1429.0,380.0){\rule[-0.200pt]{2.409pt}{0.400pt}}
\put(371.0,386.0){\rule[-0.200pt]{2.409pt}{0.400pt}}
\put(1429.0,386.0){\rule[-0.200pt]{2.409pt}{0.400pt}}
\put(371.0,390.0){\rule[-0.200pt]{2.409pt}{0.400pt}}
\put(1429.0,390.0){\rule[-0.200pt]{2.409pt}{0.400pt}}
\put(371.0,394.0){\rule[-0.200pt]{2.409pt}{0.400pt}}
\put(1429.0,394.0){\rule[-0.200pt]{2.409pt}{0.400pt}}
\put(371.0,397.0){\rule[-0.200pt]{2.409pt}{0.400pt}}
\put(1429.0,397.0){\rule[-0.200pt]{2.409pt}{0.400pt}}
\put(371.0,400.0){\rule[-0.200pt]{2.409pt}{0.400pt}}
\put(1429.0,400.0){\rule[-0.200pt]{2.409pt}{0.400pt}}
\put(371.0,402.0){\rule[-0.200pt]{2.409pt}{0.400pt}}
\put(1429.0,402.0){\rule[-0.200pt]{2.409pt}{0.400pt}}
\put(371.0,404.0){\rule[-0.200pt]{2.409pt}{0.400pt}}
\put(1429.0,404.0){\rule[-0.200pt]{2.409pt}{0.400pt}}
\put(371.0,406.0){\rule[-0.200pt]{2.409pt}{0.400pt}}
\put(1429.0,406.0){\rule[-0.200pt]{2.409pt}{0.400pt}}
\put(371.0,408.0){\rule[-0.200pt]{2.409pt}{0.400pt}}
\put(1429.0,408.0){\rule[-0.200pt]{2.409pt}{0.400pt}}
\put(371.0,409.0){\rule[-0.200pt]{2.409pt}{0.400pt}}
\put(1429.0,409.0){\rule[-0.200pt]{2.409pt}{0.400pt}}
\put(371.0,411.0){\rule[-0.200pt]{2.409pt}{0.400pt}}
\put(1429.0,411.0){\rule[-0.200pt]{2.409pt}{0.400pt}}
\put(371.0,412.0){\rule[-0.200pt]{2.409pt}{0.400pt}}
\put(1429.0,412.0){\rule[-0.200pt]{2.409pt}{0.400pt}}
\put(371.0,413.0){\rule[-0.200pt]{2.409pt}{0.400pt}}
\put(1429.0,413.0){\rule[-0.200pt]{2.409pt}{0.400pt}}
\put(371.0,414.0){\rule[-0.200pt]{2.409pt}{0.400pt}}
\put(1429.0,414.0){\rule[-0.200pt]{2.409pt}{0.400pt}}
\put(371.0,415.0){\rule[-0.200pt]{2.409pt}{0.400pt}}
\put(1429.0,415.0){\rule[-0.200pt]{2.409pt}{0.400pt}}
\put(371.0,416.0){\rule[-0.200pt]{2.409pt}{0.400pt}}
\put(1429.0,416.0){\rule[-0.200pt]{2.409pt}{0.400pt}}
\put(371.0,417.0){\rule[-0.200pt]{2.409pt}{0.400pt}}
\put(1429.0,417.0){\rule[-0.200pt]{2.409pt}{0.400pt}}
\put(371.0,418.0){\rule[-0.200pt]{2.409pt}{0.400pt}}
\put(1429.0,418.0){\rule[-0.200pt]{2.409pt}{0.400pt}}
\put(371.0,419.0){\rule[-0.200pt]{2.409pt}{0.400pt}}
\put(1429.0,419.0){\rule[-0.200pt]{2.409pt}{0.400pt}}
\put(371.0,420.0){\rule[-0.200pt]{2.409pt}{0.400pt}}
\put(1429.0,420.0){\rule[-0.200pt]{2.409pt}{0.400pt}}
\put(371.0,421.0){\rule[-0.200pt]{2.409pt}{0.400pt}}
\put(1429.0,421.0){\rule[-0.200pt]{2.409pt}{0.400pt}}
\put(371.0,422.0){\rule[-0.200pt]{2.409pt}{0.400pt}}
\put(1429.0,422.0){\rule[-0.200pt]{2.409pt}{0.400pt}}
\put(371.0,422.0){\rule[-0.200pt]{2.409pt}{0.400pt}}
\put(1429.0,422.0){\rule[-0.200pt]{2.409pt}{0.400pt}}
\put(371.0,423.0){\rule[-0.200pt]{2.409pt}{0.400pt}}
\put(1429.0,423.0){\rule[-0.200pt]{2.409pt}{0.400pt}}
\put(371.0,424.0){\rule[-0.200pt]{2.409pt}{0.400pt}}
\put(1429.0,424.0){\rule[-0.200pt]{2.409pt}{0.400pt}}
\put(371.0,425.0){\rule[-0.200pt]{2.409pt}{0.400pt}}
\put(1429.0,425.0){\rule[-0.200pt]{2.409pt}{0.400pt}}
\put(371.0,425.0){\rule[-0.200pt]{2.409pt}{0.400pt}}
\put(1429.0,425.0){\rule[-0.200pt]{2.409pt}{0.400pt}}
\put(371.0,426.0){\rule[-0.200pt]{2.409pt}{0.400pt}}
\put(1429.0,426.0){\rule[-0.200pt]{2.409pt}{0.400pt}}
\put(371.0,426.0){\rule[-0.200pt]{2.409pt}{0.400pt}}
\put(1429.0,426.0){\rule[-0.200pt]{2.409pt}{0.400pt}}
\put(371.0,427.0){\rule[-0.200pt]{2.409pt}{0.400pt}}
\put(1429.0,427.0){\rule[-0.200pt]{2.409pt}{0.400pt}}
\put(371.0,428.0){\rule[-0.200pt]{2.409pt}{0.400pt}}
\put(1429.0,428.0){\rule[-0.200pt]{2.409pt}{0.400pt}}
\put(371.0,428.0){\rule[-0.200pt]{2.409pt}{0.400pt}}
\put(1429.0,428.0){\rule[-0.200pt]{2.409pt}{0.400pt}}
\put(371.0,429.0){\rule[-0.200pt]{2.409pt}{0.400pt}}
\put(1429.0,429.0){\rule[-0.200pt]{2.409pt}{0.400pt}}
\put(371.0,429.0){\rule[-0.200pt]{2.409pt}{0.400pt}}
\put(1429.0,429.0){\rule[-0.200pt]{2.409pt}{0.400pt}}
\put(371.0,430.0){\rule[-0.200pt]{2.409pt}{0.400pt}}
\put(1429.0,430.0){\rule[-0.200pt]{2.409pt}{0.400pt}}
\put(371.0,430.0){\rule[-0.200pt]{2.409pt}{0.400pt}}
\put(1429.0,430.0){\rule[-0.200pt]{2.409pt}{0.400pt}}
\put(371.0,431.0){\rule[-0.200pt]{2.409pt}{0.400pt}}
\put(1429.0,431.0){\rule[-0.200pt]{2.409pt}{0.400pt}}
\put(371.0,431.0){\rule[-0.200pt]{2.409pt}{0.400pt}}
\put(1429.0,431.0){\rule[-0.200pt]{2.409pt}{0.400pt}}
\put(371.0,432.0){\rule[-0.200pt]{2.409pt}{0.400pt}}
\put(1429.0,432.0){\rule[-0.200pt]{2.409pt}{0.400pt}}
\put(371.0,432.0){\rule[-0.200pt]{2.409pt}{0.400pt}}
\put(1429.0,432.0){\rule[-0.200pt]{2.409pt}{0.400pt}}
\put(371.0,432.0){\rule[-0.200pt]{2.409pt}{0.400pt}}
\put(1429.0,432.0){\rule[-0.200pt]{2.409pt}{0.400pt}}
\put(371.0,433.0){\rule[-0.200pt]{2.409pt}{0.400pt}}
\put(1429.0,433.0){\rule[-0.200pt]{2.409pt}{0.400pt}}
\put(371.0,433.0){\rule[-0.200pt]{2.409pt}{0.400pt}}
\put(1429.0,433.0){\rule[-0.200pt]{2.409pt}{0.400pt}}
\put(371.0,434.0){\rule[-0.200pt]{2.409pt}{0.400pt}}
\put(1429.0,434.0){\rule[-0.200pt]{2.409pt}{0.400pt}}
\put(371.0,434.0){\rule[-0.200pt]{2.409pt}{0.400pt}}
\put(1429.0,434.0){\rule[-0.200pt]{2.409pt}{0.400pt}}
\put(371.0,434.0){\rule[-0.200pt]{2.409pt}{0.400pt}}
\put(1429.0,434.0){\rule[-0.200pt]{2.409pt}{0.400pt}}
\put(371.0,435.0){\rule[-0.200pt]{2.409pt}{0.400pt}}
\put(1429.0,435.0){\rule[-0.200pt]{2.409pt}{0.400pt}}
\put(371.0,435.0){\rule[-0.200pt]{2.409pt}{0.400pt}}
\put(1429.0,435.0){\rule[-0.200pt]{2.409pt}{0.400pt}}
\put(371.0,436.0){\rule[-0.200pt]{2.409pt}{0.400pt}}
\put(1429.0,436.0){\rule[-0.200pt]{2.409pt}{0.400pt}}
\put(371.0,436.0){\rule[-0.200pt]{2.409pt}{0.400pt}}
\put(1429.0,436.0){\rule[-0.200pt]{2.409pt}{0.400pt}}
\put(371.0,436.0){\rule[-0.200pt]{2.409pt}{0.400pt}}
\put(1429.0,436.0){\rule[-0.200pt]{2.409pt}{0.400pt}}
\put(371.0,437.0){\rule[-0.200pt]{2.409pt}{0.400pt}}
\put(1429.0,437.0){\rule[-0.200pt]{2.409pt}{0.400pt}}
\put(371.0,437.0){\rule[-0.200pt]{2.409pt}{0.400pt}}
\put(1429.0,437.0){\rule[-0.200pt]{2.409pt}{0.400pt}}
\put(371.0,437.0){\rule[-0.200pt]{2.409pt}{0.400pt}}
\put(1429.0,437.0){\rule[-0.200pt]{2.409pt}{0.400pt}}
\put(371.0,438.0){\rule[-0.200pt]{2.409pt}{0.400pt}}
\put(1429.0,438.0){\rule[-0.200pt]{2.409pt}{0.400pt}}
\put(371.0,438.0){\rule[-0.200pt]{2.409pt}{0.400pt}}
\put(1429.0,438.0){\rule[-0.200pt]{2.409pt}{0.400pt}}
\put(371.0,438.0){\rule[-0.200pt]{2.409pt}{0.400pt}}
\put(1429.0,438.0){\rule[-0.200pt]{2.409pt}{0.400pt}}
\put(371.0,439.0){\rule[-0.200pt]{2.409pt}{0.400pt}}
\put(1429.0,439.0){\rule[-0.200pt]{2.409pt}{0.400pt}}
\put(371.0,439.0){\rule[-0.200pt]{2.409pt}{0.400pt}}
\put(1429.0,439.0){\rule[-0.200pt]{2.409pt}{0.400pt}}
\put(371.0,439.0){\rule[-0.200pt]{2.409pt}{0.400pt}}
\put(1429.0,439.0){\rule[-0.200pt]{2.409pt}{0.400pt}}
\put(371.0,439.0){\rule[-0.200pt]{2.409pt}{0.400pt}}
\put(1429.0,439.0){\rule[-0.200pt]{2.409pt}{0.400pt}}
\put(371.0,440.0){\rule[-0.200pt]{2.409pt}{0.400pt}}
\put(1429.0,440.0){\rule[-0.200pt]{2.409pt}{0.400pt}}
\put(371.0,440.0){\rule[-0.200pt]{2.409pt}{0.400pt}}
\put(1429.0,440.0){\rule[-0.200pt]{2.409pt}{0.400pt}}
\put(371.0,440.0){\rule[-0.200pt]{2.409pt}{0.400pt}}
\put(1429.0,440.0){\rule[-0.200pt]{2.409pt}{0.400pt}}
\put(371.0,441.0){\rule[-0.200pt]{2.409pt}{0.400pt}}
\put(1429.0,441.0){\rule[-0.200pt]{2.409pt}{0.400pt}}
\put(371.0,441.0){\rule[-0.200pt]{2.409pt}{0.400pt}}
\put(1429.0,441.0){\rule[-0.200pt]{2.409pt}{0.400pt}}
\put(371.0,441.0){\rule[-0.200pt]{2.409pt}{0.400pt}}
\put(1429.0,441.0){\rule[-0.200pt]{2.409pt}{0.400pt}}
\put(371.0,441.0){\rule[-0.200pt]{2.409pt}{0.400pt}}
\put(1429.0,441.0){\rule[-0.200pt]{2.409pt}{0.400pt}}
\put(371.0,442.0){\rule[-0.200pt]{2.409pt}{0.400pt}}
\put(1429.0,442.0){\rule[-0.200pt]{2.409pt}{0.400pt}}
\put(371.0,442.0){\rule[-0.200pt]{2.409pt}{0.400pt}}
\put(1429.0,442.0){\rule[-0.200pt]{2.409pt}{0.400pt}}
\put(371.0,442.0){\rule[-0.200pt]{2.409pt}{0.400pt}}
\put(1429.0,442.0){\rule[-0.200pt]{2.409pt}{0.400pt}}
\put(371.0,442.0){\rule[-0.200pt]{2.409pt}{0.400pt}}
\put(1429.0,442.0){\rule[-0.200pt]{2.409pt}{0.400pt}}
\put(371.0,443.0){\rule[-0.200pt]{2.409pt}{0.400pt}}
\put(1429.0,443.0){\rule[-0.200pt]{2.409pt}{0.400pt}}
\put(371.0,443.0){\rule[-0.200pt]{2.409pt}{0.400pt}}
\put(1429.0,443.0){\rule[-0.200pt]{2.409pt}{0.400pt}}
\put(371.0,443.0){\rule[-0.200pt]{2.409pt}{0.400pt}}
\put(1429.0,443.0){\rule[-0.200pt]{2.409pt}{0.400pt}}
\put(371.0,443.0){\rule[-0.200pt]{2.409pt}{0.400pt}}
\put(1429.0,443.0){\rule[-0.200pt]{2.409pt}{0.400pt}}
\put(371.0,444.0){\rule[-0.200pt]{2.409pt}{0.400pt}}
\put(1429.0,444.0){\rule[-0.200pt]{2.409pt}{0.400pt}}
\put(371.0,444.0){\rule[-0.200pt]{2.409pt}{0.400pt}}
\put(1429.0,444.0){\rule[-0.200pt]{2.409pt}{0.400pt}}
\put(371.0,444.0){\rule[-0.200pt]{2.409pt}{0.400pt}}
\put(1429.0,444.0){\rule[-0.200pt]{2.409pt}{0.400pt}}
\put(371.0,444.0){\rule[-0.200pt]{2.409pt}{0.400pt}}
\put(1429.0,444.0){\rule[-0.200pt]{2.409pt}{0.400pt}}
\put(371.0,445.0){\rule[-0.200pt]{2.409pt}{0.400pt}}
\put(1429.0,445.0){\rule[-0.200pt]{2.409pt}{0.400pt}}
\put(371.0,445.0){\rule[-0.200pt]{2.409pt}{0.400pt}}
\put(1429.0,445.0){\rule[-0.200pt]{2.409pt}{0.400pt}}
\put(371.0,445.0){\rule[-0.200pt]{2.409pt}{0.400pt}}
\put(1429.0,445.0){\rule[-0.200pt]{2.409pt}{0.400pt}}
\put(371.0,445.0){\rule[-0.200pt]{2.409pt}{0.400pt}}
\put(1429.0,445.0){\rule[-0.200pt]{2.409pt}{0.400pt}}
\put(371.0,445.0){\rule[-0.200pt]{2.409pt}{0.400pt}}
\put(1429.0,445.0){\rule[-0.200pt]{2.409pt}{0.400pt}}
\put(371.0,446.0){\rule[-0.200pt]{2.409pt}{0.400pt}}
\put(1429.0,446.0){\rule[-0.200pt]{2.409pt}{0.400pt}}
\put(371.0,446.0){\rule[-0.200pt]{2.409pt}{0.400pt}}
\put(1429.0,446.0){\rule[-0.200pt]{2.409pt}{0.400pt}}
\put(371.0,446.0){\rule[-0.200pt]{2.409pt}{0.400pt}}
\put(1429.0,446.0){\rule[-0.200pt]{2.409pt}{0.400pt}}
\put(371.0,446.0){\rule[-0.200pt]{2.409pt}{0.400pt}}
\put(1429.0,446.0){\rule[-0.200pt]{2.409pt}{0.400pt}}
\put(371.0,447.0){\rule[-0.200pt]{2.409pt}{0.400pt}}
\put(1429.0,447.0){\rule[-0.200pt]{2.409pt}{0.400pt}}
\put(371.0,447.0){\rule[-0.200pt]{2.409pt}{0.400pt}}
\put(1429.0,447.0){\rule[-0.200pt]{2.409pt}{0.400pt}}
\put(371.0,447.0){\rule[-0.200pt]{2.409pt}{0.400pt}}
\put(1429.0,447.0){\rule[-0.200pt]{2.409pt}{0.400pt}}
\put(371.0,447.0){\rule[-0.200pt]{2.409pt}{0.400pt}}
\put(1429.0,447.0){\rule[-0.200pt]{2.409pt}{0.400pt}}
\put(371.0,447.0){\rule[-0.200pt]{2.409pt}{0.400pt}}
\put(1429.0,447.0){\rule[-0.200pt]{2.409pt}{0.400pt}}
\put(371.0,447.0){\rule[-0.200pt]{2.409pt}{0.400pt}}
\put(1429.0,447.0){\rule[-0.200pt]{2.409pt}{0.400pt}}
\put(371.0,448.0){\rule[-0.200pt]{2.409pt}{0.400pt}}
\put(1429.0,448.0){\rule[-0.200pt]{2.409pt}{0.400pt}}
\put(371.0,448.0){\rule[-0.200pt]{2.409pt}{0.400pt}}
\put(1429.0,448.0){\rule[-0.200pt]{2.409pt}{0.400pt}}
\put(371.0,448.0){\rule[-0.200pt]{2.409pt}{0.400pt}}
\put(1429.0,448.0){\rule[-0.200pt]{2.409pt}{0.400pt}}
\put(371.0,448.0){\rule[-0.200pt]{2.409pt}{0.400pt}}
\put(1429.0,448.0){\rule[-0.200pt]{2.409pt}{0.400pt}}
\put(371.0,448.0){\rule[-0.200pt]{2.409pt}{0.400pt}}
\put(1429.0,448.0){\rule[-0.200pt]{2.409pt}{0.400pt}}
\put(371.0,449.0){\rule[-0.200pt]{2.409pt}{0.400pt}}
\put(1429.0,449.0){\rule[-0.200pt]{2.409pt}{0.400pt}}
\put(371.0,449.0){\rule[-0.200pt]{2.409pt}{0.400pt}}
\put(1429.0,449.0){\rule[-0.200pt]{2.409pt}{0.400pt}}
\put(371.0,449.0){\rule[-0.200pt]{2.409pt}{0.400pt}}
\put(1429.0,449.0){\rule[-0.200pt]{2.409pt}{0.400pt}}
\put(371.0,449.0){\rule[-0.200pt]{2.409pt}{0.400pt}}
\put(1429.0,449.0){\rule[-0.200pt]{2.409pt}{0.400pt}}
\put(371.0,449.0){\rule[-0.200pt]{2.409pt}{0.400pt}}
\put(1429.0,449.0){\rule[-0.200pt]{2.409pt}{0.400pt}}
\put(371.0,449.0){\rule[-0.200pt]{2.409pt}{0.400pt}}
\put(1429.0,449.0){\rule[-0.200pt]{2.409pt}{0.400pt}}
\put(371.0,450.0){\rule[-0.200pt]{2.409pt}{0.400pt}}
\put(1429.0,450.0){\rule[-0.200pt]{2.409pt}{0.400pt}}
\put(371.0,450.0){\rule[-0.200pt]{2.409pt}{0.400pt}}
\put(1429.0,450.0){\rule[-0.200pt]{2.409pt}{0.400pt}}
\put(371.0,450.0){\rule[-0.200pt]{2.409pt}{0.400pt}}
\put(1429.0,450.0){\rule[-0.200pt]{2.409pt}{0.400pt}}
\put(371.0,450.0){\rule[-0.200pt]{2.409pt}{0.400pt}}
\put(1429.0,450.0){\rule[-0.200pt]{2.409pt}{0.400pt}}
\put(371.0,450.0){\rule[-0.200pt]{2.409pt}{0.400pt}}
\put(1429.0,450.0){\rule[-0.200pt]{2.409pt}{0.400pt}}
\put(371.0,450.0){\rule[-0.200pt]{2.409pt}{0.400pt}}
\put(1429.0,450.0){\rule[-0.200pt]{2.409pt}{0.400pt}}
\put(371.0,451.0){\rule[-0.200pt]{2.409pt}{0.400pt}}
\put(1429.0,451.0){\rule[-0.200pt]{2.409pt}{0.400pt}}
\put(371.0,451.0){\rule[-0.200pt]{2.409pt}{0.400pt}}
\put(1429.0,451.0){\rule[-0.200pt]{2.409pt}{0.400pt}}
\put(371.0,451.0){\rule[-0.200pt]{2.409pt}{0.400pt}}
\put(1429.0,451.0){\rule[-0.200pt]{2.409pt}{0.400pt}}
\put(371.0,451.0){\rule[-0.200pt]{2.409pt}{0.400pt}}
\put(1429.0,451.0){\rule[-0.200pt]{2.409pt}{0.400pt}}
\put(371.0,451.0){\rule[-0.200pt]{2.409pt}{0.400pt}}
\put(1429.0,451.0){\rule[-0.200pt]{2.409pt}{0.400pt}}
\put(371.0,451.0){\rule[-0.200pt]{2.409pt}{0.400pt}}
\put(1429.0,451.0){\rule[-0.200pt]{2.409pt}{0.400pt}}
\put(371.0,452.0){\rule[-0.200pt]{2.409pt}{0.400pt}}
\put(1429.0,452.0){\rule[-0.200pt]{2.409pt}{0.400pt}}
\put(371.0,452.0){\rule[-0.200pt]{2.409pt}{0.400pt}}
\put(1429.0,452.0){\rule[-0.200pt]{2.409pt}{0.400pt}}
\put(371.0,452.0){\rule[-0.200pt]{2.409pt}{0.400pt}}
\put(1429.0,452.0){\rule[-0.200pt]{2.409pt}{0.400pt}}
\put(371.0,452.0){\rule[-0.200pt]{2.409pt}{0.400pt}}
\put(1429.0,452.0){\rule[-0.200pt]{2.409pt}{0.400pt}}
\put(371.0,452.0){\rule[-0.200pt]{2.409pt}{0.400pt}}
\put(1429.0,452.0){\rule[-0.200pt]{2.409pt}{0.400pt}}
\put(371.0,452.0){\rule[-0.200pt]{2.409pt}{0.400pt}}
\put(1429.0,452.0){\rule[-0.200pt]{2.409pt}{0.400pt}}
\put(371.0,453.0){\rule[-0.200pt]{2.409pt}{0.400pt}}
\put(1429.0,453.0){\rule[-0.200pt]{2.409pt}{0.400pt}}
\put(371.0,453.0){\rule[-0.200pt]{2.409pt}{0.400pt}}
\put(1429.0,453.0){\rule[-0.200pt]{2.409pt}{0.400pt}}
\put(371.0,453.0){\rule[-0.200pt]{2.409pt}{0.400pt}}
\put(1429.0,453.0){\rule[-0.200pt]{2.409pt}{0.400pt}}
\put(371.0,453.0){\rule[-0.200pt]{2.409pt}{0.400pt}}
\put(1429.0,453.0){\rule[-0.200pt]{2.409pt}{0.400pt}}
\put(371.0,453.0){\rule[-0.200pt]{2.409pt}{0.400pt}}
\put(1429.0,453.0){\rule[-0.200pt]{2.409pt}{0.400pt}}
\put(371.0,453.0){\rule[-0.200pt]{2.409pt}{0.400pt}}
\put(1429.0,453.0){\rule[-0.200pt]{2.409pt}{0.400pt}}
\put(371.0,453.0){\rule[-0.200pt]{2.409pt}{0.400pt}}
\put(1429.0,453.0){\rule[-0.200pt]{2.409pt}{0.400pt}}
\put(371.0,454.0){\rule[-0.200pt]{2.409pt}{0.400pt}}
\put(1429.0,454.0){\rule[-0.200pt]{2.409pt}{0.400pt}}
\put(371.0,454.0){\rule[-0.200pt]{2.409pt}{0.400pt}}
\put(1429.0,454.0){\rule[-0.200pt]{2.409pt}{0.400pt}}
\put(371.0,454.0){\rule[-0.200pt]{2.409pt}{0.400pt}}
\put(1429.0,454.0){\rule[-0.200pt]{2.409pt}{0.400pt}}
\put(371.0,454.0){\rule[-0.200pt]{2.409pt}{0.400pt}}
\put(1429.0,454.0){\rule[-0.200pt]{2.409pt}{0.400pt}}
\put(371.0,454.0){\rule[-0.200pt]{2.409pt}{0.400pt}}
\put(1429.0,454.0){\rule[-0.200pt]{2.409pt}{0.400pt}}
\put(371.0,454.0){\rule[-0.200pt]{2.409pt}{0.400pt}}
\put(1429.0,454.0){\rule[-0.200pt]{2.409pt}{0.400pt}}
\put(371.0,454.0){\rule[-0.200pt]{2.409pt}{0.400pt}}
\put(1429.0,454.0){\rule[-0.200pt]{2.409pt}{0.400pt}}
\put(371.0,454.0){\rule[-0.200pt]{2.409pt}{0.400pt}}
\put(1429.0,454.0){\rule[-0.200pt]{2.409pt}{0.400pt}}
\put(371.0,455.0){\rule[-0.200pt]{2.409pt}{0.400pt}}
\put(1429.0,455.0){\rule[-0.200pt]{2.409pt}{0.400pt}}
\put(371.0,455.0){\rule[-0.200pt]{2.409pt}{0.400pt}}
\put(1429.0,455.0){\rule[-0.200pt]{2.409pt}{0.400pt}}
\put(371.0,455.0){\rule[-0.200pt]{2.409pt}{0.400pt}}
\put(1429.0,455.0){\rule[-0.200pt]{2.409pt}{0.400pt}}
\put(371.0,455.0){\rule[-0.200pt]{2.409pt}{0.400pt}}
\put(1429.0,455.0){\rule[-0.200pt]{2.409pt}{0.400pt}}
\put(371.0,455.0){\rule[-0.200pt]{2.409pt}{0.400pt}}
\put(1429.0,455.0){\rule[-0.200pt]{2.409pt}{0.400pt}}
\put(371.0,455.0){\rule[-0.200pt]{2.409pt}{0.400pt}}
\put(1429.0,455.0){\rule[-0.200pt]{2.409pt}{0.400pt}}
\put(371.0,455.0){\rule[-0.200pt]{2.409pt}{0.400pt}}
\put(1429.0,455.0){\rule[-0.200pt]{2.409pt}{0.400pt}}
\put(371.0,455.0){\rule[-0.200pt]{2.409pt}{0.400pt}}
\put(1429.0,455.0){\rule[-0.200pt]{2.409pt}{0.400pt}}
\put(371.0,456.0){\rule[-0.200pt]{2.409pt}{0.400pt}}
\put(1429.0,456.0){\rule[-0.200pt]{2.409pt}{0.400pt}}
\put(371.0,456.0){\rule[-0.200pt]{2.409pt}{0.400pt}}
\put(1429.0,456.0){\rule[-0.200pt]{2.409pt}{0.400pt}}
\put(371.0,456.0){\rule[-0.200pt]{2.409pt}{0.400pt}}
\put(1429.0,456.0){\rule[-0.200pt]{2.409pt}{0.400pt}}
\put(371.0,456.0){\rule[-0.200pt]{2.409pt}{0.400pt}}
\put(1429.0,456.0){\rule[-0.200pt]{2.409pt}{0.400pt}}
\put(371.0,456.0){\rule[-0.200pt]{2.409pt}{0.400pt}}
\put(1429.0,456.0){\rule[-0.200pt]{2.409pt}{0.400pt}}
\put(371.0,456.0){\rule[-0.200pt]{2.409pt}{0.400pt}}
\put(1429.0,456.0){\rule[-0.200pt]{2.409pt}{0.400pt}}
\put(371.0,456.0){\rule[-0.200pt]{2.409pt}{0.400pt}}
\put(1429.0,456.0){\rule[-0.200pt]{2.409pt}{0.400pt}}
\put(371.0,456.0){\rule[-0.200pt]{2.409pt}{0.400pt}}
\put(1429.0,456.0){\rule[-0.200pt]{2.409pt}{0.400pt}}
\put(371.0,457.0){\rule[-0.200pt]{2.409pt}{0.400pt}}
\put(1429.0,457.0){\rule[-0.200pt]{2.409pt}{0.400pt}}
\put(371.0,457.0){\rule[-0.200pt]{2.409pt}{0.400pt}}
\put(1429.0,457.0){\rule[-0.200pt]{2.409pt}{0.400pt}}
\put(371.0,457.0){\rule[-0.200pt]{2.409pt}{0.400pt}}
\put(1429.0,457.0){\rule[-0.200pt]{2.409pt}{0.400pt}}
\put(371.0,457.0){\rule[-0.200pt]{2.409pt}{0.400pt}}
\put(1429.0,457.0){\rule[-0.200pt]{2.409pt}{0.400pt}}
\put(371.0,457.0){\rule[-0.200pt]{2.409pt}{0.400pt}}
\put(1429.0,457.0){\rule[-0.200pt]{2.409pt}{0.400pt}}
\put(371.0,457.0){\rule[-0.200pt]{2.409pt}{0.400pt}}
\put(1429.0,457.0){\rule[-0.200pt]{2.409pt}{0.400pt}}
\put(371.0,457.0){\rule[-0.200pt]{2.409pt}{0.400pt}}
\put(1429.0,457.0){\rule[-0.200pt]{2.409pt}{0.400pt}}
\put(371.0,457.0){\rule[-0.200pt]{2.409pt}{0.400pt}}
\put(1429.0,457.0){\rule[-0.200pt]{2.409pt}{0.400pt}}
\put(371.0,458.0){\rule[-0.200pt]{2.409pt}{0.400pt}}
\put(1429.0,458.0){\rule[-0.200pt]{2.409pt}{0.400pt}}
\put(371.0,458.0){\rule[-0.200pt]{2.409pt}{0.400pt}}
\put(1429.0,458.0){\rule[-0.200pt]{2.409pt}{0.400pt}}
\put(371.0,458.0){\rule[-0.200pt]{2.409pt}{0.400pt}}
\put(1429.0,458.0){\rule[-0.200pt]{2.409pt}{0.400pt}}
\put(371.0,458.0){\rule[-0.200pt]{2.409pt}{0.400pt}}
\put(1429.0,458.0){\rule[-0.200pt]{2.409pt}{0.400pt}}
\put(371.0,458.0){\rule[-0.200pt]{2.409pt}{0.400pt}}
\put(1429.0,458.0){\rule[-0.200pt]{2.409pt}{0.400pt}}
\put(371.0,458.0){\rule[-0.200pt]{2.409pt}{0.400pt}}
\put(1429.0,458.0){\rule[-0.200pt]{2.409pt}{0.400pt}}
\put(371.0,458.0){\rule[-0.200pt]{2.409pt}{0.400pt}}
\put(1429.0,458.0){\rule[-0.200pt]{2.409pt}{0.400pt}}
\put(371.0,458.0){\rule[-0.200pt]{2.409pt}{0.400pt}}
\put(1429.0,458.0){\rule[-0.200pt]{2.409pt}{0.400pt}}
\put(371.0,458.0){\rule[-0.200pt]{2.409pt}{0.400pt}}
\put(1429.0,458.0){\rule[-0.200pt]{2.409pt}{0.400pt}}
\put(371.0,458.0){\rule[-0.200pt]{2.409pt}{0.400pt}}
\put(1429.0,458.0){\rule[-0.200pt]{2.409pt}{0.400pt}}
\put(371.0,459.0){\rule[-0.200pt]{2.409pt}{0.400pt}}
\put(1429.0,459.0){\rule[-0.200pt]{2.409pt}{0.400pt}}
\put(371.0,459.0){\rule[-0.200pt]{2.409pt}{0.400pt}}
\put(1429.0,459.0){\rule[-0.200pt]{2.409pt}{0.400pt}}
\put(371.0,459.0){\rule[-0.200pt]{2.409pt}{0.400pt}}
\put(1429.0,459.0){\rule[-0.200pt]{2.409pt}{0.400pt}}
\put(371.0,459.0){\rule[-0.200pt]{2.409pt}{0.400pt}}
\put(1429.0,459.0){\rule[-0.200pt]{2.409pt}{0.400pt}}
\put(371.0,459.0){\rule[-0.200pt]{2.409pt}{0.400pt}}
\put(1429.0,459.0){\rule[-0.200pt]{2.409pt}{0.400pt}}
\put(371.0,459.0){\rule[-0.200pt]{2.409pt}{0.400pt}}
\put(1429.0,459.0){\rule[-0.200pt]{2.409pt}{0.400pt}}
\put(371.0,459.0){\rule[-0.200pt]{2.409pt}{0.400pt}}
\put(1429.0,459.0){\rule[-0.200pt]{2.409pt}{0.400pt}}
\put(371.0,459.0){\rule[-0.200pt]{2.409pt}{0.400pt}}
\put(1429.0,459.0){\rule[-0.200pt]{2.409pt}{0.400pt}}
\put(371.0,459.0){\rule[-0.200pt]{2.409pt}{0.400pt}}
\put(1429.0,459.0){\rule[-0.200pt]{2.409pt}{0.400pt}}
\put(371.0,460.0){\rule[-0.200pt]{2.409pt}{0.400pt}}
\put(1429.0,460.0){\rule[-0.200pt]{2.409pt}{0.400pt}}
\put(371.0,460.0){\rule[-0.200pt]{2.409pt}{0.400pt}}
\put(1429.0,460.0){\rule[-0.200pt]{2.409pt}{0.400pt}}
\put(371.0,460.0){\rule[-0.200pt]{2.409pt}{0.400pt}}
\put(1429.0,460.0){\rule[-0.200pt]{2.409pt}{0.400pt}}
\put(371.0,460.0){\rule[-0.200pt]{2.409pt}{0.400pt}}
\put(1429.0,460.0){\rule[-0.200pt]{2.409pt}{0.400pt}}
\put(371.0,460.0){\rule[-0.200pt]{2.409pt}{0.400pt}}
\put(1429.0,460.0){\rule[-0.200pt]{2.409pt}{0.400pt}}
\put(371.0,460.0){\rule[-0.200pt]{2.409pt}{0.400pt}}
\put(1429.0,460.0){\rule[-0.200pt]{2.409pt}{0.400pt}}
\put(371.0,460.0){\rule[-0.200pt]{2.409pt}{0.400pt}}
\put(1429.0,460.0){\rule[-0.200pt]{2.409pt}{0.400pt}}
\put(371.0,460.0){\rule[-0.200pt]{2.409pt}{0.400pt}}
\put(1429.0,460.0){\rule[-0.200pt]{2.409pt}{0.400pt}}
\put(371.0,460.0){\rule[-0.200pt]{2.409pt}{0.400pt}}
\put(1429.0,460.0){\rule[-0.200pt]{2.409pt}{0.400pt}}
\put(371.0,460.0){\rule[-0.200pt]{2.409pt}{0.400pt}}
\put(1429.0,460.0){\rule[-0.200pt]{2.409pt}{0.400pt}}
\put(371.0,461.0){\rule[-0.200pt]{2.409pt}{0.400pt}}
\put(1429.0,461.0){\rule[-0.200pt]{2.409pt}{0.400pt}}
\put(371.0,461.0){\rule[-0.200pt]{2.409pt}{0.400pt}}
\put(1429.0,461.0){\rule[-0.200pt]{2.409pt}{0.400pt}}
\put(371.0,461.0){\rule[-0.200pt]{2.409pt}{0.400pt}}
\put(1429.0,461.0){\rule[-0.200pt]{2.409pt}{0.400pt}}
\put(371.0,461.0){\rule[-0.200pt]{2.409pt}{0.400pt}}
\put(1429.0,461.0){\rule[-0.200pt]{2.409pt}{0.400pt}}
\put(371.0,461.0){\rule[-0.200pt]{2.409pt}{0.400pt}}
\put(1429.0,461.0){\rule[-0.200pt]{2.409pt}{0.400pt}}
\put(371.0,461.0){\rule[-0.200pt]{2.409pt}{0.400pt}}
\put(1429.0,461.0){\rule[-0.200pt]{2.409pt}{0.400pt}}
\put(371.0,461.0){\rule[-0.200pt]{2.409pt}{0.400pt}}
\put(1429.0,461.0){\rule[-0.200pt]{2.409pt}{0.400pt}}
\put(371.0,461.0){\rule[-0.200pt]{2.409pt}{0.400pt}}
\put(1429.0,461.0){\rule[-0.200pt]{2.409pt}{0.400pt}}
\put(371.0,461.0){\rule[-0.200pt]{2.409pt}{0.400pt}}
\put(1429.0,461.0){\rule[-0.200pt]{2.409pt}{0.400pt}}
\put(371.0,461.0){\rule[-0.200pt]{2.409pt}{0.400pt}}
\put(1429.0,461.0){\rule[-0.200pt]{2.409pt}{0.400pt}}
\put(371.0,461.0){\rule[-0.200pt]{2.409pt}{0.400pt}}
\put(1429.0,461.0){\rule[-0.200pt]{2.409pt}{0.400pt}}
\put(371.0,462.0){\rule[-0.200pt]{2.409pt}{0.400pt}}
\put(1429.0,462.0){\rule[-0.200pt]{2.409pt}{0.400pt}}
\put(371.0,462.0){\rule[-0.200pt]{2.409pt}{0.400pt}}
\put(1429.0,462.0){\rule[-0.200pt]{2.409pt}{0.400pt}}
\put(371.0,462.0){\rule[-0.200pt]{2.409pt}{0.400pt}}
\put(1429.0,462.0){\rule[-0.200pt]{2.409pt}{0.400pt}}
\put(371.0,462.0){\rule[-0.200pt]{2.409pt}{0.400pt}}
\put(1429.0,462.0){\rule[-0.200pt]{2.409pt}{0.400pt}}
\put(371.0,462.0){\rule[-0.200pt]{2.409pt}{0.400pt}}
\put(1429.0,462.0){\rule[-0.200pt]{2.409pt}{0.400pt}}
\put(371.0,462.0){\rule[-0.200pt]{2.409pt}{0.400pt}}
\put(1429.0,462.0){\rule[-0.200pt]{2.409pt}{0.400pt}}
\put(371.0,462.0){\rule[-0.200pt]{2.409pt}{0.400pt}}
\put(1429.0,462.0){\rule[-0.200pt]{2.409pt}{0.400pt}}
\put(371.0,462.0){\rule[-0.200pt]{2.409pt}{0.400pt}}
\put(1429.0,462.0){\rule[-0.200pt]{2.409pt}{0.400pt}}
\put(371.0,462.0){\rule[-0.200pt]{2.409pt}{0.400pt}}
\put(1429.0,462.0){\rule[-0.200pt]{2.409pt}{0.400pt}}
\put(371.0,462.0){\rule[-0.200pt]{2.409pt}{0.400pt}}
\put(1429.0,462.0){\rule[-0.200pt]{2.409pt}{0.400pt}}
\put(371.0,462.0){\rule[-0.200pt]{2.409pt}{0.400pt}}
\put(1429.0,462.0){\rule[-0.200pt]{2.409pt}{0.400pt}}
\put(371.0,463.0){\rule[-0.200pt]{2.409pt}{0.400pt}}
\put(1429.0,463.0){\rule[-0.200pt]{2.409pt}{0.400pt}}
\put(371.0,463.0){\rule[-0.200pt]{2.409pt}{0.400pt}}
\put(1429.0,463.0){\rule[-0.200pt]{2.409pt}{0.400pt}}
\put(371.0,463.0){\rule[-0.200pt]{2.409pt}{0.400pt}}
\put(1429.0,463.0){\rule[-0.200pt]{2.409pt}{0.400pt}}
\put(371.0,463.0){\rule[-0.200pt]{2.409pt}{0.400pt}}
\put(1429.0,463.0){\rule[-0.200pt]{2.409pt}{0.400pt}}
\put(371.0,463.0){\rule[-0.200pt]{2.409pt}{0.400pt}}
\put(1429.0,463.0){\rule[-0.200pt]{2.409pt}{0.400pt}}
\put(371.0,463.0){\rule[-0.200pt]{2.409pt}{0.400pt}}
\put(1429.0,463.0){\rule[-0.200pt]{2.409pt}{0.400pt}}
\put(371.0,463.0){\rule[-0.200pt]{2.409pt}{0.400pt}}
\put(1429.0,463.0){\rule[-0.200pt]{2.409pt}{0.400pt}}
\put(371.0,463.0){\rule[-0.200pt]{2.409pt}{0.400pt}}
\put(1429.0,463.0){\rule[-0.200pt]{2.409pt}{0.400pt}}
\put(371.0,463.0){\rule[-0.200pt]{2.409pt}{0.400pt}}
\put(1429.0,463.0){\rule[-0.200pt]{2.409pt}{0.400pt}}
\put(371.0,463.0){\rule[-0.200pt]{2.409pt}{0.400pt}}
\put(1429.0,463.0){\rule[-0.200pt]{2.409pt}{0.400pt}}
\put(371.0,463.0){\rule[-0.200pt]{2.409pt}{0.400pt}}
\put(1429.0,463.0){\rule[-0.200pt]{2.409pt}{0.400pt}}
\put(371.0,463.0){\rule[-0.200pt]{2.409pt}{0.400pt}}
\put(1429.0,463.0){\rule[-0.200pt]{2.409pt}{0.400pt}}
\put(371.0,464.0){\rule[-0.200pt]{2.409pt}{0.400pt}}
\put(1429.0,464.0){\rule[-0.200pt]{2.409pt}{0.400pt}}
\put(371.0,464.0){\rule[-0.200pt]{2.409pt}{0.400pt}}
\put(1429.0,464.0){\rule[-0.200pt]{2.409pt}{0.400pt}}
\put(371.0,464.0){\rule[-0.200pt]{2.409pt}{0.400pt}}
\put(1429.0,464.0){\rule[-0.200pt]{2.409pt}{0.400pt}}
\put(371.0,464.0){\rule[-0.200pt]{2.409pt}{0.400pt}}
\put(1429.0,464.0){\rule[-0.200pt]{2.409pt}{0.400pt}}
\put(371.0,464.0){\rule[-0.200pt]{2.409pt}{0.400pt}}
\put(1429.0,464.0){\rule[-0.200pt]{2.409pt}{0.400pt}}
\put(371.0,464.0){\rule[-0.200pt]{2.409pt}{0.400pt}}
\put(1429.0,464.0){\rule[-0.200pt]{2.409pt}{0.400pt}}
\put(371.0,464.0){\rule[-0.200pt]{2.409pt}{0.400pt}}
\put(1429.0,464.0){\rule[-0.200pt]{2.409pt}{0.400pt}}
\put(371.0,464.0){\rule[-0.200pt]{2.409pt}{0.400pt}}
\put(1429.0,464.0){\rule[-0.200pt]{2.409pt}{0.400pt}}
\put(371.0,464.0){\rule[-0.200pt]{2.409pt}{0.400pt}}
\put(1429.0,464.0){\rule[-0.200pt]{2.409pt}{0.400pt}}
\put(371.0,464.0){\rule[-0.200pt]{2.409pt}{0.400pt}}
\put(1429.0,464.0){\rule[-0.200pt]{2.409pt}{0.400pt}}
\put(371.0,464.0){\rule[-0.200pt]{2.409pt}{0.400pt}}
\put(1429.0,464.0){\rule[-0.200pt]{2.409pt}{0.400pt}}
\put(371.0,464.0){\rule[-0.200pt]{2.409pt}{0.400pt}}
\put(1429.0,464.0){\rule[-0.200pt]{2.409pt}{0.400pt}}
\put(371.0,464.0){\rule[-0.200pt]{2.409pt}{0.400pt}}
\put(1429.0,464.0){\rule[-0.200pt]{2.409pt}{0.400pt}}
\put(371.0,465.0){\rule[-0.200pt]{2.409pt}{0.400pt}}
\put(1429.0,465.0){\rule[-0.200pt]{2.409pt}{0.400pt}}
\put(371.0,465.0){\rule[-0.200pt]{2.409pt}{0.400pt}}
\put(1429.0,465.0){\rule[-0.200pt]{2.409pt}{0.400pt}}
\put(371.0,465.0){\rule[-0.200pt]{2.409pt}{0.400pt}}
\put(1429.0,465.0){\rule[-0.200pt]{2.409pt}{0.400pt}}
\put(371.0,465.0){\rule[-0.200pt]{2.409pt}{0.400pt}}
\put(1429.0,465.0){\rule[-0.200pt]{2.409pt}{0.400pt}}
\put(371.0,465.0){\rule[-0.200pt]{2.409pt}{0.400pt}}
\put(1429.0,465.0){\rule[-0.200pt]{2.409pt}{0.400pt}}
\put(371.0,465.0){\rule[-0.200pt]{2.409pt}{0.400pt}}
\put(1429.0,465.0){\rule[-0.200pt]{2.409pt}{0.400pt}}
\put(371.0,465.0){\rule[-0.200pt]{2.409pt}{0.400pt}}
\put(1429.0,465.0){\rule[-0.200pt]{2.409pt}{0.400pt}}
\put(371.0,465.0){\rule[-0.200pt]{2.409pt}{0.400pt}}
\put(1429.0,465.0){\rule[-0.200pt]{2.409pt}{0.400pt}}
\put(371.0,465.0){\rule[-0.200pt]{2.409pt}{0.400pt}}
\put(1429.0,465.0){\rule[-0.200pt]{2.409pt}{0.400pt}}
\put(371.0,465.0){\rule[-0.200pt]{2.409pt}{0.400pt}}
\put(1429.0,465.0){\rule[-0.200pt]{2.409pt}{0.400pt}}
\put(371.0,465.0){\rule[-0.200pt]{2.409pt}{0.400pt}}
\put(1429.0,465.0){\rule[-0.200pt]{2.409pt}{0.400pt}}
\put(371.0,465.0){\rule[-0.200pt]{2.409pt}{0.400pt}}
\put(1429.0,465.0){\rule[-0.200pt]{2.409pt}{0.400pt}}
\put(371.0,465.0){\rule[-0.200pt]{2.409pt}{0.400pt}}
\put(1429.0,465.0){\rule[-0.200pt]{2.409pt}{0.400pt}}
\put(371.0,466.0){\rule[-0.200pt]{2.409pt}{0.400pt}}
\put(1429.0,466.0){\rule[-0.200pt]{2.409pt}{0.400pt}}
\put(371.0,466.0){\rule[-0.200pt]{2.409pt}{0.400pt}}
\put(1429.0,466.0){\rule[-0.200pt]{2.409pt}{0.400pt}}
\put(371.0,466.0){\rule[-0.200pt]{2.409pt}{0.400pt}}
\put(1429.0,466.0){\rule[-0.200pt]{2.409pt}{0.400pt}}
\put(371.0,466.0){\rule[-0.200pt]{2.409pt}{0.400pt}}
\put(1429.0,466.0){\rule[-0.200pt]{2.409pt}{0.400pt}}
\put(371.0,466.0){\rule[-0.200pt]{2.409pt}{0.400pt}}
\put(1429.0,466.0){\rule[-0.200pt]{2.409pt}{0.400pt}}
\put(371.0,466.0){\rule[-0.200pt]{2.409pt}{0.400pt}}
\put(1429.0,466.0){\rule[-0.200pt]{2.409pt}{0.400pt}}
\put(371.0,466.0){\rule[-0.200pt]{2.409pt}{0.400pt}}
\put(1429.0,466.0){\rule[-0.200pt]{2.409pt}{0.400pt}}
\put(371.0,466.0){\rule[-0.200pt]{2.409pt}{0.400pt}}
\put(1429.0,466.0){\rule[-0.200pt]{2.409pt}{0.400pt}}
\put(371.0,466.0){\rule[-0.200pt]{2.409pt}{0.400pt}}
\put(1429.0,466.0){\rule[-0.200pt]{2.409pt}{0.400pt}}
\put(371.0,466.0){\rule[-0.200pt]{2.409pt}{0.400pt}}
\put(1429.0,466.0){\rule[-0.200pt]{2.409pt}{0.400pt}}
\put(371.0,466.0){\rule[-0.200pt]{2.409pt}{0.400pt}}
\put(1429.0,466.0){\rule[-0.200pt]{2.409pt}{0.400pt}}
\put(371.0,466.0){\rule[-0.200pt]{2.409pt}{0.400pt}}
\put(1429.0,466.0){\rule[-0.200pt]{2.409pt}{0.400pt}}
\put(371.0,466.0){\rule[-0.200pt]{2.409pt}{0.400pt}}
\put(1429.0,466.0){\rule[-0.200pt]{2.409pt}{0.400pt}}
\put(371.0,466.0){\rule[-0.200pt]{2.409pt}{0.400pt}}
\put(1429.0,466.0){\rule[-0.200pt]{2.409pt}{0.400pt}}
\put(371.0,467.0){\rule[-0.200pt]{2.409pt}{0.400pt}}
\put(1429.0,467.0){\rule[-0.200pt]{2.409pt}{0.400pt}}
\put(371.0,467.0){\rule[-0.200pt]{2.409pt}{0.400pt}}
\put(1429.0,467.0){\rule[-0.200pt]{2.409pt}{0.400pt}}
\put(371.0,467.0){\rule[-0.200pt]{2.409pt}{0.400pt}}
\put(1429.0,467.0){\rule[-0.200pt]{2.409pt}{0.400pt}}
\put(371.0,467.0){\rule[-0.200pt]{2.409pt}{0.400pt}}
\put(1429.0,467.0){\rule[-0.200pt]{2.409pt}{0.400pt}}
\put(371.0,467.0){\rule[-0.200pt]{2.409pt}{0.400pt}}
\put(1429.0,467.0){\rule[-0.200pt]{2.409pt}{0.400pt}}
\put(371.0,467.0){\rule[-0.200pt]{2.409pt}{0.400pt}}
\put(1429.0,467.0){\rule[-0.200pt]{2.409pt}{0.400pt}}
\put(371.0,467.0){\rule[-0.200pt]{2.409pt}{0.400pt}}
\put(1429.0,467.0){\rule[-0.200pt]{2.409pt}{0.400pt}}
\put(371.0,467.0){\rule[-0.200pt]{2.409pt}{0.400pt}}
\put(1429.0,467.0){\rule[-0.200pt]{2.409pt}{0.400pt}}
\put(371.0,467.0){\rule[-0.200pt]{2.409pt}{0.400pt}}
\put(1429.0,467.0){\rule[-0.200pt]{2.409pt}{0.400pt}}
\put(371.0,467.0){\rule[-0.200pt]{2.409pt}{0.400pt}}
\put(1429.0,467.0){\rule[-0.200pt]{2.409pt}{0.400pt}}
\put(371.0,467.0){\rule[-0.200pt]{2.409pt}{0.400pt}}
\put(1429.0,467.0){\rule[-0.200pt]{2.409pt}{0.400pt}}
\put(371.0,467.0){\rule[-0.200pt]{2.409pt}{0.400pt}}
\put(1429.0,467.0){\rule[-0.200pt]{2.409pt}{0.400pt}}
\put(371.0,467.0){\rule[-0.200pt]{2.409pt}{0.400pt}}
\put(1429.0,467.0){\rule[-0.200pt]{2.409pt}{0.400pt}}
\put(371.0,467.0){\rule[-0.200pt]{2.409pt}{0.400pt}}
\put(1429.0,467.0){\rule[-0.200pt]{2.409pt}{0.400pt}}
\put(371.0,467.0){\rule[-0.200pt]{2.409pt}{0.400pt}}
\put(1429.0,467.0){\rule[-0.200pt]{2.409pt}{0.400pt}}
\put(371.0,468.0){\rule[-0.200pt]{2.409pt}{0.400pt}}
\put(1429.0,468.0){\rule[-0.200pt]{2.409pt}{0.400pt}}
\put(371.0,468.0){\rule[-0.200pt]{2.409pt}{0.400pt}}
\put(1429.0,468.0){\rule[-0.200pt]{2.409pt}{0.400pt}}
\put(371.0,468.0){\rule[-0.200pt]{2.409pt}{0.400pt}}
\put(1429.0,468.0){\rule[-0.200pt]{2.409pt}{0.400pt}}
\put(371.0,468.0){\rule[-0.200pt]{2.409pt}{0.400pt}}
\put(1429.0,468.0){\rule[-0.200pt]{2.409pt}{0.400pt}}
\put(371.0,468.0){\rule[-0.200pt]{2.409pt}{0.400pt}}
\put(1429.0,468.0){\rule[-0.200pt]{2.409pt}{0.400pt}}
\put(371.0,468.0){\rule[-0.200pt]{2.409pt}{0.400pt}}
\put(1429.0,468.0){\rule[-0.200pt]{2.409pt}{0.400pt}}
\put(371.0,468.0){\rule[-0.200pt]{2.409pt}{0.400pt}}
\put(1429.0,468.0){\rule[-0.200pt]{2.409pt}{0.400pt}}
\put(371.0,468.0){\rule[-0.200pt]{2.409pt}{0.400pt}}
\put(1429.0,468.0){\rule[-0.200pt]{2.409pt}{0.400pt}}
\put(371.0,468.0){\rule[-0.200pt]{2.409pt}{0.400pt}}
\put(1429.0,468.0){\rule[-0.200pt]{2.409pt}{0.400pt}}
\put(371.0,468.0){\rule[-0.200pt]{2.409pt}{0.400pt}}
\put(1429.0,468.0){\rule[-0.200pt]{2.409pt}{0.400pt}}
\put(371.0,468.0){\rule[-0.200pt]{2.409pt}{0.400pt}}
\put(1429.0,468.0){\rule[-0.200pt]{2.409pt}{0.400pt}}
\put(371.0,468.0){\rule[-0.200pt]{2.409pt}{0.400pt}}
\put(1429.0,468.0){\rule[-0.200pt]{2.409pt}{0.400pt}}
\put(371.0,468.0){\rule[-0.200pt]{2.409pt}{0.400pt}}
\put(1429.0,468.0){\rule[-0.200pt]{2.409pt}{0.400pt}}
\put(371.0,468.0){\rule[-0.200pt]{2.409pt}{0.400pt}}
\put(1429.0,468.0){\rule[-0.200pt]{2.409pt}{0.400pt}}
\put(371.0,468.0){\rule[-0.200pt]{2.409pt}{0.400pt}}
\put(1429.0,468.0){\rule[-0.200pt]{2.409pt}{0.400pt}}
\put(371.0,469.0){\rule[-0.200pt]{2.409pt}{0.400pt}}
\put(1429.0,469.0){\rule[-0.200pt]{2.409pt}{0.400pt}}
\put(371.0,469.0){\rule[-0.200pt]{2.409pt}{0.400pt}}
\put(1429.0,469.0){\rule[-0.200pt]{2.409pt}{0.400pt}}
\put(371.0,469.0){\rule[-0.200pt]{2.409pt}{0.400pt}}
\put(1429.0,469.0){\rule[-0.200pt]{2.409pt}{0.400pt}}
\put(371.0,469.0){\rule[-0.200pt]{2.409pt}{0.400pt}}
\put(1429.0,469.0){\rule[-0.200pt]{2.409pt}{0.400pt}}
\put(371.0,469.0){\rule[-0.200pt]{2.409pt}{0.400pt}}
\put(1429.0,469.0){\rule[-0.200pt]{2.409pt}{0.400pt}}
\put(371.0,469.0){\rule[-0.200pt]{2.409pt}{0.400pt}}
\put(1429.0,469.0){\rule[-0.200pt]{2.409pt}{0.400pt}}
\put(371.0,469.0){\rule[-0.200pt]{2.409pt}{0.400pt}}
\put(1429.0,469.0){\rule[-0.200pt]{2.409pt}{0.400pt}}
\put(371.0,469.0){\rule[-0.200pt]{2.409pt}{0.400pt}}
\put(1429.0,469.0){\rule[-0.200pt]{2.409pt}{0.400pt}}
\put(371.0,469.0){\rule[-0.200pt]{2.409pt}{0.400pt}}
\put(1429.0,469.0){\rule[-0.200pt]{2.409pt}{0.400pt}}
\put(371.0,469.0){\rule[-0.200pt]{2.409pt}{0.400pt}}
\put(1429.0,469.0){\rule[-0.200pt]{2.409pt}{0.400pt}}
\put(371.0,469.0){\rule[-0.200pt]{2.409pt}{0.400pt}}
\put(1429.0,469.0){\rule[-0.200pt]{2.409pt}{0.400pt}}
\put(371.0,469.0){\rule[-0.200pt]{2.409pt}{0.400pt}}
\put(1429.0,469.0){\rule[-0.200pt]{2.409pt}{0.400pt}}
\put(371.0,469.0){\rule[-0.200pt]{2.409pt}{0.400pt}}
\put(1429.0,469.0){\rule[-0.200pt]{2.409pt}{0.400pt}}
\put(371.0,469.0){\rule[-0.200pt]{2.409pt}{0.400pt}}
\put(1429.0,469.0){\rule[-0.200pt]{2.409pt}{0.400pt}}
\put(371.0,469.0){\rule[-0.200pt]{2.409pt}{0.400pt}}
\put(1429.0,469.0){\rule[-0.200pt]{2.409pt}{0.400pt}}
\put(371.0,469.0){\rule[-0.200pt]{2.409pt}{0.400pt}}
\put(1429.0,469.0){\rule[-0.200pt]{2.409pt}{0.400pt}}
\put(371.0,469.0){\rule[-0.200pt]{2.409pt}{0.400pt}}
\put(1429.0,469.0){\rule[-0.200pt]{2.409pt}{0.400pt}}
\put(371.0,470.0){\rule[-0.200pt]{2.409pt}{0.400pt}}
\put(1429.0,470.0){\rule[-0.200pt]{2.409pt}{0.400pt}}
\put(371.0,470.0){\rule[-0.200pt]{2.409pt}{0.400pt}}
\put(1429.0,470.0){\rule[-0.200pt]{2.409pt}{0.400pt}}
\put(371.0,470.0){\rule[-0.200pt]{2.409pt}{0.400pt}}
\put(1429.0,470.0){\rule[-0.200pt]{2.409pt}{0.400pt}}
\put(371.0,470.0){\rule[-0.200pt]{2.409pt}{0.400pt}}
\put(1429.0,470.0){\rule[-0.200pt]{2.409pt}{0.400pt}}
\put(371.0,470.0){\rule[-0.200pt]{2.409pt}{0.400pt}}
\put(1429.0,470.0){\rule[-0.200pt]{2.409pt}{0.400pt}}
\put(371.0,470.0){\rule[-0.200pt]{2.409pt}{0.400pt}}
\put(1429.0,470.0){\rule[-0.200pt]{2.409pt}{0.400pt}}
\put(371.0,470.0){\rule[-0.200pt]{2.409pt}{0.400pt}}
\put(1429.0,470.0){\rule[-0.200pt]{2.409pt}{0.400pt}}
\put(371.0,470.0){\rule[-0.200pt]{2.409pt}{0.400pt}}
\put(1429.0,470.0){\rule[-0.200pt]{2.409pt}{0.400pt}}
\put(371.0,470.0){\rule[-0.200pt]{2.409pt}{0.400pt}}
\put(1429.0,470.0){\rule[-0.200pt]{2.409pt}{0.400pt}}
\put(371.0,470.0){\rule[-0.200pt]{2.409pt}{0.400pt}}
\put(1429.0,470.0){\rule[-0.200pt]{2.409pt}{0.400pt}}
\put(371.0,470.0){\rule[-0.200pt]{2.409pt}{0.400pt}}
\put(1429.0,470.0){\rule[-0.200pt]{2.409pt}{0.400pt}}
\put(371.0,470.0){\rule[-0.200pt]{2.409pt}{0.400pt}}
\put(1429.0,470.0){\rule[-0.200pt]{2.409pt}{0.400pt}}
\put(371.0,470.0){\rule[-0.200pt]{2.409pt}{0.400pt}}
\put(1429.0,470.0){\rule[-0.200pt]{2.409pt}{0.400pt}}
\put(371.0,470.0){\rule[-0.200pt]{2.409pt}{0.400pt}}
\put(1429.0,470.0){\rule[-0.200pt]{2.409pt}{0.400pt}}
\put(371.0,470.0){\rule[-0.200pt]{2.409pt}{0.400pt}}
\put(1429.0,470.0){\rule[-0.200pt]{2.409pt}{0.400pt}}
\put(371.0,470.0){\rule[-0.200pt]{2.409pt}{0.400pt}}
\put(1429.0,470.0){\rule[-0.200pt]{2.409pt}{0.400pt}}
\put(371.0,470.0){\rule[-0.200pt]{2.409pt}{0.400pt}}
\put(1429.0,470.0){\rule[-0.200pt]{2.409pt}{0.400pt}}
\put(371.0,471.0){\rule[-0.200pt]{4.818pt}{0.400pt}}
\put(351,471){\makebox(0,0)[r]{ 1e+06}}
\put(1419.0,471.0){\rule[-0.200pt]{4.818pt}{0.400pt}}
\put(371.0,483.0){\rule[-0.200pt]{2.409pt}{0.400pt}}
\put(1429.0,483.0){\rule[-0.200pt]{2.409pt}{0.400pt}}
\put(371.0,501.0){\rule[-0.200pt]{2.409pt}{0.400pt}}
\put(1429.0,501.0){\rule[-0.200pt]{2.409pt}{0.400pt}}
\put(371.0,509.0){\rule[-0.200pt]{2.409pt}{0.400pt}}
\put(1429.0,509.0){\rule[-0.200pt]{2.409pt}{0.400pt}}
\put(371.0,515.0){\rule[-0.200pt]{2.409pt}{0.400pt}}
\put(1429.0,515.0){\rule[-0.200pt]{2.409pt}{0.400pt}}
\put(371.0,520.0){\rule[-0.200pt]{2.409pt}{0.400pt}}
\put(1429.0,520.0){\rule[-0.200pt]{2.409pt}{0.400pt}}
\put(371.0,524.0){\rule[-0.200pt]{2.409pt}{0.400pt}}
\put(1429.0,524.0){\rule[-0.200pt]{2.409pt}{0.400pt}}
\put(371.0,527.0){\rule[-0.200pt]{2.409pt}{0.400pt}}
\put(1429.0,527.0){\rule[-0.200pt]{2.409pt}{0.400pt}}
\put(371.0,529.0){\rule[-0.200pt]{2.409pt}{0.400pt}}
\put(1429.0,529.0){\rule[-0.200pt]{2.409pt}{0.400pt}}
\put(371.0,532.0){\rule[-0.200pt]{2.409pt}{0.400pt}}
\put(1429.0,532.0){\rule[-0.200pt]{2.409pt}{0.400pt}}
\put(371.0,534.0){\rule[-0.200pt]{2.409pt}{0.400pt}}
\put(1429.0,534.0){\rule[-0.200pt]{2.409pt}{0.400pt}}
\put(371.0,535.0){\rule[-0.200pt]{2.409pt}{0.400pt}}
\put(1429.0,535.0){\rule[-0.200pt]{2.409pt}{0.400pt}}
\put(371.0,537.0){\rule[-0.200pt]{2.409pt}{0.400pt}}
\put(1429.0,537.0){\rule[-0.200pt]{2.409pt}{0.400pt}}
\put(371.0,539.0){\rule[-0.200pt]{2.409pt}{0.400pt}}
\put(1429.0,539.0){\rule[-0.200pt]{2.409pt}{0.400pt}}
\put(371.0,540.0){\rule[-0.200pt]{2.409pt}{0.400pt}}
\put(1429.0,540.0){\rule[-0.200pt]{2.409pt}{0.400pt}}
\put(371.0,541.0){\rule[-0.200pt]{2.409pt}{0.400pt}}
\put(1429.0,541.0){\rule[-0.200pt]{2.409pt}{0.400pt}}
\put(371.0,543.0){\rule[-0.200pt]{2.409pt}{0.400pt}}
\put(1429.0,543.0){\rule[-0.200pt]{2.409pt}{0.400pt}}
\put(371.0,544.0){\rule[-0.200pt]{2.409pt}{0.400pt}}
\put(1429.0,544.0){\rule[-0.200pt]{2.409pt}{0.400pt}}
\put(371.0,545.0){\rule[-0.200pt]{2.409pt}{0.400pt}}
\put(1429.0,545.0){\rule[-0.200pt]{2.409pt}{0.400pt}}
\put(371.0,546.0){\rule[-0.200pt]{2.409pt}{0.400pt}}
\put(1429.0,546.0){\rule[-0.200pt]{2.409pt}{0.400pt}}
\put(371.0,547.0){\rule[-0.200pt]{2.409pt}{0.400pt}}
\put(1429.0,547.0){\rule[-0.200pt]{2.409pt}{0.400pt}}
\put(371.0,548.0){\rule[-0.200pt]{2.409pt}{0.400pt}}
\put(1429.0,548.0){\rule[-0.200pt]{2.409pt}{0.400pt}}
\put(371.0,549.0){\rule[-0.200pt]{2.409pt}{0.400pt}}
\put(1429.0,549.0){\rule[-0.200pt]{2.409pt}{0.400pt}}
\put(371.0,550.0){\rule[-0.200pt]{2.409pt}{0.400pt}}
\put(1429.0,550.0){\rule[-0.200pt]{2.409pt}{0.400pt}}
\put(371.0,550.0){\rule[-0.200pt]{2.409pt}{0.400pt}}
\put(1429.0,550.0){\rule[-0.200pt]{2.409pt}{0.400pt}}
\put(371.0,551.0){\rule[-0.200pt]{2.409pt}{0.400pt}}
\put(1429.0,551.0){\rule[-0.200pt]{2.409pt}{0.400pt}}
\put(371.0,552.0){\rule[-0.200pt]{2.409pt}{0.400pt}}
\put(1429.0,552.0){\rule[-0.200pt]{2.409pt}{0.400pt}}
\put(371.0,553.0){\rule[-0.200pt]{2.409pt}{0.400pt}}
\put(1429.0,553.0){\rule[-0.200pt]{2.409pt}{0.400pt}}
\put(371.0,553.0){\rule[-0.200pt]{2.409pt}{0.400pt}}
\put(1429.0,553.0){\rule[-0.200pt]{2.409pt}{0.400pt}}
\put(371.0,554.0){\rule[-0.200pt]{2.409pt}{0.400pt}}
\put(1429.0,554.0){\rule[-0.200pt]{2.409pt}{0.400pt}}
\put(371.0,555.0){\rule[-0.200pt]{2.409pt}{0.400pt}}
\put(1429.0,555.0){\rule[-0.200pt]{2.409pt}{0.400pt}}
\put(371.0,555.0){\rule[-0.200pt]{2.409pt}{0.400pt}}
\put(1429.0,555.0){\rule[-0.200pt]{2.409pt}{0.400pt}}
\put(371.0,556.0){\rule[-0.200pt]{2.409pt}{0.400pt}}
\put(1429.0,556.0){\rule[-0.200pt]{2.409pt}{0.400pt}}
\put(371.0,556.0){\rule[-0.200pt]{2.409pt}{0.400pt}}
\put(1429.0,556.0){\rule[-0.200pt]{2.409pt}{0.400pt}}
\put(371.0,557.0){\rule[-0.200pt]{2.409pt}{0.400pt}}
\put(1429.0,557.0){\rule[-0.200pt]{2.409pt}{0.400pt}}
\put(371.0,558.0){\rule[-0.200pt]{2.409pt}{0.400pt}}
\put(1429.0,558.0){\rule[-0.200pt]{2.409pt}{0.400pt}}
\put(371.0,558.0){\rule[-0.200pt]{2.409pt}{0.400pt}}
\put(1429.0,558.0){\rule[-0.200pt]{2.409pt}{0.400pt}}
\put(371.0,559.0){\rule[-0.200pt]{2.409pt}{0.400pt}}
\put(1429.0,559.0){\rule[-0.200pt]{2.409pt}{0.400pt}}
\put(371.0,559.0){\rule[-0.200pt]{2.409pt}{0.400pt}}
\put(1429.0,559.0){\rule[-0.200pt]{2.409pt}{0.400pt}}
\put(371.0,560.0){\rule[-0.200pt]{2.409pt}{0.400pt}}
\put(1429.0,560.0){\rule[-0.200pt]{2.409pt}{0.400pt}}
\put(371.0,560.0){\rule[-0.200pt]{2.409pt}{0.400pt}}
\put(1429.0,560.0){\rule[-0.200pt]{2.409pt}{0.400pt}}
\put(371.0,561.0){\rule[-0.200pt]{2.409pt}{0.400pt}}
\put(1429.0,561.0){\rule[-0.200pt]{2.409pt}{0.400pt}}
\put(371.0,561.0){\rule[-0.200pt]{2.409pt}{0.400pt}}
\put(1429.0,561.0){\rule[-0.200pt]{2.409pt}{0.400pt}}
\put(371.0,561.0){\rule[-0.200pt]{2.409pt}{0.400pt}}
\put(1429.0,561.0){\rule[-0.200pt]{2.409pt}{0.400pt}}
\put(371.0,562.0){\rule[-0.200pt]{2.409pt}{0.400pt}}
\put(1429.0,562.0){\rule[-0.200pt]{2.409pt}{0.400pt}}
\put(371.0,562.0){\rule[-0.200pt]{2.409pt}{0.400pt}}
\put(1429.0,562.0){\rule[-0.200pt]{2.409pt}{0.400pt}}
\put(371.0,563.0){\rule[-0.200pt]{2.409pt}{0.400pt}}
\put(1429.0,563.0){\rule[-0.200pt]{2.409pt}{0.400pt}}
\put(371.0,563.0){\rule[-0.200pt]{2.409pt}{0.400pt}}
\put(1429.0,563.0){\rule[-0.200pt]{2.409pt}{0.400pt}}
\put(371.0,564.0){\rule[-0.200pt]{2.409pt}{0.400pt}}
\put(1429.0,564.0){\rule[-0.200pt]{2.409pt}{0.400pt}}
\put(371.0,564.0){\rule[-0.200pt]{2.409pt}{0.400pt}}
\put(1429.0,564.0){\rule[-0.200pt]{2.409pt}{0.400pt}}
\put(371.0,564.0){\rule[-0.200pt]{2.409pt}{0.400pt}}
\put(1429.0,564.0){\rule[-0.200pt]{2.409pt}{0.400pt}}
\put(371.0,565.0){\rule[-0.200pt]{2.409pt}{0.400pt}}
\put(1429.0,565.0){\rule[-0.200pt]{2.409pt}{0.400pt}}
\put(371.0,565.0){\rule[-0.200pt]{2.409pt}{0.400pt}}
\put(1429.0,565.0){\rule[-0.200pt]{2.409pt}{0.400pt}}
\put(371.0,565.0){\rule[-0.200pt]{2.409pt}{0.400pt}}
\put(1429.0,565.0){\rule[-0.200pt]{2.409pt}{0.400pt}}
\put(371.0,566.0){\rule[-0.200pt]{2.409pt}{0.400pt}}
\put(1429.0,566.0){\rule[-0.200pt]{2.409pt}{0.400pt}}
\put(371.0,566.0){\rule[-0.200pt]{2.409pt}{0.400pt}}
\put(1429.0,566.0){\rule[-0.200pt]{2.409pt}{0.400pt}}
\put(371.0,566.0){\rule[-0.200pt]{2.409pt}{0.400pt}}
\put(1429.0,566.0){\rule[-0.200pt]{2.409pt}{0.400pt}}
\put(371.0,567.0){\rule[-0.200pt]{2.409pt}{0.400pt}}
\put(1429.0,567.0){\rule[-0.200pt]{2.409pt}{0.400pt}}
\put(371.0,567.0){\rule[-0.200pt]{2.409pt}{0.400pt}}
\put(1429.0,567.0){\rule[-0.200pt]{2.409pt}{0.400pt}}
\put(371.0,567.0){\rule[-0.200pt]{2.409pt}{0.400pt}}
\put(1429.0,567.0){\rule[-0.200pt]{2.409pt}{0.400pt}}
\put(371.0,568.0){\rule[-0.200pt]{2.409pt}{0.400pt}}
\put(1429.0,568.0){\rule[-0.200pt]{2.409pt}{0.400pt}}
\put(371.0,568.0){\rule[-0.200pt]{2.409pt}{0.400pt}}
\put(1429.0,568.0){\rule[-0.200pt]{2.409pt}{0.400pt}}
\put(371.0,568.0){\rule[-0.200pt]{2.409pt}{0.400pt}}
\put(1429.0,568.0){\rule[-0.200pt]{2.409pt}{0.400pt}}
\put(371.0,569.0){\rule[-0.200pt]{2.409pt}{0.400pt}}
\put(1429.0,569.0){\rule[-0.200pt]{2.409pt}{0.400pt}}
\put(371.0,569.0){\rule[-0.200pt]{2.409pt}{0.400pt}}
\put(1429.0,569.0){\rule[-0.200pt]{2.409pt}{0.400pt}}
\put(371.0,569.0){\rule[-0.200pt]{2.409pt}{0.400pt}}
\put(1429.0,569.0){\rule[-0.200pt]{2.409pt}{0.400pt}}
\put(371.0,570.0){\rule[-0.200pt]{2.409pt}{0.400pt}}
\put(1429.0,570.0){\rule[-0.200pt]{2.409pt}{0.400pt}}
\put(371.0,570.0){\rule[-0.200pt]{2.409pt}{0.400pt}}
\put(1429.0,570.0){\rule[-0.200pt]{2.409pt}{0.400pt}}
\put(371.0,570.0){\rule[-0.200pt]{2.409pt}{0.400pt}}
\put(1429.0,570.0){\rule[-0.200pt]{2.409pt}{0.400pt}}
\put(371.0,570.0){\rule[-0.200pt]{2.409pt}{0.400pt}}
\put(1429.0,570.0){\rule[-0.200pt]{2.409pt}{0.400pt}}
\put(371.0,571.0){\rule[-0.200pt]{2.409pt}{0.400pt}}
\put(1429.0,571.0){\rule[-0.200pt]{2.409pt}{0.400pt}}
\put(371.0,571.0){\rule[-0.200pt]{2.409pt}{0.400pt}}
\put(1429.0,571.0){\rule[-0.200pt]{2.409pt}{0.400pt}}
\put(371.0,571.0){\rule[-0.200pt]{2.409pt}{0.400pt}}
\put(1429.0,571.0){\rule[-0.200pt]{2.409pt}{0.400pt}}
\put(371.0,571.0){\rule[-0.200pt]{2.409pt}{0.400pt}}
\put(1429.0,571.0){\rule[-0.200pt]{2.409pt}{0.400pt}}
\put(371.0,572.0){\rule[-0.200pt]{2.409pt}{0.400pt}}
\put(1429.0,572.0){\rule[-0.200pt]{2.409pt}{0.400pt}}
\put(371.0,572.0){\rule[-0.200pt]{2.409pt}{0.400pt}}
\put(1429.0,572.0){\rule[-0.200pt]{2.409pt}{0.400pt}}
\put(371.0,572.0){\rule[-0.200pt]{2.409pt}{0.400pt}}
\put(1429.0,572.0){\rule[-0.200pt]{2.409pt}{0.400pt}}
\put(371.0,572.0){\rule[-0.200pt]{2.409pt}{0.400pt}}
\put(1429.0,572.0){\rule[-0.200pt]{2.409pt}{0.400pt}}
\put(371.0,573.0){\rule[-0.200pt]{2.409pt}{0.400pt}}
\put(1429.0,573.0){\rule[-0.200pt]{2.409pt}{0.400pt}}
\put(371.0,573.0){\rule[-0.200pt]{2.409pt}{0.400pt}}
\put(1429.0,573.0){\rule[-0.200pt]{2.409pt}{0.400pt}}
\put(371.0,573.0){\rule[-0.200pt]{2.409pt}{0.400pt}}
\put(1429.0,573.0){\rule[-0.200pt]{2.409pt}{0.400pt}}
\put(371.0,573.0){\rule[-0.200pt]{2.409pt}{0.400pt}}
\put(1429.0,573.0){\rule[-0.200pt]{2.409pt}{0.400pt}}
\put(371.0,574.0){\rule[-0.200pt]{2.409pt}{0.400pt}}
\put(1429.0,574.0){\rule[-0.200pt]{2.409pt}{0.400pt}}
\put(371.0,574.0){\rule[-0.200pt]{2.409pt}{0.400pt}}
\put(1429.0,574.0){\rule[-0.200pt]{2.409pt}{0.400pt}}
\put(371.0,574.0){\rule[-0.200pt]{2.409pt}{0.400pt}}
\put(1429.0,574.0){\rule[-0.200pt]{2.409pt}{0.400pt}}
\put(371.0,574.0){\rule[-0.200pt]{2.409pt}{0.400pt}}
\put(1429.0,574.0){\rule[-0.200pt]{2.409pt}{0.400pt}}
\put(371.0,575.0){\rule[-0.200pt]{2.409pt}{0.400pt}}
\put(1429.0,575.0){\rule[-0.200pt]{2.409pt}{0.400pt}}
\put(371.0,575.0){\rule[-0.200pt]{2.409pt}{0.400pt}}
\put(1429.0,575.0){\rule[-0.200pt]{2.409pt}{0.400pt}}
\put(371.0,575.0){\rule[-0.200pt]{2.409pt}{0.400pt}}
\put(1429.0,575.0){\rule[-0.200pt]{2.409pt}{0.400pt}}
\put(371.0,575.0){\rule[-0.200pt]{2.409pt}{0.400pt}}
\put(1429.0,575.0){\rule[-0.200pt]{2.409pt}{0.400pt}}
\put(371.0,575.0){\rule[-0.200pt]{2.409pt}{0.400pt}}
\put(1429.0,575.0){\rule[-0.200pt]{2.409pt}{0.400pt}}
\put(371.0,576.0){\rule[-0.200pt]{2.409pt}{0.400pt}}
\put(1429.0,576.0){\rule[-0.200pt]{2.409pt}{0.400pt}}
\put(371.0,576.0){\rule[-0.200pt]{2.409pt}{0.400pt}}
\put(1429.0,576.0){\rule[-0.200pt]{2.409pt}{0.400pt}}
\put(371.0,576.0){\rule[-0.200pt]{2.409pt}{0.400pt}}
\put(1429.0,576.0){\rule[-0.200pt]{2.409pt}{0.400pt}}
\put(371.0,576.0){\rule[-0.200pt]{2.409pt}{0.400pt}}
\put(1429.0,576.0){\rule[-0.200pt]{2.409pt}{0.400pt}}
\put(371.0,576.0){\rule[-0.200pt]{2.409pt}{0.400pt}}
\put(1429.0,576.0){\rule[-0.200pt]{2.409pt}{0.400pt}}
\put(371.0,577.0){\rule[-0.200pt]{2.409pt}{0.400pt}}
\put(1429.0,577.0){\rule[-0.200pt]{2.409pt}{0.400pt}}
\put(371.0,577.0){\rule[-0.200pt]{2.409pt}{0.400pt}}
\put(1429.0,577.0){\rule[-0.200pt]{2.409pt}{0.400pt}}
\put(371.0,577.0){\rule[-0.200pt]{2.409pt}{0.400pt}}
\put(1429.0,577.0){\rule[-0.200pt]{2.409pt}{0.400pt}}
\put(371.0,577.0){\rule[-0.200pt]{2.409pt}{0.400pt}}
\put(1429.0,577.0){\rule[-0.200pt]{2.409pt}{0.400pt}}
\put(371.0,577.0){\rule[-0.200pt]{2.409pt}{0.400pt}}
\put(1429.0,577.0){\rule[-0.200pt]{2.409pt}{0.400pt}}
\put(371.0,578.0){\rule[-0.200pt]{2.409pt}{0.400pt}}
\put(1429.0,578.0){\rule[-0.200pt]{2.409pt}{0.400pt}}
\put(371.0,578.0){\rule[-0.200pt]{2.409pt}{0.400pt}}
\put(1429.0,578.0){\rule[-0.200pt]{2.409pt}{0.400pt}}
\put(371.0,578.0){\rule[-0.200pt]{2.409pt}{0.400pt}}
\put(1429.0,578.0){\rule[-0.200pt]{2.409pt}{0.400pt}}
\put(371.0,578.0){\rule[-0.200pt]{2.409pt}{0.400pt}}
\put(1429.0,578.0){\rule[-0.200pt]{2.409pt}{0.400pt}}
\put(371.0,578.0){\rule[-0.200pt]{2.409pt}{0.400pt}}
\put(1429.0,578.0){\rule[-0.200pt]{2.409pt}{0.400pt}}
\put(371.0,578.0){\rule[-0.200pt]{2.409pt}{0.400pt}}
\put(1429.0,578.0){\rule[-0.200pt]{2.409pt}{0.400pt}}
\put(371.0,579.0){\rule[-0.200pt]{2.409pt}{0.400pt}}
\put(1429.0,579.0){\rule[-0.200pt]{2.409pt}{0.400pt}}
\put(371.0,579.0){\rule[-0.200pt]{2.409pt}{0.400pt}}
\put(1429.0,579.0){\rule[-0.200pt]{2.409pt}{0.400pt}}
\put(371.0,579.0){\rule[-0.200pt]{2.409pt}{0.400pt}}
\put(1429.0,579.0){\rule[-0.200pt]{2.409pt}{0.400pt}}
\put(371.0,579.0){\rule[-0.200pt]{2.409pt}{0.400pt}}
\put(1429.0,579.0){\rule[-0.200pt]{2.409pt}{0.400pt}}
\put(371.0,579.0){\rule[-0.200pt]{2.409pt}{0.400pt}}
\put(1429.0,579.0){\rule[-0.200pt]{2.409pt}{0.400pt}}
\put(371.0,579.0){\rule[-0.200pt]{2.409pt}{0.400pt}}
\put(1429.0,579.0){\rule[-0.200pt]{2.409pt}{0.400pt}}
\put(371.0,580.0){\rule[-0.200pt]{2.409pt}{0.400pt}}
\put(1429.0,580.0){\rule[-0.200pt]{2.409pt}{0.400pt}}
\put(371.0,580.0){\rule[-0.200pt]{2.409pt}{0.400pt}}
\put(1429.0,580.0){\rule[-0.200pt]{2.409pt}{0.400pt}}
\put(371.0,580.0){\rule[-0.200pt]{2.409pt}{0.400pt}}
\put(1429.0,580.0){\rule[-0.200pt]{2.409pt}{0.400pt}}
\put(371.0,580.0){\rule[-0.200pt]{2.409pt}{0.400pt}}
\put(1429.0,580.0){\rule[-0.200pt]{2.409pt}{0.400pt}}
\put(371.0,580.0){\rule[-0.200pt]{2.409pt}{0.400pt}}
\put(1429.0,580.0){\rule[-0.200pt]{2.409pt}{0.400pt}}
\put(371.0,580.0){\rule[-0.200pt]{2.409pt}{0.400pt}}
\put(1429.0,580.0){\rule[-0.200pt]{2.409pt}{0.400pt}}
\put(371.0,581.0){\rule[-0.200pt]{2.409pt}{0.400pt}}
\put(1429.0,581.0){\rule[-0.200pt]{2.409pt}{0.400pt}}
\put(371.0,581.0){\rule[-0.200pt]{2.409pt}{0.400pt}}
\put(1429.0,581.0){\rule[-0.200pt]{2.409pt}{0.400pt}}
\put(371.0,581.0){\rule[-0.200pt]{2.409pt}{0.400pt}}
\put(1429.0,581.0){\rule[-0.200pt]{2.409pt}{0.400pt}}
\put(371.0,581.0){\rule[-0.200pt]{2.409pt}{0.400pt}}
\put(1429.0,581.0){\rule[-0.200pt]{2.409pt}{0.400pt}}
\put(371.0,581.0){\rule[-0.200pt]{2.409pt}{0.400pt}}
\put(1429.0,581.0){\rule[-0.200pt]{2.409pt}{0.400pt}}
\put(371.0,581.0){\rule[-0.200pt]{2.409pt}{0.400pt}}
\put(1429.0,581.0){\rule[-0.200pt]{2.409pt}{0.400pt}}
\put(371.0,582.0){\rule[-0.200pt]{2.409pt}{0.400pt}}
\put(1429.0,582.0){\rule[-0.200pt]{2.409pt}{0.400pt}}
\put(371.0,582.0){\rule[-0.200pt]{2.409pt}{0.400pt}}
\put(1429.0,582.0){\rule[-0.200pt]{2.409pt}{0.400pt}}
\put(371.0,582.0){\rule[-0.200pt]{2.409pt}{0.400pt}}
\put(1429.0,582.0){\rule[-0.200pt]{2.409pt}{0.400pt}}
\put(371.0,582.0){\rule[-0.200pt]{2.409pt}{0.400pt}}
\put(1429.0,582.0){\rule[-0.200pt]{2.409pt}{0.400pt}}
\put(371.0,582.0){\rule[-0.200pt]{2.409pt}{0.400pt}}
\put(1429.0,582.0){\rule[-0.200pt]{2.409pt}{0.400pt}}
\put(371.0,582.0){\rule[-0.200pt]{2.409pt}{0.400pt}}
\put(1429.0,582.0){\rule[-0.200pt]{2.409pt}{0.400pt}}
\put(371.0,582.0){\rule[-0.200pt]{2.409pt}{0.400pt}}
\put(1429.0,582.0){\rule[-0.200pt]{2.409pt}{0.400pt}}
\put(371.0,583.0){\rule[-0.200pt]{2.409pt}{0.400pt}}
\put(1429.0,583.0){\rule[-0.200pt]{2.409pt}{0.400pt}}
\put(371.0,583.0){\rule[-0.200pt]{2.409pt}{0.400pt}}
\put(1429.0,583.0){\rule[-0.200pt]{2.409pt}{0.400pt}}
\put(371.0,583.0){\rule[-0.200pt]{2.409pt}{0.400pt}}
\put(1429.0,583.0){\rule[-0.200pt]{2.409pt}{0.400pt}}
\put(371.0,583.0){\rule[-0.200pt]{2.409pt}{0.400pt}}
\put(1429.0,583.0){\rule[-0.200pt]{2.409pt}{0.400pt}}
\put(371.0,583.0){\rule[-0.200pt]{2.409pt}{0.400pt}}
\put(1429.0,583.0){\rule[-0.200pt]{2.409pt}{0.400pt}}
\put(371.0,583.0){\rule[-0.200pt]{2.409pt}{0.400pt}}
\put(1429.0,583.0){\rule[-0.200pt]{2.409pt}{0.400pt}}
\put(371.0,583.0){\rule[-0.200pt]{2.409pt}{0.400pt}}
\put(1429.0,583.0){\rule[-0.200pt]{2.409pt}{0.400pt}}
\put(371.0,584.0){\rule[-0.200pt]{2.409pt}{0.400pt}}
\put(1429.0,584.0){\rule[-0.200pt]{2.409pt}{0.400pt}}
\put(371.0,584.0){\rule[-0.200pt]{2.409pt}{0.400pt}}
\put(1429.0,584.0){\rule[-0.200pt]{2.409pt}{0.400pt}}
\put(371.0,584.0){\rule[-0.200pt]{2.409pt}{0.400pt}}
\put(1429.0,584.0){\rule[-0.200pt]{2.409pt}{0.400pt}}
\put(371.0,584.0){\rule[-0.200pt]{2.409pt}{0.400pt}}
\put(1429.0,584.0){\rule[-0.200pt]{2.409pt}{0.400pt}}
\put(371.0,584.0){\rule[-0.200pt]{2.409pt}{0.400pt}}
\put(1429.0,584.0){\rule[-0.200pt]{2.409pt}{0.400pt}}
\put(371.0,584.0){\rule[-0.200pt]{2.409pt}{0.400pt}}
\put(1429.0,584.0){\rule[-0.200pt]{2.409pt}{0.400pt}}
\put(371.0,584.0){\rule[-0.200pt]{2.409pt}{0.400pt}}
\put(1429.0,584.0){\rule[-0.200pt]{2.409pt}{0.400pt}}
\put(371.0,584.0){\rule[-0.200pt]{2.409pt}{0.400pt}}
\put(1429.0,584.0){\rule[-0.200pt]{2.409pt}{0.400pt}}
\put(371.0,585.0){\rule[-0.200pt]{2.409pt}{0.400pt}}
\put(1429.0,585.0){\rule[-0.200pt]{2.409pt}{0.400pt}}
\put(371.0,585.0){\rule[-0.200pt]{2.409pt}{0.400pt}}
\put(1429.0,585.0){\rule[-0.200pt]{2.409pt}{0.400pt}}
\put(371.0,585.0){\rule[-0.200pt]{2.409pt}{0.400pt}}
\put(1429.0,585.0){\rule[-0.200pt]{2.409pt}{0.400pt}}
\put(371.0,585.0){\rule[-0.200pt]{2.409pt}{0.400pt}}
\put(1429.0,585.0){\rule[-0.200pt]{2.409pt}{0.400pt}}
\put(371.0,585.0){\rule[-0.200pt]{2.409pt}{0.400pt}}
\put(1429.0,585.0){\rule[-0.200pt]{2.409pt}{0.400pt}}
\put(371.0,585.0){\rule[-0.200pt]{2.409pt}{0.400pt}}
\put(1429.0,585.0){\rule[-0.200pt]{2.409pt}{0.400pt}}
\put(371.0,585.0){\rule[-0.200pt]{2.409pt}{0.400pt}}
\put(1429.0,585.0){\rule[-0.200pt]{2.409pt}{0.400pt}}
\put(371.0,585.0){\rule[-0.200pt]{2.409pt}{0.400pt}}
\put(1429.0,585.0){\rule[-0.200pt]{2.409pt}{0.400pt}}
\put(371.0,586.0){\rule[-0.200pt]{2.409pt}{0.400pt}}
\put(1429.0,586.0){\rule[-0.200pt]{2.409pt}{0.400pt}}
\put(371.0,586.0){\rule[-0.200pt]{2.409pt}{0.400pt}}
\put(1429.0,586.0){\rule[-0.200pt]{2.409pt}{0.400pt}}
\put(371.0,586.0){\rule[-0.200pt]{2.409pt}{0.400pt}}
\put(1429.0,586.0){\rule[-0.200pt]{2.409pt}{0.400pt}}
\put(371.0,586.0){\rule[-0.200pt]{2.409pt}{0.400pt}}
\put(1429.0,586.0){\rule[-0.200pt]{2.409pt}{0.400pt}}
\put(371.0,586.0){\rule[-0.200pt]{2.409pt}{0.400pt}}
\put(1429.0,586.0){\rule[-0.200pt]{2.409pt}{0.400pt}}
\put(371.0,586.0){\rule[-0.200pt]{2.409pt}{0.400pt}}
\put(1429.0,586.0){\rule[-0.200pt]{2.409pt}{0.400pt}}
\put(371.0,586.0){\rule[-0.200pt]{2.409pt}{0.400pt}}
\put(1429.0,586.0){\rule[-0.200pt]{2.409pt}{0.400pt}}
\put(371.0,586.0){\rule[-0.200pt]{2.409pt}{0.400pt}}
\put(1429.0,586.0){\rule[-0.200pt]{2.409pt}{0.400pt}}
\put(371.0,587.0){\rule[-0.200pt]{2.409pt}{0.400pt}}
\put(1429.0,587.0){\rule[-0.200pt]{2.409pt}{0.400pt}}
\put(371.0,587.0){\rule[-0.200pt]{2.409pt}{0.400pt}}
\put(1429.0,587.0){\rule[-0.200pt]{2.409pt}{0.400pt}}
\put(371.0,587.0){\rule[-0.200pt]{2.409pt}{0.400pt}}
\put(1429.0,587.0){\rule[-0.200pt]{2.409pt}{0.400pt}}
\put(371.0,587.0){\rule[-0.200pt]{2.409pt}{0.400pt}}
\put(1429.0,587.0){\rule[-0.200pt]{2.409pt}{0.400pt}}
\put(371.0,587.0){\rule[-0.200pt]{2.409pt}{0.400pt}}
\put(1429.0,587.0){\rule[-0.200pt]{2.409pt}{0.400pt}}
\put(371.0,587.0){\rule[-0.200pt]{2.409pt}{0.400pt}}
\put(1429.0,587.0){\rule[-0.200pt]{2.409pt}{0.400pt}}
\put(371.0,587.0){\rule[-0.200pt]{2.409pt}{0.400pt}}
\put(1429.0,587.0){\rule[-0.200pt]{2.409pt}{0.400pt}}
\put(371.0,587.0){\rule[-0.200pt]{2.409pt}{0.400pt}}
\put(1429.0,587.0){\rule[-0.200pt]{2.409pt}{0.400pt}}
\put(371.0,587.0){\rule[-0.200pt]{2.409pt}{0.400pt}}
\put(1429.0,587.0){\rule[-0.200pt]{2.409pt}{0.400pt}}
\put(371.0,588.0){\rule[-0.200pt]{2.409pt}{0.400pt}}
\put(1429.0,588.0){\rule[-0.200pt]{2.409pt}{0.400pt}}
\put(371.0,588.0){\rule[-0.200pt]{2.409pt}{0.400pt}}
\put(1429.0,588.0){\rule[-0.200pt]{2.409pt}{0.400pt}}
\put(371.0,588.0){\rule[-0.200pt]{2.409pt}{0.400pt}}
\put(1429.0,588.0){\rule[-0.200pt]{2.409pt}{0.400pt}}
\put(371.0,588.0){\rule[-0.200pt]{2.409pt}{0.400pt}}
\put(1429.0,588.0){\rule[-0.200pt]{2.409pt}{0.400pt}}
\put(371.0,588.0){\rule[-0.200pt]{2.409pt}{0.400pt}}
\put(1429.0,588.0){\rule[-0.200pt]{2.409pt}{0.400pt}}
\put(371.0,588.0){\rule[-0.200pt]{2.409pt}{0.400pt}}
\put(1429.0,588.0){\rule[-0.200pt]{2.409pt}{0.400pt}}
\put(371.0,588.0){\rule[-0.200pt]{2.409pt}{0.400pt}}
\put(1429.0,588.0){\rule[-0.200pt]{2.409pt}{0.400pt}}
\put(371.0,588.0){\rule[-0.200pt]{2.409pt}{0.400pt}}
\put(1429.0,588.0){\rule[-0.200pt]{2.409pt}{0.400pt}}
\put(371.0,588.0){\rule[-0.200pt]{2.409pt}{0.400pt}}
\put(1429.0,588.0){\rule[-0.200pt]{2.409pt}{0.400pt}}
\put(371.0,589.0){\rule[-0.200pt]{2.409pt}{0.400pt}}
\put(1429.0,589.0){\rule[-0.200pt]{2.409pt}{0.400pt}}
\put(371.0,589.0){\rule[-0.200pt]{2.409pt}{0.400pt}}
\put(1429.0,589.0){\rule[-0.200pt]{2.409pt}{0.400pt}}
\put(371.0,589.0){\rule[-0.200pt]{2.409pt}{0.400pt}}
\put(1429.0,589.0){\rule[-0.200pt]{2.409pt}{0.400pt}}
\put(371.0,589.0){\rule[-0.200pt]{2.409pt}{0.400pt}}
\put(1429.0,589.0){\rule[-0.200pt]{2.409pt}{0.400pt}}
\put(371.0,589.0){\rule[-0.200pt]{2.409pt}{0.400pt}}
\put(1429.0,589.0){\rule[-0.200pt]{2.409pt}{0.400pt}}
\put(371.0,589.0){\rule[-0.200pt]{2.409pt}{0.400pt}}
\put(1429.0,589.0){\rule[-0.200pt]{2.409pt}{0.400pt}}
\put(371.0,589.0){\rule[-0.200pt]{2.409pt}{0.400pt}}
\put(1429.0,589.0){\rule[-0.200pt]{2.409pt}{0.400pt}}
\put(371.0,589.0){\rule[-0.200pt]{2.409pt}{0.400pt}}
\put(1429.0,589.0){\rule[-0.200pt]{2.409pt}{0.400pt}}
\put(371.0,589.0){\rule[-0.200pt]{2.409pt}{0.400pt}}
\put(1429.0,589.0){\rule[-0.200pt]{2.409pt}{0.400pt}}
\put(371.0,589.0){\rule[-0.200pt]{2.409pt}{0.400pt}}
\put(1429.0,589.0){\rule[-0.200pt]{2.409pt}{0.400pt}}
\put(371.0,590.0){\rule[-0.200pt]{2.409pt}{0.400pt}}
\put(1429.0,590.0){\rule[-0.200pt]{2.409pt}{0.400pt}}
\put(371.0,590.0){\rule[-0.200pt]{2.409pt}{0.400pt}}
\put(1429.0,590.0){\rule[-0.200pt]{2.409pt}{0.400pt}}
\put(371.0,590.0){\rule[-0.200pt]{2.409pt}{0.400pt}}
\put(1429.0,590.0){\rule[-0.200pt]{2.409pt}{0.400pt}}
\put(371.0,590.0){\rule[-0.200pt]{2.409pt}{0.400pt}}
\put(1429.0,590.0){\rule[-0.200pt]{2.409pt}{0.400pt}}
\put(371.0,590.0){\rule[-0.200pt]{2.409pt}{0.400pt}}
\put(1429.0,590.0){\rule[-0.200pt]{2.409pt}{0.400pt}}
\put(371.0,590.0){\rule[-0.200pt]{2.409pt}{0.400pt}}
\put(1429.0,590.0){\rule[-0.200pt]{2.409pt}{0.400pt}}
\put(371.0,590.0){\rule[-0.200pt]{2.409pt}{0.400pt}}
\put(1429.0,590.0){\rule[-0.200pt]{2.409pt}{0.400pt}}
\put(371.0,590.0){\rule[-0.200pt]{2.409pt}{0.400pt}}
\put(1429.0,590.0){\rule[-0.200pt]{2.409pt}{0.400pt}}
\put(371.0,590.0){\rule[-0.200pt]{2.409pt}{0.400pt}}
\put(1429.0,590.0){\rule[-0.200pt]{2.409pt}{0.400pt}}
\put(371.0,590.0){\rule[-0.200pt]{2.409pt}{0.400pt}}
\put(1429.0,590.0){\rule[-0.200pt]{2.409pt}{0.400pt}}
\put(371.0,590.0){\rule[-0.200pt]{2.409pt}{0.400pt}}
\put(1429.0,590.0){\rule[-0.200pt]{2.409pt}{0.400pt}}
\put(371.0,591.0){\rule[-0.200pt]{2.409pt}{0.400pt}}
\put(1429.0,591.0){\rule[-0.200pt]{2.409pt}{0.400pt}}
\put(371.0,591.0){\rule[-0.200pt]{2.409pt}{0.400pt}}
\put(1429.0,591.0){\rule[-0.200pt]{2.409pt}{0.400pt}}
\put(371.0,591.0){\rule[-0.200pt]{2.409pt}{0.400pt}}
\put(1429.0,591.0){\rule[-0.200pt]{2.409pt}{0.400pt}}
\put(371.0,591.0){\rule[-0.200pt]{2.409pt}{0.400pt}}
\put(1429.0,591.0){\rule[-0.200pt]{2.409pt}{0.400pt}}
\put(371.0,591.0){\rule[-0.200pt]{2.409pt}{0.400pt}}
\put(1429.0,591.0){\rule[-0.200pt]{2.409pt}{0.400pt}}
\put(371.0,591.0){\rule[-0.200pt]{2.409pt}{0.400pt}}
\put(1429.0,591.0){\rule[-0.200pt]{2.409pt}{0.400pt}}
\put(371.0,591.0){\rule[-0.200pt]{2.409pt}{0.400pt}}
\put(1429.0,591.0){\rule[-0.200pt]{2.409pt}{0.400pt}}
\put(371.0,591.0){\rule[-0.200pt]{2.409pt}{0.400pt}}
\put(1429.0,591.0){\rule[-0.200pt]{2.409pt}{0.400pt}}
\put(371.0,591.0){\rule[-0.200pt]{2.409pt}{0.400pt}}
\put(1429.0,591.0){\rule[-0.200pt]{2.409pt}{0.400pt}}
\put(371.0,591.0){\rule[-0.200pt]{2.409pt}{0.400pt}}
\put(1429.0,591.0){\rule[-0.200pt]{2.409pt}{0.400pt}}
\put(371.0,591.0){\rule[-0.200pt]{2.409pt}{0.400pt}}
\put(1429.0,591.0){\rule[-0.200pt]{2.409pt}{0.400pt}}
\put(371.0,592.0){\rule[-0.200pt]{2.409pt}{0.400pt}}
\put(1429.0,592.0){\rule[-0.200pt]{2.409pt}{0.400pt}}
\put(371.0,592.0){\rule[-0.200pt]{2.409pt}{0.400pt}}
\put(1429.0,592.0){\rule[-0.200pt]{2.409pt}{0.400pt}}
\put(371.0,592.0){\rule[-0.200pt]{2.409pt}{0.400pt}}
\put(1429.0,592.0){\rule[-0.200pt]{2.409pt}{0.400pt}}
\put(371.0,592.0){\rule[-0.200pt]{2.409pt}{0.400pt}}
\put(1429.0,592.0){\rule[-0.200pt]{2.409pt}{0.400pt}}
\put(371.0,592.0){\rule[-0.200pt]{2.409pt}{0.400pt}}
\put(1429.0,592.0){\rule[-0.200pt]{2.409pt}{0.400pt}}
\put(371.0,592.0){\rule[-0.200pt]{2.409pt}{0.400pt}}
\put(1429.0,592.0){\rule[-0.200pt]{2.409pt}{0.400pt}}
\put(371.0,592.0){\rule[-0.200pt]{2.409pt}{0.400pt}}
\put(1429.0,592.0){\rule[-0.200pt]{2.409pt}{0.400pt}}
\put(371.0,592.0){\rule[-0.200pt]{2.409pt}{0.400pt}}
\put(1429.0,592.0){\rule[-0.200pt]{2.409pt}{0.400pt}}
\put(371.0,592.0){\rule[-0.200pt]{2.409pt}{0.400pt}}
\put(1429.0,592.0){\rule[-0.200pt]{2.409pt}{0.400pt}}
\put(371.0,592.0){\rule[-0.200pt]{2.409pt}{0.400pt}}
\put(1429.0,592.0){\rule[-0.200pt]{2.409pt}{0.400pt}}
\put(371.0,592.0){\rule[-0.200pt]{2.409pt}{0.400pt}}
\put(1429.0,592.0){\rule[-0.200pt]{2.409pt}{0.400pt}}
\put(371.0,593.0){\rule[-0.200pt]{2.409pt}{0.400pt}}
\put(1429.0,593.0){\rule[-0.200pt]{2.409pt}{0.400pt}}
\put(371.0,593.0){\rule[-0.200pt]{2.409pt}{0.400pt}}
\put(1429.0,593.0){\rule[-0.200pt]{2.409pt}{0.400pt}}
\put(371.0,593.0){\rule[-0.200pt]{2.409pt}{0.400pt}}
\put(1429.0,593.0){\rule[-0.200pt]{2.409pt}{0.400pt}}
\put(371.0,593.0){\rule[-0.200pt]{2.409pt}{0.400pt}}
\put(1429.0,593.0){\rule[-0.200pt]{2.409pt}{0.400pt}}
\put(371.0,593.0){\rule[-0.200pt]{2.409pt}{0.400pt}}
\put(1429.0,593.0){\rule[-0.200pt]{2.409pt}{0.400pt}}
\put(371.0,593.0){\rule[-0.200pt]{2.409pt}{0.400pt}}
\put(1429.0,593.0){\rule[-0.200pt]{2.409pt}{0.400pt}}
\put(371.0,593.0){\rule[-0.200pt]{2.409pt}{0.400pt}}
\put(1429.0,593.0){\rule[-0.200pt]{2.409pt}{0.400pt}}
\put(371.0,593.0){\rule[-0.200pt]{2.409pt}{0.400pt}}
\put(1429.0,593.0){\rule[-0.200pt]{2.409pt}{0.400pt}}
\put(371.0,593.0){\rule[-0.200pt]{2.409pt}{0.400pt}}
\put(1429.0,593.0){\rule[-0.200pt]{2.409pt}{0.400pt}}
\put(371.0,593.0){\rule[-0.200pt]{2.409pt}{0.400pt}}
\put(1429.0,593.0){\rule[-0.200pt]{2.409pt}{0.400pt}}
\put(371.0,593.0){\rule[-0.200pt]{2.409pt}{0.400pt}}
\put(1429.0,593.0){\rule[-0.200pt]{2.409pt}{0.400pt}}
\put(371.0,593.0){\rule[-0.200pt]{2.409pt}{0.400pt}}
\put(1429.0,593.0){\rule[-0.200pt]{2.409pt}{0.400pt}}
\put(371.0,593.0){\rule[-0.200pt]{2.409pt}{0.400pt}}
\put(1429.0,593.0){\rule[-0.200pt]{2.409pt}{0.400pt}}
\put(371.0,594.0){\rule[-0.200pt]{2.409pt}{0.400pt}}
\put(1429.0,594.0){\rule[-0.200pt]{2.409pt}{0.400pt}}
\put(371.0,594.0){\rule[-0.200pt]{2.409pt}{0.400pt}}
\put(1429.0,594.0){\rule[-0.200pt]{2.409pt}{0.400pt}}
\put(371.0,594.0){\rule[-0.200pt]{2.409pt}{0.400pt}}
\put(1429.0,594.0){\rule[-0.200pt]{2.409pt}{0.400pt}}
\put(371.0,594.0){\rule[-0.200pt]{2.409pt}{0.400pt}}
\put(1429.0,594.0){\rule[-0.200pt]{2.409pt}{0.400pt}}
\put(371.0,594.0){\rule[-0.200pt]{2.409pt}{0.400pt}}
\put(1429.0,594.0){\rule[-0.200pt]{2.409pt}{0.400pt}}
\put(371.0,594.0){\rule[-0.200pt]{2.409pt}{0.400pt}}
\put(1429.0,594.0){\rule[-0.200pt]{2.409pt}{0.400pt}}
\put(371.0,594.0){\rule[-0.200pt]{2.409pt}{0.400pt}}
\put(1429.0,594.0){\rule[-0.200pt]{2.409pt}{0.400pt}}
\put(371.0,594.0){\rule[-0.200pt]{2.409pt}{0.400pt}}
\put(1429.0,594.0){\rule[-0.200pt]{2.409pt}{0.400pt}}
\put(371.0,594.0){\rule[-0.200pt]{2.409pt}{0.400pt}}
\put(1429.0,594.0){\rule[-0.200pt]{2.409pt}{0.400pt}}
\put(371.0,594.0){\rule[-0.200pt]{2.409pt}{0.400pt}}
\put(1429.0,594.0){\rule[-0.200pt]{2.409pt}{0.400pt}}
\put(371.0,594.0){\rule[-0.200pt]{2.409pt}{0.400pt}}
\put(1429.0,594.0){\rule[-0.200pt]{2.409pt}{0.400pt}}
\put(371.0,594.0){\rule[-0.200pt]{2.409pt}{0.400pt}}
\put(1429.0,594.0){\rule[-0.200pt]{2.409pt}{0.400pt}}
\put(371.0,595.0){\rule[-0.200pt]{2.409pt}{0.400pt}}
\put(1429.0,595.0){\rule[-0.200pt]{2.409pt}{0.400pt}}
\put(371.0,595.0){\rule[-0.200pt]{2.409pt}{0.400pt}}
\put(1429.0,595.0){\rule[-0.200pt]{2.409pt}{0.400pt}}
\put(371.0,595.0){\rule[-0.200pt]{2.409pt}{0.400pt}}
\put(1429.0,595.0){\rule[-0.200pt]{2.409pt}{0.400pt}}
\put(371.0,595.0){\rule[-0.200pt]{2.409pt}{0.400pt}}
\put(1429.0,595.0){\rule[-0.200pt]{2.409pt}{0.400pt}}
\put(371.0,595.0){\rule[-0.200pt]{2.409pt}{0.400pt}}
\put(1429.0,595.0){\rule[-0.200pt]{2.409pt}{0.400pt}}
\put(371.0,595.0){\rule[-0.200pt]{2.409pt}{0.400pt}}
\put(1429.0,595.0){\rule[-0.200pt]{2.409pt}{0.400pt}}
\put(371.0,595.0){\rule[-0.200pt]{2.409pt}{0.400pt}}
\put(1429.0,595.0){\rule[-0.200pt]{2.409pt}{0.400pt}}
\put(371.0,595.0){\rule[-0.200pt]{2.409pt}{0.400pt}}
\put(1429.0,595.0){\rule[-0.200pt]{2.409pt}{0.400pt}}
\put(371.0,595.0){\rule[-0.200pt]{2.409pt}{0.400pt}}
\put(1429.0,595.0){\rule[-0.200pt]{2.409pt}{0.400pt}}
\put(371.0,595.0){\rule[-0.200pt]{2.409pt}{0.400pt}}
\put(1429.0,595.0){\rule[-0.200pt]{2.409pt}{0.400pt}}
\put(371.0,595.0){\rule[-0.200pt]{2.409pt}{0.400pt}}
\put(1429.0,595.0){\rule[-0.200pt]{2.409pt}{0.400pt}}
\put(371.0,595.0){\rule[-0.200pt]{2.409pt}{0.400pt}}
\put(1429.0,595.0){\rule[-0.200pt]{2.409pt}{0.400pt}}
\put(371.0,595.0){\rule[-0.200pt]{2.409pt}{0.400pt}}
\put(1429.0,595.0){\rule[-0.200pt]{2.409pt}{0.400pt}}
\put(371.0,595.0){\rule[-0.200pt]{2.409pt}{0.400pt}}
\put(1429.0,595.0){\rule[-0.200pt]{2.409pt}{0.400pt}}
\put(371.0,596.0){\rule[-0.200pt]{2.409pt}{0.400pt}}
\put(1429.0,596.0){\rule[-0.200pt]{2.409pt}{0.400pt}}
\put(371.0,596.0){\rule[-0.200pt]{2.409pt}{0.400pt}}
\put(1429.0,596.0){\rule[-0.200pt]{2.409pt}{0.400pt}}
\put(371.0,596.0){\rule[-0.200pt]{2.409pt}{0.400pt}}
\put(1429.0,596.0){\rule[-0.200pt]{2.409pt}{0.400pt}}
\put(371.0,596.0){\rule[-0.200pt]{2.409pt}{0.400pt}}
\put(1429.0,596.0){\rule[-0.200pt]{2.409pt}{0.400pt}}
\put(371.0,596.0){\rule[-0.200pt]{2.409pt}{0.400pt}}
\put(1429.0,596.0){\rule[-0.200pt]{2.409pt}{0.400pt}}
\put(371.0,596.0){\rule[-0.200pt]{2.409pt}{0.400pt}}
\put(1429.0,596.0){\rule[-0.200pt]{2.409pt}{0.400pt}}
\put(371.0,596.0){\rule[-0.200pt]{2.409pt}{0.400pt}}
\put(1429.0,596.0){\rule[-0.200pt]{2.409pt}{0.400pt}}
\put(371.0,596.0){\rule[-0.200pt]{2.409pt}{0.400pt}}
\put(1429.0,596.0){\rule[-0.200pt]{2.409pt}{0.400pt}}
\put(371.0,596.0){\rule[-0.200pt]{2.409pt}{0.400pt}}
\put(1429.0,596.0){\rule[-0.200pt]{2.409pt}{0.400pt}}
\put(371.0,596.0){\rule[-0.200pt]{2.409pt}{0.400pt}}
\put(1429.0,596.0){\rule[-0.200pt]{2.409pt}{0.400pt}}
\put(371.0,596.0){\rule[-0.200pt]{2.409pt}{0.400pt}}
\put(1429.0,596.0){\rule[-0.200pt]{2.409pt}{0.400pt}}
\put(371.0,596.0){\rule[-0.200pt]{2.409pt}{0.400pt}}
\put(1429.0,596.0){\rule[-0.200pt]{2.409pt}{0.400pt}}
\put(371.0,596.0){\rule[-0.200pt]{2.409pt}{0.400pt}}
\put(1429.0,596.0){\rule[-0.200pt]{2.409pt}{0.400pt}}
\put(371.0,596.0){\rule[-0.200pt]{2.409pt}{0.400pt}}
\put(1429.0,596.0){\rule[-0.200pt]{2.409pt}{0.400pt}}
\put(371.0,597.0){\rule[-0.200pt]{2.409pt}{0.400pt}}
\put(1429.0,597.0){\rule[-0.200pt]{2.409pt}{0.400pt}}
\put(371.0,597.0){\rule[-0.200pt]{2.409pt}{0.400pt}}
\put(1429.0,597.0){\rule[-0.200pt]{2.409pt}{0.400pt}}
\put(371.0,597.0){\rule[-0.200pt]{2.409pt}{0.400pt}}
\put(1429.0,597.0){\rule[-0.200pt]{2.409pt}{0.400pt}}
\put(371.0,597.0){\rule[-0.200pt]{2.409pt}{0.400pt}}
\put(1429.0,597.0){\rule[-0.200pt]{2.409pt}{0.400pt}}
\put(371.0,597.0){\rule[-0.200pt]{2.409pt}{0.400pt}}
\put(1429.0,597.0){\rule[-0.200pt]{2.409pt}{0.400pt}}
\put(371.0,597.0){\rule[-0.200pt]{2.409pt}{0.400pt}}
\put(1429.0,597.0){\rule[-0.200pt]{2.409pt}{0.400pt}}
\put(371.0,597.0){\rule[-0.200pt]{2.409pt}{0.400pt}}
\put(1429.0,597.0){\rule[-0.200pt]{2.409pt}{0.400pt}}
\put(371.0,597.0){\rule[-0.200pt]{2.409pt}{0.400pt}}
\put(1429.0,597.0){\rule[-0.200pt]{2.409pt}{0.400pt}}
\put(371.0,597.0){\rule[-0.200pt]{2.409pt}{0.400pt}}
\put(1429.0,597.0){\rule[-0.200pt]{2.409pt}{0.400pt}}
\put(371.0,597.0){\rule[-0.200pt]{2.409pt}{0.400pt}}
\put(1429.0,597.0){\rule[-0.200pt]{2.409pt}{0.400pt}}
\put(371.0,597.0){\rule[-0.200pt]{2.409pt}{0.400pt}}
\put(1429.0,597.0){\rule[-0.200pt]{2.409pt}{0.400pt}}
\put(371.0,597.0){\rule[-0.200pt]{2.409pt}{0.400pt}}
\put(1429.0,597.0){\rule[-0.200pt]{2.409pt}{0.400pt}}
\put(371.0,597.0){\rule[-0.200pt]{2.409pt}{0.400pt}}
\put(1429.0,597.0){\rule[-0.200pt]{2.409pt}{0.400pt}}
\put(371.0,597.0){\rule[-0.200pt]{2.409pt}{0.400pt}}
\put(1429.0,597.0){\rule[-0.200pt]{2.409pt}{0.400pt}}
\put(371.0,597.0){\rule[-0.200pt]{2.409pt}{0.400pt}}
\put(1429.0,597.0){\rule[-0.200pt]{2.409pt}{0.400pt}}
\put(371.0,597.0){\rule[-0.200pt]{2.409pt}{0.400pt}}
\put(1429.0,597.0){\rule[-0.200pt]{2.409pt}{0.400pt}}
\put(371.0,598.0){\rule[-0.200pt]{2.409pt}{0.400pt}}
\put(1429.0,598.0){\rule[-0.200pt]{2.409pt}{0.400pt}}
\put(371.0,598.0){\rule[-0.200pt]{2.409pt}{0.400pt}}
\put(1429.0,598.0){\rule[-0.200pt]{2.409pt}{0.400pt}}
\put(371.0,598.0){\rule[-0.200pt]{2.409pt}{0.400pt}}
\put(1429.0,598.0){\rule[-0.200pt]{2.409pt}{0.400pt}}
\put(371.0,598.0){\rule[-0.200pt]{2.409pt}{0.400pt}}
\put(1429.0,598.0){\rule[-0.200pt]{2.409pt}{0.400pt}}
\put(371.0,598.0){\rule[-0.200pt]{2.409pt}{0.400pt}}
\put(1429.0,598.0){\rule[-0.200pt]{2.409pt}{0.400pt}}
\put(371.0,598.0){\rule[-0.200pt]{2.409pt}{0.400pt}}
\put(1429.0,598.0){\rule[-0.200pt]{2.409pt}{0.400pt}}
\put(371.0,598.0){\rule[-0.200pt]{2.409pt}{0.400pt}}
\put(1429.0,598.0){\rule[-0.200pt]{2.409pt}{0.400pt}}
\put(371.0,598.0){\rule[-0.200pt]{2.409pt}{0.400pt}}
\put(1429.0,598.0){\rule[-0.200pt]{2.409pt}{0.400pt}}
\put(371.0,598.0){\rule[-0.200pt]{2.409pt}{0.400pt}}
\put(1429.0,598.0){\rule[-0.200pt]{2.409pt}{0.400pt}}
\put(371.0,598.0){\rule[-0.200pt]{2.409pt}{0.400pt}}
\put(1429.0,598.0){\rule[-0.200pt]{2.409pt}{0.400pt}}
\put(371.0,598.0){\rule[-0.200pt]{2.409pt}{0.400pt}}
\put(1429.0,598.0){\rule[-0.200pt]{2.409pt}{0.400pt}}
\put(371.0,598.0){\rule[-0.200pt]{2.409pt}{0.400pt}}
\put(1429.0,598.0){\rule[-0.200pt]{2.409pt}{0.400pt}}
\put(371.0,598.0){\rule[-0.200pt]{2.409pt}{0.400pt}}
\put(1429.0,598.0){\rule[-0.200pt]{2.409pt}{0.400pt}}
\put(371.0,598.0){\rule[-0.200pt]{2.409pt}{0.400pt}}
\put(1429.0,598.0){\rule[-0.200pt]{2.409pt}{0.400pt}}
\put(371.0,598.0){\rule[-0.200pt]{2.409pt}{0.400pt}}
\put(1429.0,598.0){\rule[-0.200pt]{2.409pt}{0.400pt}}
\put(371.0,598.0){\rule[-0.200pt]{2.409pt}{0.400pt}}
\put(1429.0,598.0){\rule[-0.200pt]{2.409pt}{0.400pt}}
\put(371.0,599.0){\rule[-0.200pt]{2.409pt}{0.400pt}}
\put(1429.0,599.0){\rule[-0.200pt]{2.409pt}{0.400pt}}
\put(371.0,599.0){\rule[-0.200pt]{2.409pt}{0.400pt}}
\put(1429.0,599.0){\rule[-0.200pt]{2.409pt}{0.400pt}}
\put(371.0,599.0){\rule[-0.200pt]{2.409pt}{0.400pt}}
\put(1429.0,599.0){\rule[-0.200pt]{2.409pt}{0.400pt}}
\put(371.0,599.0){\rule[-0.200pt]{2.409pt}{0.400pt}}
\put(1429.0,599.0){\rule[-0.200pt]{2.409pt}{0.400pt}}
\put(371.0,599.0){\rule[-0.200pt]{2.409pt}{0.400pt}}
\put(1429.0,599.0){\rule[-0.200pt]{2.409pt}{0.400pt}}
\put(371.0,599.0){\rule[-0.200pt]{2.409pt}{0.400pt}}
\put(1429.0,599.0){\rule[-0.200pt]{2.409pt}{0.400pt}}
\put(371.0,599.0){\rule[-0.200pt]{2.409pt}{0.400pt}}
\put(1429.0,599.0){\rule[-0.200pt]{2.409pt}{0.400pt}}
\put(371.0,599.0){\rule[-0.200pt]{2.409pt}{0.400pt}}
\put(1429.0,599.0){\rule[-0.200pt]{2.409pt}{0.400pt}}
\put(371.0,599.0){\rule[-0.200pt]{2.409pt}{0.400pt}}
\put(1429.0,599.0){\rule[-0.200pt]{2.409pt}{0.400pt}}
\put(371.0,599.0){\rule[-0.200pt]{2.409pt}{0.400pt}}
\put(1429.0,599.0){\rule[-0.200pt]{2.409pt}{0.400pt}}
\put(371.0,599.0){\rule[-0.200pt]{2.409pt}{0.400pt}}
\put(1429.0,599.0){\rule[-0.200pt]{2.409pt}{0.400pt}}
\put(371.0,599.0){\rule[-0.200pt]{2.409pt}{0.400pt}}
\put(1429.0,599.0){\rule[-0.200pt]{2.409pt}{0.400pt}}
\put(371.0,599.0){\rule[-0.200pt]{2.409pt}{0.400pt}}
\put(1429.0,599.0){\rule[-0.200pt]{2.409pt}{0.400pt}}
\put(371.0,599.0){\rule[-0.200pt]{2.409pt}{0.400pt}}
\put(1429.0,599.0){\rule[-0.200pt]{2.409pt}{0.400pt}}
\put(371.0,599.0){\rule[-0.200pt]{2.409pt}{0.400pt}}
\put(1429.0,599.0){\rule[-0.200pt]{2.409pt}{0.400pt}}
\put(371.0,599.0){\rule[-0.200pt]{2.409pt}{0.400pt}}
\put(1429.0,599.0){\rule[-0.200pt]{2.409pt}{0.400pt}}
\put(371.0,600.0){\rule[-0.200pt]{2.409pt}{0.400pt}}
\put(1429.0,600.0){\rule[-0.200pt]{2.409pt}{0.400pt}}
\put(371.0,600.0){\rule[-0.200pt]{2.409pt}{0.400pt}}
\put(1429.0,600.0){\rule[-0.200pt]{2.409pt}{0.400pt}}
\put(371.0,600.0){\rule[-0.200pt]{2.409pt}{0.400pt}}
\put(1429.0,600.0){\rule[-0.200pt]{2.409pt}{0.400pt}}
\put(371.0,600.0){\rule[-0.200pt]{2.409pt}{0.400pt}}
\put(1429.0,600.0){\rule[-0.200pt]{2.409pt}{0.400pt}}
\put(371.0,600.0){\rule[-0.200pt]{2.409pt}{0.400pt}}
\put(1429.0,600.0){\rule[-0.200pt]{2.409pt}{0.400pt}}
\put(371.0,600.0){\rule[-0.200pt]{2.409pt}{0.400pt}}
\put(1429.0,600.0){\rule[-0.200pt]{2.409pt}{0.400pt}}
\put(371.0,600.0){\rule[-0.200pt]{2.409pt}{0.400pt}}
\put(1429.0,600.0){\rule[-0.200pt]{2.409pt}{0.400pt}}
\put(371.0,600.0){\rule[-0.200pt]{2.409pt}{0.400pt}}
\put(1429.0,600.0){\rule[-0.200pt]{2.409pt}{0.400pt}}
\put(371.0,600.0){\rule[-0.200pt]{2.409pt}{0.400pt}}
\put(1429.0,600.0){\rule[-0.200pt]{2.409pt}{0.400pt}}
\put(371.0,600.0){\rule[-0.200pt]{4.818pt}{0.400pt}}
\put(351,600){\makebox(0,0)[r]{ 1e+09}}
\put(1419.0,600.0){\rule[-0.200pt]{4.818pt}{0.400pt}}
\put(371.0,613.0){\rule[-0.200pt]{2.409pt}{0.400pt}}
\put(1429.0,613.0){\rule[-0.200pt]{2.409pt}{0.400pt}}
\put(371.0,630.0){\rule[-0.200pt]{2.409pt}{0.400pt}}
\put(1429.0,630.0){\rule[-0.200pt]{2.409pt}{0.400pt}}
\put(371.0,639.0){\rule[-0.200pt]{2.409pt}{0.400pt}}
\put(1429.0,639.0){\rule[-0.200pt]{2.409pt}{0.400pt}}
\put(371.0,645.0){\rule[-0.200pt]{2.409pt}{0.400pt}}
\put(1429.0,645.0){\rule[-0.200pt]{2.409pt}{0.400pt}}
\put(371.0,649.0){\rule[-0.200pt]{2.409pt}{0.400pt}}
\put(1429.0,649.0){\rule[-0.200pt]{2.409pt}{0.400pt}}
\put(371.0,653.0){\rule[-0.200pt]{2.409pt}{0.400pt}}
\put(1429.0,653.0){\rule[-0.200pt]{2.409pt}{0.400pt}}
\put(371.0,656.0){\rule[-0.200pt]{2.409pt}{0.400pt}}
\put(1429.0,656.0){\rule[-0.200pt]{2.409pt}{0.400pt}}
\put(371.0,659.0){\rule[-0.200pt]{2.409pt}{0.400pt}}
\put(1429.0,659.0){\rule[-0.200pt]{2.409pt}{0.400pt}}
\put(371.0,661.0){\rule[-0.200pt]{2.409pt}{0.400pt}}
\put(1429.0,661.0){\rule[-0.200pt]{2.409pt}{0.400pt}}
\put(371.0,663.0){\rule[-0.200pt]{2.409pt}{0.400pt}}
\put(1429.0,663.0){\rule[-0.200pt]{2.409pt}{0.400pt}}
\put(371.0,665.0){\rule[-0.200pt]{2.409pt}{0.400pt}}
\put(1429.0,665.0){\rule[-0.200pt]{2.409pt}{0.400pt}}
\put(371.0,667.0){\rule[-0.200pt]{2.409pt}{0.400pt}}
\put(1429.0,667.0){\rule[-0.200pt]{2.409pt}{0.400pt}}
\put(371.0,668.0){\rule[-0.200pt]{2.409pt}{0.400pt}}
\put(1429.0,668.0){\rule[-0.200pt]{2.409pt}{0.400pt}}
\put(371.0,670.0){\rule[-0.200pt]{2.409pt}{0.400pt}}
\put(1429.0,670.0){\rule[-0.200pt]{2.409pt}{0.400pt}}
\put(371.0,671.0){\rule[-0.200pt]{2.409pt}{0.400pt}}
\put(1429.0,671.0){\rule[-0.200pt]{2.409pt}{0.400pt}}
\put(371.0,672.0){\rule[-0.200pt]{2.409pt}{0.400pt}}
\put(1429.0,672.0){\rule[-0.200pt]{2.409pt}{0.400pt}}
\put(371.0,673.0){\rule[-0.200pt]{2.409pt}{0.400pt}}
\put(1429.0,673.0){\rule[-0.200pt]{2.409pt}{0.400pt}}
\put(371.0,674.0){\rule[-0.200pt]{2.409pt}{0.400pt}}
\put(1429.0,674.0){\rule[-0.200pt]{2.409pt}{0.400pt}}
\put(371.0,675.0){\rule[-0.200pt]{2.409pt}{0.400pt}}
\put(1429.0,675.0){\rule[-0.200pt]{2.409pt}{0.400pt}}
\put(371.0,676.0){\rule[-0.200pt]{2.409pt}{0.400pt}}
\put(1429.0,676.0){\rule[-0.200pt]{2.409pt}{0.400pt}}
\put(371.0,677.0){\rule[-0.200pt]{2.409pt}{0.400pt}}
\put(1429.0,677.0){\rule[-0.200pt]{2.409pt}{0.400pt}}
\put(371.0,678.0){\rule[-0.200pt]{2.409pt}{0.400pt}}
\put(1429.0,678.0){\rule[-0.200pt]{2.409pt}{0.400pt}}
\put(371.0,679.0){\rule[-0.200pt]{2.409pt}{0.400pt}}
\put(1429.0,679.0){\rule[-0.200pt]{2.409pt}{0.400pt}}
\put(371.0,680.0){\rule[-0.200pt]{2.409pt}{0.400pt}}
\put(1429.0,680.0){\rule[-0.200pt]{2.409pt}{0.400pt}}
\put(371.0,681.0){\rule[-0.200pt]{2.409pt}{0.400pt}}
\put(1429.0,681.0){\rule[-0.200pt]{2.409pt}{0.400pt}}
\put(371.0,681.0){\rule[-0.200pt]{2.409pt}{0.400pt}}
\put(1429.0,681.0){\rule[-0.200pt]{2.409pt}{0.400pt}}
\put(371.0,682.0){\rule[-0.200pt]{2.409pt}{0.400pt}}
\put(1429.0,682.0){\rule[-0.200pt]{2.409pt}{0.400pt}}
\put(371.0,683.0){\rule[-0.200pt]{2.409pt}{0.400pt}}
\put(1429.0,683.0){\rule[-0.200pt]{2.409pt}{0.400pt}}
\put(371.0,684.0){\rule[-0.200pt]{2.409pt}{0.400pt}}
\put(1429.0,684.0){\rule[-0.200pt]{2.409pt}{0.400pt}}
\put(371.0,684.0){\rule[-0.200pt]{2.409pt}{0.400pt}}
\put(1429.0,684.0){\rule[-0.200pt]{2.409pt}{0.400pt}}
\put(371.0,685.0){\rule[-0.200pt]{2.409pt}{0.400pt}}
\put(1429.0,685.0){\rule[-0.200pt]{2.409pt}{0.400pt}}
\put(371.0,685.0){\rule[-0.200pt]{2.409pt}{0.400pt}}
\put(1429.0,685.0){\rule[-0.200pt]{2.409pt}{0.400pt}}
\put(371.0,686.0){\rule[-0.200pt]{2.409pt}{0.400pt}}
\put(1429.0,686.0){\rule[-0.200pt]{2.409pt}{0.400pt}}
\put(371.0,687.0){\rule[-0.200pt]{2.409pt}{0.400pt}}
\put(1429.0,687.0){\rule[-0.200pt]{2.409pt}{0.400pt}}
\put(371.0,687.0){\rule[-0.200pt]{2.409pt}{0.400pt}}
\put(1429.0,687.0){\rule[-0.200pt]{2.409pt}{0.400pt}}
\put(371.0,688.0){\rule[-0.200pt]{2.409pt}{0.400pt}}
\put(1429.0,688.0){\rule[-0.200pt]{2.409pt}{0.400pt}}
\put(371.0,688.0){\rule[-0.200pt]{2.409pt}{0.400pt}}
\put(1429.0,688.0){\rule[-0.200pt]{2.409pt}{0.400pt}}
\put(371.0,689.0){\rule[-0.200pt]{2.409pt}{0.400pt}}
\put(1429.0,689.0){\rule[-0.200pt]{2.409pt}{0.400pt}}
\put(371.0,689.0){\rule[-0.200pt]{2.409pt}{0.400pt}}
\put(1429.0,689.0){\rule[-0.200pt]{2.409pt}{0.400pt}}
\put(371.0,690.0){\rule[-0.200pt]{2.409pt}{0.400pt}}
\put(1429.0,690.0){\rule[-0.200pt]{2.409pt}{0.400pt}}
\put(371.0,690.0){\rule[-0.200pt]{2.409pt}{0.400pt}}
\put(1429.0,690.0){\rule[-0.200pt]{2.409pt}{0.400pt}}
\put(371.0,691.0){\rule[-0.200pt]{2.409pt}{0.400pt}}
\put(1429.0,691.0){\rule[-0.200pt]{2.409pt}{0.400pt}}
\put(371.0,691.0){\rule[-0.200pt]{2.409pt}{0.400pt}}
\put(1429.0,691.0){\rule[-0.200pt]{2.409pt}{0.400pt}}
\put(371.0,691.0){\rule[-0.200pt]{2.409pt}{0.400pt}}
\put(1429.0,691.0){\rule[-0.200pt]{2.409pt}{0.400pt}}
\put(371.0,692.0){\rule[-0.200pt]{2.409pt}{0.400pt}}
\put(1429.0,692.0){\rule[-0.200pt]{2.409pt}{0.400pt}}
\put(371.0,692.0){\rule[-0.200pt]{2.409pt}{0.400pt}}
\put(1429.0,692.0){\rule[-0.200pt]{2.409pt}{0.400pt}}
\put(371.0,693.0){\rule[-0.200pt]{2.409pt}{0.400pt}}
\put(1429.0,693.0){\rule[-0.200pt]{2.409pt}{0.400pt}}
\put(371.0,693.0){\rule[-0.200pt]{2.409pt}{0.400pt}}
\put(1429.0,693.0){\rule[-0.200pt]{2.409pt}{0.400pt}}
\put(371.0,693.0){\rule[-0.200pt]{2.409pt}{0.400pt}}
\put(1429.0,693.0){\rule[-0.200pt]{2.409pt}{0.400pt}}
\put(371.0,694.0){\rule[-0.200pt]{2.409pt}{0.400pt}}
\put(1429.0,694.0){\rule[-0.200pt]{2.409pt}{0.400pt}}
\put(371.0,694.0){\rule[-0.200pt]{2.409pt}{0.400pt}}
\put(1429.0,694.0){\rule[-0.200pt]{2.409pt}{0.400pt}}
\put(371.0,695.0){\rule[-0.200pt]{2.409pt}{0.400pt}}
\put(1429.0,695.0){\rule[-0.200pt]{2.409pt}{0.400pt}}
\put(371.0,695.0){\rule[-0.200pt]{2.409pt}{0.400pt}}
\put(1429.0,695.0){\rule[-0.200pt]{2.409pt}{0.400pt}}
\put(371.0,695.0){\rule[-0.200pt]{2.409pt}{0.400pt}}
\put(1429.0,695.0){\rule[-0.200pt]{2.409pt}{0.400pt}}
\put(371.0,696.0){\rule[-0.200pt]{2.409pt}{0.400pt}}
\put(1429.0,696.0){\rule[-0.200pt]{2.409pt}{0.400pt}}
\put(371.0,696.0){\rule[-0.200pt]{2.409pt}{0.400pt}}
\put(1429.0,696.0){\rule[-0.200pt]{2.409pt}{0.400pt}}
\put(371.0,696.0){\rule[-0.200pt]{2.409pt}{0.400pt}}
\put(1429.0,696.0){\rule[-0.200pt]{2.409pt}{0.400pt}}
\put(371.0,697.0){\rule[-0.200pt]{2.409pt}{0.400pt}}
\put(1429.0,697.0){\rule[-0.200pt]{2.409pt}{0.400pt}}
\put(371.0,697.0){\rule[-0.200pt]{2.409pt}{0.400pt}}
\put(1429.0,697.0){\rule[-0.200pt]{2.409pt}{0.400pt}}
\put(371.0,697.0){\rule[-0.200pt]{2.409pt}{0.400pt}}
\put(1429.0,697.0){\rule[-0.200pt]{2.409pt}{0.400pt}}
\put(371.0,698.0){\rule[-0.200pt]{2.409pt}{0.400pt}}
\put(1429.0,698.0){\rule[-0.200pt]{2.409pt}{0.400pt}}
\put(371.0,698.0){\rule[-0.200pt]{2.409pt}{0.400pt}}
\put(1429.0,698.0){\rule[-0.200pt]{2.409pt}{0.400pt}}
\put(371.0,698.0){\rule[-0.200pt]{2.409pt}{0.400pt}}
\put(1429.0,698.0){\rule[-0.200pt]{2.409pt}{0.400pt}}
\put(371.0,698.0){\rule[-0.200pt]{2.409pt}{0.400pt}}
\put(1429.0,698.0){\rule[-0.200pt]{2.409pt}{0.400pt}}
\put(371.0,699.0){\rule[-0.200pt]{2.409pt}{0.400pt}}
\put(1429.0,699.0){\rule[-0.200pt]{2.409pt}{0.400pt}}
\put(371.0,699.0){\rule[-0.200pt]{2.409pt}{0.400pt}}
\put(1429.0,699.0){\rule[-0.200pt]{2.409pt}{0.400pt}}
\put(371.0,699.0){\rule[-0.200pt]{2.409pt}{0.400pt}}
\put(1429.0,699.0){\rule[-0.200pt]{2.409pt}{0.400pt}}
\put(371.0,700.0){\rule[-0.200pt]{2.409pt}{0.400pt}}
\put(1429.0,700.0){\rule[-0.200pt]{2.409pt}{0.400pt}}
\put(371.0,700.0){\rule[-0.200pt]{2.409pt}{0.400pt}}
\put(1429.0,700.0){\rule[-0.200pt]{2.409pt}{0.400pt}}
\put(371.0,700.0){\rule[-0.200pt]{2.409pt}{0.400pt}}
\put(1429.0,700.0){\rule[-0.200pt]{2.409pt}{0.400pt}}
\put(371.0,700.0){\rule[-0.200pt]{2.409pt}{0.400pt}}
\put(1429.0,700.0){\rule[-0.200pt]{2.409pt}{0.400pt}}
\put(371.0,701.0){\rule[-0.200pt]{2.409pt}{0.400pt}}
\put(1429.0,701.0){\rule[-0.200pt]{2.409pt}{0.400pt}}
\put(371.0,701.0){\rule[-0.200pt]{2.409pt}{0.400pt}}
\put(1429.0,701.0){\rule[-0.200pt]{2.409pt}{0.400pt}}
\put(371.0,701.0){\rule[-0.200pt]{2.409pt}{0.400pt}}
\put(1429.0,701.0){\rule[-0.200pt]{2.409pt}{0.400pt}}
\put(371.0,701.0){\rule[-0.200pt]{2.409pt}{0.400pt}}
\put(1429.0,701.0){\rule[-0.200pt]{2.409pt}{0.400pt}}
\put(371.0,702.0){\rule[-0.200pt]{2.409pt}{0.400pt}}
\put(1429.0,702.0){\rule[-0.200pt]{2.409pt}{0.400pt}}
\put(371.0,702.0){\rule[-0.200pt]{2.409pt}{0.400pt}}
\put(1429.0,702.0){\rule[-0.200pt]{2.409pt}{0.400pt}}
\put(371.0,702.0){\rule[-0.200pt]{2.409pt}{0.400pt}}
\put(1429.0,702.0){\rule[-0.200pt]{2.409pt}{0.400pt}}
\put(371.0,702.0){\rule[-0.200pt]{2.409pt}{0.400pt}}
\put(1429.0,702.0){\rule[-0.200pt]{2.409pt}{0.400pt}}
\put(371.0,703.0){\rule[-0.200pt]{2.409pt}{0.400pt}}
\put(1429.0,703.0){\rule[-0.200pt]{2.409pt}{0.400pt}}
\put(371.0,703.0){\rule[-0.200pt]{2.409pt}{0.400pt}}
\put(1429.0,703.0){\rule[-0.200pt]{2.409pt}{0.400pt}}
\put(371.0,703.0){\rule[-0.200pt]{2.409pt}{0.400pt}}
\put(1429.0,703.0){\rule[-0.200pt]{2.409pt}{0.400pt}}
\put(371.0,703.0){\rule[-0.200pt]{2.409pt}{0.400pt}}
\put(1429.0,703.0){\rule[-0.200pt]{2.409pt}{0.400pt}}
\put(371.0,704.0){\rule[-0.200pt]{2.409pt}{0.400pt}}
\put(1429.0,704.0){\rule[-0.200pt]{2.409pt}{0.400pt}}
\put(371.0,704.0){\rule[-0.200pt]{2.409pt}{0.400pt}}
\put(1429.0,704.0){\rule[-0.200pt]{2.409pt}{0.400pt}}
\put(371.0,704.0){\rule[-0.200pt]{2.409pt}{0.400pt}}
\put(1429.0,704.0){\rule[-0.200pt]{2.409pt}{0.400pt}}
\put(371.0,704.0){\rule[-0.200pt]{2.409pt}{0.400pt}}
\put(1429.0,704.0){\rule[-0.200pt]{2.409pt}{0.400pt}}
\put(371.0,704.0){\rule[-0.200pt]{2.409pt}{0.400pt}}
\put(1429.0,704.0){\rule[-0.200pt]{2.409pt}{0.400pt}}
\put(371.0,705.0){\rule[-0.200pt]{2.409pt}{0.400pt}}
\put(1429.0,705.0){\rule[-0.200pt]{2.409pt}{0.400pt}}
\put(371.0,705.0){\rule[-0.200pt]{2.409pt}{0.400pt}}
\put(1429.0,705.0){\rule[-0.200pt]{2.409pt}{0.400pt}}
\put(371.0,705.0){\rule[-0.200pt]{2.409pt}{0.400pt}}
\put(1429.0,705.0){\rule[-0.200pt]{2.409pt}{0.400pt}}
\put(371.0,705.0){\rule[-0.200pt]{2.409pt}{0.400pt}}
\put(1429.0,705.0){\rule[-0.200pt]{2.409pt}{0.400pt}}
\put(371.0,706.0){\rule[-0.200pt]{2.409pt}{0.400pt}}
\put(1429.0,706.0){\rule[-0.200pt]{2.409pt}{0.400pt}}
\put(371.0,706.0){\rule[-0.200pt]{2.409pt}{0.400pt}}
\put(1429.0,706.0){\rule[-0.200pt]{2.409pt}{0.400pt}}
\put(371.0,706.0){\rule[-0.200pt]{2.409pt}{0.400pt}}
\put(1429.0,706.0){\rule[-0.200pt]{2.409pt}{0.400pt}}
\put(371.0,706.0){\rule[-0.200pt]{2.409pt}{0.400pt}}
\put(1429.0,706.0){\rule[-0.200pt]{2.409pt}{0.400pt}}
\put(371.0,706.0){\rule[-0.200pt]{2.409pt}{0.400pt}}
\put(1429.0,706.0){\rule[-0.200pt]{2.409pt}{0.400pt}}
\put(371.0,706.0){\rule[-0.200pt]{2.409pt}{0.400pt}}
\put(1429.0,706.0){\rule[-0.200pt]{2.409pt}{0.400pt}}
\put(371.0,707.0){\rule[-0.200pt]{2.409pt}{0.400pt}}
\put(1429.0,707.0){\rule[-0.200pt]{2.409pt}{0.400pt}}
\put(371.0,707.0){\rule[-0.200pt]{2.409pt}{0.400pt}}
\put(1429.0,707.0){\rule[-0.200pt]{2.409pt}{0.400pt}}
\put(371.0,707.0){\rule[-0.200pt]{2.409pt}{0.400pt}}
\put(1429.0,707.0){\rule[-0.200pt]{2.409pt}{0.400pt}}
\put(371.0,707.0){\rule[-0.200pt]{2.409pt}{0.400pt}}
\put(1429.0,707.0){\rule[-0.200pt]{2.409pt}{0.400pt}}
\put(371.0,707.0){\rule[-0.200pt]{2.409pt}{0.400pt}}
\put(1429.0,707.0){\rule[-0.200pt]{2.409pt}{0.400pt}}
\put(371.0,708.0){\rule[-0.200pt]{2.409pt}{0.400pt}}
\put(1429.0,708.0){\rule[-0.200pt]{2.409pt}{0.400pt}}
\put(371.0,708.0){\rule[-0.200pt]{2.409pt}{0.400pt}}
\put(1429.0,708.0){\rule[-0.200pt]{2.409pt}{0.400pt}}
\put(371.0,708.0){\rule[-0.200pt]{2.409pt}{0.400pt}}
\put(1429.0,708.0){\rule[-0.200pt]{2.409pt}{0.400pt}}
\put(371.0,708.0){\rule[-0.200pt]{2.409pt}{0.400pt}}
\put(1429.0,708.0){\rule[-0.200pt]{2.409pt}{0.400pt}}
\put(371.0,708.0){\rule[-0.200pt]{2.409pt}{0.400pt}}
\put(1429.0,708.0){\rule[-0.200pt]{2.409pt}{0.400pt}}
\put(371.0,708.0){\rule[-0.200pt]{2.409pt}{0.400pt}}
\put(1429.0,708.0){\rule[-0.200pt]{2.409pt}{0.400pt}}
\put(371.0,709.0){\rule[-0.200pt]{2.409pt}{0.400pt}}
\put(1429.0,709.0){\rule[-0.200pt]{2.409pt}{0.400pt}}
\put(371.0,709.0){\rule[-0.200pt]{2.409pt}{0.400pt}}
\put(1429.0,709.0){\rule[-0.200pt]{2.409pt}{0.400pt}}
\put(371.0,709.0){\rule[-0.200pt]{2.409pt}{0.400pt}}
\put(1429.0,709.0){\rule[-0.200pt]{2.409pt}{0.400pt}}
\put(371.0,709.0){\rule[-0.200pt]{2.409pt}{0.400pt}}
\put(1429.0,709.0){\rule[-0.200pt]{2.409pt}{0.400pt}}
\put(371.0,709.0){\rule[-0.200pt]{2.409pt}{0.400pt}}
\put(1429.0,709.0){\rule[-0.200pt]{2.409pt}{0.400pt}}
\put(371.0,709.0){\rule[-0.200pt]{2.409pt}{0.400pt}}
\put(1429.0,709.0){\rule[-0.200pt]{2.409pt}{0.400pt}}
\put(371.0,710.0){\rule[-0.200pt]{2.409pt}{0.400pt}}
\put(1429.0,710.0){\rule[-0.200pt]{2.409pt}{0.400pt}}
\put(371.0,710.0){\rule[-0.200pt]{2.409pt}{0.400pt}}
\put(1429.0,710.0){\rule[-0.200pt]{2.409pt}{0.400pt}}
\put(371.0,710.0){\rule[-0.200pt]{2.409pt}{0.400pt}}
\put(1429.0,710.0){\rule[-0.200pt]{2.409pt}{0.400pt}}
\put(371.0,710.0){\rule[-0.200pt]{2.409pt}{0.400pt}}
\put(1429.0,710.0){\rule[-0.200pt]{2.409pt}{0.400pt}}
\put(371.0,710.0){\rule[-0.200pt]{2.409pt}{0.400pt}}
\put(1429.0,710.0){\rule[-0.200pt]{2.409pt}{0.400pt}}
\put(371.0,710.0){\rule[-0.200pt]{2.409pt}{0.400pt}}
\put(1429.0,710.0){\rule[-0.200pt]{2.409pt}{0.400pt}}
\put(371.0,711.0){\rule[-0.200pt]{2.409pt}{0.400pt}}
\put(1429.0,711.0){\rule[-0.200pt]{2.409pt}{0.400pt}}
\put(371.0,711.0){\rule[-0.200pt]{2.409pt}{0.400pt}}
\put(1429.0,711.0){\rule[-0.200pt]{2.409pt}{0.400pt}}
\put(371.0,711.0){\rule[-0.200pt]{2.409pt}{0.400pt}}
\put(1429.0,711.0){\rule[-0.200pt]{2.409pt}{0.400pt}}
\put(371.0,711.0){\rule[-0.200pt]{2.409pt}{0.400pt}}
\put(1429.0,711.0){\rule[-0.200pt]{2.409pt}{0.400pt}}
\put(371.0,711.0){\rule[-0.200pt]{2.409pt}{0.400pt}}
\put(1429.0,711.0){\rule[-0.200pt]{2.409pt}{0.400pt}}
\put(371.0,711.0){\rule[-0.200pt]{2.409pt}{0.400pt}}
\put(1429.0,711.0){\rule[-0.200pt]{2.409pt}{0.400pt}}
\put(371.0,712.0){\rule[-0.200pt]{2.409pt}{0.400pt}}
\put(1429.0,712.0){\rule[-0.200pt]{2.409pt}{0.400pt}}
\put(371.0,712.0){\rule[-0.200pt]{2.409pt}{0.400pt}}
\put(1429.0,712.0){\rule[-0.200pt]{2.409pt}{0.400pt}}
\put(371.0,712.0){\rule[-0.200pt]{2.409pt}{0.400pt}}
\put(1429.0,712.0){\rule[-0.200pt]{2.409pt}{0.400pt}}
\put(371.0,712.0){\rule[-0.200pt]{2.409pt}{0.400pt}}
\put(1429.0,712.0){\rule[-0.200pt]{2.409pt}{0.400pt}}
\put(371.0,712.0){\rule[-0.200pt]{2.409pt}{0.400pt}}
\put(1429.0,712.0){\rule[-0.200pt]{2.409pt}{0.400pt}}
\put(371.0,712.0){\rule[-0.200pt]{2.409pt}{0.400pt}}
\put(1429.0,712.0){\rule[-0.200pt]{2.409pt}{0.400pt}}
\put(371.0,712.0){\rule[-0.200pt]{2.409pt}{0.400pt}}
\put(1429.0,712.0){\rule[-0.200pt]{2.409pt}{0.400pt}}
\put(371.0,713.0){\rule[-0.200pt]{2.409pt}{0.400pt}}
\put(1429.0,713.0){\rule[-0.200pt]{2.409pt}{0.400pt}}
\put(371.0,713.0){\rule[-0.200pt]{2.409pt}{0.400pt}}
\put(1429.0,713.0){\rule[-0.200pt]{2.409pt}{0.400pt}}
\put(371.0,713.0){\rule[-0.200pt]{2.409pt}{0.400pt}}
\put(1429.0,713.0){\rule[-0.200pt]{2.409pt}{0.400pt}}
\put(371.0,713.0){\rule[-0.200pt]{2.409pt}{0.400pt}}
\put(1429.0,713.0){\rule[-0.200pt]{2.409pt}{0.400pt}}
\put(371.0,713.0){\rule[-0.200pt]{2.409pt}{0.400pt}}
\put(1429.0,713.0){\rule[-0.200pt]{2.409pt}{0.400pt}}
\put(371.0,713.0){\rule[-0.200pt]{2.409pt}{0.400pt}}
\put(1429.0,713.0){\rule[-0.200pt]{2.409pt}{0.400pt}}
\put(371.0,713.0){\rule[-0.200pt]{2.409pt}{0.400pt}}
\put(1429.0,713.0){\rule[-0.200pt]{2.409pt}{0.400pt}}
\put(371.0,713.0){\rule[-0.200pt]{2.409pt}{0.400pt}}
\put(1429.0,713.0){\rule[-0.200pt]{2.409pt}{0.400pt}}
\put(371.0,714.0){\rule[-0.200pt]{2.409pt}{0.400pt}}
\put(1429.0,714.0){\rule[-0.200pt]{2.409pt}{0.400pt}}
\put(371.0,714.0){\rule[-0.200pt]{2.409pt}{0.400pt}}
\put(1429.0,714.0){\rule[-0.200pt]{2.409pt}{0.400pt}}
\put(371.0,714.0){\rule[-0.200pt]{2.409pt}{0.400pt}}
\put(1429.0,714.0){\rule[-0.200pt]{2.409pt}{0.400pt}}
\put(371.0,714.0){\rule[-0.200pt]{2.409pt}{0.400pt}}
\put(1429.0,714.0){\rule[-0.200pt]{2.409pt}{0.400pt}}
\put(371.0,714.0){\rule[-0.200pt]{2.409pt}{0.400pt}}
\put(1429.0,714.0){\rule[-0.200pt]{2.409pt}{0.400pt}}
\put(371.0,714.0){\rule[-0.200pt]{2.409pt}{0.400pt}}
\put(1429.0,714.0){\rule[-0.200pt]{2.409pt}{0.400pt}}
\put(371.0,714.0){\rule[-0.200pt]{2.409pt}{0.400pt}}
\put(1429.0,714.0){\rule[-0.200pt]{2.409pt}{0.400pt}}
\put(371.0,714.0){\rule[-0.200pt]{2.409pt}{0.400pt}}
\put(1429.0,714.0){\rule[-0.200pt]{2.409pt}{0.400pt}}
\put(371.0,715.0){\rule[-0.200pt]{2.409pt}{0.400pt}}
\put(1429.0,715.0){\rule[-0.200pt]{2.409pt}{0.400pt}}
\put(371.0,715.0){\rule[-0.200pt]{2.409pt}{0.400pt}}
\put(1429.0,715.0){\rule[-0.200pt]{2.409pt}{0.400pt}}
\put(371.0,715.0){\rule[-0.200pt]{2.409pt}{0.400pt}}
\put(1429.0,715.0){\rule[-0.200pt]{2.409pt}{0.400pt}}
\put(371.0,715.0){\rule[-0.200pt]{2.409pt}{0.400pt}}
\put(1429.0,715.0){\rule[-0.200pt]{2.409pt}{0.400pt}}
\put(371.0,715.0){\rule[-0.200pt]{2.409pt}{0.400pt}}
\put(1429.0,715.0){\rule[-0.200pt]{2.409pt}{0.400pt}}
\put(371.0,715.0){\rule[-0.200pt]{2.409pt}{0.400pt}}
\put(1429.0,715.0){\rule[-0.200pt]{2.409pt}{0.400pt}}
\put(371.0,715.0){\rule[-0.200pt]{2.409pt}{0.400pt}}
\put(1429.0,715.0){\rule[-0.200pt]{2.409pt}{0.400pt}}
\put(371.0,715.0){\rule[-0.200pt]{2.409pt}{0.400pt}}
\put(1429.0,715.0){\rule[-0.200pt]{2.409pt}{0.400pt}}
\put(371.0,716.0){\rule[-0.200pt]{2.409pt}{0.400pt}}
\put(1429.0,716.0){\rule[-0.200pt]{2.409pt}{0.400pt}}
\put(371.0,716.0){\rule[-0.200pt]{2.409pt}{0.400pt}}
\put(1429.0,716.0){\rule[-0.200pt]{2.409pt}{0.400pt}}
\put(371.0,716.0){\rule[-0.200pt]{2.409pt}{0.400pt}}
\put(1429.0,716.0){\rule[-0.200pt]{2.409pt}{0.400pt}}
\put(371.0,716.0){\rule[-0.200pt]{2.409pt}{0.400pt}}
\put(1429.0,716.0){\rule[-0.200pt]{2.409pt}{0.400pt}}
\put(371.0,716.0){\rule[-0.200pt]{2.409pt}{0.400pt}}
\put(1429.0,716.0){\rule[-0.200pt]{2.409pt}{0.400pt}}
\put(371.0,716.0){\rule[-0.200pt]{2.409pt}{0.400pt}}
\put(1429.0,716.0){\rule[-0.200pt]{2.409pt}{0.400pt}}
\put(371.0,716.0){\rule[-0.200pt]{2.409pt}{0.400pt}}
\put(1429.0,716.0){\rule[-0.200pt]{2.409pt}{0.400pt}}
\put(371.0,716.0){\rule[-0.200pt]{2.409pt}{0.400pt}}
\put(1429.0,716.0){\rule[-0.200pt]{2.409pt}{0.400pt}}
\put(371.0,717.0){\rule[-0.200pt]{2.409pt}{0.400pt}}
\put(1429.0,717.0){\rule[-0.200pt]{2.409pt}{0.400pt}}
\put(371.0,717.0){\rule[-0.200pt]{2.409pt}{0.400pt}}
\put(1429.0,717.0){\rule[-0.200pt]{2.409pt}{0.400pt}}
\put(371.0,717.0){\rule[-0.200pt]{2.409pt}{0.400pt}}
\put(1429.0,717.0){\rule[-0.200pt]{2.409pt}{0.400pt}}
\put(371.0,717.0){\rule[-0.200pt]{2.409pt}{0.400pt}}
\put(1429.0,717.0){\rule[-0.200pt]{2.409pt}{0.400pt}}
\put(371.0,717.0){\rule[-0.200pt]{2.409pt}{0.400pt}}
\put(1429.0,717.0){\rule[-0.200pt]{2.409pt}{0.400pt}}
\put(371.0,717.0){\rule[-0.200pt]{2.409pt}{0.400pt}}
\put(1429.0,717.0){\rule[-0.200pt]{2.409pt}{0.400pt}}
\put(371.0,717.0){\rule[-0.200pt]{2.409pt}{0.400pt}}
\put(1429.0,717.0){\rule[-0.200pt]{2.409pt}{0.400pt}}
\put(371.0,717.0){\rule[-0.200pt]{2.409pt}{0.400pt}}
\put(1429.0,717.0){\rule[-0.200pt]{2.409pt}{0.400pt}}
\put(371.0,717.0){\rule[-0.200pt]{2.409pt}{0.400pt}}
\put(1429.0,717.0){\rule[-0.200pt]{2.409pt}{0.400pt}}
\put(371.0,717.0){\rule[-0.200pt]{2.409pt}{0.400pt}}
\put(1429.0,717.0){\rule[-0.200pt]{2.409pt}{0.400pt}}
\put(371.0,718.0){\rule[-0.200pt]{2.409pt}{0.400pt}}
\put(1429.0,718.0){\rule[-0.200pt]{2.409pt}{0.400pt}}
\put(371.0,718.0){\rule[-0.200pt]{2.409pt}{0.400pt}}
\put(1429.0,718.0){\rule[-0.200pt]{2.409pt}{0.400pt}}
\put(371.0,718.0){\rule[-0.200pt]{2.409pt}{0.400pt}}
\put(1429.0,718.0){\rule[-0.200pt]{2.409pt}{0.400pt}}
\put(371.0,718.0){\rule[-0.200pt]{2.409pt}{0.400pt}}
\put(1429.0,718.0){\rule[-0.200pt]{2.409pt}{0.400pt}}
\put(371.0,718.0){\rule[-0.200pt]{2.409pt}{0.400pt}}
\put(1429.0,718.0){\rule[-0.200pt]{2.409pt}{0.400pt}}
\put(371.0,718.0){\rule[-0.200pt]{2.409pt}{0.400pt}}
\put(1429.0,718.0){\rule[-0.200pt]{2.409pt}{0.400pt}}
\put(371.0,718.0){\rule[-0.200pt]{2.409pt}{0.400pt}}
\put(1429.0,718.0){\rule[-0.200pt]{2.409pt}{0.400pt}}
\put(371.0,718.0){\rule[-0.200pt]{2.409pt}{0.400pt}}
\put(1429.0,718.0){\rule[-0.200pt]{2.409pt}{0.400pt}}
\put(371.0,718.0){\rule[-0.200pt]{2.409pt}{0.400pt}}
\put(1429.0,718.0){\rule[-0.200pt]{2.409pt}{0.400pt}}
\put(371.0,719.0){\rule[-0.200pt]{2.409pt}{0.400pt}}
\put(1429.0,719.0){\rule[-0.200pt]{2.409pt}{0.400pt}}
\put(371.0,719.0){\rule[-0.200pt]{2.409pt}{0.400pt}}
\put(1429.0,719.0){\rule[-0.200pt]{2.409pt}{0.400pt}}
\put(371.0,719.0){\rule[-0.200pt]{2.409pt}{0.400pt}}
\put(1429.0,719.0){\rule[-0.200pt]{2.409pt}{0.400pt}}
\put(371.0,719.0){\rule[-0.200pt]{2.409pt}{0.400pt}}
\put(1429.0,719.0){\rule[-0.200pt]{2.409pt}{0.400pt}}
\put(371.0,719.0){\rule[-0.200pt]{2.409pt}{0.400pt}}
\put(1429.0,719.0){\rule[-0.200pt]{2.409pt}{0.400pt}}
\put(371.0,719.0){\rule[-0.200pt]{2.409pt}{0.400pt}}
\put(1429.0,719.0){\rule[-0.200pt]{2.409pt}{0.400pt}}
\put(371.0,719.0){\rule[-0.200pt]{2.409pt}{0.400pt}}
\put(1429.0,719.0){\rule[-0.200pt]{2.409pt}{0.400pt}}
\put(371.0,719.0){\rule[-0.200pt]{2.409pt}{0.400pt}}
\put(1429.0,719.0){\rule[-0.200pt]{2.409pt}{0.400pt}}
\put(371.0,719.0){\rule[-0.200pt]{2.409pt}{0.400pt}}
\put(1429.0,719.0){\rule[-0.200pt]{2.409pt}{0.400pt}}
\put(371.0,719.0){\rule[-0.200pt]{2.409pt}{0.400pt}}
\put(1429.0,719.0){\rule[-0.200pt]{2.409pt}{0.400pt}}
\put(371.0,720.0){\rule[-0.200pt]{2.409pt}{0.400pt}}
\put(1429.0,720.0){\rule[-0.200pt]{2.409pt}{0.400pt}}
\put(371.0,720.0){\rule[-0.200pt]{2.409pt}{0.400pt}}
\put(1429.0,720.0){\rule[-0.200pt]{2.409pt}{0.400pt}}
\put(371.0,720.0){\rule[-0.200pt]{2.409pt}{0.400pt}}
\put(1429.0,720.0){\rule[-0.200pt]{2.409pt}{0.400pt}}
\put(371.0,720.0){\rule[-0.200pt]{2.409pt}{0.400pt}}
\put(1429.0,720.0){\rule[-0.200pt]{2.409pt}{0.400pt}}
\put(371.0,720.0){\rule[-0.200pt]{2.409pt}{0.400pt}}
\put(1429.0,720.0){\rule[-0.200pt]{2.409pt}{0.400pt}}
\put(371.0,720.0){\rule[-0.200pt]{2.409pt}{0.400pt}}
\put(1429.0,720.0){\rule[-0.200pt]{2.409pt}{0.400pt}}
\put(371.0,720.0){\rule[-0.200pt]{2.409pt}{0.400pt}}
\put(1429.0,720.0){\rule[-0.200pt]{2.409pt}{0.400pt}}
\put(371.0,720.0){\rule[-0.200pt]{2.409pt}{0.400pt}}
\put(1429.0,720.0){\rule[-0.200pt]{2.409pt}{0.400pt}}
\put(371.0,720.0){\rule[-0.200pt]{2.409pt}{0.400pt}}
\put(1429.0,720.0){\rule[-0.200pt]{2.409pt}{0.400pt}}
\put(371.0,720.0){\rule[-0.200pt]{2.409pt}{0.400pt}}
\put(1429.0,720.0){\rule[-0.200pt]{2.409pt}{0.400pt}}
\put(371.0,720.0){\rule[-0.200pt]{2.409pt}{0.400pt}}
\put(1429.0,720.0){\rule[-0.200pt]{2.409pt}{0.400pt}}
\put(371.0,721.0){\rule[-0.200pt]{2.409pt}{0.400pt}}
\put(1429.0,721.0){\rule[-0.200pt]{2.409pt}{0.400pt}}
\put(371.0,721.0){\rule[-0.200pt]{2.409pt}{0.400pt}}
\put(1429.0,721.0){\rule[-0.200pt]{2.409pt}{0.400pt}}
\put(371.0,721.0){\rule[-0.200pt]{2.409pt}{0.400pt}}
\put(1429.0,721.0){\rule[-0.200pt]{2.409pt}{0.400pt}}
\put(371.0,721.0){\rule[-0.200pt]{2.409pt}{0.400pt}}
\put(1429.0,721.0){\rule[-0.200pt]{2.409pt}{0.400pt}}
\put(371.0,721.0){\rule[-0.200pt]{2.409pt}{0.400pt}}
\put(1429.0,721.0){\rule[-0.200pt]{2.409pt}{0.400pt}}
\put(371.0,721.0){\rule[-0.200pt]{2.409pt}{0.400pt}}
\put(1429.0,721.0){\rule[-0.200pt]{2.409pt}{0.400pt}}
\put(371.0,721.0){\rule[-0.200pt]{2.409pt}{0.400pt}}
\put(1429.0,721.0){\rule[-0.200pt]{2.409pt}{0.400pt}}
\put(371.0,721.0){\rule[-0.200pt]{2.409pt}{0.400pt}}
\put(1429.0,721.0){\rule[-0.200pt]{2.409pt}{0.400pt}}
\put(371.0,721.0){\rule[-0.200pt]{2.409pt}{0.400pt}}
\put(1429.0,721.0){\rule[-0.200pt]{2.409pt}{0.400pt}}
\put(371.0,721.0){\rule[-0.200pt]{2.409pt}{0.400pt}}
\put(1429.0,721.0){\rule[-0.200pt]{2.409pt}{0.400pt}}
\put(371.0,721.0){\rule[-0.200pt]{2.409pt}{0.400pt}}
\put(1429.0,721.0){\rule[-0.200pt]{2.409pt}{0.400pt}}
\put(371.0,722.0){\rule[-0.200pt]{2.409pt}{0.400pt}}
\put(1429.0,722.0){\rule[-0.200pt]{2.409pt}{0.400pt}}
\put(371.0,722.0){\rule[-0.200pt]{2.409pt}{0.400pt}}
\put(1429.0,722.0){\rule[-0.200pt]{2.409pt}{0.400pt}}
\put(371.0,722.0){\rule[-0.200pt]{2.409pt}{0.400pt}}
\put(1429.0,722.0){\rule[-0.200pt]{2.409pt}{0.400pt}}
\put(371.0,722.0){\rule[-0.200pt]{2.409pt}{0.400pt}}
\put(1429.0,722.0){\rule[-0.200pt]{2.409pt}{0.400pt}}
\put(371.0,722.0){\rule[-0.200pt]{2.409pt}{0.400pt}}
\put(1429.0,722.0){\rule[-0.200pt]{2.409pt}{0.400pt}}
\put(371.0,722.0){\rule[-0.200pt]{2.409pt}{0.400pt}}
\put(1429.0,722.0){\rule[-0.200pt]{2.409pt}{0.400pt}}
\put(371.0,722.0){\rule[-0.200pt]{2.409pt}{0.400pt}}
\put(1429.0,722.0){\rule[-0.200pt]{2.409pt}{0.400pt}}
\put(371.0,722.0){\rule[-0.200pt]{2.409pt}{0.400pt}}
\put(1429.0,722.0){\rule[-0.200pt]{2.409pt}{0.400pt}}
\put(371.0,722.0){\rule[-0.200pt]{2.409pt}{0.400pt}}
\put(1429.0,722.0){\rule[-0.200pt]{2.409pt}{0.400pt}}
\put(371.0,722.0){\rule[-0.200pt]{2.409pt}{0.400pt}}
\put(1429.0,722.0){\rule[-0.200pt]{2.409pt}{0.400pt}}
\put(371.0,722.0){\rule[-0.200pt]{2.409pt}{0.400pt}}
\put(1429.0,722.0){\rule[-0.200pt]{2.409pt}{0.400pt}}
\put(371.0,722.0){\rule[-0.200pt]{2.409pt}{0.400pt}}
\put(1429.0,722.0){\rule[-0.200pt]{2.409pt}{0.400pt}}
\put(371.0,723.0){\rule[-0.200pt]{2.409pt}{0.400pt}}
\put(1429.0,723.0){\rule[-0.200pt]{2.409pt}{0.400pt}}
\put(371.0,723.0){\rule[-0.200pt]{2.409pt}{0.400pt}}
\put(1429.0,723.0){\rule[-0.200pt]{2.409pt}{0.400pt}}
\put(371.0,723.0){\rule[-0.200pt]{2.409pt}{0.400pt}}
\put(1429.0,723.0){\rule[-0.200pt]{2.409pt}{0.400pt}}
\put(371.0,723.0){\rule[-0.200pt]{2.409pt}{0.400pt}}
\put(1429.0,723.0){\rule[-0.200pt]{2.409pt}{0.400pt}}
\put(371.0,723.0){\rule[-0.200pt]{2.409pt}{0.400pt}}
\put(1429.0,723.0){\rule[-0.200pt]{2.409pt}{0.400pt}}
\put(371.0,723.0){\rule[-0.200pt]{2.409pt}{0.400pt}}
\put(1429.0,723.0){\rule[-0.200pt]{2.409pt}{0.400pt}}
\put(371.0,723.0){\rule[-0.200pt]{2.409pt}{0.400pt}}
\put(1429.0,723.0){\rule[-0.200pt]{2.409pt}{0.400pt}}
\put(371.0,723.0){\rule[-0.200pt]{2.409pt}{0.400pt}}
\put(1429.0,723.0){\rule[-0.200pt]{2.409pt}{0.400pt}}
\put(371.0,723.0){\rule[-0.200pt]{2.409pt}{0.400pt}}
\put(1429.0,723.0){\rule[-0.200pt]{2.409pt}{0.400pt}}
\put(371.0,723.0){\rule[-0.200pt]{2.409pt}{0.400pt}}
\put(1429.0,723.0){\rule[-0.200pt]{2.409pt}{0.400pt}}
\put(371.0,723.0){\rule[-0.200pt]{2.409pt}{0.400pt}}
\put(1429.0,723.0){\rule[-0.200pt]{2.409pt}{0.400pt}}
\put(371.0,723.0){\rule[-0.200pt]{2.409pt}{0.400pt}}
\put(1429.0,723.0){\rule[-0.200pt]{2.409pt}{0.400pt}}
\put(371.0,723.0){\rule[-0.200pt]{2.409pt}{0.400pt}}
\put(1429.0,723.0){\rule[-0.200pt]{2.409pt}{0.400pt}}
\put(371.0,724.0){\rule[-0.200pt]{2.409pt}{0.400pt}}
\put(1429.0,724.0){\rule[-0.200pt]{2.409pt}{0.400pt}}
\put(371.0,724.0){\rule[-0.200pt]{2.409pt}{0.400pt}}
\put(1429.0,724.0){\rule[-0.200pt]{2.409pt}{0.400pt}}
\put(371.0,724.0){\rule[-0.200pt]{2.409pt}{0.400pt}}
\put(1429.0,724.0){\rule[-0.200pt]{2.409pt}{0.400pt}}
\put(371.0,724.0){\rule[-0.200pt]{2.409pt}{0.400pt}}
\put(1429.0,724.0){\rule[-0.200pt]{2.409pt}{0.400pt}}
\put(371.0,724.0){\rule[-0.200pt]{2.409pt}{0.400pt}}
\put(1429.0,724.0){\rule[-0.200pt]{2.409pt}{0.400pt}}
\put(371.0,724.0){\rule[-0.200pt]{2.409pt}{0.400pt}}
\put(1429.0,724.0){\rule[-0.200pt]{2.409pt}{0.400pt}}
\put(371.0,724.0){\rule[-0.200pt]{2.409pt}{0.400pt}}
\put(1429.0,724.0){\rule[-0.200pt]{2.409pt}{0.400pt}}
\put(371.0,724.0){\rule[-0.200pt]{2.409pt}{0.400pt}}
\put(1429.0,724.0){\rule[-0.200pt]{2.409pt}{0.400pt}}
\put(371.0,724.0){\rule[-0.200pt]{2.409pt}{0.400pt}}
\put(1429.0,724.0){\rule[-0.200pt]{2.409pt}{0.400pt}}
\put(371.0,724.0){\rule[-0.200pt]{2.409pt}{0.400pt}}
\put(1429.0,724.0){\rule[-0.200pt]{2.409pt}{0.400pt}}
\put(371.0,724.0){\rule[-0.200pt]{2.409pt}{0.400pt}}
\put(1429.0,724.0){\rule[-0.200pt]{2.409pt}{0.400pt}}
\put(371.0,724.0){\rule[-0.200pt]{2.409pt}{0.400pt}}
\put(1429.0,724.0){\rule[-0.200pt]{2.409pt}{0.400pt}}
\put(371.0,724.0){\rule[-0.200pt]{2.409pt}{0.400pt}}
\put(1429.0,724.0){\rule[-0.200pt]{2.409pt}{0.400pt}}
\put(371.0,725.0){\rule[-0.200pt]{2.409pt}{0.400pt}}
\put(1429.0,725.0){\rule[-0.200pt]{2.409pt}{0.400pt}}
\put(371.0,725.0){\rule[-0.200pt]{2.409pt}{0.400pt}}
\put(1429.0,725.0){\rule[-0.200pt]{2.409pt}{0.400pt}}
\put(371.0,725.0){\rule[-0.200pt]{2.409pt}{0.400pt}}
\put(1429.0,725.0){\rule[-0.200pt]{2.409pt}{0.400pt}}
\put(371.0,725.0){\rule[-0.200pt]{2.409pt}{0.400pt}}
\put(1429.0,725.0){\rule[-0.200pt]{2.409pt}{0.400pt}}
\put(371.0,725.0){\rule[-0.200pt]{2.409pt}{0.400pt}}
\put(1429.0,725.0){\rule[-0.200pt]{2.409pt}{0.400pt}}
\put(371.0,725.0){\rule[-0.200pt]{2.409pt}{0.400pt}}
\put(1429.0,725.0){\rule[-0.200pt]{2.409pt}{0.400pt}}
\put(371.0,725.0){\rule[-0.200pt]{2.409pt}{0.400pt}}
\put(1429.0,725.0){\rule[-0.200pt]{2.409pt}{0.400pt}}
\put(371.0,725.0){\rule[-0.200pt]{2.409pt}{0.400pt}}
\put(1429.0,725.0){\rule[-0.200pt]{2.409pt}{0.400pt}}
\put(371.0,725.0){\rule[-0.200pt]{2.409pt}{0.400pt}}
\put(1429.0,725.0){\rule[-0.200pt]{2.409pt}{0.400pt}}
\put(371.0,725.0){\rule[-0.200pt]{2.409pt}{0.400pt}}
\put(1429.0,725.0){\rule[-0.200pt]{2.409pt}{0.400pt}}
\put(371.0,725.0){\rule[-0.200pt]{2.409pt}{0.400pt}}
\put(1429.0,725.0){\rule[-0.200pt]{2.409pt}{0.400pt}}
\put(371.0,725.0){\rule[-0.200pt]{2.409pt}{0.400pt}}
\put(1429.0,725.0){\rule[-0.200pt]{2.409pt}{0.400pt}}
\put(371.0,725.0){\rule[-0.200pt]{2.409pt}{0.400pt}}
\put(1429.0,725.0){\rule[-0.200pt]{2.409pt}{0.400pt}}
\put(371.0,725.0){\rule[-0.200pt]{2.409pt}{0.400pt}}
\put(1429.0,725.0){\rule[-0.200pt]{2.409pt}{0.400pt}}
\put(371.0,726.0){\rule[-0.200pt]{2.409pt}{0.400pt}}
\put(1429.0,726.0){\rule[-0.200pt]{2.409pt}{0.400pt}}
\put(371.0,726.0){\rule[-0.200pt]{2.409pt}{0.400pt}}
\put(1429.0,726.0){\rule[-0.200pt]{2.409pt}{0.400pt}}
\put(371.0,726.0){\rule[-0.200pt]{2.409pt}{0.400pt}}
\put(1429.0,726.0){\rule[-0.200pt]{2.409pt}{0.400pt}}
\put(371.0,726.0){\rule[-0.200pt]{2.409pt}{0.400pt}}
\put(1429.0,726.0){\rule[-0.200pt]{2.409pt}{0.400pt}}
\put(371.0,726.0){\rule[-0.200pt]{2.409pt}{0.400pt}}
\put(1429.0,726.0){\rule[-0.200pt]{2.409pt}{0.400pt}}
\put(371.0,726.0){\rule[-0.200pt]{2.409pt}{0.400pt}}
\put(1429.0,726.0){\rule[-0.200pt]{2.409pt}{0.400pt}}
\put(371.0,726.0){\rule[-0.200pt]{2.409pt}{0.400pt}}
\put(1429.0,726.0){\rule[-0.200pt]{2.409pt}{0.400pt}}
\put(371.0,726.0){\rule[-0.200pt]{2.409pt}{0.400pt}}
\put(1429.0,726.0){\rule[-0.200pt]{2.409pt}{0.400pt}}
\put(371.0,726.0){\rule[-0.200pt]{2.409pt}{0.400pt}}
\put(1429.0,726.0){\rule[-0.200pt]{2.409pt}{0.400pt}}
\put(371.0,726.0){\rule[-0.200pt]{2.409pt}{0.400pt}}
\put(1429.0,726.0){\rule[-0.200pt]{2.409pt}{0.400pt}}
\put(371.0,726.0){\rule[-0.200pt]{2.409pt}{0.400pt}}
\put(1429.0,726.0){\rule[-0.200pt]{2.409pt}{0.400pt}}
\put(371.0,726.0){\rule[-0.200pt]{2.409pt}{0.400pt}}
\put(1429.0,726.0){\rule[-0.200pt]{2.409pt}{0.400pt}}
\put(371.0,726.0){\rule[-0.200pt]{2.409pt}{0.400pt}}
\put(1429.0,726.0){\rule[-0.200pt]{2.409pt}{0.400pt}}
\put(371.0,726.0){\rule[-0.200pt]{2.409pt}{0.400pt}}
\put(1429.0,726.0){\rule[-0.200pt]{2.409pt}{0.400pt}}
\put(371.0,726.0){\rule[-0.200pt]{2.409pt}{0.400pt}}
\put(1429.0,726.0){\rule[-0.200pt]{2.409pt}{0.400pt}}
\put(371.0,727.0){\rule[-0.200pt]{2.409pt}{0.400pt}}
\put(1429.0,727.0){\rule[-0.200pt]{2.409pt}{0.400pt}}
\put(371.0,727.0){\rule[-0.200pt]{2.409pt}{0.400pt}}
\put(1429.0,727.0){\rule[-0.200pt]{2.409pt}{0.400pt}}
\put(371.0,727.0){\rule[-0.200pt]{2.409pt}{0.400pt}}
\put(1429.0,727.0){\rule[-0.200pt]{2.409pt}{0.400pt}}
\put(371.0,727.0){\rule[-0.200pt]{2.409pt}{0.400pt}}
\put(1429.0,727.0){\rule[-0.200pt]{2.409pt}{0.400pt}}
\put(371.0,727.0){\rule[-0.200pt]{2.409pt}{0.400pt}}
\put(1429.0,727.0){\rule[-0.200pt]{2.409pt}{0.400pt}}
\put(371.0,727.0){\rule[-0.200pt]{2.409pt}{0.400pt}}
\put(1429.0,727.0){\rule[-0.200pt]{2.409pt}{0.400pt}}
\put(371.0,727.0){\rule[-0.200pt]{2.409pt}{0.400pt}}
\put(1429.0,727.0){\rule[-0.200pt]{2.409pt}{0.400pt}}
\put(371.0,727.0){\rule[-0.200pt]{2.409pt}{0.400pt}}
\put(1429.0,727.0){\rule[-0.200pt]{2.409pt}{0.400pt}}
\put(371.0,727.0){\rule[-0.200pt]{2.409pt}{0.400pt}}
\put(1429.0,727.0){\rule[-0.200pt]{2.409pt}{0.400pt}}
\put(371.0,727.0){\rule[-0.200pt]{2.409pt}{0.400pt}}
\put(1429.0,727.0){\rule[-0.200pt]{2.409pt}{0.400pt}}
\put(371.0,727.0){\rule[-0.200pt]{2.409pt}{0.400pt}}
\put(1429.0,727.0){\rule[-0.200pt]{2.409pt}{0.400pt}}
\put(371.0,727.0){\rule[-0.200pt]{2.409pt}{0.400pt}}
\put(1429.0,727.0){\rule[-0.200pt]{2.409pt}{0.400pt}}
\put(371.0,727.0){\rule[-0.200pt]{2.409pt}{0.400pt}}
\put(1429.0,727.0){\rule[-0.200pt]{2.409pt}{0.400pt}}
\put(371.0,727.0){\rule[-0.200pt]{2.409pt}{0.400pt}}
\put(1429.0,727.0){\rule[-0.200pt]{2.409pt}{0.400pt}}
\put(371.0,727.0){\rule[-0.200pt]{2.409pt}{0.400pt}}
\put(1429.0,727.0){\rule[-0.200pt]{2.409pt}{0.400pt}}
\put(371.0,728.0){\rule[-0.200pt]{2.409pt}{0.400pt}}
\put(1429.0,728.0){\rule[-0.200pt]{2.409pt}{0.400pt}}
\put(371.0,728.0){\rule[-0.200pt]{2.409pt}{0.400pt}}
\put(1429.0,728.0){\rule[-0.200pt]{2.409pt}{0.400pt}}
\put(371.0,728.0){\rule[-0.200pt]{2.409pt}{0.400pt}}
\put(1429.0,728.0){\rule[-0.200pt]{2.409pt}{0.400pt}}
\put(371.0,728.0){\rule[-0.200pt]{2.409pt}{0.400pt}}
\put(1429.0,728.0){\rule[-0.200pt]{2.409pt}{0.400pt}}
\put(371.0,728.0){\rule[-0.200pt]{2.409pt}{0.400pt}}
\put(1429.0,728.0){\rule[-0.200pt]{2.409pt}{0.400pt}}
\put(371.0,728.0){\rule[-0.200pt]{2.409pt}{0.400pt}}
\put(1429.0,728.0){\rule[-0.200pt]{2.409pt}{0.400pt}}
\put(371.0,728.0){\rule[-0.200pt]{2.409pt}{0.400pt}}
\put(1429.0,728.0){\rule[-0.200pt]{2.409pt}{0.400pt}}
\put(371.0,728.0){\rule[-0.200pt]{2.409pt}{0.400pt}}
\put(1429.0,728.0){\rule[-0.200pt]{2.409pt}{0.400pt}}
\put(371.0,728.0){\rule[-0.200pt]{2.409pt}{0.400pt}}
\put(1429.0,728.0){\rule[-0.200pt]{2.409pt}{0.400pt}}
\put(371.0,728.0){\rule[-0.200pt]{2.409pt}{0.400pt}}
\put(1429.0,728.0){\rule[-0.200pt]{2.409pt}{0.400pt}}
\put(371.0,728.0){\rule[-0.200pt]{2.409pt}{0.400pt}}
\put(1429.0,728.0){\rule[-0.200pt]{2.409pt}{0.400pt}}
\put(371.0,728.0){\rule[-0.200pt]{2.409pt}{0.400pt}}
\put(1429.0,728.0){\rule[-0.200pt]{2.409pt}{0.400pt}}
\put(371.0,728.0){\rule[-0.200pt]{2.409pt}{0.400pt}}
\put(1429.0,728.0){\rule[-0.200pt]{2.409pt}{0.400pt}}
\put(371.0,728.0){\rule[-0.200pt]{2.409pt}{0.400pt}}
\put(1429.0,728.0){\rule[-0.200pt]{2.409pt}{0.400pt}}
\put(371.0,728.0){\rule[-0.200pt]{2.409pt}{0.400pt}}
\put(1429.0,728.0){\rule[-0.200pt]{2.409pt}{0.400pt}}
\put(371.0,728.0){\rule[-0.200pt]{2.409pt}{0.400pt}}
\put(1429.0,728.0){\rule[-0.200pt]{2.409pt}{0.400pt}}
\put(371.0,728.0){\rule[-0.200pt]{2.409pt}{0.400pt}}
\put(1429.0,728.0){\rule[-0.200pt]{2.409pt}{0.400pt}}
\put(371.0,729.0){\rule[-0.200pt]{2.409pt}{0.400pt}}
\put(1429.0,729.0){\rule[-0.200pt]{2.409pt}{0.400pt}}
\put(371.0,729.0){\rule[-0.200pt]{2.409pt}{0.400pt}}
\put(1429.0,729.0){\rule[-0.200pt]{2.409pt}{0.400pt}}
\put(371.0,729.0){\rule[-0.200pt]{2.409pt}{0.400pt}}
\put(1429.0,729.0){\rule[-0.200pt]{2.409pt}{0.400pt}}
\put(371.0,729.0){\rule[-0.200pt]{2.409pt}{0.400pt}}
\put(1429.0,729.0){\rule[-0.200pt]{2.409pt}{0.400pt}}
\put(371.0,729.0){\rule[-0.200pt]{2.409pt}{0.400pt}}
\put(1429.0,729.0){\rule[-0.200pt]{2.409pt}{0.400pt}}
\put(371.0,729.0){\rule[-0.200pt]{2.409pt}{0.400pt}}
\put(1429.0,729.0){\rule[-0.200pt]{2.409pt}{0.400pt}}
\put(371.0,729.0){\rule[-0.200pt]{2.409pt}{0.400pt}}
\put(1429.0,729.0){\rule[-0.200pt]{2.409pt}{0.400pt}}
\put(371.0,729.0){\rule[-0.200pt]{2.409pt}{0.400pt}}
\put(1429.0,729.0){\rule[-0.200pt]{2.409pt}{0.400pt}}
\put(371.0,729.0){\rule[-0.200pt]{2.409pt}{0.400pt}}
\put(1429.0,729.0){\rule[-0.200pt]{2.409pt}{0.400pt}}
\put(371.0,729.0){\rule[-0.200pt]{2.409pt}{0.400pt}}
\put(1429.0,729.0){\rule[-0.200pt]{2.409pt}{0.400pt}}
\put(371.0,729.0){\rule[-0.200pt]{2.409pt}{0.400pt}}
\put(1429.0,729.0){\rule[-0.200pt]{2.409pt}{0.400pt}}
\put(371.0,729.0){\rule[-0.200pt]{2.409pt}{0.400pt}}
\put(1429.0,729.0){\rule[-0.200pt]{2.409pt}{0.400pt}}
\put(371.0,729.0){\rule[-0.200pt]{2.409pt}{0.400pt}}
\put(1429.0,729.0){\rule[-0.200pt]{2.409pt}{0.400pt}}
\put(371.0,729.0){\rule[-0.200pt]{2.409pt}{0.400pt}}
\put(1429.0,729.0){\rule[-0.200pt]{2.409pt}{0.400pt}}
\put(371.0,729.0){\rule[-0.200pt]{2.409pt}{0.400pt}}
\put(1429.0,729.0){\rule[-0.200pt]{2.409pt}{0.400pt}}
\put(371.0,729.0){\rule[-0.200pt]{2.409pt}{0.400pt}}
\put(1429.0,729.0){\rule[-0.200pt]{2.409pt}{0.400pt}}
\put(371.0,729.0){\rule[-0.200pt]{2.409pt}{0.400pt}}
\put(1429.0,729.0){\rule[-0.200pt]{2.409pt}{0.400pt}}
\put(371.0,730.0){\rule[-0.200pt]{4.818pt}{0.400pt}}
\put(351,730){\makebox(0,0)[r]{ 1e+12}}
\put(1419.0,730.0){\rule[-0.200pt]{4.818pt}{0.400pt}}
\put(371.0,742.0){\rule[-0.200pt]{2.409pt}{0.400pt}}
\put(1429.0,742.0){\rule[-0.200pt]{2.409pt}{0.400pt}}
\put(371.0,760.0){\rule[-0.200pt]{2.409pt}{0.400pt}}
\put(1429.0,760.0){\rule[-0.200pt]{2.409pt}{0.400pt}}
\put(371.0,768.0){\rule[-0.200pt]{2.409pt}{0.400pt}}
\put(1429.0,768.0){\rule[-0.200pt]{2.409pt}{0.400pt}}
\put(371.0,774.0){\rule[-0.200pt]{2.409pt}{0.400pt}}
\put(1429.0,774.0){\rule[-0.200pt]{2.409pt}{0.400pt}}
\put(371.0,779.0){\rule[-0.200pt]{2.409pt}{0.400pt}}
\put(1429.0,779.0){\rule[-0.200pt]{2.409pt}{0.400pt}}
\put(371.0,783.0){\rule[-0.200pt]{2.409pt}{0.400pt}}
\put(1429.0,783.0){\rule[-0.200pt]{2.409pt}{0.400pt}}
\put(371.0,786.0){\rule[-0.200pt]{2.409pt}{0.400pt}}
\put(1429.0,786.0){\rule[-0.200pt]{2.409pt}{0.400pt}}
\put(371.0,788.0){\rule[-0.200pt]{2.409pt}{0.400pt}}
\put(1429.0,788.0){\rule[-0.200pt]{2.409pt}{0.400pt}}
\put(371.0,791.0){\rule[-0.200pt]{2.409pt}{0.400pt}}
\put(1429.0,791.0){\rule[-0.200pt]{2.409pt}{0.400pt}}
\put(371.0,793.0){\rule[-0.200pt]{2.409pt}{0.400pt}}
\put(1429.0,793.0){\rule[-0.200pt]{2.409pt}{0.400pt}}
\put(371.0,794.0){\rule[-0.200pt]{2.409pt}{0.400pt}}
\put(1429.0,794.0){\rule[-0.200pt]{2.409pt}{0.400pt}}
\put(371.0,796.0){\rule[-0.200pt]{2.409pt}{0.400pt}}
\put(1429.0,796.0){\rule[-0.200pt]{2.409pt}{0.400pt}}
\put(371.0,798.0){\rule[-0.200pt]{2.409pt}{0.400pt}}
\put(1429.0,798.0){\rule[-0.200pt]{2.409pt}{0.400pt}}
\put(371.0,799.0){\rule[-0.200pt]{2.409pt}{0.400pt}}
\put(1429.0,799.0){\rule[-0.200pt]{2.409pt}{0.400pt}}
\put(371.0,800.0){\rule[-0.200pt]{2.409pt}{0.400pt}}
\put(1429.0,800.0){\rule[-0.200pt]{2.409pt}{0.400pt}}
\put(371.0,802.0){\rule[-0.200pt]{2.409pt}{0.400pt}}
\put(1429.0,802.0){\rule[-0.200pt]{2.409pt}{0.400pt}}
\put(371.0,803.0){\rule[-0.200pt]{2.409pt}{0.400pt}}
\put(1429.0,803.0){\rule[-0.200pt]{2.409pt}{0.400pt}}
\put(371.0,804.0){\rule[-0.200pt]{2.409pt}{0.400pt}}
\put(1429.0,804.0){\rule[-0.200pt]{2.409pt}{0.400pt}}
\put(371.0,805.0){\rule[-0.200pt]{2.409pt}{0.400pt}}
\put(1429.0,805.0){\rule[-0.200pt]{2.409pt}{0.400pt}}
\put(371.0,806.0){\rule[-0.200pt]{2.409pt}{0.400pt}}
\put(1429.0,806.0){\rule[-0.200pt]{2.409pt}{0.400pt}}
\put(371.0,807.0){\rule[-0.200pt]{2.409pt}{0.400pt}}
\put(1429.0,807.0){\rule[-0.200pt]{2.409pt}{0.400pt}}
\put(371.0,808.0){\rule[-0.200pt]{2.409pt}{0.400pt}}
\put(1429.0,808.0){\rule[-0.200pt]{2.409pt}{0.400pt}}
\put(371.0,809.0){\rule[-0.200pt]{2.409pt}{0.400pt}}
\put(1429.0,809.0){\rule[-0.200pt]{2.409pt}{0.400pt}}
\put(371.0,809.0){\rule[-0.200pt]{2.409pt}{0.400pt}}
\put(1429.0,809.0){\rule[-0.200pt]{2.409pt}{0.400pt}}
\put(371.0,810.0){\rule[-0.200pt]{2.409pt}{0.400pt}}
\put(1429.0,810.0){\rule[-0.200pt]{2.409pt}{0.400pt}}
\put(371.0,811.0){\rule[-0.200pt]{2.409pt}{0.400pt}}
\put(1429.0,811.0){\rule[-0.200pt]{2.409pt}{0.400pt}}
\put(371.0,812.0){\rule[-0.200pt]{2.409pt}{0.400pt}}
\put(1429.0,812.0){\rule[-0.200pt]{2.409pt}{0.400pt}}
\put(371.0,812.0){\rule[-0.200pt]{2.409pt}{0.400pt}}
\put(1429.0,812.0){\rule[-0.200pt]{2.409pt}{0.400pt}}
\put(371.0,813.0){\rule[-0.200pt]{2.409pt}{0.400pt}}
\put(1429.0,813.0){\rule[-0.200pt]{2.409pt}{0.400pt}}
\put(371.0,814.0){\rule[-0.200pt]{2.409pt}{0.400pt}}
\put(1429.0,814.0){\rule[-0.200pt]{2.409pt}{0.400pt}}
\put(371.0,814.0){\rule[-0.200pt]{2.409pt}{0.400pt}}
\put(1429.0,814.0){\rule[-0.200pt]{2.409pt}{0.400pt}}
\put(371.0,815.0){\rule[-0.200pt]{2.409pt}{0.400pt}}
\put(1429.0,815.0){\rule[-0.200pt]{2.409pt}{0.400pt}}
\put(371.0,815.0){\rule[-0.200pt]{2.409pt}{0.400pt}}
\put(1429.0,815.0){\rule[-0.200pt]{2.409pt}{0.400pt}}
\put(371.0,816.0){\rule[-0.200pt]{2.409pt}{0.400pt}}
\put(1429.0,816.0){\rule[-0.200pt]{2.409pt}{0.400pt}}
\put(371.0,817.0){\rule[-0.200pt]{2.409pt}{0.400pt}}
\put(1429.0,817.0){\rule[-0.200pt]{2.409pt}{0.400pt}}
\put(371.0,817.0){\rule[-0.200pt]{2.409pt}{0.400pt}}
\put(1429.0,817.0){\rule[-0.200pt]{2.409pt}{0.400pt}}
\put(371.0,818.0){\rule[-0.200pt]{2.409pt}{0.400pt}}
\put(1429.0,818.0){\rule[-0.200pt]{2.409pt}{0.400pt}}
\put(371.0,818.0){\rule[-0.200pt]{2.409pt}{0.400pt}}
\put(1429.0,818.0){\rule[-0.200pt]{2.409pt}{0.400pt}}
\put(371.0,819.0){\rule[-0.200pt]{2.409pt}{0.400pt}}
\put(1429.0,819.0){\rule[-0.200pt]{2.409pt}{0.400pt}}
\put(371.0,819.0){\rule[-0.200pt]{2.409pt}{0.400pt}}
\put(1429.0,819.0){\rule[-0.200pt]{2.409pt}{0.400pt}}
\put(371.0,820.0){\rule[-0.200pt]{2.409pt}{0.400pt}}
\put(1429.0,820.0){\rule[-0.200pt]{2.409pt}{0.400pt}}
\put(371.0,820.0){\rule[-0.200pt]{2.409pt}{0.400pt}}
\put(1429.0,820.0){\rule[-0.200pt]{2.409pt}{0.400pt}}
\put(371.0,820.0){\rule[-0.200pt]{2.409pt}{0.400pt}}
\put(1429.0,820.0){\rule[-0.200pt]{2.409pt}{0.400pt}}
\put(371.0,821.0){\rule[-0.200pt]{2.409pt}{0.400pt}}
\put(1429.0,821.0){\rule[-0.200pt]{2.409pt}{0.400pt}}
\put(371.0,821.0){\rule[-0.200pt]{2.409pt}{0.400pt}}
\put(1429.0,821.0){\rule[-0.200pt]{2.409pt}{0.400pt}}
\put(371.0,822.0){\rule[-0.200pt]{2.409pt}{0.400pt}}
\put(1429.0,822.0){\rule[-0.200pt]{2.409pt}{0.400pt}}
\put(371.0,822.0){\rule[-0.200pt]{2.409pt}{0.400pt}}
\put(1429.0,822.0){\rule[-0.200pt]{2.409pt}{0.400pt}}
\put(371.0,823.0){\rule[-0.200pt]{2.409pt}{0.400pt}}
\put(1429.0,823.0){\rule[-0.200pt]{2.409pt}{0.400pt}}
\put(371.0,823.0){\rule[-0.200pt]{2.409pt}{0.400pt}}
\put(1429.0,823.0){\rule[-0.200pt]{2.409pt}{0.400pt}}
\put(371.0,823.0){\rule[-0.200pt]{2.409pt}{0.400pt}}
\put(1429.0,823.0){\rule[-0.200pt]{2.409pt}{0.400pt}}
\put(371.0,824.0){\rule[-0.200pt]{2.409pt}{0.400pt}}
\put(1429.0,824.0){\rule[-0.200pt]{2.409pt}{0.400pt}}
\put(371.0,824.0){\rule[-0.200pt]{2.409pt}{0.400pt}}
\put(1429.0,824.0){\rule[-0.200pt]{2.409pt}{0.400pt}}
\put(371.0,824.0){\rule[-0.200pt]{2.409pt}{0.400pt}}
\put(1429.0,824.0){\rule[-0.200pt]{2.409pt}{0.400pt}}
\put(371.0,825.0){\rule[-0.200pt]{2.409pt}{0.400pt}}
\put(1429.0,825.0){\rule[-0.200pt]{2.409pt}{0.400pt}}
\put(371.0,825.0){\rule[-0.200pt]{2.409pt}{0.400pt}}
\put(1429.0,825.0){\rule[-0.200pt]{2.409pt}{0.400pt}}
\put(371.0,825.0){\rule[-0.200pt]{2.409pt}{0.400pt}}
\put(1429.0,825.0){\rule[-0.200pt]{2.409pt}{0.400pt}}
\put(371.0,826.0){\rule[-0.200pt]{2.409pt}{0.400pt}}
\put(1429.0,826.0){\rule[-0.200pt]{2.409pt}{0.400pt}}
\put(371.0,826.0){\rule[-0.200pt]{2.409pt}{0.400pt}}
\put(1429.0,826.0){\rule[-0.200pt]{2.409pt}{0.400pt}}
\put(371.0,826.0){\rule[-0.200pt]{2.409pt}{0.400pt}}
\put(1429.0,826.0){\rule[-0.200pt]{2.409pt}{0.400pt}}
\put(371.0,827.0){\rule[-0.200pt]{2.409pt}{0.400pt}}
\put(1429.0,827.0){\rule[-0.200pt]{2.409pt}{0.400pt}}
\put(371.0,827.0){\rule[-0.200pt]{2.409pt}{0.400pt}}
\put(1429.0,827.0){\rule[-0.200pt]{2.409pt}{0.400pt}}
\put(371.0,827.0){\rule[-0.200pt]{2.409pt}{0.400pt}}
\put(1429.0,827.0){\rule[-0.200pt]{2.409pt}{0.400pt}}
\put(371.0,828.0){\rule[-0.200pt]{2.409pt}{0.400pt}}
\put(1429.0,828.0){\rule[-0.200pt]{2.409pt}{0.400pt}}
\put(371.0,828.0){\rule[-0.200pt]{2.409pt}{0.400pt}}
\put(1429.0,828.0){\rule[-0.200pt]{2.409pt}{0.400pt}}
\put(371.0,828.0){\rule[-0.200pt]{2.409pt}{0.400pt}}
\put(1429.0,828.0){\rule[-0.200pt]{2.409pt}{0.400pt}}
\put(371.0,829.0){\rule[-0.200pt]{2.409pt}{0.400pt}}
\put(1429.0,829.0){\rule[-0.200pt]{2.409pt}{0.400pt}}
\put(371.0,829.0){\rule[-0.200pt]{2.409pt}{0.400pt}}
\put(1429.0,829.0){\rule[-0.200pt]{2.409pt}{0.400pt}}
\put(371.0,829.0){\rule[-0.200pt]{2.409pt}{0.400pt}}
\put(1429.0,829.0){\rule[-0.200pt]{2.409pt}{0.400pt}}
\put(371.0,829.0){\rule[-0.200pt]{2.409pt}{0.400pt}}
\put(1429.0,829.0){\rule[-0.200pt]{2.409pt}{0.400pt}}
\put(371.0,830.0){\rule[-0.200pt]{2.409pt}{0.400pt}}
\put(1429.0,830.0){\rule[-0.200pt]{2.409pt}{0.400pt}}
\put(371.0,830.0){\rule[-0.200pt]{2.409pt}{0.400pt}}
\put(1429.0,830.0){\rule[-0.200pt]{2.409pt}{0.400pt}}
\put(371.0,830.0){\rule[-0.200pt]{2.409pt}{0.400pt}}
\put(1429.0,830.0){\rule[-0.200pt]{2.409pt}{0.400pt}}
\put(371.0,830.0){\rule[-0.200pt]{2.409pt}{0.400pt}}
\put(1429.0,830.0){\rule[-0.200pt]{2.409pt}{0.400pt}}
\put(371.0,831.0){\rule[-0.200pt]{2.409pt}{0.400pt}}
\put(1429.0,831.0){\rule[-0.200pt]{2.409pt}{0.400pt}}
\put(371.0,831.0){\rule[-0.200pt]{2.409pt}{0.400pt}}
\put(1429.0,831.0){\rule[-0.200pt]{2.409pt}{0.400pt}}
\put(371.0,831.0){\rule[-0.200pt]{2.409pt}{0.400pt}}
\put(1429.0,831.0){\rule[-0.200pt]{2.409pt}{0.400pt}}
\put(371.0,831.0){\rule[-0.200pt]{2.409pt}{0.400pt}}
\put(1429.0,831.0){\rule[-0.200pt]{2.409pt}{0.400pt}}
\put(371.0,832.0){\rule[-0.200pt]{2.409pt}{0.400pt}}
\put(1429.0,832.0){\rule[-0.200pt]{2.409pt}{0.400pt}}
\put(371.0,832.0){\rule[-0.200pt]{2.409pt}{0.400pt}}
\put(1429.0,832.0){\rule[-0.200pt]{2.409pt}{0.400pt}}
\put(371.0,832.0){\rule[-0.200pt]{2.409pt}{0.400pt}}
\put(1429.0,832.0){\rule[-0.200pt]{2.409pt}{0.400pt}}
\put(371.0,832.0){\rule[-0.200pt]{2.409pt}{0.400pt}}
\put(1429.0,832.0){\rule[-0.200pt]{2.409pt}{0.400pt}}
\put(371.0,833.0){\rule[-0.200pt]{2.409pt}{0.400pt}}
\put(1429.0,833.0){\rule[-0.200pt]{2.409pt}{0.400pt}}
\put(371.0,833.0){\rule[-0.200pt]{2.409pt}{0.400pt}}
\put(1429.0,833.0){\rule[-0.200pt]{2.409pt}{0.400pt}}
\put(371.0,833.0){\rule[-0.200pt]{2.409pt}{0.400pt}}
\put(1429.0,833.0){\rule[-0.200pt]{2.409pt}{0.400pt}}
\put(371.0,833.0){\rule[-0.200pt]{2.409pt}{0.400pt}}
\put(1429.0,833.0){\rule[-0.200pt]{2.409pt}{0.400pt}}
\put(371.0,834.0){\rule[-0.200pt]{2.409pt}{0.400pt}}
\put(1429.0,834.0){\rule[-0.200pt]{2.409pt}{0.400pt}}
\put(371.0,834.0){\rule[-0.200pt]{2.409pt}{0.400pt}}
\put(1429.0,834.0){\rule[-0.200pt]{2.409pt}{0.400pt}}
\put(371.0,834.0){\rule[-0.200pt]{2.409pt}{0.400pt}}
\put(1429.0,834.0){\rule[-0.200pt]{2.409pt}{0.400pt}}
\put(371.0,834.0){\rule[-0.200pt]{2.409pt}{0.400pt}}
\put(1429.0,834.0){\rule[-0.200pt]{2.409pt}{0.400pt}}
\put(371.0,834.0){\rule[-0.200pt]{2.409pt}{0.400pt}}
\put(1429.0,834.0){\rule[-0.200pt]{2.409pt}{0.400pt}}
\put(371.0,835.0){\rule[-0.200pt]{2.409pt}{0.400pt}}
\put(1429.0,835.0){\rule[-0.200pt]{2.409pt}{0.400pt}}
\put(371.0,835.0){\rule[-0.200pt]{2.409pt}{0.400pt}}
\put(1429.0,835.0){\rule[-0.200pt]{2.409pt}{0.400pt}}
\put(371.0,835.0){\rule[-0.200pt]{2.409pt}{0.400pt}}
\put(1429.0,835.0){\rule[-0.200pt]{2.409pt}{0.400pt}}
\put(371.0,835.0){\rule[-0.200pt]{2.409pt}{0.400pt}}
\put(1429.0,835.0){\rule[-0.200pt]{2.409pt}{0.400pt}}
\put(371.0,835.0){\rule[-0.200pt]{2.409pt}{0.400pt}}
\put(1429.0,835.0){\rule[-0.200pt]{2.409pt}{0.400pt}}
\put(371.0,836.0){\rule[-0.200pt]{2.409pt}{0.400pt}}
\put(1429.0,836.0){\rule[-0.200pt]{2.409pt}{0.400pt}}
\put(371.0,836.0){\rule[-0.200pt]{2.409pt}{0.400pt}}
\put(1429.0,836.0){\rule[-0.200pt]{2.409pt}{0.400pt}}
\put(371.0,836.0){\rule[-0.200pt]{2.409pt}{0.400pt}}
\put(1429.0,836.0){\rule[-0.200pt]{2.409pt}{0.400pt}}
\put(371.0,836.0){\rule[-0.200pt]{2.409pt}{0.400pt}}
\put(1429.0,836.0){\rule[-0.200pt]{2.409pt}{0.400pt}}
\put(371.0,836.0){\rule[-0.200pt]{2.409pt}{0.400pt}}
\put(1429.0,836.0){\rule[-0.200pt]{2.409pt}{0.400pt}}
\put(371.0,837.0){\rule[-0.200pt]{2.409pt}{0.400pt}}
\put(1429.0,837.0){\rule[-0.200pt]{2.409pt}{0.400pt}}
\put(371.0,837.0){\rule[-0.200pt]{2.409pt}{0.400pt}}
\put(1429.0,837.0){\rule[-0.200pt]{2.409pt}{0.400pt}}
\put(371.0,837.0){\rule[-0.200pt]{2.409pt}{0.400pt}}
\put(1429.0,837.0){\rule[-0.200pt]{2.409pt}{0.400pt}}
\put(371.0,837.0){\rule[-0.200pt]{2.409pt}{0.400pt}}
\put(1429.0,837.0){\rule[-0.200pt]{2.409pt}{0.400pt}}
\put(371.0,837.0){\rule[-0.200pt]{2.409pt}{0.400pt}}
\put(1429.0,837.0){\rule[-0.200pt]{2.409pt}{0.400pt}}
\put(371.0,837.0){\rule[-0.200pt]{2.409pt}{0.400pt}}
\put(1429.0,837.0){\rule[-0.200pt]{2.409pt}{0.400pt}}
\put(371.0,838.0){\rule[-0.200pt]{2.409pt}{0.400pt}}
\put(1429.0,838.0){\rule[-0.200pt]{2.409pt}{0.400pt}}
\put(371.0,838.0){\rule[-0.200pt]{2.409pt}{0.400pt}}
\put(1429.0,838.0){\rule[-0.200pt]{2.409pt}{0.400pt}}
\put(371.0,838.0){\rule[-0.200pt]{2.409pt}{0.400pt}}
\put(1429.0,838.0){\rule[-0.200pt]{2.409pt}{0.400pt}}
\put(371.0,838.0){\rule[-0.200pt]{2.409pt}{0.400pt}}
\put(1429.0,838.0){\rule[-0.200pt]{2.409pt}{0.400pt}}
\put(371.0,838.0){\rule[-0.200pt]{2.409pt}{0.400pt}}
\put(1429.0,838.0){\rule[-0.200pt]{2.409pt}{0.400pt}}
\put(371.0,838.0){\rule[-0.200pt]{2.409pt}{0.400pt}}
\put(1429.0,838.0){\rule[-0.200pt]{2.409pt}{0.400pt}}
\put(371.0,839.0){\rule[-0.200pt]{2.409pt}{0.400pt}}
\put(1429.0,839.0){\rule[-0.200pt]{2.409pt}{0.400pt}}
\put(371.0,839.0){\rule[-0.200pt]{2.409pt}{0.400pt}}
\put(1429.0,839.0){\rule[-0.200pt]{2.409pt}{0.400pt}}
\put(371.0,839.0){\rule[-0.200pt]{2.409pt}{0.400pt}}
\put(1429.0,839.0){\rule[-0.200pt]{2.409pt}{0.400pt}}
\put(371.0,839.0){\rule[-0.200pt]{2.409pt}{0.400pt}}
\put(1429.0,839.0){\rule[-0.200pt]{2.409pt}{0.400pt}}
\put(371.0,839.0){\rule[-0.200pt]{2.409pt}{0.400pt}}
\put(1429.0,839.0){\rule[-0.200pt]{2.409pt}{0.400pt}}
\put(371.0,839.0){\rule[-0.200pt]{2.409pt}{0.400pt}}
\put(1429.0,839.0){\rule[-0.200pt]{2.409pt}{0.400pt}}
\put(371.0,840.0){\rule[-0.200pt]{2.409pt}{0.400pt}}
\put(1429.0,840.0){\rule[-0.200pt]{2.409pt}{0.400pt}}
\put(371.0,840.0){\rule[-0.200pt]{2.409pt}{0.400pt}}
\put(1429.0,840.0){\rule[-0.200pt]{2.409pt}{0.400pt}}
\put(371.0,840.0){\rule[-0.200pt]{2.409pt}{0.400pt}}
\put(1429.0,840.0){\rule[-0.200pt]{2.409pt}{0.400pt}}
\put(371.0,840.0){\rule[-0.200pt]{2.409pt}{0.400pt}}
\put(1429.0,840.0){\rule[-0.200pt]{2.409pt}{0.400pt}}
\put(371.0,840.0){\rule[-0.200pt]{2.409pt}{0.400pt}}
\put(1429.0,840.0){\rule[-0.200pt]{2.409pt}{0.400pt}}
\put(371.0,840.0){\rule[-0.200pt]{2.409pt}{0.400pt}}
\put(1429.0,840.0){\rule[-0.200pt]{2.409pt}{0.400pt}}
\put(371.0,841.0){\rule[-0.200pt]{2.409pt}{0.400pt}}
\put(1429.0,841.0){\rule[-0.200pt]{2.409pt}{0.400pt}}
\put(371.0,841.0){\rule[-0.200pt]{2.409pt}{0.400pt}}
\put(1429.0,841.0){\rule[-0.200pt]{2.409pt}{0.400pt}}
\put(371.0,841.0){\rule[-0.200pt]{2.409pt}{0.400pt}}
\put(1429.0,841.0){\rule[-0.200pt]{2.409pt}{0.400pt}}
\put(371.0,841.0){\rule[-0.200pt]{2.409pt}{0.400pt}}
\put(1429.0,841.0){\rule[-0.200pt]{2.409pt}{0.400pt}}
\put(371.0,841.0){\rule[-0.200pt]{2.409pt}{0.400pt}}
\put(1429.0,841.0){\rule[-0.200pt]{2.409pt}{0.400pt}}
\put(371.0,841.0){\rule[-0.200pt]{2.409pt}{0.400pt}}
\put(1429.0,841.0){\rule[-0.200pt]{2.409pt}{0.400pt}}
\put(371.0,841.0){\rule[-0.200pt]{2.409pt}{0.400pt}}
\put(1429.0,841.0){\rule[-0.200pt]{2.409pt}{0.400pt}}
\put(371.0,842.0){\rule[-0.200pt]{2.409pt}{0.400pt}}
\put(1429.0,842.0){\rule[-0.200pt]{2.409pt}{0.400pt}}
\put(371.0,842.0){\rule[-0.200pt]{2.409pt}{0.400pt}}
\put(1429.0,842.0){\rule[-0.200pt]{2.409pt}{0.400pt}}
\put(371.0,842.0){\rule[-0.200pt]{2.409pt}{0.400pt}}
\put(1429.0,842.0){\rule[-0.200pt]{2.409pt}{0.400pt}}
\put(371.0,842.0){\rule[-0.200pt]{2.409pt}{0.400pt}}
\put(1429.0,842.0){\rule[-0.200pt]{2.409pt}{0.400pt}}
\put(371.0,842.0){\rule[-0.200pt]{2.409pt}{0.400pt}}
\put(1429.0,842.0){\rule[-0.200pt]{2.409pt}{0.400pt}}
\put(371.0,842.0){\rule[-0.200pt]{2.409pt}{0.400pt}}
\put(1429.0,842.0){\rule[-0.200pt]{2.409pt}{0.400pt}}
\put(371.0,842.0){\rule[-0.200pt]{2.409pt}{0.400pt}}
\put(1429.0,842.0){\rule[-0.200pt]{2.409pt}{0.400pt}}
\put(371.0,843.0){\rule[-0.200pt]{2.409pt}{0.400pt}}
\put(1429.0,843.0){\rule[-0.200pt]{2.409pt}{0.400pt}}
\put(371.0,843.0){\rule[-0.200pt]{2.409pt}{0.400pt}}
\put(1429.0,843.0){\rule[-0.200pt]{2.409pt}{0.400pt}}
\put(371.0,843.0){\rule[-0.200pt]{2.409pt}{0.400pt}}
\put(1429.0,843.0){\rule[-0.200pt]{2.409pt}{0.400pt}}
\put(371.0,843.0){\rule[-0.200pt]{2.409pt}{0.400pt}}
\put(1429.0,843.0){\rule[-0.200pt]{2.409pt}{0.400pt}}
\put(371.0,843.0){\rule[-0.200pt]{2.409pt}{0.400pt}}
\put(1429.0,843.0){\rule[-0.200pt]{2.409pt}{0.400pt}}
\put(371.0,843.0){\rule[-0.200pt]{2.409pt}{0.400pt}}
\put(1429.0,843.0){\rule[-0.200pt]{2.409pt}{0.400pt}}
\put(371.0,843.0){\rule[-0.200pt]{2.409pt}{0.400pt}}
\put(1429.0,843.0){\rule[-0.200pt]{2.409pt}{0.400pt}}
\put(371.0,843.0){\rule[-0.200pt]{2.409pt}{0.400pt}}
\put(1429.0,843.0){\rule[-0.200pt]{2.409pt}{0.400pt}}
\put(371.0,844.0){\rule[-0.200pt]{2.409pt}{0.400pt}}
\put(1429.0,844.0){\rule[-0.200pt]{2.409pt}{0.400pt}}
\put(371.0,844.0){\rule[-0.200pt]{2.409pt}{0.400pt}}
\put(1429.0,844.0){\rule[-0.200pt]{2.409pt}{0.400pt}}
\put(371.0,844.0){\rule[-0.200pt]{2.409pt}{0.400pt}}
\put(1429.0,844.0){\rule[-0.200pt]{2.409pt}{0.400pt}}
\put(371.0,844.0){\rule[-0.200pt]{2.409pt}{0.400pt}}
\put(1429.0,844.0){\rule[-0.200pt]{2.409pt}{0.400pt}}
\put(371.0,844.0){\rule[-0.200pt]{2.409pt}{0.400pt}}
\put(1429.0,844.0){\rule[-0.200pt]{2.409pt}{0.400pt}}
\put(371.0,844.0){\rule[-0.200pt]{2.409pt}{0.400pt}}
\put(1429.0,844.0){\rule[-0.200pt]{2.409pt}{0.400pt}}
\put(371.0,844.0){\rule[-0.200pt]{2.409pt}{0.400pt}}
\put(1429.0,844.0){\rule[-0.200pt]{2.409pt}{0.400pt}}
\put(371.0,844.0){\rule[-0.200pt]{2.409pt}{0.400pt}}
\put(1429.0,844.0){\rule[-0.200pt]{2.409pt}{0.400pt}}
\put(371.0,845.0){\rule[-0.200pt]{2.409pt}{0.400pt}}
\put(1429.0,845.0){\rule[-0.200pt]{2.409pt}{0.400pt}}
\put(371.0,845.0){\rule[-0.200pt]{2.409pt}{0.400pt}}
\put(1429.0,845.0){\rule[-0.200pt]{2.409pt}{0.400pt}}
\put(371.0,845.0){\rule[-0.200pt]{2.409pt}{0.400pt}}
\put(1429.0,845.0){\rule[-0.200pt]{2.409pt}{0.400pt}}
\put(371.0,845.0){\rule[-0.200pt]{2.409pt}{0.400pt}}
\put(1429.0,845.0){\rule[-0.200pt]{2.409pt}{0.400pt}}
\put(371.0,845.0){\rule[-0.200pt]{2.409pt}{0.400pt}}
\put(1429.0,845.0){\rule[-0.200pt]{2.409pt}{0.400pt}}
\put(371.0,845.0){\rule[-0.200pt]{2.409pt}{0.400pt}}
\put(1429.0,845.0){\rule[-0.200pt]{2.409pt}{0.400pt}}
\put(371.0,845.0){\rule[-0.200pt]{2.409pt}{0.400pt}}
\put(1429.0,845.0){\rule[-0.200pt]{2.409pt}{0.400pt}}
\put(371.0,845.0){\rule[-0.200pt]{2.409pt}{0.400pt}}
\put(1429.0,845.0){\rule[-0.200pt]{2.409pt}{0.400pt}}
\put(371.0,846.0){\rule[-0.200pt]{2.409pt}{0.400pt}}
\put(1429.0,846.0){\rule[-0.200pt]{2.409pt}{0.400pt}}
\put(371.0,846.0){\rule[-0.200pt]{2.409pt}{0.400pt}}
\put(1429.0,846.0){\rule[-0.200pt]{2.409pt}{0.400pt}}
\put(371.0,846.0){\rule[-0.200pt]{2.409pt}{0.400pt}}
\put(1429.0,846.0){\rule[-0.200pt]{2.409pt}{0.400pt}}
\put(371.0,846.0){\rule[-0.200pt]{2.409pt}{0.400pt}}
\put(1429.0,846.0){\rule[-0.200pt]{2.409pt}{0.400pt}}
\put(371.0,846.0){\rule[-0.200pt]{2.409pt}{0.400pt}}
\put(1429.0,846.0){\rule[-0.200pt]{2.409pt}{0.400pt}}
\put(371.0,846.0){\rule[-0.200pt]{2.409pt}{0.400pt}}
\put(1429.0,846.0){\rule[-0.200pt]{2.409pt}{0.400pt}}
\put(371.0,846.0){\rule[-0.200pt]{2.409pt}{0.400pt}}
\put(1429.0,846.0){\rule[-0.200pt]{2.409pt}{0.400pt}}
\put(371.0,846.0){\rule[-0.200pt]{2.409pt}{0.400pt}}
\put(1429.0,846.0){\rule[-0.200pt]{2.409pt}{0.400pt}}
\put(371.0,846.0){\rule[-0.200pt]{2.409pt}{0.400pt}}
\put(1429.0,846.0){\rule[-0.200pt]{2.409pt}{0.400pt}}
\put(371.0,847.0){\rule[-0.200pt]{2.409pt}{0.400pt}}
\put(1429.0,847.0){\rule[-0.200pt]{2.409pt}{0.400pt}}
\put(371.0,847.0){\rule[-0.200pt]{2.409pt}{0.400pt}}
\put(1429.0,847.0){\rule[-0.200pt]{2.409pt}{0.400pt}}
\put(371.0,847.0){\rule[-0.200pt]{2.409pt}{0.400pt}}
\put(1429.0,847.0){\rule[-0.200pt]{2.409pt}{0.400pt}}
\put(371.0,847.0){\rule[-0.200pt]{2.409pt}{0.400pt}}
\put(1429.0,847.0){\rule[-0.200pt]{2.409pt}{0.400pt}}
\put(371.0,847.0){\rule[-0.200pt]{2.409pt}{0.400pt}}
\put(1429.0,847.0){\rule[-0.200pt]{2.409pt}{0.400pt}}
\put(371.0,847.0){\rule[-0.200pt]{2.409pt}{0.400pt}}
\put(1429.0,847.0){\rule[-0.200pt]{2.409pt}{0.400pt}}
\put(371.0,847.0){\rule[-0.200pt]{2.409pt}{0.400pt}}
\put(1429.0,847.0){\rule[-0.200pt]{2.409pt}{0.400pt}}
\put(371.0,847.0){\rule[-0.200pt]{2.409pt}{0.400pt}}
\put(1429.0,847.0){\rule[-0.200pt]{2.409pt}{0.400pt}}
\put(371.0,847.0){\rule[-0.200pt]{2.409pt}{0.400pt}}
\put(1429.0,847.0){\rule[-0.200pt]{2.409pt}{0.400pt}}
\put(371.0,848.0){\rule[-0.200pt]{2.409pt}{0.400pt}}
\put(1429.0,848.0){\rule[-0.200pt]{2.409pt}{0.400pt}}
\put(371.0,848.0){\rule[-0.200pt]{2.409pt}{0.400pt}}
\put(1429.0,848.0){\rule[-0.200pt]{2.409pt}{0.400pt}}
\put(371.0,848.0){\rule[-0.200pt]{2.409pt}{0.400pt}}
\put(1429.0,848.0){\rule[-0.200pt]{2.409pt}{0.400pt}}
\put(371.0,848.0){\rule[-0.200pt]{2.409pt}{0.400pt}}
\put(1429.0,848.0){\rule[-0.200pt]{2.409pt}{0.400pt}}
\put(371.0,848.0){\rule[-0.200pt]{2.409pt}{0.400pt}}
\put(1429.0,848.0){\rule[-0.200pt]{2.409pt}{0.400pt}}
\put(371.0,848.0){\rule[-0.200pt]{2.409pt}{0.400pt}}
\put(1429.0,848.0){\rule[-0.200pt]{2.409pt}{0.400pt}}
\put(371.0,848.0){\rule[-0.200pt]{2.409pt}{0.400pt}}
\put(1429.0,848.0){\rule[-0.200pt]{2.409pt}{0.400pt}}
\put(371.0,848.0){\rule[-0.200pt]{2.409pt}{0.400pt}}
\put(1429.0,848.0){\rule[-0.200pt]{2.409pt}{0.400pt}}
\put(371.0,848.0){\rule[-0.200pt]{2.409pt}{0.400pt}}
\put(1429.0,848.0){\rule[-0.200pt]{2.409pt}{0.400pt}}
\put(371.0,848.0){\rule[-0.200pt]{2.409pt}{0.400pt}}
\put(1429.0,848.0){\rule[-0.200pt]{2.409pt}{0.400pt}}
\put(371.0,849.0){\rule[-0.200pt]{2.409pt}{0.400pt}}
\put(1429.0,849.0){\rule[-0.200pt]{2.409pt}{0.400pt}}
\put(371.0,849.0){\rule[-0.200pt]{2.409pt}{0.400pt}}
\put(1429.0,849.0){\rule[-0.200pt]{2.409pt}{0.400pt}}
\put(371.0,849.0){\rule[-0.200pt]{2.409pt}{0.400pt}}
\put(1429.0,849.0){\rule[-0.200pt]{2.409pt}{0.400pt}}
\put(371.0,849.0){\rule[-0.200pt]{2.409pt}{0.400pt}}
\put(1429.0,849.0){\rule[-0.200pt]{2.409pt}{0.400pt}}
\put(371.0,849.0){\rule[-0.200pt]{2.409pt}{0.400pt}}
\put(1429.0,849.0){\rule[-0.200pt]{2.409pt}{0.400pt}}
\put(371.0,849.0){\rule[-0.200pt]{2.409pt}{0.400pt}}
\put(1429.0,849.0){\rule[-0.200pt]{2.409pt}{0.400pt}}
\put(371.0,849.0){\rule[-0.200pt]{2.409pt}{0.400pt}}
\put(1429.0,849.0){\rule[-0.200pt]{2.409pt}{0.400pt}}
\put(371.0,849.0){\rule[-0.200pt]{2.409pt}{0.400pt}}
\put(1429.0,849.0){\rule[-0.200pt]{2.409pt}{0.400pt}}
\put(371.0,849.0){\rule[-0.200pt]{2.409pt}{0.400pt}}
\put(1429.0,849.0){\rule[-0.200pt]{2.409pt}{0.400pt}}
\put(371.0,849.0){\rule[-0.200pt]{2.409pt}{0.400pt}}
\put(1429.0,849.0){\rule[-0.200pt]{2.409pt}{0.400pt}}
\put(371.0,849.0){\rule[-0.200pt]{2.409pt}{0.400pt}}
\put(1429.0,849.0){\rule[-0.200pt]{2.409pt}{0.400pt}}
\put(371.0,850.0){\rule[-0.200pt]{2.409pt}{0.400pt}}
\put(1429.0,850.0){\rule[-0.200pt]{2.409pt}{0.400pt}}
\put(371.0,850.0){\rule[-0.200pt]{2.409pt}{0.400pt}}
\put(1429.0,850.0){\rule[-0.200pt]{2.409pt}{0.400pt}}
\put(371.0,850.0){\rule[-0.200pt]{2.409pt}{0.400pt}}
\put(1429.0,850.0){\rule[-0.200pt]{2.409pt}{0.400pt}}
\put(371.0,850.0){\rule[-0.200pt]{2.409pt}{0.400pt}}
\put(1429.0,850.0){\rule[-0.200pt]{2.409pt}{0.400pt}}
\put(371.0,850.0){\rule[-0.200pt]{2.409pt}{0.400pt}}
\put(1429.0,850.0){\rule[-0.200pt]{2.409pt}{0.400pt}}
\put(371.0,850.0){\rule[-0.200pt]{2.409pt}{0.400pt}}
\put(1429.0,850.0){\rule[-0.200pt]{2.409pt}{0.400pt}}
\put(371.0,850.0){\rule[-0.200pt]{2.409pt}{0.400pt}}
\put(1429.0,850.0){\rule[-0.200pt]{2.409pt}{0.400pt}}
\put(371.0,850.0){\rule[-0.200pt]{2.409pt}{0.400pt}}
\put(1429.0,850.0){\rule[-0.200pt]{2.409pt}{0.400pt}}
\put(371.0,850.0){\rule[-0.200pt]{2.409pt}{0.400pt}}
\put(1429.0,850.0){\rule[-0.200pt]{2.409pt}{0.400pt}}
\put(371.0,850.0){\rule[-0.200pt]{2.409pt}{0.400pt}}
\put(1429.0,850.0){\rule[-0.200pt]{2.409pt}{0.400pt}}
\put(371.0,850.0){\rule[-0.200pt]{2.409pt}{0.400pt}}
\put(1429.0,850.0){\rule[-0.200pt]{2.409pt}{0.400pt}}
\put(371.0,851.0){\rule[-0.200pt]{2.409pt}{0.400pt}}
\put(1429.0,851.0){\rule[-0.200pt]{2.409pt}{0.400pt}}
\put(371.0,851.0){\rule[-0.200pt]{2.409pt}{0.400pt}}
\put(1429.0,851.0){\rule[-0.200pt]{2.409pt}{0.400pt}}
\put(371.0,851.0){\rule[-0.200pt]{2.409pt}{0.400pt}}
\put(1429.0,851.0){\rule[-0.200pt]{2.409pt}{0.400pt}}
\put(371.0,851.0){\rule[-0.200pt]{2.409pt}{0.400pt}}
\put(1429.0,851.0){\rule[-0.200pt]{2.409pt}{0.400pt}}
\put(371.0,851.0){\rule[-0.200pt]{2.409pt}{0.400pt}}
\put(1429.0,851.0){\rule[-0.200pt]{2.409pt}{0.400pt}}
\put(371.0,851.0){\rule[-0.200pt]{2.409pt}{0.400pt}}
\put(1429.0,851.0){\rule[-0.200pt]{2.409pt}{0.400pt}}
\put(371.0,851.0){\rule[-0.200pt]{2.409pt}{0.400pt}}
\put(1429.0,851.0){\rule[-0.200pt]{2.409pt}{0.400pt}}
\put(371.0,851.0){\rule[-0.200pt]{2.409pt}{0.400pt}}
\put(1429.0,851.0){\rule[-0.200pt]{2.409pt}{0.400pt}}
\put(371.0,851.0){\rule[-0.200pt]{2.409pt}{0.400pt}}
\put(1429.0,851.0){\rule[-0.200pt]{2.409pt}{0.400pt}}
\put(371.0,851.0){\rule[-0.200pt]{2.409pt}{0.400pt}}
\put(1429.0,851.0){\rule[-0.200pt]{2.409pt}{0.400pt}}
\put(371.0,851.0){\rule[-0.200pt]{2.409pt}{0.400pt}}
\put(1429.0,851.0){\rule[-0.200pt]{2.409pt}{0.400pt}}
\put(371.0,852.0){\rule[-0.200pt]{2.409pt}{0.400pt}}
\put(1429.0,852.0){\rule[-0.200pt]{2.409pt}{0.400pt}}
\put(371.0,852.0){\rule[-0.200pt]{2.409pt}{0.400pt}}
\put(1429.0,852.0){\rule[-0.200pt]{2.409pt}{0.400pt}}
\put(371.0,852.0){\rule[-0.200pt]{2.409pt}{0.400pt}}
\put(1429.0,852.0){\rule[-0.200pt]{2.409pt}{0.400pt}}
\put(371.0,852.0){\rule[-0.200pt]{2.409pt}{0.400pt}}
\put(1429.0,852.0){\rule[-0.200pt]{2.409pt}{0.400pt}}
\put(371.0,852.0){\rule[-0.200pt]{2.409pt}{0.400pt}}
\put(1429.0,852.0){\rule[-0.200pt]{2.409pt}{0.400pt}}
\put(371.0,852.0){\rule[-0.200pt]{2.409pt}{0.400pt}}
\put(1429.0,852.0){\rule[-0.200pt]{2.409pt}{0.400pt}}
\put(371.0,852.0){\rule[-0.200pt]{2.409pt}{0.400pt}}
\put(1429.0,852.0){\rule[-0.200pt]{2.409pt}{0.400pt}}
\put(371.0,852.0){\rule[-0.200pt]{2.409pt}{0.400pt}}
\put(1429.0,852.0){\rule[-0.200pt]{2.409pt}{0.400pt}}
\put(371.0,852.0){\rule[-0.200pt]{2.409pt}{0.400pt}}
\put(1429.0,852.0){\rule[-0.200pt]{2.409pt}{0.400pt}}
\put(371.0,852.0){\rule[-0.200pt]{2.409pt}{0.400pt}}
\put(1429.0,852.0){\rule[-0.200pt]{2.409pt}{0.400pt}}
\put(371.0,852.0){\rule[-0.200pt]{2.409pt}{0.400pt}}
\put(1429.0,852.0){\rule[-0.200pt]{2.409pt}{0.400pt}}
\put(371.0,852.0){\rule[-0.200pt]{2.409pt}{0.400pt}}
\put(1429.0,852.0){\rule[-0.200pt]{2.409pt}{0.400pt}}
\put(371.0,852.0){\rule[-0.200pt]{2.409pt}{0.400pt}}
\put(1429.0,852.0){\rule[-0.200pt]{2.409pt}{0.400pt}}
\put(371.0,853.0){\rule[-0.200pt]{2.409pt}{0.400pt}}
\put(1429.0,853.0){\rule[-0.200pt]{2.409pt}{0.400pt}}
\put(371.0,853.0){\rule[-0.200pt]{2.409pt}{0.400pt}}
\put(1429.0,853.0){\rule[-0.200pt]{2.409pt}{0.400pt}}
\put(371.0,853.0){\rule[-0.200pt]{2.409pt}{0.400pt}}
\put(1429.0,853.0){\rule[-0.200pt]{2.409pt}{0.400pt}}
\put(371.0,853.0){\rule[-0.200pt]{2.409pt}{0.400pt}}
\put(1429.0,853.0){\rule[-0.200pt]{2.409pt}{0.400pt}}
\put(371.0,853.0){\rule[-0.200pt]{2.409pt}{0.400pt}}
\put(1429.0,853.0){\rule[-0.200pt]{2.409pt}{0.400pt}}
\put(371.0,853.0){\rule[-0.200pt]{2.409pt}{0.400pt}}
\put(1429.0,853.0){\rule[-0.200pt]{2.409pt}{0.400pt}}
\put(371.0,853.0){\rule[-0.200pt]{2.409pt}{0.400pt}}
\put(1429.0,853.0){\rule[-0.200pt]{2.409pt}{0.400pt}}
\put(371.0,853.0){\rule[-0.200pt]{2.409pt}{0.400pt}}
\put(1429.0,853.0){\rule[-0.200pt]{2.409pt}{0.400pt}}
\put(371.0,853.0){\rule[-0.200pt]{2.409pt}{0.400pt}}
\put(1429.0,853.0){\rule[-0.200pt]{2.409pt}{0.400pt}}
\put(371.0,853.0){\rule[-0.200pt]{2.409pt}{0.400pt}}
\put(1429.0,853.0){\rule[-0.200pt]{2.409pt}{0.400pt}}
\put(371.0,853.0){\rule[-0.200pt]{2.409pt}{0.400pt}}
\put(1429.0,853.0){\rule[-0.200pt]{2.409pt}{0.400pt}}
\put(371.0,853.0){\rule[-0.200pt]{2.409pt}{0.400pt}}
\put(1429.0,853.0){\rule[-0.200pt]{2.409pt}{0.400pt}}
\put(371.0,854.0){\rule[-0.200pt]{2.409pt}{0.400pt}}
\put(1429.0,854.0){\rule[-0.200pt]{2.409pt}{0.400pt}}
\put(371.0,854.0){\rule[-0.200pt]{2.409pt}{0.400pt}}
\put(1429.0,854.0){\rule[-0.200pt]{2.409pt}{0.400pt}}
\put(371.0,854.0){\rule[-0.200pt]{2.409pt}{0.400pt}}
\put(1429.0,854.0){\rule[-0.200pt]{2.409pt}{0.400pt}}
\put(371.0,854.0){\rule[-0.200pt]{2.409pt}{0.400pt}}
\put(1429.0,854.0){\rule[-0.200pt]{2.409pt}{0.400pt}}
\put(371.0,854.0){\rule[-0.200pt]{2.409pt}{0.400pt}}
\put(1429.0,854.0){\rule[-0.200pt]{2.409pt}{0.400pt}}
\put(371.0,854.0){\rule[-0.200pt]{2.409pt}{0.400pt}}
\put(1429.0,854.0){\rule[-0.200pt]{2.409pt}{0.400pt}}
\put(371.0,854.0){\rule[-0.200pt]{2.409pt}{0.400pt}}
\put(1429.0,854.0){\rule[-0.200pt]{2.409pt}{0.400pt}}
\put(371.0,854.0){\rule[-0.200pt]{2.409pt}{0.400pt}}
\put(1429.0,854.0){\rule[-0.200pt]{2.409pt}{0.400pt}}
\put(371.0,854.0){\rule[-0.200pt]{2.409pt}{0.400pt}}
\put(1429.0,854.0){\rule[-0.200pt]{2.409pt}{0.400pt}}
\put(371.0,854.0){\rule[-0.200pt]{2.409pt}{0.400pt}}
\put(1429.0,854.0){\rule[-0.200pt]{2.409pt}{0.400pt}}
\put(371.0,854.0){\rule[-0.200pt]{2.409pt}{0.400pt}}
\put(1429.0,854.0){\rule[-0.200pt]{2.409pt}{0.400pt}}
\put(371.0,854.0){\rule[-0.200pt]{2.409pt}{0.400pt}}
\put(1429.0,854.0){\rule[-0.200pt]{2.409pt}{0.400pt}}
\put(371.0,854.0){\rule[-0.200pt]{2.409pt}{0.400pt}}
\put(1429.0,854.0){\rule[-0.200pt]{2.409pt}{0.400pt}}
\put(371.0,854.0){\rule[-0.200pt]{2.409pt}{0.400pt}}
\put(1429.0,854.0){\rule[-0.200pt]{2.409pt}{0.400pt}}
\put(371.0,855.0){\rule[-0.200pt]{2.409pt}{0.400pt}}
\put(1429.0,855.0){\rule[-0.200pt]{2.409pt}{0.400pt}}
\put(371.0,855.0){\rule[-0.200pt]{2.409pt}{0.400pt}}
\put(1429.0,855.0){\rule[-0.200pt]{2.409pt}{0.400pt}}
\put(371.0,855.0){\rule[-0.200pt]{2.409pt}{0.400pt}}
\put(1429.0,855.0){\rule[-0.200pt]{2.409pt}{0.400pt}}
\put(371.0,855.0){\rule[-0.200pt]{2.409pt}{0.400pt}}
\put(1429.0,855.0){\rule[-0.200pt]{2.409pt}{0.400pt}}
\put(371.0,855.0){\rule[-0.200pt]{2.409pt}{0.400pt}}
\put(1429.0,855.0){\rule[-0.200pt]{2.409pt}{0.400pt}}
\put(371.0,855.0){\rule[-0.200pt]{2.409pt}{0.400pt}}
\put(1429.0,855.0){\rule[-0.200pt]{2.409pt}{0.400pt}}
\put(371.0,855.0){\rule[-0.200pt]{2.409pt}{0.400pt}}
\put(1429.0,855.0){\rule[-0.200pt]{2.409pt}{0.400pt}}
\put(371.0,855.0){\rule[-0.200pt]{2.409pt}{0.400pt}}
\put(1429.0,855.0){\rule[-0.200pt]{2.409pt}{0.400pt}}
\put(371.0,855.0){\rule[-0.200pt]{2.409pt}{0.400pt}}
\put(1429.0,855.0){\rule[-0.200pt]{2.409pt}{0.400pt}}
\put(371.0,855.0){\rule[-0.200pt]{2.409pt}{0.400pt}}
\put(1429.0,855.0){\rule[-0.200pt]{2.409pt}{0.400pt}}
\put(371.0,855.0){\rule[-0.200pt]{2.409pt}{0.400pt}}
\put(1429.0,855.0){\rule[-0.200pt]{2.409pt}{0.400pt}}
\put(371.0,855.0){\rule[-0.200pt]{2.409pt}{0.400pt}}
\put(1429.0,855.0){\rule[-0.200pt]{2.409pt}{0.400pt}}
\put(371.0,855.0){\rule[-0.200pt]{2.409pt}{0.400pt}}
\put(1429.0,855.0){\rule[-0.200pt]{2.409pt}{0.400pt}}
\put(371.0,855.0){\rule[-0.200pt]{2.409pt}{0.400pt}}
\put(1429.0,855.0){\rule[-0.200pt]{2.409pt}{0.400pt}}
\put(371.0,856.0){\rule[-0.200pt]{2.409pt}{0.400pt}}
\put(1429.0,856.0){\rule[-0.200pt]{2.409pt}{0.400pt}}
\put(371.0,856.0){\rule[-0.200pt]{2.409pt}{0.400pt}}
\put(1429.0,856.0){\rule[-0.200pt]{2.409pt}{0.400pt}}
\put(371.0,856.0){\rule[-0.200pt]{2.409pt}{0.400pt}}
\put(1429.0,856.0){\rule[-0.200pt]{2.409pt}{0.400pt}}
\put(371.0,856.0){\rule[-0.200pt]{2.409pt}{0.400pt}}
\put(1429.0,856.0){\rule[-0.200pt]{2.409pt}{0.400pt}}
\put(371.0,856.0){\rule[-0.200pt]{2.409pt}{0.400pt}}
\put(1429.0,856.0){\rule[-0.200pt]{2.409pt}{0.400pt}}
\put(371.0,856.0){\rule[-0.200pt]{2.409pt}{0.400pt}}
\put(1429.0,856.0){\rule[-0.200pt]{2.409pt}{0.400pt}}
\put(371.0,856.0){\rule[-0.200pt]{2.409pt}{0.400pt}}
\put(1429.0,856.0){\rule[-0.200pt]{2.409pt}{0.400pt}}
\put(371.0,856.0){\rule[-0.200pt]{2.409pt}{0.400pt}}
\put(1429.0,856.0){\rule[-0.200pt]{2.409pt}{0.400pt}}
\put(371.0,856.0){\rule[-0.200pt]{2.409pt}{0.400pt}}
\put(1429.0,856.0){\rule[-0.200pt]{2.409pt}{0.400pt}}
\put(371.0,856.0){\rule[-0.200pt]{2.409pt}{0.400pt}}
\put(1429.0,856.0){\rule[-0.200pt]{2.409pt}{0.400pt}}
\put(371.0,856.0){\rule[-0.200pt]{2.409pt}{0.400pt}}
\put(1429.0,856.0){\rule[-0.200pt]{2.409pt}{0.400pt}}
\put(371.0,856.0){\rule[-0.200pt]{2.409pt}{0.400pt}}
\put(1429.0,856.0){\rule[-0.200pt]{2.409pt}{0.400pt}}
\put(371.0,856.0){\rule[-0.200pt]{2.409pt}{0.400pt}}
\put(1429.0,856.0){\rule[-0.200pt]{2.409pt}{0.400pt}}
\put(371.0,856.0){\rule[-0.200pt]{2.409pt}{0.400pt}}
\put(1429.0,856.0){\rule[-0.200pt]{2.409pt}{0.400pt}}
\put(371.0,856.0){\rule[-0.200pt]{2.409pt}{0.400pt}}
\put(1429.0,856.0){\rule[-0.200pt]{2.409pt}{0.400pt}}
\put(371.0,856.0){\rule[-0.200pt]{2.409pt}{0.400pt}}
\put(1429.0,856.0){\rule[-0.200pt]{2.409pt}{0.400pt}}
\put(371.0,857.0){\rule[-0.200pt]{2.409pt}{0.400pt}}
\put(1429.0,857.0){\rule[-0.200pt]{2.409pt}{0.400pt}}
\put(371.0,857.0){\rule[-0.200pt]{2.409pt}{0.400pt}}
\put(1429.0,857.0){\rule[-0.200pt]{2.409pt}{0.400pt}}
\put(371.0,857.0){\rule[-0.200pt]{2.409pt}{0.400pt}}
\put(1429.0,857.0){\rule[-0.200pt]{2.409pt}{0.400pt}}
\put(371.0,857.0){\rule[-0.200pt]{2.409pt}{0.400pt}}
\put(1429.0,857.0){\rule[-0.200pt]{2.409pt}{0.400pt}}
\put(371.0,857.0){\rule[-0.200pt]{2.409pt}{0.400pt}}
\put(1429.0,857.0){\rule[-0.200pt]{2.409pt}{0.400pt}}
\put(371.0,857.0){\rule[-0.200pt]{2.409pt}{0.400pt}}
\put(1429.0,857.0){\rule[-0.200pt]{2.409pt}{0.400pt}}
\put(371.0,857.0){\rule[-0.200pt]{2.409pt}{0.400pt}}
\put(1429.0,857.0){\rule[-0.200pt]{2.409pt}{0.400pt}}
\put(371.0,857.0){\rule[-0.200pt]{2.409pt}{0.400pt}}
\put(1429.0,857.0){\rule[-0.200pt]{2.409pt}{0.400pt}}
\put(371.0,857.0){\rule[-0.200pt]{2.409pt}{0.400pt}}
\put(1429.0,857.0){\rule[-0.200pt]{2.409pt}{0.400pt}}
\put(371.0,857.0){\rule[-0.200pt]{2.409pt}{0.400pt}}
\put(1429.0,857.0){\rule[-0.200pt]{2.409pt}{0.400pt}}
\put(371.0,857.0){\rule[-0.200pt]{2.409pt}{0.400pt}}
\put(1429.0,857.0){\rule[-0.200pt]{2.409pt}{0.400pt}}
\put(371.0,857.0){\rule[-0.200pt]{2.409pt}{0.400pt}}
\put(1429.0,857.0){\rule[-0.200pt]{2.409pt}{0.400pt}}
\put(371.0,857.0){\rule[-0.200pt]{2.409pt}{0.400pt}}
\put(1429.0,857.0){\rule[-0.200pt]{2.409pt}{0.400pt}}
\put(371.0,857.0){\rule[-0.200pt]{2.409pt}{0.400pt}}
\put(1429.0,857.0){\rule[-0.200pt]{2.409pt}{0.400pt}}
\put(371.0,857.0){\rule[-0.200pt]{2.409pt}{0.400pt}}
\put(1429.0,857.0){\rule[-0.200pt]{2.409pt}{0.400pt}}
\put(371.0,857.0){\rule[-0.200pt]{2.409pt}{0.400pt}}
\put(1429.0,857.0){\rule[-0.200pt]{2.409pt}{0.400pt}}
\put(371.0,858.0){\rule[-0.200pt]{2.409pt}{0.400pt}}
\put(1429.0,858.0){\rule[-0.200pt]{2.409pt}{0.400pt}}
\put(371.0,858.0){\rule[-0.200pt]{2.409pt}{0.400pt}}
\put(1429.0,858.0){\rule[-0.200pt]{2.409pt}{0.400pt}}
\put(371.0,858.0){\rule[-0.200pt]{2.409pt}{0.400pt}}
\put(1429.0,858.0){\rule[-0.200pt]{2.409pt}{0.400pt}}
\put(371.0,858.0){\rule[-0.200pt]{2.409pt}{0.400pt}}
\put(1429.0,858.0){\rule[-0.200pt]{2.409pt}{0.400pt}}
\put(371.0,858.0){\rule[-0.200pt]{2.409pt}{0.400pt}}
\put(1429.0,858.0){\rule[-0.200pt]{2.409pt}{0.400pt}}
\put(371.0,858.0){\rule[-0.200pt]{2.409pt}{0.400pt}}
\put(1429.0,858.0){\rule[-0.200pt]{2.409pt}{0.400pt}}
\put(371.0,858.0){\rule[-0.200pt]{2.409pt}{0.400pt}}
\put(1429.0,858.0){\rule[-0.200pt]{2.409pt}{0.400pt}}
\put(371.0,858.0){\rule[-0.200pt]{2.409pt}{0.400pt}}
\put(1429.0,858.0){\rule[-0.200pt]{2.409pt}{0.400pt}}
\put(371.0,858.0){\rule[-0.200pt]{2.409pt}{0.400pt}}
\put(1429.0,858.0){\rule[-0.200pt]{2.409pt}{0.400pt}}
\put(371.0,858.0){\rule[-0.200pt]{2.409pt}{0.400pt}}
\put(1429.0,858.0){\rule[-0.200pt]{2.409pt}{0.400pt}}
\put(371.0,858.0){\rule[-0.200pt]{2.409pt}{0.400pt}}
\put(1429.0,858.0){\rule[-0.200pt]{2.409pt}{0.400pt}}
\put(371.0,858.0){\rule[-0.200pt]{2.409pt}{0.400pt}}
\put(1429.0,858.0){\rule[-0.200pt]{2.409pt}{0.400pt}}
\put(371.0,858.0){\rule[-0.200pt]{2.409pt}{0.400pt}}
\put(1429.0,858.0){\rule[-0.200pt]{2.409pt}{0.400pt}}
\put(371.0,858.0){\rule[-0.200pt]{2.409pt}{0.400pt}}
\put(1429.0,858.0){\rule[-0.200pt]{2.409pt}{0.400pt}}
\put(371.0,858.0){\rule[-0.200pt]{2.409pt}{0.400pt}}
\put(1429.0,858.0){\rule[-0.200pt]{2.409pt}{0.400pt}}
\put(371.0,858.0){\rule[-0.200pt]{2.409pt}{0.400pt}}
\put(1429.0,858.0){\rule[-0.200pt]{2.409pt}{0.400pt}}
\put(371.0,859.0){\rule[-0.200pt]{2.409pt}{0.400pt}}
\put(1429.0,859.0){\rule[-0.200pt]{2.409pt}{0.400pt}}
\put(371.0,859.0){\rule[-0.200pt]{2.409pt}{0.400pt}}
\put(1429.0,859.0){\rule[-0.200pt]{2.409pt}{0.400pt}}
\put(371.0,859.0){\rule[-0.200pt]{2.409pt}{0.400pt}}
\put(1429.0,859.0){\rule[-0.200pt]{2.409pt}{0.400pt}}
\put(371.0,859.0){\rule[-0.200pt]{2.409pt}{0.400pt}}
\put(1429.0,859.0){\rule[-0.200pt]{2.409pt}{0.400pt}}
\put(371.0,859.0){\rule[-0.200pt]{2.409pt}{0.400pt}}
\put(1429.0,859.0){\rule[-0.200pt]{2.409pt}{0.400pt}}
\put(371.0,859.0){\rule[-0.200pt]{2.409pt}{0.400pt}}
\put(1429.0,859.0){\rule[-0.200pt]{2.409pt}{0.400pt}}
\put(371.0,859.0){\rule[-0.200pt]{2.409pt}{0.400pt}}
\put(1429.0,859.0){\rule[-0.200pt]{2.409pt}{0.400pt}}
\put(371.0,859.0){\rule[-0.200pt]{2.409pt}{0.400pt}}
\put(1429.0,859.0){\rule[-0.200pt]{2.409pt}{0.400pt}}
\put(371.0,859.0){\rule[-0.200pt]{2.409pt}{0.400pt}}
\put(1429.0,859.0){\rule[-0.200pt]{2.409pt}{0.400pt}}
\put(371.0,859.0){\rule[-0.200pt]{4.818pt}{0.400pt}}
\put(351,859){\makebox(0,0)[r]{ 1e+15}}
\put(1419.0,859.0){\rule[-0.200pt]{4.818pt}{0.400pt}}
\put(371.0,82.0){\rule[-0.200pt]{0.400pt}{4.818pt}}
\put(371,41){\makebox(0,0){ 0}}
\put(371.0,839.0){\rule[-0.200pt]{0.400pt}{4.818pt}}
\put(585.0,82.0){\rule[-0.200pt]{0.400pt}{4.818pt}}
\put(585,41){\makebox(0,0){ 100}}
\put(585.0,839.0){\rule[-0.200pt]{0.400pt}{4.818pt}}
\put(798.0,82.0){\rule[-0.200pt]{0.400pt}{4.818pt}}
\put(798,41){\makebox(0,0){ 200}}
\put(798.0,839.0){\rule[-0.200pt]{0.400pt}{4.818pt}}
\put(1012.0,82.0){\rule[-0.200pt]{0.400pt}{4.818pt}}
\put(1012,41){\makebox(0,0){ 300}}
\put(1012.0,839.0){\rule[-0.200pt]{0.400pt}{4.818pt}}
\put(1225.0,82.0){\rule[-0.200pt]{0.400pt}{4.818pt}}
\put(1225,41){\makebox(0,0){ 400}}
\put(1225.0,839.0){\rule[-0.200pt]{0.400pt}{4.818pt}}
\put(1439.0,82.0){\rule[-0.200pt]{0.400pt}{4.818pt}}
\put(1439,41){\makebox(0,0){ 500}}
\put(1439.0,839.0){\rule[-0.200pt]{0.400pt}{4.818pt}}
\put(371.0,82.0){\rule[-0.200pt]{0.400pt}{187.179pt}}
\put(371.0,82.0){\rule[-0.200pt]{257.281pt}{0.400pt}}
\put(1439.0,82.0){\rule[-0.200pt]{0.400pt}{187.179pt}}
\put(371.0,859.0){\rule[-0.200pt]{257.281pt}{0.400pt}}
\put(30,880){\makebox(0,0){Jednostka: 1e-30}}
\put(1279,205){\makebox(0,0)[r]{algorytm naturalny}}
\put(1299.0,205.0){\rule[-0.200pt]{24.090pt}{0.400pt}}
\put(375,103){\usebox{\plotpoint}}
\multiput(375.58,103.00)(0.492,5.449){19}{\rule{0.118pt}{4.318pt}}
\multiput(374.17,103.00)(11.000,107.037){2}{\rule{0.400pt}{2.159pt}}
\multiput(386.58,219.00)(0.492,2.618){19}{\rule{0.118pt}{2.136pt}}
\multiput(385.17,219.00)(11.000,51.566){2}{\rule{0.400pt}{1.068pt}}
\multiput(397.58,275.00)(0.492,1.675){19}{\rule{0.118pt}{1.409pt}}
\multiput(396.17,275.00)(11.000,33.075){2}{\rule{0.400pt}{0.705pt}}
\multiput(408.58,311.00)(0.492,1.156){19}{\rule{0.118pt}{1.009pt}}
\multiput(407.17,311.00)(11.000,22.906){2}{\rule{0.400pt}{0.505pt}}
\multiput(419.58,336.00)(0.492,0.920){19}{\rule{0.118pt}{0.827pt}}
\multiput(418.17,336.00)(11.000,18.283){2}{\rule{0.400pt}{0.414pt}}
\multiput(430.58,356.00)(0.492,0.779){19}{\rule{0.118pt}{0.718pt}}
\multiput(429.17,356.00)(11.000,15.509){2}{\rule{0.400pt}{0.359pt}}
\multiput(441.58,373.00)(0.492,0.637){19}{\rule{0.118pt}{0.609pt}}
\multiput(440.17,373.00)(11.000,12.736){2}{\rule{0.400pt}{0.305pt}}
\multiput(452.58,387.00)(0.492,0.543){19}{\rule{0.118pt}{0.536pt}}
\multiput(451.17,387.00)(11.000,10.887){2}{\rule{0.400pt}{0.268pt}}
\multiput(463.00,399.58)(0.496,0.492){19}{\rule{0.500pt}{0.118pt}}
\multiput(463.00,398.17)(9.962,11.000){2}{\rule{0.250pt}{0.400pt}}
\multiput(474.00,410.58)(0.547,0.491){17}{\rule{0.540pt}{0.118pt}}
\multiput(474.00,409.17)(9.879,10.000){2}{\rule{0.270pt}{0.400pt}}
\multiput(485.00,420.59)(0.692,0.488){13}{\rule{0.650pt}{0.117pt}}
\multiput(485.00,419.17)(9.651,8.000){2}{\rule{0.325pt}{0.400pt}}
\multiput(496.00,428.59)(0.692,0.488){13}{\rule{0.650pt}{0.117pt}}
\multiput(496.00,427.17)(9.651,8.000){2}{\rule{0.325pt}{0.400pt}}
\multiput(507.00,436.59)(0.692,0.488){13}{\rule{0.650pt}{0.117pt}}
\multiput(507.00,435.17)(9.651,8.000){2}{\rule{0.325pt}{0.400pt}}
\multiput(518.00,444.59)(0.798,0.485){11}{\rule{0.729pt}{0.117pt}}
\multiput(518.00,443.17)(9.488,7.000){2}{\rule{0.364pt}{0.400pt}}
\multiput(529.00,451.59)(0.943,0.482){9}{\rule{0.833pt}{0.116pt}}
\multiput(529.00,450.17)(9.270,6.000){2}{\rule{0.417pt}{0.400pt}}
\multiput(540.00,457.59)(0.943,0.482){9}{\rule{0.833pt}{0.116pt}}
\multiput(540.00,456.17)(9.270,6.000){2}{\rule{0.417pt}{0.400pt}}
\multiput(551.00,463.59)(0.943,0.482){9}{\rule{0.833pt}{0.116pt}}
\multiput(551.00,462.17)(9.270,6.000){2}{\rule{0.417pt}{0.400pt}}
\multiput(562.00,469.59)(1.155,0.477){7}{\rule{0.980pt}{0.115pt}}
\multiput(562.00,468.17)(8.966,5.000){2}{\rule{0.490pt}{0.400pt}}
\multiput(573.00,474.59)(1.155,0.477){7}{\rule{0.980pt}{0.115pt}}
\multiput(573.00,473.17)(8.966,5.000){2}{\rule{0.490pt}{0.400pt}}
\multiput(584.00,479.59)(1.044,0.477){7}{\rule{0.900pt}{0.115pt}}
\multiput(584.00,478.17)(8.132,5.000){2}{\rule{0.450pt}{0.400pt}}
\multiput(594.00,484.60)(1.505,0.468){5}{\rule{1.200pt}{0.113pt}}
\multiput(594.00,483.17)(8.509,4.000){2}{\rule{0.600pt}{0.400pt}}
\multiput(605.00,488.59)(1.155,0.477){7}{\rule{0.980pt}{0.115pt}}
\multiput(605.00,487.17)(8.966,5.000){2}{\rule{0.490pt}{0.400pt}}
\multiput(616.00,493.60)(1.505,0.468){5}{\rule{1.200pt}{0.113pt}}
\multiput(616.00,492.17)(8.509,4.000){2}{\rule{0.600pt}{0.400pt}}
\multiput(627.00,497.60)(1.505,0.468){5}{\rule{1.200pt}{0.113pt}}
\multiput(627.00,496.17)(8.509,4.000){2}{\rule{0.600pt}{0.400pt}}
\multiput(638.00,501.61)(2.248,0.447){3}{\rule{1.567pt}{0.108pt}}
\multiput(638.00,500.17)(7.748,3.000){2}{\rule{0.783pt}{0.400pt}}
\multiput(649.00,504.60)(1.505,0.468){5}{\rule{1.200pt}{0.113pt}}
\multiput(649.00,503.17)(8.509,4.000){2}{\rule{0.600pt}{0.400pt}}
\multiput(660.00,508.60)(1.505,0.468){5}{\rule{1.200pt}{0.113pt}}
\multiput(660.00,507.17)(8.509,4.000){2}{\rule{0.600pt}{0.400pt}}
\multiput(671.00,512.61)(2.248,0.447){3}{\rule{1.567pt}{0.108pt}}
\multiput(671.00,511.17)(7.748,3.000){2}{\rule{0.783pt}{0.400pt}}
\multiput(682.00,515.61)(2.248,0.447){3}{\rule{1.567pt}{0.108pt}}
\multiput(682.00,514.17)(7.748,3.000){2}{\rule{0.783pt}{0.400pt}}
\multiput(693.00,518.61)(2.248,0.447){3}{\rule{1.567pt}{0.108pt}}
\multiput(693.00,517.17)(7.748,3.000){2}{\rule{0.783pt}{0.400pt}}
\multiput(704.00,521.61)(2.248,0.447){3}{\rule{1.567pt}{0.108pt}}
\multiput(704.00,520.17)(7.748,3.000){2}{\rule{0.783pt}{0.400pt}}
\multiput(715.00,524.61)(2.248,0.447){3}{\rule{1.567pt}{0.108pt}}
\multiput(715.00,523.17)(7.748,3.000){2}{\rule{0.783pt}{0.400pt}}
\multiput(726.00,527.61)(2.248,0.447){3}{\rule{1.567pt}{0.108pt}}
\multiput(726.00,526.17)(7.748,3.000){2}{\rule{0.783pt}{0.400pt}}
\multiput(737.00,530.61)(2.248,0.447){3}{\rule{1.567pt}{0.108pt}}
\multiput(737.00,529.17)(7.748,3.000){2}{\rule{0.783pt}{0.400pt}}
\multiput(748.00,533.61)(2.248,0.447){3}{\rule{1.567pt}{0.108pt}}
\multiput(748.00,532.17)(7.748,3.000){2}{\rule{0.783pt}{0.400pt}}
\put(759,536.17){\rule{2.300pt}{0.400pt}}
\multiput(759.00,535.17)(6.226,2.000){2}{\rule{1.150pt}{0.400pt}}
\multiput(770.00,538.61)(2.248,0.447){3}{\rule{1.567pt}{0.108pt}}
\multiput(770.00,537.17)(7.748,3.000){2}{\rule{0.783pt}{0.400pt}}
\put(781,541.17){\rule{2.300pt}{0.400pt}}
\multiput(781.00,540.17)(6.226,2.000){2}{\rule{1.150pt}{0.400pt}}
\multiput(792.00,543.61)(2.248,0.447){3}{\rule{1.567pt}{0.108pt}}
\multiput(792.00,542.17)(7.748,3.000){2}{\rule{0.783pt}{0.400pt}}
\put(803,546.17){\rule{2.300pt}{0.400pt}}
\multiput(803.00,545.17)(6.226,2.000){2}{\rule{1.150pt}{0.400pt}}
\put(814,548.17){\rule{2.300pt}{0.400pt}}
\multiput(814.00,547.17)(6.226,2.000){2}{\rule{1.150pt}{0.400pt}}
\multiput(825.00,550.61)(2.248,0.447){3}{\rule{1.567pt}{0.108pt}}
\multiput(825.00,549.17)(7.748,3.000){2}{\rule{0.783pt}{0.400pt}}
\put(836,553.17){\rule{2.300pt}{0.400pt}}
\multiput(836.00,552.17)(6.226,2.000){2}{\rule{1.150pt}{0.400pt}}
\put(847,555.17){\rule{2.300pt}{0.400pt}}
\multiput(847.00,554.17)(6.226,2.000){2}{\rule{1.150pt}{0.400pt}}
\put(858,557.17){\rule{2.100pt}{0.400pt}}
\multiput(858.00,556.17)(5.641,2.000){2}{\rule{1.050pt}{0.400pt}}
\put(868,559.17){\rule{2.300pt}{0.400pt}}
\multiput(868.00,558.17)(6.226,2.000){2}{\rule{1.150pt}{0.400pt}}
\put(879,561.17){\rule{2.300pt}{0.400pt}}
\multiput(879.00,560.17)(6.226,2.000){2}{\rule{1.150pt}{0.400pt}}
\put(890,563.17){\rule{2.300pt}{0.400pt}}
\multiput(890.00,562.17)(6.226,2.000){2}{\rule{1.150pt}{0.400pt}}
\put(901,565.17){\rule{2.300pt}{0.400pt}}
\multiput(901.00,564.17)(6.226,2.000){2}{\rule{1.150pt}{0.400pt}}
\put(912,567.17){\rule{2.300pt}{0.400pt}}
\multiput(912.00,566.17)(6.226,2.000){2}{\rule{1.150pt}{0.400pt}}
\put(923,569.17){\rule{2.300pt}{0.400pt}}
\multiput(923.00,568.17)(6.226,2.000){2}{\rule{1.150pt}{0.400pt}}
\put(934,571.17){\rule{2.300pt}{0.400pt}}
\multiput(934.00,570.17)(6.226,2.000){2}{\rule{1.150pt}{0.400pt}}
\put(945,572.67){\rule{2.650pt}{0.400pt}}
\multiput(945.00,572.17)(5.500,1.000){2}{\rule{1.325pt}{0.400pt}}
\put(956,574.17){\rule{2.300pt}{0.400pt}}
\multiput(956.00,573.17)(6.226,2.000){2}{\rule{1.150pt}{0.400pt}}
\put(967,576.17){\rule{2.300pt}{0.400pt}}
\multiput(967.00,575.17)(6.226,2.000){2}{\rule{1.150pt}{0.400pt}}
\put(978,578.17){\rule{2.300pt}{0.400pt}}
\multiput(978.00,577.17)(6.226,2.000){2}{\rule{1.150pt}{0.400pt}}
\put(989,579.67){\rule{2.650pt}{0.400pt}}
\multiput(989.00,579.17)(5.500,1.000){2}{\rule{1.325pt}{0.400pt}}
\put(1000,581.17){\rule{2.300pt}{0.400pt}}
\multiput(1000.00,580.17)(6.226,2.000){2}{\rule{1.150pt}{0.400pt}}
\put(1011,582.67){\rule{2.650pt}{0.400pt}}
\multiput(1011.00,582.17)(5.500,1.000){2}{\rule{1.325pt}{0.400pt}}
\put(1022,584.17){\rule{2.300pt}{0.400pt}}
\multiput(1022.00,583.17)(6.226,2.000){2}{\rule{1.150pt}{0.400pt}}
\put(1033,586.17){\rule{2.300pt}{0.400pt}}
\multiput(1033.00,585.17)(6.226,2.000){2}{\rule{1.150pt}{0.400pt}}
\put(1044,587.67){\rule{2.650pt}{0.400pt}}
\multiput(1044.00,587.17)(5.500,1.000){2}{\rule{1.325pt}{0.400pt}}
\put(1055,589.17){\rule{2.300pt}{0.400pt}}
\multiput(1055.00,588.17)(6.226,2.000){2}{\rule{1.150pt}{0.400pt}}
\put(1066,590.67){\rule{2.650pt}{0.400pt}}
\multiput(1066.00,590.17)(5.500,1.000){2}{\rule{1.325pt}{0.400pt}}
\put(1077,591.67){\rule{2.650pt}{0.400pt}}
\multiput(1077.00,591.17)(5.500,1.000){2}{\rule{1.325pt}{0.400pt}}
\put(1088,593.17){\rule{2.300pt}{0.400pt}}
\multiput(1088.00,592.17)(6.226,2.000){2}{\rule{1.150pt}{0.400pt}}
\put(1099,594.67){\rule{2.650pt}{0.400pt}}
\multiput(1099.00,594.17)(5.500,1.000){2}{\rule{1.325pt}{0.400pt}}
\put(1110,596.17){\rule{2.300pt}{0.400pt}}
\multiput(1110.00,595.17)(6.226,2.000){2}{\rule{1.150pt}{0.400pt}}
\put(1121,597.67){\rule{2.650pt}{0.400pt}}
\multiput(1121.00,597.17)(5.500,1.000){2}{\rule{1.325pt}{0.400pt}}
\put(1132,598.67){\rule{2.650pt}{0.400pt}}
\multiput(1132.00,598.17)(5.500,1.000){2}{\rule{1.325pt}{0.400pt}}
\put(1143,600.17){\rule{2.100pt}{0.400pt}}
\multiput(1143.00,599.17)(5.641,2.000){2}{\rule{1.050pt}{0.400pt}}
\put(1153,601.67){\rule{2.650pt}{0.400pt}}
\multiput(1153.00,601.17)(5.500,1.000){2}{\rule{1.325pt}{0.400pt}}
\put(1164,602.67){\rule{2.650pt}{0.400pt}}
\multiput(1164.00,602.17)(5.500,1.000){2}{\rule{1.325pt}{0.400pt}}
\put(1175,604.17){\rule{2.300pt}{0.400pt}}
\multiput(1175.00,603.17)(6.226,2.000){2}{\rule{1.150pt}{0.400pt}}
\put(1186,605.67){\rule{2.650pt}{0.400pt}}
\multiput(1186.00,605.17)(5.500,1.000){2}{\rule{1.325pt}{0.400pt}}
\put(1197,606.67){\rule{2.650pt}{0.400pt}}
\multiput(1197.00,606.17)(5.500,1.000){2}{\rule{1.325pt}{0.400pt}}
\put(1208,607.67){\rule{2.650pt}{0.400pt}}
\multiput(1208.00,607.17)(5.500,1.000){2}{\rule{1.325pt}{0.400pt}}
\put(1219,609.17){\rule{2.300pt}{0.400pt}}
\multiput(1219.00,608.17)(6.226,2.000){2}{\rule{1.150pt}{0.400pt}}
\put(1230,610.67){\rule{2.650pt}{0.400pt}}
\multiput(1230.00,610.17)(5.500,1.000){2}{\rule{1.325pt}{0.400pt}}
\put(1241,611.67){\rule{2.650pt}{0.400pt}}
\multiput(1241.00,611.17)(5.500,1.000){2}{\rule{1.325pt}{0.400pt}}
\put(1252,612.67){\rule{2.650pt}{0.400pt}}
\multiput(1252.00,612.17)(5.500,1.000){2}{\rule{1.325pt}{0.400pt}}
\put(1263,613.67){\rule{2.650pt}{0.400pt}}
\multiput(1263.00,613.17)(5.500,1.000){2}{\rule{1.325pt}{0.400pt}}
\put(1274,614.67){\rule{2.650pt}{0.400pt}}
\multiput(1274.00,614.17)(5.500,1.000){2}{\rule{1.325pt}{0.400pt}}
\put(1285,615.67){\rule{2.650pt}{0.400pt}}
\multiput(1285.00,615.17)(5.500,1.000){2}{\rule{1.325pt}{0.400pt}}
\put(1296,616.67){\rule{2.650pt}{0.400pt}}
\multiput(1296.00,616.17)(5.500,1.000){2}{\rule{1.325pt}{0.400pt}}
\put(1307,618.17){\rule{2.300pt}{0.400pt}}
\multiput(1307.00,617.17)(6.226,2.000){2}{\rule{1.150pt}{0.400pt}}
\put(1318,619.67){\rule{2.650pt}{0.400pt}}
\multiput(1318.00,619.17)(5.500,1.000){2}{\rule{1.325pt}{0.400pt}}
\put(1329,620.67){\rule{2.650pt}{0.400pt}}
\multiput(1329.00,620.17)(5.500,1.000){2}{\rule{1.325pt}{0.400pt}}
\put(1340,621.67){\rule{2.650pt}{0.400pt}}
\multiput(1340.00,621.17)(5.500,1.000){2}{\rule{1.325pt}{0.400pt}}
\put(1351,622.67){\rule{2.650pt}{0.400pt}}
\multiput(1351.00,622.17)(5.500,1.000){2}{\rule{1.325pt}{0.400pt}}
\put(1362,623.67){\rule{2.650pt}{0.400pt}}
\multiput(1362.00,623.17)(5.500,1.000){2}{\rule{1.325pt}{0.400pt}}
\put(1373,624.67){\rule{2.650pt}{0.400pt}}
\multiput(1373.00,624.17)(5.500,1.000){2}{\rule{1.325pt}{0.400pt}}
\put(1384,625.67){\rule{2.650pt}{0.400pt}}
\multiput(1384.00,625.17)(5.500,1.000){2}{\rule{1.325pt}{0.400pt}}
\put(1395,626.67){\rule{2.650pt}{0.400pt}}
\multiput(1395.00,626.17)(5.500,1.000){2}{\rule{1.325pt}{0.400pt}}
\put(1406,627.67){\rule{2.650pt}{0.400pt}}
\multiput(1406.00,627.17)(5.500,1.000){2}{\rule{1.325pt}{0.400pt}}
\put(1417,628.67){\rule{2.409pt}{0.400pt}}
\multiput(1417.00,628.17)(5.000,1.000){2}{\rule{1.204pt}{0.400pt}}
\put(1427,629.67){\rule{2.650pt}{0.400pt}}
\multiput(1427.00,629.17)(5.500,1.000){2}{\rule{1.325pt}{0.400pt}}
\put(1438.0,631.0){\usebox{\plotpoint}}
\sbox{\plotpoint}{\rule[-0.500pt]{1.000pt}{1.000pt}}%
\sbox{\plotpoint}{\rule[-0.200pt]{0.400pt}{0.400pt}}%
\put(1279,164){\makebox(0,0)[r]{algorytm Strassena}}
\sbox{\plotpoint}{\rule[-0.500pt]{1.000pt}{1.000pt}}%
\multiput(1299,164)(20.756,0.000){5}{\usebox{\plotpoint}}
\put(1399,164){\usebox{\plotpoint}}
\put(375,134){\usebox{\plotpoint}}
\multiput(375,134)(1.637,20.691){7}{\usebox{\plotpoint}}
\multiput(386,273)(3.052,20.530){4}{\usebox{\plotpoint}}
\multiput(397,347)(4.730,20.209){2}{\usebox{\plotpoint}}
\multiput(408,394)(6.563,19.690){2}{\usebox{\plotpoint}}
\put(424.65,439.84){\usebox{\plotpoint}}
\put(433.60,458.55){\usebox{\plotpoint}}
\put(443.82,476.61){\usebox{\plotpoint}}
\put(454.99,494.08){\usebox{\plotpoint}}
\put(467.66,510.51){\usebox{\plotpoint}}
\put(481.39,526.07){\usebox{\plotpoint}}
\put(495.90,540.90){\usebox{\plotpoint}}
\put(511.96,554.06){\usebox{\plotpoint}}
\put(528.47,566.62){\usebox{\plotpoint}}
\put(545.49,578.49){\usebox{\plotpoint}}
\put(563.04,589.57){\usebox{\plotpoint}}
\put(580.94,600.05){\usebox{\plotpoint}}
\put(598.97,610.26){\usebox{\plotpoint}}
\put(617.46,619.66){\usebox{\plotpoint}}
\put(636.66,627.51){\usebox{\plotpoint}}
\put(655.81,635.48){\usebox{\plotpoint}}
\put(675.43,642.21){\usebox{\plotpoint}}
\put(695.10,648.77){\usebox{\plotpoint}}
\put(714.61,655.86){\usebox{\plotpoint}}
\put(734.40,662.05){\usebox{\plotpoint}}
\put(754.19,668.25){\usebox{\plotpoint}}
\put(773.98,674.45){\usebox{\plotpoint}}
\put(793.82,680.50){\usebox{\plotpoint}}
\put(813.84,685.96){\usebox{\plotpoint}}
\put(833.87,691.42){\usebox{\plotpoint}}
\put(853.89,696.88){\usebox{\plotpoint}}
\put(874.08,701.66){\usebox{\plotpoint}}
\put(894.40,705.80){\usebox{\plotpoint}}
\put(914.82,709.51){\usebox{\plotpoint}}
\put(935.24,713.23){\usebox{\plotpoint}}
\put(955.66,716.94){\usebox{\plotpoint}}
\put(976.22,719.68){\usebox{\plotpoint}}
\put(996.64,723.39){\usebox{\plotpoint}}
\put(1017.19,726.13){\usebox{\plotpoint}}
\put(1037.61,729.84){\usebox{\plotpoint}}
\put(1058.03,733.55){\usebox{\plotpoint}}
\put(1078.59,736.29){\usebox{\plotpoint}}
\put(1099.01,740.00){\usebox{\plotpoint}}
\put(1119.43,743.71){\usebox{\plotpoint}}
\put(1139.85,747.43){\usebox{\plotpoint}}
\put(1160.38,750.34){\usebox{\plotpoint}}
\put(1180.80,754.05){\usebox{\plotpoint}}
\put(1201.35,756.79){\usebox{\plotpoint}}
\put(1221.91,759.53){\usebox{\plotpoint}}
\put(1242.34,763.12){\usebox{\plotpoint}}
\put(1262.88,765.98){\usebox{\plotpoint}}
\put(1283.44,768.72){\usebox{\plotpoint}}
\put(1303.99,771.45){\usebox{\plotpoint}}
\put(1324.62,773.60){\usebox{\plotpoint}}
\put(1345.16,776.47){\usebox{\plotpoint}}
\put(1365.78,778.69){\usebox{\plotpoint}}
\put(1386.36,781.21){\usebox{\plotpoint}}
\put(1406.90,784.08){\usebox{\plotpoint}}
\put(1427.56,786.05){\usebox{\plotpoint}}
\put(1439,787){\usebox{\plotpoint}}
\sbox{\plotpoint}{\rule[-0.600pt]{1.200pt}{1.200pt}}%
\sbox{\plotpoint}{\rule[-0.200pt]{0.400pt}{0.400pt}}%
\put(1279,123){\makebox(0,0)[r]{algorytm z progiem}}
\sbox{\plotpoint}{\rule[-0.600pt]{1.200pt}{1.200pt}}%
\put(1299.0,123.0){\rule[-0.600pt]{24.090pt}{1.200pt}}
\put(375,103){\usebox{\plotpoint}}
\multiput(377.24,103.00)(0.502,5.657){12}{\rule{0.121pt}{12.955pt}}
\multiput(372.51,103.00)(11.000,89.112){2}{\rule{1.200pt}{6.477pt}}
\multiput(388.24,219.00)(0.502,2.672){12}{\rule{0.121pt}{6.409pt}}
\multiput(383.51,219.00)(11.000,42.698){2}{\rule{1.200pt}{3.205pt}}
\multiput(399.24,275.00)(0.502,1.677){12}{\rule{0.121pt}{4.227pt}}
\multiput(394.51,275.00)(11.000,27.226){2}{\rule{1.200pt}{2.114pt}}
\multiput(410.24,311.00)(0.502,1.130){12}{\rule{0.121pt}{3.027pt}}
\multiput(405.51,311.00)(11.000,18.717){2}{\rule{1.200pt}{1.514pt}}
\multiput(421.24,336.00)(0.502,0.882){12}{\rule{0.121pt}{2.482pt}}
\multiput(416.51,336.00)(11.000,14.849){2}{\rule{1.200pt}{1.241pt}}
\multiput(432.24,356.00)(0.502,0.732){12}{\rule{0.121pt}{2.155pt}}
\multiput(427.51,356.00)(11.000,12.528){2}{\rule{1.200pt}{1.077pt}}
\multiput(443.24,373.00)(0.502,0.583){12}{\rule{0.121pt}{1.827pt}}
\multiput(438.51,373.00)(11.000,10.207){2}{\rule{1.200pt}{0.914pt}}
\multiput(454.24,387.00)(0.502,0.484){12}{\rule{0.121pt}{1.609pt}}
\multiput(449.51,387.00)(11.000,8.660){2}{\rule{1.200pt}{0.805pt}}
\multiput(463.00,401.24)(0.434,0.502){12}{\rule{1.500pt}{0.121pt}}
\multiput(463.00,396.51)(7.887,11.000){2}{\rule{0.750pt}{1.200pt}}
\multiput(474.00,412.24)(0.475,0.502){10}{\rule{1.620pt}{0.121pt}}
\multiput(474.00,407.51)(7.638,10.000){2}{\rule{0.810pt}{1.200pt}}
\multiput(485.00,422.24)(0.581,0.503){6}{\rule{1.950pt}{0.121pt}}
\multiput(485.00,417.51)(6.953,8.000){2}{\rule{0.975pt}{1.200pt}}
\multiput(496.00,430.24)(0.581,0.503){6}{\rule{1.950pt}{0.121pt}}
\multiput(496.00,425.51)(6.953,8.000){2}{\rule{0.975pt}{1.200pt}}
\multiput(507.00,438.24)(0.581,0.503){6}{\rule{1.950pt}{0.121pt}}
\multiput(507.00,433.51)(6.953,8.000){2}{\rule{0.975pt}{1.200pt}}
\multiput(518.00,446.24)(0.642,0.505){4}{\rule{2.186pt}{0.122pt}}
\multiput(518.00,441.51)(6.463,7.000){2}{\rule{1.093pt}{1.200pt}}
\multiput(529.00,453.24)(0.622,0.509){2}{\rule{2.500pt}{0.123pt}}
\multiput(529.00,448.51)(5.811,6.000){2}{\rule{1.250pt}{1.200pt}}
\multiput(540.00,459.24)(0.622,0.509){2}{\rule{2.500pt}{0.123pt}}
\multiput(540.00,454.51)(5.811,6.000){2}{\rule{1.250pt}{1.200pt}}
\multiput(551.00,465.24)(0.622,0.509){2}{\rule{2.500pt}{0.123pt}}
\multiput(551.00,460.51)(5.811,6.000){2}{\rule{1.250pt}{1.200pt}}
\put(562,469.01){\rule{2.650pt}{1.200pt}}
\multiput(562.00,466.51)(5.500,5.000){2}{\rule{1.325pt}{1.200pt}}
\put(573,474.01){\rule{2.650pt}{1.200pt}}
\multiput(573.00,471.51)(5.500,5.000){2}{\rule{1.325pt}{1.200pt}}
\put(584,479.01){\rule{2.409pt}{1.200pt}}
\multiput(584.00,476.51)(5.000,5.000){2}{\rule{1.204pt}{1.200pt}}
\put(594,483.51){\rule{2.650pt}{1.200pt}}
\multiput(594.00,481.51)(5.500,4.000){2}{\rule{1.325pt}{1.200pt}}
\put(605,488.01){\rule{2.650pt}{1.200pt}}
\multiput(605.00,485.51)(5.500,5.000){2}{\rule{1.325pt}{1.200pt}}
\put(616,492.51){\rule{2.650pt}{1.200pt}}
\multiput(616.00,490.51)(5.500,4.000){2}{\rule{1.325pt}{1.200pt}}
\put(627,496.51){\rule{2.650pt}{1.200pt}}
\multiput(627.00,494.51)(5.500,4.000){2}{\rule{1.325pt}{1.200pt}}
\put(638,500.51){\rule{2.650pt}{1.200pt}}
\multiput(638.00,498.51)(5.500,4.000){2}{\rule{1.325pt}{1.200pt}}
\put(649,504.51){\rule{2.650pt}{1.200pt}}
\multiput(649.00,502.51)(5.500,4.000){2}{\rule{1.325pt}{1.200pt}}
\put(660,508.51){\rule{2.650pt}{1.200pt}}
\multiput(660.00,506.51)(5.500,4.000){2}{\rule{1.325pt}{1.200pt}}
\put(671,512.51){\rule{2.650pt}{1.200pt}}
\multiput(671.00,510.51)(5.500,4.000){2}{\rule{1.325pt}{1.200pt}}
\put(682,516.51){\rule{2.650pt}{1.200pt}}
\multiput(682.00,514.51)(5.500,4.000){2}{\rule{1.325pt}{1.200pt}}
\put(693,520.51){\rule{2.650pt}{1.200pt}}
\multiput(693.00,518.51)(5.500,4.000){2}{\rule{1.325pt}{1.200pt}}
\put(704,524.51){\rule{2.650pt}{1.200pt}}
\multiput(704.00,522.51)(5.500,4.000){2}{\rule{1.325pt}{1.200pt}}
\put(715,528.01){\rule{2.650pt}{1.200pt}}
\multiput(715.00,526.51)(5.500,3.000){2}{\rule{1.325pt}{1.200pt}}
\put(726,531.51){\rule{2.650pt}{1.200pt}}
\multiput(726.00,529.51)(5.500,4.000){2}{\rule{1.325pt}{1.200pt}}
\put(737,535.51){\rule{2.650pt}{1.200pt}}
\multiput(737.00,533.51)(5.500,4.000){2}{\rule{1.325pt}{1.200pt}}
\put(748,539.01){\rule{2.650pt}{1.200pt}}
\multiput(748.00,537.51)(5.500,3.000){2}{\rule{1.325pt}{1.200pt}}
\put(759,542.51){\rule{2.650pt}{1.200pt}}
\multiput(759.00,540.51)(5.500,4.000){2}{\rule{1.325pt}{1.200pt}}
\put(770,546.01){\rule{2.650pt}{1.200pt}}
\multiput(770.00,544.51)(5.500,3.000){2}{\rule{1.325pt}{1.200pt}}
\put(781,549.01){\rule{2.650pt}{1.200pt}}
\multiput(781.00,547.51)(5.500,3.000){2}{\rule{1.325pt}{1.200pt}}
\put(792,552.01){\rule{2.650pt}{1.200pt}}
\multiput(792.00,550.51)(5.500,3.000){2}{\rule{1.325pt}{1.200pt}}
\put(803,555.01){\rule{2.650pt}{1.200pt}}
\multiput(803.00,553.51)(5.500,3.000){2}{\rule{1.325pt}{1.200pt}}
\put(814,558.01){\rule{2.650pt}{1.200pt}}
\multiput(814.00,556.51)(5.500,3.000){2}{\rule{1.325pt}{1.200pt}}
\put(825,560.51){\rule{2.650pt}{1.200pt}}
\multiput(825.00,559.51)(5.500,2.000){2}{\rule{1.325pt}{1.200pt}}
\put(836,563.01){\rule{2.650pt}{1.200pt}}
\multiput(836.00,561.51)(5.500,3.000){2}{\rule{1.325pt}{1.200pt}}
\put(847,565.51){\rule{2.650pt}{1.200pt}}
\multiput(847.00,564.51)(5.500,2.000){2}{\rule{1.325pt}{1.200pt}}
\put(858,568.01){\rule{2.409pt}{1.200pt}}
\multiput(858.00,566.51)(5.000,3.000){2}{\rule{1.204pt}{1.200pt}}
\put(868,570.51){\rule{2.650pt}{1.200pt}}
\multiput(868.00,569.51)(5.500,2.000){2}{\rule{1.325pt}{1.200pt}}
\put(879,572.51){\rule{2.650pt}{1.200pt}}
\multiput(879.00,571.51)(5.500,2.000){2}{\rule{1.325pt}{1.200pt}}
\put(890,574.51){\rule{2.650pt}{1.200pt}}
\multiput(890.00,573.51)(5.500,2.000){2}{\rule{1.325pt}{1.200pt}}
\put(901,576.51){\rule{2.650pt}{1.200pt}}
\multiput(901.00,575.51)(5.500,2.000){2}{\rule{1.325pt}{1.200pt}}
\put(912,578.51){\rule{2.650pt}{1.200pt}}
\multiput(912.00,577.51)(5.500,2.000){2}{\rule{1.325pt}{1.200pt}}
\put(923,580.51){\rule{2.650pt}{1.200pt}}
\multiput(923.00,579.51)(5.500,2.000){2}{\rule{1.325pt}{1.200pt}}
\put(934,582.51){\rule{2.650pt}{1.200pt}}
\multiput(934.00,581.51)(5.500,2.000){2}{\rule{1.325pt}{1.200pt}}
\put(945,584.01){\rule{2.650pt}{1.200pt}}
\multiput(945.00,583.51)(5.500,1.000){2}{\rule{1.325pt}{1.200pt}}
\put(956,585.51){\rule{2.650pt}{1.200pt}}
\multiput(956.00,584.51)(5.500,2.000){2}{\rule{1.325pt}{1.200pt}}
\put(967,587.51){\rule{2.650pt}{1.200pt}}
\multiput(967.00,586.51)(5.500,2.000){2}{\rule{1.325pt}{1.200pt}}
\put(978,589.51){\rule{2.650pt}{1.200pt}}
\multiput(978.00,588.51)(5.500,2.000){2}{\rule{1.325pt}{1.200pt}}
\put(989,591.51){\rule{2.650pt}{1.200pt}}
\multiput(989.00,590.51)(5.500,2.000){2}{\rule{1.325pt}{1.200pt}}
\put(1000,593.51){\rule{2.650pt}{1.200pt}}
\multiput(1000.00,592.51)(5.500,2.000){2}{\rule{1.325pt}{1.200pt}}
\put(1011,595.51){\rule{2.650pt}{1.200pt}}
\multiput(1011.00,594.51)(5.500,2.000){2}{\rule{1.325pt}{1.200pt}}
\put(1022,597.51){\rule{2.650pt}{1.200pt}}
\multiput(1022.00,596.51)(5.500,2.000){2}{\rule{1.325pt}{1.200pt}}
\put(1033,599.51){\rule{2.650pt}{1.200pt}}
\multiput(1033.00,598.51)(5.500,2.000){2}{\rule{1.325pt}{1.200pt}}
\put(1044,601.51){\rule{2.650pt}{1.200pt}}
\multiput(1044.00,600.51)(5.500,2.000){2}{\rule{1.325pt}{1.200pt}}
\put(1055,603.51){\rule{2.650pt}{1.200pt}}
\multiput(1055.00,602.51)(5.500,2.000){2}{\rule{1.325pt}{1.200pt}}
\put(1066,605.01){\rule{2.650pt}{1.200pt}}
\multiput(1066.00,604.51)(5.500,1.000){2}{\rule{1.325pt}{1.200pt}}
\put(1077,606.51){\rule{2.650pt}{1.200pt}}
\multiput(1077.00,605.51)(5.500,2.000){2}{\rule{1.325pt}{1.200pt}}
\put(1088,608.51){\rule{2.650pt}{1.200pt}}
\multiput(1088.00,607.51)(5.500,2.000){2}{\rule{1.325pt}{1.200pt}}
\put(1099,610.51){\rule{2.650pt}{1.200pt}}
\multiput(1099.00,609.51)(5.500,2.000){2}{\rule{1.325pt}{1.200pt}}
\put(1110,612.51){\rule{2.650pt}{1.200pt}}
\multiput(1110.00,611.51)(5.500,2.000){2}{\rule{1.325pt}{1.200pt}}
\put(1121,614.51){\rule{2.650pt}{1.200pt}}
\multiput(1121.00,613.51)(5.500,2.000){2}{\rule{1.325pt}{1.200pt}}
\put(1132,616.01){\rule{2.650pt}{1.200pt}}
\multiput(1132.00,615.51)(5.500,1.000){2}{\rule{1.325pt}{1.200pt}}
\put(1143,617.51){\rule{2.409pt}{1.200pt}}
\multiput(1143.00,616.51)(5.000,2.000){2}{\rule{1.204pt}{1.200pt}}
\put(1153,619.51){\rule{2.650pt}{1.200pt}}
\multiput(1153.00,618.51)(5.500,2.000){2}{\rule{1.325pt}{1.200pt}}
\put(1164,621.51){\rule{2.650pt}{1.200pt}}
\multiput(1164.00,620.51)(5.500,2.000){2}{\rule{1.325pt}{1.200pt}}
\put(1175,623.01){\rule{2.650pt}{1.200pt}}
\multiput(1175.00,622.51)(5.500,1.000){2}{\rule{1.325pt}{1.200pt}}
\put(1186,624.51){\rule{2.650pt}{1.200pt}}
\multiput(1186.00,623.51)(5.500,2.000){2}{\rule{1.325pt}{1.200pt}}
\put(1197,626.51){\rule{2.650pt}{1.200pt}}
\multiput(1197.00,625.51)(5.500,2.000){2}{\rule{1.325pt}{1.200pt}}
\put(1208,628.01){\rule{2.650pt}{1.200pt}}
\multiput(1208.00,627.51)(5.500,1.000){2}{\rule{1.325pt}{1.200pt}}
\put(1219,629.51){\rule{2.650pt}{1.200pt}}
\multiput(1219.00,628.51)(5.500,2.000){2}{\rule{1.325pt}{1.200pt}}
\put(1230,631.01){\rule{2.650pt}{1.200pt}}
\multiput(1230.00,630.51)(5.500,1.000){2}{\rule{1.325pt}{1.200pt}}
\put(1241,632.51){\rule{2.650pt}{1.200pt}}
\multiput(1241.00,631.51)(5.500,2.000){2}{\rule{1.325pt}{1.200pt}}
\put(1252,634.51){\rule{2.650pt}{1.200pt}}
\multiput(1252.00,633.51)(5.500,2.000){2}{\rule{1.325pt}{1.200pt}}
\put(1263,636.01){\rule{2.650pt}{1.200pt}}
\multiput(1263.00,635.51)(5.500,1.000){2}{\rule{1.325pt}{1.200pt}}
\put(1274,637.51){\rule{2.650pt}{1.200pt}}
\multiput(1274.00,636.51)(5.500,2.000){2}{\rule{1.325pt}{1.200pt}}
\put(1285,639.01){\rule{2.650pt}{1.200pt}}
\multiput(1285.00,638.51)(5.500,1.000){2}{\rule{1.325pt}{1.200pt}}
\put(1296,640.01){\rule{2.650pt}{1.200pt}}
\multiput(1296.00,639.51)(5.500,1.000){2}{\rule{1.325pt}{1.200pt}}
\put(1307,641.51){\rule{2.650pt}{1.200pt}}
\multiput(1307.00,640.51)(5.500,2.000){2}{\rule{1.325pt}{1.200pt}}
\put(1318,643.01){\rule{2.650pt}{1.200pt}}
\multiput(1318.00,642.51)(5.500,1.000){2}{\rule{1.325pt}{1.200pt}}
\put(1329,644.51){\rule{2.650pt}{1.200pt}}
\multiput(1329.00,643.51)(5.500,2.000){2}{\rule{1.325pt}{1.200pt}}
\put(1340,646.01){\rule{2.650pt}{1.200pt}}
\multiput(1340.00,645.51)(5.500,1.000){2}{\rule{1.325pt}{1.200pt}}
\put(1351,647.01){\rule{2.650pt}{1.200pt}}
\multiput(1351.00,646.51)(5.500,1.000){2}{\rule{1.325pt}{1.200pt}}
\put(1362,648.01){\rule{2.650pt}{1.200pt}}
\multiput(1362.00,647.51)(5.500,1.000){2}{\rule{1.325pt}{1.200pt}}
\put(1373,649.51){\rule{2.650pt}{1.200pt}}
\multiput(1373.00,648.51)(5.500,2.000){2}{\rule{1.325pt}{1.200pt}}
\put(1384,651.01){\rule{2.650pt}{1.200pt}}
\multiput(1384.00,650.51)(5.500,1.000){2}{\rule{1.325pt}{1.200pt}}
\put(1395,652.01){\rule{2.650pt}{1.200pt}}
\multiput(1395.00,651.51)(5.500,1.000){2}{\rule{1.325pt}{1.200pt}}
\put(1406,653.01){\rule{2.650pt}{1.200pt}}
\multiput(1406.00,652.51)(5.500,1.000){2}{\rule{1.325pt}{1.200pt}}
\put(1417,654.01){\rule{2.409pt}{1.200pt}}
\multiput(1417.00,653.51)(5.000,1.000){2}{\rule{1.204pt}{1.200pt}}
\put(1427,655.01){\rule{2.650pt}{1.200pt}}
\multiput(1427.00,654.51)(5.500,1.000){2}{\rule{1.325pt}{1.200pt}}
\put(1438.0,658.0){\usebox{\plotpoint}}
\sbox{\plotpoint}{\rule[-0.200pt]{0.400pt}{0.400pt}}%
\put(371.0,82.0){\rule[-0.200pt]{0.400pt}{187.179pt}}
\put(371.0,82.0){\rule[-0.200pt]{257.281pt}{0.400pt}}
\put(1439.0,82.0){\rule[-0.200pt]{0.400pt}{187.179pt}}
\put(371.0,859.0){\rule[-0.200pt]{257.281pt}{0.400pt}}
\end{picture}

\caption{Wykres zależności współczynnika $\Delta((XY)V-X(YV))$ od wielkości macierzy w arytmetyce double}
\end{center}
\end{figure}
%**********************************************************************

Ten sam współczynnik przebadaliśmy wykonując obliczenia za pomocą arytmetyki single, a wyniki tego doświadczenia prezentuje rysunek {\bf 3.8}.
Jak widać, z przyczyny zmniejszenia się dokładności arytmetyki, uzyskane wyniki mają wartości nawet $10^{16}$ razy większe w porównaniu do poprzednio użytej
arytmetyki, jednakże kształt wykresu pozostał niezmieniony. Zmiana arytmetyki nie wpływa więc na zmianę wzajemnej zależności wartości współczynnika dla badanych algorytmów.
\begin{figure}[h!tb]
\begin{center}
% GNUPLOT: LaTeX picture
\setlength{\unitlength}{0.240900pt}
\ifx\plotpoint\undefined\newsavebox{\plotpoint}\fi
\sbox{\plotpoint}{\rule[-0.200pt]{0.400pt}{0.400pt}}%
\begin{picture}(1500,900)(0,0)
\sbox{\plotpoint}{\rule[-0.200pt]{0.400pt}{0.400pt}}%
\put(170.0,82.0){\rule[-0.200pt]{4.818pt}{0.400pt}}
\put(150,82){\makebox(0,0)[r]{ 1e-18}}
\put(1419.0,82.0){\rule[-0.200pt]{4.818pt}{0.400pt}}
\put(170.0,93.0){\rule[-0.200pt]{2.409pt}{0.400pt}}
\put(1429.0,93.0){\rule[-0.200pt]{2.409pt}{0.400pt}}
\put(170.0,108.0){\rule[-0.200pt]{2.409pt}{0.400pt}}
\put(1429.0,108.0){\rule[-0.200pt]{2.409pt}{0.400pt}}
\put(170.0,115.0){\rule[-0.200pt]{2.409pt}{0.400pt}}
\put(1429.0,115.0){\rule[-0.200pt]{2.409pt}{0.400pt}}
\put(170.0,121.0){\rule[-0.200pt]{2.409pt}{0.400pt}}
\put(1429.0,121.0){\rule[-0.200pt]{2.409pt}{0.400pt}}
\put(170.0,124.0){\rule[-0.200pt]{2.409pt}{0.400pt}}
\put(1429.0,124.0){\rule[-0.200pt]{2.409pt}{0.400pt}}
\put(170.0,128.0){\rule[-0.200pt]{2.409pt}{0.400pt}}
\put(1429.0,128.0){\rule[-0.200pt]{2.409pt}{0.400pt}}
\put(170.0,130.0){\rule[-0.200pt]{2.409pt}{0.400pt}}
\put(1429.0,130.0){\rule[-0.200pt]{2.409pt}{0.400pt}}
\put(170.0,132.0){\rule[-0.200pt]{2.409pt}{0.400pt}}
\put(1429.0,132.0){\rule[-0.200pt]{2.409pt}{0.400pt}}
\put(170.0,134.0){\rule[-0.200pt]{2.409pt}{0.400pt}}
\put(1429.0,134.0){\rule[-0.200pt]{2.409pt}{0.400pt}}
\put(170.0,136.0){\rule[-0.200pt]{2.409pt}{0.400pt}}
\put(1429.0,136.0){\rule[-0.200pt]{2.409pt}{0.400pt}}
\put(170.0,138.0){\rule[-0.200pt]{2.409pt}{0.400pt}}
\put(1429.0,138.0){\rule[-0.200pt]{2.409pt}{0.400pt}}
\put(170.0,139.0){\rule[-0.200pt]{2.409pt}{0.400pt}}
\put(1429.0,139.0){\rule[-0.200pt]{2.409pt}{0.400pt}}
\put(170.0,140.0){\rule[-0.200pt]{2.409pt}{0.400pt}}
\put(1429.0,140.0){\rule[-0.200pt]{2.409pt}{0.400pt}}
\put(170.0,142.0){\rule[-0.200pt]{2.409pt}{0.400pt}}
\put(1429.0,142.0){\rule[-0.200pt]{2.409pt}{0.400pt}}
\put(170.0,143.0){\rule[-0.200pt]{2.409pt}{0.400pt}}
\put(1429.0,143.0){\rule[-0.200pt]{2.409pt}{0.400pt}}
\put(170.0,144.0){\rule[-0.200pt]{2.409pt}{0.400pt}}
\put(1429.0,144.0){\rule[-0.200pt]{2.409pt}{0.400pt}}
\put(170.0,145.0){\rule[-0.200pt]{2.409pt}{0.400pt}}
\put(1429.0,145.0){\rule[-0.200pt]{2.409pt}{0.400pt}}
\put(170.0,146.0){\rule[-0.200pt]{2.409pt}{0.400pt}}
\put(1429.0,146.0){\rule[-0.200pt]{2.409pt}{0.400pt}}
\put(170.0,147.0){\rule[-0.200pt]{2.409pt}{0.400pt}}
\put(1429.0,147.0){\rule[-0.200pt]{2.409pt}{0.400pt}}
\put(170.0,148.0){\rule[-0.200pt]{2.409pt}{0.400pt}}
\put(1429.0,148.0){\rule[-0.200pt]{2.409pt}{0.400pt}}
\put(170.0,148.0){\rule[-0.200pt]{2.409pt}{0.400pt}}
\put(1429.0,148.0){\rule[-0.200pt]{2.409pt}{0.400pt}}
\put(170.0,149.0){\rule[-0.200pt]{2.409pt}{0.400pt}}
\put(1429.0,149.0){\rule[-0.200pt]{2.409pt}{0.400pt}}
\put(170.0,150.0){\rule[-0.200pt]{2.409pt}{0.400pt}}
\put(1429.0,150.0){\rule[-0.200pt]{2.409pt}{0.400pt}}
\put(170.0,150.0){\rule[-0.200pt]{2.409pt}{0.400pt}}
\put(1429.0,150.0){\rule[-0.200pt]{2.409pt}{0.400pt}}
\put(170.0,151.0){\rule[-0.200pt]{2.409pt}{0.400pt}}
\put(1429.0,151.0){\rule[-0.200pt]{2.409pt}{0.400pt}}
\put(170.0,152.0){\rule[-0.200pt]{2.409pt}{0.400pt}}
\put(1429.0,152.0){\rule[-0.200pt]{2.409pt}{0.400pt}}
\put(170.0,152.0){\rule[-0.200pt]{2.409pt}{0.400pt}}
\put(1429.0,152.0){\rule[-0.200pt]{2.409pt}{0.400pt}}
\put(170.0,153.0){\rule[-0.200pt]{2.409pt}{0.400pt}}
\put(1429.0,153.0){\rule[-0.200pt]{2.409pt}{0.400pt}}
\put(170.0,154.0){\rule[-0.200pt]{2.409pt}{0.400pt}}
\put(1429.0,154.0){\rule[-0.200pt]{2.409pt}{0.400pt}}
\put(170.0,154.0){\rule[-0.200pt]{2.409pt}{0.400pt}}
\put(1429.0,154.0){\rule[-0.200pt]{2.409pt}{0.400pt}}
\put(170.0,155.0){\rule[-0.200pt]{2.409pt}{0.400pt}}
\put(1429.0,155.0){\rule[-0.200pt]{2.409pt}{0.400pt}}
\put(170.0,155.0){\rule[-0.200pt]{2.409pt}{0.400pt}}
\put(1429.0,155.0){\rule[-0.200pt]{2.409pt}{0.400pt}}
\put(170.0,156.0){\rule[-0.200pt]{2.409pt}{0.400pt}}
\put(1429.0,156.0){\rule[-0.200pt]{2.409pt}{0.400pt}}
\put(170.0,156.0){\rule[-0.200pt]{2.409pt}{0.400pt}}
\put(1429.0,156.0){\rule[-0.200pt]{2.409pt}{0.400pt}}
\put(170.0,157.0){\rule[-0.200pt]{2.409pt}{0.400pt}}
\put(1429.0,157.0){\rule[-0.200pt]{2.409pt}{0.400pt}}
\put(170.0,157.0){\rule[-0.200pt]{2.409pt}{0.400pt}}
\put(1429.0,157.0){\rule[-0.200pt]{2.409pt}{0.400pt}}
\put(170.0,158.0){\rule[-0.200pt]{2.409pt}{0.400pt}}
\put(1429.0,158.0){\rule[-0.200pt]{2.409pt}{0.400pt}}
\put(170.0,158.0){\rule[-0.200pt]{2.409pt}{0.400pt}}
\put(1429.0,158.0){\rule[-0.200pt]{2.409pt}{0.400pt}}
\put(170.0,158.0){\rule[-0.200pt]{2.409pt}{0.400pt}}
\put(1429.0,158.0){\rule[-0.200pt]{2.409pt}{0.400pt}}
\put(170.0,159.0){\rule[-0.200pt]{2.409pt}{0.400pt}}
\put(1429.0,159.0){\rule[-0.200pt]{2.409pt}{0.400pt}}
\put(170.0,159.0){\rule[-0.200pt]{2.409pt}{0.400pt}}
\put(1429.0,159.0){\rule[-0.200pt]{2.409pt}{0.400pt}}
\put(170.0,160.0){\rule[-0.200pt]{2.409pt}{0.400pt}}
\put(1429.0,160.0){\rule[-0.200pt]{2.409pt}{0.400pt}}
\put(170.0,160.0){\rule[-0.200pt]{2.409pt}{0.400pt}}
\put(1429.0,160.0){\rule[-0.200pt]{2.409pt}{0.400pt}}
\put(170.0,160.0){\rule[-0.200pt]{2.409pt}{0.400pt}}
\put(1429.0,160.0){\rule[-0.200pt]{2.409pt}{0.400pt}}
\put(170.0,161.0){\rule[-0.200pt]{2.409pt}{0.400pt}}
\put(1429.0,161.0){\rule[-0.200pt]{2.409pt}{0.400pt}}
\put(170.0,161.0){\rule[-0.200pt]{2.409pt}{0.400pt}}
\put(1429.0,161.0){\rule[-0.200pt]{2.409pt}{0.400pt}}
\put(170.0,161.0){\rule[-0.200pt]{2.409pt}{0.400pt}}
\put(1429.0,161.0){\rule[-0.200pt]{2.409pt}{0.400pt}}
\put(170.0,162.0){\rule[-0.200pt]{2.409pt}{0.400pt}}
\put(1429.0,162.0){\rule[-0.200pt]{2.409pt}{0.400pt}}
\put(170.0,162.0){\rule[-0.200pt]{2.409pt}{0.400pt}}
\put(1429.0,162.0){\rule[-0.200pt]{2.409pt}{0.400pt}}
\put(170.0,162.0){\rule[-0.200pt]{2.409pt}{0.400pt}}
\put(1429.0,162.0){\rule[-0.200pt]{2.409pt}{0.400pt}}
\put(170.0,163.0){\rule[-0.200pt]{2.409pt}{0.400pt}}
\put(1429.0,163.0){\rule[-0.200pt]{2.409pt}{0.400pt}}
\put(170.0,163.0){\rule[-0.200pt]{2.409pt}{0.400pt}}
\put(1429.0,163.0){\rule[-0.200pt]{2.409pt}{0.400pt}}
\put(170.0,163.0){\rule[-0.200pt]{2.409pt}{0.400pt}}
\put(1429.0,163.0){\rule[-0.200pt]{2.409pt}{0.400pt}}
\put(170.0,164.0){\rule[-0.200pt]{2.409pt}{0.400pt}}
\put(1429.0,164.0){\rule[-0.200pt]{2.409pt}{0.400pt}}
\put(170.0,164.0){\rule[-0.200pt]{2.409pt}{0.400pt}}
\put(1429.0,164.0){\rule[-0.200pt]{2.409pt}{0.400pt}}
\put(170.0,164.0){\rule[-0.200pt]{2.409pt}{0.400pt}}
\put(1429.0,164.0){\rule[-0.200pt]{2.409pt}{0.400pt}}
\put(170.0,165.0){\rule[-0.200pt]{2.409pt}{0.400pt}}
\put(1429.0,165.0){\rule[-0.200pt]{2.409pt}{0.400pt}}
\put(170.0,165.0){\rule[-0.200pt]{2.409pt}{0.400pt}}
\put(1429.0,165.0){\rule[-0.200pt]{2.409pt}{0.400pt}}
\put(170.0,165.0){\rule[-0.200pt]{2.409pt}{0.400pt}}
\put(1429.0,165.0){\rule[-0.200pt]{2.409pt}{0.400pt}}
\put(170.0,165.0){\rule[-0.200pt]{2.409pt}{0.400pt}}
\put(1429.0,165.0){\rule[-0.200pt]{2.409pt}{0.400pt}}
\put(170.0,166.0){\rule[-0.200pt]{2.409pt}{0.400pt}}
\put(1429.0,166.0){\rule[-0.200pt]{2.409pt}{0.400pt}}
\put(170.0,166.0){\rule[-0.200pt]{2.409pt}{0.400pt}}
\put(1429.0,166.0){\rule[-0.200pt]{2.409pt}{0.400pt}}
\put(170.0,166.0){\rule[-0.200pt]{2.409pt}{0.400pt}}
\put(1429.0,166.0){\rule[-0.200pt]{2.409pt}{0.400pt}}
\put(170.0,166.0){\rule[-0.200pt]{2.409pt}{0.400pt}}
\put(1429.0,166.0){\rule[-0.200pt]{2.409pt}{0.400pt}}
\put(170.0,167.0){\rule[-0.200pt]{2.409pt}{0.400pt}}
\put(1429.0,167.0){\rule[-0.200pt]{2.409pt}{0.400pt}}
\put(170.0,167.0){\rule[-0.200pt]{2.409pt}{0.400pt}}
\put(1429.0,167.0){\rule[-0.200pt]{2.409pt}{0.400pt}}
\put(170.0,167.0){\rule[-0.200pt]{2.409pt}{0.400pt}}
\put(1429.0,167.0){\rule[-0.200pt]{2.409pt}{0.400pt}}
\put(170.0,167.0){\rule[-0.200pt]{2.409pt}{0.400pt}}
\put(1429.0,167.0){\rule[-0.200pt]{2.409pt}{0.400pt}}
\put(170.0,168.0){\rule[-0.200pt]{2.409pt}{0.400pt}}
\put(1429.0,168.0){\rule[-0.200pt]{2.409pt}{0.400pt}}
\put(170.0,168.0){\rule[-0.200pt]{2.409pt}{0.400pt}}
\put(1429.0,168.0){\rule[-0.200pt]{2.409pt}{0.400pt}}
\put(170.0,168.0){\rule[-0.200pt]{2.409pt}{0.400pt}}
\put(1429.0,168.0){\rule[-0.200pt]{2.409pt}{0.400pt}}
\put(170.0,168.0){\rule[-0.200pt]{2.409pt}{0.400pt}}
\put(1429.0,168.0){\rule[-0.200pt]{2.409pt}{0.400pt}}
\put(170.0,169.0){\rule[-0.200pt]{2.409pt}{0.400pt}}
\put(1429.0,169.0){\rule[-0.200pt]{2.409pt}{0.400pt}}
\put(170.0,169.0){\rule[-0.200pt]{2.409pt}{0.400pt}}
\put(1429.0,169.0){\rule[-0.200pt]{2.409pt}{0.400pt}}
\put(170.0,169.0){\rule[-0.200pt]{2.409pt}{0.400pt}}
\put(1429.0,169.0){\rule[-0.200pt]{2.409pt}{0.400pt}}
\put(170.0,169.0){\rule[-0.200pt]{2.409pt}{0.400pt}}
\put(1429.0,169.0){\rule[-0.200pt]{2.409pt}{0.400pt}}
\put(170.0,169.0){\rule[-0.200pt]{2.409pt}{0.400pt}}
\put(1429.0,169.0){\rule[-0.200pt]{2.409pt}{0.400pt}}
\put(170.0,170.0){\rule[-0.200pt]{2.409pt}{0.400pt}}
\put(1429.0,170.0){\rule[-0.200pt]{2.409pt}{0.400pt}}
\put(170.0,170.0){\rule[-0.200pt]{2.409pt}{0.400pt}}
\put(1429.0,170.0){\rule[-0.200pt]{2.409pt}{0.400pt}}
\put(170.0,170.0){\rule[-0.200pt]{2.409pt}{0.400pt}}
\put(1429.0,170.0){\rule[-0.200pt]{2.409pt}{0.400pt}}
\put(170.0,170.0){\rule[-0.200pt]{2.409pt}{0.400pt}}
\put(1429.0,170.0){\rule[-0.200pt]{2.409pt}{0.400pt}}
\put(170.0,170.0){\rule[-0.200pt]{2.409pt}{0.400pt}}
\put(1429.0,170.0){\rule[-0.200pt]{2.409pt}{0.400pt}}
\put(170.0,171.0){\rule[-0.200pt]{2.409pt}{0.400pt}}
\put(1429.0,171.0){\rule[-0.200pt]{2.409pt}{0.400pt}}
\put(170.0,171.0){\rule[-0.200pt]{2.409pt}{0.400pt}}
\put(1429.0,171.0){\rule[-0.200pt]{2.409pt}{0.400pt}}
\put(170.0,171.0){\rule[-0.200pt]{2.409pt}{0.400pt}}
\put(1429.0,171.0){\rule[-0.200pt]{2.409pt}{0.400pt}}
\put(170.0,171.0){\rule[-0.200pt]{2.409pt}{0.400pt}}
\put(1429.0,171.0){\rule[-0.200pt]{2.409pt}{0.400pt}}
\put(170.0,171.0){\rule[-0.200pt]{2.409pt}{0.400pt}}
\put(1429.0,171.0){\rule[-0.200pt]{2.409pt}{0.400pt}}
\put(170.0,172.0){\rule[-0.200pt]{2.409pt}{0.400pt}}
\put(1429.0,172.0){\rule[-0.200pt]{2.409pt}{0.400pt}}
\put(170.0,172.0){\rule[-0.200pt]{2.409pt}{0.400pt}}
\put(1429.0,172.0){\rule[-0.200pt]{2.409pt}{0.400pt}}
\put(170.0,172.0){\rule[-0.200pt]{2.409pt}{0.400pt}}
\put(1429.0,172.0){\rule[-0.200pt]{2.409pt}{0.400pt}}
\put(170.0,172.0){\rule[-0.200pt]{2.409pt}{0.400pt}}
\put(1429.0,172.0){\rule[-0.200pt]{2.409pt}{0.400pt}}
\put(170.0,172.0){\rule[-0.200pt]{2.409pt}{0.400pt}}
\put(1429.0,172.0){\rule[-0.200pt]{2.409pt}{0.400pt}}
\put(170.0,172.0){\rule[-0.200pt]{2.409pt}{0.400pt}}
\put(1429.0,172.0){\rule[-0.200pt]{2.409pt}{0.400pt}}
\put(170.0,173.0){\rule[-0.200pt]{2.409pt}{0.400pt}}
\put(1429.0,173.0){\rule[-0.200pt]{2.409pt}{0.400pt}}
\put(170.0,173.0){\rule[-0.200pt]{2.409pt}{0.400pt}}
\put(1429.0,173.0){\rule[-0.200pt]{2.409pt}{0.400pt}}
\put(170.0,173.0){\rule[-0.200pt]{2.409pt}{0.400pt}}
\put(1429.0,173.0){\rule[-0.200pt]{2.409pt}{0.400pt}}
\put(170.0,173.0){\rule[-0.200pt]{2.409pt}{0.400pt}}
\put(1429.0,173.0){\rule[-0.200pt]{2.409pt}{0.400pt}}
\put(170.0,173.0){\rule[-0.200pt]{2.409pt}{0.400pt}}
\put(1429.0,173.0){\rule[-0.200pt]{2.409pt}{0.400pt}}
\put(170.0,173.0){\rule[-0.200pt]{2.409pt}{0.400pt}}
\put(1429.0,173.0){\rule[-0.200pt]{2.409pt}{0.400pt}}
\put(170.0,174.0){\rule[-0.200pt]{2.409pt}{0.400pt}}
\put(1429.0,174.0){\rule[-0.200pt]{2.409pt}{0.400pt}}
\put(170.0,174.0){\rule[-0.200pt]{2.409pt}{0.400pt}}
\put(1429.0,174.0){\rule[-0.200pt]{2.409pt}{0.400pt}}
\put(170.0,174.0){\rule[-0.200pt]{2.409pt}{0.400pt}}
\put(1429.0,174.0){\rule[-0.200pt]{2.409pt}{0.400pt}}
\put(170.0,174.0){\rule[-0.200pt]{2.409pt}{0.400pt}}
\put(1429.0,174.0){\rule[-0.200pt]{2.409pt}{0.400pt}}
\put(170.0,174.0){\rule[-0.200pt]{2.409pt}{0.400pt}}
\put(1429.0,174.0){\rule[-0.200pt]{2.409pt}{0.400pt}}
\put(170.0,174.0){\rule[-0.200pt]{2.409pt}{0.400pt}}
\put(1429.0,174.0){\rule[-0.200pt]{2.409pt}{0.400pt}}
\put(170.0,175.0){\rule[-0.200pt]{2.409pt}{0.400pt}}
\put(1429.0,175.0){\rule[-0.200pt]{2.409pt}{0.400pt}}
\put(170.0,175.0){\rule[-0.200pt]{2.409pt}{0.400pt}}
\put(1429.0,175.0){\rule[-0.200pt]{2.409pt}{0.400pt}}
\put(170.0,175.0){\rule[-0.200pt]{2.409pt}{0.400pt}}
\put(1429.0,175.0){\rule[-0.200pt]{2.409pt}{0.400pt}}
\put(170.0,175.0){\rule[-0.200pt]{2.409pt}{0.400pt}}
\put(1429.0,175.0){\rule[-0.200pt]{2.409pt}{0.400pt}}
\put(170.0,175.0){\rule[-0.200pt]{2.409pt}{0.400pt}}
\put(1429.0,175.0){\rule[-0.200pt]{2.409pt}{0.400pt}}
\put(170.0,175.0){\rule[-0.200pt]{2.409pt}{0.400pt}}
\put(1429.0,175.0){\rule[-0.200pt]{2.409pt}{0.400pt}}
\put(170.0,175.0){\rule[-0.200pt]{2.409pt}{0.400pt}}
\put(1429.0,175.0){\rule[-0.200pt]{2.409pt}{0.400pt}}
\put(170.0,176.0){\rule[-0.200pt]{2.409pt}{0.400pt}}
\put(1429.0,176.0){\rule[-0.200pt]{2.409pt}{0.400pt}}
\put(170.0,176.0){\rule[-0.200pt]{2.409pt}{0.400pt}}
\put(1429.0,176.0){\rule[-0.200pt]{2.409pt}{0.400pt}}
\put(170.0,176.0){\rule[-0.200pt]{2.409pt}{0.400pt}}
\put(1429.0,176.0){\rule[-0.200pt]{2.409pt}{0.400pt}}
\put(170.0,176.0){\rule[-0.200pt]{2.409pt}{0.400pt}}
\put(1429.0,176.0){\rule[-0.200pt]{2.409pt}{0.400pt}}
\put(170.0,176.0){\rule[-0.200pt]{2.409pt}{0.400pt}}
\put(1429.0,176.0){\rule[-0.200pt]{2.409pt}{0.400pt}}
\put(170.0,176.0){\rule[-0.200pt]{2.409pt}{0.400pt}}
\put(1429.0,176.0){\rule[-0.200pt]{2.409pt}{0.400pt}}
\put(170.0,176.0){\rule[-0.200pt]{2.409pt}{0.400pt}}
\put(1429.0,176.0){\rule[-0.200pt]{2.409pt}{0.400pt}}
\put(170.0,177.0){\rule[-0.200pt]{2.409pt}{0.400pt}}
\put(1429.0,177.0){\rule[-0.200pt]{2.409pt}{0.400pt}}
\put(170.0,177.0){\rule[-0.200pt]{2.409pt}{0.400pt}}
\put(1429.0,177.0){\rule[-0.200pt]{2.409pt}{0.400pt}}
\put(170.0,177.0){\rule[-0.200pt]{2.409pt}{0.400pt}}
\put(1429.0,177.0){\rule[-0.200pt]{2.409pt}{0.400pt}}
\put(170.0,177.0){\rule[-0.200pt]{2.409pt}{0.400pt}}
\put(1429.0,177.0){\rule[-0.200pt]{2.409pt}{0.400pt}}
\put(170.0,177.0){\rule[-0.200pt]{2.409pt}{0.400pt}}
\put(1429.0,177.0){\rule[-0.200pt]{2.409pt}{0.400pt}}
\put(170.0,177.0){\rule[-0.200pt]{2.409pt}{0.400pt}}
\put(1429.0,177.0){\rule[-0.200pt]{2.409pt}{0.400pt}}
\put(170.0,177.0){\rule[-0.200pt]{2.409pt}{0.400pt}}
\put(1429.0,177.0){\rule[-0.200pt]{2.409pt}{0.400pt}}
\put(170.0,177.0){\rule[-0.200pt]{2.409pt}{0.400pt}}
\put(1429.0,177.0){\rule[-0.200pt]{2.409pt}{0.400pt}}
\put(170.0,178.0){\rule[-0.200pt]{2.409pt}{0.400pt}}
\put(1429.0,178.0){\rule[-0.200pt]{2.409pt}{0.400pt}}
\put(170.0,178.0){\rule[-0.200pt]{2.409pt}{0.400pt}}
\put(1429.0,178.0){\rule[-0.200pt]{2.409pt}{0.400pt}}
\put(170.0,178.0){\rule[-0.200pt]{2.409pt}{0.400pt}}
\put(1429.0,178.0){\rule[-0.200pt]{2.409pt}{0.400pt}}
\put(170.0,178.0){\rule[-0.200pt]{2.409pt}{0.400pt}}
\put(1429.0,178.0){\rule[-0.200pt]{2.409pt}{0.400pt}}
\put(170.0,178.0){\rule[-0.200pt]{2.409pt}{0.400pt}}
\put(1429.0,178.0){\rule[-0.200pt]{2.409pt}{0.400pt}}
\put(170.0,178.0){\rule[-0.200pt]{2.409pt}{0.400pt}}
\put(1429.0,178.0){\rule[-0.200pt]{2.409pt}{0.400pt}}
\put(170.0,178.0){\rule[-0.200pt]{2.409pt}{0.400pt}}
\put(1429.0,178.0){\rule[-0.200pt]{2.409pt}{0.400pt}}
\put(170.0,178.0){\rule[-0.200pt]{2.409pt}{0.400pt}}
\put(1429.0,178.0){\rule[-0.200pt]{2.409pt}{0.400pt}}
\put(170.0,179.0){\rule[-0.200pt]{2.409pt}{0.400pt}}
\put(1429.0,179.0){\rule[-0.200pt]{2.409pt}{0.400pt}}
\put(170.0,179.0){\rule[-0.200pt]{2.409pt}{0.400pt}}
\put(1429.0,179.0){\rule[-0.200pt]{2.409pt}{0.400pt}}
\put(170.0,179.0){\rule[-0.200pt]{2.409pt}{0.400pt}}
\put(1429.0,179.0){\rule[-0.200pt]{2.409pt}{0.400pt}}
\put(170.0,179.0){\rule[-0.200pt]{2.409pt}{0.400pt}}
\put(1429.0,179.0){\rule[-0.200pt]{2.409pt}{0.400pt}}
\put(170.0,179.0){\rule[-0.200pt]{2.409pt}{0.400pt}}
\put(1429.0,179.0){\rule[-0.200pt]{2.409pt}{0.400pt}}
\put(170.0,179.0){\rule[-0.200pt]{2.409pt}{0.400pt}}
\put(1429.0,179.0){\rule[-0.200pt]{2.409pt}{0.400pt}}
\put(170.0,179.0){\rule[-0.200pt]{2.409pt}{0.400pt}}
\put(1429.0,179.0){\rule[-0.200pt]{2.409pt}{0.400pt}}
\put(170.0,179.0){\rule[-0.200pt]{2.409pt}{0.400pt}}
\put(1429.0,179.0){\rule[-0.200pt]{2.409pt}{0.400pt}}
\put(170.0,179.0){\rule[-0.200pt]{2.409pt}{0.400pt}}
\put(1429.0,179.0){\rule[-0.200pt]{2.409pt}{0.400pt}}
\put(170.0,180.0){\rule[-0.200pt]{2.409pt}{0.400pt}}
\put(1429.0,180.0){\rule[-0.200pt]{2.409pt}{0.400pt}}
\put(170.0,180.0){\rule[-0.200pt]{2.409pt}{0.400pt}}
\put(1429.0,180.0){\rule[-0.200pt]{2.409pt}{0.400pt}}
\put(170.0,180.0){\rule[-0.200pt]{2.409pt}{0.400pt}}
\put(1429.0,180.0){\rule[-0.200pt]{2.409pt}{0.400pt}}
\put(170.0,180.0){\rule[-0.200pt]{2.409pt}{0.400pt}}
\put(1429.0,180.0){\rule[-0.200pt]{2.409pt}{0.400pt}}
\put(170.0,180.0){\rule[-0.200pt]{2.409pt}{0.400pt}}
\put(1429.0,180.0){\rule[-0.200pt]{2.409pt}{0.400pt}}
\put(170.0,180.0){\rule[-0.200pt]{2.409pt}{0.400pt}}
\put(1429.0,180.0){\rule[-0.200pt]{2.409pt}{0.400pt}}
\put(170.0,180.0){\rule[-0.200pt]{2.409pt}{0.400pt}}
\put(1429.0,180.0){\rule[-0.200pt]{2.409pt}{0.400pt}}
\put(170.0,180.0){\rule[-0.200pt]{2.409pt}{0.400pt}}
\put(1429.0,180.0){\rule[-0.200pt]{2.409pt}{0.400pt}}
\put(170.0,180.0){\rule[-0.200pt]{2.409pt}{0.400pt}}
\put(1429.0,180.0){\rule[-0.200pt]{2.409pt}{0.400pt}}
\put(170.0,181.0){\rule[-0.200pt]{2.409pt}{0.400pt}}
\put(1429.0,181.0){\rule[-0.200pt]{2.409pt}{0.400pt}}
\put(170.0,181.0){\rule[-0.200pt]{2.409pt}{0.400pt}}
\put(1429.0,181.0){\rule[-0.200pt]{2.409pt}{0.400pt}}
\put(170.0,181.0){\rule[-0.200pt]{2.409pt}{0.400pt}}
\put(1429.0,181.0){\rule[-0.200pt]{2.409pt}{0.400pt}}
\put(170.0,181.0){\rule[-0.200pt]{2.409pt}{0.400pt}}
\put(1429.0,181.0){\rule[-0.200pt]{2.409pt}{0.400pt}}
\put(170.0,181.0){\rule[-0.200pt]{2.409pt}{0.400pt}}
\put(1429.0,181.0){\rule[-0.200pt]{2.409pt}{0.400pt}}
\put(170.0,181.0){\rule[-0.200pt]{2.409pt}{0.400pt}}
\put(1429.0,181.0){\rule[-0.200pt]{2.409pt}{0.400pt}}
\put(170.0,181.0){\rule[-0.200pt]{2.409pt}{0.400pt}}
\put(1429.0,181.0){\rule[-0.200pt]{2.409pt}{0.400pt}}
\put(170.0,181.0){\rule[-0.200pt]{2.409pt}{0.400pt}}
\put(1429.0,181.0){\rule[-0.200pt]{2.409pt}{0.400pt}}
\put(170.0,181.0){\rule[-0.200pt]{2.409pt}{0.400pt}}
\put(1429.0,181.0){\rule[-0.200pt]{2.409pt}{0.400pt}}
\put(170.0,181.0){\rule[-0.200pt]{2.409pt}{0.400pt}}
\put(1429.0,181.0){\rule[-0.200pt]{2.409pt}{0.400pt}}
\put(170.0,182.0){\rule[-0.200pt]{2.409pt}{0.400pt}}
\put(1429.0,182.0){\rule[-0.200pt]{2.409pt}{0.400pt}}
\put(170.0,182.0){\rule[-0.200pt]{2.409pt}{0.400pt}}
\put(1429.0,182.0){\rule[-0.200pt]{2.409pt}{0.400pt}}
\put(170.0,182.0){\rule[-0.200pt]{2.409pt}{0.400pt}}
\put(1429.0,182.0){\rule[-0.200pt]{2.409pt}{0.400pt}}
\put(170.0,182.0){\rule[-0.200pt]{2.409pt}{0.400pt}}
\put(1429.0,182.0){\rule[-0.200pt]{2.409pt}{0.400pt}}
\put(170.0,182.0){\rule[-0.200pt]{2.409pt}{0.400pt}}
\put(1429.0,182.0){\rule[-0.200pt]{2.409pt}{0.400pt}}
\put(170.0,182.0){\rule[-0.200pt]{2.409pt}{0.400pt}}
\put(1429.0,182.0){\rule[-0.200pt]{2.409pt}{0.400pt}}
\put(170.0,182.0){\rule[-0.200pt]{2.409pt}{0.400pt}}
\put(1429.0,182.0){\rule[-0.200pt]{2.409pt}{0.400pt}}
\put(170.0,182.0){\rule[-0.200pt]{2.409pt}{0.400pt}}
\put(1429.0,182.0){\rule[-0.200pt]{2.409pt}{0.400pt}}
\put(170.0,182.0){\rule[-0.200pt]{2.409pt}{0.400pt}}
\put(1429.0,182.0){\rule[-0.200pt]{2.409pt}{0.400pt}}
\put(170.0,182.0){\rule[-0.200pt]{2.409pt}{0.400pt}}
\put(1429.0,182.0){\rule[-0.200pt]{2.409pt}{0.400pt}}
\put(170.0,183.0){\rule[-0.200pt]{2.409pt}{0.400pt}}
\put(1429.0,183.0){\rule[-0.200pt]{2.409pt}{0.400pt}}
\put(170.0,183.0){\rule[-0.200pt]{2.409pt}{0.400pt}}
\put(1429.0,183.0){\rule[-0.200pt]{2.409pt}{0.400pt}}
\put(170.0,183.0){\rule[-0.200pt]{2.409pt}{0.400pt}}
\put(1429.0,183.0){\rule[-0.200pt]{2.409pt}{0.400pt}}
\put(170.0,183.0){\rule[-0.200pt]{2.409pt}{0.400pt}}
\put(1429.0,183.0){\rule[-0.200pt]{2.409pt}{0.400pt}}
\put(170.0,183.0){\rule[-0.200pt]{2.409pt}{0.400pt}}
\put(1429.0,183.0){\rule[-0.200pt]{2.409pt}{0.400pt}}
\put(170.0,183.0){\rule[-0.200pt]{2.409pt}{0.400pt}}
\put(1429.0,183.0){\rule[-0.200pt]{2.409pt}{0.400pt}}
\put(170.0,183.0){\rule[-0.200pt]{2.409pt}{0.400pt}}
\put(1429.0,183.0){\rule[-0.200pt]{2.409pt}{0.400pt}}
\put(170.0,183.0){\rule[-0.200pt]{2.409pt}{0.400pt}}
\put(1429.0,183.0){\rule[-0.200pt]{2.409pt}{0.400pt}}
\put(170.0,183.0){\rule[-0.200pt]{2.409pt}{0.400pt}}
\put(1429.0,183.0){\rule[-0.200pt]{2.409pt}{0.400pt}}
\put(170.0,183.0){\rule[-0.200pt]{2.409pt}{0.400pt}}
\put(1429.0,183.0){\rule[-0.200pt]{2.409pt}{0.400pt}}
\put(170.0,183.0){\rule[-0.200pt]{2.409pt}{0.400pt}}
\put(1429.0,183.0){\rule[-0.200pt]{2.409pt}{0.400pt}}
\put(170.0,184.0){\rule[-0.200pt]{2.409pt}{0.400pt}}
\put(1429.0,184.0){\rule[-0.200pt]{2.409pt}{0.400pt}}
\put(170.0,184.0){\rule[-0.200pt]{2.409pt}{0.400pt}}
\put(1429.0,184.0){\rule[-0.200pt]{2.409pt}{0.400pt}}
\put(170.0,184.0){\rule[-0.200pt]{2.409pt}{0.400pt}}
\put(1429.0,184.0){\rule[-0.200pt]{2.409pt}{0.400pt}}
\put(170.0,184.0){\rule[-0.200pt]{2.409pt}{0.400pt}}
\put(1429.0,184.0){\rule[-0.200pt]{2.409pt}{0.400pt}}
\put(170.0,184.0){\rule[-0.200pt]{2.409pt}{0.400pt}}
\put(1429.0,184.0){\rule[-0.200pt]{2.409pt}{0.400pt}}
\put(170.0,184.0){\rule[-0.200pt]{2.409pt}{0.400pt}}
\put(1429.0,184.0){\rule[-0.200pt]{2.409pt}{0.400pt}}
\put(170.0,184.0){\rule[-0.200pt]{2.409pt}{0.400pt}}
\put(1429.0,184.0){\rule[-0.200pt]{2.409pt}{0.400pt}}
\put(170.0,184.0){\rule[-0.200pt]{2.409pt}{0.400pt}}
\put(1429.0,184.0){\rule[-0.200pt]{2.409pt}{0.400pt}}
\put(170.0,184.0){\rule[-0.200pt]{2.409pt}{0.400pt}}
\put(1429.0,184.0){\rule[-0.200pt]{2.409pt}{0.400pt}}
\put(170.0,184.0){\rule[-0.200pt]{2.409pt}{0.400pt}}
\put(1429.0,184.0){\rule[-0.200pt]{2.409pt}{0.400pt}}
\put(170.0,184.0){\rule[-0.200pt]{2.409pt}{0.400pt}}
\put(1429.0,184.0){\rule[-0.200pt]{2.409pt}{0.400pt}}
\put(170.0,184.0){\rule[-0.200pt]{2.409pt}{0.400pt}}
\put(1429.0,184.0){\rule[-0.200pt]{2.409pt}{0.400pt}}
\put(170.0,185.0){\rule[-0.200pt]{2.409pt}{0.400pt}}
\put(1429.0,185.0){\rule[-0.200pt]{2.409pt}{0.400pt}}
\put(170.0,185.0){\rule[-0.200pt]{2.409pt}{0.400pt}}
\put(1429.0,185.0){\rule[-0.200pt]{2.409pt}{0.400pt}}
\put(170.0,185.0){\rule[-0.200pt]{2.409pt}{0.400pt}}
\put(1429.0,185.0){\rule[-0.200pt]{2.409pt}{0.400pt}}
\put(170.0,185.0){\rule[-0.200pt]{2.409pt}{0.400pt}}
\put(1429.0,185.0){\rule[-0.200pt]{2.409pt}{0.400pt}}
\put(170.0,185.0){\rule[-0.200pt]{2.409pt}{0.400pt}}
\put(1429.0,185.0){\rule[-0.200pt]{2.409pt}{0.400pt}}
\put(170.0,185.0){\rule[-0.200pt]{2.409pt}{0.400pt}}
\put(1429.0,185.0){\rule[-0.200pt]{2.409pt}{0.400pt}}
\put(170.0,185.0){\rule[-0.200pt]{2.409pt}{0.400pt}}
\put(1429.0,185.0){\rule[-0.200pt]{2.409pt}{0.400pt}}
\put(170.0,185.0){\rule[-0.200pt]{2.409pt}{0.400pt}}
\put(1429.0,185.0){\rule[-0.200pt]{2.409pt}{0.400pt}}
\put(170.0,185.0){\rule[-0.200pt]{2.409pt}{0.400pt}}
\put(1429.0,185.0){\rule[-0.200pt]{2.409pt}{0.400pt}}
\put(170.0,185.0){\rule[-0.200pt]{2.409pt}{0.400pt}}
\put(1429.0,185.0){\rule[-0.200pt]{2.409pt}{0.400pt}}
\put(170.0,185.0){\rule[-0.200pt]{2.409pt}{0.400pt}}
\put(1429.0,185.0){\rule[-0.200pt]{2.409pt}{0.400pt}}
\put(170.0,185.0){\rule[-0.200pt]{2.409pt}{0.400pt}}
\put(1429.0,185.0){\rule[-0.200pt]{2.409pt}{0.400pt}}
\put(170.0,185.0){\rule[-0.200pt]{2.409pt}{0.400pt}}
\put(1429.0,185.0){\rule[-0.200pt]{2.409pt}{0.400pt}}
\put(170.0,186.0){\rule[-0.200pt]{2.409pt}{0.400pt}}
\put(1429.0,186.0){\rule[-0.200pt]{2.409pt}{0.400pt}}
\put(170.0,186.0){\rule[-0.200pt]{2.409pt}{0.400pt}}
\put(1429.0,186.0){\rule[-0.200pt]{2.409pt}{0.400pt}}
\put(170.0,186.0){\rule[-0.200pt]{2.409pt}{0.400pt}}
\put(1429.0,186.0){\rule[-0.200pt]{2.409pt}{0.400pt}}
\put(170.0,186.0){\rule[-0.200pt]{2.409pt}{0.400pt}}
\put(1429.0,186.0){\rule[-0.200pt]{2.409pt}{0.400pt}}
\put(170.0,186.0){\rule[-0.200pt]{2.409pt}{0.400pt}}
\put(1429.0,186.0){\rule[-0.200pt]{2.409pt}{0.400pt}}
\put(170.0,186.0){\rule[-0.200pt]{2.409pt}{0.400pt}}
\put(1429.0,186.0){\rule[-0.200pt]{2.409pt}{0.400pt}}
\put(170.0,186.0){\rule[-0.200pt]{2.409pt}{0.400pt}}
\put(1429.0,186.0){\rule[-0.200pt]{2.409pt}{0.400pt}}
\put(170.0,186.0){\rule[-0.200pt]{2.409pt}{0.400pt}}
\put(1429.0,186.0){\rule[-0.200pt]{2.409pt}{0.400pt}}
\put(170.0,186.0){\rule[-0.200pt]{2.409pt}{0.400pt}}
\put(1429.0,186.0){\rule[-0.200pt]{2.409pt}{0.400pt}}
\put(170.0,186.0){\rule[-0.200pt]{2.409pt}{0.400pt}}
\put(1429.0,186.0){\rule[-0.200pt]{2.409pt}{0.400pt}}
\put(170.0,186.0){\rule[-0.200pt]{2.409pt}{0.400pt}}
\put(1429.0,186.0){\rule[-0.200pt]{2.409pt}{0.400pt}}
\put(170.0,186.0){\rule[-0.200pt]{2.409pt}{0.400pt}}
\put(1429.0,186.0){\rule[-0.200pt]{2.409pt}{0.400pt}}
\put(170.0,186.0){\rule[-0.200pt]{2.409pt}{0.400pt}}
\put(1429.0,186.0){\rule[-0.200pt]{2.409pt}{0.400pt}}
\put(170.0,187.0){\rule[-0.200pt]{2.409pt}{0.400pt}}
\put(1429.0,187.0){\rule[-0.200pt]{2.409pt}{0.400pt}}
\put(170.0,187.0){\rule[-0.200pt]{2.409pt}{0.400pt}}
\put(1429.0,187.0){\rule[-0.200pt]{2.409pt}{0.400pt}}
\put(170.0,187.0){\rule[-0.200pt]{2.409pt}{0.400pt}}
\put(1429.0,187.0){\rule[-0.200pt]{2.409pt}{0.400pt}}
\put(170.0,187.0){\rule[-0.200pt]{2.409pt}{0.400pt}}
\put(1429.0,187.0){\rule[-0.200pt]{2.409pt}{0.400pt}}
\put(170.0,187.0){\rule[-0.200pt]{2.409pt}{0.400pt}}
\put(1429.0,187.0){\rule[-0.200pt]{2.409pt}{0.400pt}}
\put(170.0,187.0){\rule[-0.200pt]{2.409pt}{0.400pt}}
\put(1429.0,187.0){\rule[-0.200pt]{2.409pt}{0.400pt}}
\put(170.0,187.0){\rule[-0.200pt]{2.409pt}{0.400pt}}
\put(1429.0,187.0){\rule[-0.200pt]{2.409pt}{0.400pt}}
\put(170.0,187.0){\rule[-0.200pt]{2.409pt}{0.400pt}}
\put(1429.0,187.0){\rule[-0.200pt]{2.409pt}{0.400pt}}
\put(170.0,187.0){\rule[-0.200pt]{2.409pt}{0.400pt}}
\put(1429.0,187.0){\rule[-0.200pt]{2.409pt}{0.400pt}}
\put(170.0,187.0){\rule[-0.200pt]{2.409pt}{0.400pt}}
\put(1429.0,187.0){\rule[-0.200pt]{2.409pt}{0.400pt}}
\put(170.0,187.0){\rule[-0.200pt]{2.409pt}{0.400pt}}
\put(1429.0,187.0){\rule[-0.200pt]{2.409pt}{0.400pt}}
\put(170.0,187.0){\rule[-0.200pt]{2.409pt}{0.400pt}}
\put(1429.0,187.0){\rule[-0.200pt]{2.409pt}{0.400pt}}
\put(170.0,187.0){\rule[-0.200pt]{2.409pt}{0.400pt}}
\put(1429.0,187.0){\rule[-0.200pt]{2.409pt}{0.400pt}}
\put(170.0,187.0){\rule[-0.200pt]{2.409pt}{0.400pt}}
\put(1429.0,187.0){\rule[-0.200pt]{2.409pt}{0.400pt}}
\put(170.0,187.0){\rule[-0.200pt]{2.409pt}{0.400pt}}
\put(1429.0,187.0){\rule[-0.200pt]{2.409pt}{0.400pt}}
\put(170.0,188.0){\rule[-0.200pt]{2.409pt}{0.400pt}}
\put(1429.0,188.0){\rule[-0.200pt]{2.409pt}{0.400pt}}
\put(170.0,188.0){\rule[-0.200pt]{2.409pt}{0.400pt}}
\put(1429.0,188.0){\rule[-0.200pt]{2.409pt}{0.400pt}}
\put(170.0,188.0){\rule[-0.200pt]{2.409pt}{0.400pt}}
\put(1429.0,188.0){\rule[-0.200pt]{2.409pt}{0.400pt}}
\put(170.0,188.0){\rule[-0.200pt]{2.409pt}{0.400pt}}
\put(1429.0,188.0){\rule[-0.200pt]{2.409pt}{0.400pt}}
\put(170.0,188.0){\rule[-0.200pt]{2.409pt}{0.400pt}}
\put(1429.0,188.0){\rule[-0.200pt]{2.409pt}{0.400pt}}
\put(170.0,188.0){\rule[-0.200pt]{2.409pt}{0.400pt}}
\put(1429.0,188.0){\rule[-0.200pt]{2.409pt}{0.400pt}}
\put(170.0,188.0){\rule[-0.200pt]{2.409pt}{0.400pt}}
\put(1429.0,188.0){\rule[-0.200pt]{2.409pt}{0.400pt}}
\put(170.0,188.0){\rule[-0.200pt]{2.409pt}{0.400pt}}
\put(1429.0,188.0){\rule[-0.200pt]{2.409pt}{0.400pt}}
\put(170.0,188.0){\rule[-0.200pt]{2.409pt}{0.400pt}}
\put(1429.0,188.0){\rule[-0.200pt]{2.409pt}{0.400pt}}
\put(170.0,188.0){\rule[-0.200pt]{2.409pt}{0.400pt}}
\put(1429.0,188.0){\rule[-0.200pt]{2.409pt}{0.400pt}}
\put(170.0,188.0){\rule[-0.200pt]{2.409pt}{0.400pt}}
\put(1429.0,188.0){\rule[-0.200pt]{2.409pt}{0.400pt}}
\put(170.0,188.0){\rule[-0.200pt]{2.409pt}{0.400pt}}
\put(1429.0,188.0){\rule[-0.200pt]{2.409pt}{0.400pt}}
\put(170.0,188.0){\rule[-0.200pt]{2.409pt}{0.400pt}}
\put(1429.0,188.0){\rule[-0.200pt]{2.409pt}{0.400pt}}
\put(170.0,188.0){\rule[-0.200pt]{2.409pt}{0.400pt}}
\put(1429.0,188.0){\rule[-0.200pt]{2.409pt}{0.400pt}}
\put(170.0,188.0){\rule[-0.200pt]{2.409pt}{0.400pt}}
\put(1429.0,188.0){\rule[-0.200pt]{2.409pt}{0.400pt}}
\put(170.0,189.0){\rule[-0.200pt]{2.409pt}{0.400pt}}
\put(1429.0,189.0){\rule[-0.200pt]{2.409pt}{0.400pt}}
\put(170.0,189.0){\rule[-0.200pt]{2.409pt}{0.400pt}}
\put(1429.0,189.0){\rule[-0.200pt]{2.409pt}{0.400pt}}
\put(170.0,189.0){\rule[-0.200pt]{2.409pt}{0.400pt}}
\put(1429.0,189.0){\rule[-0.200pt]{2.409pt}{0.400pt}}
\put(170.0,189.0){\rule[-0.200pt]{2.409pt}{0.400pt}}
\put(1429.0,189.0){\rule[-0.200pt]{2.409pt}{0.400pt}}
\put(170.0,189.0){\rule[-0.200pt]{2.409pt}{0.400pt}}
\put(1429.0,189.0){\rule[-0.200pt]{2.409pt}{0.400pt}}
\put(170.0,189.0){\rule[-0.200pt]{2.409pt}{0.400pt}}
\put(1429.0,189.0){\rule[-0.200pt]{2.409pt}{0.400pt}}
\put(170.0,189.0){\rule[-0.200pt]{2.409pt}{0.400pt}}
\put(1429.0,189.0){\rule[-0.200pt]{2.409pt}{0.400pt}}
\put(170.0,189.0){\rule[-0.200pt]{2.409pt}{0.400pt}}
\put(1429.0,189.0){\rule[-0.200pt]{2.409pt}{0.400pt}}
\put(170.0,189.0){\rule[-0.200pt]{2.409pt}{0.400pt}}
\put(1429.0,189.0){\rule[-0.200pt]{2.409pt}{0.400pt}}
\put(170.0,189.0){\rule[-0.200pt]{2.409pt}{0.400pt}}
\put(1429.0,189.0){\rule[-0.200pt]{2.409pt}{0.400pt}}
\put(170.0,189.0){\rule[-0.200pt]{2.409pt}{0.400pt}}
\put(1429.0,189.0){\rule[-0.200pt]{2.409pt}{0.400pt}}
\put(170.0,189.0){\rule[-0.200pt]{2.409pt}{0.400pt}}
\put(1429.0,189.0){\rule[-0.200pt]{2.409pt}{0.400pt}}
\put(170.0,189.0){\rule[-0.200pt]{2.409pt}{0.400pt}}
\put(1429.0,189.0){\rule[-0.200pt]{2.409pt}{0.400pt}}
\put(170.0,189.0){\rule[-0.200pt]{2.409pt}{0.400pt}}
\put(1429.0,189.0){\rule[-0.200pt]{2.409pt}{0.400pt}}
\put(170.0,189.0){\rule[-0.200pt]{2.409pt}{0.400pt}}
\put(1429.0,189.0){\rule[-0.200pt]{2.409pt}{0.400pt}}
\put(170.0,189.0){\rule[-0.200pt]{2.409pt}{0.400pt}}
\put(1429.0,189.0){\rule[-0.200pt]{2.409pt}{0.400pt}}
\put(170.0,190.0){\rule[-0.200pt]{2.409pt}{0.400pt}}
\put(1429.0,190.0){\rule[-0.200pt]{2.409pt}{0.400pt}}
\put(170.0,190.0){\rule[-0.200pt]{2.409pt}{0.400pt}}
\put(1429.0,190.0){\rule[-0.200pt]{2.409pt}{0.400pt}}
\put(170.0,190.0){\rule[-0.200pt]{2.409pt}{0.400pt}}
\put(1429.0,190.0){\rule[-0.200pt]{2.409pt}{0.400pt}}
\put(170.0,190.0){\rule[-0.200pt]{2.409pt}{0.400pt}}
\put(1429.0,190.0){\rule[-0.200pt]{2.409pt}{0.400pt}}
\put(170.0,190.0){\rule[-0.200pt]{2.409pt}{0.400pt}}
\put(1429.0,190.0){\rule[-0.200pt]{2.409pt}{0.400pt}}
\put(170.0,190.0){\rule[-0.200pt]{2.409pt}{0.400pt}}
\put(1429.0,190.0){\rule[-0.200pt]{2.409pt}{0.400pt}}
\put(170.0,190.0){\rule[-0.200pt]{2.409pt}{0.400pt}}
\put(1429.0,190.0){\rule[-0.200pt]{2.409pt}{0.400pt}}
\put(170.0,190.0){\rule[-0.200pt]{2.409pt}{0.400pt}}
\put(1429.0,190.0){\rule[-0.200pt]{2.409pt}{0.400pt}}
\put(170.0,190.0){\rule[-0.200pt]{2.409pt}{0.400pt}}
\put(1429.0,190.0){\rule[-0.200pt]{2.409pt}{0.400pt}}
\put(170.0,190.0){\rule[-0.200pt]{2.409pt}{0.400pt}}
\put(1429.0,190.0){\rule[-0.200pt]{2.409pt}{0.400pt}}
\put(170.0,190.0){\rule[-0.200pt]{2.409pt}{0.400pt}}
\put(1429.0,190.0){\rule[-0.200pt]{2.409pt}{0.400pt}}
\put(170.0,190.0){\rule[-0.200pt]{2.409pt}{0.400pt}}
\put(1429.0,190.0){\rule[-0.200pt]{2.409pt}{0.400pt}}
\put(170.0,190.0){\rule[-0.200pt]{2.409pt}{0.400pt}}
\put(1429.0,190.0){\rule[-0.200pt]{2.409pt}{0.400pt}}
\put(170.0,190.0){\rule[-0.200pt]{2.409pt}{0.400pt}}
\put(1429.0,190.0){\rule[-0.200pt]{2.409pt}{0.400pt}}
\put(170.0,190.0){\rule[-0.200pt]{2.409pt}{0.400pt}}
\put(1429.0,190.0){\rule[-0.200pt]{2.409pt}{0.400pt}}
\put(170.0,190.0){\rule[-0.200pt]{2.409pt}{0.400pt}}
\put(1429.0,190.0){\rule[-0.200pt]{2.409pt}{0.400pt}}
\put(170.0,190.0){\rule[-0.200pt]{2.409pt}{0.400pt}}
\put(1429.0,190.0){\rule[-0.200pt]{2.409pt}{0.400pt}}
\put(170.0,191.0){\rule[-0.200pt]{2.409pt}{0.400pt}}
\put(1429.0,191.0){\rule[-0.200pt]{2.409pt}{0.400pt}}
\put(170.0,191.0){\rule[-0.200pt]{2.409pt}{0.400pt}}
\put(1429.0,191.0){\rule[-0.200pt]{2.409pt}{0.400pt}}
\put(170.0,191.0){\rule[-0.200pt]{2.409pt}{0.400pt}}
\put(1429.0,191.0){\rule[-0.200pt]{2.409pt}{0.400pt}}
\put(170.0,191.0){\rule[-0.200pt]{2.409pt}{0.400pt}}
\put(1429.0,191.0){\rule[-0.200pt]{2.409pt}{0.400pt}}
\put(170.0,191.0){\rule[-0.200pt]{2.409pt}{0.400pt}}
\put(1429.0,191.0){\rule[-0.200pt]{2.409pt}{0.400pt}}
\put(170.0,191.0){\rule[-0.200pt]{2.409pt}{0.400pt}}
\put(1429.0,191.0){\rule[-0.200pt]{2.409pt}{0.400pt}}
\put(170.0,191.0){\rule[-0.200pt]{2.409pt}{0.400pt}}
\put(1429.0,191.0){\rule[-0.200pt]{2.409pt}{0.400pt}}
\put(170.0,191.0){\rule[-0.200pt]{2.409pt}{0.400pt}}
\put(1429.0,191.0){\rule[-0.200pt]{2.409pt}{0.400pt}}
\put(170.0,191.0){\rule[-0.200pt]{2.409pt}{0.400pt}}
\put(1429.0,191.0){\rule[-0.200pt]{2.409pt}{0.400pt}}
\put(170.0,191.0){\rule[-0.200pt]{2.409pt}{0.400pt}}
\put(1429.0,191.0){\rule[-0.200pt]{2.409pt}{0.400pt}}
\put(170.0,191.0){\rule[-0.200pt]{2.409pt}{0.400pt}}
\put(1429.0,191.0){\rule[-0.200pt]{2.409pt}{0.400pt}}
\put(170.0,191.0){\rule[-0.200pt]{2.409pt}{0.400pt}}
\put(1429.0,191.0){\rule[-0.200pt]{2.409pt}{0.400pt}}
\put(170.0,191.0){\rule[-0.200pt]{2.409pt}{0.400pt}}
\put(1429.0,191.0){\rule[-0.200pt]{2.409pt}{0.400pt}}
\put(170.0,191.0){\rule[-0.200pt]{2.409pt}{0.400pt}}
\put(1429.0,191.0){\rule[-0.200pt]{2.409pt}{0.400pt}}
\put(170.0,191.0){\rule[-0.200pt]{2.409pt}{0.400pt}}
\put(1429.0,191.0){\rule[-0.200pt]{2.409pt}{0.400pt}}
\put(170.0,191.0){\rule[-0.200pt]{2.409pt}{0.400pt}}
\put(1429.0,191.0){\rule[-0.200pt]{2.409pt}{0.400pt}}
\put(170.0,191.0){\rule[-0.200pt]{2.409pt}{0.400pt}}
\put(1429.0,191.0){\rule[-0.200pt]{2.409pt}{0.400pt}}
\put(170.0,191.0){\rule[-0.200pt]{2.409pt}{0.400pt}}
\put(1429.0,191.0){\rule[-0.200pt]{2.409pt}{0.400pt}}
\put(170.0,192.0){\rule[-0.200pt]{2.409pt}{0.400pt}}
\put(1429.0,192.0){\rule[-0.200pt]{2.409pt}{0.400pt}}
\put(170.0,192.0){\rule[-0.200pt]{2.409pt}{0.400pt}}
\put(1429.0,192.0){\rule[-0.200pt]{2.409pt}{0.400pt}}
\put(170.0,192.0){\rule[-0.200pt]{2.409pt}{0.400pt}}
\put(1429.0,192.0){\rule[-0.200pt]{2.409pt}{0.400pt}}
\put(170.0,192.0){\rule[-0.200pt]{2.409pt}{0.400pt}}
\put(1429.0,192.0){\rule[-0.200pt]{2.409pt}{0.400pt}}
\put(170.0,192.0){\rule[-0.200pt]{2.409pt}{0.400pt}}
\put(1429.0,192.0){\rule[-0.200pt]{2.409pt}{0.400pt}}
\put(170.0,192.0){\rule[-0.200pt]{2.409pt}{0.400pt}}
\put(1429.0,192.0){\rule[-0.200pt]{2.409pt}{0.400pt}}
\put(170.0,192.0){\rule[-0.200pt]{2.409pt}{0.400pt}}
\put(1429.0,192.0){\rule[-0.200pt]{2.409pt}{0.400pt}}
\put(170.0,192.0){\rule[-0.200pt]{2.409pt}{0.400pt}}
\put(1429.0,192.0){\rule[-0.200pt]{2.409pt}{0.400pt}}
\put(170.0,192.0){\rule[-0.200pt]{2.409pt}{0.400pt}}
\put(1429.0,192.0){\rule[-0.200pt]{2.409pt}{0.400pt}}
\put(170.0,192.0){\rule[-0.200pt]{2.409pt}{0.400pt}}
\put(1429.0,192.0){\rule[-0.200pt]{2.409pt}{0.400pt}}
\put(170.0,192.0){\rule[-0.200pt]{2.409pt}{0.400pt}}
\put(1429.0,192.0){\rule[-0.200pt]{2.409pt}{0.400pt}}
\put(170.0,192.0){\rule[-0.200pt]{2.409pt}{0.400pt}}
\put(1429.0,192.0){\rule[-0.200pt]{2.409pt}{0.400pt}}
\put(170.0,192.0){\rule[-0.200pt]{2.409pt}{0.400pt}}
\put(1429.0,192.0){\rule[-0.200pt]{2.409pt}{0.400pt}}
\put(170.0,192.0){\rule[-0.200pt]{2.409pt}{0.400pt}}
\put(1429.0,192.0){\rule[-0.200pt]{2.409pt}{0.400pt}}
\put(170.0,192.0){\rule[-0.200pt]{2.409pt}{0.400pt}}
\put(1429.0,192.0){\rule[-0.200pt]{2.409pt}{0.400pt}}
\put(170.0,192.0){\rule[-0.200pt]{2.409pt}{0.400pt}}
\put(1429.0,192.0){\rule[-0.200pt]{2.409pt}{0.400pt}}
\put(170.0,192.0){\rule[-0.200pt]{2.409pt}{0.400pt}}
\put(1429.0,192.0){\rule[-0.200pt]{2.409pt}{0.400pt}}
\put(170.0,192.0){\rule[-0.200pt]{2.409pt}{0.400pt}}
\put(1429.0,192.0){\rule[-0.200pt]{2.409pt}{0.400pt}}
\put(170.0,192.0){\rule[-0.200pt]{2.409pt}{0.400pt}}
\put(1429.0,192.0){\rule[-0.200pt]{2.409pt}{0.400pt}}
\put(170.0,192.0){\rule[-0.200pt]{2.409pt}{0.400pt}}
\put(1429.0,192.0){\rule[-0.200pt]{2.409pt}{0.400pt}}
\put(170.0,193.0){\rule[-0.200pt]{2.409pt}{0.400pt}}
\put(1429.0,193.0){\rule[-0.200pt]{2.409pt}{0.400pt}}
\put(170.0,193.0){\rule[-0.200pt]{2.409pt}{0.400pt}}
\put(1429.0,193.0){\rule[-0.200pt]{2.409pt}{0.400pt}}
\put(170.0,193.0){\rule[-0.200pt]{2.409pt}{0.400pt}}
\put(1429.0,193.0){\rule[-0.200pt]{2.409pt}{0.400pt}}
\put(170.0,193.0){\rule[-0.200pt]{2.409pt}{0.400pt}}
\put(1429.0,193.0){\rule[-0.200pt]{2.409pt}{0.400pt}}
\put(170.0,193.0){\rule[-0.200pt]{2.409pt}{0.400pt}}
\put(1429.0,193.0){\rule[-0.200pt]{2.409pt}{0.400pt}}
\put(170.0,193.0){\rule[-0.200pt]{2.409pt}{0.400pt}}
\put(1429.0,193.0){\rule[-0.200pt]{2.409pt}{0.400pt}}
\put(170.0,193.0){\rule[-0.200pt]{2.409pt}{0.400pt}}
\put(1429.0,193.0){\rule[-0.200pt]{2.409pt}{0.400pt}}
\put(170.0,193.0){\rule[-0.200pt]{2.409pt}{0.400pt}}
\put(1429.0,193.0){\rule[-0.200pt]{2.409pt}{0.400pt}}
\put(170.0,193.0){\rule[-0.200pt]{2.409pt}{0.400pt}}
\put(1429.0,193.0){\rule[-0.200pt]{2.409pt}{0.400pt}}
\put(170.0,193.0){\rule[-0.200pt]{2.409pt}{0.400pt}}
\put(1429.0,193.0){\rule[-0.200pt]{2.409pt}{0.400pt}}
\put(170.0,193.0){\rule[-0.200pt]{4.818pt}{0.400pt}}
\put(150,193){\makebox(0,0)[r]{ 1e-15}}
\put(1419.0,193.0){\rule[-0.200pt]{4.818pt}{0.400pt}}
\put(170.0,204.0){\rule[-0.200pt]{2.409pt}{0.400pt}}
\put(1429.0,204.0){\rule[-0.200pt]{2.409pt}{0.400pt}}
\put(170.0,219.0){\rule[-0.200pt]{2.409pt}{0.400pt}}
\put(1429.0,219.0){\rule[-0.200pt]{2.409pt}{0.400pt}}
\put(170.0,226.0){\rule[-0.200pt]{2.409pt}{0.400pt}}
\put(1429.0,226.0){\rule[-0.200pt]{2.409pt}{0.400pt}}
\put(170.0,232.0){\rule[-0.200pt]{2.409pt}{0.400pt}}
\put(1429.0,232.0){\rule[-0.200pt]{2.409pt}{0.400pt}}
\put(170.0,235.0){\rule[-0.200pt]{2.409pt}{0.400pt}}
\put(1429.0,235.0){\rule[-0.200pt]{2.409pt}{0.400pt}}
\put(170.0,239.0){\rule[-0.200pt]{2.409pt}{0.400pt}}
\put(1429.0,239.0){\rule[-0.200pt]{2.409pt}{0.400pt}}
\put(170.0,241.0){\rule[-0.200pt]{2.409pt}{0.400pt}}
\put(1429.0,241.0){\rule[-0.200pt]{2.409pt}{0.400pt}}
\put(170.0,243.0){\rule[-0.200pt]{2.409pt}{0.400pt}}
\put(1429.0,243.0){\rule[-0.200pt]{2.409pt}{0.400pt}}
\put(170.0,245.0){\rule[-0.200pt]{2.409pt}{0.400pt}}
\put(1429.0,245.0){\rule[-0.200pt]{2.409pt}{0.400pt}}
\put(170.0,247.0){\rule[-0.200pt]{2.409pt}{0.400pt}}
\put(1429.0,247.0){\rule[-0.200pt]{2.409pt}{0.400pt}}
\put(170.0,249.0){\rule[-0.200pt]{2.409pt}{0.400pt}}
\put(1429.0,249.0){\rule[-0.200pt]{2.409pt}{0.400pt}}
\put(170.0,250.0){\rule[-0.200pt]{2.409pt}{0.400pt}}
\put(1429.0,250.0){\rule[-0.200pt]{2.409pt}{0.400pt}}
\put(170.0,251.0){\rule[-0.200pt]{2.409pt}{0.400pt}}
\put(1429.0,251.0){\rule[-0.200pt]{2.409pt}{0.400pt}}
\put(170.0,253.0){\rule[-0.200pt]{2.409pt}{0.400pt}}
\put(1429.0,253.0){\rule[-0.200pt]{2.409pt}{0.400pt}}
\put(170.0,254.0){\rule[-0.200pt]{2.409pt}{0.400pt}}
\put(1429.0,254.0){\rule[-0.200pt]{2.409pt}{0.400pt}}
\put(170.0,255.0){\rule[-0.200pt]{2.409pt}{0.400pt}}
\put(1429.0,255.0){\rule[-0.200pt]{2.409pt}{0.400pt}}
\put(170.0,256.0){\rule[-0.200pt]{2.409pt}{0.400pt}}
\put(1429.0,256.0){\rule[-0.200pt]{2.409pt}{0.400pt}}
\put(170.0,257.0){\rule[-0.200pt]{2.409pt}{0.400pt}}
\put(1429.0,257.0){\rule[-0.200pt]{2.409pt}{0.400pt}}
\put(170.0,258.0){\rule[-0.200pt]{2.409pt}{0.400pt}}
\put(1429.0,258.0){\rule[-0.200pt]{2.409pt}{0.400pt}}
\put(170.0,259.0){\rule[-0.200pt]{2.409pt}{0.400pt}}
\put(1429.0,259.0){\rule[-0.200pt]{2.409pt}{0.400pt}}
\put(170.0,259.0){\rule[-0.200pt]{2.409pt}{0.400pt}}
\put(1429.0,259.0){\rule[-0.200pt]{2.409pt}{0.400pt}}
\put(170.0,260.0){\rule[-0.200pt]{2.409pt}{0.400pt}}
\put(1429.0,260.0){\rule[-0.200pt]{2.409pt}{0.400pt}}
\put(170.0,261.0){\rule[-0.200pt]{2.409pt}{0.400pt}}
\put(1429.0,261.0){\rule[-0.200pt]{2.409pt}{0.400pt}}
\put(170.0,261.0){\rule[-0.200pt]{2.409pt}{0.400pt}}
\put(1429.0,261.0){\rule[-0.200pt]{2.409pt}{0.400pt}}
\put(170.0,262.0){\rule[-0.200pt]{2.409pt}{0.400pt}}
\put(1429.0,262.0){\rule[-0.200pt]{2.409pt}{0.400pt}}
\put(170.0,263.0){\rule[-0.200pt]{2.409pt}{0.400pt}}
\put(1429.0,263.0){\rule[-0.200pt]{2.409pt}{0.400pt}}
\put(170.0,263.0){\rule[-0.200pt]{2.409pt}{0.400pt}}
\put(1429.0,263.0){\rule[-0.200pt]{2.409pt}{0.400pt}}
\put(170.0,264.0){\rule[-0.200pt]{2.409pt}{0.400pt}}
\put(1429.0,264.0){\rule[-0.200pt]{2.409pt}{0.400pt}}
\put(170.0,265.0){\rule[-0.200pt]{2.409pt}{0.400pt}}
\put(1429.0,265.0){\rule[-0.200pt]{2.409pt}{0.400pt}}
\put(170.0,265.0){\rule[-0.200pt]{2.409pt}{0.400pt}}
\put(1429.0,265.0){\rule[-0.200pt]{2.409pt}{0.400pt}}
\put(170.0,266.0){\rule[-0.200pt]{2.409pt}{0.400pt}}
\put(1429.0,266.0){\rule[-0.200pt]{2.409pt}{0.400pt}}
\put(170.0,266.0){\rule[-0.200pt]{2.409pt}{0.400pt}}
\put(1429.0,266.0){\rule[-0.200pt]{2.409pt}{0.400pt}}
\put(170.0,267.0){\rule[-0.200pt]{2.409pt}{0.400pt}}
\put(1429.0,267.0){\rule[-0.200pt]{2.409pt}{0.400pt}}
\put(170.0,267.0){\rule[-0.200pt]{2.409pt}{0.400pt}}
\put(1429.0,267.0){\rule[-0.200pt]{2.409pt}{0.400pt}}
\put(170.0,268.0){\rule[-0.200pt]{2.409pt}{0.400pt}}
\put(1429.0,268.0){\rule[-0.200pt]{2.409pt}{0.400pt}}
\put(170.0,268.0){\rule[-0.200pt]{2.409pt}{0.400pt}}
\put(1429.0,268.0){\rule[-0.200pt]{2.409pt}{0.400pt}}
\put(170.0,269.0){\rule[-0.200pt]{2.409pt}{0.400pt}}
\put(1429.0,269.0){\rule[-0.200pt]{2.409pt}{0.400pt}}
\put(170.0,269.0){\rule[-0.200pt]{2.409pt}{0.400pt}}
\put(1429.0,269.0){\rule[-0.200pt]{2.409pt}{0.400pt}}
\put(170.0,269.0){\rule[-0.200pt]{2.409pt}{0.400pt}}
\put(1429.0,269.0){\rule[-0.200pt]{2.409pt}{0.400pt}}
\put(170.0,270.0){\rule[-0.200pt]{2.409pt}{0.400pt}}
\put(1429.0,270.0){\rule[-0.200pt]{2.409pt}{0.400pt}}
\put(170.0,270.0){\rule[-0.200pt]{2.409pt}{0.400pt}}
\put(1429.0,270.0){\rule[-0.200pt]{2.409pt}{0.400pt}}
\put(170.0,271.0){\rule[-0.200pt]{2.409pt}{0.400pt}}
\put(1429.0,271.0){\rule[-0.200pt]{2.409pt}{0.400pt}}
\put(170.0,271.0){\rule[-0.200pt]{2.409pt}{0.400pt}}
\put(1429.0,271.0){\rule[-0.200pt]{2.409pt}{0.400pt}}
\put(170.0,271.0){\rule[-0.200pt]{2.409pt}{0.400pt}}
\put(1429.0,271.0){\rule[-0.200pt]{2.409pt}{0.400pt}}
\put(170.0,272.0){\rule[-0.200pt]{2.409pt}{0.400pt}}
\put(1429.0,272.0){\rule[-0.200pt]{2.409pt}{0.400pt}}
\put(170.0,272.0){\rule[-0.200pt]{2.409pt}{0.400pt}}
\put(1429.0,272.0){\rule[-0.200pt]{2.409pt}{0.400pt}}
\put(170.0,272.0){\rule[-0.200pt]{2.409pt}{0.400pt}}
\put(1429.0,272.0){\rule[-0.200pt]{2.409pt}{0.400pt}}
\put(170.0,273.0){\rule[-0.200pt]{2.409pt}{0.400pt}}
\put(1429.0,273.0){\rule[-0.200pt]{2.409pt}{0.400pt}}
\put(170.0,273.0){\rule[-0.200pt]{2.409pt}{0.400pt}}
\put(1429.0,273.0){\rule[-0.200pt]{2.409pt}{0.400pt}}
\put(170.0,273.0){\rule[-0.200pt]{2.409pt}{0.400pt}}
\put(1429.0,273.0){\rule[-0.200pt]{2.409pt}{0.400pt}}
\put(170.0,274.0){\rule[-0.200pt]{2.409pt}{0.400pt}}
\put(1429.0,274.0){\rule[-0.200pt]{2.409pt}{0.400pt}}
\put(170.0,274.0){\rule[-0.200pt]{2.409pt}{0.400pt}}
\put(1429.0,274.0){\rule[-0.200pt]{2.409pt}{0.400pt}}
\put(170.0,274.0){\rule[-0.200pt]{2.409pt}{0.400pt}}
\put(1429.0,274.0){\rule[-0.200pt]{2.409pt}{0.400pt}}
\put(170.0,275.0){\rule[-0.200pt]{2.409pt}{0.400pt}}
\put(1429.0,275.0){\rule[-0.200pt]{2.409pt}{0.400pt}}
\put(170.0,275.0){\rule[-0.200pt]{2.409pt}{0.400pt}}
\put(1429.0,275.0){\rule[-0.200pt]{2.409pt}{0.400pt}}
\put(170.0,275.0){\rule[-0.200pt]{2.409pt}{0.400pt}}
\put(1429.0,275.0){\rule[-0.200pt]{2.409pt}{0.400pt}}
\put(170.0,276.0){\rule[-0.200pt]{2.409pt}{0.400pt}}
\put(1429.0,276.0){\rule[-0.200pt]{2.409pt}{0.400pt}}
\put(170.0,276.0){\rule[-0.200pt]{2.409pt}{0.400pt}}
\put(1429.0,276.0){\rule[-0.200pt]{2.409pt}{0.400pt}}
\put(170.0,276.0){\rule[-0.200pt]{2.409pt}{0.400pt}}
\put(1429.0,276.0){\rule[-0.200pt]{2.409pt}{0.400pt}}
\put(170.0,276.0){\rule[-0.200pt]{2.409pt}{0.400pt}}
\put(1429.0,276.0){\rule[-0.200pt]{2.409pt}{0.400pt}}
\put(170.0,277.0){\rule[-0.200pt]{2.409pt}{0.400pt}}
\put(1429.0,277.0){\rule[-0.200pt]{2.409pt}{0.400pt}}
\put(170.0,277.0){\rule[-0.200pt]{2.409pt}{0.400pt}}
\put(1429.0,277.0){\rule[-0.200pt]{2.409pt}{0.400pt}}
\put(170.0,277.0){\rule[-0.200pt]{2.409pt}{0.400pt}}
\put(1429.0,277.0){\rule[-0.200pt]{2.409pt}{0.400pt}}
\put(170.0,277.0){\rule[-0.200pt]{2.409pt}{0.400pt}}
\put(1429.0,277.0){\rule[-0.200pt]{2.409pt}{0.400pt}}
\put(170.0,278.0){\rule[-0.200pt]{2.409pt}{0.400pt}}
\put(1429.0,278.0){\rule[-0.200pt]{2.409pt}{0.400pt}}
\put(170.0,278.0){\rule[-0.200pt]{2.409pt}{0.400pt}}
\put(1429.0,278.0){\rule[-0.200pt]{2.409pt}{0.400pt}}
\put(170.0,278.0){\rule[-0.200pt]{2.409pt}{0.400pt}}
\put(1429.0,278.0){\rule[-0.200pt]{2.409pt}{0.400pt}}
\put(170.0,278.0){\rule[-0.200pt]{2.409pt}{0.400pt}}
\put(1429.0,278.0){\rule[-0.200pt]{2.409pt}{0.400pt}}
\put(170.0,279.0){\rule[-0.200pt]{2.409pt}{0.400pt}}
\put(1429.0,279.0){\rule[-0.200pt]{2.409pt}{0.400pt}}
\put(170.0,279.0){\rule[-0.200pt]{2.409pt}{0.400pt}}
\put(1429.0,279.0){\rule[-0.200pt]{2.409pt}{0.400pt}}
\put(170.0,279.0){\rule[-0.200pt]{2.409pt}{0.400pt}}
\put(1429.0,279.0){\rule[-0.200pt]{2.409pt}{0.400pt}}
\put(170.0,279.0){\rule[-0.200pt]{2.409pt}{0.400pt}}
\put(1429.0,279.0){\rule[-0.200pt]{2.409pt}{0.400pt}}
\put(170.0,280.0){\rule[-0.200pt]{2.409pt}{0.400pt}}
\put(1429.0,280.0){\rule[-0.200pt]{2.409pt}{0.400pt}}
\put(170.0,280.0){\rule[-0.200pt]{2.409pt}{0.400pt}}
\put(1429.0,280.0){\rule[-0.200pt]{2.409pt}{0.400pt}}
\put(170.0,280.0){\rule[-0.200pt]{2.409pt}{0.400pt}}
\put(1429.0,280.0){\rule[-0.200pt]{2.409pt}{0.400pt}}
\put(170.0,280.0){\rule[-0.200pt]{2.409pt}{0.400pt}}
\put(1429.0,280.0){\rule[-0.200pt]{2.409pt}{0.400pt}}
\put(170.0,280.0){\rule[-0.200pt]{2.409pt}{0.400pt}}
\put(1429.0,280.0){\rule[-0.200pt]{2.409pt}{0.400pt}}
\put(170.0,281.0){\rule[-0.200pt]{2.409pt}{0.400pt}}
\put(1429.0,281.0){\rule[-0.200pt]{2.409pt}{0.400pt}}
\put(170.0,281.0){\rule[-0.200pt]{2.409pt}{0.400pt}}
\put(1429.0,281.0){\rule[-0.200pt]{2.409pt}{0.400pt}}
\put(170.0,281.0){\rule[-0.200pt]{2.409pt}{0.400pt}}
\put(1429.0,281.0){\rule[-0.200pt]{2.409pt}{0.400pt}}
\put(170.0,281.0){\rule[-0.200pt]{2.409pt}{0.400pt}}
\put(1429.0,281.0){\rule[-0.200pt]{2.409pt}{0.400pt}}
\put(170.0,281.0){\rule[-0.200pt]{2.409pt}{0.400pt}}
\put(1429.0,281.0){\rule[-0.200pt]{2.409pt}{0.400pt}}
\put(170.0,282.0){\rule[-0.200pt]{2.409pt}{0.400pt}}
\put(1429.0,282.0){\rule[-0.200pt]{2.409pt}{0.400pt}}
\put(170.0,282.0){\rule[-0.200pt]{2.409pt}{0.400pt}}
\put(1429.0,282.0){\rule[-0.200pt]{2.409pt}{0.400pt}}
\put(170.0,282.0){\rule[-0.200pt]{2.409pt}{0.400pt}}
\put(1429.0,282.0){\rule[-0.200pt]{2.409pt}{0.400pt}}
\put(170.0,282.0){\rule[-0.200pt]{2.409pt}{0.400pt}}
\put(1429.0,282.0){\rule[-0.200pt]{2.409pt}{0.400pt}}
\put(170.0,282.0){\rule[-0.200pt]{2.409pt}{0.400pt}}
\put(1429.0,282.0){\rule[-0.200pt]{2.409pt}{0.400pt}}
\put(170.0,283.0){\rule[-0.200pt]{2.409pt}{0.400pt}}
\put(1429.0,283.0){\rule[-0.200pt]{2.409pt}{0.400pt}}
\put(170.0,283.0){\rule[-0.200pt]{2.409pt}{0.400pt}}
\put(1429.0,283.0){\rule[-0.200pt]{2.409pt}{0.400pt}}
\put(170.0,283.0){\rule[-0.200pt]{2.409pt}{0.400pt}}
\put(1429.0,283.0){\rule[-0.200pt]{2.409pt}{0.400pt}}
\put(170.0,283.0){\rule[-0.200pt]{2.409pt}{0.400pt}}
\put(1429.0,283.0){\rule[-0.200pt]{2.409pt}{0.400pt}}
\put(170.0,283.0){\rule[-0.200pt]{2.409pt}{0.400pt}}
\put(1429.0,283.0){\rule[-0.200pt]{2.409pt}{0.400pt}}
\put(170.0,283.0){\rule[-0.200pt]{2.409pt}{0.400pt}}
\put(1429.0,283.0){\rule[-0.200pt]{2.409pt}{0.400pt}}
\put(170.0,284.0){\rule[-0.200pt]{2.409pt}{0.400pt}}
\put(1429.0,284.0){\rule[-0.200pt]{2.409pt}{0.400pt}}
\put(170.0,284.0){\rule[-0.200pt]{2.409pt}{0.400pt}}
\put(1429.0,284.0){\rule[-0.200pt]{2.409pt}{0.400pt}}
\put(170.0,284.0){\rule[-0.200pt]{2.409pt}{0.400pt}}
\put(1429.0,284.0){\rule[-0.200pt]{2.409pt}{0.400pt}}
\put(170.0,284.0){\rule[-0.200pt]{2.409pt}{0.400pt}}
\put(1429.0,284.0){\rule[-0.200pt]{2.409pt}{0.400pt}}
\put(170.0,284.0){\rule[-0.200pt]{2.409pt}{0.400pt}}
\put(1429.0,284.0){\rule[-0.200pt]{2.409pt}{0.400pt}}
\put(170.0,284.0){\rule[-0.200pt]{2.409pt}{0.400pt}}
\put(1429.0,284.0){\rule[-0.200pt]{2.409pt}{0.400pt}}
\put(170.0,285.0){\rule[-0.200pt]{2.409pt}{0.400pt}}
\put(1429.0,285.0){\rule[-0.200pt]{2.409pt}{0.400pt}}
\put(170.0,285.0){\rule[-0.200pt]{2.409pt}{0.400pt}}
\put(1429.0,285.0){\rule[-0.200pt]{2.409pt}{0.400pt}}
\put(170.0,285.0){\rule[-0.200pt]{2.409pt}{0.400pt}}
\put(1429.0,285.0){\rule[-0.200pt]{2.409pt}{0.400pt}}
\put(170.0,285.0){\rule[-0.200pt]{2.409pt}{0.400pt}}
\put(1429.0,285.0){\rule[-0.200pt]{2.409pt}{0.400pt}}
\put(170.0,285.0){\rule[-0.200pt]{2.409pt}{0.400pt}}
\put(1429.0,285.0){\rule[-0.200pt]{2.409pt}{0.400pt}}
\put(170.0,285.0){\rule[-0.200pt]{2.409pt}{0.400pt}}
\put(1429.0,285.0){\rule[-0.200pt]{2.409pt}{0.400pt}}
\put(170.0,286.0){\rule[-0.200pt]{2.409pt}{0.400pt}}
\put(1429.0,286.0){\rule[-0.200pt]{2.409pt}{0.400pt}}
\put(170.0,286.0){\rule[-0.200pt]{2.409pt}{0.400pt}}
\put(1429.0,286.0){\rule[-0.200pt]{2.409pt}{0.400pt}}
\put(170.0,286.0){\rule[-0.200pt]{2.409pt}{0.400pt}}
\put(1429.0,286.0){\rule[-0.200pt]{2.409pt}{0.400pt}}
\put(170.0,286.0){\rule[-0.200pt]{2.409pt}{0.400pt}}
\put(1429.0,286.0){\rule[-0.200pt]{2.409pt}{0.400pt}}
\put(170.0,286.0){\rule[-0.200pt]{2.409pt}{0.400pt}}
\put(1429.0,286.0){\rule[-0.200pt]{2.409pt}{0.400pt}}
\put(170.0,286.0){\rule[-0.200pt]{2.409pt}{0.400pt}}
\put(1429.0,286.0){\rule[-0.200pt]{2.409pt}{0.400pt}}
\put(170.0,286.0){\rule[-0.200pt]{2.409pt}{0.400pt}}
\put(1429.0,286.0){\rule[-0.200pt]{2.409pt}{0.400pt}}
\put(170.0,287.0){\rule[-0.200pt]{2.409pt}{0.400pt}}
\put(1429.0,287.0){\rule[-0.200pt]{2.409pt}{0.400pt}}
\put(170.0,287.0){\rule[-0.200pt]{2.409pt}{0.400pt}}
\put(1429.0,287.0){\rule[-0.200pt]{2.409pt}{0.400pt}}
\put(170.0,287.0){\rule[-0.200pt]{2.409pt}{0.400pt}}
\put(1429.0,287.0){\rule[-0.200pt]{2.409pt}{0.400pt}}
\put(170.0,287.0){\rule[-0.200pt]{2.409pt}{0.400pt}}
\put(1429.0,287.0){\rule[-0.200pt]{2.409pt}{0.400pt}}
\put(170.0,287.0){\rule[-0.200pt]{2.409pt}{0.400pt}}
\put(1429.0,287.0){\rule[-0.200pt]{2.409pt}{0.400pt}}
\put(170.0,287.0){\rule[-0.200pt]{2.409pt}{0.400pt}}
\put(1429.0,287.0){\rule[-0.200pt]{2.409pt}{0.400pt}}
\put(170.0,287.0){\rule[-0.200pt]{2.409pt}{0.400pt}}
\put(1429.0,287.0){\rule[-0.200pt]{2.409pt}{0.400pt}}
\put(170.0,288.0){\rule[-0.200pt]{2.409pt}{0.400pt}}
\put(1429.0,288.0){\rule[-0.200pt]{2.409pt}{0.400pt}}
\put(170.0,288.0){\rule[-0.200pt]{2.409pt}{0.400pt}}
\put(1429.0,288.0){\rule[-0.200pt]{2.409pt}{0.400pt}}
\put(170.0,288.0){\rule[-0.200pt]{2.409pt}{0.400pt}}
\put(1429.0,288.0){\rule[-0.200pt]{2.409pt}{0.400pt}}
\put(170.0,288.0){\rule[-0.200pt]{2.409pt}{0.400pt}}
\put(1429.0,288.0){\rule[-0.200pt]{2.409pt}{0.400pt}}
\put(170.0,288.0){\rule[-0.200pt]{2.409pt}{0.400pt}}
\put(1429.0,288.0){\rule[-0.200pt]{2.409pt}{0.400pt}}
\put(170.0,288.0){\rule[-0.200pt]{2.409pt}{0.400pt}}
\put(1429.0,288.0){\rule[-0.200pt]{2.409pt}{0.400pt}}
\put(170.0,288.0){\rule[-0.200pt]{2.409pt}{0.400pt}}
\put(1429.0,288.0){\rule[-0.200pt]{2.409pt}{0.400pt}}
\put(170.0,288.0){\rule[-0.200pt]{2.409pt}{0.400pt}}
\put(1429.0,288.0){\rule[-0.200pt]{2.409pt}{0.400pt}}
\put(170.0,289.0){\rule[-0.200pt]{2.409pt}{0.400pt}}
\put(1429.0,289.0){\rule[-0.200pt]{2.409pt}{0.400pt}}
\put(170.0,289.0){\rule[-0.200pt]{2.409pt}{0.400pt}}
\put(1429.0,289.0){\rule[-0.200pt]{2.409pt}{0.400pt}}
\put(170.0,289.0){\rule[-0.200pt]{2.409pt}{0.400pt}}
\put(1429.0,289.0){\rule[-0.200pt]{2.409pt}{0.400pt}}
\put(170.0,289.0){\rule[-0.200pt]{2.409pt}{0.400pt}}
\put(1429.0,289.0){\rule[-0.200pt]{2.409pt}{0.400pt}}
\put(170.0,289.0){\rule[-0.200pt]{2.409pt}{0.400pt}}
\put(1429.0,289.0){\rule[-0.200pt]{2.409pt}{0.400pt}}
\put(170.0,289.0){\rule[-0.200pt]{2.409pt}{0.400pt}}
\put(1429.0,289.0){\rule[-0.200pt]{2.409pt}{0.400pt}}
\put(170.0,289.0){\rule[-0.200pt]{2.409pt}{0.400pt}}
\put(1429.0,289.0){\rule[-0.200pt]{2.409pt}{0.400pt}}
\put(170.0,289.0){\rule[-0.200pt]{2.409pt}{0.400pt}}
\put(1429.0,289.0){\rule[-0.200pt]{2.409pt}{0.400pt}}
\put(170.0,290.0){\rule[-0.200pt]{2.409pt}{0.400pt}}
\put(1429.0,290.0){\rule[-0.200pt]{2.409pt}{0.400pt}}
\put(170.0,290.0){\rule[-0.200pt]{2.409pt}{0.400pt}}
\put(1429.0,290.0){\rule[-0.200pt]{2.409pt}{0.400pt}}
\put(170.0,290.0){\rule[-0.200pt]{2.409pt}{0.400pt}}
\put(1429.0,290.0){\rule[-0.200pt]{2.409pt}{0.400pt}}
\put(170.0,290.0){\rule[-0.200pt]{2.409pt}{0.400pt}}
\put(1429.0,290.0){\rule[-0.200pt]{2.409pt}{0.400pt}}
\put(170.0,290.0){\rule[-0.200pt]{2.409pt}{0.400pt}}
\put(1429.0,290.0){\rule[-0.200pt]{2.409pt}{0.400pt}}
\put(170.0,290.0){\rule[-0.200pt]{2.409pt}{0.400pt}}
\put(1429.0,290.0){\rule[-0.200pt]{2.409pt}{0.400pt}}
\put(170.0,290.0){\rule[-0.200pt]{2.409pt}{0.400pt}}
\put(1429.0,290.0){\rule[-0.200pt]{2.409pt}{0.400pt}}
\put(170.0,290.0){\rule[-0.200pt]{2.409pt}{0.400pt}}
\put(1429.0,290.0){\rule[-0.200pt]{2.409pt}{0.400pt}}
\put(170.0,290.0){\rule[-0.200pt]{2.409pt}{0.400pt}}
\put(1429.0,290.0){\rule[-0.200pt]{2.409pt}{0.400pt}}
\put(170.0,291.0){\rule[-0.200pt]{2.409pt}{0.400pt}}
\put(1429.0,291.0){\rule[-0.200pt]{2.409pt}{0.400pt}}
\put(170.0,291.0){\rule[-0.200pt]{2.409pt}{0.400pt}}
\put(1429.0,291.0){\rule[-0.200pt]{2.409pt}{0.400pt}}
\put(170.0,291.0){\rule[-0.200pt]{2.409pt}{0.400pt}}
\put(1429.0,291.0){\rule[-0.200pt]{2.409pt}{0.400pt}}
\put(170.0,291.0){\rule[-0.200pt]{2.409pt}{0.400pt}}
\put(1429.0,291.0){\rule[-0.200pt]{2.409pt}{0.400pt}}
\put(170.0,291.0){\rule[-0.200pt]{2.409pt}{0.400pt}}
\put(1429.0,291.0){\rule[-0.200pt]{2.409pt}{0.400pt}}
\put(170.0,291.0){\rule[-0.200pt]{2.409pt}{0.400pt}}
\put(1429.0,291.0){\rule[-0.200pt]{2.409pt}{0.400pt}}
\put(170.0,291.0){\rule[-0.200pt]{2.409pt}{0.400pt}}
\put(1429.0,291.0){\rule[-0.200pt]{2.409pt}{0.400pt}}
\put(170.0,291.0){\rule[-0.200pt]{2.409pt}{0.400pt}}
\put(1429.0,291.0){\rule[-0.200pt]{2.409pt}{0.400pt}}
\put(170.0,291.0){\rule[-0.200pt]{2.409pt}{0.400pt}}
\put(1429.0,291.0){\rule[-0.200pt]{2.409pt}{0.400pt}}
\put(170.0,292.0){\rule[-0.200pt]{2.409pt}{0.400pt}}
\put(1429.0,292.0){\rule[-0.200pt]{2.409pt}{0.400pt}}
\put(170.0,292.0){\rule[-0.200pt]{2.409pt}{0.400pt}}
\put(1429.0,292.0){\rule[-0.200pt]{2.409pt}{0.400pt}}
\put(170.0,292.0){\rule[-0.200pt]{2.409pt}{0.400pt}}
\put(1429.0,292.0){\rule[-0.200pt]{2.409pt}{0.400pt}}
\put(170.0,292.0){\rule[-0.200pt]{2.409pt}{0.400pt}}
\put(1429.0,292.0){\rule[-0.200pt]{2.409pt}{0.400pt}}
\put(170.0,292.0){\rule[-0.200pt]{2.409pt}{0.400pt}}
\put(1429.0,292.0){\rule[-0.200pt]{2.409pt}{0.400pt}}
\put(170.0,292.0){\rule[-0.200pt]{2.409pt}{0.400pt}}
\put(1429.0,292.0){\rule[-0.200pt]{2.409pt}{0.400pt}}
\put(170.0,292.0){\rule[-0.200pt]{2.409pt}{0.400pt}}
\put(1429.0,292.0){\rule[-0.200pt]{2.409pt}{0.400pt}}
\put(170.0,292.0){\rule[-0.200pt]{2.409pt}{0.400pt}}
\put(1429.0,292.0){\rule[-0.200pt]{2.409pt}{0.400pt}}
\put(170.0,292.0){\rule[-0.200pt]{2.409pt}{0.400pt}}
\put(1429.0,292.0){\rule[-0.200pt]{2.409pt}{0.400pt}}
\put(170.0,292.0){\rule[-0.200pt]{2.409pt}{0.400pt}}
\put(1429.0,292.0){\rule[-0.200pt]{2.409pt}{0.400pt}}
\put(170.0,293.0){\rule[-0.200pt]{2.409pt}{0.400pt}}
\put(1429.0,293.0){\rule[-0.200pt]{2.409pt}{0.400pt}}
\put(170.0,293.0){\rule[-0.200pt]{2.409pt}{0.400pt}}
\put(1429.0,293.0){\rule[-0.200pt]{2.409pt}{0.400pt}}
\put(170.0,293.0){\rule[-0.200pt]{2.409pt}{0.400pt}}
\put(1429.0,293.0){\rule[-0.200pt]{2.409pt}{0.400pt}}
\put(170.0,293.0){\rule[-0.200pt]{2.409pt}{0.400pt}}
\put(1429.0,293.0){\rule[-0.200pt]{2.409pt}{0.400pt}}
\put(170.0,293.0){\rule[-0.200pt]{2.409pt}{0.400pt}}
\put(1429.0,293.0){\rule[-0.200pt]{2.409pt}{0.400pt}}
\put(170.0,293.0){\rule[-0.200pt]{2.409pt}{0.400pt}}
\put(1429.0,293.0){\rule[-0.200pt]{2.409pt}{0.400pt}}
\put(170.0,293.0){\rule[-0.200pt]{2.409pt}{0.400pt}}
\put(1429.0,293.0){\rule[-0.200pt]{2.409pt}{0.400pt}}
\put(170.0,293.0){\rule[-0.200pt]{2.409pt}{0.400pt}}
\put(1429.0,293.0){\rule[-0.200pt]{2.409pt}{0.400pt}}
\put(170.0,293.0){\rule[-0.200pt]{2.409pt}{0.400pt}}
\put(1429.0,293.0){\rule[-0.200pt]{2.409pt}{0.400pt}}
\put(170.0,293.0){\rule[-0.200pt]{2.409pt}{0.400pt}}
\put(1429.0,293.0){\rule[-0.200pt]{2.409pt}{0.400pt}}
\put(170.0,294.0){\rule[-0.200pt]{2.409pt}{0.400pt}}
\put(1429.0,294.0){\rule[-0.200pt]{2.409pt}{0.400pt}}
\put(170.0,294.0){\rule[-0.200pt]{2.409pt}{0.400pt}}
\put(1429.0,294.0){\rule[-0.200pt]{2.409pt}{0.400pt}}
\put(170.0,294.0){\rule[-0.200pt]{2.409pt}{0.400pt}}
\put(1429.0,294.0){\rule[-0.200pt]{2.409pt}{0.400pt}}
\put(170.0,294.0){\rule[-0.200pt]{2.409pt}{0.400pt}}
\put(1429.0,294.0){\rule[-0.200pt]{2.409pt}{0.400pt}}
\put(170.0,294.0){\rule[-0.200pt]{2.409pt}{0.400pt}}
\put(1429.0,294.0){\rule[-0.200pt]{2.409pt}{0.400pt}}
\put(170.0,294.0){\rule[-0.200pt]{2.409pt}{0.400pt}}
\put(1429.0,294.0){\rule[-0.200pt]{2.409pt}{0.400pt}}
\put(170.0,294.0){\rule[-0.200pt]{2.409pt}{0.400pt}}
\put(1429.0,294.0){\rule[-0.200pt]{2.409pt}{0.400pt}}
\put(170.0,294.0){\rule[-0.200pt]{2.409pt}{0.400pt}}
\put(1429.0,294.0){\rule[-0.200pt]{2.409pt}{0.400pt}}
\put(170.0,294.0){\rule[-0.200pt]{2.409pt}{0.400pt}}
\put(1429.0,294.0){\rule[-0.200pt]{2.409pt}{0.400pt}}
\put(170.0,294.0){\rule[-0.200pt]{2.409pt}{0.400pt}}
\put(1429.0,294.0){\rule[-0.200pt]{2.409pt}{0.400pt}}
\put(170.0,294.0){\rule[-0.200pt]{2.409pt}{0.400pt}}
\put(1429.0,294.0){\rule[-0.200pt]{2.409pt}{0.400pt}}
\put(170.0,295.0){\rule[-0.200pt]{2.409pt}{0.400pt}}
\put(1429.0,295.0){\rule[-0.200pt]{2.409pt}{0.400pt}}
\put(170.0,295.0){\rule[-0.200pt]{2.409pt}{0.400pt}}
\put(1429.0,295.0){\rule[-0.200pt]{2.409pt}{0.400pt}}
\put(170.0,295.0){\rule[-0.200pt]{2.409pt}{0.400pt}}
\put(1429.0,295.0){\rule[-0.200pt]{2.409pt}{0.400pt}}
\put(170.0,295.0){\rule[-0.200pt]{2.409pt}{0.400pt}}
\put(1429.0,295.0){\rule[-0.200pt]{2.409pt}{0.400pt}}
\put(170.0,295.0){\rule[-0.200pt]{2.409pt}{0.400pt}}
\put(1429.0,295.0){\rule[-0.200pt]{2.409pt}{0.400pt}}
\put(170.0,295.0){\rule[-0.200pt]{2.409pt}{0.400pt}}
\put(1429.0,295.0){\rule[-0.200pt]{2.409pt}{0.400pt}}
\put(170.0,295.0){\rule[-0.200pt]{2.409pt}{0.400pt}}
\put(1429.0,295.0){\rule[-0.200pt]{2.409pt}{0.400pt}}
\put(170.0,295.0){\rule[-0.200pt]{2.409pt}{0.400pt}}
\put(1429.0,295.0){\rule[-0.200pt]{2.409pt}{0.400pt}}
\put(170.0,295.0){\rule[-0.200pt]{2.409pt}{0.400pt}}
\put(1429.0,295.0){\rule[-0.200pt]{2.409pt}{0.400pt}}
\put(170.0,295.0){\rule[-0.200pt]{2.409pt}{0.400pt}}
\put(1429.0,295.0){\rule[-0.200pt]{2.409pt}{0.400pt}}
\put(170.0,295.0){\rule[-0.200pt]{2.409pt}{0.400pt}}
\put(1429.0,295.0){\rule[-0.200pt]{2.409pt}{0.400pt}}
\put(170.0,295.0){\rule[-0.200pt]{2.409pt}{0.400pt}}
\put(1429.0,295.0){\rule[-0.200pt]{2.409pt}{0.400pt}}
\put(170.0,296.0){\rule[-0.200pt]{2.409pt}{0.400pt}}
\put(1429.0,296.0){\rule[-0.200pt]{2.409pt}{0.400pt}}
\put(170.0,296.0){\rule[-0.200pt]{2.409pt}{0.400pt}}
\put(1429.0,296.0){\rule[-0.200pt]{2.409pt}{0.400pt}}
\put(170.0,296.0){\rule[-0.200pt]{2.409pt}{0.400pt}}
\put(1429.0,296.0){\rule[-0.200pt]{2.409pt}{0.400pt}}
\put(170.0,296.0){\rule[-0.200pt]{2.409pt}{0.400pt}}
\put(1429.0,296.0){\rule[-0.200pt]{2.409pt}{0.400pt}}
\put(170.0,296.0){\rule[-0.200pt]{2.409pt}{0.400pt}}
\put(1429.0,296.0){\rule[-0.200pt]{2.409pt}{0.400pt}}
\put(170.0,296.0){\rule[-0.200pt]{2.409pt}{0.400pt}}
\put(1429.0,296.0){\rule[-0.200pt]{2.409pt}{0.400pt}}
\put(170.0,296.0){\rule[-0.200pt]{2.409pt}{0.400pt}}
\put(1429.0,296.0){\rule[-0.200pt]{2.409pt}{0.400pt}}
\put(170.0,296.0){\rule[-0.200pt]{2.409pt}{0.400pt}}
\put(1429.0,296.0){\rule[-0.200pt]{2.409pt}{0.400pt}}
\put(170.0,296.0){\rule[-0.200pt]{2.409pt}{0.400pt}}
\put(1429.0,296.0){\rule[-0.200pt]{2.409pt}{0.400pt}}
\put(170.0,296.0){\rule[-0.200pt]{2.409pt}{0.400pt}}
\put(1429.0,296.0){\rule[-0.200pt]{2.409pt}{0.400pt}}
\put(170.0,296.0){\rule[-0.200pt]{2.409pt}{0.400pt}}
\put(1429.0,296.0){\rule[-0.200pt]{2.409pt}{0.400pt}}
\put(170.0,296.0){\rule[-0.200pt]{2.409pt}{0.400pt}}
\put(1429.0,296.0){\rule[-0.200pt]{2.409pt}{0.400pt}}
\put(170.0,296.0){\rule[-0.200pt]{2.409pt}{0.400pt}}
\put(1429.0,296.0){\rule[-0.200pt]{2.409pt}{0.400pt}}
\put(170.0,297.0){\rule[-0.200pt]{2.409pt}{0.400pt}}
\put(1429.0,297.0){\rule[-0.200pt]{2.409pt}{0.400pt}}
\put(170.0,297.0){\rule[-0.200pt]{2.409pt}{0.400pt}}
\put(1429.0,297.0){\rule[-0.200pt]{2.409pt}{0.400pt}}
\put(170.0,297.0){\rule[-0.200pt]{2.409pt}{0.400pt}}
\put(1429.0,297.0){\rule[-0.200pt]{2.409pt}{0.400pt}}
\put(170.0,297.0){\rule[-0.200pt]{2.409pt}{0.400pt}}
\put(1429.0,297.0){\rule[-0.200pt]{2.409pt}{0.400pt}}
\put(170.0,297.0){\rule[-0.200pt]{2.409pt}{0.400pt}}
\put(1429.0,297.0){\rule[-0.200pt]{2.409pt}{0.400pt}}
\put(170.0,297.0){\rule[-0.200pt]{2.409pt}{0.400pt}}
\put(1429.0,297.0){\rule[-0.200pt]{2.409pt}{0.400pt}}
\put(170.0,297.0){\rule[-0.200pt]{2.409pt}{0.400pt}}
\put(1429.0,297.0){\rule[-0.200pt]{2.409pt}{0.400pt}}
\put(170.0,297.0){\rule[-0.200pt]{2.409pt}{0.400pt}}
\put(1429.0,297.0){\rule[-0.200pt]{2.409pt}{0.400pt}}
\put(170.0,297.0){\rule[-0.200pt]{2.409pt}{0.400pt}}
\put(1429.0,297.0){\rule[-0.200pt]{2.409pt}{0.400pt}}
\put(170.0,297.0){\rule[-0.200pt]{2.409pt}{0.400pt}}
\put(1429.0,297.0){\rule[-0.200pt]{2.409pt}{0.400pt}}
\put(170.0,297.0){\rule[-0.200pt]{2.409pt}{0.400pt}}
\put(1429.0,297.0){\rule[-0.200pt]{2.409pt}{0.400pt}}
\put(170.0,297.0){\rule[-0.200pt]{2.409pt}{0.400pt}}
\put(1429.0,297.0){\rule[-0.200pt]{2.409pt}{0.400pt}}
\put(170.0,297.0){\rule[-0.200pt]{2.409pt}{0.400pt}}
\put(1429.0,297.0){\rule[-0.200pt]{2.409pt}{0.400pt}}
\put(170.0,298.0){\rule[-0.200pt]{2.409pt}{0.400pt}}
\put(1429.0,298.0){\rule[-0.200pt]{2.409pt}{0.400pt}}
\put(170.0,298.0){\rule[-0.200pt]{2.409pt}{0.400pt}}
\put(1429.0,298.0){\rule[-0.200pt]{2.409pt}{0.400pt}}
\put(170.0,298.0){\rule[-0.200pt]{2.409pt}{0.400pt}}
\put(1429.0,298.0){\rule[-0.200pt]{2.409pt}{0.400pt}}
\put(170.0,298.0){\rule[-0.200pt]{2.409pt}{0.400pt}}
\put(1429.0,298.0){\rule[-0.200pt]{2.409pt}{0.400pt}}
\put(170.0,298.0){\rule[-0.200pt]{2.409pt}{0.400pt}}
\put(1429.0,298.0){\rule[-0.200pt]{2.409pt}{0.400pt}}
\put(170.0,298.0){\rule[-0.200pt]{2.409pt}{0.400pt}}
\put(1429.0,298.0){\rule[-0.200pt]{2.409pt}{0.400pt}}
\put(170.0,298.0){\rule[-0.200pt]{2.409pt}{0.400pt}}
\put(1429.0,298.0){\rule[-0.200pt]{2.409pt}{0.400pt}}
\put(170.0,298.0){\rule[-0.200pt]{2.409pt}{0.400pt}}
\put(1429.0,298.0){\rule[-0.200pt]{2.409pt}{0.400pt}}
\put(170.0,298.0){\rule[-0.200pt]{2.409pt}{0.400pt}}
\put(1429.0,298.0){\rule[-0.200pt]{2.409pt}{0.400pt}}
\put(170.0,298.0){\rule[-0.200pt]{2.409pt}{0.400pt}}
\put(1429.0,298.0){\rule[-0.200pt]{2.409pt}{0.400pt}}
\put(170.0,298.0){\rule[-0.200pt]{2.409pt}{0.400pt}}
\put(1429.0,298.0){\rule[-0.200pt]{2.409pt}{0.400pt}}
\put(170.0,298.0){\rule[-0.200pt]{2.409pt}{0.400pt}}
\put(1429.0,298.0){\rule[-0.200pt]{2.409pt}{0.400pt}}
\put(170.0,298.0){\rule[-0.200pt]{2.409pt}{0.400pt}}
\put(1429.0,298.0){\rule[-0.200pt]{2.409pt}{0.400pt}}
\put(170.0,298.0){\rule[-0.200pt]{2.409pt}{0.400pt}}
\put(1429.0,298.0){\rule[-0.200pt]{2.409pt}{0.400pt}}
\put(170.0,298.0){\rule[-0.200pt]{2.409pt}{0.400pt}}
\put(1429.0,298.0){\rule[-0.200pt]{2.409pt}{0.400pt}}
\put(170.0,299.0){\rule[-0.200pt]{2.409pt}{0.400pt}}
\put(1429.0,299.0){\rule[-0.200pt]{2.409pt}{0.400pt}}
\put(170.0,299.0){\rule[-0.200pt]{2.409pt}{0.400pt}}
\put(1429.0,299.0){\rule[-0.200pt]{2.409pt}{0.400pt}}
\put(170.0,299.0){\rule[-0.200pt]{2.409pt}{0.400pt}}
\put(1429.0,299.0){\rule[-0.200pt]{2.409pt}{0.400pt}}
\put(170.0,299.0){\rule[-0.200pt]{2.409pt}{0.400pt}}
\put(1429.0,299.0){\rule[-0.200pt]{2.409pt}{0.400pt}}
\put(170.0,299.0){\rule[-0.200pt]{2.409pt}{0.400pt}}
\put(1429.0,299.0){\rule[-0.200pt]{2.409pt}{0.400pt}}
\put(170.0,299.0){\rule[-0.200pt]{2.409pt}{0.400pt}}
\put(1429.0,299.0){\rule[-0.200pt]{2.409pt}{0.400pt}}
\put(170.0,299.0){\rule[-0.200pt]{2.409pt}{0.400pt}}
\put(1429.0,299.0){\rule[-0.200pt]{2.409pt}{0.400pt}}
\put(170.0,299.0){\rule[-0.200pt]{2.409pt}{0.400pt}}
\put(1429.0,299.0){\rule[-0.200pt]{2.409pt}{0.400pt}}
\put(170.0,299.0){\rule[-0.200pt]{2.409pt}{0.400pt}}
\put(1429.0,299.0){\rule[-0.200pt]{2.409pt}{0.400pt}}
\put(170.0,299.0){\rule[-0.200pt]{2.409pt}{0.400pt}}
\put(1429.0,299.0){\rule[-0.200pt]{2.409pt}{0.400pt}}
\put(170.0,299.0){\rule[-0.200pt]{2.409pt}{0.400pt}}
\put(1429.0,299.0){\rule[-0.200pt]{2.409pt}{0.400pt}}
\put(170.0,299.0){\rule[-0.200pt]{2.409pt}{0.400pt}}
\put(1429.0,299.0){\rule[-0.200pt]{2.409pt}{0.400pt}}
\put(170.0,299.0){\rule[-0.200pt]{2.409pt}{0.400pt}}
\put(1429.0,299.0){\rule[-0.200pt]{2.409pt}{0.400pt}}
\put(170.0,299.0){\rule[-0.200pt]{2.409pt}{0.400pt}}
\put(1429.0,299.0){\rule[-0.200pt]{2.409pt}{0.400pt}}
\put(170.0,299.0){\rule[-0.200pt]{2.409pt}{0.400pt}}
\put(1429.0,299.0){\rule[-0.200pt]{2.409pt}{0.400pt}}
\put(170.0,300.0){\rule[-0.200pt]{2.409pt}{0.400pt}}
\put(1429.0,300.0){\rule[-0.200pt]{2.409pt}{0.400pt}}
\put(170.0,300.0){\rule[-0.200pt]{2.409pt}{0.400pt}}
\put(1429.0,300.0){\rule[-0.200pt]{2.409pt}{0.400pt}}
\put(170.0,300.0){\rule[-0.200pt]{2.409pt}{0.400pt}}
\put(1429.0,300.0){\rule[-0.200pt]{2.409pt}{0.400pt}}
\put(170.0,300.0){\rule[-0.200pt]{2.409pt}{0.400pt}}
\put(1429.0,300.0){\rule[-0.200pt]{2.409pt}{0.400pt}}
\put(170.0,300.0){\rule[-0.200pt]{2.409pt}{0.400pt}}
\put(1429.0,300.0){\rule[-0.200pt]{2.409pt}{0.400pt}}
\put(170.0,300.0){\rule[-0.200pt]{2.409pt}{0.400pt}}
\put(1429.0,300.0){\rule[-0.200pt]{2.409pt}{0.400pt}}
\put(170.0,300.0){\rule[-0.200pt]{2.409pt}{0.400pt}}
\put(1429.0,300.0){\rule[-0.200pt]{2.409pt}{0.400pt}}
\put(170.0,300.0){\rule[-0.200pt]{2.409pt}{0.400pt}}
\put(1429.0,300.0){\rule[-0.200pt]{2.409pt}{0.400pt}}
\put(170.0,300.0){\rule[-0.200pt]{2.409pt}{0.400pt}}
\put(1429.0,300.0){\rule[-0.200pt]{2.409pt}{0.400pt}}
\put(170.0,300.0){\rule[-0.200pt]{2.409pt}{0.400pt}}
\put(1429.0,300.0){\rule[-0.200pt]{2.409pt}{0.400pt}}
\put(170.0,300.0){\rule[-0.200pt]{2.409pt}{0.400pt}}
\put(1429.0,300.0){\rule[-0.200pt]{2.409pt}{0.400pt}}
\put(170.0,300.0){\rule[-0.200pt]{2.409pt}{0.400pt}}
\put(1429.0,300.0){\rule[-0.200pt]{2.409pt}{0.400pt}}
\put(170.0,300.0){\rule[-0.200pt]{2.409pt}{0.400pt}}
\put(1429.0,300.0){\rule[-0.200pt]{2.409pt}{0.400pt}}
\put(170.0,300.0){\rule[-0.200pt]{2.409pt}{0.400pt}}
\put(1429.0,300.0){\rule[-0.200pt]{2.409pt}{0.400pt}}
\put(170.0,300.0){\rule[-0.200pt]{2.409pt}{0.400pt}}
\put(1429.0,300.0){\rule[-0.200pt]{2.409pt}{0.400pt}}
\put(170.0,300.0){\rule[-0.200pt]{2.409pt}{0.400pt}}
\put(1429.0,300.0){\rule[-0.200pt]{2.409pt}{0.400pt}}
\put(170.0,301.0){\rule[-0.200pt]{2.409pt}{0.400pt}}
\put(1429.0,301.0){\rule[-0.200pt]{2.409pt}{0.400pt}}
\put(170.0,301.0){\rule[-0.200pt]{2.409pt}{0.400pt}}
\put(1429.0,301.0){\rule[-0.200pt]{2.409pt}{0.400pt}}
\put(170.0,301.0){\rule[-0.200pt]{2.409pt}{0.400pt}}
\put(1429.0,301.0){\rule[-0.200pt]{2.409pt}{0.400pt}}
\put(170.0,301.0){\rule[-0.200pt]{2.409pt}{0.400pt}}
\put(1429.0,301.0){\rule[-0.200pt]{2.409pt}{0.400pt}}
\put(170.0,301.0){\rule[-0.200pt]{2.409pt}{0.400pt}}
\put(1429.0,301.0){\rule[-0.200pt]{2.409pt}{0.400pt}}
\put(170.0,301.0){\rule[-0.200pt]{2.409pt}{0.400pt}}
\put(1429.0,301.0){\rule[-0.200pt]{2.409pt}{0.400pt}}
\put(170.0,301.0){\rule[-0.200pt]{2.409pt}{0.400pt}}
\put(1429.0,301.0){\rule[-0.200pt]{2.409pt}{0.400pt}}
\put(170.0,301.0){\rule[-0.200pt]{2.409pt}{0.400pt}}
\put(1429.0,301.0){\rule[-0.200pt]{2.409pt}{0.400pt}}
\put(170.0,301.0){\rule[-0.200pt]{2.409pt}{0.400pt}}
\put(1429.0,301.0){\rule[-0.200pt]{2.409pt}{0.400pt}}
\put(170.0,301.0){\rule[-0.200pt]{2.409pt}{0.400pt}}
\put(1429.0,301.0){\rule[-0.200pt]{2.409pt}{0.400pt}}
\put(170.0,301.0){\rule[-0.200pt]{2.409pt}{0.400pt}}
\put(1429.0,301.0){\rule[-0.200pt]{2.409pt}{0.400pt}}
\put(170.0,301.0){\rule[-0.200pt]{2.409pt}{0.400pt}}
\put(1429.0,301.0){\rule[-0.200pt]{2.409pt}{0.400pt}}
\put(170.0,301.0){\rule[-0.200pt]{2.409pt}{0.400pt}}
\put(1429.0,301.0){\rule[-0.200pt]{2.409pt}{0.400pt}}
\put(170.0,301.0){\rule[-0.200pt]{2.409pt}{0.400pt}}
\put(1429.0,301.0){\rule[-0.200pt]{2.409pt}{0.400pt}}
\put(170.0,301.0){\rule[-0.200pt]{2.409pt}{0.400pt}}
\put(1429.0,301.0){\rule[-0.200pt]{2.409pt}{0.400pt}}
\put(170.0,301.0){\rule[-0.200pt]{2.409pt}{0.400pt}}
\put(1429.0,301.0){\rule[-0.200pt]{2.409pt}{0.400pt}}
\put(170.0,301.0){\rule[-0.200pt]{2.409pt}{0.400pt}}
\put(1429.0,301.0){\rule[-0.200pt]{2.409pt}{0.400pt}}
\put(170.0,302.0){\rule[-0.200pt]{2.409pt}{0.400pt}}
\put(1429.0,302.0){\rule[-0.200pt]{2.409pt}{0.400pt}}
\put(170.0,302.0){\rule[-0.200pt]{2.409pt}{0.400pt}}
\put(1429.0,302.0){\rule[-0.200pt]{2.409pt}{0.400pt}}
\put(170.0,302.0){\rule[-0.200pt]{2.409pt}{0.400pt}}
\put(1429.0,302.0){\rule[-0.200pt]{2.409pt}{0.400pt}}
\put(170.0,302.0){\rule[-0.200pt]{2.409pt}{0.400pt}}
\put(1429.0,302.0){\rule[-0.200pt]{2.409pt}{0.400pt}}
\put(170.0,302.0){\rule[-0.200pt]{2.409pt}{0.400pt}}
\put(1429.0,302.0){\rule[-0.200pt]{2.409pt}{0.400pt}}
\put(170.0,302.0){\rule[-0.200pt]{2.409pt}{0.400pt}}
\put(1429.0,302.0){\rule[-0.200pt]{2.409pt}{0.400pt}}
\put(170.0,302.0){\rule[-0.200pt]{2.409pt}{0.400pt}}
\put(1429.0,302.0){\rule[-0.200pt]{2.409pt}{0.400pt}}
\put(170.0,302.0){\rule[-0.200pt]{2.409pt}{0.400pt}}
\put(1429.0,302.0){\rule[-0.200pt]{2.409pt}{0.400pt}}
\put(170.0,302.0){\rule[-0.200pt]{2.409pt}{0.400pt}}
\put(1429.0,302.0){\rule[-0.200pt]{2.409pt}{0.400pt}}
\put(170.0,302.0){\rule[-0.200pt]{2.409pt}{0.400pt}}
\put(1429.0,302.0){\rule[-0.200pt]{2.409pt}{0.400pt}}
\put(170.0,302.0){\rule[-0.200pt]{2.409pt}{0.400pt}}
\put(1429.0,302.0){\rule[-0.200pt]{2.409pt}{0.400pt}}
\put(170.0,302.0){\rule[-0.200pt]{2.409pt}{0.400pt}}
\put(1429.0,302.0){\rule[-0.200pt]{2.409pt}{0.400pt}}
\put(170.0,302.0){\rule[-0.200pt]{2.409pt}{0.400pt}}
\put(1429.0,302.0){\rule[-0.200pt]{2.409pt}{0.400pt}}
\put(170.0,302.0){\rule[-0.200pt]{2.409pt}{0.400pt}}
\put(1429.0,302.0){\rule[-0.200pt]{2.409pt}{0.400pt}}
\put(170.0,302.0){\rule[-0.200pt]{2.409pt}{0.400pt}}
\put(1429.0,302.0){\rule[-0.200pt]{2.409pt}{0.400pt}}
\put(170.0,302.0){\rule[-0.200pt]{2.409pt}{0.400pt}}
\put(1429.0,302.0){\rule[-0.200pt]{2.409pt}{0.400pt}}
\put(170.0,302.0){\rule[-0.200pt]{2.409pt}{0.400pt}}
\put(1429.0,302.0){\rule[-0.200pt]{2.409pt}{0.400pt}}
\put(170.0,302.0){\rule[-0.200pt]{2.409pt}{0.400pt}}
\put(1429.0,302.0){\rule[-0.200pt]{2.409pt}{0.400pt}}
\put(170.0,303.0){\rule[-0.200pt]{2.409pt}{0.400pt}}
\put(1429.0,303.0){\rule[-0.200pt]{2.409pt}{0.400pt}}
\put(170.0,303.0){\rule[-0.200pt]{2.409pt}{0.400pt}}
\put(1429.0,303.0){\rule[-0.200pt]{2.409pt}{0.400pt}}
\put(170.0,303.0){\rule[-0.200pt]{2.409pt}{0.400pt}}
\put(1429.0,303.0){\rule[-0.200pt]{2.409pt}{0.400pt}}
\put(170.0,303.0){\rule[-0.200pt]{2.409pt}{0.400pt}}
\put(1429.0,303.0){\rule[-0.200pt]{2.409pt}{0.400pt}}
\put(170.0,303.0){\rule[-0.200pt]{2.409pt}{0.400pt}}
\put(1429.0,303.0){\rule[-0.200pt]{2.409pt}{0.400pt}}
\put(170.0,303.0){\rule[-0.200pt]{2.409pt}{0.400pt}}
\put(1429.0,303.0){\rule[-0.200pt]{2.409pt}{0.400pt}}
\put(170.0,303.0){\rule[-0.200pt]{2.409pt}{0.400pt}}
\put(1429.0,303.0){\rule[-0.200pt]{2.409pt}{0.400pt}}
\put(170.0,303.0){\rule[-0.200pt]{2.409pt}{0.400pt}}
\put(1429.0,303.0){\rule[-0.200pt]{2.409pt}{0.400pt}}
\put(170.0,303.0){\rule[-0.200pt]{2.409pt}{0.400pt}}
\put(1429.0,303.0){\rule[-0.200pt]{2.409pt}{0.400pt}}
\put(170.0,303.0){\rule[-0.200pt]{2.409pt}{0.400pt}}
\put(1429.0,303.0){\rule[-0.200pt]{2.409pt}{0.400pt}}
\put(170.0,303.0){\rule[-0.200pt]{2.409pt}{0.400pt}}
\put(1429.0,303.0){\rule[-0.200pt]{2.409pt}{0.400pt}}
\put(170.0,303.0){\rule[-0.200pt]{2.409pt}{0.400pt}}
\put(1429.0,303.0){\rule[-0.200pt]{2.409pt}{0.400pt}}
\put(170.0,303.0){\rule[-0.200pt]{2.409pt}{0.400pt}}
\put(1429.0,303.0){\rule[-0.200pt]{2.409pt}{0.400pt}}
\put(170.0,303.0){\rule[-0.200pt]{2.409pt}{0.400pt}}
\put(1429.0,303.0){\rule[-0.200pt]{2.409pt}{0.400pt}}
\put(170.0,303.0){\rule[-0.200pt]{2.409pt}{0.400pt}}
\put(1429.0,303.0){\rule[-0.200pt]{2.409pt}{0.400pt}}
\put(170.0,303.0){\rule[-0.200pt]{2.409pt}{0.400pt}}
\put(1429.0,303.0){\rule[-0.200pt]{2.409pt}{0.400pt}}
\put(170.0,303.0){\rule[-0.200pt]{2.409pt}{0.400pt}}
\put(1429.0,303.0){\rule[-0.200pt]{2.409pt}{0.400pt}}
\put(170.0,303.0){\rule[-0.200pt]{2.409pt}{0.400pt}}
\put(1429.0,303.0){\rule[-0.200pt]{2.409pt}{0.400pt}}
\put(170.0,303.0){\rule[-0.200pt]{2.409pt}{0.400pt}}
\put(1429.0,303.0){\rule[-0.200pt]{2.409pt}{0.400pt}}
\put(170.0,303.0){\rule[-0.200pt]{2.409pt}{0.400pt}}
\put(1429.0,303.0){\rule[-0.200pt]{2.409pt}{0.400pt}}
\put(170.0,304.0){\rule[-0.200pt]{2.409pt}{0.400pt}}
\put(1429.0,304.0){\rule[-0.200pt]{2.409pt}{0.400pt}}
\put(170.0,304.0){\rule[-0.200pt]{2.409pt}{0.400pt}}
\put(1429.0,304.0){\rule[-0.200pt]{2.409pt}{0.400pt}}
\put(170.0,304.0){\rule[-0.200pt]{2.409pt}{0.400pt}}
\put(1429.0,304.0){\rule[-0.200pt]{2.409pt}{0.400pt}}
\put(170.0,304.0){\rule[-0.200pt]{2.409pt}{0.400pt}}
\put(1429.0,304.0){\rule[-0.200pt]{2.409pt}{0.400pt}}
\put(170.0,304.0){\rule[-0.200pt]{2.409pt}{0.400pt}}
\put(1429.0,304.0){\rule[-0.200pt]{2.409pt}{0.400pt}}
\put(170.0,304.0){\rule[-0.200pt]{2.409pt}{0.400pt}}
\put(1429.0,304.0){\rule[-0.200pt]{2.409pt}{0.400pt}}
\put(170.0,304.0){\rule[-0.200pt]{2.409pt}{0.400pt}}
\put(1429.0,304.0){\rule[-0.200pt]{2.409pt}{0.400pt}}
\put(170.0,304.0){\rule[-0.200pt]{2.409pt}{0.400pt}}
\put(1429.0,304.0){\rule[-0.200pt]{2.409pt}{0.400pt}}
\put(170.0,304.0){\rule[-0.200pt]{2.409pt}{0.400pt}}
\put(1429.0,304.0){\rule[-0.200pt]{2.409pt}{0.400pt}}
\put(170.0,304.0){\rule[-0.200pt]{2.409pt}{0.400pt}}
\put(1429.0,304.0){\rule[-0.200pt]{2.409pt}{0.400pt}}
\put(170.0,304.0){\rule[-0.200pt]{4.818pt}{0.400pt}}
\put(150,304){\makebox(0,0)[r]{ 1e-12}}
\put(1419.0,304.0){\rule[-0.200pt]{4.818pt}{0.400pt}}
\put(170.0,315.0){\rule[-0.200pt]{2.409pt}{0.400pt}}
\put(1429.0,315.0){\rule[-0.200pt]{2.409pt}{0.400pt}}
\put(170.0,330.0){\rule[-0.200pt]{2.409pt}{0.400pt}}
\put(1429.0,330.0){\rule[-0.200pt]{2.409pt}{0.400pt}}
\put(170.0,337.0){\rule[-0.200pt]{2.409pt}{0.400pt}}
\put(1429.0,337.0){\rule[-0.200pt]{2.409pt}{0.400pt}}
\put(170.0,343.0){\rule[-0.200pt]{2.409pt}{0.400pt}}
\put(1429.0,343.0){\rule[-0.200pt]{2.409pt}{0.400pt}}
\put(170.0,346.0){\rule[-0.200pt]{2.409pt}{0.400pt}}
\put(1429.0,346.0){\rule[-0.200pt]{2.409pt}{0.400pt}}
\put(170.0,350.0){\rule[-0.200pt]{2.409pt}{0.400pt}}
\put(1429.0,350.0){\rule[-0.200pt]{2.409pt}{0.400pt}}
\put(170.0,352.0){\rule[-0.200pt]{2.409pt}{0.400pt}}
\put(1429.0,352.0){\rule[-0.200pt]{2.409pt}{0.400pt}}
\put(170.0,354.0){\rule[-0.200pt]{2.409pt}{0.400pt}}
\put(1429.0,354.0){\rule[-0.200pt]{2.409pt}{0.400pt}}
\put(170.0,356.0){\rule[-0.200pt]{2.409pt}{0.400pt}}
\put(1429.0,356.0){\rule[-0.200pt]{2.409pt}{0.400pt}}
\put(170.0,358.0){\rule[-0.200pt]{2.409pt}{0.400pt}}
\put(1429.0,358.0){\rule[-0.200pt]{2.409pt}{0.400pt}}
\put(170.0,360.0){\rule[-0.200pt]{2.409pt}{0.400pt}}
\put(1429.0,360.0){\rule[-0.200pt]{2.409pt}{0.400pt}}
\put(170.0,361.0){\rule[-0.200pt]{2.409pt}{0.400pt}}
\put(1429.0,361.0){\rule[-0.200pt]{2.409pt}{0.400pt}}
\put(170.0,362.0){\rule[-0.200pt]{2.409pt}{0.400pt}}
\put(1429.0,362.0){\rule[-0.200pt]{2.409pt}{0.400pt}}
\put(170.0,364.0){\rule[-0.200pt]{2.409pt}{0.400pt}}
\put(1429.0,364.0){\rule[-0.200pt]{2.409pt}{0.400pt}}
\put(170.0,365.0){\rule[-0.200pt]{2.409pt}{0.400pt}}
\put(1429.0,365.0){\rule[-0.200pt]{2.409pt}{0.400pt}}
\put(170.0,366.0){\rule[-0.200pt]{2.409pt}{0.400pt}}
\put(1429.0,366.0){\rule[-0.200pt]{2.409pt}{0.400pt}}
\put(170.0,367.0){\rule[-0.200pt]{2.409pt}{0.400pt}}
\put(1429.0,367.0){\rule[-0.200pt]{2.409pt}{0.400pt}}
\put(170.0,368.0){\rule[-0.200pt]{2.409pt}{0.400pt}}
\put(1429.0,368.0){\rule[-0.200pt]{2.409pt}{0.400pt}}
\put(170.0,369.0){\rule[-0.200pt]{2.409pt}{0.400pt}}
\put(1429.0,369.0){\rule[-0.200pt]{2.409pt}{0.400pt}}
\put(170.0,370.0){\rule[-0.200pt]{2.409pt}{0.400pt}}
\put(1429.0,370.0){\rule[-0.200pt]{2.409pt}{0.400pt}}
\put(170.0,370.0){\rule[-0.200pt]{2.409pt}{0.400pt}}
\put(1429.0,370.0){\rule[-0.200pt]{2.409pt}{0.400pt}}
\put(170.0,371.0){\rule[-0.200pt]{2.409pt}{0.400pt}}
\put(1429.0,371.0){\rule[-0.200pt]{2.409pt}{0.400pt}}
\put(170.0,372.0){\rule[-0.200pt]{2.409pt}{0.400pt}}
\put(1429.0,372.0){\rule[-0.200pt]{2.409pt}{0.400pt}}
\put(170.0,372.0){\rule[-0.200pt]{2.409pt}{0.400pt}}
\put(1429.0,372.0){\rule[-0.200pt]{2.409pt}{0.400pt}}
\put(170.0,373.0){\rule[-0.200pt]{2.409pt}{0.400pt}}
\put(1429.0,373.0){\rule[-0.200pt]{2.409pt}{0.400pt}}
\put(170.0,374.0){\rule[-0.200pt]{2.409pt}{0.400pt}}
\put(1429.0,374.0){\rule[-0.200pt]{2.409pt}{0.400pt}}
\put(170.0,374.0){\rule[-0.200pt]{2.409pt}{0.400pt}}
\put(1429.0,374.0){\rule[-0.200pt]{2.409pt}{0.400pt}}
\put(170.0,375.0){\rule[-0.200pt]{2.409pt}{0.400pt}}
\put(1429.0,375.0){\rule[-0.200pt]{2.409pt}{0.400pt}}
\put(170.0,376.0){\rule[-0.200pt]{2.409pt}{0.400pt}}
\put(1429.0,376.0){\rule[-0.200pt]{2.409pt}{0.400pt}}
\put(170.0,376.0){\rule[-0.200pt]{2.409pt}{0.400pt}}
\put(1429.0,376.0){\rule[-0.200pt]{2.409pt}{0.400pt}}
\put(170.0,377.0){\rule[-0.200pt]{2.409pt}{0.400pt}}
\put(1429.0,377.0){\rule[-0.200pt]{2.409pt}{0.400pt}}
\put(170.0,377.0){\rule[-0.200pt]{2.409pt}{0.400pt}}
\put(1429.0,377.0){\rule[-0.200pt]{2.409pt}{0.400pt}}
\put(170.0,378.0){\rule[-0.200pt]{2.409pt}{0.400pt}}
\put(1429.0,378.0){\rule[-0.200pt]{2.409pt}{0.400pt}}
\put(170.0,378.0){\rule[-0.200pt]{2.409pt}{0.400pt}}
\put(1429.0,378.0){\rule[-0.200pt]{2.409pt}{0.400pt}}
\put(170.0,379.0){\rule[-0.200pt]{2.409pt}{0.400pt}}
\put(1429.0,379.0){\rule[-0.200pt]{2.409pt}{0.400pt}}
\put(170.0,379.0){\rule[-0.200pt]{2.409pt}{0.400pt}}
\put(1429.0,379.0){\rule[-0.200pt]{2.409pt}{0.400pt}}
\put(170.0,380.0){\rule[-0.200pt]{2.409pt}{0.400pt}}
\put(1429.0,380.0){\rule[-0.200pt]{2.409pt}{0.400pt}}
\put(170.0,380.0){\rule[-0.200pt]{2.409pt}{0.400pt}}
\put(1429.0,380.0){\rule[-0.200pt]{2.409pt}{0.400pt}}
\put(170.0,380.0){\rule[-0.200pt]{2.409pt}{0.400pt}}
\put(1429.0,380.0){\rule[-0.200pt]{2.409pt}{0.400pt}}
\put(170.0,381.0){\rule[-0.200pt]{2.409pt}{0.400pt}}
\put(1429.0,381.0){\rule[-0.200pt]{2.409pt}{0.400pt}}
\put(170.0,381.0){\rule[-0.200pt]{2.409pt}{0.400pt}}
\put(1429.0,381.0){\rule[-0.200pt]{2.409pt}{0.400pt}}
\put(170.0,382.0){\rule[-0.200pt]{2.409pt}{0.400pt}}
\put(1429.0,382.0){\rule[-0.200pt]{2.409pt}{0.400pt}}
\put(170.0,382.0){\rule[-0.200pt]{2.409pt}{0.400pt}}
\put(1429.0,382.0){\rule[-0.200pt]{2.409pt}{0.400pt}}
\put(170.0,382.0){\rule[-0.200pt]{2.409pt}{0.400pt}}
\put(1429.0,382.0){\rule[-0.200pt]{2.409pt}{0.400pt}}
\put(170.0,383.0){\rule[-0.200pt]{2.409pt}{0.400pt}}
\put(1429.0,383.0){\rule[-0.200pt]{2.409pt}{0.400pt}}
\put(170.0,383.0){\rule[-0.200pt]{2.409pt}{0.400pt}}
\put(1429.0,383.0){\rule[-0.200pt]{2.409pt}{0.400pt}}
\put(170.0,383.0){\rule[-0.200pt]{2.409pt}{0.400pt}}
\put(1429.0,383.0){\rule[-0.200pt]{2.409pt}{0.400pt}}
\put(170.0,384.0){\rule[-0.200pt]{2.409pt}{0.400pt}}
\put(1429.0,384.0){\rule[-0.200pt]{2.409pt}{0.400pt}}
\put(170.0,384.0){\rule[-0.200pt]{2.409pt}{0.400pt}}
\put(1429.0,384.0){\rule[-0.200pt]{2.409pt}{0.400pt}}
\put(170.0,384.0){\rule[-0.200pt]{2.409pt}{0.400pt}}
\put(1429.0,384.0){\rule[-0.200pt]{2.409pt}{0.400pt}}
\put(170.0,385.0){\rule[-0.200pt]{2.409pt}{0.400pt}}
\put(1429.0,385.0){\rule[-0.200pt]{2.409pt}{0.400pt}}
\put(170.0,385.0){\rule[-0.200pt]{2.409pt}{0.400pt}}
\put(1429.0,385.0){\rule[-0.200pt]{2.409pt}{0.400pt}}
\put(170.0,385.0){\rule[-0.200pt]{2.409pt}{0.400pt}}
\put(1429.0,385.0){\rule[-0.200pt]{2.409pt}{0.400pt}}
\put(170.0,386.0){\rule[-0.200pt]{2.409pt}{0.400pt}}
\put(1429.0,386.0){\rule[-0.200pt]{2.409pt}{0.400pt}}
\put(170.0,386.0){\rule[-0.200pt]{2.409pt}{0.400pt}}
\put(1429.0,386.0){\rule[-0.200pt]{2.409pt}{0.400pt}}
\put(170.0,386.0){\rule[-0.200pt]{2.409pt}{0.400pt}}
\put(1429.0,386.0){\rule[-0.200pt]{2.409pt}{0.400pt}}
\put(170.0,387.0){\rule[-0.200pt]{2.409pt}{0.400pt}}
\put(1429.0,387.0){\rule[-0.200pt]{2.409pt}{0.400pt}}
\put(170.0,387.0){\rule[-0.200pt]{2.409pt}{0.400pt}}
\put(1429.0,387.0){\rule[-0.200pt]{2.409pt}{0.400pt}}
\put(170.0,387.0){\rule[-0.200pt]{2.409pt}{0.400pt}}
\put(1429.0,387.0){\rule[-0.200pt]{2.409pt}{0.400pt}}
\put(170.0,387.0){\rule[-0.200pt]{2.409pt}{0.400pt}}
\put(1429.0,387.0){\rule[-0.200pt]{2.409pt}{0.400pt}}
\put(170.0,388.0){\rule[-0.200pt]{2.409pt}{0.400pt}}
\put(1429.0,388.0){\rule[-0.200pt]{2.409pt}{0.400pt}}
\put(170.0,388.0){\rule[-0.200pt]{2.409pt}{0.400pt}}
\put(1429.0,388.0){\rule[-0.200pt]{2.409pt}{0.400pt}}
\put(170.0,388.0){\rule[-0.200pt]{2.409pt}{0.400pt}}
\put(1429.0,388.0){\rule[-0.200pt]{2.409pt}{0.400pt}}
\put(170.0,388.0){\rule[-0.200pt]{2.409pt}{0.400pt}}
\put(1429.0,388.0){\rule[-0.200pt]{2.409pt}{0.400pt}}
\put(170.0,389.0){\rule[-0.200pt]{2.409pt}{0.400pt}}
\put(1429.0,389.0){\rule[-0.200pt]{2.409pt}{0.400pt}}
\put(170.0,389.0){\rule[-0.200pt]{2.409pt}{0.400pt}}
\put(1429.0,389.0){\rule[-0.200pt]{2.409pt}{0.400pt}}
\put(170.0,389.0){\rule[-0.200pt]{2.409pt}{0.400pt}}
\put(1429.0,389.0){\rule[-0.200pt]{2.409pt}{0.400pt}}
\put(170.0,389.0){\rule[-0.200pt]{2.409pt}{0.400pt}}
\put(1429.0,389.0){\rule[-0.200pt]{2.409pt}{0.400pt}}
\put(170.0,390.0){\rule[-0.200pt]{2.409pt}{0.400pt}}
\put(1429.0,390.0){\rule[-0.200pt]{2.409pt}{0.400pt}}
\put(170.0,390.0){\rule[-0.200pt]{2.409pt}{0.400pt}}
\put(1429.0,390.0){\rule[-0.200pt]{2.409pt}{0.400pt}}
\put(170.0,390.0){\rule[-0.200pt]{2.409pt}{0.400pt}}
\put(1429.0,390.0){\rule[-0.200pt]{2.409pt}{0.400pt}}
\put(170.0,390.0){\rule[-0.200pt]{2.409pt}{0.400pt}}
\put(1429.0,390.0){\rule[-0.200pt]{2.409pt}{0.400pt}}
\put(170.0,391.0){\rule[-0.200pt]{2.409pt}{0.400pt}}
\put(1429.0,391.0){\rule[-0.200pt]{2.409pt}{0.400pt}}
\put(170.0,391.0){\rule[-0.200pt]{2.409pt}{0.400pt}}
\put(1429.0,391.0){\rule[-0.200pt]{2.409pt}{0.400pt}}
\put(170.0,391.0){\rule[-0.200pt]{2.409pt}{0.400pt}}
\put(1429.0,391.0){\rule[-0.200pt]{2.409pt}{0.400pt}}
\put(170.0,391.0){\rule[-0.200pt]{2.409pt}{0.400pt}}
\put(1429.0,391.0){\rule[-0.200pt]{2.409pt}{0.400pt}}
\put(170.0,391.0){\rule[-0.200pt]{2.409pt}{0.400pt}}
\put(1429.0,391.0){\rule[-0.200pt]{2.409pt}{0.400pt}}
\put(170.0,392.0){\rule[-0.200pt]{2.409pt}{0.400pt}}
\put(1429.0,392.0){\rule[-0.200pt]{2.409pt}{0.400pt}}
\put(170.0,392.0){\rule[-0.200pt]{2.409pt}{0.400pt}}
\put(1429.0,392.0){\rule[-0.200pt]{2.409pt}{0.400pt}}
\put(170.0,392.0){\rule[-0.200pt]{2.409pt}{0.400pt}}
\put(1429.0,392.0){\rule[-0.200pt]{2.409pt}{0.400pt}}
\put(170.0,392.0){\rule[-0.200pt]{2.409pt}{0.400pt}}
\put(1429.0,392.0){\rule[-0.200pt]{2.409pt}{0.400pt}}
\put(170.0,392.0){\rule[-0.200pt]{2.409pt}{0.400pt}}
\put(1429.0,392.0){\rule[-0.200pt]{2.409pt}{0.400pt}}
\put(170.0,393.0){\rule[-0.200pt]{2.409pt}{0.400pt}}
\put(1429.0,393.0){\rule[-0.200pt]{2.409pt}{0.400pt}}
\put(170.0,393.0){\rule[-0.200pt]{2.409pt}{0.400pt}}
\put(1429.0,393.0){\rule[-0.200pt]{2.409pt}{0.400pt}}
\put(170.0,393.0){\rule[-0.200pt]{2.409pt}{0.400pt}}
\put(1429.0,393.0){\rule[-0.200pt]{2.409pt}{0.400pt}}
\put(170.0,393.0){\rule[-0.200pt]{2.409pt}{0.400pt}}
\put(1429.0,393.0){\rule[-0.200pt]{2.409pt}{0.400pt}}
\put(170.0,393.0){\rule[-0.200pt]{2.409pt}{0.400pt}}
\put(1429.0,393.0){\rule[-0.200pt]{2.409pt}{0.400pt}}
\put(170.0,394.0){\rule[-0.200pt]{2.409pt}{0.400pt}}
\put(1429.0,394.0){\rule[-0.200pt]{2.409pt}{0.400pt}}
\put(170.0,394.0){\rule[-0.200pt]{2.409pt}{0.400pt}}
\put(1429.0,394.0){\rule[-0.200pt]{2.409pt}{0.400pt}}
\put(170.0,394.0){\rule[-0.200pt]{2.409pt}{0.400pt}}
\put(1429.0,394.0){\rule[-0.200pt]{2.409pt}{0.400pt}}
\put(170.0,394.0){\rule[-0.200pt]{2.409pt}{0.400pt}}
\put(1429.0,394.0){\rule[-0.200pt]{2.409pt}{0.400pt}}
\put(170.0,394.0){\rule[-0.200pt]{2.409pt}{0.400pt}}
\put(1429.0,394.0){\rule[-0.200pt]{2.409pt}{0.400pt}}
\put(170.0,394.0){\rule[-0.200pt]{2.409pt}{0.400pt}}
\put(1429.0,394.0){\rule[-0.200pt]{2.409pt}{0.400pt}}
\put(170.0,395.0){\rule[-0.200pt]{2.409pt}{0.400pt}}
\put(1429.0,395.0){\rule[-0.200pt]{2.409pt}{0.400pt}}
\put(170.0,395.0){\rule[-0.200pt]{2.409pt}{0.400pt}}
\put(1429.0,395.0){\rule[-0.200pt]{2.409pt}{0.400pt}}
\put(170.0,395.0){\rule[-0.200pt]{2.409pt}{0.400pt}}
\put(1429.0,395.0){\rule[-0.200pt]{2.409pt}{0.400pt}}
\put(170.0,395.0){\rule[-0.200pt]{2.409pt}{0.400pt}}
\put(1429.0,395.0){\rule[-0.200pt]{2.409pt}{0.400pt}}
\put(170.0,395.0){\rule[-0.200pt]{2.409pt}{0.400pt}}
\put(1429.0,395.0){\rule[-0.200pt]{2.409pt}{0.400pt}}
\put(170.0,395.0){\rule[-0.200pt]{2.409pt}{0.400pt}}
\put(1429.0,395.0){\rule[-0.200pt]{2.409pt}{0.400pt}}
\put(170.0,396.0){\rule[-0.200pt]{2.409pt}{0.400pt}}
\put(1429.0,396.0){\rule[-0.200pt]{2.409pt}{0.400pt}}
\put(170.0,396.0){\rule[-0.200pt]{2.409pt}{0.400pt}}
\put(1429.0,396.0){\rule[-0.200pt]{2.409pt}{0.400pt}}
\put(170.0,396.0){\rule[-0.200pt]{2.409pt}{0.400pt}}
\put(1429.0,396.0){\rule[-0.200pt]{2.409pt}{0.400pt}}
\put(170.0,396.0){\rule[-0.200pt]{2.409pt}{0.400pt}}
\put(1429.0,396.0){\rule[-0.200pt]{2.409pt}{0.400pt}}
\put(170.0,396.0){\rule[-0.200pt]{2.409pt}{0.400pt}}
\put(1429.0,396.0){\rule[-0.200pt]{2.409pt}{0.400pt}}
\put(170.0,396.0){\rule[-0.200pt]{2.409pt}{0.400pt}}
\put(1429.0,396.0){\rule[-0.200pt]{2.409pt}{0.400pt}}
\put(170.0,397.0){\rule[-0.200pt]{2.409pt}{0.400pt}}
\put(1429.0,397.0){\rule[-0.200pt]{2.409pt}{0.400pt}}
\put(170.0,397.0){\rule[-0.200pt]{2.409pt}{0.400pt}}
\put(1429.0,397.0){\rule[-0.200pt]{2.409pt}{0.400pt}}
\put(170.0,397.0){\rule[-0.200pt]{2.409pt}{0.400pt}}
\put(1429.0,397.0){\rule[-0.200pt]{2.409pt}{0.400pt}}
\put(170.0,397.0){\rule[-0.200pt]{2.409pt}{0.400pt}}
\put(1429.0,397.0){\rule[-0.200pt]{2.409pt}{0.400pt}}
\put(170.0,397.0){\rule[-0.200pt]{2.409pt}{0.400pt}}
\put(1429.0,397.0){\rule[-0.200pt]{2.409pt}{0.400pt}}
\put(170.0,397.0){\rule[-0.200pt]{2.409pt}{0.400pt}}
\put(1429.0,397.0){\rule[-0.200pt]{2.409pt}{0.400pt}}
\put(170.0,397.0){\rule[-0.200pt]{2.409pt}{0.400pt}}
\put(1429.0,397.0){\rule[-0.200pt]{2.409pt}{0.400pt}}
\put(170.0,398.0){\rule[-0.200pt]{2.409pt}{0.400pt}}
\put(1429.0,398.0){\rule[-0.200pt]{2.409pt}{0.400pt}}
\put(170.0,398.0){\rule[-0.200pt]{2.409pt}{0.400pt}}
\put(1429.0,398.0){\rule[-0.200pt]{2.409pt}{0.400pt}}
\put(170.0,398.0){\rule[-0.200pt]{2.409pt}{0.400pt}}
\put(1429.0,398.0){\rule[-0.200pt]{2.409pt}{0.400pt}}
\put(170.0,398.0){\rule[-0.200pt]{2.409pt}{0.400pt}}
\put(1429.0,398.0){\rule[-0.200pt]{2.409pt}{0.400pt}}
\put(170.0,398.0){\rule[-0.200pt]{2.409pt}{0.400pt}}
\put(1429.0,398.0){\rule[-0.200pt]{2.409pt}{0.400pt}}
\put(170.0,398.0){\rule[-0.200pt]{2.409pt}{0.400pt}}
\put(1429.0,398.0){\rule[-0.200pt]{2.409pt}{0.400pt}}
\put(170.0,398.0){\rule[-0.200pt]{2.409pt}{0.400pt}}
\put(1429.0,398.0){\rule[-0.200pt]{2.409pt}{0.400pt}}
\put(170.0,399.0){\rule[-0.200pt]{2.409pt}{0.400pt}}
\put(1429.0,399.0){\rule[-0.200pt]{2.409pt}{0.400pt}}
\put(170.0,399.0){\rule[-0.200pt]{2.409pt}{0.400pt}}
\put(1429.0,399.0){\rule[-0.200pt]{2.409pt}{0.400pt}}
\put(170.0,399.0){\rule[-0.200pt]{2.409pt}{0.400pt}}
\put(1429.0,399.0){\rule[-0.200pt]{2.409pt}{0.400pt}}
\put(170.0,399.0){\rule[-0.200pt]{2.409pt}{0.400pt}}
\put(1429.0,399.0){\rule[-0.200pt]{2.409pt}{0.400pt}}
\put(170.0,399.0){\rule[-0.200pt]{2.409pt}{0.400pt}}
\put(1429.0,399.0){\rule[-0.200pt]{2.409pt}{0.400pt}}
\put(170.0,399.0){\rule[-0.200pt]{2.409pt}{0.400pt}}
\put(1429.0,399.0){\rule[-0.200pt]{2.409pt}{0.400pt}}
\put(170.0,399.0){\rule[-0.200pt]{2.409pt}{0.400pt}}
\put(1429.0,399.0){\rule[-0.200pt]{2.409pt}{0.400pt}}
\put(170.0,399.0){\rule[-0.200pt]{2.409pt}{0.400pt}}
\put(1429.0,399.0){\rule[-0.200pt]{2.409pt}{0.400pt}}
\put(170.0,400.0){\rule[-0.200pt]{2.409pt}{0.400pt}}
\put(1429.0,400.0){\rule[-0.200pt]{2.409pt}{0.400pt}}
\put(170.0,400.0){\rule[-0.200pt]{2.409pt}{0.400pt}}
\put(1429.0,400.0){\rule[-0.200pt]{2.409pt}{0.400pt}}
\put(170.0,400.0){\rule[-0.200pt]{2.409pt}{0.400pt}}
\put(1429.0,400.0){\rule[-0.200pt]{2.409pt}{0.400pt}}
\put(170.0,400.0){\rule[-0.200pt]{2.409pt}{0.400pt}}
\put(1429.0,400.0){\rule[-0.200pt]{2.409pt}{0.400pt}}
\put(170.0,400.0){\rule[-0.200pt]{2.409pt}{0.400pt}}
\put(1429.0,400.0){\rule[-0.200pt]{2.409pt}{0.400pt}}
\put(170.0,400.0){\rule[-0.200pt]{2.409pt}{0.400pt}}
\put(1429.0,400.0){\rule[-0.200pt]{2.409pt}{0.400pt}}
\put(170.0,400.0){\rule[-0.200pt]{2.409pt}{0.400pt}}
\put(1429.0,400.0){\rule[-0.200pt]{2.409pt}{0.400pt}}
\put(170.0,400.0){\rule[-0.200pt]{2.409pt}{0.400pt}}
\put(1429.0,400.0){\rule[-0.200pt]{2.409pt}{0.400pt}}
\put(170.0,401.0){\rule[-0.200pt]{2.409pt}{0.400pt}}
\put(1429.0,401.0){\rule[-0.200pt]{2.409pt}{0.400pt}}
\put(170.0,401.0){\rule[-0.200pt]{2.409pt}{0.400pt}}
\put(1429.0,401.0){\rule[-0.200pt]{2.409pt}{0.400pt}}
\put(170.0,401.0){\rule[-0.200pt]{2.409pt}{0.400pt}}
\put(1429.0,401.0){\rule[-0.200pt]{2.409pt}{0.400pt}}
\put(170.0,401.0){\rule[-0.200pt]{2.409pt}{0.400pt}}
\put(1429.0,401.0){\rule[-0.200pt]{2.409pt}{0.400pt}}
\put(170.0,401.0){\rule[-0.200pt]{2.409pt}{0.400pt}}
\put(1429.0,401.0){\rule[-0.200pt]{2.409pt}{0.400pt}}
\put(170.0,401.0){\rule[-0.200pt]{2.409pt}{0.400pt}}
\put(1429.0,401.0){\rule[-0.200pt]{2.409pt}{0.400pt}}
\put(170.0,401.0){\rule[-0.200pt]{2.409pt}{0.400pt}}
\put(1429.0,401.0){\rule[-0.200pt]{2.409pt}{0.400pt}}
\put(170.0,401.0){\rule[-0.200pt]{2.409pt}{0.400pt}}
\put(1429.0,401.0){\rule[-0.200pt]{2.409pt}{0.400pt}}
\put(170.0,401.0){\rule[-0.200pt]{2.409pt}{0.400pt}}
\put(1429.0,401.0){\rule[-0.200pt]{2.409pt}{0.400pt}}
\put(170.0,402.0){\rule[-0.200pt]{2.409pt}{0.400pt}}
\put(1429.0,402.0){\rule[-0.200pt]{2.409pt}{0.400pt}}
\put(170.0,402.0){\rule[-0.200pt]{2.409pt}{0.400pt}}
\put(1429.0,402.0){\rule[-0.200pt]{2.409pt}{0.400pt}}
\put(170.0,402.0){\rule[-0.200pt]{2.409pt}{0.400pt}}
\put(1429.0,402.0){\rule[-0.200pt]{2.409pt}{0.400pt}}
\put(170.0,402.0){\rule[-0.200pt]{2.409pt}{0.400pt}}
\put(1429.0,402.0){\rule[-0.200pt]{2.409pt}{0.400pt}}
\put(170.0,402.0){\rule[-0.200pt]{2.409pt}{0.400pt}}
\put(1429.0,402.0){\rule[-0.200pt]{2.409pt}{0.400pt}}
\put(170.0,402.0){\rule[-0.200pt]{2.409pt}{0.400pt}}
\put(1429.0,402.0){\rule[-0.200pt]{2.409pt}{0.400pt}}
\put(170.0,402.0){\rule[-0.200pt]{2.409pt}{0.400pt}}
\put(1429.0,402.0){\rule[-0.200pt]{2.409pt}{0.400pt}}
\put(170.0,402.0){\rule[-0.200pt]{2.409pt}{0.400pt}}
\put(1429.0,402.0){\rule[-0.200pt]{2.409pt}{0.400pt}}
\put(170.0,402.0){\rule[-0.200pt]{2.409pt}{0.400pt}}
\put(1429.0,402.0){\rule[-0.200pt]{2.409pt}{0.400pt}}
\put(170.0,403.0){\rule[-0.200pt]{2.409pt}{0.400pt}}
\put(1429.0,403.0){\rule[-0.200pt]{2.409pt}{0.400pt}}
\put(170.0,403.0){\rule[-0.200pt]{2.409pt}{0.400pt}}
\put(1429.0,403.0){\rule[-0.200pt]{2.409pt}{0.400pt}}
\put(170.0,403.0){\rule[-0.200pt]{2.409pt}{0.400pt}}
\put(1429.0,403.0){\rule[-0.200pt]{2.409pt}{0.400pt}}
\put(170.0,403.0){\rule[-0.200pt]{2.409pt}{0.400pt}}
\put(1429.0,403.0){\rule[-0.200pt]{2.409pt}{0.400pt}}
\put(170.0,403.0){\rule[-0.200pt]{2.409pt}{0.400pt}}
\put(1429.0,403.0){\rule[-0.200pt]{2.409pt}{0.400pt}}
\put(170.0,403.0){\rule[-0.200pt]{2.409pt}{0.400pt}}
\put(1429.0,403.0){\rule[-0.200pt]{2.409pt}{0.400pt}}
\put(170.0,403.0){\rule[-0.200pt]{2.409pt}{0.400pt}}
\put(1429.0,403.0){\rule[-0.200pt]{2.409pt}{0.400pt}}
\put(170.0,403.0){\rule[-0.200pt]{2.409pt}{0.400pt}}
\put(1429.0,403.0){\rule[-0.200pt]{2.409pt}{0.400pt}}
\put(170.0,403.0){\rule[-0.200pt]{2.409pt}{0.400pt}}
\put(1429.0,403.0){\rule[-0.200pt]{2.409pt}{0.400pt}}
\put(170.0,403.0){\rule[-0.200pt]{2.409pt}{0.400pt}}
\put(1429.0,403.0){\rule[-0.200pt]{2.409pt}{0.400pt}}
\put(170.0,404.0){\rule[-0.200pt]{2.409pt}{0.400pt}}
\put(1429.0,404.0){\rule[-0.200pt]{2.409pt}{0.400pt}}
\put(170.0,404.0){\rule[-0.200pt]{2.409pt}{0.400pt}}
\put(1429.0,404.0){\rule[-0.200pt]{2.409pt}{0.400pt}}
\put(170.0,404.0){\rule[-0.200pt]{2.409pt}{0.400pt}}
\put(1429.0,404.0){\rule[-0.200pt]{2.409pt}{0.400pt}}
\put(170.0,404.0){\rule[-0.200pt]{2.409pt}{0.400pt}}
\put(1429.0,404.0){\rule[-0.200pt]{2.409pt}{0.400pt}}
\put(170.0,404.0){\rule[-0.200pt]{2.409pt}{0.400pt}}
\put(1429.0,404.0){\rule[-0.200pt]{2.409pt}{0.400pt}}
\put(170.0,404.0){\rule[-0.200pt]{2.409pt}{0.400pt}}
\put(1429.0,404.0){\rule[-0.200pt]{2.409pt}{0.400pt}}
\put(170.0,404.0){\rule[-0.200pt]{2.409pt}{0.400pt}}
\put(1429.0,404.0){\rule[-0.200pt]{2.409pt}{0.400pt}}
\put(170.0,404.0){\rule[-0.200pt]{2.409pt}{0.400pt}}
\put(1429.0,404.0){\rule[-0.200pt]{2.409pt}{0.400pt}}
\put(170.0,404.0){\rule[-0.200pt]{2.409pt}{0.400pt}}
\put(1429.0,404.0){\rule[-0.200pt]{2.409pt}{0.400pt}}
\put(170.0,404.0){\rule[-0.200pt]{2.409pt}{0.400pt}}
\put(1429.0,404.0){\rule[-0.200pt]{2.409pt}{0.400pt}}
\put(170.0,405.0){\rule[-0.200pt]{2.409pt}{0.400pt}}
\put(1429.0,405.0){\rule[-0.200pt]{2.409pt}{0.400pt}}
\put(170.0,405.0){\rule[-0.200pt]{2.409pt}{0.400pt}}
\put(1429.0,405.0){\rule[-0.200pt]{2.409pt}{0.400pt}}
\put(170.0,405.0){\rule[-0.200pt]{2.409pt}{0.400pt}}
\put(1429.0,405.0){\rule[-0.200pt]{2.409pt}{0.400pt}}
\put(170.0,405.0){\rule[-0.200pt]{2.409pt}{0.400pt}}
\put(1429.0,405.0){\rule[-0.200pt]{2.409pt}{0.400pt}}
\put(170.0,405.0){\rule[-0.200pt]{2.409pt}{0.400pt}}
\put(1429.0,405.0){\rule[-0.200pt]{2.409pt}{0.400pt}}
\put(170.0,405.0){\rule[-0.200pt]{2.409pt}{0.400pt}}
\put(1429.0,405.0){\rule[-0.200pt]{2.409pt}{0.400pt}}
\put(170.0,405.0){\rule[-0.200pt]{2.409pt}{0.400pt}}
\put(1429.0,405.0){\rule[-0.200pt]{2.409pt}{0.400pt}}
\put(170.0,405.0){\rule[-0.200pt]{2.409pt}{0.400pt}}
\put(1429.0,405.0){\rule[-0.200pt]{2.409pt}{0.400pt}}
\put(170.0,405.0){\rule[-0.200pt]{2.409pt}{0.400pt}}
\put(1429.0,405.0){\rule[-0.200pt]{2.409pt}{0.400pt}}
\put(170.0,405.0){\rule[-0.200pt]{2.409pt}{0.400pt}}
\put(1429.0,405.0){\rule[-0.200pt]{2.409pt}{0.400pt}}
\put(170.0,405.0){\rule[-0.200pt]{2.409pt}{0.400pt}}
\put(1429.0,405.0){\rule[-0.200pt]{2.409pt}{0.400pt}}
\put(170.0,406.0){\rule[-0.200pt]{2.409pt}{0.400pt}}
\put(1429.0,406.0){\rule[-0.200pt]{2.409pt}{0.400pt}}
\put(170.0,406.0){\rule[-0.200pt]{2.409pt}{0.400pt}}
\put(1429.0,406.0){\rule[-0.200pt]{2.409pt}{0.400pt}}
\put(170.0,406.0){\rule[-0.200pt]{2.409pt}{0.400pt}}
\put(1429.0,406.0){\rule[-0.200pt]{2.409pt}{0.400pt}}
\put(170.0,406.0){\rule[-0.200pt]{2.409pt}{0.400pt}}
\put(1429.0,406.0){\rule[-0.200pt]{2.409pt}{0.400pt}}
\put(170.0,406.0){\rule[-0.200pt]{2.409pt}{0.400pt}}
\put(1429.0,406.0){\rule[-0.200pt]{2.409pt}{0.400pt}}
\put(170.0,406.0){\rule[-0.200pt]{2.409pt}{0.400pt}}
\put(1429.0,406.0){\rule[-0.200pt]{2.409pt}{0.400pt}}
\put(170.0,406.0){\rule[-0.200pt]{2.409pt}{0.400pt}}
\put(1429.0,406.0){\rule[-0.200pt]{2.409pt}{0.400pt}}
\put(170.0,406.0){\rule[-0.200pt]{2.409pt}{0.400pt}}
\put(1429.0,406.0){\rule[-0.200pt]{2.409pt}{0.400pt}}
\put(170.0,406.0){\rule[-0.200pt]{2.409pt}{0.400pt}}
\put(1429.0,406.0){\rule[-0.200pt]{2.409pt}{0.400pt}}
\put(170.0,406.0){\rule[-0.200pt]{2.409pt}{0.400pt}}
\put(1429.0,406.0){\rule[-0.200pt]{2.409pt}{0.400pt}}
\put(170.0,406.0){\rule[-0.200pt]{2.409pt}{0.400pt}}
\put(1429.0,406.0){\rule[-0.200pt]{2.409pt}{0.400pt}}
\put(170.0,406.0){\rule[-0.200pt]{2.409pt}{0.400pt}}
\put(1429.0,406.0){\rule[-0.200pt]{2.409pt}{0.400pt}}
\put(170.0,407.0){\rule[-0.200pt]{2.409pt}{0.400pt}}
\put(1429.0,407.0){\rule[-0.200pt]{2.409pt}{0.400pt}}
\put(170.0,407.0){\rule[-0.200pt]{2.409pt}{0.400pt}}
\put(1429.0,407.0){\rule[-0.200pt]{2.409pt}{0.400pt}}
\put(170.0,407.0){\rule[-0.200pt]{2.409pt}{0.400pt}}
\put(1429.0,407.0){\rule[-0.200pt]{2.409pt}{0.400pt}}
\put(170.0,407.0){\rule[-0.200pt]{2.409pt}{0.400pt}}
\put(1429.0,407.0){\rule[-0.200pt]{2.409pt}{0.400pt}}
\put(170.0,407.0){\rule[-0.200pt]{2.409pt}{0.400pt}}
\put(1429.0,407.0){\rule[-0.200pt]{2.409pt}{0.400pt}}
\put(170.0,407.0){\rule[-0.200pt]{2.409pt}{0.400pt}}
\put(1429.0,407.0){\rule[-0.200pt]{2.409pt}{0.400pt}}
\put(170.0,407.0){\rule[-0.200pt]{2.409pt}{0.400pt}}
\put(1429.0,407.0){\rule[-0.200pt]{2.409pt}{0.400pt}}
\put(170.0,407.0){\rule[-0.200pt]{2.409pt}{0.400pt}}
\put(1429.0,407.0){\rule[-0.200pt]{2.409pt}{0.400pt}}
\put(170.0,407.0){\rule[-0.200pt]{2.409pt}{0.400pt}}
\put(1429.0,407.0){\rule[-0.200pt]{2.409pt}{0.400pt}}
\put(170.0,407.0){\rule[-0.200pt]{2.409pt}{0.400pt}}
\put(1429.0,407.0){\rule[-0.200pt]{2.409pt}{0.400pt}}
\put(170.0,407.0){\rule[-0.200pt]{2.409pt}{0.400pt}}
\put(1429.0,407.0){\rule[-0.200pt]{2.409pt}{0.400pt}}
\put(170.0,407.0){\rule[-0.200pt]{2.409pt}{0.400pt}}
\put(1429.0,407.0){\rule[-0.200pt]{2.409pt}{0.400pt}}
\put(170.0,407.0){\rule[-0.200pt]{2.409pt}{0.400pt}}
\put(1429.0,407.0){\rule[-0.200pt]{2.409pt}{0.400pt}}
\put(170.0,408.0){\rule[-0.200pt]{2.409pt}{0.400pt}}
\put(1429.0,408.0){\rule[-0.200pt]{2.409pt}{0.400pt}}
\put(170.0,408.0){\rule[-0.200pt]{2.409pt}{0.400pt}}
\put(1429.0,408.0){\rule[-0.200pt]{2.409pt}{0.400pt}}
\put(170.0,408.0){\rule[-0.200pt]{2.409pt}{0.400pt}}
\put(1429.0,408.0){\rule[-0.200pt]{2.409pt}{0.400pt}}
\put(170.0,408.0){\rule[-0.200pt]{2.409pt}{0.400pt}}
\put(1429.0,408.0){\rule[-0.200pt]{2.409pt}{0.400pt}}
\put(170.0,408.0){\rule[-0.200pt]{2.409pt}{0.400pt}}
\put(1429.0,408.0){\rule[-0.200pt]{2.409pt}{0.400pt}}
\put(170.0,408.0){\rule[-0.200pt]{2.409pt}{0.400pt}}
\put(1429.0,408.0){\rule[-0.200pt]{2.409pt}{0.400pt}}
\put(170.0,408.0){\rule[-0.200pt]{2.409pt}{0.400pt}}
\put(1429.0,408.0){\rule[-0.200pt]{2.409pt}{0.400pt}}
\put(170.0,408.0){\rule[-0.200pt]{2.409pt}{0.400pt}}
\put(1429.0,408.0){\rule[-0.200pt]{2.409pt}{0.400pt}}
\put(170.0,408.0){\rule[-0.200pt]{2.409pt}{0.400pt}}
\put(1429.0,408.0){\rule[-0.200pt]{2.409pt}{0.400pt}}
\put(170.0,408.0){\rule[-0.200pt]{2.409pt}{0.400pt}}
\put(1429.0,408.0){\rule[-0.200pt]{2.409pt}{0.400pt}}
\put(170.0,408.0){\rule[-0.200pt]{2.409pt}{0.400pt}}
\put(1429.0,408.0){\rule[-0.200pt]{2.409pt}{0.400pt}}
\put(170.0,408.0){\rule[-0.200pt]{2.409pt}{0.400pt}}
\put(1429.0,408.0){\rule[-0.200pt]{2.409pt}{0.400pt}}
\put(170.0,408.0){\rule[-0.200pt]{2.409pt}{0.400pt}}
\put(1429.0,408.0){\rule[-0.200pt]{2.409pt}{0.400pt}}
\put(170.0,409.0){\rule[-0.200pt]{2.409pt}{0.400pt}}
\put(1429.0,409.0){\rule[-0.200pt]{2.409pt}{0.400pt}}
\put(170.0,409.0){\rule[-0.200pt]{2.409pt}{0.400pt}}
\put(1429.0,409.0){\rule[-0.200pt]{2.409pt}{0.400pt}}
\put(170.0,409.0){\rule[-0.200pt]{2.409pt}{0.400pt}}
\put(1429.0,409.0){\rule[-0.200pt]{2.409pt}{0.400pt}}
\put(170.0,409.0){\rule[-0.200pt]{2.409pt}{0.400pt}}
\put(1429.0,409.0){\rule[-0.200pt]{2.409pt}{0.400pt}}
\put(170.0,409.0){\rule[-0.200pt]{2.409pt}{0.400pt}}
\put(1429.0,409.0){\rule[-0.200pt]{2.409pt}{0.400pt}}
\put(170.0,409.0){\rule[-0.200pt]{2.409pt}{0.400pt}}
\put(1429.0,409.0){\rule[-0.200pt]{2.409pt}{0.400pt}}
\put(170.0,409.0){\rule[-0.200pt]{2.409pt}{0.400pt}}
\put(1429.0,409.0){\rule[-0.200pt]{2.409pt}{0.400pt}}
\put(170.0,409.0){\rule[-0.200pt]{2.409pt}{0.400pt}}
\put(1429.0,409.0){\rule[-0.200pt]{2.409pt}{0.400pt}}
\put(170.0,409.0){\rule[-0.200pt]{2.409pt}{0.400pt}}
\put(1429.0,409.0){\rule[-0.200pt]{2.409pt}{0.400pt}}
\put(170.0,409.0){\rule[-0.200pt]{2.409pt}{0.400pt}}
\put(1429.0,409.0){\rule[-0.200pt]{2.409pt}{0.400pt}}
\put(170.0,409.0){\rule[-0.200pt]{2.409pt}{0.400pt}}
\put(1429.0,409.0){\rule[-0.200pt]{2.409pt}{0.400pt}}
\put(170.0,409.0){\rule[-0.200pt]{2.409pt}{0.400pt}}
\put(1429.0,409.0){\rule[-0.200pt]{2.409pt}{0.400pt}}
\put(170.0,409.0){\rule[-0.200pt]{2.409pt}{0.400pt}}
\put(1429.0,409.0){\rule[-0.200pt]{2.409pt}{0.400pt}}
\put(170.0,409.0){\rule[-0.200pt]{2.409pt}{0.400pt}}
\put(1429.0,409.0){\rule[-0.200pt]{2.409pt}{0.400pt}}
\put(170.0,409.0){\rule[-0.200pt]{2.409pt}{0.400pt}}
\put(1429.0,409.0){\rule[-0.200pt]{2.409pt}{0.400pt}}
\put(170.0,410.0){\rule[-0.200pt]{2.409pt}{0.400pt}}
\put(1429.0,410.0){\rule[-0.200pt]{2.409pt}{0.400pt}}
\put(170.0,410.0){\rule[-0.200pt]{2.409pt}{0.400pt}}
\put(1429.0,410.0){\rule[-0.200pt]{2.409pt}{0.400pt}}
\put(170.0,410.0){\rule[-0.200pt]{2.409pt}{0.400pt}}
\put(1429.0,410.0){\rule[-0.200pt]{2.409pt}{0.400pt}}
\put(170.0,410.0){\rule[-0.200pt]{2.409pt}{0.400pt}}
\put(1429.0,410.0){\rule[-0.200pt]{2.409pt}{0.400pt}}
\put(170.0,410.0){\rule[-0.200pt]{2.409pt}{0.400pt}}
\put(1429.0,410.0){\rule[-0.200pt]{2.409pt}{0.400pt}}
\put(170.0,410.0){\rule[-0.200pt]{2.409pt}{0.400pt}}
\put(1429.0,410.0){\rule[-0.200pt]{2.409pt}{0.400pt}}
\put(170.0,410.0){\rule[-0.200pt]{2.409pt}{0.400pt}}
\put(1429.0,410.0){\rule[-0.200pt]{2.409pt}{0.400pt}}
\put(170.0,410.0){\rule[-0.200pt]{2.409pt}{0.400pt}}
\put(1429.0,410.0){\rule[-0.200pt]{2.409pt}{0.400pt}}
\put(170.0,410.0){\rule[-0.200pt]{2.409pt}{0.400pt}}
\put(1429.0,410.0){\rule[-0.200pt]{2.409pt}{0.400pt}}
\put(170.0,410.0){\rule[-0.200pt]{2.409pt}{0.400pt}}
\put(1429.0,410.0){\rule[-0.200pt]{2.409pt}{0.400pt}}
\put(170.0,410.0){\rule[-0.200pt]{2.409pt}{0.400pt}}
\put(1429.0,410.0){\rule[-0.200pt]{2.409pt}{0.400pt}}
\put(170.0,410.0){\rule[-0.200pt]{2.409pt}{0.400pt}}
\put(1429.0,410.0){\rule[-0.200pt]{2.409pt}{0.400pt}}
\put(170.0,410.0){\rule[-0.200pt]{2.409pt}{0.400pt}}
\put(1429.0,410.0){\rule[-0.200pt]{2.409pt}{0.400pt}}
\put(170.0,410.0){\rule[-0.200pt]{2.409pt}{0.400pt}}
\put(1429.0,410.0){\rule[-0.200pt]{2.409pt}{0.400pt}}
\put(170.0,410.0){\rule[-0.200pt]{2.409pt}{0.400pt}}
\put(1429.0,410.0){\rule[-0.200pt]{2.409pt}{0.400pt}}
\put(170.0,411.0){\rule[-0.200pt]{2.409pt}{0.400pt}}
\put(1429.0,411.0){\rule[-0.200pt]{2.409pt}{0.400pt}}
\put(170.0,411.0){\rule[-0.200pt]{2.409pt}{0.400pt}}
\put(1429.0,411.0){\rule[-0.200pt]{2.409pt}{0.400pt}}
\put(170.0,411.0){\rule[-0.200pt]{2.409pt}{0.400pt}}
\put(1429.0,411.0){\rule[-0.200pt]{2.409pt}{0.400pt}}
\put(170.0,411.0){\rule[-0.200pt]{2.409pt}{0.400pt}}
\put(1429.0,411.0){\rule[-0.200pt]{2.409pt}{0.400pt}}
\put(170.0,411.0){\rule[-0.200pt]{2.409pt}{0.400pt}}
\put(1429.0,411.0){\rule[-0.200pt]{2.409pt}{0.400pt}}
\put(170.0,411.0){\rule[-0.200pt]{2.409pt}{0.400pt}}
\put(1429.0,411.0){\rule[-0.200pt]{2.409pt}{0.400pt}}
\put(170.0,411.0){\rule[-0.200pt]{2.409pt}{0.400pt}}
\put(1429.0,411.0){\rule[-0.200pt]{2.409pt}{0.400pt}}
\put(170.0,411.0){\rule[-0.200pt]{2.409pt}{0.400pt}}
\put(1429.0,411.0){\rule[-0.200pt]{2.409pt}{0.400pt}}
\put(170.0,411.0){\rule[-0.200pt]{2.409pt}{0.400pt}}
\put(1429.0,411.0){\rule[-0.200pt]{2.409pt}{0.400pt}}
\put(170.0,411.0){\rule[-0.200pt]{2.409pt}{0.400pt}}
\put(1429.0,411.0){\rule[-0.200pt]{2.409pt}{0.400pt}}
\put(170.0,411.0){\rule[-0.200pt]{2.409pt}{0.400pt}}
\put(1429.0,411.0){\rule[-0.200pt]{2.409pt}{0.400pt}}
\put(170.0,411.0){\rule[-0.200pt]{2.409pt}{0.400pt}}
\put(1429.0,411.0){\rule[-0.200pt]{2.409pt}{0.400pt}}
\put(170.0,411.0){\rule[-0.200pt]{2.409pt}{0.400pt}}
\put(1429.0,411.0){\rule[-0.200pt]{2.409pt}{0.400pt}}
\put(170.0,411.0){\rule[-0.200pt]{2.409pt}{0.400pt}}
\put(1429.0,411.0){\rule[-0.200pt]{2.409pt}{0.400pt}}
\put(170.0,411.0){\rule[-0.200pt]{2.409pt}{0.400pt}}
\put(1429.0,411.0){\rule[-0.200pt]{2.409pt}{0.400pt}}
\put(170.0,411.0){\rule[-0.200pt]{2.409pt}{0.400pt}}
\put(1429.0,411.0){\rule[-0.200pt]{2.409pt}{0.400pt}}
\put(170.0,412.0){\rule[-0.200pt]{2.409pt}{0.400pt}}
\put(1429.0,412.0){\rule[-0.200pt]{2.409pt}{0.400pt}}
\put(170.0,412.0){\rule[-0.200pt]{2.409pt}{0.400pt}}
\put(1429.0,412.0){\rule[-0.200pt]{2.409pt}{0.400pt}}
\put(170.0,412.0){\rule[-0.200pt]{2.409pt}{0.400pt}}
\put(1429.0,412.0){\rule[-0.200pt]{2.409pt}{0.400pt}}
\put(170.0,412.0){\rule[-0.200pt]{2.409pt}{0.400pt}}
\put(1429.0,412.0){\rule[-0.200pt]{2.409pt}{0.400pt}}
\put(170.0,412.0){\rule[-0.200pt]{2.409pt}{0.400pt}}
\put(1429.0,412.0){\rule[-0.200pt]{2.409pt}{0.400pt}}
\put(170.0,412.0){\rule[-0.200pt]{2.409pt}{0.400pt}}
\put(1429.0,412.0){\rule[-0.200pt]{2.409pt}{0.400pt}}
\put(170.0,412.0){\rule[-0.200pt]{2.409pt}{0.400pt}}
\put(1429.0,412.0){\rule[-0.200pt]{2.409pt}{0.400pt}}
\put(170.0,412.0){\rule[-0.200pt]{2.409pt}{0.400pt}}
\put(1429.0,412.0){\rule[-0.200pt]{2.409pt}{0.400pt}}
\put(170.0,412.0){\rule[-0.200pt]{2.409pt}{0.400pt}}
\put(1429.0,412.0){\rule[-0.200pt]{2.409pt}{0.400pt}}
\put(170.0,412.0){\rule[-0.200pt]{2.409pt}{0.400pt}}
\put(1429.0,412.0){\rule[-0.200pt]{2.409pt}{0.400pt}}
\put(170.0,412.0){\rule[-0.200pt]{2.409pt}{0.400pt}}
\put(1429.0,412.0){\rule[-0.200pt]{2.409pt}{0.400pt}}
\put(170.0,412.0){\rule[-0.200pt]{2.409pt}{0.400pt}}
\put(1429.0,412.0){\rule[-0.200pt]{2.409pt}{0.400pt}}
\put(170.0,412.0){\rule[-0.200pt]{2.409pt}{0.400pt}}
\put(1429.0,412.0){\rule[-0.200pt]{2.409pt}{0.400pt}}
\put(170.0,412.0){\rule[-0.200pt]{2.409pt}{0.400pt}}
\put(1429.0,412.0){\rule[-0.200pt]{2.409pt}{0.400pt}}
\put(170.0,412.0){\rule[-0.200pt]{2.409pt}{0.400pt}}
\put(1429.0,412.0){\rule[-0.200pt]{2.409pt}{0.400pt}}
\put(170.0,412.0){\rule[-0.200pt]{2.409pt}{0.400pt}}
\put(1429.0,412.0){\rule[-0.200pt]{2.409pt}{0.400pt}}
\put(170.0,412.0){\rule[-0.200pt]{2.409pt}{0.400pt}}
\put(1429.0,412.0){\rule[-0.200pt]{2.409pt}{0.400pt}}
\put(170.0,413.0){\rule[-0.200pt]{2.409pt}{0.400pt}}
\put(1429.0,413.0){\rule[-0.200pt]{2.409pt}{0.400pt}}
\put(170.0,413.0){\rule[-0.200pt]{2.409pt}{0.400pt}}
\put(1429.0,413.0){\rule[-0.200pt]{2.409pt}{0.400pt}}
\put(170.0,413.0){\rule[-0.200pt]{2.409pt}{0.400pt}}
\put(1429.0,413.0){\rule[-0.200pt]{2.409pt}{0.400pt}}
\put(170.0,413.0){\rule[-0.200pt]{2.409pt}{0.400pt}}
\put(1429.0,413.0){\rule[-0.200pt]{2.409pt}{0.400pt}}
\put(170.0,413.0){\rule[-0.200pt]{2.409pt}{0.400pt}}
\put(1429.0,413.0){\rule[-0.200pt]{2.409pt}{0.400pt}}
\put(170.0,413.0){\rule[-0.200pt]{2.409pt}{0.400pt}}
\put(1429.0,413.0){\rule[-0.200pt]{2.409pt}{0.400pt}}
\put(170.0,413.0){\rule[-0.200pt]{2.409pt}{0.400pt}}
\put(1429.0,413.0){\rule[-0.200pt]{2.409pt}{0.400pt}}
\put(170.0,413.0){\rule[-0.200pt]{2.409pt}{0.400pt}}
\put(1429.0,413.0){\rule[-0.200pt]{2.409pt}{0.400pt}}
\put(170.0,413.0){\rule[-0.200pt]{2.409pt}{0.400pt}}
\put(1429.0,413.0){\rule[-0.200pt]{2.409pt}{0.400pt}}
\put(170.0,413.0){\rule[-0.200pt]{2.409pt}{0.400pt}}
\put(1429.0,413.0){\rule[-0.200pt]{2.409pt}{0.400pt}}
\put(170.0,413.0){\rule[-0.200pt]{2.409pt}{0.400pt}}
\put(1429.0,413.0){\rule[-0.200pt]{2.409pt}{0.400pt}}
\put(170.0,413.0){\rule[-0.200pt]{2.409pt}{0.400pt}}
\put(1429.0,413.0){\rule[-0.200pt]{2.409pt}{0.400pt}}
\put(170.0,413.0){\rule[-0.200pt]{2.409pt}{0.400pt}}
\put(1429.0,413.0){\rule[-0.200pt]{2.409pt}{0.400pt}}
\put(170.0,413.0){\rule[-0.200pt]{2.409pt}{0.400pt}}
\put(1429.0,413.0){\rule[-0.200pt]{2.409pt}{0.400pt}}
\put(170.0,413.0){\rule[-0.200pt]{2.409pt}{0.400pt}}
\put(1429.0,413.0){\rule[-0.200pt]{2.409pt}{0.400pt}}
\put(170.0,413.0){\rule[-0.200pt]{2.409pt}{0.400pt}}
\put(1429.0,413.0){\rule[-0.200pt]{2.409pt}{0.400pt}}
\put(170.0,413.0){\rule[-0.200pt]{2.409pt}{0.400pt}}
\put(1429.0,413.0){\rule[-0.200pt]{2.409pt}{0.400pt}}
\put(170.0,413.0){\rule[-0.200pt]{2.409pt}{0.400pt}}
\put(1429.0,413.0){\rule[-0.200pt]{2.409pt}{0.400pt}}
\put(170.0,414.0){\rule[-0.200pt]{2.409pt}{0.400pt}}
\put(1429.0,414.0){\rule[-0.200pt]{2.409pt}{0.400pt}}
\put(170.0,414.0){\rule[-0.200pt]{2.409pt}{0.400pt}}
\put(1429.0,414.0){\rule[-0.200pt]{2.409pt}{0.400pt}}
\put(170.0,414.0){\rule[-0.200pt]{2.409pt}{0.400pt}}
\put(1429.0,414.0){\rule[-0.200pt]{2.409pt}{0.400pt}}
\put(170.0,414.0){\rule[-0.200pt]{2.409pt}{0.400pt}}
\put(1429.0,414.0){\rule[-0.200pt]{2.409pt}{0.400pt}}
\put(170.0,414.0){\rule[-0.200pt]{2.409pt}{0.400pt}}
\put(1429.0,414.0){\rule[-0.200pt]{2.409pt}{0.400pt}}
\put(170.0,414.0){\rule[-0.200pt]{2.409pt}{0.400pt}}
\put(1429.0,414.0){\rule[-0.200pt]{2.409pt}{0.400pt}}
\put(170.0,414.0){\rule[-0.200pt]{2.409pt}{0.400pt}}
\put(1429.0,414.0){\rule[-0.200pt]{2.409pt}{0.400pt}}
\put(170.0,414.0){\rule[-0.200pt]{2.409pt}{0.400pt}}
\put(1429.0,414.0){\rule[-0.200pt]{2.409pt}{0.400pt}}
\put(170.0,414.0){\rule[-0.200pt]{2.409pt}{0.400pt}}
\put(1429.0,414.0){\rule[-0.200pt]{2.409pt}{0.400pt}}
\put(170.0,414.0){\rule[-0.200pt]{2.409pt}{0.400pt}}
\put(1429.0,414.0){\rule[-0.200pt]{2.409pt}{0.400pt}}
\put(170.0,414.0){\rule[-0.200pt]{2.409pt}{0.400pt}}
\put(1429.0,414.0){\rule[-0.200pt]{2.409pt}{0.400pt}}
\put(170.0,414.0){\rule[-0.200pt]{2.409pt}{0.400pt}}
\put(1429.0,414.0){\rule[-0.200pt]{2.409pt}{0.400pt}}
\put(170.0,414.0){\rule[-0.200pt]{2.409pt}{0.400pt}}
\put(1429.0,414.0){\rule[-0.200pt]{2.409pt}{0.400pt}}
\put(170.0,414.0){\rule[-0.200pt]{2.409pt}{0.400pt}}
\put(1429.0,414.0){\rule[-0.200pt]{2.409pt}{0.400pt}}
\put(170.0,414.0){\rule[-0.200pt]{2.409pt}{0.400pt}}
\put(1429.0,414.0){\rule[-0.200pt]{2.409pt}{0.400pt}}
\put(170.0,414.0){\rule[-0.200pt]{2.409pt}{0.400pt}}
\put(1429.0,414.0){\rule[-0.200pt]{2.409pt}{0.400pt}}
\put(170.0,414.0){\rule[-0.200pt]{2.409pt}{0.400pt}}
\put(1429.0,414.0){\rule[-0.200pt]{2.409pt}{0.400pt}}
\put(170.0,414.0){\rule[-0.200pt]{2.409pt}{0.400pt}}
\put(1429.0,414.0){\rule[-0.200pt]{2.409pt}{0.400pt}}
\put(170.0,414.0){\rule[-0.200pt]{2.409pt}{0.400pt}}
\put(1429.0,414.0){\rule[-0.200pt]{2.409pt}{0.400pt}}
\put(170.0,414.0){\rule[-0.200pt]{2.409pt}{0.400pt}}
\put(1429.0,414.0){\rule[-0.200pt]{2.409pt}{0.400pt}}
\put(170.0,415.0){\rule[-0.200pt]{2.409pt}{0.400pt}}
\put(1429.0,415.0){\rule[-0.200pt]{2.409pt}{0.400pt}}
\put(170.0,415.0){\rule[-0.200pt]{2.409pt}{0.400pt}}
\put(1429.0,415.0){\rule[-0.200pt]{2.409pt}{0.400pt}}
\put(170.0,415.0){\rule[-0.200pt]{2.409pt}{0.400pt}}
\put(1429.0,415.0){\rule[-0.200pt]{2.409pt}{0.400pt}}
\put(170.0,415.0){\rule[-0.200pt]{2.409pt}{0.400pt}}
\put(1429.0,415.0){\rule[-0.200pt]{2.409pt}{0.400pt}}
\put(170.0,415.0){\rule[-0.200pt]{2.409pt}{0.400pt}}
\put(1429.0,415.0){\rule[-0.200pt]{2.409pt}{0.400pt}}
\put(170.0,415.0){\rule[-0.200pt]{2.409pt}{0.400pt}}
\put(1429.0,415.0){\rule[-0.200pt]{2.409pt}{0.400pt}}
\put(170.0,415.0){\rule[-0.200pt]{2.409pt}{0.400pt}}
\put(1429.0,415.0){\rule[-0.200pt]{2.409pt}{0.400pt}}
\put(170.0,415.0){\rule[-0.200pt]{2.409pt}{0.400pt}}
\put(1429.0,415.0){\rule[-0.200pt]{2.409pt}{0.400pt}}
\put(170.0,415.0){\rule[-0.200pt]{2.409pt}{0.400pt}}
\put(1429.0,415.0){\rule[-0.200pt]{2.409pt}{0.400pt}}
\put(170.0,415.0){\rule[-0.200pt]{2.409pt}{0.400pt}}
\put(1429.0,415.0){\rule[-0.200pt]{2.409pt}{0.400pt}}
\put(170.0,415.0){\rule[-0.200pt]{4.818pt}{0.400pt}}
\put(150,415){\makebox(0,0)[r]{ 1e-09}}
\put(1419.0,415.0){\rule[-0.200pt]{4.818pt}{0.400pt}}
\put(170.0,426.0){\rule[-0.200pt]{2.409pt}{0.400pt}}
\put(1429.0,426.0){\rule[-0.200pt]{2.409pt}{0.400pt}}
\put(170.0,441.0){\rule[-0.200pt]{2.409pt}{0.400pt}}
\put(1429.0,441.0){\rule[-0.200pt]{2.409pt}{0.400pt}}
\put(170.0,448.0){\rule[-0.200pt]{2.409pt}{0.400pt}}
\put(1429.0,448.0){\rule[-0.200pt]{2.409pt}{0.400pt}}
\put(170.0,454.0){\rule[-0.200pt]{2.409pt}{0.400pt}}
\put(1429.0,454.0){\rule[-0.200pt]{2.409pt}{0.400pt}}
\put(170.0,457.0){\rule[-0.200pt]{2.409pt}{0.400pt}}
\put(1429.0,457.0){\rule[-0.200pt]{2.409pt}{0.400pt}}
\put(170.0,461.0){\rule[-0.200pt]{2.409pt}{0.400pt}}
\put(1429.0,461.0){\rule[-0.200pt]{2.409pt}{0.400pt}}
\put(170.0,463.0){\rule[-0.200pt]{2.409pt}{0.400pt}}
\put(1429.0,463.0){\rule[-0.200pt]{2.409pt}{0.400pt}}
\put(170.0,465.0){\rule[-0.200pt]{2.409pt}{0.400pt}}
\put(1429.0,465.0){\rule[-0.200pt]{2.409pt}{0.400pt}}
\put(170.0,467.0){\rule[-0.200pt]{2.409pt}{0.400pt}}
\put(1429.0,467.0){\rule[-0.200pt]{2.409pt}{0.400pt}}
\put(170.0,469.0){\rule[-0.200pt]{2.409pt}{0.400pt}}
\put(1429.0,469.0){\rule[-0.200pt]{2.409pt}{0.400pt}}
\put(170.0,471.0){\rule[-0.200pt]{2.409pt}{0.400pt}}
\put(1429.0,471.0){\rule[-0.200pt]{2.409pt}{0.400pt}}
\put(170.0,472.0){\rule[-0.200pt]{2.409pt}{0.400pt}}
\put(1429.0,472.0){\rule[-0.200pt]{2.409pt}{0.400pt}}
\put(170.0,473.0){\rule[-0.200pt]{2.409pt}{0.400pt}}
\put(1429.0,473.0){\rule[-0.200pt]{2.409pt}{0.400pt}}
\put(170.0,475.0){\rule[-0.200pt]{2.409pt}{0.400pt}}
\put(1429.0,475.0){\rule[-0.200pt]{2.409pt}{0.400pt}}
\put(170.0,476.0){\rule[-0.200pt]{2.409pt}{0.400pt}}
\put(1429.0,476.0){\rule[-0.200pt]{2.409pt}{0.400pt}}
\put(170.0,477.0){\rule[-0.200pt]{2.409pt}{0.400pt}}
\put(1429.0,477.0){\rule[-0.200pt]{2.409pt}{0.400pt}}
\put(170.0,478.0){\rule[-0.200pt]{2.409pt}{0.400pt}}
\put(1429.0,478.0){\rule[-0.200pt]{2.409pt}{0.400pt}}
\put(170.0,479.0){\rule[-0.200pt]{2.409pt}{0.400pt}}
\put(1429.0,479.0){\rule[-0.200pt]{2.409pt}{0.400pt}}
\put(170.0,480.0){\rule[-0.200pt]{2.409pt}{0.400pt}}
\put(1429.0,480.0){\rule[-0.200pt]{2.409pt}{0.400pt}}
\put(170.0,481.0){\rule[-0.200pt]{2.409pt}{0.400pt}}
\put(1429.0,481.0){\rule[-0.200pt]{2.409pt}{0.400pt}}
\put(170.0,481.0){\rule[-0.200pt]{2.409pt}{0.400pt}}
\put(1429.0,481.0){\rule[-0.200pt]{2.409pt}{0.400pt}}
\put(170.0,482.0){\rule[-0.200pt]{2.409pt}{0.400pt}}
\put(1429.0,482.0){\rule[-0.200pt]{2.409pt}{0.400pt}}
\put(170.0,483.0){\rule[-0.200pt]{2.409pt}{0.400pt}}
\put(1429.0,483.0){\rule[-0.200pt]{2.409pt}{0.400pt}}
\put(170.0,483.0){\rule[-0.200pt]{2.409pt}{0.400pt}}
\put(1429.0,483.0){\rule[-0.200pt]{2.409pt}{0.400pt}}
\put(170.0,484.0){\rule[-0.200pt]{2.409pt}{0.400pt}}
\put(1429.0,484.0){\rule[-0.200pt]{2.409pt}{0.400pt}}
\put(170.0,485.0){\rule[-0.200pt]{2.409pt}{0.400pt}}
\put(1429.0,485.0){\rule[-0.200pt]{2.409pt}{0.400pt}}
\put(170.0,485.0){\rule[-0.200pt]{2.409pt}{0.400pt}}
\put(1429.0,485.0){\rule[-0.200pt]{2.409pt}{0.400pt}}
\put(170.0,486.0){\rule[-0.200pt]{2.409pt}{0.400pt}}
\put(1429.0,486.0){\rule[-0.200pt]{2.409pt}{0.400pt}}
\put(170.0,487.0){\rule[-0.200pt]{2.409pt}{0.400pt}}
\put(1429.0,487.0){\rule[-0.200pt]{2.409pt}{0.400pt}}
\put(170.0,487.0){\rule[-0.200pt]{2.409pt}{0.400pt}}
\put(1429.0,487.0){\rule[-0.200pt]{2.409pt}{0.400pt}}
\put(170.0,488.0){\rule[-0.200pt]{2.409pt}{0.400pt}}
\put(1429.0,488.0){\rule[-0.200pt]{2.409pt}{0.400pt}}
\put(170.0,488.0){\rule[-0.200pt]{2.409pt}{0.400pt}}
\put(1429.0,488.0){\rule[-0.200pt]{2.409pt}{0.400pt}}
\put(170.0,489.0){\rule[-0.200pt]{2.409pt}{0.400pt}}
\put(1429.0,489.0){\rule[-0.200pt]{2.409pt}{0.400pt}}
\put(170.0,489.0){\rule[-0.200pt]{2.409pt}{0.400pt}}
\put(1429.0,489.0){\rule[-0.200pt]{2.409pt}{0.400pt}}
\put(170.0,490.0){\rule[-0.200pt]{2.409pt}{0.400pt}}
\put(1429.0,490.0){\rule[-0.200pt]{2.409pt}{0.400pt}}
\put(170.0,490.0){\rule[-0.200pt]{2.409pt}{0.400pt}}
\put(1429.0,490.0){\rule[-0.200pt]{2.409pt}{0.400pt}}
\put(170.0,491.0){\rule[-0.200pt]{2.409pt}{0.400pt}}
\put(1429.0,491.0){\rule[-0.200pt]{2.409pt}{0.400pt}}
\put(170.0,491.0){\rule[-0.200pt]{2.409pt}{0.400pt}}
\put(1429.0,491.0){\rule[-0.200pt]{2.409pt}{0.400pt}}
\put(170.0,491.0){\rule[-0.200pt]{2.409pt}{0.400pt}}
\put(1429.0,491.0){\rule[-0.200pt]{2.409pt}{0.400pt}}
\put(170.0,492.0){\rule[-0.200pt]{2.409pt}{0.400pt}}
\put(1429.0,492.0){\rule[-0.200pt]{2.409pt}{0.400pt}}
\put(170.0,492.0){\rule[-0.200pt]{2.409pt}{0.400pt}}
\put(1429.0,492.0){\rule[-0.200pt]{2.409pt}{0.400pt}}
\put(170.0,493.0){\rule[-0.200pt]{2.409pt}{0.400pt}}
\put(1429.0,493.0){\rule[-0.200pt]{2.409pt}{0.400pt}}
\put(170.0,493.0){\rule[-0.200pt]{2.409pt}{0.400pt}}
\put(1429.0,493.0){\rule[-0.200pt]{2.409pt}{0.400pt}}
\put(170.0,493.0){\rule[-0.200pt]{2.409pt}{0.400pt}}
\put(1429.0,493.0){\rule[-0.200pt]{2.409pt}{0.400pt}}
\put(170.0,494.0){\rule[-0.200pt]{2.409pt}{0.400pt}}
\put(1429.0,494.0){\rule[-0.200pt]{2.409pt}{0.400pt}}
\put(170.0,494.0){\rule[-0.200pt]{2.409pt}{0.400pt}}
\put(1429.0,494.0){\rule[-0.200pt]{2.409pt}{0.400pt}}
\put(170.0,494.0){\rule[-0.200pt]{2.409pt}{0.400pt}}
\put(1429.0,494.0){\rule[-0.200pt]{2.409pt}{0.400pt}}
\put(170.0,495.0){\rule[-0.200pt]{2.409pt}{0.400pt}}
\put(1429.0,495.0){\rule[-0.200pt]{2.409pt}{0.400pt}}
\put(170.0,495.0){\rule[-0.200pt]{2.409pt}{0.400pt}}
\put(1429.0,495.0){\rule[-0.200pt]{2.409pt}{0.400pt}}
\put(170.0,495.0){\rule[-0.200pt]{2.409pt}{0.400pt}}
\put(1429.0,495.0){\rule[-0.200pt]{2.409pt}{0.400pt}}
\put(170.0,496.0){\rule[-0.200pt]{2.409pt}{0.400pt}}
\put(1429.0,496.0){\rule[-0.200pt]{2.409pt}{0.400pt}}
\put(170.0,496.0){\rule[-0.200pt]{2.409pt}{0.400pt}}
\put(1429.0,496.0){\rule[-0.200pt]{2.409pt}{0.400pt}}
\put(170.0,496.0){\rule[-0.200pt]{2.409pt}{0.400pt}}
\put(1429.0,496.0){\rule[-0.200pt]{2.409pt}{0.400pt}}
\put(170.0,497.0){\rule[-0.200pt]{2.409pt}{0.400pt}}
\put(1429.0,497.0){\rule[-0.200pt]{2.409pt}{0.400pt}}
\put(170.0,497.0){\rule[-0.200pt]{2.409pt}{0.400pt}}
\put(1429.0,497.0){\rule[-0.200pt]{2.409pt}{0.400pt}}
\put(170.0,497.0){\rule[-0.200pt]{2.409pt}{0.400pt}}
\put(1429.0,497.0){\rule[-0.200pt]{2.409pt}{0.400pt}}
\put(170.0,498.0){\rule[-0.200pt]{2.409pt}{0.400pt}}
\put(1429.0,498.0){\rule[-0.200pt]{2.409pt}{0.400pt}}
\put(170.0,498.0){\rule[-0.200pt]{2.409pt}{0.400pt}}
\put(1429.0,498.0){\rule[-0.200pt]{2.409pt}{0.400pt}}
\put(170.0,498.0){\rule[-0.200pt]{2.409pt}{0.400pt}}
\put(1429.0,498.0){\rule[-0.200pt]{2.409pt}{0.400pt}}
\put(170.0,498.0){\rule[-0.200pt]{2.409pt}{0.400pt}}
\put(1429.0,498.0){\rule[-0.200pt]{2.409pt}{0.400pt}}
\put(170.0,499.0){\rule[-0.200pt]{2.409pt}{0.400pt}}
\put(1429.0,499.0){\rule[-0.200pt]{2.409pt}{0.400pt}}
\put(170.0,499.0){\rule[-0.200pt]{2.409pt}{0.400pt}}
\put(1429.0,499.0){\rule[-0.200pt]{2.409pt}{0.400pt}}
\put(170.0,499.0){\rule[-0.200pt]{2.409pt}{0.400pt}}
\put(1429.0,499.0){\rule[-0.200pt]{2.409pt}{0.400pt}}
\put(170.0,499.0){\rule[-0.200pt]{2.409pt}{0.400pt}}
\put(1429.0,499.0){\rule[-0.200pt]{2.409pt}{0.400pt}}
\put(170.0,500.0){\rule[-0.200pt]{2.409pt}{0.400pt}}
\put(1429.0,500.0){\rule[-0.200pt]{2.409pt}{0.400pt}}
\put(170.0,500.0){\rule[-0.200pt]{2.409pt}{0.400pt}}
\put(1429.0,500.0){\rule[-0.200pt]{2.409pt}{0.400pt}}
\put(170.0,500.0){\rule[-0.200pt]{2.409pt}{0.400pt}}
\put(1429.0,500.0){\rule[-0.200pt]{2.409pt}{0.400pt}}
\put(170.0,500.0){\rule[-0.200pt]{2.409pt}{0.400pt}}
\put(1429.0,500.0){\rule[-0.200pt]{2.409pt}{0.400pt}}
\put(170.0,501.0){\rule[-0.200pt]{2.409pt}{0.400pt}}
\put(1429.0,501.0){\rule[-0.200pt]{2.409pt}{0.400pt}}
\put(170.0,501.0){\rule[-0.200pt]{2.409pt}{0.400pt}}
\put(1429.0,501.0){\rule[-0.200pt]{2.409pt}{0.400pt}}
\put(170.0,501.0){\rule[-0.200pt]{2.409pt}{0.400pt}}
\put(1429.0,501.0){\rule[-0.200pt]{2.409pt}{0.400pt}}
\put(170.0,501.0){\rule[-0.200pt]{2.409pt}{0.400pt}}
\put(1429.0,501.0){\rule[-0.200pt]{2.409pt}{0.400pt}}
\put(170.0,502.0){\rule[-0.200pt]{2.409pt}{0.400pt}}
\put(1429.0,502.0){\rule[-0.200pt]{2.409pt}{0.400pt}}
\put(170.0,502.0){\rule[-0.200pt]{2.409pt}{0.400pt}}
\put(1429.0,502.0){\rule[-0.200pt]{2.409pt}{0.400pt}}
\put(170.0,502.0){\rule[-0.200pt]{2.409pt}{0.400pt}}
\put(1429.0,502.0){\rule[-0.200pt]{2.409pt}{0.400pt}}
\put(170.0,502.0){\rule[-0.200pt]{2.409pt}{0.400pt}}
\put(1429.0,502.0){\rule[-0.200pt]{2.409pt}{0.400pt}}
\put(170.0,502.0){\rule[-0.200pt]{2.409pt}{0.400pt}}
\put(1429.0,502.0){\rule[-0.200pt]{2.409pt}{0.400pt}}
\put(170.0,503.0){\rule[-0.200pt]{2.409pt}{0.400pt}}
\put(1429.0,503.0){\rule[-0.200pt]{2.409pt}{0.400pt}}
\put(170.0,503.0){\rule[-0.200pt]{2.409pt}{0.400pt}}
\put(1429.0,503.0){\rule[-0.200pt]{2.409pt}{0.400pt}}
\put(170.0,503.0){\rule[-0.200pt]{2.409pt}{0.400pt}}
\put(1429.0,503.0){\rule[-0.200pt]{2.409pt}{0.400pt}}
\put(170.0,503.0){\rule[-0.200pt]{2.409pt}{0.400pt}}
\put(1429.0,503.0){\rule[-0.200pt]{2.409pt}{0.400pt}}
\put(170.0,503.0){\rule[-0.200pt]{2.409pt}{0.400pt}}
\put(1429.0,503.0){\rule[-0.200pt]{2.409pt}{0.400pt}}
\put(170.0,504.0){\rule[-0.200pt]{2.409pt}{0.400pt}}
\put(1429.0,504.0){\rule[-0.200pt]{2.409pt}{0.400pt}}
\put(170.0,504.0){\rule[-0.200pt]{2.409pt}{0.400pt}}
\put(1429.0,504.0){\rule[-0.200pt]{2.409pt}{0.400pt}}
\put(170.0,504.0){\rule[-0.200pt]{2.409pt}{0.400pt}}
\put(1429.0,504.0){\rule[-0.200pt]{2.409pt}{0.400pt}}
\put(170.0,504.0){\rule[-0.200pt]{2.409pt}{0.400pt}}
\put(1429.0,504.0){\rule[-0.200pt]{2.409pt}{0.400pt}}
\put(170.0,504.0){\rule[-0.200pt]{2.409pt}{0.400pt}}
\put(1429.0,504.0){\rule[-0.200pt]{2.409pt}{0.400pt}}
\put(170.0,505.0){\rule[-0.200pt]{2.409pt}{0.400pt}}
\put(1429.0,505.0){\rule[-0.200pt]{2.409pt}{0.400pt}}
\put(170.0,505.0){\rule[-0.200pt]{2.409pt}{0.400pt}}
\put(1429.0,505.0){\rule[-0.200pt]{2.409pt}{0.400pt}}
\put(170.0,505.0){\rule[-0.200pt]{2.409pt}{0.400pt}}
\put(1429.0,505.0){\rule[-0.200pt]{2.409pt}{0.400pt}}
\put(170.0,505.0){\rule[-0.200pt]{2.409pt}{0.400pt}}
\put(1429.0,505.0){\rule[-0.200pt]{2.409pt}{0.400pt}}
\put(170.0,505.0){\rule[-0.200pt]{2.409pt}{0.400pt}}
\put(1429.0,505.0){\rule[-0.200pt]{2.409pt}{0.400pt}}
\put(170.0,505.0){\rule[-0.200pt]{2.409pt}{0.400pt}}
\put(1429.0,505.0){\rule[-0.200pt]{2.409pt}{0.400pt}}
\put(170.0,506.0){\rule[-0.200pt]{2.409pt}{0.400pt}}
\put(1429.0,506.0){\rule[-0.200pt]{2.409pt}{0.400pt}}
\put(170.0,506.0){\rule[-0.200pt]{2.409pt}{0.400pt}}
\put(1429.0,506.0){\rule[-0.200pt]{2.409pt}{0.400pt}}
\put(170.0,506.0){\rule[-0.200pt]{2.409pt}{0.400pt}}
\put(1429.0,506.0){\rule[-0.200pt]{2.409pt}{0.400pt}}
\put(170.0,506.0){\rule[-0.200pt]{2.409pt}{0.400pt}}
\put(1429.0,506.0){\rule[-0.200pt]{2.409pt}{0.400pt}}
\put(170.0,506.0){\rule[-0.200pt]{2.409pt}{0.400pt}}
\put(1429.0,506.0){\rule[-0.200pt]{2.409pt}{0.400pt}}
\put(170.0,506.0){\rule[-0.200pt]{2.409pt}{0.400pt}}
\put(1429.0,506.0){\rule[-0.200pt]{2.409pt}{0.400pt}}
\put(170.0,507.0){\rule[-0.200pt]{2.409pt}{0.400pt}}
\put(1429.0,507.0){\rule[-0.200pt]{2.409pt}{0.400pt}}
\put(170.0,507.0){\rule[-0.200pt]{2.409pt}{0.400pt}}
\put(1429.0,507.0){\rule[-0.200pt]{2.409pt}{0.400pt}}
\put(170.0,507.0){\rule[-0.200pt]{2.409pt}{0.400pt}}
\put(1429.0,507.0){\rule[-0.200pt]{2.409pt}{0.400pt}}
\put(170.0,507.0){\rule[-0.200pt]{2.409pt}{0.400pt}}
\put(1429.0,507.0){\rule[-0.200pt]{2.409pt}{0.400pt}}
\put(170.0,507.0){\rule[-0.200pt]{2.409pt}{0.400pt}}
\put(1429.0,507.0){\rule[-0.200pt]{2.409pt}{0.400pt}}
\put(170.0,507.0){\rule[-0.200pt]{2.409pt}{0.400pt}}
\put(1429.0,507.0){\rule[-0.200pt]{2.409pt}{0.400pt}}
\put(170.0,508.0){\rule[-0.200pt]{2.409pt}{0.400pt}}
\put(1429.0,508.0){\rule[-0.200pt]{2.409pt}{0.400pt}}
\put(170.0,508.0){\rule[-0.200pt]{2.409pt}{0.400pt}}
\put(1429.0,508.0){\rule[-0.200pt]{2.409pt}{0.400pt}}
\put(170.0,508.0){\rule[-0.200pt]{2.409pt}{0.400pt}}
\put(1429.0,508.0){\rule[-0.200pt]{2.409pt}{0.400pt}}
\put(170.0,508.0){\rule[-0.200pt]{2.409pt}{0.400pt}}
\put(1429.0,508.0){\rule[-0.200pt]{2.409pt}{0.400pt}}
\put(170.0,508.0){\rule[-0.200pt]{2.409pt}{0.400pt}}
\put(1429.0,508.0){\rule[-0.200pt]{2.409pt}{0.400pt}}
\put(170.0,508.0){\rule[-0.200pt]{2.409pt}{0.400pt}}
\put(1429.0,508.0){\rule[-0.200pt]{2.409pt}{0.400pt}}
\put(170.0,508.0){\rule[-0.200pt]{2.409pt}{0.400pt}}
\put(1429.0,508.0){\rule[-0.200pt]{2.409pt}{0.400pt}}
\put(170.0,509.0){\rule[-0.200pt]{2.409pt}{0.400pt}}
\put(1429.0,509.0){\rule[-0.200pt]{2.409pt}{0.400pt}}
\put(170.0,509.0){\rule[-0.200pt]{2.409pt}{0.400pt}}
\put(1429.0,509.0){\rule[-0.200pt]{2.409pt}{0.400pt}}
\put(170.0,509.0){\rule[-0.200pt]{2.409pt}{0.400pt}}
\put(1429.0,509.0){\rule[-0.200pt]{2.409pt}{0.400pt}}
\put(170.0,509.0){\rule[-0.200pt]{2.409pt}{0.400pt}}
\put(1429.0,509.0){\rule[-0.200pt]{2.409pt}{0.400pt}}
\put(170.0,509.0){\rule[-0.200pt]{2.409pt}{0.400pt}}
\put(1429.0,509.0){\rule[-0.200pt]{2.409pt}{0.400pt}}
\put(170.0,509.0){\rule[-0.200pt]{2.409pt}{0.400pt}}
\put(1429.0,509.0){\rule[-0.200pt]{2.409pt}{0.400pt}}
\put(170.0,509.0){\rule[-0.200pt]{2.409pt}{0.400pt}}
\put(1429.0,509.0){\rule[-0.200pt]{2.409pt}{0.400pt}}
\put(170.0,510.0){\rule[-0.200pt]{2.409pt}{0.400pt}}
\put(1429.0,510.0){\rule[-0.200pt]{2.409pt}{0.400pt}}
\put(170.0,510.0){\rule[-0.200pt]{2.409pt}{0.400pt}}
\put(1429.0,510.0){\rule[-0.200pt]{2.409pt}{0.400pt}}
\put(170.0,510.0){\rule[-0.200pt]{2.409pt}{0.400pt}}
\put(1429.0,510.0){\rule[-0.200pt]{2.409pt}{0.400pt}}
\put(170.0,510.0){\rule[-0.200pt]{2.409pt}{0.400pt}}
\put(1429.0,510.0){\rule[-0.200pt]{2.409pt}{0.400pt}}
\put(170.0,510.0){\rule[-0.200pt]{2.409pt}{0.400pt}}
\put(1429.0,510.0){\rule[-0.200pt]{2.409pt}{0.400pt}}
\put(170.0,510.0){\rule[-0.200pt]{2.409pt}{0.400pt}}
\put(1429.0,510.0){\rule[-0.200pt]{2.409pt}{0.400pt}}
\put(170.0,510.0){\rule[-0.200pt]{2.409pt}{0.400pt}}
\put(1429.0,510.0){\rule[-0.200pt]{2.409pt}{0.400pt}}
\put(170.0,510.0){\rule[-0.200pt]{2.409pt}{0.400pt}}
\put(1429.0,510.0){\rule[-0.200pt]{2.409pt}{0.400pt}}
\put(170.0,511.0){\rule[-0.200pt]{2.409pt}{0.400pt}}
\put(1429.0,511.0){\rule[-0.200pt]{2.409pt}{0.400pt}}
\put(170.0,511.0){\rule[-0.200pt]{2.409pt}{0.400pt}}
\put(1429.0,511.0){\rule[-0.200pt]{2.409pt}{0.400pt}}
\put(170.0,511.0){\rule[-0.200pt]{2.409pt}{0.400pt}}
\put(1429.0,511.0){\rule[-0.200pt]{2.409pt}{0.400pt}}
\put(170.0,511.0){\rule[-0.200pt]{2.409pt}{0.400pt}}
\put(1429.0,511.0){\rule[-0.200pt]{2.409pt}{0.400pt}}
\put(170.0,511.0){\rule[-0.200pt]{2.409pt}{0.400pt}}
\put(1429.0,511.0){\rule[-0.200pt]{2.409pt}{0.400pt}}
\put(170.0,511.0){\rule[-0.200pt]{2.409pt}{0.400pt}}
\put(1429.0,511.0){\rule[-0.200pt]{2.409pt}{0.400pt}}
\put(170.0,511.0){\rule[-0.200pt]{2.409pt}{0.400pt}}
\put(1429.0,511.0){\rule[-0.200pt]{2.409pt}{0.400pt}}
\put(170.0,511.0){\rule[-0.200pt]{2.409pt}{0.400pt}}
\put(1429.0,511.0){\rule[-0.200pt]{2.409pt}{0.400pt}}
\put(170.0,512.0){\rule[-0.200pt]{2.409pt}{0.400pt}}
\put(1429.0,512.0){\rule[-0.200pt]{2.409pt}{0.400pt}}
\put(170.0,512.0){\rule[-0.200pt]{2.409pt}{0.400pt}}
\put(1429.0,512.0){\rule[-0.200pt]{2.409pt}{0.400pt}}
\put(170.0,512.0){\rule[-0.200pt]{2.409pt}{0.400pt}}
\put(1429.0,512.0){\rule[-0.200pt]{2.409pt}{0.400pt}}
\put(170.0,512.0){\rule[-0.200pt]{2.409pt}{0.400pt}}
\put(1429.0,512.0){\rule[-0.200pt]{2.409pt}{0.400pt}}
\put(170.0,512.0){\rule[-0.200pt]{2.409pt}{0.400pt}}
\put(1429.0,512.0){\rule[-0.200pt]{2.409pt}{0.400pt}}
\put(170.0,512.0){\rule[-0.200pt]{2.409pt}{0.400pt}}
\put(1429.0,512.0){\rule[-0.200pt]{2.409pt}{0.400pt}}
\put(170.0,512.0){\rule[-0.200pt]{2.409pt}{0.400pt}}
\put(1429.0,512.0){\rule[-0.200pt]{2.409pt}{0.400pt}}
\put(170.0,512.0){\rule[-0.200pt]{2.409pt}{0.400pt}}
\put(1429.0,512.0){\rule[-0.200pt]{2.409pt}{0.400pt}}
\put(170.0,512.0){\rule[-0.200pt]{2.409pt}{0.400pt}}
\put(1429.0,512.0){\rule[-0.200pt]{2.409pt}{0.400pt}}
\put(170.0,513.0){\rule[-0.200pt]{2.409pt}{0.400pt}}
\put(1429.0,513.0){\rule[-0.200pt]{2.409pt}{0.400pt}}
\put(170.0,513.0){\rule[-0.200pt]{2.409pt}{0.400pt}}
\put(1429.0,513.0){\rule[-0.200pt]{2.409pt}{0.400pt}}
\put(170.0,513.0){\rule[-0.200pt]{2.409pt}{0.400pt}}
\put(1429.0,513.0){\rule[-0.200pt]{2.409pt}{0.400pt}}
\put(170.0,513.0){\rule[-0.200pt]{2.409pt}{0.400pt}}
\put(1429.0,513.0){\rule[-0.200pt]{2.409pt}{0.400pt}}
\put(170.0,513.0){\rule[-0.200pt]{2.409pt}{0.400pt}}
\put(1429.0,513.0){\rule[-0.200pt]{2.409pt}{0.400pt}}
\put(170.0,513.0){\rule[-0.200pt]{2.409pt}{0.400pt}}
\put(1429.0,513.0){\rule[-0.200pt]{2.409pt}{0.400pt}}
\put(170.0,513.0){\rule[-0.200pt]{2.409pt}{0.400pt}}
\put(1429.0,513.0){\rule[-0.200pt]{2.409pt}{0.400pt}}
\put(170.0,513.0){\rule[-0.200pt]{2.409pt}{0.400pt}}
\put(1429.0,513.0){\rule[-0.200pt]{2.409pt}{0.400pt}}
\put(170.0,513.0){\rule[-0.200pt]{2.409pt}{0.400pt}}
\put(1429.0,513.0){\rule[-0.200pt]{2.409pt}{0.400pt}}
\put(170.0,514.0){\rule[-0.200pt]{2.409pt}{0.400pt}}
\put(1429.0,514.0){\rule[-0.200pt]{2.409pt}{0.400pt}}
\put(170.0,514.0){\rule[-0.200pt]{2.409pt}{0.400pt}}
\put(1429.0,514.0){\rule[-0.200pt]{2.409pt}{0.400pt}}
\put(170.0,514.0){\rule[-0.200pt]{2.409pt}{0.400pt}}
\put(1429.0,514.0){\rule[-0.200pt]{2.409pt}{0.400pt}}
\put(170.0,514.0){\rule[-0.200pt]{2.409pt}{0.400pt}}
\put(1429.0,514.0){\rule[-0.200pt]{2.409pt}{0.400pt}}
\put(170.0,514.0){\rule[-0.200pt]{2.409pt}{0.400pt}}
\put(1429.0,514.0){\rule[-0.200pt]{2.409pt}{0.400pt}}
\put(170.0,514.0){\rule[-0.200pt]{2.409pt}{0.400pt}}
\put(1429.0,514.0){\rule[-0.200pt]{2.409pt}{0.400pt}}
\put(170.0,514.0){\rule[-0.200pt]{2.409pt}{0.400pt}}
\put(1429.0,514.0){\rule[-0.200pt]{2.409pt}{0.400pt}}
\put(170.0,514.0){\rule[-0.200pt]{2.409pt}{0.400pt}}
\put(1429.0,514.0){\rule[-0.200pt]{2.409pt}{0.400pt}}
\put(170.0,514.0){\rule[-0.200pt]{2.409pt}{0.400pt}}
\put(1429.0,514.0){\rule[-0.200pt]{2.409pt}{0.400pt}}
\put(170.0,514.0){\rule[-0.200pt]{2.409pt}{0.400pt}}
\put(1429.0,514.0){\rule[-0.200pt]{2.409pt}{0.400pt}}
\put(170.0,515.0){\rule[-0.200pt]{2.409pt}{0.400pt}}
\put(1429.0,515.0){\rule[-0.200pt]{2.409pt}{0.400pt}}
\put(170.0,515.0){\rule[-0.200pt]{2.409pt}{0.400pt}}
\put(1429.0,515.0){\rule[-0.200pt]{2.409pt}{0.400pt}}
\put(170.0,515.0){\rule[-0.200pt]{2.409pt}{0.400pt}}
\put(1429.0,515.0){\rule[-0.200pt]{2.409pt}{0.400pt}}
\put(170.0,515.0){\rule[-0.200pt]{2.409pt}{0.400pt}}
\put(1429.0,515.0){\rule[-0.200pt]{2.409pt}{0.400pt}}
\put(170.0,515.0){\rule[-0.200pt]{2.409pt}{0.400pt}}
\put(1429.0,515.0){\rule[-0.200pt]{2.409pt}{0.400pt}}
\put(170.0,515.0){\rule[-0.200pt]{2.409pt}{0.400pt}}
\put(1429.0,515.0){\rule[-0.200pt]{2.409pt}{0.400pt}}
\put(170.0,515.0){\rule[-0.200pt]{2.409pt}{0.400pt}}
\put(1429.0,515.0){\rule[-0.200pt]{2.409pt}{0.400pt}}
\put(170.0,515.0){\rule[-0.200pt]{2.409pt}{0.400pt}}
\put(1429.0,515.0){\rule[-0.200pt]{2.409pt}{0.400pt}}
\put(170.0,515.0){\rule[-0.200pt]{2.409pt}{0.400pt}}
\put(1429.0,515.0){\rule[-0.200pt]{2.409pt}{0.400pt}}
\put(170.0,515.0){\rule[-0.200pt]{2.409pt}{0.400pt}}
\put(1429.0,515.0){\rule[-0.200pt]{2.409pt}{0.400pt}}
\put(170.0,516.0){\rule[-0.200pt]{2.409pt}{0.400pt}}
\put(1429.0,516.0){\rule[-0.200pt]{2.409pt}{0.400pt}}
\put(170.0,516.0){\rule[-0.200pt]{2.409pt}{0.400pt}}
\put(1429.0,516.0){\rule[-0.200pt]{2.409pt}{0.400pt}}
\put(170.0,516.0){\rule[-0.200pt]{2.409pt}{0.400pt}}
\put(1429.0,516.0){\rule[-0.200pt]{2.409pt}{0.400pt}}
\put(170.0,516.0){\rule[-0.200pt]{2.409pt}{0.400pt}}
\put(1429.0,516.0){\rule[-0.200pt]{2.409pt}{0.400pt}}
\put(170.0,516.0){\rule[-0.200pt]{2.409pt}{0.400pt}}
\put(1429.0,516.0){\rule[-0.200pt]{2.409pt}{0.400pt}}
\put(170.0,516.0){\rule[-0.200pt]{2.409pt}{0.400pt}}
\put(1429.0,516.0){\rule[-0.200pt]{2.409pt}{0.400pt}}
\put(170.0,516.0){\rule[-0.200pt]{2.409pt}{0.400pt}}
\put(1429.0,516.0){\rule[-0.200pt]{2.409pt}{0.400pt}}
\put(170.0,516.0){\rule[-0.200pt]{2.409pt}{0.400pt}}
\put(1429.0,516.0){\rule[-0.200pt]{2.409pt}{0.400pt}}
\put(170.0,516.0){\rule[-0.200pt]{2.409pt}{0.400pt}}
\put(1429.0,516.0){\rule[-0.200pt]{2.409pt}{0.400pt}}
\put(170.0,516.0){\rule[-0.200pt]{2.409pt}{0.400pt}}
\put(1429.0,516.0){\rule[-0.200pt]{2.409pt}{0.400pt}}
\put(170.0,516.0){\rule[-0.200pt]{2.409pt}{0.400pt}}
\put(1429.0,516.0){\rule[-0.200pt]{2.409pt}{0.400pt}}
\put(170.0,517.0){\rule[-0.200pt]{2.409pt}{0.400pt}}
\put(1429.0,517.0){\rule[-0.200pt]{2.409pt}{0.400pt}}
\put(170.0,517.0){\rule[-0.200pt]{2.409pt}{0.400pt}}
\put(1429.0,517.0){\rule[-0.200pt]{2.409pt}{0.400pt}}
\put(170.0,517.0){\rule[-0.200pt]{2.409pt}{0.400pt}}
\put(1429.0,517.0){\rule[-0.200pt]{2.409pt}{0.400pt}}
\put(170.0,517.0){\rule[-0.200pt]{2.409pt}{0.400pt}}
\put(1429.0,517.0){\rule[-0.200pt]{2.409pt}{0.400pt}}
\put(170.0,517.0){\rule[-0.200pt]{2.409pt}{0.400pt}}
\put(1429.0,517.0){\rule[-0.200pt]{2.409pt}{0.400pt}}
\put(170.0,517.0){\rule[-0.200pt]{2.409pt}{0.400pt}}
\put(1429.0,517.0){\rule[-0.200pt]{2.409pt}{0.400pt}}
\put(170.0,517.0){\rule[-0.200pt]{2.409pt}{0.400pt}}
\put(1429.0,517.0){\rule[-0.200pt]{2.409pt}{0.400pt}}
\put(170.0,517.0){\rule[-0.200pt]{2.409pt}{0.400pt}}
\put(1429.0,517.0){\rule[-0.200pt]{2.409pt}{0.400pt}}
\put(170.0,517.0){\rule[-0.200pt]{2.409pt}{0.400pt}}
\put(1429.0,517.0){\rule[-0.200pt]{2.409pt}{0.400pt}}
\put(170.0,517.0){\rule[-0.200pt]{2.409pt}{0.400pt}}
\put(1429.0,517.0){\rule[-0.200pt]{2.409pt}{0.400pt}}
\put(170.0,517.0){\rule[-0.200pt]{2.409pt}{0.400pt}}
\put(1429.0,517.0){\rule[-0.200pt]{2.409pt}{0.400pt}}
\put(170.0,517.0){\rule[-0.200pt]{2.409pt}{0.400pt}}
\put(1429.0,517.0){\rule[-0.200pt]{2.409pt}{0.400pt}}
\put(170.0,518.0){\rule[-0.200pt]{2.409pt}{0.400pt}}
\put(1429.0,518.0){\rule[-0.200pt]{2.409pt}{0.400pt}}
\put(170.0,518.0){\rule[-0.200pt]{2.409pt}{0.400pt}}
\put(1429.0,518.0){\rule[-0.200pt]{2.409pt}{0.400pt}}
\put(170.0,518.0){\rule[-0.200pt]{2.409pt}{0.400pt}}
\put(1429.0,518.0){\rule[-0.200pt]{2.409pt}{0.400pt}}
\put(170.0,518.0){\rule[-0.200pt]{2.409pt}{0.400pt}}
\put(1429.0,518.0){\rule[-0.200pt]{2.409pt}{0.400pt}}
\put(170.0,518.0){\rule[-0.200pt]{2.409pt}{0.400pt}}
\put(1429.0,518.0){\rule[-0.200pt]{2.409pt}{0.400pt}}
\put(170.0,518.0){\rule[-0.200pt]{2.409pt}{0.400pt}}
\put(1429.0,518.0){\rule[-0.200pt]{2.409pt}{0.400pt}}
\put(170.0,518.0){\rule[-0.200pt]{2.409pt}{0.400pt}}
\put(1429.0,518.0){\rule[-0.200pt]{2.409pt}{0.400pt}}
\put(170.0,518.0){\rule[-0.200pt]{2.409pt}{0.400pt}}
\put(1429.0,518.0){\rule[-0.200pt]{2.409pt}{0.400pt}}
\put(170.0,518.0){\rule[-0.200pt]{2.409pt}{0.400pt}}
\put(1429.0,518.0){\rule[-0.200pt]{2.409pt}{0.400pt}}
\put(170.0,518.0){\rule[-0.200pt]{2.409pt}{0.400pt}}
\put(1429.0,518.0){\rule[-0.200pt]{2.409pt}{0.400pt}}
\put(170.0,518.0){\rule[-0.200pt]{2.409pt}{0.400pt}}
\put(1429.0,518.0){\rule[-0.200pt]{2.409pt}{0.400pt}}
\put(170.0,518.0){\rule[-0.200pt]{2.409pt}{0.400pt}}
\put(1429.0,518.0){\rule[-0.200pt]{2.409pt}{0.400pt}}
\put(170.0,518.0){\rule[-0.200pt]{2.409pt}{0.400pt}}
\put(1429.0,518.0){\rule[-0.200pt]{2.409pt}{0.400pt}}
\put(170.0,519.0){\rule[-0.200pt]{2.409pt}{0.400pt}}
\put(1429.0,519.0){\rule[-0.200pt]{2.409pt}{0.400pt}}
\put(170.0,519.0){\rule[-0.200pt]{2.409pt}{0.400pt}}
\put(1429.0,519.0){\rule[-0.200pt]{2.409pt}{0.400pt}}
\put(170.0,519.0){\rule[-0.200pt]{2.409pt}{0.400pt}}
\put(1429.0,519.0){\rule[-0.200pt]{2.409pt}{0.400pt}}
\put(170.0,519.0){\rule[-0.200pt]{2.409pt}{0.400pt}}
\put(1429.0,519.0){\rule[-0.200pt]{2.409pt}{0.400pt}}
\put(170.0,519.0){\rule[-0.200pt]{2.409pt}{0.400pt}}
\put(1429.0,519.0){\rule[-0.200pt]{2.409pt}{0.400pt}}
\put(170.0,519.0){\rule[-0.200pt]{2.409pt}{0.400pt}}
\put(1429.0,519.0){\rule[-0.200pt]{2.409pt}{0.400pt}}
\put(170.0,519.0){\rule[-0.200pt]{2.409pt}{0.400pt}}
\put(1429.0,519.0){\rule[-0.200pt]{2.409pt}{0.400pt}}
\put(170.0,519.0){\rule[-0.200pt]{2.409pt}{0.400pt}}
\put(1429.0,519.0){\rule[-0.200pt]{2.409pt}{0.400pt}}
\put(170.0,519.0){\rule[-0.200pt]{2.409pt}{0.400pt}}
\put(1429.0,519.0){\rule[-0.200pt]{2.409pt}{0.400pt}}
\put(170.0,519.0){\rule[-0.200pt]{2.409pt}{0.400pt}}
\put(1429.0,519.0){\rule[-0.200pt]{2.409pt}{0.400pt}}
\put(170.0,519.0){\rule[-0.200pt]{2.409pt}{0.400pt}}
\put(1429.0,519.0){\rule[-0.200pt]{2.409pt}{0.400pt}}
\put(170.0,519.0){\rule[-0.200pt]{2.409pt}{0.400pt}}
\put(1429.0,519.0){\rule[-0.200pt]{2.409pt}{0.400pt}}
\put(170.0,519.0){\rule[-0.200pt]{2.409pt}{0.400pt}}
\put(1429.0,519.0){\rule[-0.200pt]{2.409pt}{0.400pt}}
\put(170.0,520.0){\rule[-0.200pt]{2.409pt}{0.400pt}}
\put(1429.0,520.0){\rule[-0.200pt]{2.409pt}{0.400pt}}
\put(170.0,520.0){\rule[-0.200pt]{2.409pt}{0.400pt}}
\put(1429.0,520.0){\rule[-0.200pt]{2.409pt}{0.400pt}}
\put(170.0,520.0){\rule[-0.200pt]{2.409pt}{0.400pt}}
\put(1429.0,520.0){\rule[-0.200pt]{2.409pt}{0.400pt}}
\put(170.0,520.0){\rule[-0.200pt]{2.409pt}{0.400pt}}
\put(1429.0,520.0){\rule[-0.200pt]{2.409pt}{0.400pt}}
\put(170.0,520.0){\rule[-0.200pt]{2.409pt}{0.400pt}}
\put(1429.0,520.0){\rule[-0.200pt]{2.409pt}{0.400pt}}
\put(170.0,520.0){\rule[-0.200pt]{2.409pt}{0.400pt}}
\put(1429.0,520.0){\rule[-0.200pt]{2.409pt}{0.400pt}}
\put(170.0,520.0){\rule[-0.200pt]{2.409pt}{0.400pt}}
\put(1429.0,520.0){\rule[-0.200pt]{2.409pt}{0.400pt}}
\put(170.0,520.0){\rule[-0.200pt]{2.409pt}{0.400pt}}
\put(1429.0,520.0){\rule[-0.200pt]{2.409pt}{0.400pt}}
\put(170.0,520.0){\rule[-0.200pt]{2.409pt}{0.400pt}}
\put(1429.0,520.0){\rule[-0.200pt]{2.409pt}{0.400pt}}
\put(170.0,520.0){\rule[-0.200pt]{2.409pt}{0.400pt}}
\put(1429.0,520.0){\rule[-0.200pt]{2.409pt}{0.400pt}}
\put(170.0,520.0){\rule[-0.200pt]{2.409pt}{0.400pt}}
\put(1429.0,520.0){\rule[-0.200pt]{2.409pt}{0.400pt}}
\put(170.0,520.0){\rule[-0.200pt]{2.409pt}{0.400pt}}
\put(1429.0,520.0){\rule[-0.200pt]{2.409pt}{0.400pt}}
\put(170.0,520.0){\rule[-0.200pt]{2.409pt}{0.400pt}}
\put(1429.0,520.0){\rule[-0.200pt]{2.409pt}{0.400pt}}
\put(170.0,520.0){\rule[-0.200pt]{2.409pt}{0.400pt}}
\put(1429.0,520.0){\rule[-0.200pt]{2.409pt}{0.400pt}}
\put(170.0,520.0){\rule[-0.200pt]{2.409pt}{0.400pt}}
\put(1429.0,520.0){\rule[-0.200pt]{2.409pt}{0.400pt}}
\put(170.0,521.0){\rule[-0.200pt]{2.409pt}{0.400pt}}
\put(1429.0,521.0){\rule[-0.200pt]{2.409pt}{0.400pt}}
\put(170.0,521.0){\rule[-0.200pt]{2.409pt}{0.400pt}}
\put(1429.0,521.0){\rule[-0.200pt]{2.409pt}{0.400pt}}
\put(170.0,521.0){\rule[-0.200pt]{2.409pt}{0.400pt}}
\put(1429.0,521.0){\rule[-0.200pt]{2.409pt}{0.400pt}}
\put(170.0,521.0){\rule[-0.200pt]{2.409pt}{0.400pt}}
\put(1429.0,521.0){\rule[-0.200pt]{2.409pt}{0.400pt}}
\put(170.0,521.0){\rule[-0.200pt]{2.409pt}{0.400pt}}
\put(1429.0,521.0){\rule[-0.200pt]{2.409pt}{0.400pt}}
\put(170.0,521.0){\rule[-0.200pt]{2.409pt}{0.400pt}}
\put(1429.0,521.0){\rule[-0.200pt]{2.409pt}{0.400pt}}
\put(170.0,521.0){\rule[-0.200pt]{2.409pt}{0.400pt}}
\put(1429.0,521.0){\rule[-0.200pt]{2.409pt}{0.400pt}}
\put(170.0,521.0){\rule[-0.200pt]{2.409pt}{0.400pt}}
\put(1429.0,521.0){\rule[-0.200pt]{2.409pt}{0.400pt}}
\put(170.0,521.0){\rule[-0.200pt]{2.409pt}{0.400pt}}
\put(1429.0,521.0){\rule[-0.200pt]{2.409pt}{0.400pt}}
\put(170.0,521.0){\rule[-0.200pt]{2.409pt}{0.400pt}}
\put(1429.0,521.0){\rule[-0.200pt]{2.409pt}{0.400pt}}
\put(170.0,521.0){\rule[-0.200pt]{2.409pt}{0.400pt}}
\put(1429.0,521.0){\rule[-0.200pt]{2.409pt}{0.400pt}}
\put(170.0,521.0){\rule[-0.200pt]{2.409pt}{0.400pt}}
\put(1429.0,521.0){\rule[-0.200pt]{2.409pt}{0.400pt}}
\put(170.0,521.0){\rule[-0.200pt]{2.409pt}{0.400pt}}
\put(1429.0,521.0){\rule[-0.200pt]{2.409pt}{0.400pt}}
\put(170.0,521.0){\rule[-0.200pt]{2.409pt}{0.400pt}}
\put(1429.0,521.0){\rule[-0.200pt]{2.409pt}{0.400pt}}
\put(170.0,521.0){\rule[-0.200pt]{2.409pt}{0.400pt}}
\put(1429.0,521.0){\rule[-0.200pt]{2.409pt}{0.400pt}}
\put(170.0,522.0){\rule[-0.200pt]{2.409pt}{0.400pt}}
\put(1429.0,522.0){\rule[-0.200pt]{2.409pt}{0.400pt}}
\put(170.0,522.0){\rule[-0.200pt]{2.409pt}{0.400pt}}
\put(1429.0,522.0){\rule[-0.200pt]{2.409pt}{0.400pt}}
\put(170.0,522.0){\rule[-0.200pt]{2.409pt}{0.400pt}}
\put(1429.0,522.0){\rule[-0.200pt]{2.409pt}{0.400pt}}
\put(170.0,522.0){\rule[-0.200pt]{2.409pt}{0.400pt}}
\put(1429.0,522.0){\rule[-0.200pt]{2.409pt}{0.400pt}}
\put(170.0,522.0){\rule[-0.200pt]{2.409pt}{0.400pt}}
\put(1429.0,522.0){\rule[-0.200pt]{2.409pt}{0.400pt}}
\put(170.0,522.0){\rule[-0.200pt]{2.409pt}{0.400pt}}
\put(1429.0,522.0){\rule[-0.200pt]{2.409pt}{0.400pt}}
\put(170.0,522.0){\rule[-0.200pt]{2.409pt}{0.400pt}}
\put(1429.0,522.0){\rule[-0.200pt]{2.409pt}{0.400pt}}
\put(170.0,522.0){\rule[-0.200pt]{2.409pt}{0.400pt}}
\put(1429.0,522.0){\rule[-0.200pt]{2.409pt}{0.400pt}}
\put(170.0,522.0){\rule[-0.200pt]{2.409pt}{0.400pt}}
\put(1429.0,522.0){\rule[-0.200pt]{2.409pt}{0.400pt}}
\put(170.0,522.0){\rule[-0.200pt]{2.409pt}{0.400pt}}
\put(1429.0,522.0){\rule[-0.200pt]{2.409pt}{0.400pt}}
\put(170.0,522.0){\rule[-0.200pt]{2.409pt}{0.400pt}}
\put(1429.0,522.0){\rule[-0.200pt]{2.409pt}{0.400pt}}
\put(170.0,522.0){\rule[-0.200pt]{2.409pt}{0.400pt}}
\put(1429.0,522.0){\rule[-0.200pt]{2.409pt}{0.400pt}}
\put(170.0,522.0){\rule[-0.200pt]{2.409pt}{0.400pt}}
\put(1429.0,522.0){\rule[-0.200pt]{2.409pt}{0.400pt}}
\put(170.0,522.0){\rule[-0.200pt]{2.409pt}{0.400pt}}
\put(1429.0,522.0){\rule[-0.200pt]{2.409pt}{0.400pt}}
\put(170.0,522.0){\rule[-0.200pt]{2.409pt}{0.400pt}}
\put(1429.0,522.0){\rule[-0.200pt]{2.409pt}{0.400pt}}
\put(170.0,522.0){\rule[-0.200pt]{2.409pt}{0.400pt}}
\put(1429.0,522.0){\rule[-0.200pt]{2.409pt}{0.400pt}}
\put(170.0,523.0){\rule[-0.200pt]{2.409pt}{0.400pt}}
\put(1429.0,523.0){\rule[-0.200pt]{2.409pt}{0.400pt}}
\put(170.0,523.0){\rule[-0.200pt]{2.409pt}{0.400pt}}
\put(1429.0,523.0){\rule[-0.200pt]{2.409pt}{0.400pt}}
\put(170.0,523.0){\rule[-0.200pt]{2.409pt}{0.400pt}}
\put(1429.0,523.0){\rule[-0.200pt]{2.409pt}{0.400pt}}
\put(170.0,523.0){\rule[-0.200pt]{2.409pt}{0.400pt}}
\put(1429.0,523.0){\rule[-0.200pt]{2.409pt}{0.400pt}}
\put(170.0,523.0){\rule[-0.200pt]{2.409pt}{0.400pt}}
\put(1429.0,523.0){\rule[-0.200pt]{2.409pt}{0.400pt}}
\put(170.0,523.0){\rule[-0.200pt]{2.409pt}{0.400pt}}
\put(1429.0,523.0){\rule[-0.200pt]{2.409pt}{0.400pt}}
\put(170.0,523.0){\rule[-0.200pt]{2.409pt}{0.400pt}}
\put(1429.0,523.0){\rule[-0.200pt]{2.409pt}{0.400pt}}
\put(170.0,523.0){\rule[-0.200pt]{2.409pt}{0.400pt}}
\put(1429.0,523.0){\rule[-0.200pt]{2.409pt}{0.400pt}}
\put(170.0,523.0){\rule[-0.200pt]{2.409pt}{0.400pt}}
\put(1429.0,523.0){\rule[-0.200pt]{2.409pt}{0.400pt}}
\put(170.0,523.0){\rule[-0.200pt]{2.409pt}{0.400pt}}
\put(1429.0,523.0){\rule[-0.200pt]{2.409pt}{0.400pt}}
\put(170.0,523.0){\rule[-0.200pt]{2.409pt}{0.400pt}}
\put(1429.0,523.0){\rule[-0.200pt]{2.409pt}{0.400pt}}
\put(170.0,523.0){\rule[-0.200pt]{2.409pt}{0.400pt}}
\put(1429.0,523.0){\rule[-0.200pt]{2.409pt}{0.400pt}}
\put(170.0,523.0){\rule[-0.200pt]{2.409pt}{0.400pt}}
\put(1429.0,523.0){\rule[-0.200pt]{2.409pt}{0.400pt}}
\put(170.0,523.0){\rule[-0.200pt]{2.409pt}{0.400pt}}
\put(1429.0,523.0){\rule[-0.200pt]{2.409pt}{0.400pt}}
\put(170.0,523.0){\rule[-0.200pt]{2.409pt}{0.400pt}}
\put(1429.0,523.0){\rule[-0.200pt]{2.409pt}{0.400pt}}
\put(170.0,523.0){\rule[-0.200pt]{2.409pt}{0.400pt}}
\put(1429.0,523.0){\rule[-0.200pt]{2.409pt}{0.400pt}}
\put(170.0,523.0){\rule[-0.200pt]{2.409pt}{0.400pt}}
\put(1429.0,523.0){\rule[-0.200pt]{2.409pt}{0.400pt}}
\put(170.0,524.0){\rule[-0.200pt]{2.409pt}{0.400pt}}
\put(1429.0,524.0){\rule[-0.200pt]{2.409pt}{0.400pt}}
\put(170.0,524.0){\rule[-0.200pt]{2.409pt}{0.400pt}}
\put(1429.0,524.0){\rule[-0.200pt]{2.409pt}{0.400pt}}
\put(170.0,524.0){\rule[-0.200pt]{2.409pt}{0.400pt}}
\put(1429.0,524.0){\rule[-0.200pt]{2.409pt}{0.400pt}}
\put(170.0,524.0){\rule[-0.200pt]{2.409pt}{0.400pt}}
\put(1429.0,524.0){\rule[-0.200pt]{2.409pt}{0.400pt}}
\put(170.0,524.0){\rule[-0.200pt]{2.409pt}{0.400pt}}
\put(1429.0,524.0){\rule[-0.200pt]{2.409pt}{0.400pt}}
\put(170.0,524.0){\rule[-0.200pt]{2.409pt}{0.400pt}}
\put(1429.0,524.0){\rule[-0.200pt]{2.409pt}{0.400pt}}
\put(170.0,524.0){\rule[-0.200pt]{2.409pt}{0.400pt}}
\put(1429.0,524.0){\rule[-0.200pt]{2.409pt}{0.400pt}}
\put(170.0,524.0){\rule[-0.200pt]{2.409pt}{0.400pt}}
\put(1429.0,524.0){\rule[-0.200pt]{2.409pt}{0.400pt}}
\put(170.0,524.0){\rule[-0.200pt]{2.409pt}{0.400pt}}
\put(1429.0,524.0){\rule[-0.200pt]{2.409pt}{0.400pt}}
\put(170.0,524.0){\rule[-0.200pt]{2.409pt}{0.400pt}}
\put(1429.0,524.0){\rule[-0.200pt]{2.409pt}{0.400pt}}
\put(170.0,524.0){\rule[-0.200pt]{2.409pt}{0.400pt}}
\put(1429.0,524.0){\rule[-0.200pt]{2.409pt}{0.400pt}}
\put(170.0,524.0){\rule[-0.200pt]{2.409pt}{0.400pt}}
\put(1429.0,524.0){\rule[-0.200pt]{2.409pt}{0.400pt}}
\put(170.0,524.0){\rule[-0.200pt]{2.409pt}{0.400pt}}
\put(1429.0,524.0){\rule[-0.200pt]{2.409pt}{0.400pt}}
\put(170.0,524.0){\rule[-0.200pt]{2.409pt}{0.400pt}}
\put(1429.0,524.0){\rule[-0.200pt]{2.409pt}{0.400pt}}
\put(170.0,524.0){\rule[-0.200pt]{2.409pt}{0.400pt}}
\put(1429.0,524.0){\rule[-0.200pt]{2.409pt}{0.400pt}}
\put(170.0,524.0){\rule[-0.200pt]{2.409pt}{0.400pt}}
\put(1429.0,524.0){\rule[-0.200pt]{2.409pt}{0.400pt}}
\put(170.0,524.0){\rule[-0.200pt]{2.409pt}{0.400pt}}
\put(1429.0,524.0){\rule[-0.200pt]{2.409pt}{0.400pt}}
\put(170.0,524.0){\rule[-0.200pt]{2.409pt}{0.400pt}}
\put(1429.0,524.0){\rule[-0.200pt]{2.409pt}{0.400pt}}
\put(170.0,525.0){\rule[-0.200pt]{2.409pt}{0.400pt}}
\put(1429.0,525.0){\rule[-0.200pt]{2.409pt}{0.400pt}}
\put(170.0,525.0){\rule[-0.200pt]{2.409pt}{0.400pt}}
\put(1429.0,525.0){\rule[-0.200pt]{2.409pt}{0.400pt}}
\put(170.0,525.0){\rule[-0.200pt]{2.409pt}{0.400pt}}
\put(1429.0,525.0){\rule[-0.200pt]{2.409pt}{0.400pt}}
\put(170.0,525.0){\rule[-0.200pt]{2.409pt}{0.400pt}}
\put(1429.0,525.0){\rule[-0.200pt]{2.409pt}{0.400pt}}
\put(170.0,525.0){\rule[-0.200pt]{2.409pt}{0.400pt}}
\put(1429.0,525.0){\rule[-0.200pt]{2.409pt}{0.400pt}}
\put(170.0,525.0){\rule[-0.200pt]{2.409pt}{0.400pt}}
\put(1429.0,525.0){\rule[-0.200pt]{2.409pt}{0.400pt}}
\put(170.0,525.0){\rule[-0.200pt]{2.409pt}{0.400pt}}
\put(1429.0,525.0){\rule[-0.200pt]{2.409pt}{0.400pt}}
\put(170.0,525.0){\rule[-0.200pt]{2.409pt}{0.400pt}}
\put(1429.0,525.0){\rule[-0.200pt]{2.409pt}{0.400pt}}
\put(170.0,525.0){\rule[-0.200pt]{2.409pt}{0.400pt}}
\put(1429.0,525.0){\rule[-0.200pt]{2.409pt}{0.400pt}}
\put(170.0,525.0){\rule[-0.200pt]{2.409pt}{0.400pt}}
\put(1429.0,525.0){\rule[-0.200pt]{2.409pt}{0.400pt}}
\put(170.0,525.0){\rule[-0.200pt]{2.409pt}{0.400pt}}
\put(1429.0,525.0){\rule[-0.200pt]{2.409pt}{0.400pt}}
\put(170.0,525.0){\rule[-0.200pt]{2.409pt}{0.400pt}}
\put(1429.0,525.0){\rule[-0.200pt]{2.409pt}{0.400pt}}
\put(170.0,525.0){\rule[-0.200pt]{2.409pt}{0.400pt}}
\put(1429.0,525.0){\rule[-0.200pt]{2.409pt}{0.400pt}}
\put(170.0,525.0){\rule[-0.200pt]{2.409pt}{0.400pt}}
\put(1429.0,525.0){\rule[-0.200pt]{2.409pt}{0.400pt}}
\put(170.0,525.0){\rule[-0.200pt]{2.409pt}{0.400pt}}
\put(1429.0,525.0){\rule[-0.200pt]{2.409pt}{0.400pt}}
\put(170.0,525.0){\rule[-0.200pt]{2.409pt}{0.400pt}}
\put(1429.0,525.0){\rule[-0.200pt]{2.409pt}{0.400pt}}
\put(170.0,525.0){\rule[-0.200pt]{2.409pt}{0.400pt}}
\put(1429.0,525.0){\rule[-0.200pt]{2.409pt}{0.400pt}}
\put(170.0,525.0){\rule[-0.200pt]{2.409pt}{0.400pt}}
\put(1429.0,525.0){\rule[-0.200pt]{2.409pt}{0.400pt}}
\put(170.0,525.0){\rule[-0.200pt]{2.409pt}{0.400pt}}
\put(1429.0,525.0){\rule[-0.200pt]{2.409pt}{0.400pt}}
\put(170.0,525.0){\rule[-0.200pt]{2.409pt}{0.400pt}}
\put(1429.0,525.0){\rule[-0.200pt]{2.409pt}{0.400pt}}
\put(170.0,526.0){\rule[-0.200pt]{2.409pt}{0.400pt}}
\put(1429.0,526.0){\rule[-0.200pt]{2.409pt}{0.400pt}}
\put(170.0,526.0){\rule[-0.200pt]{2.409pt}{0.400pt}}
\put(1429.0,526.0){\rule[-0.200pt]{2.409pt}{0.400pt}}
\put(170.0,526.0){\rule[-0.200pt]{2.409pt}{0.400pt}}
\put(1429.0,526.0){\rule[-0.200pt]{2.409pt}{0.400pt}}
\put(170.0,526.0){\rule[-0.200pt]{2.409pt}{0.400pt}}
\put(1429.0,526.0){\rule[-0.200pt]{2.409pt}{0.400pt}}
\put(170.0,526.0){\rule[-0.200pt]{2.409pt}{0.400pt}}
\put(1429.0,526.0){\rule[-0.200pt]{2.409pt}{0.400pt}}
\put(170.0,526.0){\rule[-0.200pt]{2.409pt}{0.400pt}}
\put(1429.0,526.0){\rule[-0.200pt]{2.409pt}{0.400pt}}
\put(170.0,526.0){\rule[-0.200pt]{2.409pt}{0.400pt}}
\put(1429.0,526.0){\rule[-0.200pt]{2.409pt}{0.400pt}}
\put(170.0,526.0){\rule[-0.200pt]{2.409pt}{0.400pt}}
\put(1429.0,526.0){\rule[-0.200pt]{2.409pt}{0.400pt}}
\put(170.0,526.0){\rule[-0.200pt]{2.409pt}{0.400pt}}
\put(1429.0,526.0){\rule[-0.200pt]{2.409pt}{0.400pt}}
\put(170.0,526.0){\rule[-0.200pt]{2.409pt}{0.400pt}}
\put(1429.0,526.0){\rule[-0.200pt]{2.409pt}{0.400pt}}
\put(170.0,526.0){\rule[-0.200pt]{4.818pt}{0.400pt}}
\put(150,526){\makebox(0,0)[r]{ 1e-06}}
\put(1419.0,526.0){\rule[-0.200pt]{4.818pt}{0.400pt}}
\put(170.0,537.0){\rule[-0.200pt]{2.409pt}{0.400pt}}
\put(1429.0,537.0){\rule[-0.200pt]{2.409pt}{0.400pt}}
\put(170.0,552.0){\rule[-0.200pt]{2.409pt}{0.400pt}}
\put(1429.0,552.0){\rule[-0.200pt]{2.409pt}{0.400pt}}
\put(170.0,559.0){\rule[-0.200pt]{2.409pt}{0.400pt}}
\put(1429.0,559.0){\rule[-0.200pt]{2.409pt}{0.400pt}}
\put(170.0,565.0){\rule[-0.200pt]{2.409pt}{0.400pt}}
\put(1429.0,565.0){\rule[-0.200pt]{2.409pt}{0.400pt}}
\put(170.0,568.0){\rule[-0.200pt]{2.409pt}{0.400pt}}
\put(1429.0,568.0){\rule[-0.200pt]{2.409pt}{0.400pt}}
\put(170.0,572.0){\rule[-0.200pt]{2.409pt}{0.400pt}}
\put(1429.0,572.0){\rule[-0.200pt]{2.409pt}{0.400pt}}
\put(170.0,574.0){\rule[-0.200pt]{2.409pt}{0.400pt}}
\put(1429.0,574.0){\rule[-0.200pt]{2.409pt}{0.400pt}}
\put(170.0,576.0){\rule[-0.200pt]{2.409pt}{0.400pt}}
\put(1429.0,576.0){\rule[-0.200pt]{2.409pt}{0.400pt}}
\put(170.0,578.0){\rule[-0.200pt]{2.409pt}{0.400pt}}
\put(1429.0,578.0){\rule[-0.200pt]{2.409pt}{0.400pt}}
\put(170.0,580.0){\rule[-0.200pt]{2.409pt}{0.400pt}}
\put(1429.0,580.0){\rule[-0.200pt]{2.409pt}{0.400pt}}
\put(170.0,582.0){\rule[-0.200pt]{2.409pt}{0.400pt}}
\put(1429.0,582.0){\rule[-0.200pt]{2.409pt}{0.400pt}}
\put(170.0,583.0){\rule[-0.200pt]{2.409pt}{0.400pt}}
\put(1429.0,583.0){\rule[-0.200pt]{2.409pt}{0.400pt}}
\put(170.0,584.0){\rule[-0.200pt]{2.409pt}{0.400pt}}
\put(1429.0,584.0){\rule[-0.200pt]{2.409pt}{0.400pt}}
\put(170.0,586.0){\rule[-0.200pt]{2.409pt}{0.400pt}}
\put(1429.0,586.0){\rule[-0.200pt]{2.409pt}{0.400pt}}
\put(170.0,587.0){\rule[-0.200pt]{2.409pt}{0.400pt}}
\put(1429.0,587.0){\rule[-0.200pt]{2.409pt}{0.400pt}}
\put(170.0,588.0){\rule[-0.200pt]{2.409pt}{0.400pt}}
\put(1429.0,588.0){\rule[-0.200pt]{2.409pt}{0.400pt}}
\put(170.0,589.0){\rule[-0.200pt]{2.409pt}{0.400pt}}
\put(1429.0,589.0){\rule[-0.200pt]{2.409pt}{0.400pt}}
\put(170.0,590.0){\rule[-0.200pt]{2.409pt}{0.400pt}}
\put(1429.0,590.0){\rule[-0.200pt]{2.409pt}{0.400pt}}
\put(170.0,591.0){\rule[-0.200pt]{2.409pt}{0.400pt}}
\put(1429.0,591.0){\rule[-0.200pt]{2.409pt}{0.400pt}}
\put(170.0,592.0){\rule[-0.200pt]{2.409pt}{0.400pt}}
\put(1429.0,592.0){\rule[-0.200pt]{2.409pt}{0.400pt}}
\put(170.0,592.0){\rule[-0.200pt]{2.409pt}{0.400pt}}
\put(1429.0,592.0){\rule[-0.200pt]{2.409pt}{0.400pt}}
\put(170.0,593.0){\rule[-0.200pt]{2.409pt}{0.400pt}}
\put(1429.0,593.0){\rule[-0.200pt]{2.409pt}{0.400pt}}
\put(170.0,594.0){\rule[-0.200pt]{2.409pt}{0.400pt}}
\put(1429.0,594.0){\rule[-0.200pt]{2.409pt}{0.400pt}}
\put(170.0,594.0){\rule[-0.200pt]{2.409pt}{0.400pt}}
\put(1429.0,594.0){\rule[-0.200pt]{2.409pt}{0.400pt}}
\put(170.0,595.0){\rule[-0.200pt]{2.409pt}{0.400pt}}
\put(1429.0,595.0){\rule[-0.200pt]{2.409pt}{0.400pt}}
\put(170.0,596.0){\rule[-0.200pt]{2.409pt}{0.400pt}}
\put(1429.0,596.0){\rule[-0.200pt]{2.409pt}{0.400pt}}
\put(170.0,596.0){\rule[-0.200pt]{2.409pt}{0.400pt}}
\put(1429.0,596.0){\rule[-0.200pt]{2.409pt}{0.400pt}}
\put(170.0,597.0){\rule[-0.200pt]{2.409pt}{0.400pt}}
\put(1429.0,597.0){\rule[-0.200pt]{2.409pt}{0.400pt}}
\put(170.0,598.0){\rule[-0.200pt]{2.409pt}{0.400pt}}
\put(1429.0,598.0){\rule[-0.200pt]{2.409pt}{0.400pt}}
\put(170.0,598.0){\rule[-0.200pt]{2.409pt}{0.400pt}}
\put(1429.0,598.0){\rule[-0.200pt]{2.409pt}{0.400pt}}
\put(170.0,599.0){\rule[-0.200pt]{2.409pt}{0.400pt}}
\put(1429.0,599.0){\rule[-0.200pt]{2.409pt}{0.400pt}}
\put(170.0,599.0){\rule[-0.200pt]{2.409pt}{0.400pt}}
\put(1429.0,599.0){\rule[-0.200pt]{2.409pt}{0.400pt}}
\put(170.0,600.0){\rule[-0.200pt]{2.409pt}{0.400pt}}
\put(1429.0,600.0){\rule[-0.200pt]{2.409pt}{0.400pt}}
\put(170.0,600.0){\rule[-0.200pt]{2.409pt}{0.400pt}}
\put(1429.0,600.0){\rule[-0.200pt]{2.409pt}{0.400pt}}
\put(170.0,601.0){\rule[-0.200pt]{2.409pt}{0.400pt}}
\put(1429.0,601.0){\rule[-0.200pt]{2.409pt}{0.400pt}}
\put(170.0,601.0){\rule[-0.200pt]{2.409pt}{0.400pt}}
\put(1429.0,601.0){\rule[-0.200pt]{2.409pt}{0.400pt}}
\put(170.0,602.0){\rule[-0.200pt]{2.409pt}{0.400pt}}
\put(1429.0,602.0){\rule[-0.200pt]{2.409pt}{0.400pt}}
\put(170.0,602.0){\rule[-0.200pt]{2.409pt}{0.400pt}}
\put(1429.0,602.0){\rule[-0.200pt]{2.409pt}{0.400pt}}
\put(170.0,602.0){\rule[-0.200pt]{2.409pt}{0.400pt}}
\put(1429.0,602.0){\rule[-0.200pt]{2.409pt}{0.400pt}}
\put(170.0,603.0){\rule[-0.200pt]{2.409pt}{0.400pt}}
\put(1429.0,603.0){\rule[-0.200pt]{2.409pt}{0.400pt}}
\put(170.0,603.0){\rule[-0.200pt]{2.409pt}{0.400pt}}
\put(1429.0,603.0){\rule[-0.200pt]{2.409pt}{0.400pt}}
\put(170.0,604.0){\rule[-0.200pt]{2.409pt}{0.400pt}}
\put(1429.0,604.0){\rule[-0.200pt]{2.409pt}{0.400pt}}
\put(170.0,604.0){\rule[-0.200pt]{2.409pt}{0.400pt}}
\put(1429.0,604.0){\rule[-0.200pt]{2.409pt}{0.400pt}}
\put(170.0,604.0){\rule[-0.200pt]{2.409pt}{0.400pt}}
\put(1429.0,604.0){\rule[-0.200pt]{2.409pt}{0.400pt}}
\put(170.0,605.0){\rule[-0.200pt]{2.409pt}{0.400pt}}
\put(1429.0,605.0){\rule[-0.200pt]{2.409pt}{0.400pt}}
\put(170.0,605.0){\rule[-0.200pt]{2.409pt}{0.400pt}}
\put(1429.0,605.0){\rule[-0.200pt]{2.409pt}{0.400pt}}
\put(170.0,605.0){\rule[-0.200pt]{2.409pt}{0.400pt}}
\put(1429.0,605.0){\rule[-0.200pt]{2.409pt}{0.400pt}}
\put(170.0,606.0){\rule[-0.200pt]{2.409pt}{0.400pt}}
\put(1429.0,606.0){\rule[-0.200pt]{2.409pt}{0.400pt}}
\put(170.0,606.0){\rule[-0.200pt]{2.409pt}{0.400pt}}
\put(1429.0,606.0){\rule[-0.200pt]{2.409pt}{0.400pt}}
\put(170.0,606.0){\rule[-0.200pt]{2.409pt}{0.400pt}}
\put(1429.0,606.0){\rule[-0.200pt]{2.409pt}{0.400pt}}
\put(170.0,607.0){\rule[-0.200pt]{2.409pt}{0.400pt}}
\put(1429.0,607.0){\rule[-0.200pt]{2.409pt}{0.400pt}}
\put(170.0,607.0){\rule[-0.200pt]{2.409pt}{0.400pt}}
\put(1429.0,607.0){\rule[-0.200pt]{2.409pt}{0.400pt}}
\put(170.0,607.0){\rule[-0.200pt]{2.409pt}{0.400pt}}
\put(1429.0,607.0){\rule[-0.200pt]{2.409pt}{0.400pt}}
\put(170.0,608.0){\rule[-0.200pt]{2.409pt}{0.400pt}}
\put(1429.0,608.0){\rule[-0.200pt]{2.409pt}{0.400pt}}
\put(170.0,608.0){\rule[-0.200pt]{2.409pt}{0.400pt}}
\put(1429.0,608.0){\rule[-0.200pt]{2.409pt}{0.400pt}}
\put(170.0,608.0){\rule[-0.200pt]{2.409pt}{0.400pt}}
\put(1429.0,608.0){\rule[-0.200pt]{2.409pt}{0.400pt}}
\put(170.0,609.0){\rule[-0.200pt]{2.409pt}{0.400pt}}
\put(1429.0,609.0){\rule[-0.200pt]{2.409pt}{0.400pt}}
\put(170.0,609.0){\rule[-0.200pt]{2.409pt}{0.400pt}}
\put(1429.0,609.0){\rule[-0.200pt]{2.409pt}{0.400pt}}
\put(170.0,609.0){\rule[-0.200pt]{2.409pt}{0.400pt}}
\put(1429.0,609.0){\rule[-0.200pt]{2.409pt}{0.400pt}}
\put(170.0,609.0){\rule[-0.200pt]{2.409pt}{0.400pt}}
\put(1429.0,609.0){\rule[-0.200pt]{2.409pt}{0.400pt}}
\put(170.0,610.0){\rule[-0.200pt]{2.409pt}{0.400pt}}
\put(1429.0,610.0){\rule[-0.200pt]{2.409pt}{0.400pt}}
\put(170.0,610.0){\rule[-0.200pt]{2.409pt}{0.400pt}}
\put(1429.0,610.0){\rule[-0.200pt]{2.409pt}{0.400pt}}
\put(170.0,610.0){\rule[-0.200pt]{2.409pt}{0.400pt}}
\put(1429.0,610.0){\rule[-0.200pt]{2.409pt}{0.400pt}}
\put(170.0,610.0){\rule[-0.200pt]{2.409pt}{0.400pt}}
\put(1429.0,610.0){\rule[-0.200pt]{2.409pt}{0.400pt}}
\put(170.0,611.0){\rule[-0.200pt]{2.409pt}{0.400pt}}
\put(1429.0,611.0){\rule[-0.200pt]{2.409pt}{0.400pt}}
\put(170.0,611.0){\rule[-0.200pt]{2.409pt}{0.400pt}}
\put(1429.0,611.0){\rule[-0.200pt]{2.409pt}{0.400pt}}
\put(170.0,611.0){\rule[-0.200pt]{2.409pt}{0.400pt}}
\put(1429.0,611.0){\rule[-0.200pt]{2.409pt}{0.400pt}}
\put(170.0,611.0){\rule[-0.200pt]{2.409pt}{0.400pt}}
\put(1429.0,611.0){\rule[-0.200pt]{2.409pt}{0.400pt}}
\put(170.0,612.0){\rule[-0.200pt]{2.409pt}{0.400pt}}
\put(1429.0,612.0){\rule[-0.200pt]{2.409pt}{0.400pt}}
\put(170.0,612.0){\rule[-0.200pt]{2.409pt}{0.400pt}}
\put(1429.0,612.0){\rule[-0.200pt]{2.409pt}{0.400pt}}
\put(170.0,612.0){\rule[-0.200pt]{2.409pt}{0.400pt}}
\put(1429.0,612.0){\rule[-0.200pt]{2.409pt}{0.400pt}}
\put(170.0,612.0){\rule[-0.200pt]{2.409pt}{0.400pt}}
\put(1429.0,612.0){\rule[-0.200pt]{2.409pt}{0.400pt}}
\put(170.0,613.0){\rule[-0.200pt]{2.409pt}{0.400pt}}
\put(1429.0,613.0){\rule[-0.200pt]{2.409pt}{0.400pt}}
\put(170.0,613.0){\rule[-0.200pt]{2.409pt}{0.400pt}}
\put(1429.0,613.0){\rule[-0.200pt]{2.409pt}{0.400pt}}
\put(170.0,613.0){\rule[-0.200pt]{2.409pt}{0.400pt}}
\put(1429.0,613.0){\rule[-0.200pt]{2.409pt}{0.400pt}}
\put(170.0,613.0){\rule[-0.200pt]{2.409pt}{0.400pt}}
\put(1429.0,613.0){\rule[-0.200pt]{2.409pt}{0.400pt}}
\put(170.0,613.0){\rule[-0.200pt]{2.409pt}{0.400pt}}
\put(1429.0,613.0){\rule[-0.200pt]{2.409pt}{0.400pt}}
\put(170.0,614.0){\rule[-0.200pt]{2.409pt}{0.400pt}}
\put(1429.0,614.0){\rule[-0.200pt]{2.409pt}{0.400pt}}
\put(170.0,614.0){\rule[-0.200pt]{2.409pt}{0.400pt}}
\put(1429.0,614.0){\rule[-0.200pt]{2.409pt}{0.400pt}}
\put(170.0,614.0){\rule[-0.200pt]{2.409pt}{0.400pt}}
\put(1429.0,614.0){\rule[-0.200pt]{2.409pt}{0.400pt}}
\put(170.0,614.0){\rule[-0.200pt]{2.409pt}{0.400pt}}
\put(1429.0,614.0){\rule[-0.200pt]{2.409pt}{0.400pt}}
\put(170.0,614.0){\rule[-0.200pt]{2.409pt}{0.400pt}}
\put(1429.0,614.0){\rule[-0.200pt]{2.409pt}{0.400pt}}
\put(170.0,615.0){\rule[-0.200pt]{2.409pt}{0.400pt}}
\put(1429.0,615.0){\rule[-0.200pt]{2.409pt}{0.400pt}}
\put(170.0,615.0){\rule[-0.200pt]{2.409pt}{0.400pt}}
\put(1429.0,615.0){\rule[-0.200pt]{2.409pt}{0.400pt}}
\put(170.0,615.0){\rule[-0.200pt]{2.409pt}{0.400pt}}
\put(1429.0,615.0){\rule[-0.200pt]{2.409pt}{0.400pt}}
\put(170.0,615.0){\rule[-0.200pt]{2.409pt}{0.400pt}}
\put(1429.0,615.0){\rule[-0.200pt]{2.409pt}{0.400pt}}
\put(170.0,615.0){\rule[-0.200pt]{2.409pt}{0.400pt}}
\put(1429.0,615.0){\rule[-0.200pt]{2.409pt}{0.400pt}}
\put(170.0,616.0){\rule[-0.200pt]{2.409pt}{0.400pt}}
\put(1429.0,616.0){\rule[-0.200pt]{2.409pt}{0.400pt}}
\put(170.0,616.0){\rule[-0.200pt]{2.409pt}{0.400pt}}
\put(1429.0,616.0){\rule[-0.200pt]{2.409pt}{0.400pt}}
\put(170.0,616.0){\rule[-0.200pt]{2.409pt}{0.400pt}}
\put(1429.0,616.0){\rule[-0.200pt]{2.409pt}{0.400pt}}
\put(170.0,616.0){\rule[-0.200pt]{2.409pt}{0.400pt}}
\put(1429.0,616.0){\rule[-0.200pt]{2.409pt}{0.400pt}}
\put(170.0,616.0){\rule[-0.200pt]{2.409pt}{0.400pt}}
\put(1429.0,616.0){\rule[-0.200pt]{2.409pt}{0.400pt}}
\put(170.0,616.0){\rule[-0.200pt]{2.409pt}{0.400pt}}
\put(1429.0,616.0){\rule[-0.200pt]{2.409pt}{0.400pt}}
\put(170.0,617.0){\rule[-0.200pt]{2.409pt}{0.400pt}}
\put(1429.0,617.0){\rule[-0.200pt]{2.409pt}{0.400pt}}
\put(170.0,617.0){\rule[-0.200pt]{2.409pt}{0.400pt}}
\put(1429.0,617.0){\rule[-0.200pt]{2.409pt}{0.400pt}}
\put(170.0,617.0){\rule[-0.200pt]{2.409pt}{0.400pt}}
\put(1429.0,617.0){\rule[-0.200pt]{2.409pt}{0.400pt}}
\put(170.0,617.0){\rule[-0.200pt]{2.409pt}{0.400pt}}
\put(1429.0,617.0){\rule[-0.200pt]{2.409pt}{0.400pt}}
\put(170.0,617.0){\rule[-0.200pt]{2.409pt}{0.400pt}}
\put(1429.0,617.0){\rule[-0.200pt]{2.409pt}{0.400pt}}
\put(170.0,617.0){\rule[-0.200pt]{2.409pt}{0.400pt}}
\put(1429.0,617.0){\rule[-0.200pt]{2.409pt}{0.400pt}}
\put(170.0,618.0){\rule[-0.200pt]{2.409pt}{0.400pt}}
\put(1429.0,618.0){\rule[-0.200pt]{2.409pt}{0.400pt}}
\put(170.0,618.0){\rule[-0.200pt]{2.409pt}{0.400pt}}
\put(1429.0,618.0){\rule[-0.200pt]{2.409pt}{0.400pt}}
\put(170.0,618.0){\rule[-0.200pt]{2.409pt}{0.400pt}}
\put(1429.0,618.0){\rule[-0.200pt]{2.409pt}{0.400pt}}
\put(170.0,618.0){\rule[-0.200pt]{2.409pt}{0.400pt}}
\put(1429.0,618.0){\rule[-0.200pt]{2.409pt}{0.400pt}}
\put(170.0,618.0){\rule[-0.200pt]{2.409pt}{0.400pt}}
\put(1429.0,618.0){\rule[-0.200pt]{2.409pt}{0.400pt}}
\put(170.0,618.0){\rule[-0.200pt]{2.409pt}{0.400pt}}
\put(1429.0,618.0){\rule[-0.200pt]{2.409pt}{0.400pt}}
\put(170.0,619.0){\rule[-0.200pt]{2.409pt}{0.400pt}}
\put(1429.0,619.0){\rule[-0.200pt]{2.409pt}{0.400pt}}
\put(170.0,619.0){\rule[-0.200pt]{2.409pt}{0.400pt}}
\put(1429.0,619.0){\rule[-0.200pt]{2.409pt}{0.400pt}}
\put(170.0,619.0){\rule[-0.200pt]{2.409pt}{0.400pt}}
\put(1429.0,619.0){\rule[-0.200pt]{2.409pt}{0.400pt}}
\put(170.0,619.0){\rule[-0.200pt]{2.409pt}{0.400pt}}
\put(1429.0,619.0){\rule[-0.200pt]{2.409pt}{0.400pt}}
\put(170.0,619.0){\rule[-0.200pt]{2.409pt}{0.400pt}}
\put(1429.0,619.0){\rule[-0.200pt]{2.409pt}{0.400pt}}
\put(170.0,619.0){\rule[-0.200pt]{2.409pt}{0.400pt}}
\put(1429.0,619.0){\rule[-0.200pt]{2.409pt}{0.400pt}}
\put(170.0,619.0){\rule[-0.200pt]{2.409pt}{0.400pt}}
\put(1429.0,619.0){\rule[-0.200pt]{2.409pt}{0.400pt}}
\put(170.0,620.0){\rule[-0.200pt]{2.409pt}{0.400pt}}
\put(1429.0,620.0){\rule[-0.200pt]{2.409pt}{0.400pt}}
\put(170.0,620.0){\rule[-0.200pt]{2.409pt}{0.400pt}}
\put(1429.0,620.0){\rule[-0.200pt]{2.409pt}{0.400pt}}
\put(170.0,620.0){\rule[-0.200pt]{2.409pt}{0.400pt}}
\put(1429.0,620.0){\rule[-0.200pt]{2.409pt}{0.400pt}}
\put(170.0,620.0){\rule[-0.200pt]{2.409pt}{0.400pt}}
\put(1429.0,620.0){\rule[-0.200pt]{2.409pt}{0.400pt}}
\put(170.0,620.0){\rule[-0.200pt]{2.409pt}{0.400pt}}
\put(1429.0,620.0){\rule[-0.200pt]{2.409pt}{0.400pt}}
\put(170.0,620.0){\rule[-0.200pt]{2.409pt}{0.400pt}}
\put(1429.0,620.0){\rule[-0.200pt]{2.409pt}{0.400pt}}
\put(170.0,620.0){\rule[-0.200pt]{2.409pt}{0.400pt}}
\put(1429.0,620.0){\rule[-0.200pt]{2.409pt}{0.400pt}}
\put(170.0,621.0){\rule[-0.200pt]{2.409pt}{0.400pt}}
\put(1429.0,621.0){\rule[-0.200pt]{2.409pt}{0.400pt}}
\put(170.0,621.0){\rule[-0.200pt]{2.409pt}{0.400pt}}
\put(1429.0,621.0){\rule[-0.200pt]{2.409pt}{0.400pt}}
\put(170.0,621.0){\rule[-0.200pt]{2.409pt}{0.400pt}}
\put(1429.0,621.0){\rule[-0.200pt]{2.409pt}{0.400pt}}
\put(170.0,621.0){\rule[-0.200pt]{2.409pt}{0.400pt}}
\put(1429.0,621.0){\rule[-0.200pt]{2.409pt}{0.400pt}}
\put(170.0,621.0){\rule[-0.200pt]{2.409pt}{0.400pt}}
\put(1429.0,621.0){\rule[-0.200pt]{2.409pt}{0.400pt}}
\put(170.0,621.0){\rule[-0.200pt]{2.409pt}{0.400pt}}
\put(1429.0,621.0){\rule[-0.200pt]{2.409pt}{0.400pt}}
\put(170.0,621.0){\rule[-0.200pt]{2.409pt}{0.400pt}}
\put(1429.0,621.0){\rule[-0.200pt]{2.409pt}{0.400pt}}
\put(170.0,621.0){\rule[-0.200pt]{2.409pt}{0.400pt}}
\put(1429.0,621.0){\rule[-0.200pt]{2.409pt}{0.400pt}}
\put(170.0,622.0){\rule[-0.200pt]{2.409pt}{0.400pt}}
\put(1429.0,622.0){\rule[-0.200pt]{2.409pt}{0.400pt}}
\put(170.0,622.0){\rule[-0.200pt]{2.409pt}{0.400pt}}
\put(1429.0,622.0){\rule[-0.200pt]{2.409pt}{0.400pt}}
\put(170.0,622.0){\rule[-0.200pt]{2.409pt}{0.400pt}}
\put(1429.0,622.0){\rule[-0.200pt]{2.409pt}{0.400pt}}
\put(170.0,622.0){\rule[-0.200pt]{2.409pt}{0.400pt}}
\put(1429.0,622.0){\rule[-0.200pt]{2.409pt}{0.400pt}}
\put(170.0,622.0){\rule[-0.200pt]{2.409pt}{0.400pt}}
\put(1429.0,622.0){\rule[-0.200pt]{2.409pt}{0.400pt}}
\put(170.0,622.0){\rule[-0.200pt]{2.409pt}{0.400pt}}
\put(1429.0,622.0){\rule[-0.200pt]{2.409pt}{0.400pt}}
\put(170.0,622.0){\rule[-0.200pt]{2.409pt}{0.400pt}}
\put(1429.0,622.0){\rule[-0.200pt]{2.409pt}{0.400pt}}
\put(170.0,622.0){\rule[-0.200pt]{2.409pt}{0.400pt}}
\put(1429.0,622.0){\rule[-0.200pt]{2.409pt}{0.400pt}}
\put(170.0,623.0){\rule[-0.200pt]{2.409pt}{0.400pt}}
\put(1429.0,623.0){\rule[-0.200pt]{2.409pt}{0.400pt}}
\put(170.0,623.0){\rule[-0.200pt]{2.409pt}{0.400pt}}
\put(1429.0,623.0){\rule[-0.200pt]{2.409pt}{0.400pt}}
\put(170.0,623.0){\rule[-0.200pt]{2.409pt}{0.400pt}}
\put(1429.0,623.0){\rule[-0.200pt]{2.409pt}{0.400pt}}
\put(170.0,623.0){\rule[-0.200pt]{2.409pt}{0.400pt}}
\put(1429.0,623.0){\rule[-0.200pt]{2.409pt}{0.400pt}}
\put(170.0,623.0){\rule[-0.200pt]{2.409pt}{0.400pt}}
\put(1429.0,623.0){\rule[-0.200pt]{2.409pt}{0.400pt}}
\put(170.0,623.0){\rule[-0.200pt]{2.409pt}{0.400pt}}
\put(1429.0,623.0){\rule[-0.200pt]{2.409pt}{0.400pt}}
\put(170.0,623.0){\rule[-0.200pt]{2.409pt}{0.400pt}}
\put(1429.0,623.0){\rule[-0.200pt]{2.409pt}{0.400pt}}
\put(170.0,623.0){\rule[-0.200pt]{2.409pt}{0.400pt}}
\put(1429.0,623.0){\rule[-0.200pt]{2.409pt}{0.400pt}}
\put(170.0,623.0){\rule[-0.200pt]{2.409pt}{0.400pt}}
\put(1429.0,623.0){\rule[-0.200pt]{2.409pt}{0.400pt}}
\put(170.0,624.0){\rule[-0.200pt]{2.409pt}{0.400pt}}
\put(1429.0,624.0){\rule[-0.200pt]{2.409pt}{0.400pt}}
\put(170.0,624.0){\rule[-0.200pt]{2.409pt}{0.400pt}}
\put(1429.0,624.0){\rule[-0.200pt]{2.409pt}{0.400pt}}
\put(170.0,624.0){\rule[-0.200pt]{2.409pt}{0.400pt}}
\put(1429.0,624.0){\rule[-0.200pt]{2.409pt}{0.400pt}}
\put(170.0,624.0){\rule[-0.200pt]{2.409pt}{0.400pt}}
\put(1429.0,624.0){\rule[-0.200pt]{2.409pt}{0.400pt}}
\put(170.0,624.0){\rule[-0.200pt]{2.409pt}{0.400pt}}
\put(1429.0,624.0){\rule[-0.200pt]{2.409pt}{0.400pt}}
\put(170.0,624.0){\rule[-0.200pt]{2.409pt}{0.400pt}}
\put(1429.0,624.0){\rule[-0.200pt]{2.409pt}{0.400pt}}
\put(170.0,624.0){\rule[-0.200pt]{2.409pt}{0.400pt}}
\put(1429.0,624.0){\rule[-0.200pt]{2.409pt}{0.400pt}}
\put(170.0,624.0){\rule[-0.200pt]{2.409pt}{0.400pt}}
\put(1429.0,624.0){\rule[-0.200pt]{2.409pt}{0.400pt}}
\put(170.0,624.0){\rule[-0.200pt]{2.409pt}{0.400pt}}
\put(1429.0,624.0){\rule[-0.200pt]{2.409pt}{0.400pt}}
\put(170.0,625.0){\rule[-0.200pt]{2.409pt}{0.400pt}}
\put(1429.0,625.0){\rule[-0.200pt]{2.409pt}{0.400pt}}
\put(170.0,625.0){\rule[-0.200pt]{2.409pt}{0.400pt}}
\put(1429.0,625.0){\rule[-0.200pt]{2.409pt}{0.400pt}}
\put(170.0,625.0){\rule[-0.200pt]{2.409pt}{0.400pt}}
\put(1429.0,625.0){\rule[-0.200pt]{2.409pt}{0.400pt}}
\put(170.0,625.0){\rule[-0.200pt]{2.409pt}{0.400pt}}
\put(1429.0,625.0){\rule[-0.200pt]{2.409pt}{0.400pt}}
\put(170.0,625.0){\rule[-0.200pt]{2.409pt}{0.400pt}}
\put(1429.0,625.0){\rule[-0.200pt]{2.409pt}{0.400pt}}
\put(170.0,625.0){\rule[-0.200pt]{2.409pt}{0.400pt}}
\put(1429.0,625.0){\rule[-0.200pt]{2.409pt}{0.400pt}}
\put(170.0,625.0){\rule[-0.200pt]{2.409pt}{0.400pt}}
\put(1429.0,625.0){\rule[-0.200pt]{2.409pt}{0.400pt}}
\put(170.0,625.0){\rule[-0.200pt]{2.409pt}{0.400pt}}
\put(1429.0,625.0){\rule[-0.200pt]{2.409pt}{0.400pt}}
\put(170.0,625.0){\rule[-0.200pt]{2.409pt}{0.400pt}}
\put(1429.0,625.0){\rule[-0.200pt]{2.409pt}{0.400pt}}
\put(170.0,625.0){\rule[-0.200pt]{2.409pt}{0.400pt}}
\put(1429.0,625.0){\rule[-0.200pt]{2.409pt}{0.400pt}}
\put(170.0,626.0){\rule[-0.200pt]{2.409pt}{0.400pt}}
\put(1429.0,626.0){\rule[-0.200pt]{2.409pt}{0.400pt}}
\put(170.0,626.0){\rule[-0.200pt]{2.409pt}{0.400pt}}
\put(1429.0,626.0){\rule[-0.200pt]{2.409pt}{0.400pt}}
\put(170.0,626.0){\rule[-0.200pt]{2.409pt}{0.400pt}}
\put(1429.0,626.0){\rule[-0.200pt]{2.409pt}{0.400pt}}
\put(170.0,626.0){\rule[-0.200pt]{2.409pt}{0.400pt}}
\put(1429.0,626.0){\rule[-0.200pt]{2.409pt}{0.400pt}}
\put(170.0,626.0){\rule[-0.200pt]{2.409pt}{0.400pt}}
\put(1429.0,626.0){\rule[-0.200pt]{2.409pt}{0.400pt}}
\put(170.0,626.0){\rule[-0.200pt]{2.409pt}{0.400pt}}
\put(1429.0,626.0){\rule[-0.200pt]{2.409pt}{0.400pt}}
\put(170.0,626.0){\rule[-0.200pt]{2.409pt}{0.400pt}}
\put(1429.0,626.0){\rule[-0.200pt]{2.409pt}{0.400pt}}
\put(170.0,626.0){\rule[-0.200pt]{2.409pt}{0.400pt}}
\put(1429.0,626.0){\rule[-0.200pt]{2.409pt}{0.400pt}}
\put(170.0,626.0){\rule[-0.200pt]{2.409pt}{0.400pt}}
\put(1429.0,626.0){\rule[-0.200pt]{2.409pt}{0.400pt}}
\put(170.0,626.0){\rule[-0.200pt]{2.409pt}{0.400pt}}
\put(1429.0,626.0){\rule[-0.200pt]{2.409pt}{0.400pt}}
\put(170.0,627.0){\rule[-0.200pt]{2.409pt}{0.400pt}}
\put(1429.0,627.0){\rule[-0.200pt]{2.409pt}{0.400pt}}
\put(170.0,627.0){\rule[-0.200pt]{2.409pt}{0.400pt}}
\put(1429.0,627.0){\rule[-0.200pt]{2.409pt}{0.400pt}}
\put(170.0,627.0){\rule[-0.200pt]{2.409pt}{0.400pt}}
\put(1429.0,627.0){\rule[-0.200pt]{2.409pt}{0.400pt}}
\put(170.0,627.0){\rule[-0.200pt]{2.409pt}{0.400pt}}
\put(1429.0,627.0){\rule[-0.200pt]{2.409pt}{0.400pt}}
\put(170.0,627.0){\rule[-0.200pt]{2.409pt}{0.400pt}}
\put(1429.0,627.0){\rule[-0.200pt]{2.409pt}{0.400pt}}
\put(170.0,627.0){\rule[-0.200pt]{2.409pt}{0.400pt}}
\put(1429.0,627.0){\rule[-0.200pt]{2.409pt}{0.400pt}}
\put(170.0,627.0){\rule[-0.200pt]{2.409pt}{0.400pt}}
\put(1429.0,627.0){\rule[-0.200pt]{2.409pt}{0.400pt}}
\put(170.0,627.0){\rule[-0.200pt]{2.409pt}{0.400pt}}
\put(1429.0,627.0){\rule[-0.200pt]{2.409pt}{0.400pt}}
\put(170.0,627.0){\rule[-0.200pt]{2.409pt}{0.400pt}}
\put(1429.0,627.0){\rule[-0.200pt]{2.409pt}{0.400pt}}
\put(170.0,627.0){\rule[-0.200pt]{2.409pt}{0.400pt}}
\put(1429.0,627.0){\rule[-0.200pt]{2.409pt}{0.400pt}}
\put(170.0,627.0){\rule[-0.200pt]{2.409pt}{0.400pt}}
\put(1429.0,627.0){\rule[-0.200pt]{2.409pt}{0.400pt}}
\put(170.0,628.0){\rule[-0.200pt]{2.409pt}{0.400pt}}
\put(1429.0,628.0){\rule[-0.200pt]{2.409pt}{0.400pt}}
\put(170.0,628.0){\rule[-0.200pt]{2.409pt}{0.400pt}}
\put(1429.0,628.0){\rule[-0.200pt]{2.409pt}{0.400pt}}
\put(170.0,628.0){\rule[-0.200pt]{2.409pt}{0.400pt}}
\put(1429.0,628.0){\rule[-0.200pt]{2.409pt}{0.400pt}}
\put(170.0,628.0){\rule[-0.200pt]{2.409pt}{0.400pt}}
\put(1429.0,628.0){\rule[-0.200pt]{2.409pt}{0.400pt}}
\put(170.0,628.0){\rule[-0.200pt]{2.409pt}{0.400pt}}
\put(1429.0,628.0){\rule[-0.200pt]{2.409pt}{0.400pt}}
\put(170.0,628.0){\rule[-0.200pt]{2.409pt}{0.400pt}}
\put(1429.0,628.0){\rule[-0.200pt]{2.409pt}{0.400pt}}
\put(170.0,628.0){\rule[-0.200pt]{2.409pt}{0.400pt}}
\put(1429.0,628.0){\rule[-0.200pt]{2.409pt}{0.400pt}}
\put(170.0,628.0){\rule[-0.200pt]{2.409pt}{0.400pt}}
\put(1429.0,628.0){\rule[-0.200pt]{2.409pt}{0.400pt}}
\put(170.0,628.0){\rule[-0.200pt]{2.409pt}{0.400pt}}
\put(1429.0,628.0){\rule[-0.200pt]{2.409pt}{0.400pt}}
\put(170.0,628.0){\rule[-0.200pt]{2.409pt}{0.400pt}}
\put(1429.0,628.0){\rule[-0.200pt]{2.409pt}{0.400pt}}
\put(170.0,628.0){\rule[-0.200pt]{2.409pt}{0.400pt}}
\put(1429.0,628.0){\rule[-0.200pt]{2.409pt}{0.400pt}}
\put(170.0,628.0){\rule[-0.200pt]{2.409pt}{0.400pt}}
\put(1429.0,628.0){\rule[-0.200pt]{2.409pt}{0.400pt}}
\put(170.0,629.0){\rule[-0.200pt]{2.409pt}{0.400pt}}
\put(1429.0,629.0){\rule[-0.200pt]{2.409pt}{0.400pt}}
\put(170.0,629.0){\rule[-0.200pt]{2.409pt}{0.400pt}}
\put(1429.0,629.0){\rule[-0.200pt]{2.409pt}{0.400pt}}
\put(170.0,629.0){\rule[-0.200pt]{2.409pt}{0.400pt}}
\put(1429.0,629.0){\rule[-0.200pt]{2.409pt}{0.400pt}}
\put(170.0,629.0){\rule[-0.200pt]{2.409pt}{0.400pt}}
\put(1429.0,629.0){\rule[-0.200pt]{2.409pt}{0.400pt}}
\put(170.0,629.0){\rule[-0.200pt]{2.409pt}{0.400pt}}
\put(1429.0,629.0){\rule[-0.200pt]{2.409pt}{0.400pt}}
\put(170.0,629.0){\rule[-0.200pt]{2.409pt}{0.400pt}}
\put(1429.0,629.0){\rule[-0.200pt]{2.409pt}{0.400pt}}
\put(170.0,629.0){\rule[-0.200pt]{2.409pt}{0.400pt}}
\put(1429.0,629.0){\rule[-0.200pt]{2.409pt}{0.400pt}}
\put(170.0,629.0){\rule[-0.200pt]{2.409pt}{0.400pt}}
\put(1429.0,629.0){\rule[-0.200pt]{2.409pt}{0.400pt}}
\put(170.0,629.0){\rule[-0.200pt]{2.409pt}{0.400pt}}
\put(1429.0,629.0){\rule[-0.200pt]{2.409pt}{0.400pt}}
\put(170.0,629.0){\rule[-0.200pt]{2.409pt}{0.400pt}}
\put(1429.0,629.0){\rule[-0.200pt]{2.409pt}{0.400pt}}
\put(170.0,629.0){\rule[-0.200pt]{2.409pt}{0.400pt}}
\put(1429.0,629.0){\rule[-0.200pt]{2.409pt}{0.400pt}}
\put(170.0,629.0){\rule[-0.200pt]{2.409pt}{0.400pt}}
\put(1429.0,629.0){\rule[-0.200pt]{2.409pt}{0.400pt}}
\put(170.0,629.0){\rule[-0.200pt]{2.409pt}{0.400pt}}
\put(1429.0,629.0){\rule[-0.200pt]{2.409pt}{0.400pt}}
\put(170.0,630.0){\rule[-0.200pt]{2.409pt}{0.400pt}}
\put(1429.0,630.0){\rule[-0.200pt]{2.409pt}{0.400pt}}
\put(170.0,630.0){\rule[-0.200pt]{2.409pt}{0.400pt}}
\put(1429.0,630.0){\rule[-0.200pt]{2.409pt}{0.400pt}}
\put(170.0,630.0){\rule[-0.200pt]{2.409pt}{0.400pt}}
\put(1429.0,630.0){\rule[-0.200pt]{2.409pt}{0.400pt}}
\put(170.0,630.0){\rule[-0.200pt]{2.409pt}{0.400pt}}
\put(1429.0,630.0){\rule[-0.200pt]{2.409pt}{0.400pt}}
\put(170.0,630.0){\rule[-0.200pt]{2.409pt}{0.400pt}}
\put(1429.0,630.0){\rule[-0.200pt]{2.409pt}{0.400pt}}
\put(170.0,630.0){\rule[-0.200pt]{2.409pt}{0.400pt}}
\put(1429.0,630.0){\rule[-0.200pt]{2.409pt}{0.400pt}}
\put(170.0,630.0){\rule[-0.200pt]{2.409pt}{0.400pt}}
\put(1429.0,630.0){\rule[-0.200pt]{2.409pt}{0.400pt}}
\put(170.0,630.0){\rule[-0.200pt]{2.409pt}{0.400pt}}
\put(1429.0,630.0){\rule[-0.200pt]{2.409pt}{0.400pt}}
\put(170.0,630.0){\rule[-0.200pt]{2.409pt}{0.400pt}}
\put(1429.0,630.0){\rule[-0.200pt]{2.409pt}{0.400pt}}
\put(170.0,630.0){\rule[-0.200pt]{2.409pt}{0.400pt}}
\put(1429.0,630.0){\rule[-0.200pt]{2.409pt}{0.400pt}}
\put(170.0,630.0){\rule[-0.200pt]{2.409pt}{0.400pt}}
\put(1429.0,630.0){\rule[-0.200pt]{2.409pt}{0.400pt}}
\put(170.0,630.0){\rule[-0.200pt]{2.409pt}{0.400pt}}
\put(1429.0,630.0){\rule[-0.200pt]{2.409pt}{0.400pt}}
\put(170.0,630.0){\rule[-0.200pt]{2.409pt}{0.400pt}}
\put(1429.0,630.0){\rule[-0.200pt]{2.409pt}{0.400pt}}
\put(170.0,631.0){\rule[-0.200pt]{2.409pt}{0.400pt}}
\put(1429.0,631.0){\rule[-0.200pt]{2.409pt}{0.400pt}}
\put(170.0,631.0){\rule[-0.200pt]{2.409pt}{0.400pt}}
\put(1429.0,631.0){\rule[-0.200pt]{2.409pt}{0.400pt}}
\put(170.0,631.0){\rule[-0.200pt]{2.409pt}{0.400pt}}
\put(1429.0,631.0){\rule[-0.200pt]{2.409pt}{0.400pt}}
\put(170.0,631.0){\rule[-0.200pt]{2.409pt}{0.400pt}}
\put(1429.0,631.0){\rule[-0.200pt]{2.409pt}{0.400pt}}
\put(170.0,631.0){\rule[-0.200pt]{2.409pt}{0.400pt}}
\put(1429.0,631.0){\rule[-0.200pt]{2.409pt}{0.400pt}}
\put(170.0,631.0){\rule[-0.200pt]{2.409pt}{0.400pt}}
\put(1429.0,631.0){\rule[-0.200pt]{2.409pt}{0.400pt}}
\put(170.0,631.0){\rule[-0.200pt]{2.409pt}{0.400pt}}
\put(1429.0,631.0){\rule[-0.200pt]{2.409pt}{0.400pt}}
\put(170.0,631.0){\rule[-0.200pt]{2.409pt}{0.400pt}}
\put(1429.0,631.0){\rule[-0.200pt]{2.409pt}{0.400pt}}
\put(170.0,631.0){\rule[-0.200pt]{2.409pt}{0.400pt}}
\put(1429.0,631.0){\rule[-0.200pt]{2.409pt}{0.400pt}}
\put(170.0,631.0){\rule[-0.200pt]{2.409pt}{0.400pt}}
\put(1429.0,631.0){\rule[-0.200pt]{2.409pt}{0.400pt}}
\put(170.0,631.0){\rule[-0.200pt]{2.409pt}{0.400pt}}
\put(1429.0,631.0){\rule[-0.200pt]{2.409pt}{0.400pt}}
\put(170.0,631.0){\rule[-0.200pt]{2.409pt}{0.400pt}}
\put(1429.0,631.0){\rule[-0.200pt]{2.409pt}{0.400pt}}
\put(170.0,631.0){\rule[-0.200pt]{2.409pt}{0.400pt}}
\put(1429.0,631.0){\rule[-0.200pt]{2.409pt}{0.400pt}}
\put(170.0,631.0){\rule[-0.200pt]{2.409pt}{0.400pt}}
\put(1429.0,631.0){\rule[-0.200pt]{2.409pt}{0.400pt}}
\put(170.0,631.0){\rule[-0.200pt]{2.409pt}{0.400pt}}
\put(1429.0,631.0){\rule[-0.200pt]{2.409pt}{0.400pt}}
\put(170.0,632.0){\rule[-0.200pt]{2.409pt}{0.400pt}}
\put(1429.0,632.0){\rule[-0.200pt]{2.409pt}{0.400pt}}
\put(170.0,632.0){\rule[-0.200pt]{2.409pt}{0.400pt}}
\put(1429.0,632.0){\rule[-0.200pt]{2.409pt}{0.400pt}}
\put(170.0,632.0){\rule[-0.200pt]{2.409pt}{0.400pt}}
\put(1429.0,632.0){\rule[-0.200pt]{2.409pt}{0.400pt}}
\put(170.0,632.0){\rule[-0.200pt]{2.409pt}{0.400pt}}
\put(1429.0,632.0){\rule[-0.200pt]{2.409pt}{0.400pt}}
\put(170.0,632.0){\rule[-0.200pt]{2.409pt}{0.400pt}}
\put(1429.0,632.0){\rule[-0.200pt]{2.409pt}{0.400pt}}
\put(170.0,632.0){\rule[-0.200pt]{2.409pt}{0.400pt}}
\put(1429.0,632.0){\rule[-0.200pt]{2.409pt}{0.400pt}}
\put(170.0,632.0){\rule[-0.200pt]{2.409pt}{0.400pt}}
\put(1429.0,632.0){\rule[-0.200pt]{2.409pt}{0.400pt}}
\put(170.0,632.0){\rule[-0.200pt]{2.409pt}{0.400pt}}
\put(1429.0,632.0){\rule[-0.200pt]{2.409pt}{0.400pt}}
\put(170.0,632.0){\rule[-0.200pt]{2.409pt}{0.400pt}}
\put(1429.0,632.0){\rule[-0.200pt]{2.409pt}{0.400pt}}
\put(170.0,632.0){\rule[-0.200pt]{2.409pt}{0.400pt}}
\put(1429.0,632.0){\rule[-0.200pt]{2.409pt}{0.400pt}}
\put(170.0,632.0){\rule[-0.200pt]{2.409pt}{0.400pt}}
\put(1429.0,632.0){\rule[-0.200pt]{2.409pt}{0.400pt}}
\put(170.0,632.0){\rule[-0.200pt]{2.409pt}{0.400pt}}
\put(1429.0,632.0){\rule[-0.200pt]{2.409pt}{0.400pt}}
\put(170.0,632.0){\rule[-0.200pt]{2.409pt}{0.400pt}}
\put(1429.0,632.0){\rule[-0.200pt]{2.409pt}{0.400pt}}
\put(170.0,632.0){\rule[-0.200pt]{2.409pt}{0.400pt}}
\put(1429.0,632.0){\rule[-0.200pt]{2.409pt}{0.400pt}}
\put(170.0,632.0){\rule[-0.200pt]{2.409pt}{0.400pt}}
\put(1429.0,632.0){\rule[-0.200pt]{2.409pt}{0.400pt}}
\put(170.0,633.0){\rule[-0.200pt]{2.409pt}{0.400pt}}
\put(1429.0,633.0){\rule[-0.200pt]{2.409pt}{0.400pt}}
\put(170.0,633.0){\rule[-0.200pt]{2.409pt}{0.400pt}}
\put(1429.0,633.0){\rule[-0.200pt]{2.409pt}{0.400pt}}
\put(170.0,633.0){\rule[-0.200pt]{2.409pt}{0.400pt}}
\put(1429.0,633.0){\rule[-0.200pt]{2.409pt}{0.400pt}}
\put(170.0,633.0){\rule[-0.200pt]{2.409pt}{0.400pt}}
\put(1429.0,633.0){\rule[-0.200pt]{2.409pt}{0.400pt}}
\put(170.0,633.0){\rule[-0.200pt]{2.409pt}{0.400pt}}
\put(1429.0,633.0){\rule[-0.200pt]{2.409pt}{0.400pt}}
\put(170.0,633.0){\rule[-0.200pt]{2.409pt}{0.400pt}}
\put(1429.0,633.0){\rule[-0.200pt]{2.409pt}{0.400pt}}
\put(170.0,633.0){\rule[-0.200pt]{2.409pt}{0.400pt}}
\put(1429.0,633.0){\rule[-0.200pt]{2.409pt}{0.400pt}}
\put(170.0,633.0){\rule[-0.200pt]{2.409pt}{0.400pt}}
\put(1429.0,633.0){\rule[-0.200pt]{2.409pt}{0.400pt}}
\put(170.0,633.0){\rule[-0.200pt]{2.409pt}{0.400pt}}
\put(1429.0,633.0){\rule[-0.200pt]{2.409pt}{0.400pt}}
\put(170.0,633.0){\rule[-0.200pt]{2.409pt}{0.400pt}}
\put(1429.0,633.0){\rule[-0.200pt]{2.409pt}{0.400pt}}
\put(170.0,633.0){\rule[-0.200pt]{2.409pt}{0.400pt}}
\put(1429.0,633.0){\rule[-0.200pt]{2.409pt}{0.400pt}}
\put(170.0,633.0){\rule[-0.200pt]{2.409pt}{0.400pt}}
\put(1429.0,633.0){\rule[-0.200pt]{2.409pt}{0.400pt}}
\put(170.0,633.0){\rule[-0.200pt]{2.409pt}{0.400pt}}
\put(1429.0,633.0){\rule[-0.200pt]{2.409pt}{0.400pt}}
\put(170.0,633.0){\rule[-0.200pt]{2.409pt}{0.400pt}}
\put(1429.0,633.0){\rule[-0.200pt]{2.409pt}{0.400pt}}
\put(170.0,633.0){\rule[-0.200pt]{2.409pt}{0.400pt}}
\put(1429.0,633.0){\rule[-0.200pt]{2.409pt}{0.400pt}}
\put(170.0,633.0){\rule[-0.200pt]{2.409pt}{0.400pt}}
\put(1429.0,633.0){\rule[-0.200pt]{2.409pt}{0.400pt}}
\put(170.0,634.0){\rule[-0.200pt]{2.409pt}{0.400pt}}
\put(1429.0,634.0){\rule[-0.200pt]{2.409pt}{0.400pt}}
\put(170.0,634.0){\rule[-0.200pt]{2.409pt}{0.400pt}}
\put(1429.0,634.0){\rule[-0.200pt]{2.409pt}{0.400pt}}
\put(170.0,634.0){\rule[-0.200pt]{2.409pt}{0.400pt}}
\put(1429.0,634.0){\rule[-0.200pt]{2.409pt}{0.400pt}}
\put(170.0,634.0){\rule[-0.200pt]{2.409pt}{0.400pt}}
\put(1429.0,634.0){\rule[-0.200pt]{2.409pt}{0.400pt}}
\put(170.0,634.0){\rule[-0.200pt]{2.409pt}{0.400pt}}
\put(1429.0,634.0){\rule[-0.200pt]{2.409pt}{0.400pt}}
\put(170.0,634.0){\rule[-0.200pt]{2.409pt}{0.400pt}}
\put(1429.0,634.0){\rule[-0.200pt]{2.409pt}{0.400pt}}
\put(170.0,634.0){\rule[-0.200pt]{2.409pt}{0.400pt}}
\put(1429.0,634.0){\rule[-0.200pt]{2.409pt}{0.400pt}}
\put(170.0,634.0){\rule[-0.200pt]{2.409pt}{0.400pt}}
\put(1429.0,634.0){\rule[-0.200pt]{2.409pt}{0.400pt}}
\put(170.0,634.0){\rule[-0.200pt]{2.409pt}{0.400pt}}
\put(1429.0,634.0){\rule[-0.200pt]{2.409pt}{0.400pt}}
\put(170.0,634.0){\rule[-0.200pt]{2.409pt}{0.400pt}}
\put(1429.0,634.0){\rule[-0.200pt]{2.409pt}{0.400pt}}
\put(170.0,634.0){\rule[-0.200pt]{2.409pt}{0.400pt}}
\put(1429.0,634.0){\rule[-0.200pt]{2.409pt}{0.400pt}}
\put(170.0,634.0){\rule[-0.200pt]{2.409pt}{0.400pt}}
\put(1429.0,634.0){\rule[-0.200pt]{2.409pt}{0.400pt}}
\put(170.0,634.0){\rule[-0.200pt]{2.409pt}{0.400pt}}
\put(1429.0,634.0){\rule[-0.200pt]{2.409pt}{0.400pt}}
\put(170.0,634.0){\rule[-0.200pt]{2.409pt}{0.400pt}}
\put(1429.0,634.0){\rule[-0.200pt]{2.409pt}{0.400pt}}
\put(170.0,634.0){\rule[-0.200pt]{2.409pt}{0.400pt}}
\put(1429.0,634.0){\rule[-0.200pt]{2.409pt}{0.400pt}}
\put(170.0,634.0){\rule[-0.200pt]{2.409pt}{0.400pt}}
\put(1429.0,634.0){\rule[-0.200pt]{2.409pt}{0.400pt}}
\put(170.0,634.0){\rule[-0.200pt]{2.409pt}{0.400pt}}
\put(1429.0,634.0){\rule[-0.200pt]{2.409pt}{0.400pt}}
\put(170.0,635.0){\rule[-0.200pt]{2.409pt}{0.400pt}}
\put(1429.0,635.0){\rule[-0.200pt]{2.409pt}{0.400pt}}
\put(170.0,635.0){\rule[-0.200pt]{2.409pt}{0.400pt}}
\put(1429.0,635.0){\rule[-0.200pt]{2.409pt}{0.400pt}}
\put(170.0,635.0){\rule[-0.200pt]{2.409pt}{0.400pt}}
\put(1429.0,635.0){\rule[-0.200pt]{2.409pt}{0.400pt}}
\put(170.0,635.0){\rule[-0.200pt]{2.409pt}{0.400pt}}
\put(1429.0,635.0){\rule[-0.200pt]{2.409pt}{0.400pt}}
\put(170.0,635.0){\rule[-0.200pt]{2.409pt}{0.400pt}}
\put(1429.0,635.0){\rule[-0.200pt]{2.409pt}{0.400pt}}
\put(170.0,635.0){\rule[-0.200pt]{2.409pt}{0.400pt}}
\put(1429.0,635.0){\rule[-0.200pt]{2.409pt}{0.400pt}}
\put(170.0,635.0){\rule[-0.200pt]{2.409pt}{0.400pt}}
\put(1429.0,635.0){\rule[-0.200pt]{2.409pt}{0.400pt}}
\put(170.0,635.0){\rule[-0.200pt]{2.409pt}{0.400pt}}
\put(1429.0,635.0){\rule[-0.200pt]{2.409pt}{0.400pt}}
\put(170.0,635.0){\rule[-0.200pt]{2.409pt}{0.400pt}}
\put(1429.0,635.0){\rule[-0.200pt]{2.409pt}{0.400pt}}
\put(170.0,635.0){\rule[-0.200pt]{2.409pt}{0.400pt}}
\put(1429.0,635.0){\rule[-0.200pt]{2.409pt}{0.400pt}}
\put(170.0,635.0){\rule[-0.200pt]{2.409pt}{0.400pt}}
\put(1429.0,635.0){\rule[-0.200pt]{2.409pt}{0.400pt}}
\put(170.0,635.0){\rule[-0.200pt]{2.409pt}{0.400pt}}
\put(1429.0,635.0){\rule[-0.200pt]{2.409pt}{0.400pt}}
\put(170.0,635.0){\rule[-0.200pt]{2.409pt}{0.400pt}}
\put(1429.0,635.0){\rule[-0.200pt]{2.409pt}{0.400pt}}
\put(170.0,635.0){\rule[-0.200pt]{2.409pt}{0.400pt}}
\put(1429.0,635.0){\rule[-0.200pt]{2.409pt}{0.400pt}}
\put(170.0,635.0){\rule[-0.200pt]{2.409pt}{0.400pt}}
\put(1429.0,635.0){\rule[-0.200pt]{2.409pt}{0.400pt}}
\put(170.0,635.0){\rule[-0.200pt]{2.409pt}{0.400pt}}
\put(1429.0,635.0){\rule[-0.200pt]{2.409pt}{0.400pt}}
\put(170.0,635.0){\rule[-0.200pt]{2.409pt}{0.400pt}}
\put(1429.0,635.0){\rule[-0.200pt]{2.409pt}{0.400pt}}
\put(170.0,635.0){\rule[-0.200pt]{2.409pt}{0.400pt}}
\put(1429.0,635.0){\rule[-0.200pt]{2.409pt}{0.400pt}}
\put(170.0,636.0){\rule[-0.200pt]{2.409pt}{0.400pt}}
\put(1429.0,636.0){\rule[-0.200pt]{2.409pt}{0.400pt}}
\put(170.0,636.0){\rule[-0.200pt]{2.409pt}{0.400pt}}
\put(1429.0,636.0){\rule[-0.200pt]{2.409pt}{0.400pt}}
\put(170.0,636.0){\rule[-0.200pt]{2.409pt}{0.400pt}}
\put(1429.0,636.0){\rule[-0.200pt]{2.409pt}{0.400pt}}
\put(170.0,636.0){\rule[-0.200pt]{2.409pt}{0.400pt}}
\put(1429.0,636.0){\rule[-0.200pt]{2.409pt}{0.400pt}}
\put(170.0,636.0){\rule[-0.200pt]{2.409pt}{0.400pt}}
\put(1429.0,636.0){\rule[-0.200pt]{2.409pt}{0.400pt}}
\put(170.0,636.0){\rule[-0.200pt]{2.409pt}{0.400pt}}
\put(1429.0,636.0){\rule[-0.200pt]{2.409pt}{0.400pt}}
\put(170.0,636.0){\rule[-0.200pt]{2.409pt}{0.400pt}}
\put(1429.0,636.0){\rule[-0.200pt]{2.409pt}{0.400pt}}
\put(170.0,636.0){\rule[-0.200pt]{2.409pt}{0.400pt}}
\put(1429.0,636.0){\rule[-0.200pt]{2.409pt}{0.400pt}}
\put(170.0,636.0){\rule[-0.200pt]{2.409pt}{0.400pt}}
\put(1429.0,636.0){\rule[-0.200pt]{2.409pt}{0.400pt}}
\put(170.0,636.0){\rule[-0.200pt]{2.409pt}{0.400pt}}
\put(1429.0,636.0){\rule[-0.200pt]{2.409pt}{0.400pt}}
\put(170.0,636.0){\rule[-0.200pt]{2.409pt}{0.400pt}}
\put(1429.0,636.0){\rule[-0.200pt]{2.409pt}{0.400pt}}
\put(170.0,636.0){\rule[-0.200pt]{2.409pt}{0.400pt}}
\put(1429.0,636.0){\rule[-0.200pt]{2.409pt}{0.400pt}}
\put(170.0,636.0){\rule[-0.200pt]{2.409pt}{0.400pt}}
\put(1429.0,636.0){\rule[-0.200pt]{2.409pt}{0.400pt}}
\put(170.0,636.0){\rule[-0.200pt]{2.409pt}{0.400pt}}
\put(1429.0,636.0){\rule[-0.200pt]{2.409pt}{0.400pt}}
\put(170.0,636.0){\rule[-0.200pt]{2.409pt}{0.400pt}}
\put(1429.0,636.0){\rule[-0.200pt]{2.409pt}{0.400pt}}
\put(170.0,636.0){\rule[-0.200pt]{2.409pt}{0.400pt}}
\put(1429.0,636.0){\rule[-0.200pt]{2.409pt}{0.400pt}}
\put(170.0,636.0){\rule[-0.200pt]{2.409pt}{0.400pt}}
\put(1429.0,636.0){\rule[-0.200pt]{2.409pt}{0.400pt}}
\put(170.0,636.0){\rule[-0.200pt]{2.409pt}{0.400pt}}
\put(1429.0,636.0){\rule[-0.200pt]{2.409pt}{0.400pt}}
\put(170.0,636.0){\rule[-0.200pt]{2.409pt}{0.400pt}}
\put(1429.0,636.0){\rule[-0.200pt]{2.409pt}{0.400pt}}
\put(170.0,636.0){\rule[-0.200pt]{2.409pt}{0.400pt}}
\put(1429.0,636.0){\rule[-0.200pt]{2.409pt}{0.400pt}}
\put(170.0,637.0){\rule[-0.200pt]{2.409pt}{0.400pt}}
\put(1429.0,637.0){\rule[-0.200pt]{2.409pt}{0.400pt}}
\put(170.0,637.0){\rule[-0.200pt]{2.409pt}{0.400pt}}
\put(1429.0,637.0){\rule[-0.200pt]{2.409pt}{0.400pt}}
\put(170.0,637.0){\rule[-0.200pt]{2.409pt}{0.400pt}}
\put(1429.0,637.0){\rule[-0.200pt]{2.409pt}{0.400pt}}
\put(170.0,637.0){\rule[-0.200pt]{2.409pt}{0.400pt}}
\put(1429.0,637.0){\rule[-0.200pt]{2.409pt}{0.400pt}}
\put(170.0,637.0){\rule[-0.200pt]{2.409pt}{0.400pt}}
\put(1429.0,637.0){\rule[-0.200pt]{2.409pt}{0.400pt}}
\put(170.0,637.0){\rule[-0.200pt]{2.409pt}{0.400pt}}
\put(1429.0,637.0){\rule[-0.200pt]{2.409pt}{0.400pt}}
\put(170.0,637.0){\rule[-0.200pt]{2.409pt}{0.400pt}}
\put(1429.0,637.0){\rule[-0.200pt]{2.409pt}{0.400pt}}
\put(170.0,637.0){\rule[-0.200pt]{2.409pt}{0.400pt}}
\put(1429.0,637.0){\rule[-0.200pt]{2.409pt}{0.400pt}}
\put(170.0,637.0){\rule[-0.200pt]{2.409pt}{0.400pt}}
\put(1429.0,637.0){\rule[-0.200pt]{2.409pt}{0.400pt}}
\put(170.0,637.0){\rule[-0.200pt]{2.409pt}{0.400pt}}
\put(1429.0,637.0){\rule[-0.200pt]{2.409pt}{0.400pt}}
\put(170.0,637.0){\rule[-0.200pt]{4.818pt}{0.400pt}}
\put(150,637){\makebox(0,0)[r]{ 0.001}}
\put(1419.0,637.0){\rule[-0.200pt]{4.818pt}{0.400pt}}
\put(170.0,648.0){\rule[-0.200pt]{2.409pt}{0.400pt}}
\put(1429.0,648.0){\rule[-0.200pt]{2.409pt}{0.400pt}}
\put(170.0,663.0){\rule[-0.200pt]{2.409pt}{0.400pt}}
\put(1429.0,663.0){\rule[-0.200pt]{2.409pt}{0.400pt}}
\put(170.0,670.0){\rule[-0.200pt]{2.409pt}{0.400pt}}
\put(1429.0,670.0){\rule[-0.200pt]{2.409pt}{0.400pt}}
\put(170.0,676.0){\rule[-0.200pt]{2.409pt}{0.400pt}}
\put(1429.0,676.0){\rule[-0.200pt]{2.409pt}{0.400pt}}
\put(170.0,679.0){\rule[-0.200pt]{2.409pt}{0.400pt}}
\put(1429.0,679.0){\rule[-0.200pt]{2.409pt}{0.400pt}}
\put(170.0,683.0){\rule[-0.200pt]{2.409pt}{0.400pt}}
\put(1429.0,683.0){\rule[-0.200pt]{2.409pt}{0.400pt}}
\put(170.0,685.0){\rule[-0.200pt]{2.409pt}{0.400pt}}
\put(1429.0,685.0){\rule[-0.200pt]{2.409pt}{0.400pt}}
\put(170.0,687.0){\rule[-0.200pt]{2.409pt}{0.400pt}}
\put(1429.0,687.0){\rule[-0.200pt]{2.409pt}{0.400pt}}
\put(170.0,689.0){\rule[-0.200pt]{2.409pt}{0.400pt}}
\put(1429.0,689.0){\rule[-0.200pt]{2.409pt}{0.400pt}}
\put(170.0,691.0){\rule[-0.200pt]{2.409pt}{0.400pt}}
\put(1429.0,691.0){\rule[-0.200pt]{2.409pt}{0.400pt}}
\put(170.0,693.0){\rule[-0.200pt]{2.409pt}{0.400pt}}
\put(1429.0,693.0){\rule[-0.200pt]{2.409pt}{0.400pt}}
\put(170.0,694.0){\rule[-0.200pt]{2.409pt}{0.400pt}}
\put(1429.0,694.0){\rule[-0.200pt]{2.409pt}{0.400pt}}
\put(170.0,695.0){\rule[-0.200pt]{2.409pt}{0.400pt}}
\put(1429.0,695.0){\rule[-0.200pt]{2.409pt}{0.400pt}}
\put(170.0,697.0){\rule[-0.200pt]{2.409pt}{0.400pt}}
\put(1429.0,697.0){\rule[-0.200pt]{2.409pt}{0.400pt}}
\put(170.0,698.0){\rule[-0.200pt]{2.409pt}{0.400pt}}
\put(1429.0,698.0){\rule[-0.200pt]{2.409pt}{0.400pt}}
\put(170.0,699.0){\rule[-0.200pt]{2.409pt}{0.400pt}}
\put(1429.0,699.0){\rule[-0.200pt]{2.409pt}{0.400pt}}
\put(170.0,700.0){\rule[-0.200pt]{2.409pt}{0.400pt}}
\put(1429.0,700.0){\rule[-0.200pt]{2.409pt}{0.400pt}}
\put(170.0,701.0){\rule[-0.200pt]{2.409pt}{0.400pt}}
\put(1429.0,701.0){\rule[-0.200pt]{2.409pt}{0.400pt}}
\put(170.0,702.0){\rule[-0.200pt]{2.409pt}{0.400pt}}
\put(1429.0,702.0){\rule[-0.200pt]{2.409pt}{0.400pt}}
\put(170.0,703.0){\rule[-0.200pt]{2.409pt}{0.400pt}}
\put(1429.0,703.0){\rule[-0.200pt]{2.409pt}{0.400pt}}
\put(170.0,703.0){\rule[-0.200pt]{2.409pt}{0.400pt}}
\put(1429.0,703.0){\rule[-0.200pt]{2.409pt}{0.400pt}}
\put(170.0,704.0){\rule[-0.200pt]{2.409pt}{0.400pt}}
\put(1429.0,704.0){\rule[-0.200pt]{2.409pt}{0.400pt}}
\put(170.0,705.0){\rule[-0.200pt]{2.409pt}{0.400pt}}
\put(1429.0,705.0){\rule[-0.200pt]{2.409pt}{0.400pt}}
\put(170.0,705.0){\rule[-0.200pt]{2.409pt}{0.400pt}}
\put(1429.0,705.0){\rule[-0.200pt]{2.409pt}{0.400pt}}
\put(170.0,706.0){\rule[-0.200pt]{2.409pt}{0.400pt}}
\put(1429.0,706.0){\rule[-0.200pt]{2.409pt}{0.400pt}}
\put(170.0,707.0){\rule[-0.200pt]{2.409pt}{0.400pt}}
\put(1429.0,707.0){\rule[-0.200pt]{2.409pt}{0.400pt}}
\put(170.0,707.0){\rule[-0.200pt]{2.409pt}{0.400pt}}
\put(1429.0,707.0){\rule[-0.200pt]{2.409pt}{0.400pt}}
\put(170.0,708.0){\rule[-0.200pt]{2.409pt}{0.400pt}}
\put(1429.0,708.0){\rule[-0.200pt]{2.409pt}{0.400pt}}
\put(170.0,709.0){\rule[-0.200pt]{2.409pt}{0.400pt}}
\put(1429.0,709.0){\rule[-0.200pt]{2.409pt}{0.400pt}}
\put(170.0,709.0){\rule[-0.200pt]{2.409pt}{0.400pt}}
\put(1429.0,709.0){\rule[-0.200pt]{2.409pt}{0.400pt}}
\put(170.0,710.0){\rule[-0.200pt]{2.409pt}{0.400pt}}
\put(1429.0,710.0){\rule[-0.200pt]{2.409pt}{0.400pt}}
\put(170.0,710.0){\rule[-0.200pt]{2.409pt}{0.400pt}}
\put(1429.0,710.0){\rule[-0.200pt]{2.409pt}{0.400pt}}
\put(170.0,711.0){\rule[-0.200pt]{2.409pt}{0.400pt}}
\put(1429.0,711.0){\rule[-0.200pt]{2.409pt}{0.400pt}}
\put(170.0,711.0){\rule[-0.200pt]{2.409pt}{0.400pt}}
\put(1429.0,711.0){\rule[-0.200pt]{2.409pt}{0.400pt}}
\put(170.0,712.0){\rule[-0.200pt]{2.409pt}{0.400pt}}
\put(1429.0,712.0){\rule[-0.200pt]{2.409pt}{0.400pt}}
\put(170.0,712.0){\rule[-0.200pt]{2.409pt}{0.400pt}}
\put(1429.0,712.0){\rule[-0.200pt]{2.409pt}{0.400pt}}
\put(170.0,713.0){\rule[-0.200pt]{2.409pt}{0.400pt}}
\put(1429.0,713.0){\rule[-0.200pt]{2.409pt}{0.400pt}}
\put(170.0,713.0){\rule[-0.200pt]{2.409pt}{0.400pt}}
\put(1429.0,713.0){\rule[-0.200pt]{2.409pt}{0.400pt}}
\put(170.0,713.0){\rule[-0.200pt]{2.409pt}{0.400pt}}
\put(1429.0,713.0){\rule[-0.200pt]{2.409pt}{0.400pt}}
\put(170.0,714.0){\rule[-0.200pt]{2.409pt}{0.400pt}}
\put(1429.0,714.0){\rule[-0.200pt]{2.409pt}{0.400pt}}
\put(170.0,714.0){\rule[-0.200pt]{2.409pt}{0.400pt}}
\put(1429.0,714.0){\rule[-0.200pt]{2.409pt}{0.400pt}}
\put(170.0,715.0){\rule[-0.200pt]{2.409pt}{0.400pt}}
\put(1429.0,715.0){\rule[-0.200pt]{2.409pt}{0.400pt}}
\put(170.0,715.0){\rule[-0.200pt]{2.409pt}{0.400pt}}
\put(1429.0,715.0){\rule[-0.200pt]{2.409pt}{0.400pt}}
\put(170.0,715.0){\rule[-0.200pt]{2.409pt}{0.400pt}}
\put(1429.0,715.0){\rule[-0.200pt]{2.409pt}{0.400pt}}
\put(170.0,716.0){\rule[-0.200pt]{2.409pt}{0.400pt}}
\put(1429.0,716.0){\rule[-0.200pt]{2.409pt}{0.400pt}}
\put(170.0,716.0){\rule[-0.200pt]{2.409pt}{0.400pt}}
\put(1429.0,716.0){\rule[-0.200pt]{2.409pt}{0.400pt}}
\put(170.0,716.0){\rule[-0.200pt]{2.409pt}{0.400pt}}
\put(1429.0,716.0){\rule[-0.200pt]{2.409pt}{0.400pt}}
\put(170.0,717.0){\rule[-0.200pt]{2.409pt}{0.400pt}}
\put(1429.0,717.0){\rule[-0.200pt]{2.409pt}{0.400pt}}
\put(170.0,717.0){\rule[-0.200pt]{2.409pt}{0.400pt}}
\put(1429.0,717.0){\rule[-0.200pt]{2.409pt}{0.400pt}}
\put(170.0,717.0){\rule[-0.200pt]{2.409pt}{0.400pt}}
\put(1429.0,717.0){\rule[-0.200pt]{2.409pt}{0.400pt}}
\put(170.0,718.0){\rule[-0.200pt]{2.409pt}{0.400pt}}
\put(1429.0,718.0){\rule[-0.200pt]{2.409pt}{0.400pt}}
\put(170.0,718.0){\rule[-0.200pt]{2.409pt}{0.400pt}}
\put(1429.0,718.0){\rule[-0.200pt]{2.409pt}{0.400pt}}
\put(170.0,718.0){\rule[-0.200pt]{2.409pt}{0.400pt}}
\put(1429.0,718.0){\rule[-0.200pt]{2.409pt}{0.400pt}}
\put(170.0,719.0){\rule[-0.200pt]{2.409pt}{0.400pt}}
\put(1429.0,719.0){\rule[-0.200pt]{2.409pt}{0.400pt}}
\put(170.0,719.0){\rule[-0.200pt]{2.409pt}{0.400pt}}
\put(1429.0,719.0){\rule[-0.200pt]{2.409pt}{0.400pt}}
\put(170.0,719.0){\rule[-0.200pt]{2.409pt}{0.400pt}}
\put(1429.0,719.0){\rule[-0.200pt]{2.409pt}{0.400pt}}
\put(170.0,720.0){\rule[-0.200pt]{2.409pt}{0.400pt}}
\put(1429.0,720.0){\rule[-0.200pt]{2.409pt}{0.400pt}}
\put(170.0,720.0){\rule[-0.200pt]{2.409pt}{0.400pt}}
\put(1429.0,720.0){\rule[-0.200pt]{2.409pt}{0.400pt}}
\put(170.0,720.0){\rule[-0.200pt]{2.409pt}{0.400pt}}
\put(1429.0,720.0){\rule[-0.200pt]{2.409pt}{0.400pt}}
\put(170.0,720.0){\rule[-0.200pt]{2.409pt}{0.400pt}}
\put(1429.0,720.0){\rule[-0.200pt]{2.409pt}{0.400pt}}
\put(170.0,721.0){\rule[-0.200pt]{2.409pt}{0.400pt}}
\put(1429.0,721.0){\rule[-0.200pt]{2.409pt}{0.400pt}}
\put(170.0,721.0){\rule[-0.200pt]{2.409pt}{0.400pt}}
\put(1429.0,721.0){\rule[-0.200pt]{2.409pt}{0.400pt}}
\put(170.0,721.0){\rule[-0.200pt]{2.409pt}{0.400pt}}
\put(1429.0,721.0){\rule[-0.200pt]{2.409pt}{0.400pt}}
\put(170.0,721.0){\rule[-0.200pt]{2.409pt}{0.400pt}}
\put(1429.0,721.0){\rule[-0.200pt]{2.409pt}{0.400pt}}
\put(170.0,722.0){\rule[-0.200pt]{2.409pt}{0.400pt}}
\put(1429.0,722.0){\rule[-0.200pt]{2.409pt}{0.400pt}}
\put(170.0,722.0){\rule[-0.200pt]{2.409pt}{0.400pt}}
\put(1429.0,722.0){\rule[-0.200pt]{2.409pt}{0.400pt}}
\put(170.0,722.0){\rule[-0.200pt]{2.409pt}{0.400pt}}
\put(1429.0,722.0){\rule[-0.200pt]{2.409pt}{0.400pt}}
\put(170.0,722.0){\rule[-0.200pt]{2.409pt}{0.400pt}}
\put(1429.0,722.0){\rule[-0.200pt]{2.409pt}{0.400pt}}
\put(170.0,723.0){\rule[-0.200pt]{2.409pt}{0.400pt}}
\put(1429.0,723.0){\rule[-0.200pt]{2.409pt}{0.400pt}}
\put(170.0,723.0){\rule[-0.200pt]{2.409pt}{0.400pt}}
\put(1429.0,723.0){\rule[-0.200pt]{2.409pt}{0.400pt}}
\put(170.0,723.0){\rule[-0.200pt]{2.409pt}{0.400pt}}
\put(1429.0,723.0){\rule[-0.200pt]{2.409pt}{0.400pt}}
\put(170.0,723.0){\rule[-0.200pt]{2.409pt}{0.400pt}}
\put(1429.0,723.0){\rule[-0.200pt]{2.409pt}{0.400pt}}
\put(170.0,724.0){\rule[-0.200pt]{2.409pt}{0.400pt}}
\put(1429.0,724.0){\rule[-0.200pt]{2.409pt}{0.400pt}}
\put(170.0,724.0){\rule[-0.200pt]{2.409pt}{0.400pt}}
\put(1429.0,724.0){\rule[-0.200pt]{2.409pt}{0.400pt}}
\put(170.0,724.0){\rule[-0.200pt]{2.409pt}{0.400pt}}
\put(1429.0,724.0){\rule[-0.200pt]{2.409pt}{0.400pt}}
\put(170.0,724.0){\rule[-0.200pt]{2.409pt}{0.400pt}}
\put(1429.0,724.0){\rule[-0.200pt]{2.409pt}{0.400pt}}
\put(170.0,724.0){\rule[-0.200pt]{2.409pt}{0.400pt}}
\put(1429.0,724.0){\rule[-0.200pt]{2.409pt}{0.400pt}}
\put(170.0,725.0){\rule[-0.200pt]{2.409pt}{0.400pt}}
\put(1429.0,725.0){\rule[-0.200pt]{2.409pt}{0.400pt}}
\put(170.0,725.0){\rule[-0.200pt]{2.409pt}{0.400pt}}
\put(1429.0,725.0){\rule[-0.200pt]{2.409pt}{0.400pt}}
\put(170.0,725.0){\rule[-0.200pt]{2.409pt}{0.400pt}}
\put(1429.0,725.0){\rule[-0.200pt]{2.409pt}{0.400pt}}
\put(170.0,725.0){\rule[-0.200pt]{2.409pt}{0.400pt}}
\put(1429.0,725.0){\rule[-0.200pt]{2.409pt}{0.400pt}}
\put(170.0,725.0){\rule[-0.200pt]{2.409pt}{0.400pt}}
\put(1429.0,725.0){\rule[-0.200pt]{2.409pt}{0.400pt}}
\put(170.0,726.0){\rule[-0.200pt]{2.409pt}{0.400pt}}
\put(1429.0,726.0){\rule[-0.200pt]{2.409pt}{0.400pt}}
\put(170.0,726.0){\rule[-0.200pt]{2.409pt}{0.400pt}}
\put(1429.0,726.0){\rule[-0.200pt]{2.409pt}{0.400pt}}
\put(170.0,726.0){\rule[-0.200pt]{2.409pt}{0.400pt}}
\put(1429.0,726.0){\rule[-0.200pt]{2.409pt}{0.400pt}}
\put(170.0,726.0){\rule[-0.200pt]{2.409pt}{0.400pt}}
\put(1429.0,726.0){\rule[-0.200pt]{2.409pt}{0.400pt}}
\put(170.0,726.0){\rule[-0.200pt]{2.409pt}{0.400pt}}
\put(1429.0,726.0){\rule[-0.200pt]{2.409pt}{0.400pt}}
\put(170.0,727.0){\rule[-0.200pt]{2.409pt}{0.400pt}}
\put(1429.0,727.0){\rule[-0.200pt]{2.409pt}{0.400pt}}
\put(170.0,727.0){\rule[-0.200pt]{2.409pt}{0.400pt}}
\put(1429.0,727.0){\rule[-0.200pt]{2.409pt}{0.400pt}}
\put(170.0,727.0){\rule[-0.200pt]{2.409pt}{0.400pt}}
\put(1429.0,727.0){\rule[-0.200pt]{2.409pt}{0.400pt}}
\put(170.0,727.0){\rule[-0.200pt]{2.409pt}{0.400pt}}
\put(1429.0,727.0){\rule[-0.200pt]{2.409pt}{0.400pt}}
\put(170.0,727.0){\rule[-0.200pt]{2.409pt}{0.400pt}}
\put(1429.0,727.0){\rule[-0.200pt]{2.409pt}{0.400pt}}
\put(170.0,727.0){\rule[-0.200pt]{2.409pt}{0.400pt}}
\put(1429.0,727.0){\rule[-0.200pt]{2.409pt}{0.400pt}}
\put(170.0,728.0){\rule[-0.200pt]{2.409pt}{0.400pt}}
\put(1429.0,728.0){\rule[-0.200pt]{2.409pt}{0.400pt}}
\put(170.0,728.0){\rule[-0.200pt]{2.409pt}{0.400pt}}
\put(1429.0,728.0){\rule[-0.200pt]{2.409pt}{0.400pt}}
\put(170.0,728.0){\rule[-0.200pt]{2.409pt}{0.400pt}}
\put(1429.0,728.0){\rule[-0.200pt]{2.409pt}{0.400pt}}
\put(170.0,728.0){\rule[-0.200pt]{2.409pt}{0.400pt}}
\put(1429.0,728.0){\rule[-0.200pt]{2.409pt}{0.400pt}}
\put(170.0,728.0){\rule[-0.200pt]{2.409pt}{0.400pt}}
\put(1429.0,728.0){\rule[-0.200pt]{2.409pt}{0.400pt}}
\put(170.0,728.0){\rule[-0.200pt]{2.409pt}{0.400pt}}
\put(1429.0,728.0){\rule[-0.200pt]{2.409pt}{0.400pt}}
\put(170.0,729.0){\rule[-0.200pt]{2.409pt}{0.400pt}}
\put(1429.0,729.0){\rule[-0.200pt]{2.409pt}{0.400pt}}
\put(170.0,729.0){\rule[-0.200pt]{2.409pt}{0.400pt}}
\put(1429.0,729.0){\rule[-0.200pt]{2.409pt}{0.400pt}}
\put(170.0,729.0){\rule[-0.200pt]{2.409pt}{0.400pt}}
\put(1429.0,729.0){\rule[-0.200pt]{2.409pt}{0.400pt}}
\put(170.0,729.0){\rule[-0.200pt]{2.409pt}{0.400pt}}
\put(1429.0,729.0){\rule[-0.200pt]{2.409pt}{0.400pt}}
\put(170.0,729.0){\rule[-0.200pt]{2.409pt}{0.400pt}}
\put(1429.0,729.0){\rule[-0.200pt]{2.409pt}{0.400pt}}
\put(170.0,729.0){\rule[-0.200pt]{2.409pt}{0.400pt}}
\put(1429.0,729.0){\rule[-0.200pt]{2.409pt}{0.400pt}}
\put(170.0,730.0){\rule[-0.200pt]{2.409pt}{0.400pt}}
\put(1429.0,730.0){\rule[-0.200pt]{2.409pt}{0.400pt}}
\put(170.0,730.0){\rule[-0.200pt]{2.409pt}{0.400pt}}
\put(1429.0,730.0){\rule[-0.200pt]{2.409pt}{0.400pt}}
\put(170.0,730.0){\rule[-0.200pt]{2.409pt}{0.400pt}}
\put(1429.0,730.0){\rule[-0.200pt]{2.409pt}{0.400pt}}
\put(170.0,730.0){\rule[-0.200pt]{2.409pt}{0.400pt}}
\put(1429.0,730.0){\rule[-0.200pt]{2.409pt}{0.400pt}}
\put(170.0,730.0){\rule[-0.200pt]{2.409pt}{0.400pt}}
\put(1429.0,730.0){\rule[-0.200pt]{2.409pt}{0.400pt}}
\put(170.0,730.0){\rule[-0.200pt]{2.409pt}{0.400pt}}
\put(1429.0,730.0){\rule[-0.200pt]{2.409pt}{0.400pt}}
\put(170.0,730.0){\rule[-0.200pt]{2.409pt}{0.400pt}}
\put(1429.0,730.0){\rule[-0.200pt]{2.409pt}{0.400pt}}
\put(170.0,731.0){\rule[-0.200pt]{2.409pt}{0.400pt}}
\put(1429.0,731.0){\rule[-0.200pt]{2.409pt}{0.400pt}}
\put(170.0,731.0){\rule[-0.200pt]{2.409pt}{0.400pt}}
\put(1429.0,731.0){\rule[-0.200pt]{2.409pt}{0.400pt}}
\put(170.0,731.0){\rule[-0.200pt]{2.409pt}{0.400pt}}
\put(1429.0,731.0){\rule[-0.200pt]{2.409pt}{0.400pt}}
\put(170.0,731.0){\rule[-0.200pt]{2.409pt}{0.400pt}}
\put(1429.0,731.0){\rule[-0.200pt]{2.409pt}{0.400pt}}
\put(170.0,731.0){\rule[-0.200pt]{2.409pt}{0.400pt}}
\put(1429.0,731.0){\rule[-0.200pt]{2.409pt}{0.400pt}}
\put(170.0,731.0){\rule[-0.200pt]{2.409pt}{0.400pt}}
\put(1429.0,731.0){\rule[-0.200pt]{2.409pt}{0.400pt}}
\put(170.0,731.0){\rule[-0.200pt]{2.409pt}{0.400pt}}
\put(1429.0,731.0){\rule[-0.200pt]{2.409pt}{0.400pt}}
\put(170.0,732.0){\rule[-0.200pt]{2.409pt}{0.400pt}}
\put(1429.0,732.0){\rule[-0.200pt]{2.409pt}{0.400pt}}
\put(170.0,732.0){\rule[-0.200pt]{2.409pt}{0.400pt}}
\put(1429.0,732.0){\rule[-0.200pt]{2.409pt}{0.400pt}}
\put(170.0,732.0){\rule[-0.200pt]{2.409pt}{0.400pt}}
\put(1429.0,732.0){\rule[-0.200pt]{2.409pt}{0.400pt}}
\put(170.0,732.0){\rule[-0.200pt]{2.409pt}{0.400pt}}
\put(1429.0,732.0){\rule[-0.200pt]{2.409pt}{0.400pt}}
\put(170.0,732.0){\rule[-0.200pt]{2.409pt}{0.400pt}}
\put(1429.0,732.0){\rule[-0.200pt]{2.409pt}{0.400pt}}
\put(170.0,732.0){\rule[-0.200pt]{2.409pt}{0.400pt}}
\put(1429.0,732.0){\rule[-0.200pt]{2.409pt}{0.400pt}}
\put(170.0,732.0){\rule[-0.200pt]{2.409pt}{0.400pt}}
\put(1429.0,732.0){\rule[-0.200pt]{2.409pt}{0.400pt}}
\put(170.0,732.0){\rule[-0.200pt]{2.409pt}{0.400pt}}
\put(1429.0,732.0){\rule[-0.200pt]{2.409pt}{0.400pt}}
\put(170.0,733.0){\rule[-0.200pt]{2.409pt}{0.400pt}}
\put(1429.0,733.0){\rule[-0.200pt]{2.409pt}{0.400pt}}
\put(170.0,733.0){\rule[-0.200pt]{2.409pt}{0.400pt}}
\put(1429.0,733.0){\rule[-0.200pt]{2.409pt}{0.400pt}}
\put(170.0,733.0){\rule[-0.200pt]{2.409pt}{0.400pt}}
\put(1429.0,733.0){\rule[-0.200pt]{2.409pt}{0.400pt}}
\put(170.0,733.0){\rule[-0.200pt]{2.409pt}{0.400pt}}
\put(1429.0,733.0){\rule[-0.200pt]{2.409pt}{0.400pt}}
\put(170.0,733.0){\rule[-0.200pt]{2.409pt}{0.400pt}}
\put(1429.0,733.0){\rule[-0.200pt]{2.409pt}{0.400pt}}
\put(170.0,733.0){\rule[-0.200pt]{2.409pt}{0.400pt}}
\put(1429.0,733.0){\rule[-0.200pt]{2.409pt}{0.400pt}}
\put(170.0,733.0){\rule[-0.200pt]{2.409pt}{0.400pt}}
\put(1429.0,733.0){\rule[-0.200pt]{2.409pt}{0.400pt}}
\put(170.0,733.0){\rule[-0.200pt]{2.409pt}{0.400pt}}
\put(1429.0,733.0){\rule[-0.200pt]{2.409pt}{0.400pt}}
\put(170.0,734.0){\rule[-0.200pt]{2.409pt}{0.400pt}}
\put(1429.0,734.0){\rule[-0.200pt]{2.409pt}{0.400pt}}
\put(170.0,734.0){\rule[-0.200pt]{2.409pt}{0.400pt}}
\put(1429.0,734.0){\rule[-0.200pt]{2.409pt}{0.400pt}}
\put(170.0,734.0){\rule[-0.200pt]{2.409pt}{0.400pt}}
\put(1429.0,734.0){\rule[-0.200pt]{2.409pt}{0.400pt}}
\put(170.0,734.0){\rule[-0.200pt]{2.409pt}{0.400pt}}
\put(1429.0,734.0){\rule[-0.200pt]{2.409pt}{0.400pt}}
\put(170.0,734.0){\rule[-0.200pt]{2.409pt}{0.400pt}}
\put(1429.0,734.0){\rule[-0.200pt]{2.409pt}{0.400pt}}
\put(170.0,734.0){\rule[-0.200pt]{2.409pt}{0.400pt}}
\put(1429.0,734.0){\rule[-0.200pt]{2.409pt}{0.400pt}}
\put(170.0,734.0){\rule[-0.200pt]{2.409pt}{0.400pt}}
\put(1429.0,734.0){\rule[-0.200pt]{2.409pt}{0.400pt}}
\put(170.0,734.0){\rule[-0.200pt]{2.409pt}{0.400pt}}
\put(1429.0,734.0){\rule[-0.200pt]{2.409pt}{0.400pt}}
\put(170.0,734.0){\rule[-0.200pt]{2.409pt}{0.400pt}}
\put(1429.0,734.0){\rule[-0.200pt]{2.409pt}{0.400pt}}
\put(170.0,735.0){\rule[-0.200pt]{2.409pt}{0.400pt}}
\put(1429.0,735.0){\rule[-0.200pt]{2.409pt}{0.400pt}}
\put(170.0,735.0){\rule[-0.200pt]{2.409pt}{0.400pt}}
\put(1429.0,735.0){\rule[-0.200pt]{2.409pt}{0.400pt}}
\put(170.0,735.0){\rule[-0.200pt]{2.409pt}{0.400pt}}
\put(1429.0,735.0){\rule[-0.200pt]{2.409pt}{0.400pt}}
\put(170.0,735.0){\rule[-0.200pt]{2.409pt}{0.400pt}}
\put(1429.0,735.0){\rule[-0.200pt]{2.409pt}{0.400pt}}
\put(170.0,735.0){\rule[-0.200pt]{2.409pt}{0.400pt}}
\put(1429.0,735.0){\rule[-0.200pt]{2.409pt}{0.400pt}}
\put(170.0,735.0){\rule[-0.200pt]{2.409pt}{0.400pt}}
\put(1429.0,735.0){\rule[-0.200pt]{2.409pt}{0.400pt}}
\put(170.0,735.0){\rule[-0.200pt]{2.409pt}{0.400pt}}
\put(1429.0,735.0){\rule[-0.200pt]{2.409pt}{0.400pt}}
\put(170.0,735.0){\rule[-0.200pt]{2.409pt}{0.400pt}}
\put(1429.0,735.0){\rule[-0.200pt]{2.409pt}{0.400pt}}
\put(170.0,735.0){\rule[-0.200pt]{2.409pt}{0.400pt}}
\put(1429.0,735.0){\rule[-0.200pt]{2.409pt}{0.400pt}}
\put(170.0,736.0){\rule[-0.200pt]{2.409pt}{0.400pt}}
\put(1429.0,736.0){\rule[-0.200pt]{2.409pt}{0.400pt}}
\put(170.0,736.0){\rule[-0.200pt]{2.409pt}{0.400pt}}
\put(1429.0,736.0){\rule[-0.200pt]{2.409pt}{0.400pt}}
\put(170.0,736.0){\rule[-0.200pt]{2.409pt}{0.400pt}}
\put(1429.0,736.0){\rule[-0.200pt]{2.409pt}{0.400pt}}
\put(170.0,736.0){\rule[-0.200pt]{2.409pt}{0.400pt}}
\put(1429.0,736.0){\rule[-0.200pt]{2.409pt}{0.400pt}}
\put(170.0,736.0){\rule[-0.200pt]{2.409pt}{0.400pt}}
\put(1429.0,736.0){\rule[-0.200pt]{2.409pt}{0.400pt}}
\put(170.0,736.0){\rule[-0.200pt]{2.409pt}{0.400pt}}
\put(1429.0,736.0){\rule[-0.200pt]{2.409pt}{0.400pt}}
\put(170.0,736.0){\rule[-0.200pt]{2.409pt}{0.400pt}}
\put(1429.0,736.0){\rule[-0.200pt]{2.409pt}{0.400pt}}
\put(170.0,736.0){\rule[-0.200pt]{2.409pt}{0.400pt}}
\put(1429.0,736.0){\rule[-0.200pt]{2.409pt}{0.400pt}}
\put(170.0,736.0){\rule[-0.200pt]{2.409pt}{0.400pt}}
\put(1429.0,736.0){\rule[-0.200pt]{2.409pt}{0.400pt}}
\put(170.0,736.0){\rule[-0.200pt]{2.409pt}{0.400pt}}
\put(1429.0,736.0){\rule[-0.200pt]{2.409pt}{0.400pt}}
\put(170.0,737.0){\rule[-0.200pt]{2.409pt}{0.400pt}}
\put(1429.0,737.0){\rule[-0.200pt]{2.409pt}{0.400pt}}
\put(170.0,737.0){\rule[-0.200pt]{2.409pt}{0.400pt}}
\put(1429.0,737.0){\rule[-0.200pt]{2.409pt}{0.400pt}}
\put(170.0,737.0){\rule[-0.200pt]{2.409pt}{0.400pt}}
\put(1429.0,737.0){\rule[-0.200pt]{2.409pt}{0.400pt}}
\put(170.0,737.0){\rule[-0.200pt]{2.409pt}{0.400pt}}
\put(1429.0,737.0){\rule[-0.200pt]{2.409pt}{0.400pt}}
\put(170.0,737.0){\rule[-0.200pt]{2.409pt}{0.400pt}}
\put(1429.0,737.0){\rule[-0.200pt]{2.409pt}{0.400pt}}
\put(170.0,737.0){\rule[-0.200pt]{2.409pt}{0.400pt}}
\put(1429.0,737.0){\rule[-0.200pt]{2.409pt}{0.400pt}}
\put(170.0,737.0){\rule[-0.200pt]{2.409pt}{0.400pt}}
\put(1429.0,737.0){\rule[-0.200pt]{2.409pt}{0.400pt}}
\put(170.0,737.0){\rule[-0.200pt]{2.409pt}{0.400pt}}
\put(1429.0,737.0){\rule[-0.200pt]{2.409pt}{0.400pt}}
\put(170.0,737.0){\rule[-0.200pt]{2.409pt}{0.400pt}}
\put(1429.0,737.0){\rule[-0.200pt]{2.409pt}{0.400pt}}
\put(170.0,737.0){\rule[-0.200pt]{2.409pt}{0.400pt}}
\put(1429.0,737.0){\rule[-0.200pt]{2.409pt}{0.400pt}}
\put(170.0,738.0){\rule[-0.200pt]{2.409pt}{0.400pt}}
\put(1429.0,738.0){\rule[-0.200pt]{2.409pt}{0.400pt}}
\put(170.0,738.0){\rule[-0.200pt]{2.409pt}{0.400pt}}
\put(1429.0,738.0){\rule[-0.200pt]{2.409pt}{0.400pt}}
\put(170.0,738.0){\rule[-0.200pt]{2.409pt}{0.400pt}}
\put(1429.0,738.0){\rule[-0.200pt]{2.409pt}{0.400pt}}
\put(170.0,738.0){\rule[-0.200pt]{2.409pt}{0.400pt}}
\put(1429.0,738.0){\rule[-0.200pt]{2.409pt}{0.400pt}}
\put(170.0,738.0){\rule[-0.200pt]{2.409pt}{0.400pt}}
\put(1429.0,738.0){\rule[-0.200pt]{2.409pt}{0.400pt}}
\put(170.0,738.0){\rule[-0.200pt]{2.409pt}{0.400pt}}
\put(1429.0,738.0){\rule[-0.200pt]{2.409pt}{0.400pt}}
\put(170.0,738.0){\rule[-0.200pt]{2.409pt}{0.400pt}}
\put(1429.0,738.0){\rule[-0.200pt]{2.409pt}{0.400pt}}
\put(170.0,738.0){\rule[-0.200pt]{2.409pt}{0.400pt}}
\put(1429.0,738.0){\rule[-0.200pt]{2.409pt}{0.400pt}}
\put(170.0,738.0){\rule[-0.200pt]{2.409pt}{0.400pt}}
\put(1429.0,738.0){\rule[-0.200pt]{2.409pt}{0.400pt}}
\put(170.0,738.0){\rule[-0.200pt]{2.409pt}{0.400pt}}
\put(1429.0,738.0){\rule[-0.200pt]{2.409pt}{0.400pt}}
\put(170.0,738.0){\rule[-0.200pt]{2.409pt}{0.400pt}}
\put(1429.0,738.0){\rule[-0.200pt]{2.409pt}{0.400pt}}
\put(170.0,739.0){\rule[-0.200pt]{2.409pt}{0.400pt}}
\put(1429.0,739.0){\rule[-0.200pt]{2.409pt}{0.400pt}}
\put(170.0,739.0){\rule[-0.200pt]{2.409pt}{0.400pt}}
\put(1429.0,739.0){\rule[-0.200pt]{2.409pt}{0.400pt}}
\put(170.0,739.0){\rule[-0.200pt]{2.409pt}{0.400pt}}
\put(1429.0,739.0){\rule[-0.200pt]{2.409pt}{0.400pt}}
\put(170.0,739.0){\rule[-0.200pt]{2.409pt}{0.400pt}}
\put(1429.0,739.0){\rule[-0.200pt]{2.409pt}{0.400pt}}
\put(170.0,739.0){\rule[-0.200pt]{2.409pt}{0.400pt}}
\put(1429.0,739.0){\rule[-0.200pt]{2.409pt}{0.400pt}}
\put(170.0,739.0){\rule[-0.200pt]{2.409pt}{0.400pt}}
\put(1429.0,739.0){\rule[-0.200pt]{2.409pt}{0.400pt}}
\put(170.0,739.0){\rule[-0.200pt]{2.409pt}{0.400pt}}
\put(1429.0,739.0){\rule[-0.200pt]{2.409pt}{0.400pt}}
\put(170.0,739.0){\rule[-0.200pt]{2.409pt}{0.400pt}}
\put(1429.0,739.0){\rule[-0.200pt]{2.409pt}{0.400pt}}
\put(170.0,739.0){\rule[-0.200pt]{2.409pt}{0.400pt}}
\put(1429.0,739.0){\rule[-0.200pt]{2.409pt}{0.400pt}}
\put(170.0,739.0){\rule[-0.200pt]{2.409pt}{0.400pt}}
\put(1429.0,739.0){\rule[-0.200pt]{2.409pt}{0.400pt}}
\put(170.0,739.0){\rule[-0.200pt]{2.409pt}{0.400pt}}
\put(1429.0,739.0){\rule[-0.200pt]{2.409pt}{0.400pt}}
\put(170.0,739.0){\rule[-0.200pt]{2.409pt}{0.400pt}}
\put(1429.0,739.0){\rule[-0.200pt]{2.409pt}{0.400pt}}
\put(170.0,740.0){\rule[-0.200pt]{2.409pt}{0.400pt}}
\put(1429.0,740.0){\rule[-0.200pt]{2.409pt}{0.400pt}}
\put(170.0,740.0){\rule[-0.200pt]{2.409pt}{0.400pt}}
\put(1429.0,740.0){\rule[-0.200pt]{2.409pt}{0.400pt}}
\put(170.0,740.0){\rule[-0.200pt]{2.409pt}{0.400pt}}
\put(1429.0,740.0){\rule[-0.200pt]{2.409pt}{0.400pt}}
\put(170.0,740.0){\rule[-0.200pt]{2.409pt}{0.400pt}}
\put(1429.0,740.0){\rule[-0.200pt]{2.409pt}{0.400pt}}
\put(170.0,740.0){\rule[-0.200pt]{2.409pt}{0.400pt}}
\put(1429.0,740.0){\rule[-0.200pt]{2.409pt}{0.400pt}}
\put(170.0,740.0){\rule[-0.200pt]{2.409pt}{0.400pt}}
\put(1429.0,740.0){\rule[-0.200pt]{2.409pt}{0.400pt}}
\put(170.0,740.0){\rule[-0.200pt]{2.409pt}{0.400pt}}
\put(1429.0,740.0){\rule[-0.200pt]{2.409pt}{0.400pt}}
\put(170.0,740.0){\rule[-0.200pt]{2.409pt}{0.400pt}}
\put(1429.0,740.0){\rule[-0.200pt]{2.409pt}{0.400pt}}
\put(170.0,740.0){\rule[-0.200pt]{2.409pt}{0.400pt}}
\put(1429.0,740.0){\rule[-0.200pt]{2.409pt}{0.400pt}}
\put(170.0,740.0){\rule[-0.200pt]{2.409pt}{0.400pt}}
\put(1429.0,740.0){\rule[-0.200pt]{2.409pt}{0.400pt}}
\put(170.0,740.0){\rule[-0.200pt]{2.409pt}{0.400pt}}
\put(1429.0,740.0){\rule[-0.200pt]{2.409pt}{0.400pt}}
\put(170.0,740.0){\rule[-0.200pt]{2.409pt}{0.400pt}}
\put(1429.0,740.0){\rule[-0.200pt]{2.409pt}{0.400pt}}
\put(170.0,740.0){\rule[-0.200pt]{2.409pt}{0.400pt}}
\put(1429.0,740.0){\rule[-0.200pt]{2.409pt}{0.400pt}}
\put(170.0,741.0){\rule[-0.200pt]{2.409pt}{0.400pt}}
\put(1429.0,741.0){\rule[-0.200pt]{2.409pt}{0.400pt}}
\put(170.0,741.0){\rule[-0.200pt]{2.409pt}{0.400pt}}
\put(1429.0,741.0){\rule[-0.200pt]{2.409pt}{0.400pt}}
\put(170.0,741.0){\rule[-0.200pt]{2.409pt}{0.400pt}}
\put(1429.0,741.0){\rule[-0.200pt]{2.409pt}{0.400pt}}
\put(170.0,741.0){\rule[-0.200pt]{2.409pt}{0.400pt}}
\put(1429.0,741.0){\rule[-0.200pt]{2.409pt}{0.400pt}}
\put(170.0,741.0){\rule[-0.200pt]{2.409pt}{0.400pt}}
\put(1429.0,741.0){\rule[-0.200pt]{2.409pt}{0.400pt}}
\put(170.0,741.0){\rule[-0.200pt]{2.409pt}{0.400pt}}
\put(1429.0,741.0){\rule[-0.200pt]{2.409pt}{0.400pt}}
\put(170.0,741.0){\rule[-0.200pt]{2.409pt}{0.400pt}}
\put(1429.0,741.0){\rule[-0.200pt]{2.409pt}{0.400pt}}
\put(170.0,741.0){\rule[-0.200pt]{2.409pt}{0.400pt}}
\put(1429.0,741.0){\rule[-0.200pt]{2.409pt}{0.400pt}}
\put(170.0,741.0){\rule[-0.200pt]{2.409pt}{0.400pt}}
\put(1429.0,741.0){\rule[-0.200pt]{2.409pt}{0.400pt}}
\put(170.0,741.0){\rule[-0.200pt]{2.409pt}{0.400pt}}
\put(1429.0,741.0){\rule[-0.200pt]{2.409pt}{0.400pt}}
\put(170.0,741.0){\rule[-0.200pt]{2.409pt}{0.400pt}}
\put(1429.0,741.0){\rule[-0.200pt]{2.409pt}{0.400pt}}
\put(170.0,741.0){\rule[-0.200pt]{2.409pt}{0.400pt}}
\put(1429.0,741.0){\rule[-0.200pt]{2.409pt}{0.400pt}}
\put(170.0,741.0){\rule[-0.200pt]{2.409pt}{0.400pt}}
\put(1429.0,741.0){\rule[-0.200pt]{2.409pt}{0.400pt}}
\put(170.0,742.0){\rule[-0.200pt]{2.409pt}{0.400pt}}
\put(1429.0,742.0){\rule[-0.200pt]{2.409pt}{0.400pt}}
\put(170.0,742.0){\rule[-0.200pt]{2.409pt}{0.400pt}}
\put(1429.0,742.0){\rule[-0.200pt]{2.409pt}{0.400pt}}
\put(170.0,742.0){\rule[-0.200pt]{2.409pt}{0.400pt}}
\put(1429.0,742.0){\rule[-0.200pt]{2.409pt}{0.400pt}}
\put(170.0,742.0){\rule[-0.200pt]{2.409pt}{0.400pt}}
\put(1429.0,742.0){\rule[-0.200pt]{2.409pt}{0.400pt}}
\put(170.0,742.0){\rule[-0.200pt]{2.409pt}{0.400pt}}
\put(1429.0,742.0){\rule[-0.200pt]{2.409pt}{0.400pt}}
\put(170.0,742.0){\rule[-0.200pt]{2.409pt}{0.400pt}}
\put(1429.0,742.0){\rule[-0.200pt]{2.409pt}{0.400pt}}
\put(170.0,742.0){\rule[-0.200pt]{2.409pt}{0.400pt}}
\put(1429.0,742.0){\rule[-0.200pt]{2.409pt}{0.400pt}}
\put(170.0,742.0){\rule[-0.200pt]{2.409pt}{0.400pt}}
\put(1429.0,742.0){\rule[-0.200pt]{2.409pt}{0.400pt}}
\put(170.0,742.0){\rule[-0.200pt]{2.409pt}{0.400pt}}
\put(1429.0,742.0){\rule[-0.200pt]{2.409pt}{0.400pt}}
\put(170.0,742.0){\rule[-0.200pt]{2.409pt}{0.400pt}}
\put(1429.0,742.0){\rule[-0.200pt]{2.409pt}{0.400pt}}
\put(170.0,742.0){\rule[-0.200pt]{2.409pt}{0.400pt}}
\put(1429.0,742.0){\rule[-0.200pt]{2.409pt}{0.400pt}}
\put(170.0,742.0){\rule[-0.200pt]{2.409pt}{0.400pt}}
\put(1429.0,742.0){\rule[-0.200pt]{2.409pt}{0.400pt}}
\put(170.0,742.0){\rule[-0.200pt]{2.409pt}{0.400pt}}
\put(1429.0,742.0){\rule[-0.200pt]{2.409pt}{0.400pt}}
\put(170.0,742.0){\rule[-0.200pt]{2.409pt}{0.400pt}}
\put(1429.0,742.0){\rule[-0.200pt]{2.409pt}{0.400pt}}
\put(170.0,742.0){\rule[-0.200pt]{2.409pt}{0.400pt}}
\put(1429.0,742.0){\rule[-0.200pt]{2.409pt}{0.400pt}}
\put(170.0,743.0){\rule[-0.200pt]{2.409pt}{0.400pt}}
\put(1429.0,743.0){\rule[-0.200pt]{2.409pt}{0.400pt}}
\put(170.0,743.0){\rule[-0.200pt]{2.409pt}{0.400pt}}
\put(1429.0,743.0){\rule[-0.200pt]{2.409pt}{0.400pt}}
\put(170.0,743.0){\rule[-0.200pt]{2.409pt}{0.400pt}}
\put(1429.0,743.0){\rule[-0.200pt]{2.409pt}{0.400pt}}
\put(170.0,743.0){\rule[-0.200pt]{2.409pt}{0.400pt}}
\put(1429.0,743.0){\rule[-0.200pt]{2.409pt}{0.400pt}}
\put(170.0,743.0){\rule[-0.200pt]{2.409pt}{0.400pt}}
\put(1429.0,743.0){\rule[-0.200pt]{2.409pt}{0.400pt}}
\put(170.0,743.0){\rule[-0.200pt]{2.409pt}{0.400pt}}
\put(1429.0,743.0){\rule[-0.200pt]{2.409pt}{0.400pt}}
\put(170.0,743.0){\rule[-0.200pt]{2.409pt}{0.400pt}}
\put(1429.0,743.0){\rule[-0.200pt]{2.409pt}{0.400pt}}
\put(170.0,743.0){\rule[-0.200pt]{2.409pt}{0.400pt}}
\put(1429.0,743.0){\rule[-0.200pt]{2.409pt}{0.400pt}}
\put(170.0,743.0){\rule[-0.200pt]{2.409pt}{0.400pt}}
\put(1429.0,743.0){\rule[-0.200pt]{2.409pt}{0.400pt}}
\put(170.0,743.0){\rule[-0.200pt]{2.409pt}{0.400pt}}
\put(1429.0,743.0){\rule[-0.200pt]{2.409pt}{0.400pt}}
\put(170.0,743.0){\rule[-0.200pt]{2.409pt}{0.400pt}}
\put(1429.0,743.0){\rule[-0.200pt]{2.409pt}{0.400pt}}
\put(170.0,743.0){\rule[-0.200pt]{2.409pt}{0.400pt}}
\put(1429.0,743.0){\rule[-0.200pt]{2.409pt}{0.400pt}}
\put(170.0,743.0){\rule[-0.200pt]{2.409pt}{0.400pt}}
\put(1429.0,743.0){\rule[-0.200pt]{2.409pt}{0.400pt}}
\put(170.0,743.0){\rule[-0.200pt]{2.409pt}{0.400pt}}
\put(1429.0,743.0){\rule[-0.200pt]{2.409pt}{0.400pt}}
\put(170.0,743.0){\rule[-0.200pt]{2.409pt}{0.400pt}}
\put(1429.0,743.0){\rule[-0.200pt]{2.409pt}{0.400pt}}
\put(170.0,744.0){\rule[-0.200pt]{2.409pt}{0.400pt}}
\put(1429.0,744.0){\rule[-0.200pt]{2.409pt}{0.400pt}}
\put(170.0,744.0){\rule[-0.200pt]{2.409pt}{0.400pt}}
\put(1429.0,744.0){\rule[-0.200pt]{2.409pt}{0.400pt}}
\put(170.0,744.0){\rule[-0.200pt]{2.409pt}{0.400pt}}
\put(1429.0,744.0){\rule[-0.200pt]{2.409pt}{0.400pt}}
\put(170.0,744.0){\rule[-0.200pt]{2.409pt}{0.400pt}}
\put(1429.0,744.0){\rule[-0.200pt]{2.409pt}{0.400pt}}
\put(170.0,744.0){\rule[-0.200pt]{2.409pt}{0.400pt}}
\put(1429.0,744.0){\rule[-0.200pt]{2.409pt}{0.400pt}}
\put(170.0,744.0){\rule[-0.200pt]{2.409pt}{0.400pt}}
\put(1429.0,744.0){\rule[-0.200pt]{2.409pt}{0.400pt}}
\put(170.0,744.0){\rule[-0.200pt]{2.409pt}{0.400pt}}
\put(1429.0,744.0){\rule[-0.200pt]{2.409pt}{0.400pt}}
\put(170.0,744.0){\rule[-0.200pt]{2.409pt}{0.400pt}}
\put(1429.0,744.0){\rule[-0.200pt]{2.409pt}{0.400pt}}
\put(170.0,744.0){\rule[-0.200pt]{2.409pt}{0.400pt}}
\put(1429.0,744.0){\rule[-0.200pt]{2.409pt}{0.400pt}}
\put(170.0,744.0){\rule[-0.200pt]{2.409pt}{0.400pt}}
\put(1429.0,744.0){\rule[-0.200pt]{2.409pt}{0.400pt}}
\put(170.0,744.0){\rule[-0.200pt]{2.409pt}{0.400pt}}
\put(1429.0,744.0){\rule[-0.200pt]{2.409pt}{0.400pt}}
\put(170.0,744.0){\rule[-0.200pt]{2.409pt}{0.400pt}}
\put(1429.0,744.0){\rule[-0.200pt]{2.409pt}{0.400pt}}
\put(170.0,744.0){\rule[-0.200pt]{2.409pt}{0.400pt}}
\put(1429.0,744.0){\rule[-0.200pt]{2.409pt}{0.400pt}}
\put(170.0,744.0){\rule[-0.200pt]{2.409pt}{0.400pt}}
\put(1429.0,744.0){\rule[-0.200pt]{2.409pt}{0.400pt}}
\put(170.0,744.0){\rule[-0.200pt]{2.409pt}{0.400pt}}
\put(1429.0,744.0){\rule[-0.200pt]{2.409pt}{0.400pt}}
\put(170.0,744.0){\rule[-0.200pt]{2.409pt}{0.400pt}}
\put(1429.0,744.0){\rule[-0.200pt]{2.409pt}{0.400pt}}
\put(170.0,745.0){\rule[-0.200pt]{2.409pt}{0.400pt}}
\put(1429.0,745.0){\rule[-0.200pt]{2.409pt}{0.400pt}}
\put(170.0,745.0){\rule[-0.200pt]{2.409pt}{0.400pt}}
\put(1429.0,745.0){\rule[-0.200pt]{2.409pt}{0.400pt}}
\put(170.0,745.0){\rule[-0.200pt]{2.409pt}{0.400pt}}
\put(1429.0,745.0){\rule[-0.200pt]{2.409pt}{0.400pt}}
\put(170.0,745.0){\rule[-0.200pt]{2.409pt}{0.400pt}}
\put(1429.0,745.0){\rule[-0.200pt]{2.409pt}{0.400pt}}
\put(170.0,745.0){\rule[-0.200pt]{2.409pt}{0.400pt}}
\put(1429.0,745.0){\rule[-0.200pt]{2.409pt}{0.400pt}}
\put(170.0,745.0){\rule[-0.200pt]{2.409pt}{0.400pt}}
\put(1429.0,745.0){\rule[-0.200pt]{2.409pt}{0.400pt}}
\put(170.0,745.0){\rule[-0.200pt]{2.409pt}{0.400pt}}
\put(1429.0,745.0){\rule[-0.200pt]{2.409pt}{0.400pt}}
\put(170.0,745.0){\rule[-0.200pt]{2.409pt}{0.400pt}}
\put(1429.0,745.0){\rule[-0.200pt]{2.409pt}{0.400pt}}
\put(170.0,745.0){\rule[-0.200pt]{2.409pt}{0.400pt}}
\put(1429.0,745.0){\rule[-0.200pt]{2.409pt}{0.400pt}}
\put(170.0,745.0){\rule[-0.200pt]{2.409pt}{0.400pt}}
\put(1429.0,745.0){\rule[-0.200pt]{2.409pt}{0.400pt}}
\put(170.0,745.0){\rule[-0.200pt]{2.409pt}{0.400pt}}
\put(1429.0,745.0){\rule[-0.200pt]{2.409pt}{0.400pt}}
\put(170.0,745.0){\rule[-0.200pt]{2.409pt}{0.400pt}}
\put(1429.0,745.0){\rule[-0.200pt]{2.409pt}{0.400pt}}
\put(170.0,745.0){\rule[-0.200pt]{2.409pt}{0.400pt}}
\put(1429.0,745.0){\rule[-0.200pt]{2.409pt}{0.400pt}}
\put(170.0,745.0){\rule[-0.200pt]{2.409pt}{0.400pt}}
\put(1429.0,745.0){\rule[-0.200pt]{2.409pt}{0.400pt}}
\put(170.0,745.0){\rule[-0.200pt]{2.409pt}{0.400pt}}
\put(1429.0,745.0){\rule[-0.200pt]{2.409pt}{0.400pt}}
\put(170.0,745.0){\rule[-0.200pt]{2.409pt}{0.400pt}}
\put(1429.0,745.0){\rule[-0.200pt]{2.409pt}{0.400pt}}
\put(170.0,745.0){\rule[-0.200pt]{2.409pt}{0.400pt}}
\put(1429.0,745.0){\rule[-0.200pt]{2.409pt}{0.400pt}}
\put(170.0,746.0){\rule[-0.200pt]{2.409pt}{0.400pt}}
\put(1429.0,746.0){\rule[-0.200pt]{2.409pt}{0.400pt}}
\put(170.0,746.0){\rule[-0.200pt]{2.409pt}{0.400pt}}
\put(1429.0,746.0){\rule[-0.200pt]{2.409pt}{0.400pt}}
\put(170.0,746.0){\rule[-0.200pt]{2.409pt}{0.400pt}}
\put(1429.0,746.0){\rule[-0.200pt]{2.409pt}{0.400pt}}
\put(170.0,746.0){\rule[-0.200pt]{2.409pt}{0.400pt}}
\put(1429.0,746.0){\rule[-0.200pt]{2.409pt}{0.400pt}}
\put(170.0,746.0){\rule[-0.200pt]{2.409pt}{0.400pt}}
\put(1429.0,746.0){\rule[-0.200pt]{2.409pt}{0.400pt}}
\put(170.0,746.0){\rule[-0.200pt]{2.409pt}{0.400pt}}
\put(1429.0,746.0){\rule[-0.200pt]{2.409pt}{0.400pt}}
\put(170.0,746.0){\rule[-0.200pt]{2.409pt}{0.400pt}}
\put(1429.0,746.0){\rule[-0.200pt]{2.409pt}{0.400pt}}
\put(170.0,746.0){\rule[-0.200pt]{2.409pt}{0.400pt}}
\put(1429.0,746.0){\rule[-0.200pt]{2.409pt}{0.400pt}}
\put(170.0,746.0){\rule[-0.200pt]{2.409pt}{0.400pt}}
\put(1429.0,746.0){\rule[-0.200pt]{2.409pt}{0.400pt}}
\put(170.0,746.0){\rule[-0.200pt]{2.409pt}{0.400pt}}
\put(1429.0,746.0){\rule[-0.200pt]{2.409pt}{0.400pt}}
\put(170.0,746.0){\rule[-0.200pt]{2.409pt}{0.400pt}}
\put(1429.0,746.0){\rule[-0.200pt]{2.409pt}{0.400pt}}
\put(170.0,746.0){\rule[-0.200pt]{2.409pt}{0.400pt}}
\put(1429.0,746.0){\rule[-0.200pt]{2.409pt}{0.400pt}}
\put(170.0,746.0){\rule[-0.200pt]{2.409pt}{0.400pt}}
\put(1429.0,746.0){\rule[-0.200pt]{2.409pt}{0.400pt}}
\put(170.0,746.0){\rule[-0.200pt]{2.409pt}{0.400pt}}
\put(1429.0,746.0){\rule[-0.200pt]{2.409pt}{0.400pt}}
\put(170.0,746.0){\rule[-0.200pt]{2.409pt}{0.400pt}}
\put(1429.0,746.0){\rule[-0.200pt]{2.409pt}{0.400pt}}
\put(170.0,746.0){\rule[-0.200pt]{2.409pt}{0.400pt}}
\put(1429.0,746.0){\rule[-0.200pt]{2.409pt}{0.400pt}}
\put(170.0,746.0){\rule[-0.200pt]{2.409pt}{0.400pt}}
\put(1429.0,746.0){\rule[-0.200pt]{2.409pt}{0.400pt}}
\put(170.0,746.0){\rule[-0.200pt]{2.409pt}{0.400pt}}
\put(1429.0,746.0){\rule[-0.200pt]{2.409pt}{0.400pt}}
\put(170.0,747.0){\rule[-0.200pt]{2.409pt}{0.400pt}}
\put(1429.0,747.0){\rule[-0.200pt]{2.409pt}{0.400pt}}
\put(170.0,747.0){\rule[-0.200pt]{2.409pt}{0.400pt}}
\put(1429.0,747.0){\rule[-0.200pt]{2.409pt}{0.400pt}}
\put(170.0,747.0){\rule[-0.200pt]{2.409pt}{0.400pt}}
\put(1429.0,747.0){\rule[-0.200pt]{2.409pt}{0.400pt}}
\put(170.0,747.0){\rule[-0.200pt]{2.409pt}{0.400pt}}
\put(1429.0,747.0){\rule[-0.200pt]{2.409pt}{0.400pt}}
\put(170.0,747.0){\rule[-0.200pt]{2.409pt}{0.400pt}}
\put(1429.0,747.0){\rule[-0.200pt]{2.409pt}{0.400pt}}
\put(170.0,747.0){\rule[-0.200pt]{2.409pt}{0.400pt}}
\put(1429.0,747.0){\rule[-0.200pt]{2.409pt}{0.400pt}}
\put(170.0,747.0){\rule[-0.200pt]{2.409pt}{0.400pt}}
\put(1429.0,747.0){\rule[-0.200pt]{2.409pt}{0.400pt}}
\put(170.0,747.0){\rule[-0.200pt]{2.409pt}{0.400pt}}
\put(1429.0,747.0){\rule[-0.200pt]{2.409pt}{0.400pt}}
\put(170.0,747.0){\rule[-0.200pt]{2.409pt}{0.400pt}}
\put(1429.0,747.0){\rule[-0.200pt]{2.409pt}{0.400pt}}
\put(170.0,747.0){\rule[-0.200pt]{2.409pt}{0.400pt}}
\put(1429.0,747.0){\rule[-0.200pt]{2.409pt}{0.400pt}}
\put(170.0,747.0){\rule[-0.200pt]{2.409pt}{0.400pt}}
\put(1429.0,747.0){\rule[-0.200pt]{2.409pt}{0.400pt}}
\put(170.0,747.0){\rule[-0.200pt]{2.409pt}{0.400pt}}
\put(1429.0,747.0){\rule[-0.200pt]{2.409pt}{0.400pt}}
\put(170.0,747.0){\rule[-0.200pt]{2.409pt}{0.400pt}}
\put(1429.0,747.0){\rule[-0.200pt]{2.409pt}{0.400pt}}
\put(170.0,747.0){\rule[-0.200pt]{2.409pt}{0.400pt}}
\put(1429.0,747.0){\rule[-0.200pt]{2.409pt}{0.400pt}}
\put(170.0,747.0){\rule[-0.200pt]{2.409pt}{0.400pt}}
\put(1429.0,747.0){\rule[-0.200pt]{2.409pt}{0.400pt}}
\put(170.0,747.0){\rule[-0.200pt]{2.409pt}{0.400pt}}
\put(1429.0,747.0){\rule[-0.200pt]{2.409pt}{0.400pt}}
\put(170.0,747.0){\rule[-0.200pt]{2.409pt}{0.400pt}}
\put(1429.0,747.0){\rule[-0.200pt]{2.409pt}{0.400pt}}
\put(170.0,747.0){\rule[-0.200pt]{2.409pt}{0.400pt}}
\put(1429.0,747.0){\rule[-0.200pt]{2.409pt}{0.400pt}}
\put(170.0,747.0){\rule[-0.200pt]{2.409pt}{0.400pt}}
\put(1429.0,747.0){\rule[-0.200pt]{2.409pt}{0.400pt}}
\put(170.0,747.0){\rule[-0.200pt]{2.409pt}{0.400pt}}
\put(1429.0,747.0){\rule[-0.200pt]{2.409pt}{0.400pt}}
\put(170.0,748.0){\rule[-0.200pt]{2.409pt}{0.400pt}}
\put(1429.0,748.0){\rule[-0.200pt]{2.409pt}{0.400pt}}
\put(170.0,748.0){\rule[-0.200pt]{2.409pt}{0.400pt}}
\put(1429.0,748.0){\rule[-0.200pt]{2.409pt}{0.400pt}}
\put(170.0,748.0){\rule[-0.200pt]{2.409pt}{0.400pt}}
\put(1429.0,748.0){\rule[-0.200pt]{2.409pt}{0.400pt}}
\put(170.0,748.0){\rule[-0.200pt]{2.409pt}{0.400pt}}
\put(1429.0,748.0){\rule[-0.200pt]{2.409pt}{0.400pt}}
\put(170.0,748.0){\rule[-0.200pt]{2.409pt}{0.400pt}}
\put(1429.0,748.0){\rule[-0.200pt]{2.409pt}{0.400pt}}
\put(170.0,748.0){\rule[-0.200pt]{2.409pt}{0.400pt}}
\put(1429.0,748.0){\rule[-0.200pt]{2.409pt}{0.400pt}}
\put(170.0,748.0){\rule[-0.200pt]{2.409pt}{0.400pt}}
\put(1429.0,748.0){\rule[-0.200pt]{2.409pt}{0.400pt}}
\put(170.0,748.0){\rule[-0.200pt]{2.409pt}{0.400pt}}
\put(1429.0,748.0){\rule[-0.200pt]{2.409pt}{0.400pt}}
\put(170.0,748.0){\rule[-0.200pt]{2.409pt}{0.400pt}}
\put(1429.0,748.0){\rule[-0.200pt]{2.409pt}{0.400pt}}
\put(170.0,748.0){\rule[-0.200pt]{2.409pt}{0.400pt}}
\put(1429.0,748.0){\rule[-0.200pt]{2.409pt}{0.400pt}}
\put(170.0,748.0){\rule[-0.200pt]{4.818pt}{0.400pt}}
\put(150,748){\makebox(0,0)[r]{ 1}}
\put(1419.0,748.0){\rule[-0.200pt]{4.818pt}{0.400pt}}
\put(170.0,759.0){\rule[-0.200pt]{2.409pt}{0.400pt}}
\put(1429.0,759.0){\rule[-0.200pt]{2.409pt}{0.400pt}}
\put(170.0,774.0){\rule[-0.200pt]{2.409pt}{0.400pt}}
\put(1429.0,774.0){\rule[-0.200pt]{2.409pt}{0.400pt}}
\put(170.0,781.0){\rule[-0.200pt]{2.409pt}{0.400pt}}
\put(1429.0,781.0){\rule[-0.200pt]{2.409pt}{0.400pt}}
\put(170.0,787.0){\rule[-0.200pt]{2.409pt}{0.400pt}}
\put(1429.0,787.0){\rule[-0.200pt]{2.409pt}{0.400pt}}
\put(170.0,790.0){\rule[-0.200pt]{2.409pt}{0.400pt}}
\put(1429.0,790.0){\rule[-0.200pt]{2.409pt}{0.400pt}}
\put(170.0,794.0){\rule[-0.200pt]{2.409pt}{0.400pt}}
\put(1429.0,794.0){\rule[-0.200pt]{2.409pt}{0.400pt}}
\put(170.0,796.0){\rule[-0.200pt]{2.409pt}{0.400pt}}
\put(1429.0,796.0){\rule[-0.200pt]{2.409pt}{0.400pt}}
\put(170.0,798.0){\rule[-0.200pt]{2.409pt}{0.400pt}}
\put(1429.0,798.0){\rule[-0.200pt]{2.409pt}{0.400pt}}
\put(170.0,800.0){\rule[-0.200pt]{2.409pt}{0.400pt}}
\put(1429.0,800.0){\rule[-0.200pt]{2.409pt}{0.400pt}}
\put(170.0,802.0){\rule[-0.200pt]{2.409pt}{0.400pt}}
\put(1429.0,802.0){\rule[-0.200pt]{2.409pt}{0.400pt}}
\put(170.0,804.0){\rule[-0.200pt]{2.409pt}{0.400pt}}
\put(1429.0,804.0){\rule[-0.200pt]{2.409pt}{0.400pt}}
\put(170.0,805.0){\rule[-0.200pt]{2.409pt}{0.400pt}}
\put(1429.0,805.0){\rule[-0.200pt]{2.409pt}{0.400pt}}
\put(170.0,806.0){\rule[-0.200pt]{2.409pt}{0.400pt}}
\put(1429.0,806.0){\rule[-0.200pt]{2.409pt}{0.400pt}}
\put(170.0,808.0){\rule[-0.200pt]{2.409pt}{0.400pt}}
\put(1429.0,808.0){\rule[-0.200pt]{2.409pt}{0.400pt}}
\put(170.0,809.0){\rule[-0.200pt]{2.409pt}{0.400pt}}
\put(1429.0,809.0){\rule[-0.200pt]{2.409pt}{0.400pt}}
\put(170.0,810.0){\rule[-0.200pt]{2.409pt}{0.400pt}}
\put(1429.0,810.0){\rule[-0.200pt]{2.409pt}{0.400pt}}
\put(170.0,811.0){\rule[-0.200pt]{2.409pt}{0.400pt}}
\put(1429.0,811.0){\rule[-0.200pt]{2.409pt}{0.400pt}}
\put(170.0,812.0){\rule[-0.200pt]{2.409pt}{0.400pt}}
\put(1429.0,812.0){\rule[-0.200pt]{2.409pt}{0.400pt}}
\put(170.0,813.0){\rule[-0.200pt]{2.409pt}{0.400pt}}
\put(1429.0,813.0){\rule[-0.200pt]{2.409pt}{0.400pt}}
\put(170.0,814.0){\rule[-0.200pt]{2.409pt}{0.400pt}}
\put(1429.0,814.0){\rule[-0.200pt]{2.409pt}{0.400pt}}
\put(170.0,814.0){\rule[-0.200pt]{2.409pt}{0.400pt}}
\put(1429.0,814.0){\rule[-0.200pt]{2.409pt}{0.400pt}}
\put(170.0,815.0){\rule[-0.200pt]{2.409pt}{0.400pt}}
\put(1429.0,815.0){\rule[-0.200pt]{2.409pt}{0.400pt}}
\put(170.0,816.0){\rule[-0.200pt]{2.409pt}{0.400pt}}
\put(1429.0,816.0){\rule[-0.200pt]{2.409pt}{0.400pt}}
\put(170.0,816.0){\rule[-0.200pt]{2.409pt}{0.400pt}}
\put(1429.0,816.0){\rule[-0.200pt]{2.409pt}{0.400pt}}
\put(170.0,817.0){\rule[-0.200pt]{2.409pt}{0.400pt}}
\put(1429.0,817.0){\rule[-0.200pt]{2.409pt}{0.400pt}}
\put(170.0,818.0){\rule[-0.200pt]{2.409pt}{0.400pt}}
\put(1429.0,818.0){\rule[-0.200pt]{2.409pt}{0.400pt}}
\put(170.0,818.0){\rule[-0.200pt]{2.409pt}{0.400pt}}
\put(1429.0,818.0){\rule[-0.200pt]{2.409pt}{0.400pt}}
\put(170.0,819.0){\rule[-0.200pt]{2.409pt}{0.400pt}}
\put(1429.0,819.0){\rule[-0.200pt]{2.409pt}{0.400pt}}
\put(170.0,820.0){\rule[-0.200pt]{2.409pt}{0.400pt}}
\put(1429.0,820.0){\rule[-0.200pt]{2.409pt}{0.400pt}}
\put(170.0,820.0){\rule[-0.200pt]{2.409pt}{0.400pt}}
\put(1429.0,820.0){\rule[-0.200pt]{2.409pt}{0.400pt}}
\put(170.0,821.0){\rule[-0.200pt]{2.409pt}{0.400pt}}
\put(1429.0,821.0){\rule[-0.200pt]{2.409pt}{0.400pt}}
\put(170.0,821.0){\rule[-0.200pt]{2.409pt}{0.400pt}}
\put(1429.0,821.0){\rule[-0.200pt]{2.409pt}{0.400pt}}
\put(170.0,822.0){\rule[-0.200pt]{2.409pt}{0.400pt}}
\put(1429.0,822.0){\rule[-0.200pt]{2.409pt}{0.400pt}}
\put(170.0,822.0){\rule[-0.200pt]{2.409pt}{0.400pt}}
\put(1429.0,822.0){\rule[-0.200pt]{2.409pt}{0.400pt}}
\put(170.0,823.0){\rule[-0.200pt]{2.409pt}{0.400pt}}
\put(1429.0,823.0){\rule[-0.200pt]{2.409pt}{0.400pt}}
\put(170.0,823.0){\rule[-0.200pt]{2.409pt}{0.400pt}}
\put(1429.0,823.0){\rule[-0.200pt]{2.409pt}{0.400pt}}
\put(170.0,824.0){\rule[-0.200pt]{2.409pt}{0.400pt}}
\put(1429.0,824.0){\rule[-0.200pt]{2.409pt}{0.400pt}}
\put(170.0,824.0){\rule[-0.200pt]{2.409pt}{0.400pt}}
\put(1429.0,824.0){\rule[-0.200pt]{2.409pt}{0.400pt}}
\put(170.0,824.0){\rule[-0.200pt]{2.409pt}{0.400pt}}
\put(1429.0,824.0){\rule[-0.200pt]{2.409pt}{0.400pt}}
\put(170.0,825.0){\rule[-0.200pt]{2.409pt}{0.400pt}}
\put(1429.0,825.0){\rule[-0.200pt]{2.409pt}{0.400pt}}
\put(170.0,825.0){\rule[-0.200pt]{2.409pt}{0.400pt}}
\put(1429.0,825.0){\rule[-0.200pt]{2.409pt}{0.400pt}}
\put(170.0,826.0){\rule[-0.200pt]{2.409pt}{0.400pt}}
\put(1429.0,826.0){\rule[-0.200pt]{2.409pt}{0.400pt}}
\put(170.0,826.0){\rule[-0.200pt]{2.409pt}{0.400pt}}
\put(1429.0,826.0){\rule[-0.200pt]{2.409pt}{0.400pt}}
\put(170.0,826.0){\rule[-0.200pt]{2.409pt}{0.400pt}}
\put(1429.0,826.0){\rule[-0.200pt]{2.409pt}{0.400pt}}
\put(170.0,827.0){\rule[-0.200pt]{2.409pt}{0.400pt}}
\put(1429.0,827.0){\rule[-0.200pt]{2.409pt}{0.400pt}}
\put(170.0,827.0){\rule[-0.200pt]{2.409pt}{0.400pt}}
\put(1429.0,827.0){\rule[-0.200pt]{2.409pt}{0.400pt}}
\put(170.0,827.0){\rule[-0.200pt]{2.409pt}{0.400pt}}
\put(1429.0,827.0){\rule[-0.200pt]{2.409pt}{0.400pt}}
\put(170.0,828.0){\rule[-0.200pt]{2.409pt}{0.400pt}}
\put(1429.0,828.0){\rule[-0.200pt]{2.409pt}{0.400pt}}
\put(170.0,828.0){\rule[-0.200pt]{2.409pt}{0.400pt}}
\put(1429.0,828.0){\rule[-0.200pt]{2.409pt}{0.400pt}}
\put(170.0,828.0){\rule[-0.200pt]{2.409pt}{0.400pt}}
\put(1429.0,828.0){\rule[-0.200pt]{2.409pt}{0.400pt}}
\put(170.0,829.0){\rule[-0.200pt]{2.409pt}{0.400pt}}
\put(1429.0,829.0){\rule[-0.200pt]{2.409pt}{0.400pt}}
\put(170.0,829.0){\rule[-0.200pt]{2.409pt}{0.400pt}}
\put(1429.0,829.0){\rule[-0.200pt]{2.409pt}{0.400pt}}
\put(170.0,829.0){\rule[-0.200pt]{2.409pt}{0.400pt}}
\put(1429.0,829.0){\rule[-0.200pt]{2.409pt}{0.400pt}}
\put(170.0,830.0){\rule[-0.200pt]{2.409pt}{0.400pt}}
\put(1429.0,830.0){\rule[-0.200pt]{2.409pt}{0.400pt}}
\put(170.0,830.0){\rule[-0.200pt]{2.409pt}{0.400pt}}
\put(1429.0,830.0){\rule[-0.200pt]{2.409pt}{0.400pt}}
\put(170.0,830.0){\rule[-0.200pt]{2.409pt}{0.400pt}}
\put(1429.0,830.0){\rule[-0.200pt]{2.409pt}{0.400pt}}
\put(170.0,831.0){\rule[-0.200pt]{2.409pt}{0.400pt}}
\put(1429.0,831.0){\rule[-0.200pt]{2.409pt}{0.400pt}}
\put(170.0,831.0){\rule[-0.200pt]{2.409pt}{0.400pt}}
\put(1429.0,831.0){\rule[-0.200pt]{2.409pt}{0.400pt}}
\put(170.0,831.0){\rule[-0.200pt]{2.409pt}{0.400pt}}
\put(1429.0,831.0){\rule[-0.200pt]{2.409pt}{0.400pt}}
\put(170.0,831.0){\rule[-0.200pt]{2.409pt}{0.400pt}}
\put(1429.0,831.0){\rule[-0.200pt]{2.409pt}{0.400pt}}
\put(170.0,832.0){\rule[-0.200pt]{2.409pt}{0.400pt}}
\put(1429.0,832.0){\rule[-0.200pt]{2.409pt}{0.400pt}}
\put(170.0,832.0){\rule[-0.200pt]{2.409pt}{0.400pt}}
\put(1429.0,832.0){\rule[-0.200pt]{2.409pt}{0.400pt}}
\put(170.0,832.0){\rule[-0.200pt]{2.409pt}{0.400pt}}
\put(1429.0,832.0){\rule[-0.200pt]{2.409pt}{0.400pt}}
\put(170.0,832.0){\rule[-0.200pt]{2.409pt}{0.400pt}}
\put(1429.0,832.0){\rule[-0.200pt]{2.409pt}{0.400pt}}
\put(170.0,833.0){\rule[-0.200pt]{2.409pt}{0.400pt}}
\put(1429.0,833.0){\rule[-0.200pt]{2.409pt}{0.400pt}}
\put(170.0,833.0){\rule[-0.200pt]{2.409pt}{0.400pt}}
\put(1429.0,833.0){\rule[-0.200pt]{2.409pt}{0.400pt}}
\put(170.0,833.0){\rule[-0.200pt]{2.409pt}{0.400pt}}
\put(1429.0,833.0){\rule[-0.200pt]{2.409pt}{0.400pt}}
\put(170.0,833.0){\rule[-0.200pt]{2.409pt}{0.400pt}}
\put(1429.0,833.0){\rule[-0.200pt]{2.409pt}{0.400pt}}
\put(170.0,834.0){\rule[-0.200pt]{2.409pt}{0.400pt}}
\put(1429.0,834.0){\rule[-0.200pt]{2.409pt}{0.400pt}}
\put(170.0,834.0){\rule[-0.200pt]{2.409pt}{0.400pt}}
\put(1429.0,834.0){\rule[-0.200pt]{2.409pt}{0.400pt}}
\put(170.0,834.0){\rule[-0.200pt]{2.409pt}{0.400pt}}
\put(1429.0,834.0){\rule[-0.200pt]{2.409pt}{0.400pt}}
\put(170.0,834.0){\rule[-0.200pt]{2.409pt}{0.400pt}}
\put(1429.0,834.0){\rule[-0.200pt]{2.409pt}{0.400pt}}
\put(170.0,835.0){\rule[-0.200pt]{2.409pt}{0.400pt}}
\put(1429.0,835.0){\rule[-0.200pt]{2.409pt}{0.400pt}}
\put(170.0,835.0){\rule[-0.200pt]{2.409pt}{0.400pt}}
\put(1429.0,835.0){\rule[-0.200pt]{2.409pt}{0.400pt}}
\put(170.0,835.0){\rule[-0.200pt]{2.409pt}{0.400pt}}
\put(1429.0,835.0){\rule[-0.200pt]{2.409pt}{0.400pt}}
\put(170.0,835.0){\rule[-0.200pt]{2.409pt}{0.400pt}}
\put(1429.0,835.0){\rule[-0.200pt]{2.409pt}{0.400pt}}
\put(170.0,835.0){\rule[-0.200pt]{2.409pt}{0.400pt}}
\put(1429.0,835.0){\rule[-0.200pt]{2.409pt}{0.400pt}}
\put(170.0,836.0){\rule[-0.200pt]{2.409pt}{0.400pt}}
\put(1429.0,836.0){\rule[-0.200pt]{2.409pt}{0.400pt}}
\put(170.0,836.0){\rule[-0.200pt]{2.409pt}{0.400pt}}
\put(1429.0,836.0){\rule[-0.200pt]{2.409pt}{0.400pt}}
\put(170.0,836.0){\rule[-0.200pt]{2.409pt}{0.400pt}}
\put(1429.0,836.0){\rule[-0.200pt]{2.409pt}{0.400pt}}
\put(170.0,836.0){\rule[-0.200pt]{2.409pt}{0.400pt}}
\put(1429.0,836.0){\rule[-0.200pt]{2.409pt}{0.400pt}}
\put(170.0,836.0){\rule[-0.200pt]{2.409pt}{0.400pt}}
\put(1429.0,836.0){\rule[-0.200pt]{2.409pt}{0.400pt}}
\put(170.0,837.0){\rule[-0.200pt]{2.409pt}{0.400pt}}
\put(1429.0,837.0){\rule[-0.200pt]{2.409pt}{0.400pt}}
\put(170.0,837.0){\rule[-0.200pt]{2.409pt}{0.400pt}}
\put(1429.0,837.0){\rule[-0.200pt]{2.409pt}{0.400pt}}
\put(170.0,837.0){\rule[-0.200pt]{2.409pt}{0.400pt}}
\put(1429.0,837.0){\rule[-0.200pt]{2.409pt}{0.400pt}}
\put(170.0,837.0){\rule[-0.200pt]{2.409pt}{0.400pt}}
\put(1429.0,837.0){\rule[-0.200pt]{2.409pt}{0.400pt}}
\put(170.0,837.0){\rule[-0.200pt]{2.409pt}{0.400pt}}
\put(1429.0,837.0){\rule[-0.200pt]{2.409pt}{0.400pt}}
\put(170.0,838.0){\rule[-0.200pt]{2.409pt}{0.400pt}}
\put(1429.0,838.0){\rule[-0.200pt]{2.409pt}{0.400pt}}
\put(170.0,838.0){\rule[-0.200pt]{2.409pt}{0.400pt}}
\put(1429.0,838.0){\rule[-0.200pt]{2.409pt}{0.400pt}}
\put(170.0,838.0){\rule[-0.200pt]{2.409pt}{0.400pt}}
\put(1429.0,838.0){\rule[-0.200pt]{2.409pt}{0.400pt}}
\put(170.0,838.0){\rule[-0.200pt]{2.409pt}{0.400pt}}
\put(1429.0,838.0){\rule[-0.200pt]{2.409pt}{0.400pt}}
\put(170.0,838.0){\rule[-0.200pt]{2.409pt}{0.400pt}}
\put(1429.0,838.0){\rule[-0.200pt]{2.409pt}{0.400pt}}
\put(170.0,838.0){\rule[-0.200pt]{2.409pt}{0.400pt}}
\put(1429.0,838.0){\rule[-0.200pt]{2.409pt}{0.400pt}}
\put(170.0,839.0){\rule[-0.200pt]{2.409pt}{0.400pt}}
\put(1429.0,839.0){\rule[-0.200pt]{2.409pt}{0.400pt}}
\put(170.0,839.0){\rule[-0.200pt]{2.409pt}{0.400pt}}
\put(1429.0,839.0){\rule[-0.200pt]{2.409pt}{0.400pt}}
\put(170.0,839.0){\rule[-0.200pt]{2.409pt}{0.400pt}}
\put(1429.0,839.0){\rule[-0.200pt]{2.409pt}{0.400pt}}
\put(170.0,839.0){\rule[-0.200pt]{2.409pt}{0.400pt}}
\put(1429.0,839.0){\rule[-0.200pt]{2.409pt}{0.400pt}}
\put(170.0,839.0){\rule[-0.200pt]{2.409pt}{0.400pt}}
\put(1429.0,839.0){\rule[-0.200pt]{2.409pt}{0.400pt}}
\put(170.0,839.0){\rule[-0.200pt]{2.409pt}{0.400pt}}
\put(1429.0,839.0){\rule[-0.200pt]{2.409pt}{0.400pt}}
\put(170.0,840.0){\rule[-0.200pt]{2.409pt}{0.400pt}}
\put(1429.0,840.0){\rule[-0.200pt]{2.409pt}{0.400pt}}
\put(170.0,840.0){\rule[-0.200pt]{2.409pt}{0.400pt}}
\put(1429.0,840.0){\rule[-0.200pt]{2.409pt}{0.400pt}}
\put(170.0,840.0){\rule[-0.200pt]{2.409pt}{0.400pt}}
\put(1429.0,840.0){\rule[-0.200pt]{2.409pt}{0.400pt}}
\put(170.0,840.0){\rule[-0.200pt]{2.409pt}{0.400pt}}
\put(1429.0,840.0){\rule[-0.200pt]{2.409pt}{0.400pt}}
\put(170.0,840.0){\rule[-0.200pt]{2.409pt}{0.400pt}}
\put(1429.0,840.0){\rule[-0.200pt]{2.409pt}{0.400pt}}
\put(170.0,840.0){\rule[-0.200pt]{2.409pt}{0.400pt}}
\put(1429.0,840.0){\rule[-0.200pt]{2.409pt}{0.400pt}}
\put(170.0,841.0){\rule[-0.200pt]{2.409pt}{0.400pt}}
\put(1429.0,841.0){\rule[-0.200pt]{2.409pt}{0.400pt}}
\put(170.0,841.0){\rule[-0.200pt]{2.409pt}{0.400pt}}
\put(1429.0,841.0){\rule[-0.200pt]{2.409pt}{0.400pt}}
\put(170.0,841.0){\rule[-0.200pt]{2.409pt}{0.400pt}}
\put(1429.0,841.0){\rule[-0.200pt]{2.409pt}{0.400pt}}
\put(170.0,841.0){\rule[-0.200pt]{2.409pt}{0.400pt}}
\put(1429.0,841.0){\rule[-0.200pt]{2.409pt}{0.400pt}}
\put(170.0,841.0){\rule[-0.200pt]{2.409pt}{0.400pt}}
\put(1429.0,841.0){\rule[-0.200pt]{2.409pt}{0.400pt}}
\put(170.0,841.0){\rule[-0.200pt]{2.409pt}{0.400pt}}
\put(1429.0,841.0){\rule[-0.200pt]{2.409pt}{0.400pt}}
\put(170.0,841.0){\rule[-0.200pt]{2.409pt}{0.400pt}}
\put(1429.0,841.0){\rule[-0.200pt]{2.409pt}{0.400pt}}
\put(170.0,842.0){\rule[-0.200pt]{2.409pt}{0.400pt}}
\put(1429.0,842.0){\rule[-0.200pt]{2.409pt}{0.400pt}}
\put(170.0,842.0){\rule[-0.200pt]{2.409pt}{0.400pt}}
\put(1429.0,842.0){\rule[-0.200pt]{2.409pt}{0.400pt}}
\put(170.0,842.0){\rule[-0.200pt]{2.409pt}{0.400pt}}
\put(1429.0,842.0){\rule[-0.200pt]{2.409pt}{0.400pt}}
\put(170.0,842.0){\rule[-0.200pt]{2.409pt}{0.400pt}}
\put(1429.0,842.0){\rule[-0.200pt]{2.409pt}{0.400pt}}
\put(170.0,842.0){\rule[-0.200pt]{2.409pt}{0.400pt}}
\put(1429.0,842.0){\rule[-0.200pt]{2.409pt}{0.400pt}}
\put(170.0,842.0){\rule[-0.200pt]{2.409pt}{0.400pt}}
\put(1429.0,842.0){\rule[-0.200pt]{2.409pt}{0.400pt}}
\put(170.0,842.0){\rule[-0.200pt]{2.409pt}{0.400pt}}
\put(1429.0,842.0){\rule[-0.200pt]{2.409pt}{0.400pt}}
\put(170.0,843.0){\rule[-0.200pt]{2.409pt}{0.400pt}}
\put(1429.0,843.0){\rule[-0.200pt]{2.409pt}{0.400pt}}
\put(170.0,843.0){\rule[-0.200pt]{2.409pt}{0.400pt}}
\put(1429.0,843.0){\rule[-0.200pt]{2.409pt}{0.400pt}}
\put(170.0,843.0){\rule[-0.200pt]{2.409pt}{0.400pt}}
\put(1429.0,843.0){\rule[-0.200pt]{2.409pt}{0.400pt}}
\put(170.0,843.0){\rule[-0.200pt]{2.409pt}{0.400pt}}
\put(1429.0,843.0){\rule[-0.200pt]{2.409pt}{0.400pt}}
\put(170.0,843.0){\rule[-0.200pt]{2.409pt}{0.400pt}}
\put(1429.0,843.0){\rule[-0.200pt]{2.409pt}{0.400pt}}
\put(170.0,843.0){\rule[-0.200pt]{2.409pt}{0.400pt}}
\put(1429.0,843.0){\rule[-0.200pt]{2.409pt}{0.400pt}}
\put(170.0,843.0){\rule[-0.200pt]{2.409pt}{0.400pt}}
\put(1429.0,843.0){\rule[-0.200pt]{2.409pt}{0.400pt}}
\put(170.0,843.0){\rule[-0.200pt]{2.409pt}{0.400pt}}
\put(1429.0,843.0){\rule[-0.200pt]{2.409pt}{0.400pt}}
\put(170.0,844.0){\rule[-0.200pt]{2.409pt}{0.400pt}}
\put(1429.0,844.0){\rule[-0.200pt]{2.409pt}{0.400pt}}
\put(170.0,844.0){\rule[-0.200pt]{2.409pt}{0.400pt}}
\put(1429.0,844.0){\rule[-0.200pt]{2.409pt}{0.400pt}}
\put(170.0,844.0){\rule[-0.200pt]{2.409pt}{0.400pt}}
\put(1429.0,844.0){\rule[-0.200pt]{2.409pt}{0.400pt}}
\put(170.0,844.0){\rule[-0.200pt]{2.409pt}{0.400pt}}
\put(1429.0,844.0){\rule[-0.200pt]{2.409pt}{0.400pt}}
\put(170.0,844.0){\rule[-0.200pt]{2.409pt}{0.400pt}}
\put(1429.0,844.0){\rule[-0.200pt]{2.409pt}{0.400pt}}
\put(170.0,844.0){\rule[-0.200pt]{2.409pt}{0.400pt}}
\put(1429.0,844.0){\rule[-0.200pt]{2.409pt}{0.400pt}}
\put(170.0,844.0){\rule[-0.200pt]{2.409pt}{0.400pt}}
\put(1429.0,844.0){\rule[-0.200pt]{2.409pt}{0.400pt}}
\put(170.0,844.0){\rule[-0.200pt]{2.409pt}{0.400pt}}
\put(1429.0,844.0){\rule[-0.200pt]{2.409pt}{0.400pt}}
\put(170.0,845.0){\rule[-0.200pt]{2.409pt}{0.400pt}}
\put(1429.0,845.0){\rule[-0.200pt]{2.409pt}{0.400pt}}
\put(170.0,845.0){\rule[-0.200pt]{2.409pt}{0.400pt}}
\put(1429.0,845.0){\rule[-0.200pt]{2.409pt}{0.400pt}}
\put(170.0,845.0){\rule[-0.200pt]{2.409pt}{0.400pt}}
\put(1429.0,845.0){\rule[-0.200pt]{2.409pt}{0.400pt}}
\put(170.0,845.0){\rule[-0.200pt]{2.409pt}{0.400pt}}
\put(1429.0,845.0){\rule[-0.200pt]{2.409pt}{0.400pt}}
\put(170.0,845.0){\rule[-0.200pt]{2.409pt}{0.400pt}}
\put(1429.0,845.0){\rule[-0.200pt]{2.409pt}{0.400pt}}
\put(170.0,845.0){\rule[-0.200pt]{2.409pt}{0.400pt}}
\put(1429.0,845.0){\rule[-0.200pt]{2.409pt}{0.400pt}}
\put(170.0,845.0){\rule[-0.200pt]{2.409pt}{0.400pt}}
\put(1429.0,845.0){\rule[-0.200pt]{2.409pt}{0.400pt}}
\put(170.0,845.0){\rule[-0.200pt]{2.409pt}{0.400pt}}
\put(1429.0,845.0){\rule[-0.200pt]{2.409pt}{0.400pt}}
\put(170.0,845.0){\rule[-0.200pt]{2.409pt}{0.400pt}}
\put(1429.0,845.0){\rule[-0.200pt]{2.409pt}{0.400pt}}
\put(170.0,846.0){\rule[-0.200pt]{2.409pt}{0.400pt}}
\put(1429.0,846.0){\rule[-0.200pt]{2.409pt}{0.400pt}}
\put(170.0,846.0){\rule[-0.200pt]{2.409pt}{0.400pt}}
\put(1429.0,846.0){\rule[-0.200pt]{2.409pt}{0.400pt}}
\put(170.0,846.0){\rule[-0.200pt]{2.409pt}{0.400pt}}
\put(1429.0,846.0){\rule[-0.200pt]{2.409pt}{0.400pt}}
\put(170.0,846.0){\rule[-0.200pt]{2.409pt}{0.400pt}}
\put(1429.0,846.0){\rule[-0.200pt]{2.409pt}{0.400pt}}
\put(170.0,846.0){\rule[-0.200pt]{2.409pt}{0.400pt}}
\put(1429.0,846.0){\rule[-0.200pt]{2.409pt}{0.400pt}}
\put(170.0,846.0){\rule[-0.200pt]{2.409pt}{0.400pt}}
\put(1429.0,846.0){\rule[-0.200pt]{2.409pt}{0.400pt}}
\put(170.0,846.0){\rule[-0.200pt]{2.409pt}{0.400pt}}
\put(1429.0,846.0){\rule[-0.200pt]{2.409pt}{0.400pt}}
\put(170.0,846.0){\rule[-0.200pt]{2.409pt}{0.400pt}}
\put(1429.0,846.0){\rule[-0.200pt]{2.409pt}{0.400pt}}
\put(170.0,846.0){\rule[-0.200pt]{2.409pt}{0.400pt}}
\put(1429.0,846.0){\rule[-0.200pt]{2.409pt}{0.400pt}}
\put(170.0,847.0){\rule[-0.200pt]{2.409pt}{0.400pt}}
\put(1429.0,847.0){\rule[-0.200pt]{2.409pt}{0.400pt}}
\put(170.0,847.0){\rule[-0.200pt]{2.409pt}{0.400pt}}
\put(1429.0,847.0){\rule[-0.200pt]{2.409pt}{0.400pt}}
\put(170.0,847.0){\rule[-0.200pt]{2.409pt}{0.400pt}}
\put(1429.0,847.0){\rule[-0.200pt]{2.409pt}{0.400pt}}
\put(170.0,847.0){\rule[-0.200pt]{2.409pt}{0.400pt}}
\put(1429.0,847.0){\rule[-0.200pt]{2.409pt}{0.400pt}}
\put(170.0,847.0){\rule[-0.200pt]{2.409pt}{0.400pt}}
\put(1429.0,847.0){\rule[-0.200pt]{2.409pt}{0.400pt}}
\put(170.0,847.0){\rule[-0.200pt]{2.409pt}{0.400pt}}
\put(1429.0,847.0){\rule[-0.200pt]{2.409pt}{0.400pt}}
\put(170.0,847.0){\rule[-0.200pt]{2.409pt}{0.400pt}}
\put(1429.0,847.0){\rule[-0.200pt]{2.409pt}{0.400pt}}
\put(170.0,847.0){\rule[-0.200pt]{2.409pt}{0.400pt}}
\put(1429.0,847.0){\rule[-0.200pt]{2.409pt}{0.400pt}}
\put(170.0,847.0){\rule[-0.200pt]{2.409pt}{0.400pt}}
\put(1429.0,847.0){\rule[-0.200pt]{2.409pt}{0.400pt}}
\put(170.0,847.0){\rule[-0.200pt]{2.409pt}{0.400pt}}
\put(1429.0,847.0){\rule[-0.200pt]{2.409pt}{0.400pt}}
\put(170.0,848.0){\rule[-0.200pt]{2.409pt}{0.400pt}}
\put(1429.0,848.0){\rule[-0.200pt]{2.409pt}{0.400pt}}
\put(170.0,848.0){\rule[-0.200pt]{2.409pt}{0.400pt}}
\put(1429.0,848.0){\rule[-0.200pt]{2.409pt}{0.400pt}}
\put(170.0,848.0){\rule[-0.200pt]{2.409pt}{0.400pt}}
\put(1429.0,848.0){\rule[-0.200pt]{2.409pt}{0.400pt}}
\put(170.0,848.0){\rule[-0.200pt]{2.409pt}{0.400pt}}
\put(1429.0,848.0){\rule[-0.200pt]{2.409pt}{0.400pt}}
\put(170.0,848.0){\rule[-0.200pt]{2.409pt}{0.400pt}}
\put(1429.0,848.0){\rule[-0.200pt]{2.409pt}{0.400pt}}
\put(170.0,848.0){\rule[-0.200pt]{2.409pt}{0.400pt}}
\put(1429.0,848.0){\rule[-0.200pt]{2.409pt}{0.400pt}}
\put(170.0,848.0){\rule[-0.200pt]{2.409pt}{0.400pt}}
\put(1429.0,848.0){\rule[-0.200pt]{2.409pt}{0.400pt}}
\put(170.0,848.0){\rule[-0.200pt]{2.409pt}{0.400pt}}
\put(1429.0,848.0){\rule[-0.200pt]{2.409pt}{0.400pt}}
\put(170.0,848.0){\rule[-0.200pt]{2.409pt}{0.400pt}}
\put(1429.0,848.0){\rule[-0.200pt]{2.409pt}{0.400pt}}
\put(170.0,848.0){\rule[-0.200pt]{2.409pt}{0.400pt}}
\put(1429.0,848.0){\rule[-0.200pt]{2.409pt}{0.400pt}}
\put(170.0,849.0){\rule[-0.200pt]{2.409pt}{0.400pt}}
\put(1429.0,849.0){\rule[-0.200pt]{2.409pt}{0.400pt}}
\put(170.0,849.0){\rule[-0.200pt]{2.409pt}{0.400pt}}
\put(1429.0,849.0){\rule[-0.200pt]{2.409pt}{0.400pt}}
\put(170.0,849.0){\rule[-0.200pt]{2.409pt}{0.400pt}}
\put(1429.0,849.0){\rule[-0.200pt]{2.409pt}{0.400pt}}
\put(170.0,849.0){\rule[-0.200pt]{2.409pt}{0.400pt}}
\put(1429.0,849.0){\rule[-0.200pt]{2.409pt}{0.400pt}}
\put(170.0,849.0){\rule[-0.200pt]{2.409pt}{0.400pt}}
\put(1429.0,849.0){\rule[-0.200pt]{2.409pt}{0.400pt}}
\put(170.0,849.0){\rule[-0.200pt]{2.409pt}{0.400pt}}
\put(1429.0,849.0){\rule[-0.200pt]{2.409pt}{0.400pt}}
\put(170.0,849.0){\rule[-0.200pt]{2.409pt}{0.400pt}}
\put(1429.0,849.0){\rule[-0.200pt]{2.409pt}{0.400pt}}
\put(170.0,849.0){\rule[-0.200pt]{2.409pt}{0.400pt}}
\put(1429.0,849.0){\rule[-0.200pt]{2.409pt}{0.400pt}}
\put(170.0,849.0){\rule[-0.200pt]{2.409pt}{0.400pt}}
\put(1429.0,849.0){\rule[-0.200pt]{2.409pt}{0.400pt}}
\put(170.0,849.0){\rule[-0.200pt]{2.409pt}{0.400pt}}
\put(1429.0,849.0){\rule[-0.200pt]{2.409pt}{0.400pt}}
\put(170.0,849.0){\rule[-0.200pt]{2.409pt}{0.400pt}}
\put(1429.0,849.0){\rule[-0.200pt]{2.409pt}{0.400pt}}
\put(170.0,850.0){\rule[-0.200pt]{2.409pt}{0.400pt}}
\put(1429.0,850.0){\rule[-0.200pt]{2.409pt}{0.400pt}}
\put(170.0,850.0){\rule[-0.200pt]{2.409pt}{0.400pt}}
\put(1429.0,850.0){\rule[-0.200pt]{2.409pt}{0.400pt}}
\put(170.0,850.0){\rule[-0.200pt]{2.409pt}{0.400pt}}
\put(1429.0,850.0){\rule[-0.200pt]{2.409pt}{0.400pt}}
\put(170.0,850.0){\rule[-0.200pt]{2.409pt}{0.400pt}}
\put(1429.0,850.0){\rule[-0.200pt]{2.409pt}{0.400pt}}
\put(170.0,850.0){\rule[-0.200pt]{2.409pt}{0.400pt}}
\put(1429.0,850.0){\rule[-0.200pt]{2.409pt}{0.400pt}}
\put(170.0,850.0){\rule[-0.200pt]{2.409pt}{0.400pt}}
\put(1429.0,850.0){\rule[-0.200pt]{2.409pt}{0.400pt}}
\put(170.0,850.0){\rule[-0.200pt]{2.409pt}{0.400pt}}
\put(1429.0,850.0){\rule[-0.200pt]{2.409pt}{0.400pt}}
\put(170.0,850.0){\rule[-0.200pt]{2.409pt}{0.400pt}}
\put(1429.0,850.0){\rule[-0.200pt]{2.409pt}{0.400pt}}
\put(170.0,850.0){\rule[-0.200pt]{2.409pt}{0.400pt}}
\put(1429.0,850.0){\rule[-0.200pt]{2.409pt}{0.400pt}}
\put(170.0,850.0){\rule[-0.200pt]{2.409pt}{0.400pt}}
\put(1429.0,850.0){\rule[-0.200pt]{2.409pt}{0.400pt}}
\put(170.0,850.0){\rule[-0.200pt]{2.409pt}{0.400pt}}
\put(1429.0,850.0){\rule[-0.200pt]{2.409pt}{0.400pt}}
\put(170.0,850.0){\rule[-0.200pt]{2.409pt}{0.400pt}}
\put(1429.0,850.0){\rule[-0.200pt]{2.409pt}{0.400pt}}
\put(170.0,851.0){\rule[-0.200pt]{2.409pt}{0.400pt}}
\put(1429.0,851.0){\rule[-0.200pt]{2.409pt}{0.400pt}}
\put(170.0,851.0){\rule[-0.200pt]{2.409pt}{0.400pt}}
\put(1429.0,851.0){\rule[-0.200pt]{2.409pt}{0.400pt}}
\put(170.0,851.0){\rule[-0.200pt]{2.409pt}{0.400pt}}
\put(1429.0,851.0){\rule[-0.200pt]{2.409pt}{0.400pt}}
\put(170.0,851.0){\rule[-0.200pt]{2.409pt}{0.400pt}}
\put(1429.0,851.0){\rule[-0.200pt]{2.409pt}{0.400pt}}
\put(170.0,851.0){\rule[-0.200pt]{2.409pt}{0.400pt}}
\put(1429.0,851.0){\rule[-0.200pt]{2.409pt}{0.400pt}}
\put(170.0,851.0){\rule[-0.200pt]{2.409pt}{0.400pt}}
\put(1429.0,851.0){\rule[-0.200pt]{2.409pt}{0.400pt}}
\put(170.0,851.0){\rule[-0.200pt]{2.409pt}{0.400pt}}
\put(1429.0,851.0){\rule[-0.200pt]{2.409pt}{0.400pt}}
\put(170.0,851.0){\rule[-0.200pt]{2.409pt}{0.400pt}}
\put(1429.0,851.0){\rule[-0.200pt]{2.409pt}{0.400pt}}
\put(170.0,851.0){\rule[-0.200pt]{2.409pt}{0.400pt}}
\put(1429.0,851.0){\rule[-0.200pt]{2.409pt}{0.400pt}}
\put(170.0,851.0){\rule[-0.200pt]{2.409pt}{0.400pt}}
\put(1429.0,851.0){\rule[-0.200pt]{2.409pt}{0.400pt}}
\put(170.0,851.0){\rule[-0.200pt]{2.409pt}{0.400pt}}
\put(1429.0,851.0){\rule[-0.200pt]{2.409pt}{0.400pt}}
\put(170.0,851.0){\rule[-0.200pt]{2.409pt}{0.400pt}}
\put(1429.0,851.0){\rule[-0.200pt]{2.409pt}{0.400pt}}
\put(170.0,851.0){\rule[-0.200pt]{2.409pt}{0.400pt}}
\put(1429.0,851.0){\rule[-0.200pt]{2.409pt}{0.400pt}}
\put(170.0,852.0){\rule[-0.200pt]{2.409pt}{0.400pt}}
\put(1429.0,852.0){\rule[-0.200pt]{2.409pt}{0.400pt}}
\put(170.0,852.0){\rule[-0.200pt]{2.409pt}{0.400pt}}
\put(1429.0,852.0){\rule[-0.200pt]{2.409pt}{0.400pt}}
\put(170.0,852.0){\rule[-0.200pt]{2.409pt}{0.400pt}}
\put(1429.0,852.0){\rule[-0.200pt]{2.409pt}{0.400pt}}
\put(170.0,852.0){\rule[-0.200pt]{2.409pt}{0.400pt}}
\put(1429.0,852.0){\rule[-0.200pt]{2.409pt}{0.400pt}}
\put(170.0,852.0){\rule[-0.200pt]{2.409pt}{0.400pt}}
\put(1429.0,852.0){\rule[-0.200pt]{2.409pt}{0.400pt}}
\put(170.0,852.0){\rule[-0.200pt]{2.409pt}{0.400pt}}
\put(1429.0,852.0){\rule[-0.200pt]{2.409pt}{0.400pt}}
\put(170.0,852.0){\rule[-0.200pt]{2.409pt}{0.400pt}}
\put(1429.0,852.0){\rule[-0.200pt]{2.409pt}{0.400pt}}
\put(170.0,852.0){\rule[-0.200pt]{2.409pt}{0.400pt}}
\put(1429.0,852.0){\rule[-0.200pt]{2.409pt}{0.400pt}}
\put(170.0,852.0){\rule[-0.200pt]{2.409pt}{0.400pt}}
\put(1429.0,852.0){\rule[-0.200pt]{2.409pt}{0.400pt}}
\put(170.0,852.0){\rule[-0.200pt]{2.409pt}{0.400pt}}
\put(1429.0,852.0){\rule[-0.200pt]{2.409pt}{0.400pt}}
\put(170.0,852.0){\rule[-0.200pt]{2.409pt}{0.400pt}}
\put(1429.0,852.0){\rule[-0.200pt]{2.409pt}{0.400pt}}
\put(170.0,852.0){\rule[-0.200pt]{2.409pt}{0.400pt}}
\put(1429.0,852.0){\rule[-0.200pt]{2.409pt}{0.400pt}}
\put(170.0,852.0){\rule[-0.200pt]{2.409pt}{0.400pt}}
\put(1429.0,852.0){\rule[-0.200pt]{2.409pt}{0.400pt}}
\put(170.0,853.0){\rule[-0.200pt]{2.409pt}{0.400pt}}
\put(1429.0,853.0){\rule[-0.200pt]{2.409pt}{0.400pt}}
\put(170.0,853.0){\rule[-0.200pt]{2.409pt}{0.400pt}}
\put(1429.0,853.0){\rule[-0.200pt]{2.409pt}{0.400pt}}
\put(170.0,853.0){\rule[-0.200pt]{2.409pt}{0.400pt}}
\put(1429.0,853.0){\rule[-0.200pt]{2.409pt}{0.400pt}}
\put(170.0,853.0){\rule[-0.200pt]{2.409pt}{0.400pt}}
\put(1429.0,853.0){\rule[-0.200pt]{2.409pt}{0.400pt}}
\put(170.0,853.0){\rule[-0.200pt]{2.409pt}{0.400pt}}
\put(1429.0,853.0){\rule[-0.200pt]{2.409pt}{0.400pt}}
\put(170.0,853.0){\rule[-0.200pt]{2.409pt}{0.400pt}}
\put(1429.0,853.0){\rule[-0.200pt]{2.409pt}{0.400pt}}
\put(170.0,853.0){\rule[-0.200pt]{2.409pt}{0.400pt}}
\put(1429.0,853.0){\rule[-0.200pt]{2.409pt}{0.400pt}}
\put(170.0,853.0){\rule[-0.200pt]{2.409pt}{0.400pt}}
\put(1429.0,853.0){\rule[-0.200pt]{2.409pt}{0.400pt}}
\put(170.0,853.0){\rule[-0.200pt]{2.409pt}{0.400pt}}
\put(1429.0,853.0){\rule[-0.200pt]{2.409pt}{0.400pt}}
\put(170.0,853.0){\rule[-0.200pt]{2.409pt}{0.400pt}}
\put(1429.0,853.0){\rule[-0.200pt]{2.409pt}{0.400pt}}
\put(170.0,853.0){\rule[-0.200pt]{2.409pt}{0.400pt}}
\put(1429.0,853.0){\rule[-0.200pt]{2.409pt}{0.400pt}}
\put(170.0,853.0){\rule[-0.200pt]{2.409pt}{0.400pt}}
\put(1429.0,853.0){\rule[-0.200pt]{2.409pt}{0.400pt}}
\put(170.0,853.0){\rule[-0.200pt]{2.409pt}{0.400pt}}
\put(1429.0,853.0){\rule[-0.200pt]{2.409pt}{0.400pt}}
\put(170.0,853.0){\rule[-0.200pt]{2.409pt}{0.400pt}}
\put(1429.0,853.0){\rule[-0.200pt]{2.409pt}{0.400pt}}
\put(170.0,853.0){\rule[-0.200pt]{2.409pt}{0.400pt}}
\put(1429.0,853.0){\rule[-0.200pt]{2.409pt}{0.400pt}}
\put(170.0,854.0){\rule[-0.200pt]{2.409pt}{0.400pt}}
\put(1429.0,854.0){\rule[-0.200pt]{2.409pt}{0.400pt}}
\put(170.0,854.0){\rule[-0.200pt]{2.409pt}{0.400pt}}
\put(1429.0,854.0){\rule[-0.200pt]{2.409pt}{0.400pt}}
\put(170.0,854.0){\rule[-0.200pt]{2.409pt}{0.400pt}}
\put(1429.0,854.0){\rule[-0.200pt]{2.409pt}{0.400pt}}
\put(170.0,854.0){\rule[-0.200pt]{2.409pt}{0.400pt}}
\put(1429.0,854.0){\rule[-0.200pt]{2.409pt}{0.400pt}}
\put(170.0,854.0){\rule[-0.200pt]{2.409pt}{0.400pt}}
\put(1429.0,854.0){\rule[-0.200pt]{2.409pt}{0.400pt}}
\put(170.0,854.0){\rule[-0.200pt]{2.409pt}{0.400pt}}
\put(1429.0,854.0){\rule[-0.200pt]{2.409pt}{0.400pt}}
\put(170.0,854.0){\rule[-0.200pt]{2.409pt}{0.400pt}}
\put(1429.0,854.0){\rule[-0.200pt]{2.409pt}{0.400pt}}
\put(170.0,854.0){\rule[-0.200pt]{2.409pt}{0.400pt}}
\put(1429.0,854.0){\rule[-0.200pt]{2.409pt}{0.400pt}}
\put(170.0,854.0){\rule[-0.200pt]{2.409pt}{0.400pt}}
\put(1429.0,854.0){\rule[-0.200pt]{2.409pt}{0.400pt}}
\put(170.0,854.0){\rule[-0.200pt]{2.409pt}{0.400pt}}
\put(1429.0,854.0){\rule[-0.200pt]{2.409pt}{0.400pt}}
\put(170.0,854.0){\rule[-0.200pt]{2.409pt}{0.400pt}}
\put(1429.0,854.0){\rule[-0.200pt]{2.409pt}{0.400pt}}
\put(170.0,854.0){\rule[-0.200pt]{2.409pt}{0.400pt}}
\put(1429.0,854.0){\rule[-0.200pt]{2.409pt}{0.400pt}}
\put(170.0,854.0){\rule[-0.200pt]{2.409pt}{0.400pt}}
\put(1429.0,854.0){\rule[-0.200pt]{2.409pt}{0.400pt}}
\put(170.0,854.0){\rule[-0.200pt]{2.409pt}{0.400pt}}
\put(1429.0,854.0){\rule[-0.200pt]{2.409pt}{0.400pt}}
\put(170.0,854.0){\rule[-0.200pt]{2.409pt}{0.400pt}}
\put(1429.0,854.0){\rule[-0.200pt]{2.409pt}{0.400pt}}
\put(170.0,855.0){\rule[-0.200pt]{2.409pt}{0.400pt}}
\put(1429.0,855.0){\rule[-0.200pt]{2.409pt}{0.400pt}}
\put(170.0,855.0){\rule[-0.200pt]{2.409pt}{0.400pt}}
\put(1429.0,855.0){\rule[-0.200pt]{2.409pt}{0.400pt}}
\put(170.0,855.0){\rule[-0.200pt]{2.409pt}{0.400pt}}
\put(1429.0,855.0){\rule[-0.200pt]{2.409pt}{0.400pt}}
\put(170.0,855.0){\rule[-0.200pt]{2.409pt}{0.400pt}}
\put(1429.0,855.0){\rule[-0.200pt]{2.409pt}{0.400pt}}
\put(170.0,855.0){\rule[-0.200pt]{2.409pt}{0.400pt}}
\put(1429.0,855.0){\rule[-0.200pt]{2.409pt}{0.400pt}}
\put(170.0,855.0){\rule[-0.200pt]{2.409pt}{0.400pt}}
\put(1429.0,855.0){\rule[-0.200pt]{2.409pt}{0.400pt}}
\put(170.0,855.0){\rule[-0.200pt]{2.409pt}{0.400pt}}
\put(1429.0,855.0){\rule[-0.200pt]{2.409pt}{0.400pt}}
\put(170.0,855.0){\rule[-0.200pt]{2.409pt}{0.400pt}}
\put(1429.0,855.0){\rule[-0.200pt]{2.409pt}{0.400pt}}
\put(170.0,855.0){\rule[-0.200pt]{2.409pt}{0.400pt}}
\put(1429.0,855.0){\rule[-0.200pt]{2.409pt}{0.400pt}}
\put(170.0,855.0){\rule[-0.200pt]{2.409pt}{0.400pt}}
\put(1429.0,855.0){\rule[-0.200pt]{2.409pt}{0.400pt}}
\put(170.0,855.0){\rule[-0.200pt]{2.409pt}{0.400pt}}
\put(1429.0,855.0){\rule[-0.200pt]{2.409pt}{0.400pt}}
\put(170.0,855.0){\rule[-0.200pt]{2.409pt}{0.400pt}}
\put(1429.0,855.0){\rule[-0.200pt]{2.409pt}{0.400pt}}
\put(170.0,855.0){\rule[-0.200pt]{2.409pt}{0.400pt}}
\put(1429.0,855.0){\rule[-0.200pt]{2.409pt}{0.400pt}}
\put(170.0,855.0){\rule[-0.200pt]{2.409pt}{0.400pt}}
\put(1429.0,855.0){\rule[-0.200pt]{2.409pt}{0.400pt}}
\put(170.0,855.0){\rule[-0.200pt]{2.409pt}{0.400pt}}
\put(1429.0,855.0){\rule[-0.200pt]{2.409pt}{0.400pt}}
\put(170.0,855.0){\rule[-0.200pt]{2.409pt}{0.400pt}}
\put(1429.0,855.0){\rule[-0.200pt]{2.409pt}{0.400pt}}
\put(170.0,856.0){\rule[-0.200pt]{2.409pt}{0.400pt}}
\put(1429.0,856.0){\rule[-0.200pt]{2.409pt}{0.400pt}}
\put(170.0,856.0){\rule[-0.200pt]{2.409pt}{0.400pt}}
\put(1429.0,856.0){\rule[-0.200pt]{2.409pt}{0.400pt}}
\put(170.0,856.0){\rule[-0.200pt]{2.409pt}{0.400pt}}
\put(1429.0,856.0){\rule[-0.200pt]{2.409pt}{0.400pt}}
\put(170.0,856.0){\rule[-0.200pt]{2.409pt}{0.400pt}}
\put(1429.0,856.0){\rule[-0.200pt]{2.409pt}{0.400pt}}
\put(170.0,856.0){\rule[-0.200pt]{2.409pt}{0.400pt}}
\put(1429.0,856.0){\rule[-0.200pt]{2.409pt}{0.400pt}}
\put(170.0,856.0){\rule[-0.200pt]{2.409pt}{0.400pt}}
\put(1429.0,856.0){\rule[-0.200pt]{2.409pt}{0.400pt}}
\put(170.0,856.0){\rule[-0.200pt]{2.409pt}{0.400pt}}
\put(1429.0,856.0){\rule[-0.200pt]{2.409pt}{0.400pt}}
\put(170.0,856.0){\rule[-0.200pt]{2.409pt}{0.400pt}}
\put(1429.0,856.0){\rule[-0.200pt]{2.409pt}{0.400pt}}
\put(170.0,856.0){\rule[-0.200pt]{2.409pt}{0.400pt}}
\put(1429.0,856.0){\rule[-0.200pt]{2.409pt}{0.400pt}}
\put(170.0,856.0){\rule[-0.200pt]{2.409pt}{0.400pt}}
\put(1429.0,856.0){\rule[-0.200pt]{2.409pt}{0.400pt}}
\put(170.0,856.0){\rule[-0.200pt]{2.409pt}{0.400pt}}
\put(1429.0,856.0){\rule[-0.200pt]{2.409pt}{0.400pt}}
\put(170.0,856.0){\rule[-0.200pt]{2.409pt}{0.400pt}}
\put(1429.0,856.0){\rule[-0.200pt]{2.409pt}{0.400pt}}
\put(170.0,856.0){\rule[-0.200pt]{2.409pt}{0.400pt}}
\put(1429.0,856.0){\rule[-0.200pt]{2.409pt}{0.400pt}}
\put(170.0,856.0){\rule[-0.200pt]{2.409pt}{0.400pt}}
\put(1429.0,856.0){\rule[-0.200pt]{2.409pt}{0.400pt}}
\put(170.0,856.0){\rule[-0.200pt]{2.409pt}{0.400pt}}
\put(1429.0,856.0){\rule[-0.200pt]{2.409pt}{0.400pt}}
\put(170.0,856.0){\rule[-0.200pt]{2.409pt}{0.400pt}}
\put(1429.0,856.0){\rule[-0.200pt]{2.409pt}{0.400pt}}
\put(170.0,856.0){\rule[-0.200pt]{2.409pt}{0.400pt}}
\put(1429.0,856.0){\rule[-0.200pt]{2.409pt}{0.400pt}}
\put(170.0,857.0){\rule[-0.200pt]{2.409pt}{0.400pt}}
\put(1429.0,857.0){\rule[-0.200pt]{2.409pt}{0.400pt}}
\put(170.0,857.0){\rule[-0.200pt]{2.409pt}{0.400pt}}
\put(1429.0,857.0){\rule[-0.200pt]{2.409pt}{0.400pt}}
\put(170.0,857.0){\rule[-0.200pt]{2.409pt}{0.400pt}}
\put(1429.0,857.0){\rule[-0.200pt]{2.409pt}{0.400pt}}
\put(170.0,857.0){\rule[-0.200pt]{2.409pt}{0.400pt}}
\put(1429.0,857.0){\rule[-0.200pt]{2.409pt}{0.400pt}}
\put(170.0,857.0){\rule[-0.200pt]{2.409pt}{0.400pt}}
\put(1429.0,857.0){\rule[-0.200pt]{2.409pt}{0.400pt}}
\put(170.0,857.0){\rule[-0.200pt]{2.409pt}{0.400pt}}
\put(1429.0,857.0){\rule[-0.200pt]{2.409pt}{0.400pt}}
\put(170.0,857.0){\rule[-0.200pt]{2.409pt}{0.400pt}}
\put(1429.0,857.0){\rule[-0.200pt]{2.409pt}{0.400pt}}
\put(170.0,857.0){\rule[-0.200pt]{2.409pt}{0.400pt}}
\put(1429.0,857.0){\rule[-0.200pt]{2.409pt}{0.400pt}}
\put(170.0,857.0){\rule[-0.200pt]{2.409pt}{0.400pt}}
\put(1429.0,857.0){\rule[-0.200pt]{2.409pt}{0.400pt}}
\put(170.0,857.0){\rule[-0.200pt]{2.409pt}{0.400pt}}
\put(1429.0,857.0){\rule[-0.200pt]{2.409pt}{0.400pt}}
\put(170.0,857.0){\rule[-0.200pt]{2.409pt}{0.400pt}}
\put(1429.0,857.0){\rule[-0.200pt]{2.409pt}{0.400pt}}
\put(170.0,857.0){\rule[-0.200pt]{2.409pt}{0.400pt}}
\put(1429.0,857.0){\rule[-0.200pt]{2.409pt}{0.400pt}}
\put(170.0,857.0){\rule[-0.200pt]{2.409pt}{0.400pt}}
\put(1429.0,857.0){\rule[-0.200pt]{2.409pt}{0.400pt}}
\put(170.0,857.0){\rule[-0.200pt]{2.409pt}{0.400pt}}
\put(1429.0,857.0){\rule[-0.200pt]{2.409pt}{0.400pt}}
\put(170.0,857.0){\rule[-0.200pt]{2.409pt}{0.400pt}}
\put(1429.0,857.0){\rule[-0.200pt]{2.409pt}{0.400pt}}
\put(170.0,857.0){\rule[-0.200pt]{2.409pt}{0.400pt}}
\put(1429.0,857.0){\rule[-0.200pt]{2.409pt}{0.400pt}}
\put(170.0,857.0){\rule[-0.200pt]{2.409pt}{0.400pt}}
\put(1429.0,857.0){\rule[-0.200pt]{2.409pt}{0.400pt}}
\put(170.0,857.0){\rule[-0.200pt]{2.409pt}{0.400pt}}
\put(1429.0,857.0){\rule[-0.200pt]{2.409pt}{0.400pt}}
\put(170.0,858.0){\rule[-0.200pt]{2.409pt}{0.400pt}}
\put(1429.0,858.0){\rule[-0.200pt]{2.409pt}{0.400pt}}
\put(170.0,858.0){\rule[-0.200pt]{2.409pt}{0.400pt}}
\put(1429.0,858.0){\rule[-0.200pt]{2.409pt}{0.400pt}}
\put(170.0,858.0){\rule[-0.200pt]{2.409pt}{0.400pt}}
\put(1429.0,858.0){\rule[-0.200pt]{2.409pt}{0.400pt}}
\put(170.0,858.0){\rule[-0.200pt]{2.409pt}{0.400pt}}
\put(1429.0,858.0){\rule[-0.200pt]{2.409pt}{0.400pt}}
\put(170.0,858.0){\rule[-0.200pt]{2.409pt}{0.400pt}}
\put(1429.0,858.0){\rule[-0.200pt]{2.409pt}{0.400pt}}
\put(170.0,858.0){\rule[-0.200pt]{2.409pt}{0.400pt}}
\put(1429.0,858.0){\rule[-0.200pt]{2.409pt}{0.400pt}}
\put(170.0,858.0){\rule[-0.200pt]{2.409pt}{0.400pt}}
\put(1429.0,858.0){\rule[-0.200pt]{2.409pt}{0.400pt}}
\put(170.0,858.0){\rule[-0.200pt]{2.409pt}{0.400pt}}
\put(1429.0,858.0){\rule[-0.200pt]{2.409pt}{0.400pt}}
\put(170.0,858.0){\rule[-0.200pt]{2.409pt}{0.400pt}}
\put(1429.0,858.0){\rule[-0.200pt]{2.409pt}{0.400pt}}
\put(170.0,858.0){\rule[-0.200pt]{2.409pt}{0.400pt}}
\put(1429.0,858.0){\rule[-0.200pt]{2.409pt}{0.400pt}}
\put(170.0,858.0){\rule[-0.200pt]{2.409pt}{0.400pt}}
\put(1429.0,858.0){\rule[-0.200pt]{2.409pt}{0.400pt}}
\put(170.0,858.0){\rule[-0.200pt]{2.409pt}{0.400pt}}
\put(1429.0,858.0){\rule[-0.200pt]{2.409pt}{0.400pt}}
\put(170.0,858.0){\rule[-0.200pt]{2.409pt}{0.400pt}}
\put(1429.0,858.0){\rule[-0.200pt]{2.409pt}{0.400pt}}
\put(170.0,858.0){\rule[-0.200pt]{2.409pt}{0.400pt}}
\put(1429.0,858.0){\rule[-0.200pt]{2.409pt}{0.400pt}}
\put(170.0,858.0){\rule[-0.200pt]{2.409pt}{0.400pt}}
\put(1429.0,858.0){\rule[-0.200pt]{2.409pt}{0.400pt}}
\put(170.0,858.0){\rule[-0.200pt]{2.409pt}{0.400pt}}
\put(1429.0,858.0){\rule[-0.200pt]{2.409pt}{0.400pt}}
\put(170.0,858.0){\rule[-0.200pt]{2.409pt}{0.400pt}}
\put(1429.0,858.0){\rule[-0.200pt]{2.409pt}{0.400pt}}
\put(170.0,858.0){\rule[-0.200pt]{2.409pt}{0.400pt}}
\put(1429.0,858.0){\rule[-0.200pt]{2.409pt}{0.400pt}}
\put(170.0,858.0){\rule[-0.200pt]{2.409pt}{0.400pt}}
\put(1429.0,858.0){\rule[-0.200pt]{2.409pt}{0.400pt}}
\put(170.0,858.0){\rule[-0.200pt]{2.409pt}{0.400pt}}
\put(1429.0,858.0){\rule[-0.200pt]{2.409pt}{0.400pt}}
\put(170.0,859.0){\rule[-0.200pt]{2.409pt}{0.400pt}}
\put(1429.0,859.0){\rule[-0.200pt]{2.409pt}{0.400pt}}
\put(170.0,859.0){\rule[-0.200pt]{2.409pt}{0.400pt}}
\put(1429.0,859.0){\rule[-0.200pt]{2.409pt}{0.400pt}}
\put(170.0,859.0){\rule[-0.200pt]{2.409pt}{0.400pt}}
\put(1429.0,859.0){\rule[-0.200pt]{2.409pt}{0.400pt}}
\put(170.0,859.0){\rule[-0.200pt]{2.409pt}{0.400pt}}
\put(1429.0,859.0){\rule[-0.200pt]{2.409pt}{0.400pt}}
\put(170.0,859.0){\rule[-0.200pt]{2.409pt}{0.400pt}}
\put(1429.0,859.0){\rule[-0.200pt]{2.409pt}{0.400pt}}
\put(170.0,859.0){\rule[-0.200pt]{2.409pt}{0.400pt}}
\put(1429.0,859.0){\rule[-0.200pt]{2.409pt}{0.400pt}}
\put(170.0,859.0){\rule[-0.200pt]{2.409pt}{0.400pt}}
\put(1429.0,859.0){\rule[-0.200pt]{2.409pt}{0.400pt}}
\put(170.0,859.0){\rule[-0.200pt]{2.409pt}{0.400pt}}
\put(1429.0,859.0){\rule[-0.200pt]{2.409pt}{0.400pt}}
\put(170.0,859.0){\rule[-0.200pt]{2.409pt}{0.400pt}}
\put(1429.0,859.0){\rule[-0.200pt]{2.409pt}{0.400pt}}
\put(170.0,859.0){\rule[-0.200pt]{2.409pt}{0.400pt}}
\put(1429.0,859.0){\rule[-0.200pt]{2.409pt}{0.400pt}}
\put(170.0,859.0){\rule[-0.200pt]{4.818pt}{0.400pt}}
\put(150,859){\makebox(0,0)[r]{ 1000}}
\put(1419.0,859.0){\rule[-0.200pt]{4.818pt}{0.400pt}}
\put(170.0,82.0){\rule[-0.200pt]{0.400pt}{4.818pt}}
\put(170,41){\makebox(0,0){ 0}}
\put(170.0,839.0){\rule[-0.200pt]{0.400pt}{4.818pt}}
\put(424.0,82.0){\rule[-0.200pt]{0.400pt}{4.818pt}}
\put(424,41){\makebox(0,0){ 100}}
\put(424.0,839.0){\rule[-0.200pt]{0.400pt}{4.818pt}}
\put(678.0,82.0){\rule[-0.200pt]{0.400pt}{4.818pt}}
\put(678,41){\makebox(0,0){ 200}}
\put(678.0,839.0){\rule[-0.200pt]{0.400pt}{4.818pt}}
\put(931.0,82.0){\rule[-0.200pt]{0.400pt}{4.818pt}}
\put(931,41){\makebox(0,0){ 300}}
\put(931.0,839.0){\rule[-0.200pt]{0.400pt}{4.818pt}}
\put(1185.0,82.0){\rule[-0.200pt]{0.400pt}{4.818pt}}
\put(1185,41){\makebox(0,0){ 400}}
\put(1185.0,839.0){\rule[-0.200pt]{0.400pt}{4.818pt}}
\put(1439.0,82.0){\rule[-0.200pt]{0.400pt}{4.818pt}}
\put(1439,41){\makebox(0,0){ 500}}
\put(1439.0,839.0){\rule[-0.200pt]{0.400pt}{4.818pt}}
\put(170.0,82.0){\rule[-0.200pt]{0.400pt}{187.179pt}}
\put(170.0,82.0){\rule[-0.200pt]{305.702pt}{0.400pt}}
\put(1439.0,82.0){\rule[-0.200pt]{0.400pt}{187.179pt}}
\put(170.0,859.0){\rule[-0.200pt]{305.702pt}{0.400pt}}
\put(1279,205){\makebox(0,0)[r]{algorytm naturalny}}
\put(1299.0,205.0){\rule[-0.200pt]{24.090pt}{0.400pt}}
\put(175,192){\usebox{\plotpoint}}
\multiput(175.58,192.00)(0.493,3.787){23}{\rule{0.119pt}{3.054pt}}
\multiput(174.17,192.00)(13.000,89.662){2}{\rule{0.400pt}{1.527pt}}
\multiput(188.58,288.00)(0.493,1.924){23}{\rule{0.119pt}{1.608pt}}
\multiput(187.17,288.00)(13.000,45.663){2}{\rule{0.400pt}{0.804pt}}
\multiput(201.58,337.00)(0.493,1.210){23}{\rule{0.119pt}{1.054pt}}
\multiput(200.17,337.00)(13.000,28.813){2}{\rule{0.400pt}{0.527pt}}
\multiput(214.58,368.00)(0.493,0.853){23}{\rule{0.119pt}{0.777pt}}
\multiput(213.17,368.00)(13.000,20.387){2}{\rule{0.400pt}{0.388pt}}
\multiput(227.58,390.00)(0.493,0.695){23}{\rule{0.119pt}{0.654pt}}
\multiput(226.17,390.00)(13.000,16.643){2}{\rule{0.400pt}{0.327pt}}
\multiput(240.58,408.00)(0.493,0.536){23}{\rule{0.119pt}{0.531pt}}
\multiput(239.17,408.00)(13.000,12.898){2}{\rule{0.400pt}{0.265pt}}
\multiput(253.00,422.58)(0.539,0.492){21}{\rule{0.533pt}{0.119pt}}
\multiput(253.00,421.17)(11.893,12.000){2}{\rule{0.267pt}{0.400pt}}
\multiput(266.00,434.58)(0.590,0.492){19}{\rule{0.573pt}{0.118pt}}
\multiput(266.00,433.17)(11.811,11.000){2}{\rule{0.286pt}{0.400pt}}
\multiput(279.00,445.59)(0.728,0.489){15}{\rule{0.678pt}{0.118pt}}
\multiput(279.00,444.17)(11.593,9.000){2}{\rule{0.339pt}{0.400pt}}
\multiput(292.00,454.59)(0.824,0.488){13}{\rule{0.750pt}{0.117pt}}
\multiput(292.00,453.17)(11.443,8.000){2}{\rule{0.375pt}{0.400pt}}
\multiput(305.00,462.59)(0.824,0.488){13}{\rule{0.750pt}{0.117pt}}
\multiput(305.00,461.17)(11.443,8.000){2}{\rule{0.375pt}{0.400pt}}
\multiput(318.00,470.59)(0.950,0.485){11}{\rule{0.843pt}{0.117pt}}
\multiput(318.00,469.17)(11.251,7.000){2}{\rule{0.421pt}{0.400pt}}
\multiput(331.00,477.59)(1.123,0.482){9}{\rule{0.967pt}{0.116pt}}
\multiput(331.00,476.17)(10.994,6.000){2}{\rule{0.483pt}{0.400pt}}
\multiput(344.00,483.59)(1.123,0.482){9}{\rule{0.967pt}{0.116pt}}
\multiput(344.00,482.17)(10.994,6.000){2}{\rule{0.483pt}{0.400pt}}
\multiput(357.00,489.59)(1.123,0.482){9}{\rule{0.967pt}{0.116pt}}
\multiput(357.00,488.17)(10.994,6.000){2}{\rule{0.483pt}{0.400pt}}
\multiput(370.00,495.59)(1.378,0.477){7}{\rule{1.140pt}{0.115pt}}
\multiput(370.00,494.17)(10.634,5.000){2}{\rule{0.570pt}{0.400pt}}
\multiput(383.00,500.59)(1.378,0.477){7}{\rule{1.140pt}{0.115pt}}
\multiput(383.00,499.17)(10.634,5.000){2}{\rule{0.570pt}{0.400pt}}
\multiput(396.00,505.60)(1.797,0.468){5}{\rule{1.400pt}{0.113pt}}
\multiput(396.00,504.17)(10.094,4.000){2}{\rule{0.700pt}{0.400pt}}
\multiput(409.00,509.60)(1.943,0.468){5}{\rule{1.500pt}{0.113pt}}
\multiput(409.00,508.17)(10.887,4.000){2}{\rule{0.750pt}{0.400pt}}
\multiput(423.00,513.60)(1.797,0.468){5}{\rule{1.400pt}{0.113pt}}
\multiput(423.00,512.17)(10.094,4.000){2}{\rule{0.700pt}{0.400pt}}
\multiput(436.00,517.60)(1.797,0.468){5}{\rule{1.400pt}{0.113pt}}
\multiput(436.00,516.17)(10.094,4.000){2}{\rule{0.700pt}{0.400pt}}
\multiput(449.00,521.60)(1.797,0.468){5}{\rule{1.400pt}{0.113pt}}
\multiput(449.00,520.17)(10.094,4.000){2}{\rule{0.700pt}{0.400pt}}
\multiput(462.00,525.61)(2.695,0.447){3}{\rule{1.833pt}{0.108pt}}
\multiput(462.00,524.17)(9.195,3.000){2}{\rule{0.917pt}{0.400pt}}
\multiput(475.00,528.60)(1.797,0.468){5}{\rule{1.400pt}{0.113pt}}
\multiput(475.00,527.17)(10.094,4.000){2}{\rule{0.700pt}{0.400pt}}
\multiput(488.00,532.61)(2.695,0.447){3}{\rule{1.833pt}{0.108pt}}
\multiput(488.00,531.17)(9.195,3.000){2}{\rule{0.917pt}{0.400pt}}
\multiput(501.00,535.61)(2.695,0.447){3}{\rule{1.833pt}{0.108pt}}
\multiput(501.00,534.17)(9.195,3.000){2}{\rule{0.917pt}{0.400pt}}
\multiput(514.00,538.61)(2.695,0.447){3}{\rule{1.833pt}{0.108pt}}
\multiput(514.00,537.17)(9.195,3.000){2}{\rule{0.917pt}{0.400pt}}
\multiput(527.00,541.61)(2.695,0.447){3}{\rule{1.833pt}{0.108pt}}
\multiput(527.00,540.17)(9.195,3.000){2}{\rule{0.917pt}{0.400pt}}
\multiput(540.00,544.61)(2.695,0.447){3}{\rule{1.833pt}{0.108pt}}
\multiput(540.00,543.17)(9.195,3.000){2}{\rule{0.917pt}{0.400pt}}
\multiput(553.00,547.61)(2.695,0.447){3}{\rule{1.833pt}{0.108pt}}
\multiput(553.00,546.17)(9.195,3.000){2}{\rule{0.917pt}{0.400pt}}
\put(566,550.17){\rule{2.700pt}{0.400pt}}
\multiput(566.00,549.17)(7.396,2.000){2}{\rule{1.350pt}{0.400pt}}
\multiput(579.00,552.61)(2.695,0.447){3}{\rule{1.833pt}{0.108pt}}
\multiput(579.00,551.17)(9.195,3.000){2}{\rule{0.917pt}{0.400pt}}
\put(592,555.17){\rule{2.700pt}{0.400pt}}
\multiput(592.00,554.17)(7.396,2.000){2}{\rule{1.350pt}{0.400pt}}
\multiput(605.00,557.61)(2.695,0.447){3}{\rule{1.833pt}{0.108pt}}
\multiput(605.00,556.17)(9.195,3.000){2}{\rule{0.917pt}{0.400pt}}
\put(618,560.17){\rule{2.700pt}{0.400pt}}
\multiput(618.00,559.17)(7.396,2.000){2}{\rule{1.350pt}{0.400pt}}
\put(631,562.17){\rule{2.700pt}{0.400pt}}
\multiput(631.00,561.17)(7.396,2.000){2}{\rule{1.350pt}{0.400pt}}
\put(644,564.17){\rule{2.700pt}{0.400pt}}
\multiput(644.00,563.17)(7.396,2.000){2}{\rule{1.350pt}{0.400pt}}
\put(657,566.17){\rule{2.700pt}{0.400pt}}
\multiput(657.00,565.17)(7.396,2.000){2}{\rule{1.350pt}{0.400pt}}
\put(670,568.17){\rule{2.700pt}{0.400pt}}
\multiput(670.00,567.17)(7.396,2.000){2}{\rule{1.350pt}{0.400pt}}
\multiput(683.00,570.61)(2.695,0.447){3}{\rule{1.833pt}{0.108pt}}
\multiput(683.00,569.17)(9.195,3.000){2}{\rule{0.917pt}{0.400pt}}
\put(696,572.67){\rule{3.132pt}{0.400pt}}
\multiput(696.00,572.17)(6.500,1.000){2}{\rule{1.566pt}{0.400pt}}
\put(709,574.17){\rule{2.700pt}{0.400pt}}
\multiput(709.00,573.17)(7.396,2.000){2}{\rule{1.350pt}{0.400pt}}
\put(722,576.17){\rule{2.700pt}{0.400pt}}
\multiput(722.00,575.17)(7.396,2.000){2}{\rule{1.350pt}{0.400pt}}
\put(735,578.17){\rule{2.700pt}{0.400pt}}
\multiput(735.00,577.17)(7.396,2.000){2}{\rule{1.350pt}{0.400pt}}
\put(748,580.17){\rule{2.700pt}{0.400pt}}
\multiput(748.00,579.17)(7.396,2.000){2}{\rule{1.350pt}{0.400pt}}
\put(761,582.17){\rule{2.700pt}{0.400pt}}
\multiput(761.00,581.17)(7.396,2.000){2}{\rule{1.350pt}{0.400pt}}
\put(774,583.67){\rule{3.132pt}{0.400pt}}
\multiput(774.00,583.17)(6.500,1.000){2}{\rule{1.566pt}{0.400pt}}
\put(787,585.17){\rule{2.700pt}{0.400pt}}
\multiput(787.00,584.17)(7.396,2.000){2}{\rule{1.350pt}{0.400pt}}
\put(800,587.17){\rule{2.700pt}{0.400pt}}
\multiput(800.00,586.17)(7.396,2.000){2}{\rule{1.350pt}{0.400pt}}
\put(813,588.67){\rule{3.132pt}{0.400pt}}
\multiput(813.00,588.17)(6.500,1.000){2}{\rule{1.566pt}{0.400pt}}
\put(826,590.17){\rule{2.700pt}{0.400pt}}
\multiput(826.00,589.17)(7.396,2.000){2}{\rule{1.350pt}{0.400pt}}
\put(839,592.17){\rule{2.700pt}{0.400pt}}
\multiput(839.00,591.17)(7.396,2.000){2}{\rule{1.350pt}{0.400pt}}
\put(852,593.67){\rule{3.132pt}{0.400pt}}
\multiput(852.00,593.17)(6.500,1.000){2}{\rule{1.566pt}{0.400pt}}
\put(865,595.17){\rule{2.700pt}{0.400pt}}
\multiput(865.00,594.17)(7.396,2.000){2}{\rule{1.350pt}{0.400pt}}
\put(878,596.67){\rule{3.132pt}{0.400pt}}
\multiput(878.00,596.17)(6.500,1.000){2}{\rule{1.566pt}{0.400pt}}
\put(891,597.67){\rule{3.132pt}{0.400pt}}
\multiput(891.00,597.17)(6.500,1.000){2}{\rule{1.566pt}{0.400pt}}
\put(904,599.17){\rule{2.700pt}{0.400pt}}
\multiput(904.00,598.17)(7.396,2.000){2}{\rule{1.350pt}{0.400pt}}
\put(917,600.67){\rule{3.132pt}{0.400pt}}
\multiput(917.00,600.17)(6.500,1.000){2}{\rule{1.566pt}{0.400pt}}
\put(930,602.17){\rule{2.700pt}{0.400pt}}
\multiput(930.00,601.17)(7.396,2.000){2}{\rule{1.350pt}{0.400pt}}
\put(943,603.67){\rule{3.132pt}{0.400pt}}
\multiput(943.00,603.17)(6.500,1.000){2}{\rule{1.566pt}{0.400pt}}
\put(956,604.67){\rule{3.132pt}{0.400pt}}
\multiput(956.00,604.17)(6.500,1.000){2}{\rule{1.566pt}{0.400pt}}
\put(969,606.17){\rule{2.900pt}{0.400pt}}
\multiput(969.00,605.17)(7.981,2.000){2}{\rule{1.450pt}{0.400pt}}
\put(983,607.67){\rule{3.132pt}{0.400pt}}
\multiput(983.00,607.17)(6.500,1.000){2}{\rule{1.566pt}{0.400pt}}
\put(996,608.67){\rule{3.132pt}{0.400pt}}
\multiput(996.00,608.17)(6.500,1.000){2}{\rule{1.566pt}{0.400pt}}
\put(1009,609.67){\rule{3.132pt}{0.400pt}}
\multiput(1009.00,609.17)(6.500,1.000){2}{\rule{1.566pt}{0.400pt}}
\put(1022,611.17){\rule{2.700pt}{0.400pt}}
\multiput(1022.00,610.17)(7.396,2.000){2}{\rule{1.350pt}{0.400pt}}
\put(1035,612.67){\rule{3.132pt}{0.400pt}}
\multiput(1035.00,612.17)(6.500,1.000){2}{\rule{1.566pt}{0.400pt}}
\put(1048,613.67){\rule{3.132pt}{0.400pt}}
\multiput(1048.00,613.17)(6.500,1.000){2}{\rule{1.566pt}{0.400pt}}
\put(1061,614.67){\rule{3.132pt}{0.400pt}}
\multiput(1061.00,614.17)(6.500,1.000){2}{\rule{1.566pt}{0.400pt}}
\put(1074,615.67){\rule{3.132pt}{0.400pt}}
\multiput(1074.00,615.17)(6.500,1.000){2}{\rule{1.566pt}{0.400pt}}
\put(1087,616.67){\rule{3.132pt}{0.400pt}}
\multiput(1087.00,616.17)(6.500,1.000){2}{\rule{1.566pt}{0.400pt}}
\put(1100,618.17){\rule{2.700pt}{0.400pt}}
\multiput(1100.00,617.17)(7.396,2.000){2}{\rule{1.350pt}{0.400pt}}
\put(1113,619.67){\rule{3.132pt}{0.400pt}}
\multiput(1113.00,619.17)(6.500,1.000){2}{\rule{1.566pt}{0.400pt}}
\put(1126,620.67){\rule{3.132pt}{0.400pt}}
\multiput(1126.00,620.17)(6.500,1.000){2}{\rule{1.566pt}{0.400pt}}
\put(1139,621.67){\rule{3.132pt}{0.400pt}}
\multiput(1139.00,621.17)(6.500,1.000){2}{\rule{1.566pt}{0.400pt}}
\put(1152,622.67){\rule{3.132pt}{0.400pt}}
\multiput(1152.00,622.17)(6.500,1.000){2}{\rule{1.566pt}{0.400pt}}
\put(1165,623.67){\rule{3.132pt}{0.400pt}}
\multiput(1165.00,623.17)(6.500,1.000){2}{\rule{1.566pt}{0.400pt}}
\put(1178,624.67){\rule{3.132pt}{0.400pt}}
\multiput(1178.00,624.17)(6.500,1.000){2}{\rule{1.566pt}{0.400pt}}
\put(1191,625.67){\rule{3.132pt}{0.400pt}}
\multiput(1191.00,625.17)(6.500,1.000){2}{\rule{1.566pt}{0.400pt}}
\put(1204,626.67){\rule{3.132pt}{0.400pt}}
\multiput(1204.00,626.17)(6.500,1.000){2}{\rule{1.566pt}{0.400pt}}
\put(1217,627.67){\rule{3.132pt}{0.400pt}}
\multiput(1217.00,627.17)(6.500,1.000){2}{\rule{1.566pt}{0.400pt}}
\put(1230,628.67){\rule{3.132pt}{0.400pt}}
\multiput(1230.00,628.17)(6.500,1.000){2}{\rule{1.566pt}{0.400pt}}
\put(1243,629.67){\rule{3.132pt}{0.400pt}}
\multiput(1243.00,629.17)(6.500,1.000){2}{\rule{1.566pt}{0.400pt}}
\put(1256,630.67){\rule{3.132pt}{0.400pt}}
\multiput(1256.00,630.17)(6.500,1.000){2}{\rule{1.566pt}{0.400pt}}
\put(1269,631.67){\rule{3.132pt}{0.400pt}}
\multiput(1269.00,631.17)(6.500,1.000){2}{\rule{1.566pt}{0.400pt}}
\put(1282,632.67){\rule{3.132pt}{0.400pt}}
\multiput(1282.00,632.17)(6.500,1.000){2}{\rule{1.566pt}{0.400pt}}
\put(1295,633.67){\rule{3.132pt}{0.400pt}}
\multiput(1295.00,633.17)(6.500,1.000){2}{\rule{1.566pt}{0.400pt}}
\put(1308,634.67){\rule{3.132pt}{0.400pt}}
\multiput(1308.00,634.17)(6.500,1.000){2}{\rule{1.566pt}{0.400pt}}
\put(1321,635.67){\rule{3.132pt}{0.400pt}}
\multiput(1321.00,635.17)(6.500,1.000){2}{\rule{1.566pt}{0.400pt}}
\put(1347,636.67){\rule{3.132pt}{0.400pt}}
\multiput(1347.00,636.17)(6.500,1.000){2}{\rule{1.566pt}{0.400pt}}
\put(1360,637.67){\rule{3.132pt}{0.400pt}}
\multiput(1360.00,637.17)(6.500,1.000){2}{\rule{1.566pt}{0.400pt}}
\put(1373,638.67){\rule{3.132pt}{0.400pt}}
\multiput(1373.00,638.17)(6.500,1.000){2}{\rule{1.566pt}{0.400pt}}
\put(1386,639.67){\rule{3.132pt}{0.400pt}}
\multiput(1386.00,639.17)(6.500,1.000){2}{\rule{1.566pt}{0.400pt}}
\put(1399,640.67){\rule{3.132pt}{0.400pt}}
\multiput(1399.00,640.17)(6.500,1.000){2}{\rule{1.566pt}{0.400pt}}
\put(1412,641.67){\rule{3.132pt}{0.400pt}}
\multiput(1412.00,641.17)(6.500,1.000){2}{\rule{1.566pt}{0.400pt}}
\put(1334.0,637.0){\rule[-0.200pt]{3.132pt}{0.400pt}}
\put(1438,642.67){\rule{0.241pt}{0.400pt}}
\multiput(1438.00,642.17)(0.500,1.000){2}{\rule{0.120pt}{0.400pt}}
\put(1425.0,643.0){\rule[-0.200pt]{3.132pt}{0.400pt}}
\sbox{\plotpoint}{\rule[-0.500pt]{1.000pt}{1.000pt}}%
\sbox{\plotpoint}{\rule[-0.200pt]{0.400pt}{0.400pt}}%
\put(1279,164){\makebox(0,0)[r]{algorytm Strassena}}
\sbox{\plotpoint}{\rule[-0.500pt]{1.000pt}{1.000pt}}%
\multiput(1299,164)(20.756,0.000){5}{\usebox{\plotpoint}}
\put(1399,164){\usebox{\plotpoint}}
\put(175,160){\usebox{\plotpoint}}
\multiput(175,160)(1.620,20.692){9}{\usebox{\plotpoint}}
\multiput(188,326)(3.591,20.442){3}{\usebox{\plotpoint}}
\multiput(201,400)(6.415,19.739){2}{\usebox{\plotpoint}}
\multiput(214,440)(9.004,18.701){2}{\usebox{\plotpoint}}
\put(236.70,482.66){\usebox{\plotpoint}}
\put(248.48,499.74){\usebox{\plotpoint}}
\put(261.53,515.85){\usebox{\plotpoint}}
\put(275.48,531.21){\usebox{\plotpoint}}
\put(290.90,545.07){\usebox{\plotpoint}}
\put(306.88,558.30){\usebox{\plotpoint}}
\put(324.16,569.79){\usebox{\plotpoint}}
\put(341.83,580.67){\usebox{\plotpoint}}
\put(360.13,590.44){\usebox{\plotpoint}}
\put(378.97,599.14){\usebox{\plotpoint}}
\put(397.82,607.84){\usebox{\plotpoint}}
\put(416.95,615.84){\usebox{\plotpoint}}
\put(436.37,623.14){\usebox{\plotpoint}}
\put(455.75,630.59){\usebox{\plotpoint}}
\put(475.43,637.13){\usebox{\plotpoint}}
\put(495.27,643.24){\usebox{\plotpoint}}
\put(515.36,648.42){\usebox{\plotpoint}}
\put(535.35,653.93){\usebox{\plotpoint}}
\put(555.53,658.78){\usebox{\plotpoint}}
\put(575.55,664.20){\usebox{\plotpoint}}
\put(595.77,668.87){\usebox{\plotpoint}}
\put(616.00,673.54){\usebox{\plotpoint}}
\put(636.22,678.20){\usebox{\plotpoint}}
\put(656.45,682.87){\usebox{\plotpoint}}
\put(676.85,686.58){\usebox{\plotpoint}}
\put(697.09,691.17){\usebox{\plotpoint}}
\put(717.48,694.96){\usebox{\plotpoint}}
\put(737.89,698.67){\usebox{\plotpoint}}
\put(758.26,702.58){\usebox{\plotpoint}}
\put(778.78,705.73){\usebox{\plotpoint}}
\put(799.29,708.89){\usebox{\plotpoint}}
\put(819.86,711.53){\usebox{\plotpoint}}
\put(840.43,714.22){\usebox{\plotpoint}}
\put(861.02,716.69){\usebox{\plotpoint}}
\put(881.60,719.28){\usebox{\plotpoint}}
\put(902.20,721.72){\usebox{\plotpoint}}
\put(922.83,723.90){\usebox{\plotpoint}}
\put(943.46,726.07){\usebox{\plotpoint}}
\put(964.04,728.62){\usebox{\plotpoint}}
\put(984.62,731.25){\usebox{\plotpoint}}
\put(1005.21,733.71){\usebox{\plotpoint}}
\put(1025.79,736.29){\usebox{\plotpoint}}
\put(1046.39,738.75){\usebox{\plotpoint}}
\put(1066.95,741.46){\usebox{\plotpoint}}
\put(1087.53,744.04){\usebox{\plotpoint}}
\put(1108.16,746.25){\usebox{\plotpoint}}
\put(1128.78,748.43){\usebox{\plotpoint}}
\put(1149.39,750.80){\usebox{\plotpoint}}
\put(1170.04,752.78){\usebox{\plotpoint}}
\put(1190.66,754.97){\usebox{\plotpoint}}
\put(1211.29,757.12){\usebox{\plotpoint}}
\put(1231.94,759.15){\usebox{\plotpoint}}
\put(1252.55,761.47){\usebox{\plotpoint}}
\put(1273.21,763.32){\usebox{\plotpoint}}
\put(1293.91,764.92){\usebox{\plotpoint}}
\put(1314.54,767.01){\usebox{\plotpoint}}
\put(1335.18,769.09){\usebox{\plotpoint}}
\put(1355.88,770.68){\usebox{\plotpoint}}
\put(1376.57,772.27){\usebox{\plotpoint}}
\put(1397.26,773.87){\usebox{\plotpoint}}
\put(1417.84,776.45){\usebox{\plotpoint}}
\put(1438.54,778.00){\usebox{\plotpoint}}
\put(1439,778){\usebox{\plotpoint}}
\sbox{\plotpoint}{\rule[-0.600pt]{1.200pt}{1.200pt}}%
\sbox{\plotpoint}{\rule[-0.200pt]{0.400pt}{0.400pt}}%
\put(1279,123){\makebox(0,0)[r]{algorytm z progiem}}
\sbox{\plotpoint}{\rule[-0.600pt]{1.200pt}{1.200pt}}%
\put(1299.0,123.0){\rule[-0.600pt]{24.090pt}{1.200pt}}
\put(175,192){\usebox{\plotpoint}}
\multiput(177.24,192.00)(0.501,3.865){16}{\rule{0.121pt}{9.162pt}}
\multiput(172.51,192.00)(13.000,76.985){2}{\rule{1.200pt}{4.581pt}}
\multiput(190.24,288.00)(0.501,1.932){16}{\rule{0.121pt}{4.823pt}}
\multiput(185.51,288.00)(13.000,38.989){2}{\rule{1.200pt}{2.412pt}}
\multiput(203.24,337.00)(0.501,1.192){16}{\rule{0.121pt}{3.162pt}}
\multiput(198.51,337.00)(13.000,24.438){2}{\rule{1.200pt}{1.581pt}}
\multiput(216.24,368.00)(0.501,0.822){16}{\rule{0.121pt}{2.331pt}}
\multiput(211.51,368.00)(13.000,17.162){2}{\rule{1.200pt}{1.165pt}}
\multiput(229.24,390.00)(0.501,0.657){16}{\rule{0.121pt}{1.962pt}}
\multiput(224.51,390.00)(13.000,13.929){2}{\rule{1.200pt}{0.981pt}}
\multiput(242.24,408.00)(0.501,0.493){16}{\rule{0.121pt}{1.592pt}}
\multiput(237.51,408.00)(13.000,10.695){2}{\rule{1.200pt}{0.796pt}}
\multiput(253.00,424.24)(0.489,0.501){14}{\rule{1.600pt}{0.121pt}}
\multiput(253.00,419.51)(9.679,12.000){2}{\rule{0.800pt}{1.200pt}}
\multiput(266.00,436.24)(0.533,0.502){12}{\rule{1.718pt}{0.121pt}}
\multiput(266.00,431.51)(9.434,11.000){2}{\rule{0.859pt}{1.200pt}}
\multiput(279.00,447.24)(0.651,0.502){8}{\rule{2.033pt}{0.121pt}}
\multiput(279.00,442.51)(8.780,9.000){2}{\rule{1.017pt}{1.200pt}}
\multiput(292.00,456.24)(0.732,0.503){6}{\rule{2.250pt}{0.121pt}}
\multiput(292.00,451.51)(8.330,8.000){2}{\rule{1.125pt}{1.200pt}}
\multiput(305.00,464.24)(0.732,0.503){6}{\rule{2.250pt}{0.121pt}}
\multiput(305.00,459.51)(8.330,8.000){2}{\rule{1.125pt}{1.200pt}}
\multiput(318.00,472.24)(0.835,0.505){4}{\rule{2.529pt}{0.122pt}}
\multiput(318.00,467.51)(7.752,7.000){2}{\rule{1.264pt}{1.200pt}}
\multiput(331.00,479.24)(0.962,0.509){2}{\rule{2.900pt}{0.123pt}}
\multiput(331.00,474.51)(6.981,6.000){2}{\rule{1.450pt}{1.200pt}}
\multiput(344.00,485.24)(0.962,0.509){2}{\rule{2.900pt}{0.123pt}}
\multiput(344.00,480.51)(6.981,6.000){2}{\rule{1.450pt}{1.200pt}}
\multiput(357.00,491.24)(0.962,0.509){2}{\rule{2.900pt}{0.123pt}}
\multiput(357.00,486.51)(6.981,6.000){2}{\rule{1.450pt}{1.200pt}}
\put(370,495.01){\rule{3.132pt}{1.200pt}}
\multiput(370.00,492.51)(6.500,5.000){2}{\rule{1.566pt}{1.200pt}}
\put(383,500.01){\rule{3.132pt}{1.200pt}}
\multiput(383.00,497.51)(6.500,5.000){2}{\rule{1.566pt}{1.200pt}}
\put(396,504.51){\rule{3.132pt}{1.200pt}}
\multiput(396.00,502.51)(6.500,4.000){2}{\rule{1.566pt}{1.200pt}}
\put(409,508.51){\rule{3.373pt}{1.200pt}}
\multiput(409.00,506.51)(7.000,4.000){2}{\rule{1.686pt}{1.200pt}}
\put(423,512.51){\rule{3.132pt}{1.200pt}}
\multiput(423.00,510.51)(6.500,4.000){2}{\rule{1.566pt}{1.200pt}}
\put(436,516.51){\rule{3.132pt}{1.200pt}}
\multiput(436.00,514.51)(6.500,4.000){2}{\rule{1.566pt}{1.200pt}}
\put(449,520.51){\rule{3.132pt}{1.200pt}}
\multiput(449.00,518.51)(6.500,4.000){2}{\rule{1.566pt}{1.200pt}}
\put(462,524.01){\rule{3.132pt}{1.200pt}}
\multiput(462.00,522.51)(6.500,3.000){2}{\rule{1.566pt}{1.200pt}}
\put(475,527.51){\rule{3.132pt}{1.200pt}}
\multiput(475.00,525.51)(6.500,4.000){2}{\rule{1.566pt}{1.200pt}}
\put(488,531.01){\rule{3.132pt}{1.200pt}}
\multiput(488.00,529.51)(6.500,3.000){2}{\rule{1.566pt}{1.200pt}}
\put(501,534.51){\rule{3.132pt}{1.200pt}}
\multiput(501.00,532.51)(6.500,4.000){2}{\rule{1.566pt}{1.200pt}}
\put(514,538.01){\rule{3.132pt}{1.200pt}}
\multiput(514.00,536.51)(6.500,3.000){2}{\rule{1.566pt}{1.200pt}}
\put(527,541.51){\rule{3.132pt}{1.200pt}}
\multiput(527.00,539.51)(6.500,4.000){2}{\rule{1.566pt}{1.200pt}}
\put(540,545.01){\rule{3.132pt}{1.200pt}}
\multiput(540.00,543.51)(6.500,3.000){2}{\rule{1.566pt}{1.200pt}}
\put(553,548.01){\rule{3.132pt}{1.200pt}}
\multiput(553.00,546.51)(6.500,3.000){2}{\rule{1.566pt}{1.200pt}}
\put(566,551.51){\rule{3.132pt}{1.200pt}}
\multiput(566.00,549.51)(6.500,4.000){2}{\rule{1.566pt}{1.200pt}}
\put(579,555.01){\rule{3.132pt}{1.200pt}}
\multiput(579.00,553.51)(6.500,3.000){2}{\rule{1.566pt}{1.200pt}}
\put(592,558.01){\rule{3.132pt}{1.200pt}}
\multiput(592.00,556.51)(6.500,3.000){2}{\rule{1.566pt}{1.200pt}}
\put(605,561.01){\rule{3.132pt}{1.200pt}}
\multiput(605.00,559.51)(6.500,3.000){2}{\rule{1.566pt}{1.200pt}}
\put(618,564.01){\rule{3.132pt}{1.200pt}}
\multiput(618.00,562.51)(6.500,3.000){2}{\rule{1.566pt}{1.200pt}}
\put(631,567.01){\rule{3.132pt}{1.200pt}}
\multiput(631.00,565.51)(6.500,3.000){2}{\rule{1.566pt}{1.200pt}}
\put(644,570.01){\rule{3.132pt}{1.200pt}}
\multiput(644.00,568.51)(6.500,3.000){2}{\rule{1.566pt}{1.200pt}}
\put(657,573.01){\rule{3.132pt}{1.200pt}}
\multiput(657.00,571.51)(6.500,3.000){2}{\rule{1.566pt}{1.200pt}}
\put(670,575.51){\rule{3.132pt}{1.200pt}}
\multiput(670.00,574.51)(6.500,2.000){2}{\rule{1.566pt}{1.200pt}}
\put(683,578.01){\rule{3.132pt}{1.200pt}}
\multiput(683.00,576.51)(6.500,3.000){2}{\rule{1.566pt}{1.200pt}}
\put(696,580.51){\rule{3.132pt}{1.200pt}}
\multiput(696.00,579.51)(6.500,2.000){2}{\rule{1.566pt}{1.200pt}}
\put(709,583.01){\rule{3.132pt}{1.200pt}}
\multiput(709.00,581.51)(6.500,3.000){2}{\rule{1.566pt}{1.200pt}}
\put(722,585.51){\rule{3.132pt}{1.200pt}}
\multiput(722.00,584.51)(6.500,2.000){2}{\rule{1.566pt}{1.200pt}}
\put(735,587.51){\rule{3.132pt}{1.200pt}}
\multiput(735.00,586.51)(6.500,2.000){2}{\rule{1.566pt}{1.200pt}}
\put(748,589.51){\rule{3.132pt}{1.200pt}}
\multiput(748.00,588.51)(6.500,2.000){2}{\rule{1.566pt}{1.200pt}}
\put(761,591.51){\rule{3.132pt}{1.200pt}}
\multiput(761.00,590.51)(6.500,2.000){2}{\rule{1.566pt}{1.200pt}}
\put(774,593.51){\rule{3.132pt}{1.200pt}}
\multiput(774.00,592.51)(6.500,2.000){2}{\rule{1.566pt}{1.200pt}}
\put(787,595.01){\rule{3.132pt}{1.200pt}}
\multiput(787.00,594.51)(6.500,1.000){2}{\rule{1.566pt}{1.200pt}}
\put(800,596.51){\rule{3.132pt}{1.200pt}}
\multiput(800.00,595.51)(6.500,2.000){2}{\rule{1.566pt}{1.200pt}}
\put(813,598.51){\rule{3.132pt}{1.200pt}}
\multiput(813.00,597.51)(6.500,2.000){2}{\rule{1.566pt}{1.200pt}}
\put(826,600.01){\rule{3.132pt}{1.200pt}}
\multiput(826.00,599.51)(6.500,1.000){2}{\rule{1.566pt}{1.200pt}}
\put(839,601.51){\rule{3.132pt}{1.200pt}}
\multiput(839.00,600.51)(6.500,2.000){2}{\rule{1.566pt}{1.200pt}}
\put(852,603.01){\rule{3.132pt}{1.200pt}}
\multiput(852.00,602.51)(6.500,1.000){2}{\rule{1.566pt}{1.200pt}}
\put(865,604.51){\rule{3.132pt}{1.200pt}}
\multiput(865.00,603.51)(6.500,2.000){2}{\rule{1.566pt}{1.200pt}}
\put(878,606.01){\rule{3.132pt}{1.200pt}}
\multiput(878.00,605.51)(6.500,1.000){2}{\rule{1.566pt}{1.200pt}}
\put(891,607.51){\rule{3.132pt}{1.200pt}}
\multiput(891.00,606.51)(6.500,2.000){2}{\rule{1.566pt}{1.200pt}}
\put(904,609.51){\rule{3.132pt}{1.200pt}}
\multiput(904.00,608.51)(6.500,2.000){2}{\rule{1.566pt}{1.200pt}}
\put(917,611.01){\rule{3.132pt}{1.200pt}}
\multiput(917.00,610.51)(6.500,1.000){2}{\rule{1.566pt}{1.200pt}}
\put(930,612.51){\rule{3.132pt}{1.200pt}}
\multiput(930.00,611.51)(6.500,2.000){2}{\rule{1.566pt}{1.200pt}}
\put(943,614.51){\rule{3.132pt}{1.200pt}}
\multiput(943.00,613.51)(6.500,2.000){2}{\rule{1.566pt}{1.200pt}}
\put(956,616.01){\rule{3.132pt}{1.200pt}}
\multiput(956.00,615.51)(6.500,1.000){2}{\rule{1.566pt}{1.200pt}}
\put(969,617.51){\rule{3.373pt}{1.200pt}}
\multiput(969.00,616.51)(7.000,2.000){2}{\rule{1.686pt}{1.200pt}}
\put(983,619.51){\rule{3.132pt}{1.200pt}}
\multiput(983.00,618.51)(6.500,2.000){2}{\rule{1.566pt}{1.200pt}}
\put(996,621.01){\rule{3.132pt}{1.200pt}}
\multiput(996.00,620.51)(6.500,1.000){2}{\rule{1.566pt}{1.200pt}}
\put(1009,622.51){\rule{3.132pt}{1.200pt}}
\multiput(1009.00,621.51)(6.500,2.000){2}{\rule{1.566pt}{1.200pt}}
\put(1022,624.01){\rule{3.132pt}{1.200pt}}
\multiput(1022.00,623.51)(6.500,1.000){2}{\rule{1.566pt}{1.200pt}}
\put(1035,625.51){\rule{3.132pt}{1.200pt}}
\multiput(1035.00,624.51)(6.500,2.000){2}{\rule{1.566pt}{1.200pt}}
\put(1048,627.51){\rule{3.132pt}{1.200pt}}
\multiput(1048.00,626.51)(6.500,2.000){2}{\rule{1.566pt}{1.200pt}}
\put(1061,629.01){\rule{3.132pt}{1.200pt}}
\multiput(1061.00,628.51)(6.500,1.000){2}{\rule{1.566pt}{1.200pt}}
\put(1074,630.51){\rule{3.132pt}{1.200pt}}
\multiput(1074.00,629.51)(6.500,2.000){2}{\rule{1.566pt}{1.200pt}}
\put(1087,632.01){\rule{3.132pt}{1.200pt}}
\multiput(1087.00,631.51)(6.500,1.000){2}{\rule{1.566pt}{1.200pt}}
\put(1100,633.51){\rule{3.132pt}{1.200pt}}
\multiput(1100.00,632.51)(6.500,2.000){2}{\rule{1.566pt}{1.200pt}}
\put(1113,635.01){\rule{3.132pt}{1.200pt}}
\multiput(1113.00,634.51)(6.500,1.000){2}{\rule{1.566pt}{1.200pt}}
\put(1126,636.51){\rule{3.132pt}{1.200pt}}
\multiput(1126.00,635.51)(6.500,2.000){2}{\rule{1.566pt}{1.200pt}}
\put(1139,638.01){\rule{3.132pt}{1.200pt}}
\multiput(1139.00,637.51)(6.500,1.000){2}{\rule{1.566pt}{1.200pt}}
\put(1152,639.01){\rule{3.132pt}{1.200pt}}
\multiput(1152.00,638.51)(6.500,1.000){2}{\rule{1.566pt}{1.200pt}}
\put(1165,640.51){\rule{3.132pt}{1.200pt}}
\multiput(1165.00,639.51)(6.500,2.000){2}{\rule{1.566pt}{1.200pt}}
\put(1178,642.01){\rule{3.132pt}{1.200pt}}
\multiput(1178.00,641.51)(6.500,1.000){2}{\rule{1.566pt}{1.200pt}}
\put(1191,643.51){\rule{3.132pt}{1.200pt}}
\multiput(1191.00,642.51)(6.500,2.000){2}{\rule{1.566pt}{1.200pt}}
\put(1204,645.01){\rule{3.132pt}{1.200pt}}
\multiput(1204.00,644.51)(6.500,1.000){2}{\rule{1.566pt}{1.200pt}}
\put(1217,646.01){\rule{3.132pt}{1.200pt}}
\multiput(1217.00,645.51)(6.500,1.000){2}{\rule{1.566pt}{1.200pt}}
\put(1230,647.51){\rule{3.132pt}{1.200pt}}
\multiput(1230.00,646.51)(6.500,2.000){2}{\rule{1.566pt}{1.200pt}}
\put(1243,649.01){\rule{3.132pt}{1.200pt}}
\multiput(1243.00,648.51)(6.500,1.000){2}{\rule{1.566pt}{1.200pt}}
\put(1256,650.01){\rule{3.132pt}{1.200pt}}
\multiput(1256.00,649.51)(6.500,1.000){2}{\rule{1.566pt}{1.200pt}}
\put(1269,651.01){\rule{3.132pt}{1.200pt}}
\multiput(1269.00,650.51)(6.500,1.000){2}{\rule{1.566pt}{1.200pt}}
\put(1282,652.01){\rule{3.132pt}{1.200pt}}
\multiput(1282.00,651.51)(6.500,1.000){2}{\rule{1.566pt}{1.200pt}}
\put(1295,653.51){\rule{3.132pt}{1.200pt}}
\multiput(1295.00,652.51)(6.500,2.000){2}{\rule{1.566pt}{1.200pt}}
\put(1308,655.01){\rule{3.132pt}{1.200pt}}
\multiput(1308.00,654.51)(6.500,1.000){2}{\rule{1.566pt}{1.200pt}}
\put(1321,656.01){\rule{3.132pt}{1.200pt}}
\multiput(1321.00,655.51)(6.500,1.000){2}{\rule{1.566pt}{1.200pt}}
\put(1334,657.01){\rule{3.132pt}{1.200pt}}
\multiput(1334.00,656.51)(6.500,1.000){2}{\rule{1.566pt}{1.200pt}}
\put(1347,658.01){\rule{3.132pt}{1.200pt}}
\multiput(1347.00,657.51)(6.500,1.000){2}{\rule{1.566pt}{1.200pt}}
\put(1360,659.01){\rule{3.132pt}{1.200pt}}
\multiput(1360.00,658.51)(6.500,1.000){2}{\rule{1.566pt}{1.200pt}}
\put(1373,660.01){\rule{3.132pt}{1.200pt}}
\multiput(1373.00,659.51)(6.500,1.000){2}{\rule{1.566pt}{1.200pt}}
\put(1386,661.01){\rule{3.132pt}{1.200pt}}
\multiput(1386.00,660.51)(6.500,1.000){2}{\rule{1.566pt}{1.200pt}}
\put(1399,662.01){\rule{3.132pt}{1.200pt}}
\multiput(1399.00,661.51)(6.500,1.000){2}{\rule{1.566pt}{1.200pt}}
\put(1412,663.01){\rule{3.132pt}{1.200pt}}
\multiput(1412.00,662.51)(6.500,1.000){2}{\rule{1.566pt}{1.200pt}}
\put(1425,664.01){\rule{3.132pt}{1.200pt}}
\multiput(1425.00,663.51)(6.500,1.000){2}{\rule{1.566pt}{1.200pt}}
\put(1438.0,667.0){\usebox{\plotpoint}}
\sbox{\plotpoint}{\rule[-0.200pt]{0.400pt}{0.400pt}}%
\put(170.0,82.0){\rule[-0.200pt]{0.400pt}{187.179pt}}
\put(170.0,82.0){\rule[-0.200pt]{305.702pt}{0.400pt}}
\put(1439.0,82.0){\rule[-0.200pt]{0.400pt}{187.179pt}}
\put(170.0,859.0){\rule[-0.200pt]{305.702pt}{0.400pt}}
\end{picture}

\caption{Wykres zależności współczynnika $\Delta((XY)V-X(YV))$ od wielkości macierzy w arytmetyce single}
\end{center}
\end{figure}
