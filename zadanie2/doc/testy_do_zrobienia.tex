\documentclass{article}

\usepackage{polski}
\usepackage{a4wide}
\usepackage[utf8]{inputenc}


\begin{document}

Obrazki do testowania:
\begin{itemize}
  \item Funkcja Rungego 2D
  \item Szachownica
  \item Sinusy 2D
  \item Jakieś zdjęcie
\end{itemize}

Algorytmy do testowania:
\begin{itemize}
  \item Liniowy
  \item Najbliższy sąsiad
  \item Spline
\end{itemize}


Testy:
\begin{enumerate}
  \item Wydajność - dla jakiegoś obrazka wykonujemy zmianę wielkości wszystkimi trzema metodami,
  od 10 do 1000\% i mierzymy czas.

  \item Dokładność - norma dla obrazków proceduralnych dla wielu różnych zmian rozdzielczości dla 
  różnych kolejności, optyczne porównanie wyników dla najgorszej/najlepszej metody i pokemonów.

  \item Kolejność - norma obrazu różnicowego kolejności dla wszystkich metod i różnych konfiguracji
  zmian, + obraz różnicowy pokemonów dla skrajnego przypadka

\end{enumerate}

\end{document}
