\section{Jakość algorytmów}
Pierwszym zadaniem, które wykonaliśmy, polegało na porównaniu jakości obrazów po
zmianie jego rozdzielczości za pomocą trzech powyżej opisanych metod. Rysunek
\textbf{3.1} jest obrazem, którego użyliśmy przy wstępnych badaniach.
\begin{figure}[h!tb]
\begin{center}
\includegraphics{../obrazki/liscie_1/liscie.jpg}
\caption{Liście - pierwszy rysunek testowy.}
\end{center}
\end{figure}

Powiększyliśmy ten obraz o pięciokrotnie w pionie i poziomie używając
kolejnych algorytmów. Wynik tej operacji prezentuje rysunek \textbf{3.2}.
\begin{figure}[h!tb]
\begin{center}
\subfigure[NS]{
\includegraphics[width=4cm]{../obrazki/liscie_1/liscie_500_round.jpg}
}
\subfigure[AFS(I)]{
\includegraphics[width=4cm]{../obrazki/liscie_1/liscie_500_linear.jpg}
}
\subfigure[AFS(III)]{
\includegraphics[width=4cm]{../obrazki/liscie_1/liscie_500_cubic.jpg}
}
\caption{Fragmenty obrazków będących pięciokrotnym powiększeniem rysunku \textbf{3.1}.}
\end{center}
\end{figure}

Całościowe wyniki tego doświadczenia znajdują się w katalogu \textit{obrazki$\slash$liscie\_1}. 
Jak można od razu zauważyć, metoda NS dała najgorszy pod względem wizualnym wynik.
Widoczny jest efekt 'schodków', czyli wygląd krawędzi daleko odbiega od naszych
oczekiwań. Gdy spojrzymy na pozostałe dwie metody, dają one wynik dość zbliżony,
przy czym AFS(III) zdaje się w mniejszym stopniu powodować efekt rozmycia.

Po wykonaniu tego wstępnego rozpoznania problemu, postanowiliśmy sprawdzić, jak
poszczególne algorytmy zachowają się wobec pisanego tekstu.
Fragment tekstu, którego użyliśmy do naszych eksperymentów, jak i uzyskane wyniki znajdują się w katalogu \textit{obrazki$slash$tekst}.

Tym razem rozpoczęliśmy od zmniejszania obrazka. Spośród sprawdzonych wartości,
jako przykład do prezentacji wybraliśmy skalowanie do $70\%$ oryginalnego
rozmiaru, ponieważ tekst wtedy dla wszystkich metod pozostaje czytelny, a
widoczne są już uboczne efekty zmiany rozdzielczości. Fragment otrzymanych
wyników przedstawia rysunek \textbf{3.3}. Ponownie dla metody NS obserwujemy
efekt 'schodków' i przez to wyraźną stratę na jakości obrazu. Metoda AFS(I)
wydaje się zaś rozmazywać poszczególne znaki. Choć nieidealna, metoda AFS(III),
zdaje się radzić sobie najlepiej jeśli chodzi o jakość wyniku. Analogiczne efekty
otrzymujemy przy powiększaniu tekstu. Rysunek \textbf{3.4} prezentuje
fragmenty powyższego dokumentu powiększone dwukrotnie za pomocą wszystkich trzech metod. Chcąc
podsumować nasze wizualne obserwacje, możemy stwierdzić, że najgorszym algorytmem pod względem zachowania
jakości jest algorytm NS, najlepszym zaś zdaje się być algorytm AFS(III).
\begin{figure}[h!tb]
\begin{center}
\subfigure[NS]{
\includegraphics[width=5cm]{../obrazki/tekst/tekst_70_round.jpg}
}
\subfigure[AFS(I)]{
\includegraphics[width=5cm]{../obrazki/tekst/tekst_70_linear.jpg}
}
\subfigure[AFS(III)]{
\includegraphics[width=5cm]{../obrazki/tekst/tekst_70_cubic.jpg}
}
\caption{Pomniejszenie tekstu za pomocą trzech metod.}
\end{center}
\end{figure}
\begin{figure}
\begin{center}
\subfigure[NS]{
\includegraphics{../obrazki/tekst/tekst_200_round.jpg}
}
\subfigure[AFS(I)]{
\includegraphics{../obrazki/tekst/tekst_200_linear.jpg}
}
\subfigure[AFS(III)]{
\includegraphics{../obrazki/tekst/tekst_200_cubic.jpg}
}
\caption{Powiększenie tekstu dwukrotnie za pomocą trzech metod.}
\end{center}
\end{figure}

Aby nie opierać się wyłącznie na wizualnych obserwacjach, zbadaliśmy
metody zmiany rozdzielczości obrazka również od strony matematycznej. Załóżmy
bowiem, że mamy obrazek zadany przez dwuwymiarową funkcję. Wówczas znamy
dokładną wartość, jaką powinna ona przyjąć po przeskalowaniu. Uznaliśmy, że
wykorzystując ten fakt, możemy zbadać odchylenie standardowe wyników uzyskanych przy pomocy
poszczególnych algorytmów w przypadku zmieniania rozdzielczości obrazu, który jest
wykresem dwuwymiarowej funkcji. Oznaczmy więc przez $F$ funkcję odpowiadającą
obrazowi o zmienionych rozmiarach, których wartości to $N_x$ i $N_y$. Niech
jednocześnie $M$ będzie funkcją, która przyporządkowuje danemu pikselowi jego
wartość wyznaczoną przez jedną z metod, wtedy odchylenie standardowe określamy
jako
$$d(M)=\sqrt{\frac{\sum_{x=0}^{N_x-1}{\sum_{y=0}^{N_y-1}{(M(x,y)-F(x,y))^2}}}{N_xN_y}}.$$
Im większa wartość tego współczynnika tym bardziej uzyskany za pomocą określonej
metody skalowania wynik odbiega od wyniku pożądanego. Obliczyliśmy wartość tego
współczynnika dla wartości skalowania od $10\%$ do $500\%$ dla dwuwymiarowych
wykresów piły i funkcji sinus. Tabele na rysunku \textbf{3.5}
przedstawiają wybrane wyniki dla odpowiednio piły i funkcji sinus.
Jak widać wyniki zgadzają się z naszymi intuicjami. Największą wartość odchylenia
standardowego jest uzyskana przez NS zaś odchylenia AFS(I) i AFS(III) są do siebie zbliżone. Jak można
zauważyć, wartość tego odchylenia niewielkim stopniu od tego, jak zmieniamy obraz.
Pełen zestaw zgromadzonych danych znajduje się w odpowiednich plikach w katalogu
\textit{data}.
\begin{figure}[h!tb]
\begin{center}
\subfigure[Piła]{
\begin{tabular}{|c|c||c|c|c|}
    \hline
    OX & OY & d(NS) & d(AFS(I)) & d(AFS(III))\\
    \hline
    \hline
    10\% & 50\% & 19.7684 & 16.1438 & 15.8878\\
    \hline
    50\% & 10\% & 19.7684 & 16.1488 & 15.7155\\
    \hline
    50\% & 250\% & 19.3463 & 15.6816 & 15.1441\\
    \hline
    150\% & 450\% & 20.9178 & 15.844 & 15.3858\\
    \hline
    250\% & 50\% & 19.3463  & 15.6864 & 15.2307\\
    \hline
    450\% & 150\% & 20.9178 & 15.8438 & 15.3248\\
    \hline
\end{tabular}
}
\subfigure[Funkcja Rungego]{
\begin{tabular}{|c|c||c|c|c|}
    \hline
    OX & OY & d(NS) & d(AFS(I)) & d(AFS(III))\\
    \hline
    \hline
    10\% & 50\% & 3.38851 & 1.06066 & 1.12161\\
    \hline
    50\% & 10\% & 3.388515 & 1.0616 & 1.13159\\
    \hline
    50\% & 250\% & 4.08785 & 1.1073 & 1.13852\\
    \hline
    150\% & 450\% & 4.10309 & 1.11169 & 1.14702\\
    \hline
    250\% & 50\% & 4.08785 & 1.10646 & 1.13993\\
    \hline
    450\% & 150\% & 4.10309 & 1.11306 & 1.14848\\
    \hline
\end{tabular}
}
\caption{Tabele przedstawiające wartość odchylenia standardowego dla metod skalowania.}
\end{center}
\end{figure}

Zbadaliśmy pod względem wielkości odchylenia również funkcję Rungego. W tym przypadku spotkało
nas zaskoczenie. Wartości odchylenia standardowego dla algorytmu naturalnego sąsiedztwa wniosły
w przybliżeniu \textbf{0.65}, dla AFS(I) około \textbf{0.8}, a dla AFS(III) \textbf{1.1}, a więc
odwrotnie, niż by to mówiły wcześniejsze badania. Przyjrzyjmy się więc, jak wygląda wykres tej funkcji
(został on zamieszczony na rysunku \textbf{3.6}). Jak widać przejścia pomiędzy poszczególnymi odcieniami
są na tym rysunku gładkie, stąd metoda NS czy liniowa są jak najbardziej wskazane w tym przypadku do użycia.
Wynika stąd, że to jakiej metody warto użyć do skalowania obrazu zależy także od tego, co sam obraz przedstawia,
czy posiada ostre czy rozmyte kontury.
\begin{figure}[h!tb]
\begin{center}
\includegraphics{../obrazki/funkcje/runge_test.jpg}
\caption{Wykres funkcji Rungego}
\end{center}
\end{figure}

Po wykonaniu tego doświadczenia zastanowiliśmy się, co stanie się z obrazkiem, gdy zmniejszymy go i ponownie powiększymy do oryginalnego rozmiaru.
W celu sprawdzenia tego, zmniejszyliśmy pięciokrotnie, a następnie zwiększyliśmy pięciokrotnie trzy obrazki za pomocą wszystkich trzech metod.
Wyniki tych badań znajdują się w folderach \textit{obrazki$\slash$rybki\_zm\_zw},
\textit{obrazki$\slash$linia\_zm\_zw} oraz \textit{obrazki$\slash$labirynt\_zm\_zw}.

Pierwszy z podanych folderów zawiera zmieniony przez metody czarno-biały obrazek
przedstawiający rybki, mający wiele małych szczegółów. Jak widać, wszystkie
trzy algorytmy spowodowały znaczne ubytki na jakości. Od razu rzuca się w oczy,
że metoda NS daje najgorsze efekty. Rysunek niemal całkowicie stracił swój poprzedni wygląd, obraz jest mało płynny, 
na krawędziach pojawiają się całe kwadraty. Metoda AFS(I) rozmywa i rozjaśni obraz, ponownie więc metoda 
AFS(III) daje najbardziej pozytywny wynik, choć i tak bardzo zniekształcony w porównaniu z oryginałem.

Drugi folder zawiera obrazki przedstawiające prostą czarną linię na biały,m tle.
Wszystkie trzy algorytmy powodują pogrubienie się linii. Algorytmy, wykorzystujące
funkcje sklejane dodatkowo niekorzystnie wpływają na wygląd krawędzi, stają się
one bardzo rozmyte.

Zawartość ostatniego folderu zostały częściowo przedstawione na rysunku \textbf{3.7}.
Oryginalny obraz przedstawia labirynt. Można dostrzec, że po wykonaniu skalowania
wszystkie za pomocą trzech metod pojawia się efekt prążków, dodatkowo w wynikowym
obrazku metody AFS(I) niewątpliwie obraz jest w dużym stopniu rozmyty.
\begin{figure}[h!tb]
\begin{center}
\subfigure[NS]{
\includegraphics[width=4cm]{../obrazki/labirynt_zm_zw/lab_diff_img_closest.jpg}
}
\subfigure[AFS(I)]{
\includegraphics[width=4cm]{../obrazki/labirynt_zm_zw/lab_diff_img_linear.jpg}
}
\subfigure[AFS(III)]{
\includegraphics[width=4cm]{../obrazki/labirynt_zm_zw/lab_diff_img_cubic.jpg}
}
\caption{Labirynt pomniejszony, a następnie powiększony przez wszystkie trzy metody.}
\end{center}
\end{figure}

Widać więc,że zmiana obrazu opisana powyżej ma katastrofalne skutki. Można sobie
zadać teraz pytanie: a co, gdy zmienimy kolejność działania? Inaczej mówiąc,
co stanie się z obrazem gdy najpierw pięciokrotnie go powiększymy, a potem pomniejszymy do początkowych wymiarów?
Teoretycznie przecież, przy zwiększaniu obrazu nie tracimy żadnych informacji.
W celu zbadania tego problemu wykonaliśmy eksperymenty, których wyniki zawierają foldery
\textit{obrazki$\slash$rybki\_zw\_zm} i \textit{obrazki$\slash$labirynt\_zw\_zm}.
Obserwujemy pozytywne zjawisko. Po wykonanych operacjach skalowania,
wszystkie testowane przez nas algorytmy nie zmieniają początkowego obrazu.

Ostatnim badaniem jakie przeprowadziliśmy odnośnie jakości polegało na przyjrzeniu się,
jak nasze algorytmy poradzą sobie z obrazkami kolorowymi. Rysunek \textbf{3.8}
przedstawia kolejno od góry wynik pięciokrotnego pomniejszenia, a następnie
powiększenia obrazka \textit{pokemon.bmp} do początkowych wymiarów przez algorytm AFS(III), 
AFS(I), NS. Po lewej stronie umieszczone zostały obrazy różnicowe dla odpowiednich metod. Jak widzimy, uzyskaliśmy podobne efekty,
jak przy skalowaniu obrazów monochromatycznych, dodatkowo jednak można zaobserwować
dla wszystkich metod zanik barw widoczny zwłaszcza przy czarnych krawędziach
Czarny kontur postaci przedstawionej na rysunku traci bowiem ciągłość.
\begin{figure}[h!tb]
\begin{center}
\includegraphics[height=15cm]{../obrazki/all.jpg}
\caption{Wyniki doświadczeń i obrazy różnicowe dla rysunku kolorowego}
\end{center}
\end{figure}







