\section{Zestawienie z profesjonalnym programem do obróbki obrazów}
Używane w profesjonalnych programach do obróbki obrazu algorytmy skalowania
różnią się od tych przedstawionych przez nas. W zależności od tego jakich
efektów ubocznych chcemy uniknąć (czy efektu schodków, czy rozmycia) oraz jak
bardzo zależy nam na szybkości wykonywanch skalowań, wybiera się różne metody.
Najczęściej spotykanymi rowiązaniami są metody
\textbf{dwuliniowa}oraz \textbf{dwusześcienna}.
Główną różnicą między nimi, a
rozwiązaniami przedstawionymi przez nas jest to, że dla danego punktu nie
patrzą one jedynie na punkty w poziomie (pionie), ale na całe otoczenie.
Pierwsza z nich uwzględnia najbliższych 4 sąsiadów w pionie i poziomie
($2\times2$) i przyporządkowuje rozważąnemu pikselowi ich średnią ważoną.
W rezultacie uzyskujemy wyraźnie gładsze obrazy niż np. w metodzie
najbliższego sąsiedztwa.Metoda zwana 'bicubic interpolation' zaś idzie jeszcze krok dalej, uwzględniając 16 ($4\times4$) otaczających rozawżany punkt sąsiadów. Tu również
liczonajest średnia ważona, jednakże dalsze punkty uzysują mniejszą wagę.
Algorytm ten daje dostrzegawczo ostrzejsze obrazy i jest bliskim ideału
kompromisem między złożonością czasową, a zachowniem jakości. Stąd jest on
podstawą wielu takich programów, jak przykładowo Adobe Photoshop.

W rozdziale tym postaramy się porównać działanie najlepszego z testowanych przez
nas algorytmów, a więc AFS(III), z działaniem programu AdobePhotoshot, przy czym zadaniw to
wykonamy zarówno dla obrazów czarno-białych, jak i kolorowych. Zaczniemy od obrazów
monochromatycznych. Dwukrotnie zmniejszyliśmy, a następnie zwiększyliśmy obraz
\textit{liscie.bmp} za pomocą badanej przez nas metody AFS(III) i metod dwuliniowej
i dwusześćiennej. Fragmenty wyniku przedstawia rysunek \textbf{6.1}. Pełne wyniki
testu znajdują się w folderze \textit{obrazki$\slash$liscie\_2}. Jak widzimy, w
tym przypadku najlepiej sprawdza się metoda dwusześcienna. Niewątpliwie, metody z profesjanalnego programu
dają lepszej jakości obrazy po przeskalowaniu. Niewidoczne są efekty 'schodów',
które dostrzegamy w obrazie wynikowym algrytmu AFS(III), linie są gładkie. Wynika stąd, że patrzenie nie tylko na punkty w jednej lini, ale interpolacja
funkcji w dwuch wymiarach przyczynia się do lepszego wizualnie efektu.
\begin{figure}[h!tb]
\begin{center}
\subfigure[AFS(III)]{
\includegraphics[width=3cm]{../obrazki/liscie_2/cubic.jpg}
}
\subfigure[Dwuliniowa]{
\includegraphics[width=3cm]{../obrazki/liscie_2/liscie_dwuliniowa.jpg}
}
\subfigure[Dwusześcienna]{
\includegraphics[width=3cm]{../obrazki/liscie_2/dwuszescienna.jpg}
}
\caption{Fragmenty obrazków przeskalowanych}
\end{center}
\end{figure}

Po tym eksperymencie, przeskalowaliśmy w ten sam obrazek kolorowy \textit{pokemon.bmp}.
Wyniki doświadczenia obrazuje rysunek \textbf{6.2} (pełne wyniki zawiera folder \textit{obrazki$\slash$pokemon}).
Ponownie najlepszą okazuje się być metoda dwusześcienna programu AdobePhotoshop. W metodzie dwuliniowej
obserwujemy już częsciowy zanik krawędzi i lekkie rozmycie. Metoda AFS(III) daje jednak
wyniki najmniej zadowalające, widoczne jest zniekształcenie krawędzi.
\begin{figure}[h!tb]
\begin{center}
\subfigure[AFS(III)]{
\includegraphics[width=4cm]{../obrazki/pokemon/cubic.jpg}
}
\subfigure[Dwuliniowa]{
\includegraphics[width=4cm]{../obrazki/pokemon/dwuliniowa.jpg}
}
\subfigure[Dwusześcienna]{
\includegraphics[width=4cm]{../obrazki/pokemon/dwuszescienna.jpg}
}
\caption{Obrazki kolorowe przeskalowane za pomocą AFS(III) i metod z AdobePhotoshop}
\end{center}
\end{figure}

