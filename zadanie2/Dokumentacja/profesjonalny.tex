\section{Zestawienie z profesjonalnym programem do obróbki obrazów}
Używane w profesjonalnych programach do obróbki obrazu algorytmy skalowania
różnią się od tych przedstawionych przez nas. W zależności od tego jakich
efektów ubocznych chcemy uniknąć (czy efektu schodków, czy rozmycia) oraz jak
bardzo zależy nam na szybkości wykonywanch skalowań, wybiera się różne metody.
Najczęściej spotykanymi rowiązaniami są metody
\textbf{'bilinear interpolation'}oraz \textbf{'bicubic interpolation'}.
Główną różnicą między nimi, a
rozwiązaniami przedstawionymi przez nas jest to, że dla danego punktu nie
patrzą one jedynie na punkty w poziomie (pionie), ale na całe otoczenie.
Pierwsza z nich uwzględnia najbliższych 4 sąsiadów w pionie i poziomie
($2\times2$) i przyporządkowuje rozważąnemu pikselowi ich średnią ważoną.
W rezultacie uzyskujemy wyraźnie gładsze obrazy niż np. w metodzie
najbliższego sąsiedztwa.Metoda zwana 'bicubic interpolation' zaś idzie jeszcze krok dalej, uwzględniając 16 ($4\times4$) otaczających rozawżany punkt sąsiadów. Tu również
liczonajest średnia ważona, jednakże dalsze punkty uzysują mniejszą wagę.
Algorytm ten daje dostrzegawczo ostrzejsze obrazy i jest bliskim ideału
kompromisem między złożonością czasową, a zachowniem jakości. Stąd jest on
podstawą wielu takich programów, jak przykładowo Adobe Photoshop.

W rozdziale tym postaramy się porównać działanie najlepszego z testowanych przez
nas algorytmów, a więc AFS(III), z działaniem programu GIMP, przy czym zadaniw to
wykonamy zarówno dla obrazów czarno-białych, jak i kolorowych. Zaczniemy od obrazów
monochromatycznych. 

