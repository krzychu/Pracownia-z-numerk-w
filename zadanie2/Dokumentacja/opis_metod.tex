\section{Opis metod}
Niech $M_x$ oraz $M_y$ oznaczają wymiary danego na wejściu obrazu. Naszym
zadaniem jest przekształcić ten obraz do wymiarów $N_x$ na $N_y$. Zmiana
wielkości rysunku w badanych przez nas algorytmów odbywa się niezależnie od
siebie w pionie i poziomie, stąd w opisie metod ograniczym się jedynie do
obrazów mających wymiary $M$ na 1 punktów.

Obraz taki w naturalny sposób można traktować jako ciąg kolorów w punktach
$t_1=1,t_2=2,...,t_M=M$ (w przypadku obrazów kolorowych, ponieważ jest to w
istocie nałożenie na siebie trzech obrazów jednobarwnych, rozważamy 3 takie
ciągi)\footnote{W dalszej częci opisu algorytmów będziemy pomijać dodatkowe
rozważenie przypadku obrazów kolorowych. Wynika to z faktu, że polaga ono na
wkonaniu analogicznych działań, a jedyną różnicą jest, że wykonujemy je na
trzech ciągach.}. Zmiana rozmiaru polega jedynie na wyznaczeniu wartości koloru
$K(.)$ w punktach
$$p_i=1+(i-1)\frac{M-1}{N-1},$$
gdzie $N$ oznacza nowy rozmiar obrazu. Naszym zadaniem było przeanalizowanie
trzech metod realizujących to zadanie.

Pierwsza z nich, nazywana dalej \textit{metodą najbliższego sąsiedztwa}, polega
na wyznaczeniu wartości $K(p_i)$ przy pomocy wzoru
$$K(p_i)=K(round(p_i)), (i=1,2,...,N).$$
Intuicyjnie, algorytm ten polega na tym, że dla danego punktu szukamy najbliższy
mu punkt oryginalnego obrazka i przypożądkowujemy mu jego kolor.

Następny algorytm opiera się na konstrukcji funkcji sklejanej S pierwszego
stopnia (będziemy go nazywać \textit{AFS(I)}), której wartości w węzłach $S(t_i)
=K(t_i)$ dla $i=1,2,...,M$. Przyjmujemy następnie $K(p_i)=S(p_i) (i=1,2,...,N).$

Ostatnia metoda, którą oznaczać będziemy \textit{AFS(III)}, to stworzenie
funkcji sklejanej trzeciego stopnia $Z$, takiej, że $Z(t_i)=K(t_i)$ dla
$i=1,2,...,M$, przy czym przyjmujemy, analogicznie do poprzedniej metody,
$K(p_i)=S(p_i) (i=1,2,...,N)$. Jeżeli wyzanczane wartości przekraczają
dopuszczalny zakres wartości (czyli nie są liczbami z przedziału $[0,255]$)
zstępujemy odpowiednio liczby ujemnye przez wartość $0$, zaś liczby większe od
$255$ przez $255$.

Gdy obrazy mają obydwa wymiary większe od $1$ to nasze metody stosujemy najpierw
do zmiany obrazu w pionie, a później w poziomie (lub odwrotnie).

