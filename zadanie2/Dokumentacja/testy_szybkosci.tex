\section{Porównanie szybkości algorytmów}
W celu porównania badanych przez nas metod pod względem szybkości obliczeń,
przeprowadziliśmy testy skalowania obrazu o wymiarach $180\times135$ pikseli
\footnote{Zważywszy na to, jak metody, które badamy konstruują wynik, wiemy, że 
czas działania algorytmu nie zależy od tego, co znajduje się na obrazku. Zależy 
jedynie od jego rozmiaru.},zmieniając go tak samo na osi $X$ i $Y$. Skala naszych badań objęła zakres
$[10\%, 1100\%]$ krokiem co $5\%$. Oznacza to, że zmienialiśmy jego szerokość
oraz długość do $10\%, 15\%, ... ,1100\%$. Wyniki prezentuje wykres \textbf{5.1}.
\begin{figure}[h!tb]
\begin{center}
% GNUPLOT: LaTeX picture
\setlength{\unitlength}{0.240900pt}
\ifx\plotpoint\undefined\newsavebox{\plotpoint}\fi
\sbox{\plotpoint}{\rule[-0.200pt]{0.400pt}{0.400pt}}%
\begin{picture}(1500,900)(0,0)
\sbox{\plotpoint}{\rule[-0.200pt]{0.400pt}{0.400pt}}%
\put(190.0,82.0){\rule[-0.200pt]{4.818pt}{0.400pt}}
\put(170,82){\makebox(0,0)[r]{ 0}}
\put(1419.0,82.0){\rule[-0.200pt]{4.818pt}{0.400pt}}
\put(190.0,168.0){\rule[-0.200pt]{4.818pt}{0.400pt}}
\put(170,168){\makebox(0,0)[r]{ 50000}}
\put(1419.0,168.0){\rule[-0.200pt]{4.818pt}{0.400pt}}
\put(190.0,255.0){\rule[-0.200pt]{4.818pt}{0.400pt}}
\put(170,255){\makebox(0,0)[r]{ 100000}}
\put(1419.0,255.0){\rule[-0.200pt]{4.818pt}{0.400pt}}
\put(190.0,341.0){\rule[-0.200pt]{4.818pt}{0.400pt}}
\put(170,341){\makebox(0,0)[r]{ 150000}}
\put(1419.0,341.0){\rule[-0.200pt]{4.818pt}{0.400pt}}
\put(190.0,427.0){\rule[-0.200pt]{4.818pt}{0.400pt}}
\put(170,427){\makebox(0,0)[r]{ 200000}}
\put(1419.0,427.0){\rule[-0.200pt]{4.818pt}{0.400pt}}
\put(190.0,514.0){\rule[-0.200pt]{4.818pt}{0.400pt}}
\put(170,514){\makebox(0,0)[r]{ 250000}}
\put(1419.0,514.0){\rule[-0.200pt]{4.818pt}{0.400pt}}
\put(190.0,600.0){\rule[-0.200pt]{4.818pt}{0.400pt}}
\put(170,600){\makebox(0,0)[r]{ 300000}}
\put(1419.0,600.0){\rule[-0.200pt]{4.818pt}{0.400pt}}
\put(190.0,686.0){\rule[-0.200pt]{4.818pt}{0.400pt}}
\put(170,686){\makebox(0,0)[r]{ 350000}}
\put(1419.0,686.0){\rule[-0.200pt]{4.818pt}{0.400pt}}
\put(190.0,773.0){\rule[-0.200pt]{4.818pt}{0.400pt}}
\put(170,773){\makebox(0,0)[r]{ 400000}}
\put(1419.0,773.0){\rule[-0.200pt]{4.818pt}{0.400pt}}
\put(190.0,859.0){\rule[-0.200pt]{4.818pt}{0.400pt}}
\put(170,859){\makebox(0,0)[r]{ 450000}}
\put(1419.0,859.0){\rule[-0.200pt]{4.818pt}{0.400pt}}
\put(190.0,82.0){\rule[-0.200pt]{0.400pt}{4.818pt}}
\put(190,41){\makebox(0,0){ 0}}
\put(190.0,839.0){\rule[-0.200pt]{0.400pt}{4.818pt}}
\put(417.0,82.0){\rule[-0.200pt]{0.400pt}{4.818pt}}
\put(417,41){\makebox(0,0){ 200}}
\put(417.0,839.0){\rule[-0.200pt]{0.400pt}{4.818pt}}
\put(644.0,82.0){\rule[-0.200pt]{0.400pt}{4.818pt}}
\put(644,41){\makebox(0,0){ 400}}
\put(644.0,839.0){\rule[-0.200pt]{0.400pt}{4.818pt}}
\put(871.0,82.0){\rule[-0.200pt]{0.400pt}{4.818pt}}
\put(871,41){\makebox(0,0){ 600}}
\put(871.0,839.0){\rule[-0.200pt]{0.400pt}{4.818pt}}
\put(1098.0,82.0){\rule[-0.200pt]{0.400pt}{4.818pt}}
\put(1098,41){\makebox(0,0){ 800}}
\put(1098.0,839.0){\rule[-0.200pt]{0.400pt}{4.818pt}}
\put(1325.0,82.0){\rule[-0.200pt]{0.400pt}{4.818pt}}
\put(1325,41){\makebox(0,0){ 1000}}
\put(1325.0,839.0){\rule[-0.200pt]{0.400pt}{4.818pt}}
\put(190.0,82.0){\rule[-0.200pt]{0.400pt}{187.179pt}}
\put(190.0,82.0){\rule[-0.200pt]{300.884pt}{0.400pt}}
\put(1439.0,82.0){\rule[-0.200pt]{0.400pt}{187.179pt}}
\put(190.0,859.0){\rule[-0.200pt]{300.884pt}{0.400pt}}
\put(210,819){\makebox(0,0)[l]{algorytm NS}}
\put(670.0,819.0){\rule[-0.200pt]{24.090pt}{0.400pt}}
\put(201,82){\usebox{\plotpoint}}
\put(224,81.67){\rule{2.650pt}{0.400pt}}
\multiput(224.00,81.17)(5.500,1.000){2}{\rule{1.325pt}{0.400pt}}
\put(201.0,82.0){\rule[-0.200pt]{5.541pt}{0.400pt}}
\put(303,82.67){\rule{2.650pt}{0.400pt}}
\multiput(303.00,82.17)(5.500,1.000){2}{\rule{1.325pt}{0.400pt}}
\put(235.0,83.0){\rule[-0.200pt]{16.381pt}{0.400pt}}
\put(360,83.67){\rule{2.650pt}{0.400pt}}
\multiput(360.00,83.17)(5.500,1.000){2}{\rule{1.325pt}{0.400pt}}
\put(314.0,84.0){\rule[-0.200pt]{11.081pt}{0.400pt}}
\put(405,84.67){\rule{2.650pt}{0.400pt}}
\multiput(405.00,84.17)(5.500,1.000){2}{\rule{1.325pt}{0.400pt}}
\put(371.0,85.0){\rule[-0.200pt]{8.191pt}{0.400pt}}
\put(439,85.67){\rule{2.650pt}{0.400pt}}
\multiput(439.00,85.17)(5.500,1.000){2}{\rule{1.325pt}{0.400pt}}
\put(416.0,86.0){\rule[-0.200pt]{5.541pt}{0.400pt}}
\put(484,86.67){\rule{2.650pt}{0.400pt}}
\multiput(484.00,86.17)(5.500,1.000){2}{\rule{1.325pt}{0.400pt}}
\put(450.0,87.0){\rule[-0.200pt]{8.191pt}{0.400pt}}
\put(518,87.67){\rule{2.650pt}{0.400pt}}
\multiput(518.00,87.17)(5.500,1.000){2}{\rule{1.325pt}{0.400pt}}
\put(495.0,88.0){\rule[-0.200pt]{5.541pt}{0.400pt}}
\put(540,88.67){\rule{2.891pt}{0.400pt}}
\multiput(540.00,88.17)(6.000,1.000){2}{\rule{1.445pt}{0.400pt}}
\put(529.0,89.0){\rule[-0.200pt]{2.650pt}{0.400pt}}
\put(574,89.67){\rule{2.650pt}{0.400pt}}
\multiput(574.00,89.17)(5.500,1.000){2}{\rule{1.325pt}{0.400pt}}
\put(552.0,90.0){\rule[-0.200pt]{5.300pt}{0.400pt}}
\put(608,90.67){\rule{2.650pt}{0.400pt}}
\multiput(608.00,90.17)(5.500,1.000){2}{\rule{1.325pt}{0.400pt}}
\put(585.0,91.0){\rule[-0.200pt]{5.541pt}{0.400pt}}
\put(631,91.67){\rule{2.650pt}{0.400pt}}
\multiput(631.00,91.17)(5.500,1.000){2}{\rule{1.325pt}{0.400pt}}
\put(619.0,92.0){\rule[-0.200pt]{2.891pt}{0.400pt}}
\put(653,92.67){\rule{2.891pt}{0.400pt}}
\multiput(653.00,92.17)(6.000,1.000){2}{\rule{1.445pt}{0.400pt}}
\put(642.0,93.0){\rule[-0.200pt]{2.650pt}{0.400pt}}
\put(676,93.67){\rule{2.650pt}{0.400pt}}
\multiput(676.00,93.17)(5.500,1.000){2}{\rule{1.325pt}{0.400pt}}
\put(665.0,94.0){\rule[-0.200pt]{2.650pt}{0.400pt}}
\put(710,94.67){\rule{2.650pt}{0.400pt}}
\multiput(710.00,94.17)(5.500,1.000){2}{\rule{1.325pt}{0.400pt}}
\put(687.0,95.0){\rule[-0.200pt]{5.541pt}{0.400pt}}
\put(732,95.67){\rule{2.891pt}{0.400pt}}
\multiput(732.00,95.17)(6.000,1.000){2}{\rule{1.445pt}{0.400pt}}
\put(721.0,96.0){\rule[-0.200pt]{2.650pt}{0.400pt}}
\put(755,96.67){\rule{2.650pt}{0.400pt}}
\multiput(755.00,96.17)(5.500,1.000){2}{\rule{1.325pt}{0.400pt}}
\put(744.0,97.0){\rule[-0.200pt]{2.650pt}{0.400pt}}
\put(778,97.67){\rule{2.650pt}{0.400pt}}
\multiput(778.00,97.17)(5.500,1.000){2}{\rule{1.325pt}{0.400pt}}
\put(789,98.67){\rule{2.650pt}{0.400pt}}
\multiput(789.00,98.17)(5.500,1.000){2}{\rule{1.325pt}{0.400pt}}
\put(766.0,98.0){\rule[-0.200pt]{2.891pt}{0.400pt}}
\put(811,99.67){\rule{2.891pt}{0.400pt}}
\multiput(811.00,99.17)(6.000,1.000){2}{\rule{1.445pt}{0.400pt}}
\put(800.0,100.0){\rule[-0.200pt]{2.650pt}{0.400pt}}
\put(834,100.67){\rule{2.650pt}{0.400pt}}
\multiput(834.00,100.17)(5.500,1.000){2}{\rule{1.325pt}{0.400pt}}
\put(845,101.67){\rule{2.891pt}{0.400pt}}
\multiput(845.00,101.17)(6.000,1.000){2}{\rule{1.445pt}{0.400pt}}
\put(823.0,101.0){\rule[-0.200pt]{2.650pt}{0.400pt}}
\put(868,102.67){\rule{2.650pt}{0.400pt}}
\multiput(868.00,102.17)(5.500,1.000){2}{\rule{1.325pt}{0.400pt}}
\put(857.0,103.0){\rule[-0.200pt]{2.650pt}{0.400pt}}
\put(890,103.67){\rule{2.891pt}{0.400pt}}
\multiput(890.00,103.17)(6.000,1.000){2}{\rule{1.445pt}{0.400pt}}
\put(902,104.67){\rule{2.650pt}{0.400pt}}
\multiput(902.00,104.17)(5.500,1.000){2}{\rule{1.325pt}{0.400pt}}
\put(879.0,104.0){\rule[-0.200pt]{2.650pt}{0.400pt}}
\put(924,105.67){\rule{2.891pt}{0.400pt}}
\multiput(924.00,105.17)(6.000,1.000){2}{\rule{1.445pt}{0.400pt}}
\put(936,106.67){\rule{2.650pt}{0.400pt}}
\multiput(936.00,106.17)(5.500,1.000){2}{\rule{1.325pt}{0.400pt}}
\put(913.0,106.0){\rule[-0.200pt]{2.650pt}{0.400pt}}
\put(958,107.67){\rule{2.891pt}{0.400pt}}
\multiput(958.00,107.17)(6.000,1.000){2}{\rule{1.445pt}{0.400pt}}
\put(970,108.67){\rule{2.650pt}{0.400pt}}
\multiput(970.00,108.17)(5.500,1.000){2}{\rule{1.325pt}{0.400pt}}
\put(981,109.67){\rule{2.650pt}{0.400pt}}
\multiput(981.00,109.17)(5.500,1.000){2}{\rule{1.325pt}{0.400pt}}
\put(947.0,108.0){\rule[-0.200pt]{2.650pt}{0.400pt}}
\put(1003,110.67){\rule{2.891pt}{0.400pt}}
\multiput(1003.00,110.17)(6.000,1.000){2}{\rule{1.445pt}{0.400pt}}
\put(1015,111.67){\rule{2.650pt}{0.400pt}}
\multiput(1015.00,111.17)(5.500,1.000){2}{\rule{1.325pt}{0.400pt}}
\put(992.0,111.0){\rule[-0.200pt]{2.650pt}{0.400pt}}
\put(1037,112.67){\rule{2.891pt}{0.400pt}}
\multiput(1037.00,112.17)(6.000,1.000){2}{\rule{1.445pt}{0.400pt}}
\put(1049,113.67){\rule{2.650pt}{0.400pt}}
\multiput(1049.00,113.17)(5.500,1.000){2}{\rule{1.325pt}{0.400pt}}
\put(1060,114.67){\rule{2.650pt}{0.400pt}}
\multiput(1060.00,114.17)(5.500,1.000){2}{\rule{1.325pt}{0.400pt}}
\put(1071,115.67){\rule{2.891pt}{0.400pt}}
\multiput(1071.00,115.17)(6.000,1.000){2}{\rule{1.445pt}{0.400pt}}
\put(1026.0,113.0){\rule[-0.200pt]{2.650pt}{0.400pt}}
\put(1094,116.67){\rule{2.650pt}{0.400pt}}
\multiput(1094.00,116.17)(5.500,1.000){2}{\rule{1.325pt}{0.400pt}}
\put(1105,117.67){\rule{2.650pt}{0.400pt}}
\multiput(1105.00,117.17)(5.500,1.000){2}{\rule{1.325pt}{0.400pt}}
\put(1116,118.67){\rule{2.891pt}{0.400pt}}
\multiput(1116.00,118.17)(6.000,1.000){2}{\rule{1.445pt}{0.400pt}}
\put(1128,119.67){\rule{2.650pt}{0.400pt}}
\multiput(1128.00,119.17)(5.500,1.000){2}{\rule{1.325pt}{0.400pt}}
\put(1083.0,117.0){\rule[-0.200pt]{2.650pt}{0.400pt}}
\put(1150,120.67){\rule{2.891pt}{0.400pt}}
\multiput(1150.00,120.17)(6.000,1.000){2}{\rule{1.445pt}{0.400pt}}
\put(1162,121.67){\rule{2.650pt}{0.400pt}}
\multiput(1162.00,121.17)(5.500,1.000){2}{\rule{1.325pt}{0.400pt}}
\put(1173,122.67){\rule{2.650pt}{0.400pt}}
\multiput(1173.00,122.17)(5.500,1.000){2}{\rule{1.325pt}{0.400pt}}
\put(1184,123.67){\rule{2.891pt}{0.400pt}}
\multiput(1184.00,123.17)(6.000,1.000){2}{\rule{1.445pt}{0.400pt}}
\put(1139.0,121.0){\rule[-0.200pt]{2.650pt}{0.400pt}}
\put(1207,124.67){\rule{2.650pt}{0.400pt}}
\multiput(1207.00,124.17)(5.500,1.000){2}{\rule{1.325pt}{0.400pt}}
\put(1218,125.67){\rule{2.650pt}{0.400pt}}
\multiput(1218.00,125.17)(5.500,1.000){2}{\rule{1.325pt}{0.400pt}}
\put(1229,126.67){\rule{2.891pt}{0.400pt}}
\multiput(1229.00,126.17)(6.000,1.000){2}{\rule{1.445pt}{0.400pt}}
\put(1241,127.67){\rule{2.650pt}{0.400pt}}
\multiput(1241.00,127.17)(5.500,1.000){2}{\rule{1.325pt}{0.400pt}}
\put(1252,128.67){\rule{2.650pt}{0.400pt}}
\multiput(1252.00,128.17)(5.500,1.000){2}{\rule{1.325pt}{0.400pt}}
\put(1263,129.67){\rule{2.891pt}{0.400pt}}
\multiput(1263.00,129.17)(6.000,1.000){2}{\rule{1.445pt}{0.400pt}}
\put(1275,130.67){\rule{2.650pt}{0.400pt}}
\multiput(1275.00,130.17)(5.500,1.000){2}{\rule{1.325pt}{0.400pt}}
\put(1286,131.67){\rule{2.650pt}{0.400pt}}
\multiput(1286.00,131.17)(5.500,1.000){2}{\rule{1.325pt}{0.400pt}}
\put(1297,132.67){\rule{2.650pt}{0.400pt}}
\multiput(1297.00,132.17)(5.500,1.000){2}{\rule{1.325pt}{0.400pt}}
\put(1308,133.67){\rule{2.891pt}{0.400pt}}
\multiput(1308.00,133.17)(6.000,1.000){2}{\rule{1.445pt}{0.400pt}}
\put(1196.0,125.0){\rule[-0.200pt]{2.650pt}{0.400pt}}
\sbox{\plotpoint}{\rule[-0.500pt]{1.000pt}{1.000pt}}%
\sbox{\plotpoint}{\rule[-0.200pt]{0.400pt}{0.400pt}}%
\put(210,778){\makebox(0,0)[l]{algorytm AFS(I)}}
\sbox{\plotpoint}{\rule[-0.500pt]{1.000pt}{1.000pt}}%
\multiput(670,778)(20.756,0.000){5}{\usebox{\plotpoint}}
\put(770,778){\usebox{\plotpoint}}
\put(201,82){\usebox{\plotpoint}}
\put(201.00,82.00){\usebox{\plotpoint}}
\put(221.71,83.00){\usebox{\plotpoint}}
\put(242.44,83.62){\usebox{\plotpoint}}
\put(263.18,84.00){\usebox{\plotpoint}}
\put(283.89,85.00){\usebox{\plotpoint}}
\put(304.60,86.00){\usebox{\plotpoint}}
\put(325.32,86.94){\usebox{\plotpoint}}
\put(346.04,87.82){\usebox{\plotpoint}}
\put(366.74,89.00){\usebox{\plotpoint}}
\put(387.43,90.49){\usebox{\plotpoint}}
\put(408.15,91.29){\usebox{\plotpoint}}
\put(428.82,93.15){\usebox{\plotpoint}}
\put(449.50,94.95){\usebox{\plotpoint}}
\put(470.21,95.84){\usebox{\plotpoint}}
\put(490.89,97.63){\usebox{\plotpoint}}
\put(511.56,99.46){\usebox{\plotpoint}}
\put(532.24,101.29){\usebox{\plotpoint}}
\put(552.92,103.08){\usebox{\plotpoint}}
\put(573.59,104.96){\usebox{\plotpoint}}
\put(594.26,106.77){\usebox{\plotpoint}}
\put(614.80,109.62){\usebox{\plotpoint}}
\put(635.48,111.41){\usebox{\plotpoint}}
\put(656.12,113.52){\usebox{\plotpoint}}
\put(676.70,116.06){\usebox{\plotpoint}}
\put(697.25,118.86){\usebox{\plotpoint}}
\put(717.82,121.42){\usebox{\plotpoint}}
\put(738.39,124.07){\usebox{\plotpoint}}
\put(758.96,126.72){\usebox{\plotpoint}}
\put(779.53,129.28){\usebox{\plotpoint}}
\put(799.95,132.99){\usebox{\plotpoint}}
\put(820.53,135.59){\usebox{\plotpoint}}
\put(841.04,138.64){\usebox{\plotpoint}}
\put(861.54,141.83){\usebox{\plotpoint}}
\put(882.10,144.56){\usebox{\plotpoint}}
\put(902.55,148.10){\usebox{\plotpoint}}
\put(922.97,151.81){\usebox{\plotpoint}}
\put(943.42,155.35){\usebox{\plotpoint}}
\put(963.86,158.98){\usebox{\plotpoint}}
\put(984.29,162.60){\usebox{\plotpoint}}
\put(1004.72,166.29){\usebox{\plotpoint}}
\put(1025.17,169.85){\usebox{\plotpoint}}
\put(1045.61,173.43){\usebox{\plotpoint}}
\put(1065.92,177.61){\usebox{\plotpoint}}
\put(1086.27,181.59){\usebox{\plotpoint}}
\put(1106.66,185.45){\usebox{\plotpoint}}
\put(1126.92,189.82){\usebox{\plotpoint}}
\put(1147.18,194.23){\usebox{\plotpoint}}
\put(1167.47,198.49){\usebox{\plotpoint}}
\put(1187.79,202.63){\usebox{\plotpoint}}
\put(1208.00,207.27){\usebox{\plotpoint}}
\put(1228.22,211.86){\usebox{\plotpoint}}
\put(1248.47,216.36){\usebox{\plotpoint}}
\put(1268.59,221.40){\usebox{\plotpoint}}
\put(1288.87,225.78){\usebox{\plotpoint}}
\put(1308.91,231.15){\usebox{\plotpoint}}
\put(1320,233){\usebox{\plotpoint}}
\sbox{\plotpoint}{\rule[-0.600pt]{1.200pt}{1.200pt}}%
\sbox{\plotpoint}{\rule[-0.200pt]{0.400pt}{0.400pt}}%
\put(210,737){\makebox(0,0)[l]{algorytm AFS(III)}}
\sbox{\plotpoint}{\rule[-0.600pt]{1.200pt}{1.200pt}}%
\put(670.0,737.0){\rule[-0.600pt]{24.090pt}{1.200pt}}
\put(201,90){\usebox{\plotpoint}}
\put(201,87.01){\rule{2.891pt}{1.200pt}}
\multiput(201.00,87.51)(6.000,-1.000){2}{\rule{1.445pt}{1.200pt}}
\put(213,87.01){\rule{2.650pt}{1.200pt}}
\multiput(213.00,86.51)(5.500,1.000){2}{\rule{1.325pt}{1.200pt}}
\put(224,88.01){\rule{2.650pt}{1.200pt}}
\multiput(224.00,87.51)(5.500,1.000){2}{\rule{1.325pt}{1.200pt}}
\put(235,89.51){\rule{2.891pt}{1.200pt}}
\multiput(235.00,88.51)(6.000,2.000){2}{\rule{1.445pt}{1.200pt}}
\put(247,91.51){\rule{2.650pt}{1.200pt}}
\multiput(247.00,90.51)(5.500,2.000){2}{\rule{1.325pt}{1.200pt}}
\put(258,93.01){\rule{2.650pt}{1.200pt}}
\multiput(258.00,92.51)(5.500,1.000){2}{\rule{1.325pt}{1.200pt}}
\put(269,94.51){\rule{2.650pt}{1.200pt}}
\multiput(269.00,93.51)(5.500,2.000){2}{\rule{1.325pt}{1.200pt}}
\put(280,97.01){\rule{2.891pt}{1.200pt}}
\multiput(280.00,95.51)(6.000,3.000){2}{\rule{1.445pt}{1.200pt}}
\put(292,99.51){\rule{2.650pt}{1.200pt}}
\multiput(292.00,98.51)(5.500,2.000){2}{\rule{1.325pt}{1.200pt}}
\put(303,101.51){\rule{2.650pt}{1.200pt}}
\multiput(303.00,100.51)(5.500,2.000){2}{\rule{1.325pt}{1.200pt}}
\put(314,104.01){\rule{2.891pt}{1.200pt}}
\multiput(314.00,102.51)(6.000,3.000){2}{\rule{1.445pt}{1.200pt}}
\put(326,107.01){\rule{2.650pt}{1.200pt}}
\multiput(326.00,105.51)(5.500,3.000){2}{\rule{1.325pt}{1.200pt}}
\put(337,109.51){\rule{2.650pt}{1.200pt}}
\multiput(337.00,108.51)(5.500,2.000){2}{\rule{1.325pt}{1.200pt}}
\put(348,112.01){\rule{2.891pt}{1.200pt}}
\multiput(348.00,110.51)(6.000,3.000){2}{\rule{1.445pt}{1.200pt}}
\put(360,115.01){\rule{2.650pt}{1.200pt}}
\multiput(360.00,113.51)(5.500,3.000){2}{\rule{1.325pt}{1.200pt}}
\put(371,118.51){\rule{2.650pt}{1.200pt}}
\multiput(371.00,116.51)(5.500,4.000){2}{\rule{1.325pt}{1.200pt}}
\put(382,122.01){\rule{2.650pt}{1.200pt}}
\multiput(382.00,120.51)(5.500,3.000){2}{\rule{1.325pt}{1.200pt}}
\put(393,125.01){\rule{2.891pt}{1.200pt}}
\multiput(393.00,123.51)(6.000,3.000){2}{\rule{1.445pt}{1.200pt}}
\put(405,128.51){\rule{2.650pt}{1.200pt}}
\multiput(405.00,126.51)(5.500,4.000){2}{\rule{1.325pt}{1.200pt}}
\put(416,132.51){\rule{2.650pt}{1.200pt}}
\multiput(416.00,130.51)(5.500,4.000){2}{\rule{1.325pt}{1.200pt}}
\put(427,136.01){\rule{2.891pt}{1.200pt}}
\multiput(427.00,134.51)(6.000,3.000){2}{\rule{1.445pt}{1.200pt}}
\put(439,139.51){\rule{2.650pt}{1.200pt}}
\multiput(439.00,137.51)(5.500,4.000){2}{\rule{1.325pt}{1.200pt}}
\put(450,144.01){\rule{2.650pt}{1.200pt}}
\multiput(450.00,141.51)(5.500,5.000){2}{\rule{1.325pt}{1.200pt}}
\put(461,148.51){\rule{2.650pt}{1.200pt}}
\multiput(461.00,146.51)(5.500,4.000){2}{\rule{1.325pt}{1.200pt}}
\put(472,152.51){\rule{2.891pt}{1.200pt}}
\multiput(472.00,150.51)(6.000,4.000){2}{\rule{1.445pt}{1.200pt}}
\put(484,157.01){\rule{2.650pt}{1.200pt}}
\multiput(484.00,154.51)(5.500,5.000){2}{\rule{1.325pt}{1.200pt}}
\put(495,161.51){\rule{2.650pt}{1.200pt}}
\multiput(495.00,159.51)(5.500,4.000){2}{\rule{1.325pt}{1.200pt}}
\put(506,166.01){\rule{2.891pt}{1.200pt}}
\multiput(506.00,163.51)(6.000,5.000){2}{\rule{1.445pt}{1.200pt}}
\put(518,171.01){\rule{2.650pt}{1.200pt}}
\multiput(518.00,168.51)(5.500,5.000){2}{\rule{1.325pt}{1.200pt}}
\put(529,176.01){\rule{2.650pt}{1.200pt}}
\multiput(529.00,173.51)(5.500,5.000){2}{\rule{1.325pt}{1.200pt}}
\put(540,181.01){\rule{2.891pt}{1.200pt}}
\multiput(540.00,178.51)(6.000,5.000){2}{\rule{1.445pt}{1.200pt}}
\put(552,186.01){\rule{2.650pt}{1.200pt}}
\multiput(552.00,183.51)(5.500,5.000){2}{\rule{1.325pt}{1.200pt}}
\multiput(563.00,193.24)(0.622,0.509){2}{\rule{2.500pt}{0.123pt}}
\multiput(563.00,188.51)(5.811,6.000){2}{\rule{1.250pt}{1.200pt}}
\put(574,197.01){\rule{2.650pt}{1.200pt}}
\multiput(574.00,194.51)(5.500,5.000){2}{\rule{1.325pt}{1.200pt}}
\multiput(585.00,204.24)(0.792,0.509){2}{\rule{2.700pt}{0.123pt}}
\multiput(585.00,199.51)(6.396,6.000){2}{\rule{1.350pt}{1.200pt}}
\multiput(597.00,210.24)(0.622,0.509){2}{\rule{2.500pt}{0.123pt}}
\multiput(597.00,205.51)(5.811,6.000){2}{\rule{1.250pt}{1.200pt}}
\multiput(608.00,216.24)(0.622,0.509){2}{\rule{2.500pt}{0.123pt}}
\multiput(608.00,211.51)(5.811,6.000){2}{\rule{1.250pt}{1.200pt}}
\multiput(619.00,222.24)(0.792,0.509){2}{\rule{2.700pt}{0.123pt}}
\multiput(619.00,217.51)(6.396,6.000){2}{\rule{1.350pt}{1.200pt}}
\multiput(631.00,228.24)(0.622,0.509){2}{\rule{2.500pt}{0.123pt}}
\multiput(631.00,223.51)(5.811,6.000){2}{\rule{1.250pt}{1.200pt}}
\multiput(642.00,234.24)(0.622,0.509){2}{\rule{2.500pt}{0.123pt}}
\multiput(642.00,229.51)(5.811,6.000){2}{\rule{1.250pt}{1.200pt}}
\multiput(653.00,240.24)(0.792,0.509){2}{\rule{2.700pt}{0.123pt}}
\multiput(653.00,235.51)(6.396,6.000){2}{\rule{1.350pt}{1.200pt}}
\multiput(665.00,246.24)(0.642,0.505){4}{\rule{2.186pt}{0.122pt}}
\multiput(665.00,241.51)(6.463,7.000){2}{\rule{1.093pt}{1.200pt}}
\multiput(676.00,253.24)(0.642,0.505){4}{\rule{2.186pt}{0.122pt}}
\multiput(676.00,248.51)(6.463,7.000){2}{\rule{1.093pt}{1.200pt}}
\multiput(687.00,260.24)(0.622,0.509){2}{\rule{2.500pt}{0.123pt}}
\multiput(687.00,255.51)(5.811,6.000){2}{\rule{1.250pt}{1.200pt}}
\multiput(698.00,266.24)(0.738,0.505){4}{\rule{2.357pt}{0.122pt}}
\multiput(698.00,261.51)(7.108,7.000){2}{\rule{1.179pt}{1.200pt}}
\multiput(710.00,273.24)(0.642,0.505){4}{\rule{2.186pt}{0.122pt}}
\multiput(710.00,268.51)(6.463,7.000){2}{\rule{1.093pt}{1.200pt}}
\multiput(721.00,280.24)(0.581,0.503){6}{\rule{1.950pt}{0.121pt}}
\multiput(721.00,275.51)(6.953,8.000){2}{\rule{0.975pt}{1.200pt}}
\multiput(732.00,288.24)(0.738,0.505){4}{\rule{2.357pt}{0.122pt}}
\multiput(732.00,283.51)(7.108,7.000){2}{\rule{1.179pt}{1.200pt}}
\multiput(744.00,295.24)(0.642,0.505){4}{\rule{2.186pt}{0.122pt}}
\multiput(744.00,290.51)(6.463,7.000){2}{\rule{1.093pt}{1.200pt}}
\multiput(755.00,302.24)(0.581,0.503){6}{\rule{1.950pt}{0.121pt}}
\multiput(755.00,297.51)(6.953,8.000){2}{\rule{0.975pt}{1.200pt}}
\multiput(766.00,310.24)(0.657,0.503){6}{\rule{2.100pt}{0.121pt}}
\multiput(766.00,305.51)(7.641,8.000){2}{\rule{1.050pt}{1.200pt}}
\multiput(778.00,318.24)(0.642,0.505){4}{\rule{2.186pt}{0.122pt}}
\multiput(778.00,313.51)(6.463,7.000){2}{\rule{1.093pt}{1.200pt}}
\multiput(789.00,325.24)(0.581,0.503){6}{\rule{1.950pt}{0.121pt}}
\multiput(789.00,320.51)(6.953,8.000){2}{\rule{0.975pt}{1.200pt}}
\multiput(800.00,333.24)(0.524,0.502){8}{\rule{1.767pt}{0.121pt}}
\multiput(800.00,328.51)(7.333,9.000){2}{\rule{0.883pt}{1.200pt}}
\multiput(811.00,342.24)(0.657,0.503){6}{\rule{2.100pt}{0.121pt}}
\multiput(811.00,337.51)(7.641,8.000){2}{\rule{1.050pt}{1.200pt}}
\multiput(823.00,350.24)(0.581,0.503){6}{\rule{1.950pt}{0.121pt}}
\multiput(823.00,345.51)(6.953,8.000){2}{\rule{0.975pt}{1.200pt}}
\multiput(834.00,358.24)(0.524,0.502){8}{\rule{1.767pt}{0.121pt}}
\multiput(834.00,353.51)(7.333,9.000){2}{\rule{0.883pt}{1.200pt}}
\multiput(845.00,367.24)(0.657,0.503){6}{\rule{2.100pt}{0.121pt}}
\multiput(845.00,362.51)(7.641,8.000){2}{\rule{1.050pt}{1.200pt}}
\multiput(857.00,375.24)(0.524,0.502){8}{\rule{1.767pt}{0.121pt}}
\multiput(857.00,370.51)(7.333,9.000){2}{\rule{0.883pt}{1.200pt}}
\multiput(868.00,384.24)(0.524,0.502){8}{\rule{1.767pt}{0.121pt}}
\multiput(868.00,379.51)(7.333,9.000){2}{\rule{0.883pt}{1.200pt}}
\multiput(879.00,393.24)(0.524,0.502){8}{\rule{1.767pt}{0.121pt}}
\multiput(879.00,388.51)(7.333,9.000){2}{\rule{0.883pt}{1.200pt}}
\multiput(890.00,402.24)(0.588,0.502){8}{\rule{1.900pt}{0.121pt}}
\multiput(890.00,397.51)(8.056,9.000){2}{\rule{0.950pt}{1.200pt}}
\multiput(902.00,411.24)(0.475,0.502){10}{\rule{1.620pt}{0.121pt}}
\multiput(902.00,406.51)(7.638,10.000){2}{\rule{0.810pt}{1.200pt}}
\multiput(913.00,421.24)(0.524,0.502){8}{\rule{1.767pt}{0.121pt}}
\multiput(913.00,416.51)(7.333,9.000){2}{\rule{0.883pt}{1.200pt}}
\multiput(924.00,430.24)(0.531,0.502){10}{\rule{1.740pt}{0.121pt}}
\multiput(924.00,425.51)(8.389,10.000){2}{\rule{0.870pt}{1.200pt}}
\multiput(936.00,440.24)(0.475,0.502){10}{\rule{1.620pt}{0.121pt}}
\multiput(936.00,435.51)(7.638,10.000){2}{\rule{0.810pt}{1.200pt}}
\multiput(947.00,450.24)(0.524,0.502){8}{\rule{1.767pt}{0.121pt}}
\multiput(947.00,445.51)(7.333,9.000){2}{\rule{0.883pt}{1.200pt}}
\multiput(958.00,459.24)(0.531,0.502){10}{\rule{1.740pt}{0.121pt}}
\multiput(958.00,454.51)(8.389,10.000){2}{\rule{0.870pt}{1.200pt}}
\multiput(970.00,469.24)(0.434,0.502){12}{\rule{1.500pt}{0.121pt}}
\multiput(970.00,464.51)(7.887,11.000){2}{\rule{0.750pt}{1.200pt}}
\multiput(981.00,480.24)(0.475,0.502){10}{\rule{1.620pt}{0.121pt}}
\multiput(981.00,475.51)(7.638,10.000){2}{\rule{0.810pt}{1.200pt}}
\multiput(992.00,490.24)(0.434,0.502){12}{\rule{1.500pt}{0.121pt}}
\multiput(992.00,485.51)(7.887,11.000){2}{\rule{0.750pt}{1.200pt}}
\multiput(1003.00,501.24)(0.531,0.502){10}{\rule{1.740pt}{0.121pt}}
\multiput(1003.00,496.51)(8.389,10.000){2}{\rule{0.870pt}{1.200pt}}
\multiput(1015.00,511.24)(0.434,0.502){12}{\rule{1.500pt}{0.121pt}}
\multiput(1015.00,506.51)(7.887,11.000){2}{\rule{0.750pt}{1.200pt}}
\multiput(1026.00,522.24)(0.434,0.502){12}{\rule{1.500pt}{0.121pt}}
\multiput(1026.00,517.51)(7.887,11.000){2}{\rule{0.750pt}{1.200pt}}
\multiput(1037.00,533.24)(0.444,0.501){14}{\rule{1.500pt}{0.121pt}}
\multiput(1037.00,528.51)(8.887,12.000){2}{\rule{0.750pt}{1.200pt}}
\multiput(1049.00,545.24)(0.434,0.502){12}{\rule{1.500pt}{0.121pt}}
\multiput(1049.00,540.51)(7.887,11.000){2}{\rule{0.750pt}{1.200pt}}
\multiput(1060.00,556.24)(0.434,0.502){12}{\rule{1.500pt}{0.121pt}}
\multiput(1060.00,551.51)(7.887,11.000){2}{\rule{0.750pt}{1.200pt}}
\multiput(1071.00,567.24)(0.484,0.502){12}{\rule{1.609pt}{0.121pt}}
\multiput(1071.00,562.51)(8.660,11.000){2}{\rule{0.805pt}{1.200pt}}
\multiput(1083.00,578.24)(0.434,0.502){12}{\rule{1.500pt}{0.121pt}}
\multiput(1083.00,573.51)(7.887,11.000){2}{\rule{0.750pt}{1.200pt}}
\multiput(1094.00,589.24)(0.434,0.502){12}{\rule{1.500pt}{0.121pt}}
\multiput(1094.00,584.51)(7.887,11.000){2}{\rule{0.750pt}{1.200pt}}
\multiput(1105.00,600.24)(0.434,0.502){12}{\rule{1.500pt}{0.121pt}}
\multiput(1105.00,595.51)(7.887,11.000){2}{\rule{0.750pt}{1.200pt}}
\multiput(1116.00,611.24)(0.444,0.501){14}{\rule{1.500pt}{0.121pt}}
\multiput(1116.00,606.51)(8.887,12.000){2}{\rule{0.750pt}{1.200pt}}
\multiput(1128.00,623.24)(0.434,0.502){12}{\rule{1.500pt}{0.121pt}}
\multiput(1128.00,618.51)(7.887,11.000){2}{\rule{0.750pt}{1.200pt}}
\multiput(1141.24,632.00)(0.502,0.484){12}{\rule{0.121pt}{1.609pt}}
\multiput(1136.51,632.00)(11.000,8.660){2}{\rule{1.200pt}{0.805pt}}
\multiput(1150.00,646.24)(0.444,0.501){14}{\rule{1.500pt}{0.121pt}}
\multiput(1150.00,641.51)(8.887,12.000){2}{\rule{0.750pt}{1.200pt}}
\multiput(1164.24,656.00)(0.502,0.533){12}{\rule{0.121pt}{1.718pt}}
\multiput(1159.51,656.00)(11.000,9.434){2}{\rule{1.200pt}{0.859pt}}
\multiput(1175.24,669.00)(0.502,0.484){12}{\rule{0.121pt}{1.609pt}}
\multiput(1170.51,669.00)(11.000,8.660){2}{\rule{1.200pt}{0.805pt}}
\multiput(1186.24,681.00)(0.501,0.489){14}{\rule{0.121pt}{1.600pt}}
\multiput(1181.51,681.00)(12.000,9.679){2}{\rule{1.200pt}{0.800pt}}
\multiput(1198.24,694.00)(0.502,0.484){12}{\rule{0.121pt}{1.609pt}}
\multiput(1193.51,694.00)(11.000,8.660){2}{\rule{1.200pt}{0.805pt}}
\multiput(1209.24,706.00)(0.502,0.533){12}{\rule{0.121pt}{1.718pt}}
\multiput(1204.51,706.00)(11.000,9.434){2}{\rule{1.200pt}{0.859pt}}
\multiput(1220.24,719.00)(0.502,0.533){12}{\rule{0.121pt}{1.718pt}}
\multiput(1215.51,719.00)(11.000,9.434){2}{\rule{1.200pt}{0.859pt}}
\multiput(1231.24,732.00)(0.501,0.489){14}{\rule{0.121pt}{1.600pt}}
\multiput(1226.51,732.00)(12.000,9.679){2}{\rule{1.200pt}{0.800pt}}
\multiput(1243.24,745.00)(0.502,0.533){12}{\rule{0.121pt}{1.718pt}}
\multiput(1238.51,745.00)(11.000,9.434){2}{\rule{1.200pt}{0.859pt}}
\multiput(1254.24,758.00)(0.502,0.533){12}{\rule{0.121pt}{1.718pt}}
\multiput(1249.51,758.00)(11.000,9.434){2}{\rule{1.200pt}{0.859pt}}
\multiput(1265.24,771.00)(0.501,0.534){14}{\rule{0.121pt}{1.700pt}}
\multiput(1260.51,771.00)(12.000,10.472){2}{\rule{1.200pt}{0.850pt}}
\multiput(1277.24,785.00)(0.502,0.533){12}{\rule{0.121pt}{1.718pt}}
\multiput(1272.51,785.00)(11.000,9.434){2}{\rule{1.200pt}{0.859pt}}
\multiput(1288.24,798.00)(0.502,0.583){12}{\rule{0.121pt}{1.827pt}}
\multiput(1283.51,798.00)(11.000,10.207){2}{\rule{1.200pt}{0.914pt}}
\multiput(1299.24,812.00)(0.502,0.583){12}{\rule{0.121pt}{1.827pt}}
\multiput(1294.51,812.00)(11.000,10.207){2}{\rule{1.200pt}{0.914pt}}
\multiput(1310.24,826.00)(0.501,0.534){14}{\rule{0.121pt}{1.700pt}}
\multiput(1305.51,826.00)(12.000,10.472){2}{\rule{1.200pt}{0.850pt}}
\sbox{\plotpoint}{\rule[-0.200pt]{0.400pt}{0.400pt}}%
\put(190.0,82.0){\rule[-0.200pt]{0.400pt}{187.179pt}}
\put(190.0,82.0){\rule[-0.200pt]{300.884pt}{0.400pt}}
\put(1439.0,82.0){\rule[-0.200pt]{0.400pt}{187.179pt}}
\put(190.0,859.0){\rule[-0.200pt]{300.884pt}{0.400pt}}
\end{picture}

\caption{Zależność czasu działania algorytmów od wielkości zmiany obrazu (jednakowo w pionie i poziomie, jednostka osi 0X: $\%$).}
\end{center}
\end{figure}

Ze zgromadzonych danych można wywnioskować, że najszybszą z przebadanych przez
metod okazuje się metoda najbliższego sąsiedztwa. Jak widać to na wykresie, jest
ona niemal czterokrotnie szybsza od AFS(I) i nawet osiemnastokrotnie od AFS(III).
Pamiętamy jednakże, że metoda ta daje najbardziej zniekształcone wyniki, więc
za szybkie działanie płacimy w tym przypadku właśnie jakością obrazu.
Najwolniejszym algorytmem okazuje się być algorytm wykorzystujący naturalną
funkcję sklejaną trzeciego stopnia. Jest około pięciokrotnie wolniejszy od algorytmu AFS(I).
W jego przypadku występuje więc odwrotna sytuacja: przez uzyskanie lepszej jakości,
tracimy na czasie działania programu.

