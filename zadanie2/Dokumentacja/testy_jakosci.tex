\section{Porównanie algorytmów pod względem jakości}
Pierwszym zadaniem, które wykonaliśmy, polegało na porównaniu jakości obrazów po
zmianie jego rozdzielczości za pomocą trzech powyżej opisanych metod. Rysunek
\textbf{3.1} jest obrazem, którego użyliśmy przy wstępnych badaniach.
\begin{figure}[h!tb]
\begin{center}
\includegraphics{../obrazki/liscie.jpg}
\caption{Liście - pierwszy rysunek testowy.}
\end{center}
\end{figure}

Powiększyliśmy ten obraz o pięciokrotnie w pionie i poziomie używając
kolejnych algorytmów. Wynik tej operacji prezentuje rysunek \textbf{3.2}.
\begin{figure}[h!tb]
\begin{center}
\subfigure[NS]{
\includegraphics[width=4cm]{../obrazki/liscie_500_round.jpg}
}
\subfigure[AFS(I)]{
\includegraphics[width=4cm]{../obrazki/liscie_500_linear.jpg}
}
\subfigure[AFS(III)]{
\includegraphics[width=4cm]{../obrazki/liscie_500_cubic.jpg}
}
\caption{Fragmenty obrazków będących pięciokrotnym powiększeniem rysunku \textbf{3.1}.}
\end{center}
\end{figure}

Jak można odrazu zauważyć, metoda NS dała najgorszy pod względem wizualnym wynik.
Widoszny jest efekt "schodków", czyli wygląd krawędzi daleko odbiega od naszych
oczekiwań. Gdy spojrzymy na pozostałe dwie metody, dają one wynik dość zbliżony,
przy czym AFS(III) zdaje się dawać mniejszy efekt rozmycia.

Po wykonaniu tego wstępnego rozpozania problemu, postanowiliśmy sprawdzić, jak
poszczególne algorytmy zachowają się wobec pisanego tekstu.
Fragment tekstu, którego użyliśmy do naszych eksperymentów znajduje się w pliku
o nazwie \textit{tekst.bmp}.

Tym razem rozpoczęlismy od zmniejszania obrazka. Spośród sprawdzonych wartości,
jako przykład do prezentacji wybraliśmy skalowanie do $70\%$ oryginalnego
rozmiaru, ponieważ tekst wtedy dla wszystkich metod pozostaje czytelny, a
widoczne są już uboczne efekty zmiany rozdzielczości. Fragment otrzymanych
wyników przedstawiw rysunek \textbf{3.3}. Ponownie dla metody NS obserwujemy
efekt "schodków" i przez to wyraźną stratę na jakości obrazu. Metoda AFS(I)
wydaje się zaś rozmazywać poszczególne znaki. Choć nieidealna, metoda AFS(III),
zdaje się radzić sobie najlepiej jeśli chodzi o jakość wyniku. Analogiczne efekty
otrzymujemy przy powiększaniu tekstu. Rysunek \textbf{3.4} prezentuje
fragmenty powyższego fragmentu dokumentu powiększone pięciokrotnie. Chcąc
podsumować nasze wizuane obserwacje, najgorszym algorytmem pod względem zachowania
jakości jest algorytm NS, najlepszym zaś zdaje się być algorytm AFS(III).
\begin{figure}[h!tb]
\begin{center}
\subfigure[NS]{
\includegraphics[width=5cm]{../obrazki/tekst_70_round.jpg}
}
\subfigure[AFS(I)]{
\includegraphics[width=5cm]{../obrazki/tekst_70_linear.jpg}
}
\subfigure[AFS(III)]{
\includegraphics[width=5cm]{../obrazki/tekst_70_cubic.jpg}
}
\caption{Pomniejszenie tekstu za pomocą trzech metod.}
\end{center}
\end{figure}
\begin{figure}[h!tb]
\begin{center}
\subfigure[NS]{
\includegraphics[width=10cm]{../obrazki/tekst_500_round.jpg}
}
\subfigure[AFS(I)]{
\includegraphics[width=10cm]{../obrazki/tekst_500_linear.jpg}
}
\subfigure[AFS(III)]{
\includegraphics[width=10cm]{../obrazki/tekst_500_linear.jpg}
}
\caption{Powiększenie tekstu za pomocą trzech metod.}
\end{center}
\end{figure}







