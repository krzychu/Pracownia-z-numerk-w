\section{Wstęp}
Zdjęcia i obrazy, dawniej tak unikatowe, w dzisiejszych czasach otaczają nas
wszędzie. Oczekuje się coraz więcej od ich oryginalności jak i jakości. Nie
dziwi nas więc zainteresowanie programami komputerowymi służącymi do obróbki
obrazów, w tym do ich skalowania. Problem pojawia się wtedy, gdy chcemy zmniejszać i zwiększać
rysunki tracąc jak najmniej na ich wyglądzie. Opracowano wiele algorytmów,
które lepiej lub gorzej radzą sobie z tym zadaniem. W naszej pracy chcemy
przedstawić i zbadać pod względem szybkości działania, jak i również
skuteczności, trzy metody skalowania obrazów.

W paragrafie \S2 zamieściliśmy opis poszczególnych metod zmieniania
rozmiarów obrazu. Kolejny rozdział stanowi relację z doświadczeń
przeprowadzonych przez nas nad jakością otrzymywanych obrazów przy
użyciu trzech różnych algorytmów. Aby nie opierać się wyłącznie na swoich
subiektywnych ocenach wyglądu rysunków, wprowadziliśmy porównanie
matematyczne przy użyciu normy, która przedstawiona jest w tym samym rozdziale.
W Paragrafie \S4 znajdują się wyniki naszych badań odnośnie kolejności
wykonywanych działań. Innymi słowy, sprawdziliśmy czy jakość obrazu
wynikowego jest uzależniona od tego, w którym kierunku (w poziomie, cz w pionie) najpierw skalujemy.
W \S5 umieściliśmy opis naszych eksperymentów związanych z porównaniem
algorytmów pod względem szybkości, w następnym rozdziale zaś
postaraliśmy się wizualnie porównać najlepszy z badanych przez nas algorytmów
z wynikami, jakie daje profesjonalny program do obróbki obrazu. Ostatni
rozdział jest swoistym podsumowaniem, kompletującym wyciągnięte przez nas
wnioski.

Nasze eksperymenty przeprowadziliśmy zarówno na obrazach monochromatycznych,
jak i kolorowych. Wykorzystanymi przez nas rysunkami były wykres dwuwymiarowy
funkcji Rungego, funkcji sinus, piły, jak i wielu innych. Wszystkie umieszczone są w katalogu \textit{obrazki}. Programy, których
użyliśmy do wykonania wszelkiego rodzaju obliczeń i przeprowadzenia testów znajdują się w katalogu \textit{prog}.

