\section{Wnioski}
Dzięki zrealizowania postawionych przed nami zadań, zdołaliśmy zebrać wiele
cennych informacji, które w tym rozdziale postaramy się skompletować.

Po pierwsze, jeśli chodzi o ocenę badanych przez nas algortmów, najlepszy
pod względem jakości okazał się być AFS(III), jednakże wiązało się to z długim czasem
realizacji skalowania. Jednocześnie najgorszy pod względem jakości algorytm NS,
był niewątpliwie najszbszy. Niekiedy jednak obserwuje się przypadki odbiegające od tej reguły.
Niekiedy to, jaką metodę powinniśmy zastosować, zależy od tego, co przedstawia obraz,
czy kontury są wyraźne, czy raczej rozmyte, czy mamy wiele szczegółów, czy nie itp. Zauważyliśmy, jak bardzo badane przez nas algorytmy mogą zmienić
oryginalny rysunek, zmniejszając go a następnie powiększając do początkowych wielkości. Skutki takiego działania okazały się katastrofalne.
Dobrą informacą było jednak to, że w pierwszej kolejności zwiększając obraz, a następnie
go zmniejszając, otrzymujemy obrazy identyczne z oryginalnym dla wszystkich metod.
Powiększanie jest więc bezstratne.

Ważną obserwacją było również to, że wyniki skalowań nie zależą od tego czy
najpierw zmienimy obraz w pionie, czy w poziomie. Zauważyliśmy także, że czas
działania algorytmów badanych przez nas, nie jest zależny od tego, co jest
na nim przedstawione. Na długość trwania obliczeń ma bowiem wpływ wielkość rysunku, który skalujemy.

Dowiedzieliśmy się, że algorytmy, które w rzeczywistości wykorzysuje się do zmiany obrazu, znacznie różnią się o tych, które badaliśmy.
Przede wszystki nie opierają się wyłącznie na informacji o punktach na tej samej osi, ale patrzą na najbliższe punkty otaczające dany piksel ze wszystkich stron.
Jak się okazało jest to istotne, algorytmy dwuliniowy i dwusześćienny okazały dawać bardziej zadowalające wyniki, niż najlepszy z testowanych przez na algorytmów, a czyli AFS(III).
